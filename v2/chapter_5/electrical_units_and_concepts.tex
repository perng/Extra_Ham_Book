\section{Electrical Units and Concepts}
\label{section:electrical_units_and_concepts}

\subsection*{Introduction}
This section introduces fundamental electrical units and concepts, including electrical current, power, resistance, voltage, frequency, and the distinction between conductors and insulators. These concepts are essential for understanding how electrical circuits function and are foundational for further study in radio technology.

\subsection*{Electrical Current and Its Unit of Measurement}
Electrical current is the flow of electric charge, typically carried by electrons in a conductor. The unit of measurement for electrical current is the \textbf{Ampere} (A), often abbreviated as "Amp." One ampere is defined as the flow of one coulomb of charge per second. Mathematically, this can be expressed as:
\begin{equation}
I = \frac{Q}{t}
\end{equation}
where \( I \) is the current in amperes, \( Q \) is the charge in coulombs, and \( t \) is the time in seconds.

\subsection*{Relationship Between Voltage, Current, and Resistance}
The relationship between voltage (\( V \)), current (\( I \)), and resistance (\( R \)) is described by Ohm's Law:
\begin{equation}
V = I \cdot R
\end{equation}
This equation states that the voltage across a conductor is directly proportional to the current flowing through it and the resistance of the conductor. Voltage is measured in volts (V), current in amperes (A), and resistance in ohms (\(\Omega\)).

\subsection*{Why Metals Are Good Conductors of Electricity}
Metals are generally good conductors of electricity because they have many free electrons that can move easily through the material. These free electrons are not tightly bound to any particular atom, allowing them to flow in response to an applied electric field. This property makes metals highly effective at conducting electrical current.

\subsection*{Frequency and Its Unit of Measurement}
Frequency is the number of cycles of a periodic waveform that occur in one second. The unit of measurement for frequency is the \textbf{Hertz} (Hz), named after the German physicist Heinrich Hertz. One hertz is equivalent to one cycle per second. Frequency is a critical parameter in radio technology, as it determines the wavelength and propagation characteristics of radio waves.

\subsection*{Conductors and Insulators}
Conductors are materials that allow the free flow of electric charge, typically due to the presence of free electrons. Examples of good conductors include metals like copper and aluminum. Insulators, on the other hand, are materials that resist the flow of electric charge. Examples of good insulators include glass, rubber, and plastic. The distinction between conductors and insulators is crucial in designing electrical circuits and components.

\subsection*{Questions}
\begin{tcolorbox}[colback=gray!10!white,colframe=black!75!black,title={T5A01}]
Electrical current is measured in which of the following units?
\begin{enumerate}[label=\Alph*),noitemsep]
    \item Volts
    \item Watts
    \item Ohms
    \item \textbf{Amperes}
\end{enumerate}
\end{tcolorbox}
Electrical current is measured in amperes (A). Volts measure voltage, watts measure power, and ohms measure resistance.

%memory_trick T5A01

\begin{tcolorbox}[colback=gray!10!white,colframe=black!75!black,title={T5A02}]
Electrical power is measured in which of the following units?
\begin{enumerate}[label=\Alph*),noitemsep]
    \item Volts
    \item \textbf{Watts}
    \item Watt-hours
    \item Amperes
\end{enumerate}
\end{tcolorbox}
Electrical power is measured in watts (W). Volts measure voltage, watt-hours measure energy, and amperes measure current.

%memory_trick T5A02

\begin{tcolorbox}[colback=gray!10!white,colframe=black!75!black,title={T5A03}]
What is the name for the flow of electrons in an electric circuit?
\begin{enumerate}[label=\Alph*),noitemsep]
    \item Voltage
    \item Resistance
    \item Capacitance
    \item \textbf{Current}
\end{enumerate}
\end{tcolorbox}
The flow of electrons in an electric circuit is called current. Voltage is the force that causes the flow, resistance opposes the flow, and capacitance is the ability to store charge.

%memory_trick T5A03

\begin{tcolorbox}[colback=gray!10!white,colframe=black!75!black,title={T5A04}]
What are the units of electrical resistance?
\begin{enumerate}[label=\Alph*),noitemsep]
    \item Siemens
    \item Mhos
    \item \textbf{Ohms}
    \item Coulombs
\end{enumerate}
\end{tcolorbox}
Electrical resistance is measured in ohms (\(\Omega\)). Siemens and mhos are units of conductance, and coulombs measure electric charge.

%memory_trick T5A04

\begin{tcolorbox}[colback=gray!10!white,colframe=black!75!black,title={T5A05}]
What is the electrical term for the force that causes electron flow?
\begin{enumerate}[label=\Alph*),noitemsep]
    \item \textbf{Voltage}
    \item Ampere-hours
    \item Capacitance
    \item Inductance
\end{enumerate}
\end{tcolorbox}
The force that causes electron flow is called voltage. Ampere-hours measure charge, capacitance is the ability to store charge, and inductance is the property of a conductor to oppose changes in current.

%memory_trick T5A05

\begin{tcolorbox}[colback=gray!10!white,colframe=black!75!black,title={T5A06}]
What is the unit of frequency?
\begin{enumerate}[label=\Alph*),noitemsep]
    \item \textbf{Hertz}
    \item Henry
    \item Farad
    \item Tesla
\end{enumerate}
\end{tcolorbox}
The unit of frequency is the hertz (Hz). Henry is the unit of inductance, farad is the unit of capacitance, and tesla is the unit of magnetic flux density.

%memory_trick T5A06

\begin{tcolorbox}[colback=gray!10!white,colframe=black!75!black,title={T5A07}]
Why are metals generally good conductors of electricity?
\begin{enumerate}[label=\Alph*),noitemsep]
    \item They have relatively high density
    \item \textbf{They have many free electrons}
    \item They have many free protons
    \item All these choices are correct
\end{enumerate}
\end{tcolorbox}
Metals are good conductors because they have many free electrons that can move easily through the material. Density and protons are not relevant to conductivity.

%memory_trick T5A07

\begin{tcolorbox}[colback=gray!10!white,colframe=black!75!black,title={T5A08}]
Which of the following is a good electrical insulator?
\begin{enumerate}[label=\Alph*),noitemsep]
    \item Copper
    \item \textbf{Glass}
    \item Aluminum
    \item Mercury
\end{enumerate}
\end{tcolorbox}
Glass is a good electrical insulator because it resists the flow of electric charge. Copper, aluminum, and mercury are all good conductors.

%memory_trick T5A08

\subsection*{Summary}
This section covered the following key concepts:
\begin{itemize}
    \item \textbf{Electrical Current}: The flow of electric charge, measured in amperes (A).
    \item \textbf{Electrical Power}: The rate at which electrical energy is transferred, measured in watts (W).
    \item \textbf{Electron Flow}: The movement of electrons in a conductor, which constitutes electric current.
    \item \textbf{Electrical Resistance}: The opposition to the flow of current, measured in ohms (\(\Omega\)).
    \item \textbf{Voltage}: The force that causes electron flow, measured in volts (V).
    \item \textbf{Frequency}: The number of cycles per second, measured in hertz (Hz).
    \item \textbf{Conductors and Insulators}: Materials that allow or resist the flow of electric charge, respectively.
\end{itemize}

\subsection*{Figures}
\begin{figure}[h!]
    \centering
    %\includegraphics[width=0.8\textwidth]{electron_flow.svg}
    \caption{Electron flow in a conductor. The diagram shows the movement of free electrons through a metal wire under the influence of an applied electric field.}
    \label{fig:electron_flow}
    % Prompt: Diagram showing the flow of electrons in a conductor. The diagram should depict free electrons moving through a metal wire, with arrows indicating the direction of electron flow.
\end{figure}

\subsection*{Tables}
\begin{table}[h!]
    \centering
    \begin{tabular}{|c|c|}
        \hline
        \textbf{Quantity} & \textbf{Unit} \\
        \hline
        Current & Ampere (A) \\
        Voltage & Volt (V) \\
        Resistance & Ohm (\(\Omega\)) \\
        Power & Watt (W) \\
        Frequency & Hertz (Hz) \\
        \hline
    \end{tabular}
    \caption{Units of Electrical Measurement}
    \label{tab:electrical_units}
    % Prompt: Table comparing units of electrical measurement, including current, voltage, resistance, power, and frequency.
\end{table}
