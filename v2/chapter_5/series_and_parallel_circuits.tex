\section{Series and Parallel Circuits}
\label{section:series_and_parallel_circuits}

\subsection*{Characteristics of Series Circuits}
In a series circuit, components are connected end-to-end in a single path. The current flowing through each component is the same, as there is only one path for the current to follow. The total voltage across the circuit is the sum of the voltages across each component. This can be expressed mathematically as:
\begin{equation}
    V_{\text{total}} = V_1 + V_2 + V_3 + \dots + V_n
    \label{eq:series_voltage}
\end{equation}
where \( V_1, V_2, \dots, V_n \) are the voltages across each component.

\subsection*{Characteristics of Parallel Circuits}
In a parallel circuit, components are connected across the same voltage source, providing multiple paths for the current to flow. The voltage across each component is the same, but the current divides among the branches. The total current in the circuit is the sum of the currents through each branch:
\begin{equation}
    I_{\text{total}} = I_1 + I_2 + I_3 + \dots + I_n
    \label{eq:parallel_current}
\end{equation}
where \( I_1, I_2, \dots, I_n \) are the currents through each branch.

\subsection*{Current and Voltage in Series and Parallel Circuits}
In series circuits, the current is constant throughout, while the voltage varies across components. In parallel circuits, the voltage is constant across all components, while the current varies. These behaviors are fundamental to understanding how circuits operate and are critical for designing and troubleshooting electrical systems.

\begin{figure}[h]
    \centering
    % \includegraphics[width=0.8\textwidth]{series_parallel_circuit_diagram.svg}
    % Diagram comparing series and parallel circuits
    % The figure should show a side-by-side comparison of a series circuit and a parallel circuit.
    % The series circuit should have three resistors connected in a single path, with arrows indicating the direction of current flow.
    % The parallel circuit should have three resistors connected across the same voltage source, with arrows showing the division of current.
    \caption{Series and parallel circuits}
    \label{fig:series_parallel}
\end{figure}

\subsection*{Questions}
\begin{tcolorbox}[colback=gray!10!white,colframe=black!75!black,title={T5D13}]
    In which type of circuit is DC current the same through all components?
    \begin{enumerate}[label=\Alph*),noitemsep]
        \item \textbf{Series}
        \item Parallel
        \item Resonant
        \item Branch
    \end{enumerate}
\end{tcolorbox}
In a series circuit, the current is the same through all components because there is only one path for the current to flow. In parallel circuits, the current divides among the branches, so it is not the same through all components. Resonant and branch circuits are not relevant to this question.

%memory_trick T5D13

\begin{tcolorbox}[colback=gray!10!white,colframe=black!75!black,title={T5D14}]
    In which type of circuit is voltage the same across all components?
    \begin{enumerate}[label=\Alph*),noitemsep]
        \item Series
        \item \textbf{Parallel}
        \item Resonant
        \item Branch
    \end{enumerate}
\end{tcolorbox}
In a parallel circuit, the voltage across each component is the same because they are all connected directly to the voltage source. In series circuits, the voltage varies across components. Resonant and branch circuits are not relevant to this question.

%memory_trick T5D14

\subsection*{Summary}
\begin{itemize}
    \item \textbf{Series circuits}: Components are connected in a single path, with the same current flowing through each component. The total voltage is the sum of the voltages across each component.
    \item \textbf{Parallel circuits}: Components are connected across the same voltage source, with the same voltage across each component. The total current is the sum of the currents through each branch.
    \item \textbf{Current and voltage in circuits}: In series circuits, current is constant, and voltage varies. In parallel circuits, voltage is constant, and current varies.
\end{itemize}
