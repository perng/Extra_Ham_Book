\section{Station Identification}
\label{sec:station_id}

\subsection*{Station Identification Procedures and Timing}
Station identification is a critical requirement for amateur radio operators to ensure compliance with FCC regulations. The station must be identified using the FCC-assigned call sign at specific intervals during communication. According to FCC Part 97, the station must transmit its call sign at least every 10 minutes during a communication and at the end of the communication. This ensures that the station is properly identified and traceable by regulatory authorities.

Additionally, when using tactical call signs such as "Race Headquarters," the FCC-assigned call sign must still be used at the end of each communication and every 10 minutes during the communication. This dual identification ensures clarity and compliance with regulations.

\subsection*{Language and Method Requirements for Station Identification}
When operating in a phone sub-band, the station identification must be transmitted in English. This requirement ensures that the identification is universally understandable. The call sign can be transmitted using either a phone emission or CW (Morse code) emission. Self-assigned indicators, such as "KL7CC/W3," are also acceptable as long as they follow the prescribed format.

\begin{figure}[h!]
    \centering
    % \includegraphics[width=0.8\textwidth]{id_timeline.png} % Placeholder for the image
    \caption{Station Identification Timeline}
    \label{fig:id_timeline}
    % Prompt: Create a timeline showing station identification requirements. The timeline should include intervals of 10 minutes and the end of communication, with labels for FCC-assigned call sign and tactical call sign usage.
\end{figure}

\begin{table}[h!]
    \centering
    \caption{Station ID Requirements}
    \label{tab:id_requirements}
    \begin{tabular}{|l|l|}
        \hline
        \textbf{Requirement} & \textbf{Details} \\
        \hline
        ID Timing & Every 10 minutes and at the end of communication \\
        Language & English for phone sub-band \\
        Method & Phone or CW emission \\
        Self-Assigned Indicators & Acceptable (e.g., KL7CC/W3) \\
        \hline
    \end{tabular}
\end{table}

\subsection*{Questions}
\begin{tcolorbox}[colback=gray!10!white,colframe=black!75!black,title={T1F01}]
    When must the station and its records be available for FCC inspection?
    \begin{enumerate}[label=\Alph*),noitemsep]
        \item At any time ten days after notification by the FCC of such an inspection
        \item \textbf{At any time upon request by an FCC representative}
        \item At any time after written notification by the FCC of such inspection
        \item Only when presented with a valid warrant by an FCC official or government agent
    \end{enumerate}
\end{tcolorbox}
The station and its records must be available for inspection at any time upon request by an FCC representative. This ensures compliance with FCC regulations and allows for immediate verification of station operations.

%memory_trick T1F01

\begin{tcolorbox}[colback=gray!10!white,colframe=black!75!black,title={T1F02}]
    How often must you identify with your FCC-assigned call sign when using tactical call signs such as “Race Headquarters”?
    \begin{enumerate}[label=\Alph*),noitemsep]
        \item Never, the tactical call is sufficient
        \item Once during every hour
        \item \textbf{At the end of each communication and every ten minutes during a communication}
        \item At the end of every transmission
    \end{enumerate}
\end{tcolorbox}
Even when using tactical call signs, the FCC-assigned call sign must be used at the end of each communication and every 10 minutes during the communication. This ensures proper identification.

%memory_trick T1F02

\begin{tcolorbox}[colback=gray!10!white,colframe=black!75!black,title={T1F03}]
    When are you required to transmit your assigned call sign?
    \begin{enumerate}[label=\Alph*),noitemsep]
        \item At the beginning of each contact, and every 10 minutes thereafter
        \item At least once during each transmission
        \item At least every 15 minutes during and at the end of a communication
        \item \textbf{At least every 10 minutes during and at the end of a communication}
    \end{enumerate}
\end{tcolorbox}
The call sign must be transmitted at least every 10 minutes during and at the end of a communication. This is a key requirement under FCC Part 97.

%memory_trick T1F03

\begin{tcolorbox}[colback=gray!10!white,colframe=black!75!black,title={T1F04}]
    What language may you use for identification when operating in a phone sub-band?
    \begin{enumerate}[label=\Alph*),noitemsep]
        \item Any language recognized by the United Nations
        \item Any language recognized by the ITU
        \item \textbf{English}
        \item English, French, or Spanish
    \end{enumerate}
\end{tcolorbox}
When operating in a phone sub-band, station identification must be in English. This ensures clarity and compliance with FCC regulations.

%memory_trick T1F04

\begin{tcolorbox}[colback=gray!10!white,colframe=black!75!black,title={T1F05}]
    What method of call sign identification is required for a station transmitting phone signals?
    \begin{enumerate}[label=\Alph*),noitemsep]
        \item Send the call sign followed by the indicator RPT
        \item \textbf{Send the call sign using a CW or phone emission}
        \item Send the call sign followed by the indicator R
        \item Send the call sign using only a phone emission
    \end{enumerate}
\end{tcolorbox}
The call sign can be transmitted using either a phone emission or CW emission. This flexibility allows operators to choose the method that best suits their equipment and operating conditions.

%memory_trick T1F05

\begin{tcolorbox}[colback=gray!10!white,colframe=black!75!black,title={T1F06}]
    Which of the following self-assigned indicators are acceptable when using a phone transmission?
    \begin{enumerate}[label=\Alph*),noitemsep]
        \item KL7CC stroke W3
        \item KL7CC slant W3
        \item KL7CC slash W3
        \item \textbf{All these choices are correct}
    \end{enumerate}
\end{tcolorbox}
All the listed self-assigned indicators (stroke, slant, slash) are acceptable when using a phone transmission. This provides flexibility in how operators identify their stations.

%memory_trick T1F06

\subsection*{Summary}
This section covered the following key concepts:
\begin{itemize}
    \item \textbf{Station identification requirements}: The station must be identified using the FCC-assigned call sign at specific intervals.
    \item \textbf{ID timing rules}: Identification must occur every 10 minutes during communication and at the end of communication.
    \item \textbf{Language requirements}: English is required for identification in phone sub-bands.
    \item \textbf{Self-assigned indicators}: Acceptable formats include stroke, slant, and slash indicators.
\end{itemize}
