\section{Licensing and Call Sign Protocols}
\label{sec:licensing_call_signs}

\subsection*{License Classes and Availability}
The Federal Communications Commission (FCC) currently offers three license classes for amateur radio operators: Technician, General, and Amateur Extra. Each class grants specific operating privileges, with the Technician class being the entry-level license. The General class provides additional frequency privileges, while the Amateur Extra class offers the most extensive operating privileges, including access to all amateur radio bands.

\subsection*{Vanity Call Sign Rules}
Amateur radio operators may select a desired call sign under the vanity call sign rules. Any licensed amateur, regardless of license class, is eligible to apply for a vanity call sign. The process involves submitting an application to the FCC, which is then reviewed for compliance with the rules. 

\subsection*{International Communications}
FCC-licensed amateur radio stations are permitted to make international communications that are incidental to the purposes of the Amateur Radio Service. This includes remarks of a personal character, but excludes communications for business purposes. These rules ensure that amateur radio remains a non-commercial service.

\subsection*{License Renewal and Revocation}
Maintaining accurate contact information with the FCC is crucial. Failure to provide and maintain a correct email address can result in the revocation of the station license or suspension of the operator license. Additionally, amateur radio licenses are typically issued for a term of ten years and must be renewed before expiration.

% Table summarizing license classes and privileges
\begin{table}[h!]
    \centering
    \caption{Amateur Radio License Classes and Privileges Comparison}
    \label{tab:license_classes}
    \begin{tabular}{|l|p{9cm}|l|}
        \hline
        \textbf{License Class} & \textbf{Privileges} & \textbf{Term} \\ 
        \hline
        Technician & 
        \begin{itemize}[noitemsep,topsep=0pt]
            \item All VHF/UHF amateur bands (50 MHz and above)
            \item Limited HF privileges:
                \begin{itemize}[noitemsep]
                    \item CW on 80, 40, 15 meters
                    \item CW, RTTY, data on 10 meters
                    \item SSB phone on 10 meters (28.3-28.5 MHz)
                \end{itemize}
            \item Maximum 200 watts PEP output on 10 meters
            \item Full privileges above 50 MHz
        \end{itemize} & 
        10 years \\ 
        \hline
        General & 
        \begin{itemize}[noitemsep,topsep=0pt]
            \item All Technician privileges, plus:
            \item Most HF privileges (160-10 meters)
            \item At least 83 voice channels on HF
            \item Large segments for CW, phone, digital modes
            \item Maximum 1500 watts PEP output on most bands
            \item Some restrictions in 60, 30, 17, 12 meter bands
        \end{itemize} & 
        10 years \\ 
        \hline
        Amateur Extra & 
        \begin{itemize}[noitemsep,topsep=0pt]
            \item All General privileges, plus:
            \item Full privileges on all amateur bands
            \item Exclusive Extra-only sub-bands
            \item Additional phone privileges in:
                \begin{itemize}[noitemsep]
                    \item 75/80 meters (3.6-3.7 MHz)
                    \item 40 meters (7.125-7.175 MHz)
                    \item 20 meters (14.150-14.175 MHz)
                    \item 15 meters (21.200-21.225 MHz)
                \end{itemize}
            \item Shortest possible call signs available
        \end{itemize} & 
        10 years \\
        \hline
    \end{tabular}
\end{table}

\subsection*{Questions}
\begin{tcolorbox}[colback=gray!10!white,colframe=black!75!black,title={T1C01}]
    For which license classes are new licenses currently available from the FCC?
    \begin{enumerate}[label=\Alph*),noitemsep]
        \item Novice, Technician, General, Amateur Extra
        \item Technician, Technician Plus, General, Amateur Extra
        \item Novice, Technician Plus, General, Advanced
        \item \textbf{Technician, General, Amateur Extra}
    \end{enumerate}
\end{tcolorbox}
The FCC currently offers new licenses for the Technician, General, and Amateur Extra classes. The Novice and Technician Plus classes are no longer available.

%memory_trick T1C01

\begin{tcolorbox}[colback=gray!10!white,colframe=black!75!black,title={T1C02}]
    Who may select a desired call sign under the vanity call sign rules?
    \begin{enumerate}[label=\Alph*),noitemsep]
        \item Only a licensed amateur with a General or Amateur Extra Class license
        \item Only a licensed amateur with an Amateur Extra Class license
        \item Only a licensed amateur who has been licensed continuously for more than 10 years
        \item \textbf{Any licensed amateur}
    \end{enumerate}
\end{tcolorbox}
Any licensed amateur, regardless of license class, may apply for a vanity call sign.

%memory_trick T1C02

\begin{tcolorbox}[colback=gray!10!white,colframe=black!75!black,title={T1C03}]
    What types of international communications are an FCC-licensed amateur radio station permitted to make?
    \begin{enumerate}[label=\Alph*),noitemsep]
        \item \textbf{Communications incidental to the purposes of the Amateur Radio Service and remarks of a personal character}
        \item Communications incidental to conducting business or remarks of a personal nature
        \item Only communications incidental to contest exchanges; all other communications are prohibited
        \item Any communications that would be permitted by an international broadcast station
    \end{enumerate}
\end{tcolorbox}
FCC-licensed amateur radio stations are permitted to make international communications that are incidental to the purposes of the Amateur Radio Service, including personal remarks. Business communications are not allowed.

%memory_trick T1C03

\begin{tcolorbox}[colback=gray!10!white,colframe=black!75!black,title={T1C04}]
    What may happen if the FCC is unable to reach you by email?
    \begin{enumerate}[label=\Alph*),noitemsep]
        \item Fine and suspension of operator license
        \item \textbf{Revocation of the station license or suspension of the operator license}
        \item Revocation of access to the license record in the FCC system
        \item Nothing; there is no such requirement
    \end{enumerate}
\end{tcolorbox}
Failure to maintain a correct email address with the FCC can result in the revocation of the station license or suspension of the operator license.

%memory_trick T1C04

\subsection*{Summary}
This section covered the current license classes available from the FCC, including Technician, General, and Amateur Extra. It also explained the rules for selecting a vanity call sign and the types of international communications permitted for amateur radio stations. Finally, it outlined the importance of maintaining accurate contact information with the FCC to avoid license revocation or suspension.
