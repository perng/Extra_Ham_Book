\section{Basic Operating Rules}
\label{sec:basic_rules}

\subsection*{Permitted Communications}
Amateur radio stations are permitted to engage in various types of communications, including but not limited to:
\begin{itemize}
    \item Exchanging messages with other amateur stations.
    \item Retransmitting manned spacecraft communications.
    \item Transmitting control commands to space stations or radio-controlled craft.
    \item Notifying other amateurs of the availability of equipment for sale or trade, provided it is not done on a regular basis.
\end{itemize}

\subsection*{Prohibited Transmissions}
Certain types of transmissions are strictly prohibited under FCC regulations. These include:
\begin{itemize}
    \item Broadcasting, which refers to one-way transmissions intended for the general public.
    \item Transmissions encoded to obscure their meaning, except when sending control commands to space stations or radio-controlled craft.
    \item Transmissions containing indecent or obscene language.
    \item Communications with countries that have notified the ITU of their objection to such exchanges.
\end{itemize}

\subsection*{Communication Guidelines}
The following table summarizes the permitted and prohibited communications:

\begin{table}[h]
    \centering
    \caption{Communication Guidelines}
    \label{tab:comm_guidelines}
    \begin{tabular}{|l|l|}
        \hline
        \textbf{Permitted} & \textbf{Prohibited} \\
        \hline
        Exchanging messages with other amateur stations & Broadcasting \\
        Retransmitting manned spacecraft communications & Obscured transmissions (except for control commands) \\
        Transmitting control commands to space stations or radio-controlled craft & Indecent or obscene language \\
        Notifying other amateurs of equipment for sale or trade (non-regular basis) & Communications with objecting countries \\
        \hline
    \end{tabular}
\end{table}

\subsection*{Permissible Communications Guide}
The decision tree in Figure~\ref{fig:permitted_comms} provides a visual guide to determining whether a communication is permissible under FCC regulations.

\begin{figure}[h]
    \centering
    % \includegraphics[width=0.8\textwidth]{permitted_comms} % Placeholder for the image
    \caption{Permissible Communications Guide}
    \label{fig:permitted_comms}
    % Image prompt: Create a decision tree for determining permissible communications. The figure should include nodes for types of communications (e.g., broadcasting, control commands, equipment notifications) and branches leading to "Permitted" or "Prohibited" outcomes.
\end{figure}

\subsection*{Questions}
\begin{tcolorbox}[colback=gray!10!white,colframe=black!75!black,title={T1D01}]
    With which countries are FCC-licensed amateur radio stations prohibited from exchanging communications?
    \begin{enumerate}[label=\Alph*),noitemsep]
        \item \textbf{Any country whose administration has notified the International Telecommunication Union (ITU) that it objects to such communications}
        \item Any country whose administration has notified the American Radio Relay League (ARRL) that it objects to such communications
        \item Any country banned from such communications by the International Amateur Radio Union (IARU)
        \item Any country banned from making such communications by the American Radio Relay League (ARRL)
    \end{enumerate}
\end{tcolorbox}
FCC Part 97 prohibits communications with countries that have notified the ITU of their objection to such exchanges. This ensures compliance with international regulations.

%memory_trick T1D01

\begin{tcolorbox}[colback=gray!10!white,colframe=black!75!black,title={T1D02}]
    Under which of the following circumstances are one-way transmissions by an amateur station prohibited?
    \begin{enumerate}[label=\Alph*),noitemsep]
        \item In all circumstances
        \item \textbf{Broadcasting}
        \item International Morse Code Practice
        \item Telecommand or transmissions of telemetry
    \end{enumerate}
\end{tcolorbox}
One-way transmissions are prohibited when they constitute broadcasting, which is intended for the general public. Other one-way transmissions, such as telecommand or telemetry, are permitted.

%memory_trick T1D02

\begin{tcolorbox}[colback=gray!10!white,colframe=black!75!black,title={T1D03}]
    When is it permissible to transmit messages encoded to obscure their meaning?
    \begin{enumerate}[label=\Alph*),noitemsep]
        \item Only during contests
        \item Only when transmitting certain approved digital codes
        \item \textbf{Only when transmitting control commands to space stations or radio control craft}
        \item Never
    \end{enumerate}
\end{tcolorbox}
Encoded messages are only permitted when transmitting control commands to space stations or radio-controlled craft. This ensures that the primary purpose of amateur radio—communication—is not undermined.

%memory_trick T1D03

\begin{tcolorbox}[colback=gray!10!white,colframe=black!75!black,title={T1D04}]
    Under what conditions is an amateur station authorized to transmit music using a phone emission?
    \begin{enumerate}[label=\Alph*),noitemsep]
        \item \textbf{When incidental to an authorized retransmission of manned spacecraft communications}
        \item When the music produces no spurious emissions
        \item When transmissions are limited to less than three minutes per hour
        \item When the music is transmitted above 1280 MHz
    \end{enumerate}
\end{tcolorbox}
Music transmissions are only permitted when they are incidental to retransmitting manned spacecraft communications. This ensures that amateur radio is not used for entertainment purposes.

%memory_trick T1D04

\begin{tcolorbox}[colback=gray!10!white,colframe=black!75!black,title={T1D05}]
    When may amateur radio operators use their stations to notify other amateurs of the availability of equipment for sale or trade?
    \begin{enumerate}[label=\Alph*),noitemsep]
        \item Never
        \item When the equipment is not the personal property of either the station licensee, or the control operator, or their close relatives
        \item When no profit is made on the sale
        \item \textbf{When selling amateur radio equipment and not on a regular basis}
    \end{enumerate}
\end{tcolorbox}
Amateur radio operators may notify others of equipment for sale or trade, provided it is not done on a regular basis. This prevents the amateur radio service from being used as a commercial platform.

%memory_trick T1D05

\begin{tcolorbox}[colback=gray!10!white,colframe=black!75!black,title={T1D06}]
    What, if any, are the restrictions concerning transmission of language that may be considered indecent or obscene?
    \begin{enumerate}[label=\Alph*),noitemsep]
        \item The FCC maintains a list of words that are not permitted to be used on amateur frequencies
        \item \textbf{Any such language is prohibited}
        \item The ITU maintains a list of words that are not permitted to be used on amateur frequencies
        \item There is no such prohibition
    \end{enumerate}
\end{tcolorbox}
The FCC prohibits the transmission of any indecent or obscene language on amateur frequencies. This ensures that amateur radio remains a respectful and professional communication medium.

%memory_trick T1D06

\subsection*{Summary}
This section covered the basic operating rules for amateur radio stations, including:
\begin{itemize}
    \item \textbf{Permissible communications}: Exchanging messages, retransmitting spacecraft communications, and notifying others of equipment for sale or trade.
    \item \textbf{Prohibited transmissions}: Broadcasting, obscured transmissions (except for control commands), indecent or obscene language, and communications with objecting countries.
    \item \textbf{Basic operating restrictions}: Ensuring compliance with FCC regulations and international agreements.
\end{itemize}
