\section{License Privileges and Terms}
\label{sec:license_privileges}

This section discusses the privileges associated with different amateur radio license classes, as well as the terms and procedures for license renewal and modification. Understanding these aspects is crucial for maintaining compliance with FCC regulations and ensuring uninterrupted operation.

\subsection*{License Privileges}
Amateur radio licenses in the United States are divided into three main classes: Technician, General, and Extra. Each class grants specific operating privileges, with the Technician class being the entry-level license. The privileges increase as you progress to higher license classes, allowing access to more frequency bands and modes of operation. For a detailed comparison of the privileges, refer to Table~\ref{tab:license_privileges}.

\subsection*{License Terms and Renewal}
An FCC-issued amateur radio license is typically valid for ten years. After expiration, there is a two-year grace period during which the license can be renewed without retaking the examination. However, transmitting is not permitted during the grace period until the license is renewed. For more information on the renewal process, see the FCC's guidelines.

\subsection*{License Modifications}
License modifications, such as changes to call signs or operator information, can be requested through the FCC's Universal Licensing System (ULS). It is important to keep your license information up to date to avoid any issues with compliance.

\subsection*{Operating Restrictions}
Certain operating restrictions apply to amateur radio operators, depending on their license class and the specific frequency bands they are using. These restrictions are outlined in FCC Part 97 regulations.

\begin{figure}[h]
    \centering
    % \includegraphics[width=0.8\textwidth]{license_structure.png}
    \caption{Amateur License Class Structure}
    \label{fig:license_structure}
    % Image prompt: Create a diagram showing license class progression using Graphviz. The diagram should illustrate the hierarchy of license classes (Technician, General, Extra) and the privileges associated with each.
\end{figure}

\begin{table}[h]
    \centering
    \caption{License Class Privileges}
    \label{tab:license_privileges}
    \begin{tabular}{|l|l|l|}
        \hline
        \textbf{License Class} & \textbf{Frequency Bands} & \textbf{Operating Modes} \\
        \hline
        Technician & VHF/UHF & FM, CW, Digital \\
        General & HF, VHF/UHF & SSB, CW, Digital \\
        Extra & All amateur bands & All modes \\
        \hline
    \end{tabular}
    % Table prompt: Create a table comparing privileges of different license classes, including frequency bands and operating modes.
\end{table}

\subsection*{Questions}

\begin{tcolorbox}[colback=gray!10!white,colframe=black!75!black,title={T1C05}]
    Which of the following is a valid Technician class call sign format?
    \begin{enumerate}[label=\Alph*),noitemsep]
        \item \textbf{KF1XXX}
        \item KA1X
        \item W1XX
        \item All these choices are correct
    \end{enumerate}
\end{tcolorbox}
The correct format for a Technician class call sign is \textbf{KF1XXX}. The other options either do not follow the FCC's call sign structure or are not specific to the Technician class.

%memory_trick T1C05

\begin{tcolorbox}[colback=gray!10!white,colframe=black!75!black,title={T1C08}]
    What is the normal term for an FCC-issued amateur radio license?
    \begin{enumerate}[label=\Alph*),noitemsep]
        \item Five years
        \item Life
        \item \textbf{Ten years}
        \item Eight years
    \end{enumerate}
\end{tcolorbox}
An FCC-issued amateur radio license is valid for ten years. This is standard across all license classes.

%memory_trick T1C08

\begin{tcolorbox}[colback=gray!10!white,colframe=black!75!black,title={T1C09}]
    What is the grace period for renewal if an amateur license expires?
    \begin{enumerate}[label=\Alph*),noitemsep]
        \item \textbf{Two years}
        \item Three years
        \item Five years
        \item Ten years
    \end{enumerate}
\end{tcolorbox}
The grace period for renewing an expired amateur radio license is two years. During this time, you cannot transmit until the license is renewed.

%memory_trick T1C09

\begin{tcolorbox}[colback=gray!10!white,colframe=black!75!black,title={T1C10}]
    How soon after passing the examination for your first amateur radio license may you transmit on the amateur radio bands?
    \begin{enumerate}[label=\Alph*),noitemsep]
        \item Immediately on receiving your Certificate of Successful Completion of Examination (CSCE)
        \item As soon as your operator/station license grant appears on the ARRL website
        \item \textbf{As soon as your operator/station license grant appears in the FCC’s license database}
        \item As soon as you receive your license in the mail from the FCC
    \end{enumerate}
\end{tcolorbox}
You may begin transmitting as soon as your license grant appears in the FCC’s license database. This is the official record of your license status.

%memory_trick T1C10

\begin{tcolorbox}[colback=gray!10!white,colframe=black!75!black,title={T1C11}]
    If your license has expired and is still within the allowable grace period, may you continue to transmit on the amateur radio bands?
    \begin{enumerate}[label=\Alph*),noitemsep]
        \item Yes, for up to two years
        \item Yes, as soon as you apply for renewal
        \item Yes, for up to one year
        \item \textbf{No, you must wait until the license has been renewed}
    \end{enumerate}
\end{tcolorbox}
Transmitting is not allowed during the grace period until the license is officially renewed. This is to ensure compliance with FCC regulations.

%memory_trick T1C11

\subsection*{Summary}
This section covered the following key concepts:
\begin{itemize}
    \item \textbf{License privileges}: Different license classes grant varying levels of access to frequency bands and operating modes.
    \item \textbf{License terms and renewal}: Licenses are valid for ten years, with a two-year grace period for renewal.
    \item \textbf{License modifications}: Changes to call signs or operator information can be made through the FCC's ULS.
    \item \textbf{Operating restrictions}: Specific restrictions apply based on license class and frequency bands.
\end{itemize}
