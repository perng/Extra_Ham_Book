\section{Special Operations and Services}
\label{sec:special_ops}

\subsection*{Beacon and Space Station Operations}
In amateur radio, beacon operations and space station communications are specialized activities governed by FCC Part 97 regulations. A \textbf{beacon} is defined as an amateur station that transmits communications for the purpose of observing propagation or conducting related experimental activities. This is in contrast to other types of transmissions, such as weather bulletins or government markers, which are not considered beacons under Part 97.

A \textbf{space station}, on the other hand, is an amateur station located more than 50 km above Earth's surface. This definition excludes satellites that are not operated by amateur radio operators or those that do not meet the altitude requirement. Space stations are used for experimental and communication purposes, often involving amateur radio satellites.

\subsection*{Frequency Coordination}
Frequency coordination is a critical aspect of amateur radio operations, particularly for repeater and auxiliary stations. A \textbf{Frequency Coordinator} is a volunteer recognized by local amateur operators to recommend transmit/receive channels and other operational parameters. This role is essential to prevent interference and ensure efficient use of the radio spectrum. The selection of a Frequency Coordinator is made by amateur operators in a local or regional area whose stations are eligible to operate as repeaters or auxiliary stations.

\subsection*{Third-Party Communications}
Third-party communications involve messages transmitted by a control operator on behalf of another person. This type of communication is subject to specific restrictions, particularly when involving foreign stations. For example, the foreign station must be in a country with which the U.S. has a third-party agreement. Additionally, the licensed control operator is responsible for station identification during such communications.

\subsection*{Special Operations Overview}
\begin{figure}[h]
    \centering
    % \includegraphics[width=0.8\textwidth]{special_ops_diagram}
    \caption{Special Operations Overview}
    \label{fig:special_ops}
    % Diagram created using Graphviz. The figure shows the different types of special operations in amateur radio, including beacon operations, space station communications, and third-party communications. Each operation is represented as a node, with arrows indicating the flow of communication.
\end{figure}

\subsection*{Special Operations Requirements}
\begin{table}[h]
    \centering
    \begin{tabular}{|l|l|}
        \hline
        \textbf{Operation Type} & \textbf{Requirements} \\ \hline
        Beacon & Transmit for propagation observation or experiments \\ \hline
        Space Station & Located >50 km above Earth's surface \\ \hline
        Frequency Coordination & Volunteer coordinator recommended by local amateurs \\ \hline
        Third-Party Communications & Requires third-party agreement with foreign station \\ \hline
    \end{tabular}
    \caption{Special Operations Requirements}
    \label{tab:special_ops}
\end{table}

\subsection*{Questions}
\begin{tcolorbox}[colback=gray!10!white,colframe=black!75!black,title={T1A06}]
    What is the FCC Part 97 definition of a beacon?
    \begin{enumerate}[label=\Alph*),noitemsep]
        \item A government transmitter marking the amateur radio band edges
        \item A bulletin sent by the FCC to announce a national emergency
        \item A continuous transmission of weather information authorized in the amateur bands by the National Weather Service
        \item \textbf{An amateur station transmitting communications for the purposes of observing propagation or related experimental activities}
    \end{enumerate}
\end{tcolorbox}
According to FCC Part 97, a beacon is an amateur station used for propagation observation or experimental activities. Options A, B, and C describe other types of transmissions that do not meet the definition of a beacon.

%memory_trick T1A06

\begin{tcolorbox}[colback=gray!10!white,colframe=black!75!black,title={T1A07}]
    What is the FCC Part 97 definition of a space station?
    \begin{enumerate}[label=\Alph*),noitemsep]
        \item Any satellite orbiting Earth
        \item A manned satellite orbiting Earth
        \item \textbf{An amateur station located more than 50 km above Earth's surface}
        \item An amateur station using amateur radio satellites for relay of signals
    \end{enumerate}
\end{tcolorbox}
A space station, as defined by FCC Part 97, is an amateur station located more than 50 km above Earth's surface. Options A and B are incorrect because they do not specify the altitude requirement, and option D refers to the use of satellites rather than the station's location.

%memory_trick T1A07

\begin{tcolorbox}[colback=gray!10!white,colframe=black!75!black,title={T1A08}]
    Which of the following entities recommends transmit/receive channels and other parameters for auxiliary and repeater stations?
    \begin{enumerate}[label=\Alph*),noitemsep]
        \item Frequency Spectrum Manager appointed by the FCC
        \item \textbf{Volunteer Frequency Coordinator recognized by local amateurs}
        \item FCC Regional Field Office
        \item International Telecommunication Union
    \end{enumerate}
\end{tcolorbox}
The Volunteer Frequency Coordinator, recognized by local amateur operators, is responsible for recommending transmit/receive channels and other parameters for auxiliary and repeater stations. Options A, C, and D are incorrect because they do not involve local amateur operators.

%memory_trick T1A08

\begin{tcolorbox}[colback=gray!10!white,colframe=black!75!black,title={T1A09}]
    Who selects a Frequency Coordinator?
    \begin{enumerate}[label=\Alph*),noitemsep]
        \item The FCC Office of Spectrum Management and Coordination Policy
        \item The local chapter of the Office of National Council of Independent Frequency Coordinators
        \item \textbf{Amateur operators in a local or regional area whose stations are eligible to be repeater or auxiliary stations}
        \item FCC Regional Field Office
    \end{enumerate}
\end{tcolorbox}
Amateur operators in a local or regional area whose stations are eligible to operate as repeaters or auxiliary stations select the Frequency Coordinator. Options A, B, and D are incorrect because they do not involve local amateur operators.

%memory_trick T1A09

\begin{tcolorbox}[colback=gray!10!white,colframe=black!75!black,title={T1F07}]
    Which of the following restrictions apply when a non-licensed person is allowed to speak to a foreign station using a station under the control of a licensed amateur operator?
    \begin{enumerate}[label=\Alph*),noitemsep]
        \item The person must be a U.S. citizen
        \item \textbf{The foreign station must be in a country with which the U.S. has a third party agreement}
        \item The licensed control operator must do the station identification
        \item All these choices are correct
    \end{enumerate}
\end{tcolorbox}
The primary restriction is that the foreign station must be in a country with which the U.S. has a third-party agreement. Option A is incorrect because citizenship is not a requirement, and option C is only partially correct as it does not address the third-party agreement requirement.

%memory_trick T1F07

\begin{tcolorbox}[colback=gray!10!white,colframe=black!75!black,title={T1F08}]
    What is the definition of third party communications?
    \begin{enumerate}[label=\Alph*),noitemsep]
        \item \textbf{A message from a control operator to another amateur station control operator on behalf of another person}
        \item Amateur radio communications where three stations are in communications with one another
        \item Operation when the transmitting equipment is licensed to a person other than the control operator
        \item Temporary authorization for an unlicensed person to transmit on the amateur bands for technical experiments
    \end{enumerate}
\end{tcolorbox}
Third-party communications involve a control operator transmitting a message on behalf of another person. Options B, C, and D describe other scenarios that do not meet the definition of third-party communications.

%memory_trick T1F08

\subsection*{Summary}
This section covered key concepts in amateur radio special operations and services:
\begin{itemize}
    \item \textbf{Beacon operations}: Transmissions for propagation observation or experiments.
    \item \textbf{Space station communications}: Amateur stations located more than 50 km above Earth's surface.
    \item \textbf{Frequency coordination}: Managed by volunteer coordinators to prevent interference.
    \item \textbf{Third-party communications}: Messages transmitted by a control operator on behalf of another person, subject to specific restrictions.
\end{itemize}
