\section{License Requirements}
\label{sec:license_requirements}

\subsection*{Obtaining an Amateur Radio License}
To obtain an amateur radio license, an individual must pass an examination administered by a Volunteer Examiner Coordinator (VEC). The examination tests the applicant's knowledge of radio theory, regulations, and operating practices. Once the exam is successfully completed, the results are submitted to the Federal Communications Commission (FCC) for processing. The FCC then issues a license grant, which authorizes the individual to operate an amateur radio station. The license grant is recorded in the FCC's Universal Licensing System (ULS) database, which serves as the official record of the license.

\subsection*{FCC's Universal Licensing System (ULS)}
The FCC's Universal Licensing System (ULS) is an online database that manages all amateur radio licenses. The ULS allows licensees to apply for, renew, and modify their licenses. It also provides public access to license information, including call signs, license class, and expiration dates. The ULS is an essential tool for ensuring compliance with FCC regulations and for maintaining accurate records of amateur radio operators.

\begin{figure}[h]
    \centering
    % \includegraphics[width=0.8\textwidth]{licensing_process.png}
    \caption{Amateur Radio Licensing Process}
    \label{fig:licensing_process}
    % The figure should show a flowchart with the following steps:
    % 1. Study for the exam
    % 2. Take the exam with a VEC
    % 3. Pass the exam
    % 4. Submit results to FCC
    % 5. FCC issues license grant
    % 6. License recorded in ULS database
\end{figure}

\begin{table}[h]
    \centering
    \caption{Amateur Radio License Requirements}
    \label{tab:license_requirements}
    \begin{tabular}{|l|l|}
        \hline
        \textbf{Requirement} & \textbf{Description} \\
        \hline
        Examination & Pass a written exam administered by a VEC \\
        \hline
        License Grant & Issued by the FCC and recorded in the ULS \\
        \hline
        Call Sign & Assigned by the FCC, can be requested under vanity rules \\
        \hline
        License Class & Determines operating privileges \\
        \hline
    \end{tabular}
\end{table}

\subsection*{Questions}

\begin{tcolorbox}[colback=gray!10!white,colframe=black!75!black,title={T1A03}]
    What do the FCC rules state regarding the use of a phonetic alphabet for station identification in the Amateur Radio Service?
    \begin{enumerate}[label=\Alph*),noitemsep]
        \item It is required when transmitting emergency messages
        \item \textbf{It is encouraged}
        \item It is required when in contact with foreign stations
        \item All these choices are correct
    \end{enumerate}
\end{tcolorbox}
The use of a phonetic alphabet is encouraged by the FCC for clarity in communication, but it is not mandatory except in specific situations. The correct answer is \textbf{B}.

%memory_trick T1A03

\begin{tcolorbox}[colback=gray!10!white,colframe=black!75!black,title={T1A04}]
    How many operator/primary station license grants may be held by any one person?
    \begin{enumerate}[label=\Alph*),noitemsep]
        \item \textbf{One}
        \item No more than two
        \item One for each band on which the person plans to operate
        \item One for each permanent station location from which the person plans to operate
    \end{enumerate}
\end{tcolorbox}
The FCC rules state that an individual may hold only one operator/primary station license grant. The correct answer is \textbf{A}.

%memory_trick T1A04

\begin{tcolorbox}[colback=gray!10!white,colframe=black!75!black,title={T1A05}]
    What proves that the FCC has issued an operator/primary license grant?
    \begin{enumerate}[label=\Alph*),noitemsep]
        \item A printed copy of the certificate of successful completion of examination
        \item An email notification from the NCVEC granting the license
        \item \textbf{The license appears in the FCC ULS database}
        \item All these choices are correct
    \end{enumerate}
\end{tcolorbox}
The official proof of an FCC-issued license is its appearance in the ULS database. The correct answer is \textbf{C}.

%memory_trick T1A05

\begin{tcolorbox}[colback=gray!10!white,colframe=black!75!black,title={T1C02}]
    Who may select a desired call sign under the vanity call sign rules?
    \begin{enumerate}[label=\Alph*),noitemsep]
        \item Only a licensed amateur with a General or Amateur Extra Class license
        \item Only a licensed amateur with an Amateur Extra Class license
        \item Only a licensed amateur who has been licensed continuously for more than 10 years
        \item \textbf{Any licensed amateur}
    \end{enumerate}
\end{tcolorbox}
Any licensed amateur may request a vanity call sign, provided the desired call sign is available and meets FCC requirements. The correct answer is \textbf{D}.

%memory_trick T1C02

\begin{tcolorbox}[colback=gray!10!white,colframe=black!75!black,title={T1C04}]
    What may happen if the FCC is unable to reach you by email?
    \begin{enumerate}[label=\Alph*),noitemsep]
        \item Fine and suspension of operator license
        \item \textbf{Revocation of the station license or suspension of the operator license}
        \item Revocation of access to the license record in the FCC system
        \item Nothing; there is no such requirement
    \end{enumerate}
\end{tcolorbox}
If the FCC cannot reach a licensee by email, it may result in the revocation of the station license or suspension of the operator license. The correct answer is \textbf{B}.

%memory_trick T1C04

\subsection*{Summary}
This section covered the essential requirements for obtaining and maintaining an amateur radio license. Key concepts include:

\begin{itemize}
    \item \textbf{License grant requirements}: Passing an exam administered by a VEC and having the license recorded in the ULS database.
    \item \textbf{ULS database system}: The FCC's online system for managing amateur radio licenses.
    \item \textbf{Call sign requirements}: Call signs are assigned by the FCC and can be requested under vanity rules.
    \item \textbf{License limitations}: An individual may hold only one operator/primary station license grant.
\end{itemize}
