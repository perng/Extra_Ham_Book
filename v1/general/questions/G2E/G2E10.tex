\subsection{Establishing Contact with a Digital Messaging System Gateway Station}
\label{G2E10}

\begin{tcolorbox}[colback=gray!10!white,colframe=black!75!black,title=G2E10]
Which of the following is a way to establish contact with a digital messaging system gateway station?
\begin{enumerate}[label=\Alph*),noitemsep]
    \item Send an email to the system control operator
    \item Send QRL in Morse code
    \item Respond when the station broadcasts its SSID
    \item \textbf{Transmit a connect message on the station’s published frequency}
\end{enumerate}
\end{tcolorbox}

\subsubsection{Intuitive Explanation}
Imagine you’re trying to talk to a friend who’s using a special walkie-talkie that only understands certain messages. You can’t just shout their name or send them a text message—you need to use the right frequency and send the right kind of message. In this case, the correct way to get their attention is to send a connect message on the specific frequency they’re listening to. It’s like knocking on their door instead of yelling from across the street!

\subsubsection{Advanced Explanation}
In digital messaging systems, gateway stations operate on specific frequencies and protocols. To establish contact, you must adhere to the station’s communication standards. The correct method is to transmit a connect message on the station’s published frequency. This ensures that the gateway station recognizes your signal and initiates the communication process. 

The other options are incorrect because:
\begin{itemize}
    \item Sending an email (A) is not a direct method of communication with a radio gateway.
    \item Sending QRL in Morse code (B) is a query to check if the frequency is in use, not a method to establish contact.
    \item Responding when the station broadcasts its SSID (C) is passive and does not initiate contact.
\end{itemize}

% Diagram prompt: A diagram showing a user transmitting a connect message on a specific frequency to a gateway station, with other incorrect methods (email, Morse code, SSID response) shown as crossed out.