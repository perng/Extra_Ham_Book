\subsection{Inductance of Parallel Inductors}
\label{G5C10}

\begin{tcolorbox}[colback=gray!10!white,colframe=black!75!black,title=G5C10]
What is the inductance of three 10-millihenry inductors connected in parallel?
\begin{enumerate}[label=\Alph*),noitemsep]
    \item 0.30 henries
    \item 3.3 henries
    \item \textbf{3.3 millihenries}
    \item 30 millihenries
\end{enumerate}
\end{tcolorbox}

\subsubsection*{Intuitive Explanation}
Imagine you have three identical water pipes connected side by side. If you open all three pipes at the same time, water flows more easily than if you only opened one pipe. Similarly, when you connect inductors in parallel, the overall inductance decreases because the current has more paths to flow through. So, three 10-millihenry inductors in parallel will have a smaller inductance than just one inductor. The correct answer is 3.3 millihenries, which is like saying the water flows more easily when all three pipes are open.

\subsubsection*{Advanced Explanation}
When inductors are connected in parallel, the total inductance \( L_{\text{total}} \) is given by the reciprocal of the sum of the reciprocals of the individual inductances. Mathematically, this is expressed as:

\[
\frac{1}{L_{\text{total}}} = \frac{1}{L_1} + \frac{1}{L_2} + \frac{1}{L_3}
\]

Given that \( L_1 = L_2 = L_3 = 10 \) millihenries, we can substitute these values into the equation:

\[
\frac{1}{L_{\text{total}}} = \frac{1}{10} + \frac{1}{10} + \frac{1}{10} = \frac{3}{10}
\]

Taking the reciprocal of both sides to solve for \( L_{\text{total}} \):

\[
L_{\text{total}} = \frac{10}{3} \approx 3.3 \text{ millihenries}
\]

Thus, the total inductance of three 10-millihenry inductors connected in parallel is approximately 3.3 millihenries.

% Diagram prompt: A diagram showing three inductors connected in parallel with labels indicating the inductance values and the total inductance.