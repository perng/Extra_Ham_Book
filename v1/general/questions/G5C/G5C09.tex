\subsection{Capacitance of Series Capacitors}
\label{G5C09}

\begin{tcolorbox}[colback=gray!10!white,colframe=black!75!black,title=G5C09]
What is the capacitance of three 100-microfarad capacitors connected in series?
\begin{enumerate}[label=\Alph*),noitemsep]
    \item 0.33 microfarads
    \item 3.0 microfarads
    \item \textbf{33.3 microfarads}
    \item 300 microfarads
\end{enumerate}
\end{tcolorbox}

\subsubsection{Intuitive Explanation}
Imagine you have three water tanks connected in a row. Each tank can hold 100 liters of water. If you connect them in series, it's like making one big tank that can hold less water because the water has to flow through all three tanks. The total capacity of the tanks combined is less than each individual tank. Similarly, when capacitors are connected in series, the total capacitance decreases. So, three 100-microfarad capacitors in series give you a total capacitance of 33.3 microfarads.

\subsubsection{Advanced Explanation}
When capacitors are connected in series, the reciprocal of the total capacitance \( C_{\text{total}} \) is the sum of the reciprocals of the individual capacitances. Mathematically, this is expressed as:

\[
\frac{1}{C_{\text{total}}} = \frac{1}{C_1} + \frac{1}{C_2} + \frac{1}{C_3}
\]

Given that \( C_1 = C_2 = C_3 = 100 \) microfarads, we can substitute these values into the equation:

\[
\frac{1}{C_{\text{total}}} = \frac{1}{100} + \frac{1}{100} + \frac{1}{100} = \frac{3}{100}
\]

Taking the reciprocal of both sides to solve for \( C_{\text{total}} \):

\[
C_{\text{total}} = \frac{100}{3} \approx 33.3 \text{ microfarads}
\]

This calculation shows that the total capacitance of three 100-microfarad capacitors connected in series is approximately 33.3 microfarads.

% Diagram prompt: A diagram showing three capacitors connected in series with their respective capacitance values labeled, and the total capacitance calculated.