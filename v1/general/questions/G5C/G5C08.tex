\subsection{Equivalent Capacitance in Parallel}
\label{G5C08}

\begin{tcolorbox}[colback=gray!10!white,colframe=black!75!black,title=G5C08]
What is the equivalent capacitance of two 5.0-nanofarad capacitors and one 750-picofarad capacitor connected in parallel?
\begin{enumerate}[label=\Alph*),noitemsep]
    \item 576.9 nanofarads
    \item 1,733 picofarads
    \item 3,583 picofarads
    \item \textbf{10.750 nanofarads}
\end{enumerate}
\end{tcolorbox}

\subsubsection*{Intuitive Explanation}
Imagine you have three buckets (capacitors) that can hold water (electric charge). Two of them are big (5.0 nanofarads each), and one is small (750 picofarads). If you connect them all together in parallel, it's like combining the buckets into one giant bucket. The total amount of water the giant bucket can hold is just the sum of what each individual bucket can hold. So, you add up the capacities of the two big buckets and the small one, and voilà, you get the total capacity!

\subsubsection*{Advanced Explanation}
When capacitors are connected in parallel, the equivalent capacitance \( C_{\text{eq}} \) is the sum of the individual capacitances. Mathematically, this is expressed as:

\[
C_{\text{eq}} = C_1 + C_2 + C_3
\]

Given:
\[
C_1 = 5.0 \, \text{nF}, \quad C_2 = 5.0 \, \text{nF}, \quad C_3 = 750 \, \text{pF}
\]

First, convert all capacitances to the same unit. Here, we convert picofarads to nanofarads:
\[
750 \, \text{pF} = 0.750 \, \text{nF}
\]

Now, sum the capacitances:
\[
C_{\text{eq}} = 5.0 \, \text{nF} + 5.0 \, \text{nF} + 0.750 \, \text{nF} = 10.750 \, \text{nF}
\]

Thus, the equivalent capacitance is \( 10.750 \, \text{nF} \).

% Diagram Prompt: A diagram showing three capacitors connected in parallel with their respective capacitance values labeled.