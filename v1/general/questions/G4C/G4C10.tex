\subsection{Symptoms of Ground Loop in Audio Connections}
\label{G4C10}

\begin{tcolorbox}[colback=gray!10!white,colframe=black!75!black,title=G4C10]
What could be a symptom caused by a ground loop in your station’s audio connections?
\begin{enumerate}[label=\Alph*),noitemsep]
    \item \textbf{You receive reports of “hum” on your station’s transmitted signal}
    \item The SWR reading for one or more antennas is suddenly very high
    \item An item of station equipment starts to draw excessive amounts of current
    \item You receive reports of harmonic interference from your station
\end{enumerate}
\end{tcolorbox}

\subsubsection{Intuitive Explanation}
Imagine you’re trying to listen to your favorite song on the radio, but instead of clear music, you hear a constant annoying “hum” sound. This is like when you’re trying to talk to a friend, but someone keeps making a buzzing noise in the background. In radio stations, this “hum” can happen because of something called a ground loop. It’s like when two different paths for electricity don’t agree on where the ground is, and they start arguing, creating that annoying hum in your audio.

\subsubsection{Advanced Explanation}
A ground loop occurs when there are multiple grounding paths in an electrical system, leading to a difference in ground potential between different pieces of equipment. This potential difference can cause a small current to flow through the ground connections, which can introduce noise into the audio signal. The noise typically manifests as a low-frequency hum, often at 50 Hz or 60 Hz, depending on the local power grid frequency.

Mathematically, the noise voltage \( V_{\text{noise}} \) can be expressed as:
\[ V_{\text{noise}} = I_{\text{loop}} \times R_{\text{ground}} \]
where \( I_{\text{loop}} \) is the current flowing through the ground loop and \( R_{\text{ground}} \) is the resistance of the ground path.

To mitigate ground loops, it is essential to ensure that all equipment is grounded at a single point, reducing the potential for multiple ground paths. Additionally, using isolation transformers or balanced audio connections can help eliminate the noise introduced by ground loops.

% Prompt for diagram: A diagram showing a typical ground loop scenario in a radio station's audio connections, illustrating multiple grounding paths and the resulting noise.