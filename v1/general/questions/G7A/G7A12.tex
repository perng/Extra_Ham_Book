\subsection{Solid Core Transformer Symbol}
\label{G7A12}

\begin{tcolorbox}[colback=gray!10!white,colframe=black!75!black,title=G7A12]
Which symbol in Figure G7-1 represents a solid core transformer?
\begin{enumerate}[label=\Alph*),noitemsep]
    \item Symbol 4
    \item Symbol 7
    \item \textbf{Symbol 6}
    \item Symbol 1
\end{enumerate}
\end{tcolorbox}

\subsubsection{Intuitive Explanation}
Imagine you have a magical box that can change the strength of electricity. This box is called a transformer. Now, some transformers have a solid core inside them, like a solid piece of metal. In Figure G7-1, Symbol 6 is the one that shows this solid core transformer. It’s like the transformer is wearing a solid metal jacket!

\subsubsection{Advanced Explanation}
A transformer is an electrical device that transfers energy between two or more circuits through electromagnetic induction. The core of a transformer is typically made of ferromagnetic material, which helps in efficiently transferring the magnetic flux. In Figure G7-1, Symbol 6 represents a solid core transformer, where the core is a continuous piece of ferromagnetic material. This design minimizes energy losses and is commonly used in applications requiring high efficiency.

% Prompt for diagram: Generate a diagram showing various transformer symbols, highlighting Symbol 6 as the solid core transformer.