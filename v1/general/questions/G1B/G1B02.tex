\subsection{Compliance Conditions for Beacon Stations}
\label{G1B02}

\begin{tcolorbox}[colback=gray!10!white,colframe=black!75!black,title=G1B02]
With which of the following conditions must beacon stations comply?
\begin{enumerate}[label=\Alph*),noitemsep]
    \item \textbf{No more than one beacon station may transmit in the same band from the same station location}
    \item The frequency must be coordinated with the National Beacon Organization
    \item The frequency must be posted on the internet or published in a national periodical
    \item All these choices are correct
\end{enumerate}
\end{tcolorbox}

\subsubsection{Intuitive Explanation}
Imagine you and your friend both have walkie-talkies. If you both try to talk on the same channel at the same time, it’s going to be a big mess, right? That’s why beacon stations, which are like super fancy walkie-talkies, have a rule: only one beacon station can use the same frequency in the same area. This way, everyone can hear clearly without any confusion. So, the correct answer is A because it’s like saying, “Only one person can talk on this channel at a time!”

\subsubsection{Advanced Explanation}
Beacon stations are used to transmit signals for various purposes, such as navigation or calibration. To avoid interference, it is crucial that no two beacon stations transmit on the same frequency within the same geographical location. This is governed by regulatory bodies to ensure efficient use of the radio spectrum. 

The correct answer, A, reflects this regulatory requirement. Options B and C, while important in some contexts, are not universally mandated conditions for beacon stations. Option D is incorrect because not all the listed conditions are required.

In mathematical terms, the frequency allocation can be represented as:
\[
f_i \neq f_j \quad \text{for} \quad i \neq j
\]
where \( f_i \) and \( f_j \) are the frequencies of two different beacon stations in the same location.

% Diagram prompt: Generate a diagram showing two beacon stations in different locations transmitting on different frequencies, with a third beacon station in the same location as one of them but on a different frequency.