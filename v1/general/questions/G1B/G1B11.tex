\subsection{Determining Good Engineering and Good Amateur Practice}
\label{G1B11}

\begin{tcolorbox}[colback=gray!10!white,colframe=black!75!black,title=G1B11]
Who or what determines “good engineering and good amateur practice,” as applied to the operation of an amateur station in all respects not covered by the Part 97 rules?
\begin{enumerate}[label=\Alph*),noitemsep]
    \item \textbf{The FCC}
    \item The control operator
    \item The IEEE
    \item The ITU
\end{enumerate}
\end{tcolorbox}

\subsubsection{Intuitive Explanation}
Imagine you're building a treehouse with your friends. There are some basic rules you all agree on, like not using rotten wood. But what if you want to add a cool slide or a secret trapdoor? Who decides if that's okay? In the world of amateur radio, the FCC (Federal Communications Commission) is like the ultimate rule-maker. They decide what counts as good engineering and good amateur practice when the regular rules don't cover everything. So, if you're not sure if your radio setup is up to snuff, the FCC is the one to look to!

\subsubsection{Advanced Explanation}
In the context of amateur radio operations, the Federal Communications Commission (FCC) is the regulatory body that oversees and enforces the rules outlined in Part 97 of the Code of Federal Regulations (CFR). Part 97 provides a comprehensive set of guidelines for amateur radio operations, but it does not cover every possible scenario. In cases where specific rules are not provided, the FCC is responsible for determining what constitutes good engineering and good amateur practice.

This determination is based on a combination of technical standards, industry best practices, and the FCC's regulatory authority. The FCC ensures that amateur radio operators adhere to principles that promote safety, efficiency, and minimal interference with other communications services. This authority is derived from the Communications Act of 1934, which grants the FCC the power to regulate interstate and international communications by radio, television, wire, satellite, and cable.

In summary, while the control operator, IEEE (Institute of Electrical and Electronics Engineers), and ITU (International Telecommunication Union) play significant roles in the broader field of telecommunications, the FCC is the definitive authority in determining good engineering and good amateur practice for amateur radio operations in the United States.

% Prompt for diagram: A flowchart showing the hierarchy of regulatory bodies in amateur radio, with the FCC at the top, followed by IEEE and ITU, and then the control operator at the bottom.