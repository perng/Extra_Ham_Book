\subsection{G2B03: Good Amateur Practice During Propagation Changes}
\label{G2B03}

\begin{tcolorbox}[colback=gray!10!white,colframe=black!75!black,title=G2B03]
What is good amateur practice if propagation changes during a contact creating interference from other stations using the frequency?
\begin{enumerate}[label=\Alph*),noitemsep]
    \item Advise the interfering stations that you are on the frequency and that you have priority
    \item Decrease power and continue to transmit
    \item \textbf{Attempt to resolve the interference problem with the other stations in a mutually acceptable manner}
    \item Switch to the opposite sideband
\end{enumerate}
\end{tcolorbox}

\subsubsection{Intuitive Explanation}
Imagine you're playing a game of tag with your friends in a big park. Suddenly, another group of kids starts playing tag in the same area, and you all start bumping into each other. What do you do? You don't just yell, This is our spot, go away! That would be rude. Instead, you talk to the other group and figure out a way to share the space so everyone can have fun. That's exactly what you should do on the radio too! If other stations start interfering with your frequency, the best thing to do is to talk it out and find a solution that works for everyone.

\subsubsection{Advanced Explanation}
In amateur radio, propagation conditions can change due to factors like ionospheric variations, solar activity, or atmospheric conditions. These changes can cause interference from other stations using the same frequency. The correct approach, as outlined in the question, is to attempt to resolve the interference problem with the other stations in a mutually acceptable manner. This practice aligns with the principles of good amateur radio etiquette and the regulations set by governing bodies like the FCC.

When propagation changes, the signal paths may overlap, causing mutual interference. Simply asserting priority or reducing power may not resolve the issue effectively. Instead, engaging in a cooperative dialogue with the other operators can lead to a more efficient use of the frequency. This might involve shifting frequencies slightly, adjusting transmission times, or agreeing on a protocol to minimize interference.

Mathematically, the interference can be modeled as a function of signal overlap. If two signals \( S_1(t) \) and \( S_2(t) \) are present on the same frequency, the resulting signal \( S(t) \) can be expressed as:
\[ S(t) = S_1(t) + S_2(t) \]
To minimize interference, operators can adjust their frequencies such that \( S_1(t) \) and \( S_2(t) \) do not overlap significantly. This can be achieved by shifting frequencies by a small amount \( \Delta f \), resulting in:
\[ S(t) = S_1(t) e^{j2\pi \Delta f t} + S_2(t) \]
By coordinating with other operators, the value of \( \Delta f \) can be chosen to minimize mutual interference, ensuring clear communication for all parties involved.

% Diagram Prompt: Generate a diagram showing two overlapping signals on the same frequency and how shifting one signal by a small frequency offset reduces interference.