\subsection{LED Biasing for Light Emission}
\label{G6B08}

\begin{tcolorbox}[colback=gray!10!white,colframe=black!75!black,title=G6B08]
How is an LED biased when emitting light?
\begin{enumerate}[label=\Alph*),noitemsep]
    \item In the tunnel-effect region
    \item At the Zener voltage
    \item Reverse biased
    \item \textbf{Forward biased}
\end{enumerate}
\end{tcolorbox}

\subsubsection{Intuitive Explanation}
Imagine an LED as a tiny light bulb that only turns on when you push electricity through it the right way. If you try to push electricity the wrong way, it’s like trying to blow air into a balloon that’s already full—nothing happens! But when you push electricity the correct way (forward biased), it’s like opening a door for the electricity to flow through, and the LED lights up like a little star. So, the LED needs to be forward biased to shine!

\subsubsection{Advanced Explanation}
An LED (Light Emitting Diode) is a semiconductor device that emits light when current flows through it. For this to happen, the LED must be forward biased. Forward biasing means applying a positive voltage to the anode (p-side) and a negative voltage to the cathode (n-side) of the diode. This reduces the potential barrier at the p-n junction, allowing electrons to recombine with holes. During this recombination process, energy is released in the form of photons, which we perceive as light.

The forward bias voltage required for an LED is typically around 1.8 to 3.3 volts, depending on the material used in the semiconductor. If the LED is reverse biased, the potential barrier increases, and no significant current flows, hence no light is emitted. The tunnel-effect region and Zener voltage are not relevant to the normal operation of an LED.

Mathematically, the current \( I \) through an LED can be described by the Shockley diode equation:
\[ I = I_S \left( e^{\frac{V}{nV_T}} - 1 \right) \]
where:
\begin{itemize}
    \item \( I_S \) is the reverse saturation current,
    \item \( V \) is the voltage across the diode,
    \item \( n \) is the ideality factor (typically between 1 and 2),
    \item \( V_T \) is the thermal voltage (\( \approx 26 \) mV at room temperature).
\end{itemize}

When forward biased, \( V \) is positive, and the exponential term dominates, allowing significant current to flow and the LED to emit light.

% Diagram prompt: Generate a diagram showing the forward biasing of an LED, with the p-n junction, anode, cathode, and the direction of current flow.