\subsection{Functions in Software-Defined Radio (SDR)}
\label{G7C11}

\begin{tcolorbox}[colback=gray!10!white,colframe=black!75!black,title=G7C11]
Which of these functions is performed by software in a software-defined radio (SDR)?
\begin{enumerate}[label=\Alph*),noitemsep]
    \item Filtering
    \item Detection
    \item Modulation
    \item \textbf{All these choices are correct}
\end{enumerate}
\end{tcolorbox}

\subsubsection{Intuitive Explanation}
Imagine your radio is like a superhero that can change its powers depending on what you need. In a software-defined radio (SDR), the software is like the brain of the superhero. It can do all sorts of cool things like filtering out the noise (like blocking out the bad guys), detecting signals (like spotting a friend in a crowd), and even changing the way the signal sounds (like putting on a disguise). So, the software can do all these jobs—filtering, detection, and modulation—making it super versatile!

\subsubsection{Advanced Explanation}
In a software-defined radio (SDR), the traditional hardware components such as filters, detectors, and modulators are replaced by software algorithms. This allows for greater flexibility and reconfigurability. Here’s a breakdown of each function:

\begin{itemize}
    \item \textbf{Filtering}: In SDR, filtering is performed using digital signal processing (DSP) techniques. The software can implement various types of filters (e.g., low-pass, high-pass, band-pass) to remove unwanted frequencies from the signal.
    
    \item \textbf{Detection}: Detection involves identifying the presence of a signal within a given frequency range. Software algorithms can analyze the incoming signal and determine its characteristics, such as amplitude, frequency, and phase.
    
    \item \textbf{Modulation}: Modulation is the process of varying a carrier signal to encode information. In SDR, modulation is achieved through software that can implement different modulation schemes (e.g., AM, FM, QAM) by manipulating the signal digitally.
\end{itemize}

Since all these functions—filtering, detection, and modulation—are performed by software in an SDR, the correct answer is \textbf{D: All these choices are correct}.

% Prompt for diagram: A diagram showing the block diagram of a software-defined radio (SDR) with software components replacing traditional hardware components like filters, detectors, and modulators.