\subsection{Disqualification of a Third Party from Sending a Message via an Amateur Station}
\label{G1E01}

\begin{tcolorbox}[colback=gray!10!white,colframe=black!75!black,title=G1E01]
Which of the following would disqualify a third party from participating in sending a message via an amateur station?
\begin{enumerate}[label=\Alph*),noitemsep]
    \item \textbf{The third party’s amateur license has been revoked and not reinstated}
    \item The third party is not a US citizen
    \item The third party is speaking in a language other than English
    \item All these choices are correct
\end{enumerate}
\end{tcolorbox}

\subsubsection{Intuitive Explanation}
Imagine you're playing a game where you need a special pass to join in. If someone loses their pass and doesn't get it back, they can't play anymore, right? It's the same with sending messages through an amateur radio station. If someone's license (their special pass) is taken away and not given back, they can't send messages. It doesn't matter where they're from or what language they speak—it's all about that license!

\subsubsection{Advanced Explanation}
In the context of amateur radio operations, the Federal Communications Commission (FCC) regulates who can participate in transmitting messages. According to FCC rules, a third party can only participate in sending messages via an amateur station if they hold a valid amateur radio license. If their license has been revoked and not reinstated, they are no longer authorized to operate or participate in amateur radio communications. This is a strict requirement to ensure that all operators are knowledgeable about radio regulations and operating procedures.

The other options, such as citizenship and language, are not factors that disqualify a third party from participating. The FCC does not restrict amateur radio operations based on nationality or language, as long as the operator adheres to the rules and regulations governing amateur radio.

Therefore, the correct answer is \textbf{A}, as the revocation of a license directly impacts the eligibility of a third party to participate in amateur radio communications.

% Prompt for diagram: A flowchart showing the decision process for disqualifying a third party from sending a message via an amateur station, with branches for license status, citizenship, and language.