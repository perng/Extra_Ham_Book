\subsection{Automatically Controlled Stations and Band Communication}
\label{G1E11}

\begin{tcolorbox}[colback=gray!10!white,colframe=black!75!black,title=G1E11]
On what bands may automatically controlled stations transmitting RTTY or data emissions communicate with other automatically controlled digital stations?
\begin{enumerate}[label=\Alph*),noitemsep]
    \item On any band segment where digital operation is permitted
    \item Anywhere in the non-phone segments of the 10-meter or shorter wavelength bands
    \item Only in the non-phone Extra Class segments of the bands
    \item \textbf{Anywhere in the 6-meter or shorter wavelength bands, and in limited segments of some of the HF bands}
\end{enumerate}
\end{tcolorbox}

\subsubsection{Intuitive Explanation}
Imagine you have a robot friend who loves to send secret messages using a special type of code called RTTY or data emissions. Now, you and your robot friend want to chat, but you need to know where you're allowed to send these messages. Think of the radio bands as different playgrounds where you can play. The rules say that your robot friend can send messages in the 6-meter playground or any smaller playgrounds (shorter wavelengths). Additionally, there are a few specific spots in the bigger playgrounds (HF bands) where you can also send messages. So, the correct answer is that your robot friend can send messages in the 6-meter or shorter wavelength playgrounds, and in some special spots in the bigger playgrounds.

\subsubsection{Advanced Explanation}
Automatically controlled stations, which operate without human intervention, are permitted to transmit RTTY (Radio Teletype) or data emissions under specific regulations. According to the FCC rules, these stations can communicate with other automatically controlled digital stations in the following bands:

1. \textbf{6-meter or shorter wavelength bands}: This includes the 6-meter band (50-54 MHz), 2-meter band (144-148 MHz), and other VHF/UHF bands. These bands are typically used for local and regional communication due to their propagation characteristics.

2. \textbf{Limited segments of some HF bands}: HF bands (3-30 MHz) are generally used for long-distance communication. However, automatically controlled stations are restricted to specific segments within these bands to avoid interference with other types of communication.

The correct answer, therefore, is that automatically controlled stations can communicate anywhere in the 6-meter or shorter wavelength bands, and in limited segments of some of the HF bands. This ensures efficient use of the radio spectrum while minimizing interference with other users.

% Prompt for generating a diagram: A diagram showing the radio spectrum with highlighted sections for 6-meter and shorter wavelength bands, and limited segments of HF bands, would be beneficial for visual learners.