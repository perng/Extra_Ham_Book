\subsection{Preparing to Use a New Digital Protocol on the Air}
\label{G1C07}

\begin{tcolorbox}[colback=gray!10!white,colframe=black!75!black,title=G1C07]
What must be done before using a new digital protocol on the air?
\begin{enumerate}[label=\Alph*),noitemsep]
    \item Type-certify equipment to FCC standards
    \item Obtain an experimental license from the FCC
    \item \textbf{Publicly document the technical characteristics of the protocol}
    \item Submit a rule-making proposal to the FCC describing the codes and methods of the technique
\end{enumerate}
\end{tcolorbox}

\subsubsection{Intuitive Explanation}
Imagine you’ve just invented a new secret handshake that you want to use with your friends. Before you start using it, you’d probably want to explain how it works so everyone knows what to expect, right? Similarly, when using a new digital protocol on the air, it’s important to publicly document how it works. This way, other radio operators can understand and use it correctly, and there’s no confusion or interference. It’s like sharing the rules of a new game so everyone can play fair!

\subsubsection{Advanced Explanation}
Before deploying a new digital protocol in radio communications, it is essential to publicly document its technical characteristics. This documentation ensures transparency and allows other operators to understand the protocol’s specifications, such as modulation techniques, data rates, and error correction methods. This step is crucial for maintaining interoperability and minimizing potential interference with other communications.

The Federal Communications Commission (FCC) does not require type-certification for all digital protocols, nor is an experimental license always necessary. However, publicly documenting the protocol aligns with regulatory expectations and promotes a cooperative environment among radio operators. This practice also facilitates the adoption of new technologies while ensuring compliance with existing regulations.

% Prompt for diagram: A flowchart showing the steps to deploy a new digital protocol, including documentation, testing, and regulatory compliance.