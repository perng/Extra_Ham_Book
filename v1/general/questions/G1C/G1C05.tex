\subsection{Transmitter Power Limit on the 28 MHz Band for a General Class Control Operator}
\label{G1C05}

\begin{tcolorbox}[colback=gray!10!white,colframe=black!75!black,title=G1C05]
What is the limit for transmitter power on the 28 MHz band for a General Class control operator?
\begin{enumerate}[label=\Alph*),noitemsep]
    \item 100 watts PEP output
    \item 1000 watts PEP output
    \item \textbf{1500 watts PEP output}
    \item 2000 watts PEP output
\end{enumerate}
\end{tcolorbox}

\subsubsection{Intuitive Explanation}
Imagine you're playing with a super loud speaker in your backyard. You want to make sure it's not too loud, so you don't annoy your neighbors or break any rules. In the world of radio, the 28 MHz band is like your backyard, and the transmitter power is how loud your speaker is. For a General Class control operator, the rule is that your speaker can go up to 1500 watts PEP output. That's like turning the volume up to 1500 on your speaker—loud enough to be heard, but not so loud that it causes trouble!

\subsubsection{Advanced Explanation}
In the context of amateur radio operations, the Federal Communications Commission (FCC) sets specific power limits for different license classes to ensure efficient use of the radio spectrum and to minimize interference. The 28 MHz band, also known as the 10-meter band, is a popular frequency range for amateur radio operators.

The term PEP stands for Peak Envelope Power, which is the maximum power level that occurs during a transmission. For a General Class control operator, the FCC limits the transmitter power on the 28 MHz band to 1500 watts PEP output. This limit is designed to balance the need for effective communication with the need to prevent excessive interference with other users of the spectrum.

Mathematically, the power limit can be expressed as:
\[
P_{\text{max}} = 1500 \text{ watts PEP}
\]
where \( P_{\text{max}} \) is the maximum allowable transmitter power.

Understanding this limit is crucial for operators to comply with regulations and to operate their equipment safely and effectively. Exceeding this limit can result in penalties and can cause interference with other communications.

% Prompt for generating a diagram: A diagram showing the 28 MHz band and the power limits for different license classes could be helpful here.