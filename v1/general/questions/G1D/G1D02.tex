\subsection{G1D02: License Examinations Administered by a General Class Volunteer Examiner}
\label{G1D02}

\begin{tcolorbox}[colback=gray!10!white,colframe=black!75!black,title=G1D02]
What license examinations may you administer as an accredited Volunteer Examiner holding a General class operator license?
\begin{enumerate}[label=\Alph*),noitemsep]
    \item General and Technician
    \item None, only Amateur Extra class licensees may be accredited
    \item \textbf{Technician only}
    \item Amateur Extra, General, and Technician
\end{enumerate}
\end{tcolorbox}

\subsubsection{Intuitive Explanation}
Imagine you're a teacher, but instead of teaching math or science, you're helping people get their ham radio licenses. Now, if you're a General Class teacher, you can only help students get their Technician license. It's like being a middle school teacher—you can teach middle schoolers, but you can't teach high schoolers or college students. So, as a General Class Volunteer Examiner, your job is to help people pass the Technician exam, and that's it!

\subsubsection{Advanced Explanation}
In the context of amateur radio licensing in the United States, the Federal Communications Commission (FCC) categorizes Volunteer Examiners (VEs) based on their license class. A General class operator license allows the holder to administer only the Technician class license examinations. This is because the General class license does not provide the necessary authority to administer higher-level exams, such as the General or Amateur Extra class exams. Only individuals holding an Amateur Extra class license are authorized to administer all three levels of examinations: Technician, General, and Amateur Extra. Therefore, the correct answer is that a General class VE can only administer the Technician exam.

% Diagram Prompt: A flowchart showing the hierarchy of amateur radio license classes and the corresponding exams that each class of Volunteer Examiner can administer.