\subsection{Regulations for Remote Control Operation of a Station in South America from the US}
\label{G1D12}

\begin{tcolorbox}[colback=gray!10!white,colframe=black!75!black,title=G1D12]
When operating a station in South America by remote control over the internet from the US, what regulations apply?
\begin{enumerate}[label=\Alph*),noitemsep]
    \item Those of both the remote station’s country and the FCC
    \item Those of the remote station’s country and the FCC’s third-party regulations
    \item \textbf{Only those of the remote station’s country}
    \item Only those of the FCC
\end{enumerate}
\end{tcolorbox}

\subsubsection{Intuitive Explanation}
Imagine you’re playing a video game where you control a character in another country. Even though you’re sitting in your living room, the rules of the game are set by the country where your character is located. Similarly, when you operate a radio station in South America from the US, you have to follow the rules of the country where the station is located, not the rules of the US. It’s like saying, “When in Rome, do as the Romans do!”

\subsubsection{Advanced Explanation}
When operating a radio station remotely, the regulatory framework is determined by the jurisdiction in which the station is physically located. In this case, the station is in South America, so the regulations of that specific country apply. The Federal Communications Commission (FCC) in the US does not have jurisdiction over operations conducted in foreign countries, even if the operator is located in the US. 

The key concept here is \textit{territorial jurisdiction}, which means that the laws and regulations of a country apply to all activities conducted within its borders. Therefore, the operator must comply with the licensing, operational, and technical standards set by the regulatory authority in South America. 

Additionally, the FCC’s third-party regulations, which govern the use of remote control operations within the US, do not extend to operations conducted in foreign countries. This is because the FCC’s authority is limited to the United States and its territories.

% Prompt for generating a diagram: A map showing the US and South America with arrows indicating the remote control operation and labels for the applicable regulations.