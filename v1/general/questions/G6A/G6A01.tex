\subsection{Minimum Discharge Voltage for Lead-Acid Battery}
\label{G6A01}

\begin{tcolorbox}[colback=gray!10!white,colframe=black!75!black,title=G6A01]
What is the minimum allowable discharge voltage for maximum life of a standard 12-volt lead-acid battery?
\begin{enumerate}[label=\Alph*),noitemsep]
    \item 6 volts
    \item 8.5 volts
    \item \textbf{10.5 volts}
    \item 12 volts
\end{enumerate}
\end{tcolorbox}

\subsubsection*{Intuitive Explanation}
Imagine your lead-acid battery is like a water bottle. If you drink all the water, the bottle is empty, and it’s not good for the bottle to stay empty for too long. Similarly, if you let the battery discharge too much, it’s like emptying the bottle completely, and that’s bad for the battery’s health. To keep the battery happy and healthy, you should stop using it when it reaches 10.5 volts. That’s like leaving a little water in the bottle so it doesn’t get damaged.

\subsubsection*{Advanced Explanation}
A standard 12-volt lead-acid battery consists of six cells, each with a nominal voltage of 2 volts. The minimum allowable discharge voltage is crucial to prevent sulfation, a process where lead sulfate crystals form on the battery plates, reducing the battery's capacity and lifespan. The minimum discharge voltage for each cell is typically around 1.75 volts. Therefore, for a 12-volt battery:

\[
\text{Minimum Discharge Voltage} = 6 \text{ cells} \times 1.75 \text{ volts/cell} = 10.5 \text{ volts}
\]

Discharging the battery below this voltage can cause irreversible damage, leading to a shorter battery life. Maintaining the discharge voltage above 10.5 volts ensures optimal performance and longevity of the battery.

% Prompt for generating a diagram: A diagram showing the voltage levels of a 12-volt lead-acid battery with labels for nominal voltage, minimum discharge voltage, and fully charged voltage.