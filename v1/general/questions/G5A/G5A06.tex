\subsection{Capacitor Reaction to AC}
\label{G5A06}

\begin{tcolorbox}[colback=gray!10!white,colframe=black!75!black,title=G5A06]
How does a capacitor react to AC?
\begin{enumerate}[label=\Alph*),noitemsep]
    \item \textbf{As the frequency of the applied AC increases, the reactance decreases}
    \item As the frequency of the applied AC increases, the reactance increases
    \item As the amplitude of the applied AC increases, the reactance increases
    \item As the amplitude of the applied AC increases, the reactance decreases
\end{enumerate}
\end{tcolorbox}

\subsubsection{Intuitive Explanation}
Imagine a capacitor as a tiny, magical sponge that soaks up electric charge. When you give it an alternating current (AC), it starts to wiggle back and forth, trying to keep up with the changing direction of the current. Now, if you crank up the speed of these wiggles (that's the frequency), the sponge gets better at soaking up the charge and doesn't resist as much. So, the faster the wiggles, the less the sponge fights back—meaning the reactance decreases. It's like the sponge is saying, Bring it on, I can handle this!

\subsubsection{Advanced Explanation}
The reactance of a capacitor in an AC circuit is given by the formula:
\[
X_C = \frac{1}{2\pi f C}
\]
where \(X_C\) is the capacitive reactance, \(f\) is the frequency of the AC signal, and \(C\) is the capacitance of the capacitor. 

From the formula, it's clear that \(X_C\) is inversely proportional to the frequency \(f\). This means that as the frequency increases, the reactance decreases. This relationship is crucial in understanding how capacitors behave in AC circuits, especially in filtering and tuning applications.

The amplitude of the AC signal does not affect the reactance directly, as the reactance is solely dependent on the frequency and the capacitance. Therefore, changes in amplitude do not influence \(X_C\).

% Diagram Prompt: Generate a diagram showing the relationship between frequency and capacitive reactance, with frequency on the x-axis and reactance on the y-axis, illustrating the inverse relationship.