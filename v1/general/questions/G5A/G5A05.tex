\subsection{Inductor Reaction to AC}
\label{G5A05}

\begin{tcolorbox}[colback=gray!10!white,colframe=black!75!black,title=G5A05]
How does an inductor react to AC?
\begin{enumerate}[label=\Alph*),noitemsep]
    \item As the frequency of the applied AC increases, the reactance decreases
    \item As the amplitude of the applied AC increases, the reactance increases
    \item As the amplitude of the applied AC increases, the reactance decreases
    \item \textbf{As the frequency of the applied AC increases, the reactance increases}
\end{enumerate}
\end{tcolorbox}

\subsubsection{Intuitive Explanation}
Imagine an inductor is like a bouncer at a club. The bouncer doesn't like changes in the crowd (current) and tries to slow them down. When the music (frequency) gets faster, the bouncer works harder to keep the crowd in check. So, the faster the music, the more the bouncer resists the crowd. That's why, as the frequency of the AC increases, the inductor's reactance (resistance to change) also increases.

\subsubsection{Advanced Explanation}
The reactance of an inductor, denoted as \( X_L \), is given by the formula:
\[
X_L = 2\pi f L
\]
where:
\begin{itemize}
    \item \( X_L \) is the inductive reactance in ohms ($\Omega$)
    \item \( f \) is the frequency of the AC signal in hertz (Hz)
    \item \( L \) is the inductance in henries (H)
\end{itemize}

From the formula, it is clear that the inductive reactance \( X_L \) is directly proportional to the frequency \( f \). Therefore, as the frequency of the applied AC increases, the reactance of the inductor increases. This is because the inductor opposes changes in current, and higher frequencies mean more rapid changes, leading to greater opposition.

The amplitude of the AC signal does not affect the reactance of the inductor. Reactance is purely a function of frequency and inductance, not the amplitude of the voltage or current.

% Diagram Prompt: Generate a diagram showing the relationship between frequency and inductive reactance, with frequency on the x-axis and reactance on the y-axis, illustrating the linear increase in reactance with frequency.