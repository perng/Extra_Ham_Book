\subsection{Impedance Definition}
\label{G5A08}

\begin{tcolorbox}[colback=gray!10!white,colframe=black!75!black,title=G5A08]
What is impedance?  
\begin{enumerate}[label=\Alph*),noitemsep]
    \item The ratio of current to voltage
    \item The product of current and voltage
    \item \textbf{The ratio of voltage to current}
    \item The product of current and reactance
\end{enumerate}
\end{tcolorbox}

\subsubsection{Intuitive Explanation}
Imagine you’re trying to push a shopping cart through a crowded store. The crowd represents resistance, and the force you need to apply to move the cart is like voltage. The speed at which the cart moves is like current. Impedance is like the total pushback you feel from the crowd and the cart’s wheels combined. It’s the ratio of how much force you need (voltage) to how fast the cart moves (current). So, impedance is simply voltage divided by current!

\subsubsection{Advanced Explanation}
Impedance (\(Z\)) is a complex quantity that describes the opposition a circuit presents to the flow of alternating current (AC). It is defined as the ratio of voltage (\(V\)) to current (\(I\)) in an AC circuit:  
\[
Z = \frac{V}{I}
\]  
Impedance extends the concept of resistance to AC circuits, incorporating both resistance (\(R\)) and reactance (\(X\)). Reactance arises from the effects of inductance and capacitance in the circuit. The total impedance is given by:  
\[
Z = R + jX
\]  
where \(j\) is the imaginary unit. The magnitude of impedance is:  
\[
|Z| = \sqrt{R^2 + X^2}
\]  
This relationship shows how impedance combines resistive and reactive components to oppose current flow in AC circuits.

% [Prompt for diagram: A diagram showing a simple AC circuit with a resistor, inductor, and capacitor, labeled with voltage, current, and impedance values.]