\subsection{Length of a 1/2 Wave Dipole Antenna}
\label{G9B11}

\begin{tcolorbox}[colback=gray!10!white,colframe=black!75!black,title=G9B11]
What is the approximate length for a 1/2 wave dipole antenna cut for 3.550 MHz?
\begin{enumerate}[label=\Alph*),noitemsep]
    \item 42 feet
    \item 84 feet
    \item \textbf{132 feet}
    \item 263 feet
\end{enumerate}
\end{tcolorbox}

\subsubsection{Intuitive Explanation}
Imagine you have a piece of string that you want to cut to the perfect length so it can wiggle just right when you shake it at a certain speed. For a radio antenna, this wiggling is actually the radio waves it sends out. The length of the antenna needs to match the wiggle speed (or frequency) of the radio waves. For a frequency of 3.550 MHz, the antenna needs to be about 132 feet long to wiggle perfectly. Think of it like tuning a guitar string to the right note!

\subsubsection{Advanced Explanation}
The length of a half-wave dipole antenna can be calculated using the formula:

\[
L = \frac{492}{f}
\]

where \( L \) is the length in feet and \( f \) is the frequency in MHz. For a frequency of 3.550 MHz:

\[
L = \frac{492}{3.550} \approx 138.6 \text{ feet}
\]

However, due to various factors like the velocity factor of the wire, the actual length is slightly shorter. The closest option to this calculated value is 132 feet. 

A half-wave dipole antenna is designed to resonate at a specific frequency, meaning it efficiently radiates or receives radio waves at that frequency. The length of the antenna is crucial because it determines the wavelength of the radio waves it interacts with. The formula above is derived from the relationship between the speed of light, frequency, and wavelength.

% Diagram Prompt: Generate a diagram showing a half-wave dipole antenna with labeled dimensions and the corresponding wavelength.