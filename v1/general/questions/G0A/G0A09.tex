\subsection{Measuring RF Field Strength}
\label{G0A09}

\begin{tcolorbox}[colback=gray!10!white,colframe=black!75!black,title=G0A09]
What type of instrument can be used to accurately measure an RF field strength?
\begin{enumerate}[label=\Alph*),noitemsep]
    \item A receiver with digital signal processing (DSP) noise reduction
    \item \textbf{A calibrated field strength meter with a calibrated antenna}
    \item An SWR meter with a peak-reading function
    \item An oscilloscope with a high-stability crystal marker generator
\end{enumerate}
\end{tcolorbox}

\subsubsection{Intuitive Explanation}
Imagine you're trying to measure how strong a radio signal is in your backyard. You wouldn't use a fancy noise-canceling headphone or a ruler, right? Instead, you'd use a special tool designed just for this job—like a field strength meter with a calibrated antenna. It's like using a thermometer to measure temperature instead of guessing by how hot your forehead feels. This tool is specifically made to give you an accurate reading of how strong the radio signal is.

\subsubsection{Advanced Explanation}
To accurately measure RF (Radio Frequency) field strength, a calibrated field strength meter with a calibrated antenna is essential. The field strength meter is designed to measure the electric field intensity in volts per meter (V/m) or magnetic field intensity in amperes per meter (A/m). Calibration ensures that the measurements are accurate and traceable to international standards. The antenna must also be calibrated to ensure it responds correctly to the RF field being measured.

The field strength meter typically consists of a detector, amplifier, and display unit. The detector converts the RF signal into a measurable voltage or current, which is then amplified and displayed. The calibration process involves comparing the meter's readings to a known standard under controlled conditions.

Mathematically, the electric field strength \( E \) can be expressed as:
\[ E = \frac{V}{d} \]
where \( V \) is the voltage induced in the antenna and \( d \) is the distance from the antenna to the point of measurement.

Other instruments like receivers with DSP noise reduction, SWR meters, or oscilloscopes are not designed to measure field strength directly and would not provide accurate readings for this purpose.

% Diagram Prompt: Generate a diagram showing a calibrated field strength meter with a calibrated antenna measuring an RF field.