\subsection{RF Exposure Limits from Directional Antennas}
\label{G0A10}

\begin{tcolorbox}[colback=gray!10!white,colframe=black!75!black,title=G0A10]
What should be done if evaluation shows that a neighbor might experience more than the allowable limit of RF exposure from the main lobe of a directional antenna?
\begin{enumerate}[label=\Alph*),noitemsep]
    \item Change to a non-polarized antenna with higher gain
    \item Use an antenna with a higher front-to-back ratio
    \item \textbf{Take precautions to ensure that the antenna cannot be pointed in their direction when they are present}
    \item All these choices are correct
\end{enumerate}
\end{tcolorbox}

\subsubsection{Intuitive Explanation}
Imagine you have a super bright flashlight (the directional antenna) that you use to light up your backyard. But if you accidentally shine it into your neighbor's window, it might bother them too much. The rules say you can't let that happen! So, what do you do? You make sure you don't point the flashlight at their window when they're around. Simple, right? That's what this question is about—making sure your flashlight doesn't bother your neighbor.

\subsubsection{Advanced Explanation}
When dealing with RF exposure from a directional antenna, the main lobe is the primary source of radiation. Regulatory bodies set limits on the amount of RF exposure that individuals can safely experience. If an evaluation indicates that a neighbor might be exposed to RF levels exceeding these limits, it is crucial to mitigate the risk. 

The correct approach is to ensure that the antenna is not directed towards the neighbor when they are present. This can be achieved by implementing physical barriers, adjusting the antenna's orientation, or using automated systems to control the antenna's direction. 

Changing to a non-polarized antenna with higher gain (Option A) or using an antenna with a higher front-to-back ratio (Option B) might reduce exposure in certain directions but does not guarantee compliance with RF exposure limits. Therefore, the most effective and direct solution is to control the antenna's directionality, as stated in Option C.

% Diagram prompt: Generate a diagram showing a directional antenna with its main lobe and the neighbor's location, illustrating the concept of RF exposure limits.