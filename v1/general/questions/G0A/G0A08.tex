\subsection{RF Safety Compliance Steps}
\label{G0A08}

\begin{tcolorbox}[colback=gray!10!white,colframe=black!75!black,title=G0A08]
Which of the following steps must an amateur operator take to ensure compliance with RF safety regulations?
\begin{enumerate}[label=\Alph*),noitemsep]
    \item Post a copy of FCC Part 97.13 in the station
    \item Notify neighbors within a 100-foot radius of the antenna of the existence of the station and power levels
    \item \textbf{Perform a routine RF exposure evaluation and prevent access to any identified high exposure areas}
    \item All these choices are correct
\end{enumerate}
\end{tcolorbox}

\subsubsection{Intuitive Explanation}
Alright, imagine you’re playing with a super powerful flashlight. You wouldn’t shine it directly into someone’s eyes, right? That’s because it could hurt them. Similarly, when you’re using a radio transmitter, you need to make sure the radio waves (like the flashlight beam) aren’t too strong in certain areas where people might be. So, the smart thing to do is check how strong the radio waves are and make sure no one gets too close to the super strong spots. That’s what option C is all about—keeping everyone safe by checking and controlling the radio wave exposure.

\subsubsection{Advanced Explanation}
To ensure compliance with RF safety regulations, an amateur operator must evaluate the potential for RF exposure and take necessary precautions. This involves performing an RF exposure assessment, which calculates the power density and specific absorption rate (SAR) in the vicinity of the antenna. The evaluation considers factors such as transmitter power, antenna gain, and distance from the antenna. 

The power density \( S \) can be calculated using the formula:
\[
S = \frac{P \cdot G}{4 \pi r^2}
\]
where \( P \) is the transmitted power, \( G \) is the antenna gain, and \( r \) is the distance from the antenna. If the calculated power density exceeds the permissible exposure limits defined by the FCC, the operator must implement measures to restrict access to high exposure areas, such as erecting barriers or posting warning signs. 

Option C correctly identifies this critical step, while the other options either do not directly address RF safety or are not required by FCC regulations.

% Diagram Prompt: Generate a diagram showing the power density distribution around an antenna, highlighting high exposure areas and safe zones.