\subsection{Power Consumption in a DC Circuit}
\label{G5B03}

\begin{tcolorbox}[colback=gray!10!white,colframe=black!75!black,title=G5B03]
How many watts of electrical power are consumed if 400 VDC is supplied to an 800-ohm load?
\begin{enumerate}[label=\Alph*),noitemsep]
    \item 0.5 watts
    \item \textbf{200 watts}
    \item 400 watts
    \item 3200 watts
\end{enumerate}
\end{tcolorbox}

\subsubsection{Intuitive Explanation}
Imagine you have a water hose (the voltage) and a sponge (the load). The water pressure is 400 units, and the sponge is 800 units thick. The question is asking how much water (power) the sponge can soak up. If you think about it, the sponge isn’t too thick, so it won’t soak up a crazy amount of water. The correct answer is 200 watts, which is like saying the sponge soaked up a moderate amount of water—not too little, not too much.

\subsubsection{Advanced Explanation}
To calculate the power consumed in a DC circuit, we use the formula:
\[
P = \frac{V^2}{R}
\]
where \( P \) is the power in watts, \( V \) is the voltage in volts, and \( R \) is the resistance in ohms.

Given:
\[
V = 400 \, \text{V}, \quad R = 800 \, \Omega
\]

Substitute the values into the formula:
\[
P = \frac{400^2}{800} = \frac{160000}{800} = 200 \, \text{W}
\]

Thus, the power consumed is 200 watts.

This formula is derived from Ohm's Law, which relates voltage, current, and resistance in an electrical circuit. The power dissipated in a resistor is proportional to the square of the voltage across it and inversely proportional to its resistance.

% Diagram prompt: A simple circuit diagram showing a voltage source connected to a resistor with labels for voltage (V) and resistance (R).