\subsection{QRP Operation}
\label{G2D10}

\begin{tcolorbox}[colback=gray!10!white,colframe=black!75!black,title=G2D10]
What is QRP operation?
\begin{enumerate}[label=\Alph*),noitemsep]
    \item Remote piloted model control
    \item \textbf{Low-power transmit operation}
    \item Transmission using Quick Response Protocol
    \item Traffic relay procedure net operation
\end{enumerate}
\end{tcolorbox}

\subsubsection{Intuitive Explanation}
Imagine you're trying to talk to your friend across the playground, but instead of shouting at the top of your lungs, you decide to whisper. QRP operation is like that whisper in the world of radio communication. It's all about using very low power to send your message. Why would anyone do that? Well, it's a fun challenge for radio enthusiasts, and it can also save battery life if you're out in the wild with limited power. Plus, it's like being a radio ninja—sneaky and efficient!

\subsubsection{Advanced Explanation}
QRP operation refers to the practice of transmitting radio signals at low power levels, typically 5 watts or less for CW (Morse code) and 10 watts or less for SSB (Single Side Band) voice communication. The term QRP is derived from the Q-code, a standardized set of three-letter codes used in radio communication, where QRP means reduce power.

The primary advantage of QRP operation is the reduced power consumption, which is particularly beneficial for portable or emergency operations where power sources are limited. Additionally, QRP operation can be a test of skill, as it requires efficient antenna systems and careful tuning to ensure that the signal reaches its destination despite the low power.

Mathematically, the power output \( P \) in QRP operation is constrained by:
\[ P \leq 5 \text{ watts (CW)} \]
\[ P \leq 10 \text{ watts (SSB)} \]

This low-power transmission challenges operators to optimize their equipment and techniques to achieve successful communication over long distances.

% Diagram Prompt: Generate a diagram showing a comparison between high-power and low-power (QRP) radio transmissions, illustrating the difference in signal strength and range.