\subsection{G1A07: Amateur Frequencies for CW Emissions in the 10-Meter Band}
\label{G1A07}

\begin{tcolorbox}[colback=gray!10!white,colframe=black!75!black,title=G1A07]
On which amateur frequencies in the 10-meter band may stations with a General class control operator transmit CW emissions?
\begin{enumerate}[label=\Alph*),noitemsep]
    \item 28.000 MHz to 28.025 MHz only
    \item 28.000 MHz to 28.300 MHz only
    \item 28.025 MHz to 28.300 MHz only
    \item \textbf{The entire band}
\end{enumerate}
\end{tcolorbox}

\subsubsection{Intuitive Explanation}
Imagine the 10-meter band is like a big playground, and CW (Continuous Wave) emissions are like a game of tag. If you're a General class operator, you get to play tag anywhere on the playground—no restrictions! So, whether you're at the swings (28.000 MHz) or the slide (28.300 MHz), you can run around and have fun everywhere. That's why the correct answer is The entire band.

\subsubsection{Advanced Explanation}
The 10-meter band spans from 28.000 MHz to 28.300 MHz. For General class operators, the FCC regulations allow CW emissions across the entire 10-meter band. This means there are no sub-band restrictions for CW emissions within this range. 

To understand this better, let's break it down:
\begin{itemize}
    \item \textbf{CW Emissions}: These are continuous wave signals, typically used for Morse code communication. They are narrowband and efficient for long-distance communication.
    \item \textbf{General Class Privileges}: Operators with a General class license have broader frequency privileges compared to Technician class operators. This includes access to the entire 10-meter band for CW emissions.
\end{itemize}

Therefore, the correct answer is \textbf{D: The entire band}.

% Prompt for diagram: A frequency spectrum diagram showing the 10-meter band from 28.000 MHz to 28.300 MHz with CW emissions marked across the entire range.