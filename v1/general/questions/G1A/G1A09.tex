\subsection{Which Frequency is Within the General Class Portion of the 15-Meter Band?}
\label{G1A09}

\begin{tcolorbox}[colback=gray!10!white,colframe=black!75!black,title=G1A09]
Which of the following frequencies is within the General class portion of the 15-meter band?
\begin{enumerate}[label=\Alph*),noitemsep]
    \item 14250 kHz
    \item 18155 kHz
    \item \textbf{21300 kHz}
    \item 24900 kHz
\end{enumerate}
\end{tcolorbox}

\subsubsection{Intuitive Explanation}
Imagine the 15-meter band as a big playground for radio waves. The General class portion is like a specific area in this playground where certain radio operators are allowed to play. Now, we have four kids (frequencies) who want to enter this area. Only one of them, 21300 kHz, has the right ticket to get in. The others are either too low or too high to be allowed in this special zone. So, 21300 kHz is the lucky one that gets to play in the General class portion of the 15-meter band!

\subsubsection{Advanced Explanation}
The 15-meter band in amateur radio spans from 21.000 MHz to 21.450 MHz. The General class portion of this band is specifically allocated from 21.025 MHz to 21.200 MHz. To determine which of the given frequencies falls within this range, we convert the frequencies to MHz:

\begin{itemize}
    \item 14250 kHz = 14.250 MHz (Too low)
    \item 18155 kHz = 18.155 MHz (Too low)
    \item 21300 kHz = 21.300 MHz (Within the range)
    \item 24900 kHz = 24.900 MHz (Too high)
\end{itemize}

Thus, 21300 kHz (21.300 MHz) is the only frequency that lies within the General class portion of the 15-meter band. This frequency is suitable for General class operators to use for communication within this specific segment of the radio spectrum.

% Prompt for generating a diagram: 
% A frequency spectrum diagram showing the 15-meter band (21.000 MHz to 21.450 MHz) with the General class portion (21.025 MHz to 21.200 MHz) highlighted. Mark the frequency 21.300 MHz within this range.