\subsection{Waterfall Display Vertical Lines}
\label{G8C13}

\begin{tcolorbox}[colback=gray!10!white,colframe=black!75!black,title=G8C13]
What is indicated on a waterfall display by one or more vertical lines on either side of a data mode or RTTY signal?
\begin{enumerate}[label=\Alph*),noitemsep]
    \item Long path propagation
    \item Backscatter propagation
    \item Insufficient modulation
    \item \textbf{Overmodulation}
\end{enumerate}
\end{tcolorbox}

\subsubsection{Intuitive Explanation}
Imagine you're watching a waterfall, but instead of water, it's a display of radio signals. If you see vertical lines on either side of a signal, it's like the signal is shouting too loudly. This shouting is called overmodulation. It means the signal is too strong and might cause problems for others trying to listen in. Think of it as someone talking so loudly in a library that no one else can hear their own books!

\subsubsection{Advanced Explanation}
In radio communications, modulation is the process of varying a carrier wave to encode information. Overmodulation occurs when the modulation index exceeds 1, causing distortion and potentially generating unwanted sidebands. On a waterfall display, which visually represents the frequency spectrum over time, overmodulation manifests as vertical lines adjacent to the primary signal. These lines indicate the presence of excessive modulation, which can lead to interference with adjacent frequencies.

Mathematically, the modulation index \( m \) is given by:
\[
m = \frac{A_m}{A_c}
\]
where \( A_m \) is the amplitude of the modulating signal and \( A_c \) is the amplitude of the carrier wave. When \( m > 1 \), overmodulation occurs, leading to the generation of harmonics and sidebands that appear as vertical lines on the waterfall display.

% Diagram prompt: Generate a diagram showing a waterfall display with a primary signal and vertical lines on either side indicating overmodulation.