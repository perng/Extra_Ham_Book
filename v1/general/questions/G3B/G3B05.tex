\subsection{Ionosphere's Effect on Radio Waves}
\label{G3B05}

\begin{tcolorbox}[colback=gray!10!white,colframe=black!75!black,title=G3B05]
How does the ionosphere affect radio waves with frequencies below the MUF and above the LUF?
\begin{enumerate}[label=\Alph*),noitemsep]
    \item \textbf{They are refracted back to Earth}
    \item They pass through the ionosphere
    \item They are amplified by interaction with the ionosphere
    \item They are refracted and trapped in the ionosphere to circle Earth
\end{enumerate}
\end{tcolorbox}

\subsubsection*{Intuitive Explanation}
Imagine the ionosphere as a giant trampoline in the sky. When you throw a ball (radio wave) at it, if the ball isn't too fast (frequency below the MUF) or too slow (frequency above the LUF), the trampoline bounces it back to you. This is what happens to radio waves in the ionosphere—they get bounced back to Earth instead of going straight through or getting stuck in the trampoline.

\subsubsection*{Advanced Explanation}
The ionosphere is a layer of the Earth's atmosphere that is ionized by solar radiation. It can refract radio waves, bending their path back towards the Earth's surface. The Maximum Usable Frequency (MUF) is the highest frequency that can be refracted back to Earth for a given ionospheric condition, while the Lowest Usable Frequency (LUF) is the lowest frequency that can be effectively refracted.

For radio waves with frequencies below the MUF and above the LUF, the ionosphere acts as a refractive medium. The refractive index \( n \) of the ionosphere can be approximated by:

\[
n = \sqrt{1 - \frac{N_e e^2}{\pi m \nu^2}}
\]

where \( N_e \) is the electron density, \( e \) is the electron charge, \( m \) is the electron mass, and \( \nu \) is the frequency of the radio wave. When the frequency \( \nu \) is between the LUF and MUF, the refractive index \( n \) is such that the radio wave is bent back towards the Earth, allowing for long-distance communication.

% Diagram prompt: A diagram showing the path of radio waves being refracted back to Earth by the ionosphere, with labels for MUF and LUF.