\subsection{E Region Hop Distance}
\label{G3B10}

\begin{tcolorbox}[colback=gray!10!white,colframe=black!75!black,title=G3B10]
What is the approximate maximum distance along the Earth’s surface normally covered in one hop using the E region?
\begin{enumerate}[label=\Alph*),noitemsep]
    \item 180 miles
    \item \textbf{1,200 miles}
    \item 2,500 miles
    \item 12,000 miles
\end{enumerate}
\end{tcolorbox}

\subsubsection*{Intuitive Explanation}
Imagine you’re playing a game of catch with a friend, but instead of throwing a ball, you’re bouncing a radio wave off a layer in the sky called the E region. The E region is like a trampoline for radio waves—it bounces them back to Earth. Now, how far can this bounce take you? Well, it’s not like throwing a ball across the street; it’s more like throwing it across a few states! The E region can bounce radio waves up to about 1,200 miles in one hop. That’s like going from New York City to Miami in one bounce!

\subsubsection*{Advanced Explanation}
The E region is one of the ionospheric layers located approximately 90 to 150 kilometers above the Earth’s surface. It plays a crucial role in high-frequency (HF) radio communication by reflecting radio waves back to Earth. The maximum distance covered in one hop using the E region depends on the height of the E layer and the angle of incidence of the radio wave.

The formula to calculate the maximum distance \( D \) for one hop is given by:
\[
D = 2 \times h \times \tan(\theta)
\]
where \( h \) is the height of the E region, and \( \theta \) is the angle of incidence. For typical conditions, the height \( h \) is around 110 kilometers, and the angle \( \theta \) is such that the maximum distance \( D \) is approximately 1,200 miles.

The E region is most effective during daylight hours when solar radiation ionizes the layer, enhancing its reflective properties. At night, the E region’s ionization decreases, reducing its effectiveness for long-distance communication.

% Diagram prompt: Generate a diagram showing the Earth's surface, the E region, and the path of a radio wave bouncing off the E region to illustrate the concept of one hop.