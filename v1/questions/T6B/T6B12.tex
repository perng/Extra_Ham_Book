\subsection{Electrodes of a Bipolar Junction Transistor}
\label{T6B12}

\begin{tcolorbox}[colback=gray!10!white,colframe=black!75!black,title=T6B12]
What are the names of the electrodes of a bipolar junction transistor?
\begin{enumerate}[noitemsep]
    \item Signal, bias, power
    \item \textbf{Emitter, base, collector}
    \item Input, output, supply
    \item Pole one, pole two, output
\end{enumerate}
\end{tcolorbox}

A bipolar junction transistor (BJT) has three electrodes: the emitter, the base, and the collector. These electrodes are crucial for the transistor's operation, as they control the flow of current through the device. The emitter emits charge carriers, the base controls the flow, and the collector collects the charge carriers. This structure allows the BJT to amplify signals and switch electronic circuits.