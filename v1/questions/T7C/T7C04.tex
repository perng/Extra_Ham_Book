\subsection{SWR Meter and Impedance Match}
\label{T7C04}

\begin{tcolorbox}[colback=gray!10!white,colframe=black!75!black,title=T7C04]
What reading on an SWR meter indicates a perfect impedance match between the antenna and the feed line?
\begin{enumerate}[noitemsep]
    \item 50:50
    \item Zero
    \item \textbf{1:1}
    \item Full Scale
\end{enumerate}
\end{tcolorbox}

\subsubsection*{Intuitive Explanation}
An SWR (Standing Wave Ratio) meter measures how well the antenna and the feed line are matched in terms of impedance. Think of it like a dance between the antenna and the feed line. If they are perfectly in sync, the SWR meter will show a 1:1 ratio. This means there’s no mismatch, and the signal is being transmitted efficiently without any reflections.

\subsubsection*{Advanced Explanation}
The SWR is a measure of the impedance match between the antenna and the feed line. When the impedance of the antenna matches the impedance of the feed line, there is no reflection of the signal, and the SWR is 1:1. This is the ideal condition for maximum power transfer. If the SWR is higher than 1:1, it indicates a mismatch, which can lead to signal loss and potential damage to the transmitter. The correct answer, therefore, is \textbf{1:1}, as it signifies a perfect impedance match.