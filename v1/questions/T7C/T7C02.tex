\subsection{Determining Antenna Resonance}
\label{T7C02}

\begin{tcolorbox}[colback=gray!10!white,colframe=black!75!black,title=T7C02]
Which of the following is used to determine if an antenna is resonant at the desired operating frequency?
\begin{enumerate}[noitemsep]
    \item A VTVM
    \item \textbf{An antenna analyzer}
    \item A Q meter
    \item A frequency counter
\end{enumerate}
\end{tcolorbox}

\subsubsection*{Intuitive Explanation}
Think of an antenna like a musical instrument. Just as a guitar string needs to be tuned to the right pitch to sound good, an antenna needs to be tuned to the right frequency to work efficiently. An antenna analyzer is like a tuner for your antenna, helping you check if it's resonating at the desired frequency.

\subsubsection*{Advanced Explanation}
An antenna analyzer is a specialized tool used to measure the impedance and resonance of an antenna. Resonance occurs when the antenna's inductive and capacitive reactances cancel each other out, resulting in a purely resistive impedance. This is crucial for efficient power transfer from the transmitter to the antenna. The analyzer provides real-time feedback on the antenna's performance, allowing adjustments to be made to achieve optimal resonance at the desired operating frequency. Other tools like VTVMs, Q meters, and frequency counters do not provide the specific measurements needed to determine antenna resonance.