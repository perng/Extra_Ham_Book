\subsection{Common Modulation for VHF Packet Radio}
\label{T8A02}

\begin{tcolorbox}[colback=gray!10!white,colframe=black!75!black,title=T8A02]
What type of modulation is commonly used for VHF packet radio transmissions?
\begin{enumerate}[noitemsep]
    \item \textbf{FM or PM}
    \item SSB
    \item AM
    \item PSK
\end{enumerate}
\end{tcolorbox}

\subsubsection*{Intuitive Explanation}
Think of VHF packet radio like a conversation between two people. To make sure the message gets through clearly, you need a reliable way to send it. FM (Frequency Modulation) and PM (Phase Modulation) are like speaking in a steady, clear voice—they’re great for keeping the message intact over short to medium distances. That’s why they’re commonly used for VHF packet radio transmissions.

\subsubsection*{Advanced Explanation}
VHF (Very High Frequency) packet radio typically operates in the 144-148 MHz range. FM and PM are preferred for VHF packet radio because they are robust against noise and interference, which is crucial for maintaining data integrity over these frequencies. FM modulates the frequency of the carrier wave, while PM modulates the phase. Both methods are effective for digital data transmission, ensuring that the signal remains clear and stable even in less-than-ideal conditions.