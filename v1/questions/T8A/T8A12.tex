\subsection{Disadvantages of FM Compared to Single Sideband}
\label{T8A12}

\begin{tcolorbox}[colback=gray!10!white,colframe=black!75!black,title=T8A12]
Which of the following is a disadvantage of FM compared with single sideband?
\begin{enumerate}[noitemsep]
    \item Voice quality is poorer
    \item \textbf{Only one signal can be received at a time}
    \item FM signals are harder to tune
    \item All these choices are correct
\end{enumerate}
\end{tcolorbox}

\subsubsection*{Intuitive Explanation}
Imagine you're at a party where everyone is talking at the same time. FM is like having a loudspeaker that can only play one conversation at a time, while single sideband is like having multiple headphones that let you listen to different conversations simultaneously. FM's limitation is that it can only handle one signal at a time, making it less versatile in crowded environments.

\subsubsection*{Advanced Explanation}
Frequency Modulation (FM) and Single Sideband (SSB) are two different methods of transmitting radio signals. FM modulates the frequency of the carrier wave to encode the information, while SSB suppresses one sideband and the carrier to transmit the signal more efficiently. One of the key disadvantages of FM is its inability to handle multiple signals simultaneously on the same frequency, unlike SSB which can accommodate multiple signals by using different sidebands. This makes FM less efficient in scenarios where multiple transmissions need to be received concurrently.