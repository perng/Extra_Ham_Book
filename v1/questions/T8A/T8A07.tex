\subsection{Characteristics of Single Sideband (SSB) Compared to FM}
\label{T8A07}

\begin{tcolorbox}[colback=gray!10!white,colframe=black!75!black,title=T8A07]
What is a characteristic of single sideband (SSB) compared to FM?
\begin{enumerate}[noitemsep]
    \item SSB signals are easier to tune in correctly
    \item SSB signals are less susceptible to interference
    \item \textbf{SSB signals have narrower bandwidth}
    \item All these choices are correct
\end{enumerate}
\end{tcolorbox}

\subsubsection*{Explanation}
Single Sideband (SSB) modulation is known for its efficient use of bandwidth compared to Frequency Modulation (FM). SSB transmits only one sideband and suppresses the carrier, which results in a narrower bandwidth. This is particularly advantageous in crowded frequency bands where spectrum efficiency is crucial. FM, on the other hand, uses a wider bandwidth due to its modulation technique, which spreads the signal over a larger frequency range. Therefore, the correct answer is that SSB signals have narrower bandwidth.