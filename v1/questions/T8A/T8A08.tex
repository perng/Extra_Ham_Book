\subsection{Bandwidth of a Single Sideband (SSB) Voice Signal}
\label{T8A08}

\begin{tcolorbox}[colback=gray!10!white,colframe=black!75!black,title=T8A08]
What is the approximate bandwidth of a typical single sideband (SSB) voice signal?
\begin{enumerate}[noitemsep]
    \item 1 kHz
    \item \textbf{3 kHz}
    \item 6 kHz
    \item 15 kHz
\end{enumerate}
\end{tcolorbox}

\subsubsection*{Intuitive Explanation}
Imagine you're talking on a walkie-talkie. The sound of your voice gets converted into a radio signal, but instead of sending the entire signal, which would take up a lot of space, we only send one side of it. This is called Single Sideband (SSB). The bandwidth is like the width of the signal, and for a typical SSB voice signal, it's about 3 kHz. This is enough to carry the important parts of your voice without wasting space.

\subsubsection*{Advanced Explanation}
Single Sideband (SSB) modulation is a technique used in radio communications to transmit voice signals efficiently. In SSB, only one sideband (either the upper or lower) is transmitted, along with the carrier signal suppressed. This reduces the bandwidth required for transmission. The bandwidth of a typical SSB voice signal is approximately 3 kHz. This is because the human voice contains frequencies primarily in the range of 300 Hz to 3 kHz. By transmitting only one sideband, the bandwidth is effectively halved compared to double sideband (DSB) transmission, making SSB a more efficient use of the radio spectrum.

% Diagram prompt: Generate a frequency spectrum diagram showing the bandwidth of a typical SSB voice signal. Use Python with Matplotlib to plot the frequency range from 0 Hz to 4 kHz, highlighting the 3 kHz bandwidth of the SSB signal.