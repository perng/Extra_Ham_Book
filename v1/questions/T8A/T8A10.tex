\subsection{Bandwidth of AM Fast-Scan TV Transmissions}
\label{T8A10}

\begin{tcolorbox}[colback=gray!10!white,colframe=black!75!black,title=T8A10]
What is the approximate bandwidth of AM fast-scan TV transmissions?
\begin{enumerate}[noitemsep]
    \item More than 10 MHz
    \item \textbf{About 6 MHz}
    \item About 3 MHz
    \item About 1 MHz
\end{enumerate}
\end{tcolorbox}

\subsubsection*{Intuitive Explanation}
Imagine you're watching an old-school TV show. The picture and sound are being sent to your TV through the airwaves. The amount of space these signals take up in the airwaves is called bandwidth. For AM fast-scan TV, this space is about 6 MHz. That's like a lane on a highway just for your TV show!

\subsubsection*{Advanced Explanation}
AM fast-scan TV transmissions use amplitude modulation (AM) to carry both video and audio signals. The bandwidth required for these transmissions is determined by the range of frequencies needed to accurately represent the video and audio information. For standard AM fast-scan TV, the bandwidth is approximately 6 MHz. This includes the video carrier, the audio carrier, and the sidebands that carry the actual information. The 6 MHz bandwidth ensures that the TV signal can carry enough detail for clear picture and sound.