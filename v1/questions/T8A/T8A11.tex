\subsection{Bandwidth for CW Signal Transmission}
\label{T8A11}

\begin{tcolorbox}[colback=gray!10!white,colframe=black!75!black,title=T8A11]
What is the approximate bandwidth required to transmit a CW signal?
\begin{enumerate}[noitemsep]
    \item 2.4 kHz
    \item \textbf{150 Hz}
    \item 1000 Hz
    \item 15 kHz
\end{enumerate}
\end{tcolorbox}

\subsubsection*{Intuitive Explanation}
Imagine you're sending a message using a flashlight by turning it on and off in Morse code. The bandwidth is like the range of frequencies needed to send this simple on-off signal. Since CW (Continuous Wave) signals are just simple on-off keying, they don't need much bandwidth. Think of it as a narrow lane on a highway—just enough space for your signal to pass through without any extra room.

\subsubsection*{Advanced Explanation}
CW signals are generated by turning a carrier wave on and off, typically using Morse code. The bandwidth required for a CW signal is primarily determined by the keying speed, which is usually quite slow. The bandwidth can be approximated by the formula:

\[
\text{Bandwidth} \approx \frac{1}{\text{Keying Speed}}
\]

For typical CW keying speeds, the bandwidth is around 150 Hz. This narrow bandwidth allows for efficient use of the radio spectrum and minimizes interference with other signals.