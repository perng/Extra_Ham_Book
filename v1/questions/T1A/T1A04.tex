\subsection{Number of Operator/Primary Station Licenses}
\label{T1A04}

\begin{tcolorbox}[colback=gray!10!white,colframe=black!75!black,title=T1A04]
How many operator/primary station license grants may be held by any one person?
\begin{enumerate}[label=\Alph*),noitemsep]
    \item \textbf{One}
    \item No more than two
    \item One for each band on which the person plans to operate
    \item One for each permanent station location from which the person plans to operate
\end{enumerate}
\end{tcolorbox}

\subsubsection*{Explanation}
In the context of radio operation, an operator/primary station license is a legal authorization granted by the regulatory authority (such as the FCC in the United States) that allows an individual to operate a radio station. According to the regulations, any one person is allowed to hold only one such license. This ensures that the allocation of licenses is fair and that no single individual can monopolize the available resources or frequencies. Therefore, the correct answer is \textbf{A: One}.