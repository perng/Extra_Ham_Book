\subsection{Frequency Coordinator Selection}
\label{T1A09}

\begin{tcolorbox}[colback=gray!10!white,colframe=black!75!black,title=T1A09]
Who selects a Frequency Coordinator?
\begin{enumerate}[label=\Alph*),noitemsep]
    \item The FCC Office of Spectrum Management and Coordination Policy
    \item The local chapter of the Office of National Council of Independent Frequency Coordinators
    \item \textbf{Amateur operators in a local or regional area whose stations are eligible to be repeater or auxiliary stations}
    \item FCC Regional Field Office
\end{enumerate}
\end{tcolorbox}

\subsubsection*{Intuitive Explanation}
Imagine you and your friends are organizing a big event, like a concert, and you need to decide who will be in charge of making sure everyone gets a turn to perform without overlapping. In the world of amateur radio, this person is called the Frequency Coordinator. The Frequency Coordinator ensures that different radio stations don't interfere with each other. So, who gets to pick this important person? It's the amateur radio operators in the local area who are eligible to use repeaters or auxiliary stations. They know the local needs best and can choose someone who will do a good job.

\subsubsection*{Advanced Explanation}
In amateur radio, a Frequency Coordinator is responsible for managing the use of frequencies within a specific area to prevent interference between different stations. This role is crucial for maintaining orderly communication, especially in areas with a high density of amateur radio operators. The selection of a Frequency Coordinator is not done by a governmental body like the FCC, but rather by the amateur operators themselves. Specifically, it is the operators in a local or regional area who are eligible to operate repeater or auxiliary stations that select the Frequency Coordinator. This ensures that the coordinator is someone who understands the local communication needs and can effectively manage the frequency assignments.