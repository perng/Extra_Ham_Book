\subsection{Reducing VHF Transceiver Overload}
\label{T7B07}

\begin{tcolorbox}[colback=gray!10!white,colframe=black!75!black,title=T7B07]
Which of the following can reduce overload of a VHF transceiver by a nearby commercial FM station?
\begin{enumerate}[noitemsep]
    \item Installing an RF preamplifier
    \item Using double-shielded coaxial cable
    \item Installing bypass capacitors on the microphone cable
    \item \textbf{Installing a band-reject filter}
\end{enumerate}
\end{tcolorbox}

\subsubsection*{Explanation}
When a VHF transceiver is overloaded by a nearby commercial FM station, the issue is typically due to strong signals from the FM station interfering with the transceiver's operation. A band-reject filter, also known as a notch filter, can be installed to attenuate the specific frequency range of the FM station, thereby reducing the interference and preventing overload. This is the most effective solution among the given options. 

Installing an RF preamplifier (Option A) would amplify all incoming signals, including the unwanted FM station, making the problem worse. Using double-shielded coaxial cable (Option B) can reduce external interference but is not effective against strong nearby signals. Installing bypass capacitors on the microphone cable (Option C) is unrelated to RF interference and would not address the issue.