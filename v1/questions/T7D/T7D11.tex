\subsection{Precautions for Measuring In-Circuit Resistance}
\label{T7D11}

\begin{tcolorbox}[colback=gray!10!white,colframe=black!75!black,title=T7D11]
Which of the following precautions should be taken when measuring in-circuit resistance with an ohmmeter?
\begin{enumerate}[noitemsep]
    \item Ensure that the applied voltages are correct
    \item \textbf{Ensure that the circuit is not powered}
    \item Ensure that the circuit is grounded
    \item Ensure that the circuit is operating at the correct frequency
\end{enumerate}
\end{tcolorbox}

\subsubsection*{Intuitive Explanation}
When using an ohmmeter to measure resistance in a circuit, it's crucial to ensure that the circuit is not powered. This is because an ohmmeter works by sending a small current through the circuit to measure resistance. If the circuit is powered, the external voltage can interfere with the ohmmeter's readings, leading to inaccurate results or even damaging the ohmmeter. Think of it like trying to measure the weight of a balloon while someone is blowing air into it—you won't get the correct measurement!

\subsubsection*{Advanced Explanation}
An ohmmeter measures resistance by applying a known voltage and measuring the resulting current through the circuit using Ohm's Law (\( R = \frac{V}{I} \)). If the circuit is powered, the external voltage can alter the current flowing through the circuit, leading to incorrect resistance readings. Additionally, the external voltage can cause excessive current to flow through the ohmmeter, potentially damaging its internal components. Therefore, it is essential to ensure that the circuit is not powered when measuring in-circuit resistance with an ohmmeter. This precaution helps maintain the accuracy of the measurement and protects the ohmmeter from damage.