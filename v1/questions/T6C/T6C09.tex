\subsection{Component Identification in Figure T-2}
\label{T6C09}

\begin{tcolorbox}[colback=gray!10!white,colframe=black!75!black,title=T6C09]
What is component 4 in figure T-2?
\begin{enumerate}[noitemsep]
    \item Variable inductor
    \item Double-pole switch
    \item Potentiometer
    \item \textbf{Transformer}
\end{enumerate}
\end{tcolorbox}

\subsubsection*{Intuitive Explanation}
Imagine you have a magical box that can change the amount of electricity flowing through it. If you put in a little bit of electricity, it can make it stronger or weaker depending on what you need. That's essentially what a transformer does. It's like a volume knob for electricity, but instead of sound, it adjusts the voltage.

\subsubsection*{Advanced Explanation}
A transformer is an electrical device that transfers electrical energy between two or more circuits through electromagnetic induction. It consists of two coils of wire, known as the primary and secondary windings, which are usually wrapped around a common iron core. When an alternating current (AC) flows through the primary winding, it creates a changing magnetic field in the core, which induces a voltage in the secondary winding. This allows the transformer to step up (increase) or step down (decrease) the voltage level of the AC signal. In the context of figure T-2, component 4 is identified as a transformer based on its typical representation in circuit diagrams and its function in the circuit.

% Diagram prompt: Generate a simple circuit diagram showing a transformer with labeled primary and secondary windings, using SVG format for clarity and scalability.