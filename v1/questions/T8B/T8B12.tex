\subsection{Determining Satellite Uplink Power}
\label{T8B12}

\begin{tcolorbox}[colback=gray!10!white,colframe=black!75!black,title=T8B12]
Which of the following is a way to determine whether your satellite uplink power is neither too low nor too high?
\begin{enumerate}[noitemsep]
    \item Check your signal strength report in the telemetry data
    \item Listen for distortion on your downlink signal
    \item \textbf{Your signal strength on the downlink should be about the same as the beacon}
    \item All these choices are correct
\end{enumerate}
\end{tcolorbox}

\subsubsection*{Intuitive Explanation}
Imagine you're trying to talk to someone on a walkie-talkie. If you shout too loudly, the other person might hear you, but it could be distorted. If you whisper, they might not hear you at all. The same idea applies to satellite communication. You want your uplink power to be just right—not too loud and not too soft. One way to check this is by comparing your signal strength to the beacon, which is like a reference signal. If they match, you're good to go!

\subsubsection*{Advanced Explanation}
In satellite communication, the uplink power must be carefully controlled to ensure effective communication without causing interference or signal degradation. The beacon signal is a constant reference signal transmitted by the satellite. By comparing the strength of your downlink signal to the beacon, you can gauge whether your uplink power is appropriate. If the downlink signal strength is similar to the beacon, it indicates that the uplink power is neither too low (which would result in a weak downlink signal) nor too high (which could cause distortion or interference). This method provides a straightforward and reliable way to optimize uplink power.