\subsection{Satellite U/V Mode Operation}
\label{T8B08}

\begin{tcolorbox}[colback=gray!10!white,colframe=black!75!black,title=T8B08]
What is meant by the statement that a satellite is operating in U/V mode?
\begin{enumerate}[noitemsep]
    \item The satellite uplink is in the 15 meter band and the downlink is in the 10 meter band
    \item \textbf{The satellite uplink is in the 70 centimeter band and the downlink is in the 2 meter band}
    \item The satellite operates using ultraviolet frequencies
    \item The satellite frequencies are usually variable
\end{enumerate}
\end{tcolorbox}

\subsubsection*{Intuitive Explanation}
Imagine you're sending a message to a satellite. The U/V mode is like a two-way street where you send your message on one lane (the 70 cm band) and the satellite replies on another lane (the 2 meter band). It's not about ultraviolet light or variable frequencies; it's just about the specific radio bands used for sending and receiving.

\subsubsection*{Advanced Explanation}
In satellite communications, the terms U and V refer to specific frequency bands. The U stands for the 70 centimeter band (430-440 MHz), which is commonly used for the uplink (transmission from Earth to the satellite). The V stands for the 2 meter band (144-146 MHz), which is typically used for the downlink (transmission from the satellite to Earth). This configuration is known as U/V mode. It is important to note that these bands are part of the amateur radio spectrum and are chosen for their propagation characteristics and availability for satellite use.