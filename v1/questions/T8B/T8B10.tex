\subsection{LEO Satellite Definition}
\label{T8B10}

\begin{tcolorbox}[colback=gray!10!white,colframe=black!75!black,title=T8B10]
What is a LEO satellite?
\begin{enumerate}[noitemsep]
    \item A sun synchronous satellite
    \item A highly elliptical orbit satellite
    \item A satellite in low energy operation mode
    \item \textbf{A satellite in low earth orbit}
\end{enumerate}
\end{tcolorbox}

A LEO satellite, or Low Earth Orbit satellite, is a satellite that orbits the Earth at altitudes typically ranging from 160 to 2,000 kilometers. These satellites are commonly used for various applications, including communication, Earth observation, and scientific research, due to their relatively close proximity to the Earth's surface.