\subsection{Satellite Tracking Program Inputs}
\label{T8B06}

\begin{tcolorbox}[colback=gray!10!white,colframe=black!75!black,title=T8B06]
Which of the following are inputs to a satellite tracking program?
\begin{enumerate}[noitemsep]
    \item The satellite transmitted power
    \item \textbf{The Keplerian elements}
    \item The last observed time of zero Doppler shift
    \item All these choices are correct
\end{enumerate}
\end{tcolorbox}

\subsubsection*{Intuitive Explanation}
Imagine you're trying to track a friend's car using a GPS app. To know where the car is, you need some key information like its current location, speed, and direction. Similarly, a satellite tracking program needs specific data to predict where a satellite will be at any given time. The Keplerian elements are like the GPS coordinates for satellites—they provide the necessary orbital information to track them accurately.

\subsubsection*{Advanced Explanation}
The Keplerian elements are a set of six parameters that define the orbit of a satellite. These elements include the semi-major axis, eccentricity, inclination, right ascension of the ascending node, argument of periapsis, and mean anomaly. These parameters allow the tracking program to calculate the satellite's position and velocity at any point in time. The satellite's transmitted power and the last observed time of zero Doppler shift are not used as inputs for tracking; instead, they are related to signal reception and Doppler effect analysis, respectively. Therefore, the correct input for a satellite tracking program is the Keplerian elements.