\subsection{Designation of Station Control Operator}
\label{T1E03}

\begin{tcolorbox}[colback=gray!10!white,colframe=black!75!black,title=T1E03]
Who must designate the station control operator?
\begin{enumerate}[label=\Alph*.]
    \item \textbf{The station licensee}
    \item The FCC
    \item The frequency coordinator
    \item Any licensed operator
\end{enumerate}
\end{tcolorbox}

\subsubsection{Intuitive Explanation}
Imagine you have a cool treehouse, and you’re the boss of it. You get to decide who gets to be in charge when you’re not around. In the world of radio, the person who owns the radio station (the station licensee) is like the boss of the treehouse. They get to pick who’s in charge of running the station, called the control operator. It’s not the government (FCC), the person who helps pick the radio frequency (frequency coordinator), or just any random person with a license. It’s the boss—the station licensee!

\subsubsection{Advanced Explanation}
In the context of radio operations, the station licensee holds the legal responsibility for the station’s compliance with FCC regulations. According to FCC rules, the station licensee must designate the control operator, who is responsible for the station’s operation during a specific period. This designation ensures that the station operates within the legal framework and adheres to technical standards. The control operator must hold the appropriate license class for the station’s operation, but the authority to designate this operator lies solely with the station licensee. This process underscores the licensee’s accountability for the station’s activities.