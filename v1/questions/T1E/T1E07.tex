\subsection{Responsibility for Station Operation}
\label{T1E07}

\begin{tcolorbox}[colback=gray!10!white,colframe=black!75!black,title=T1E07]
When the control operator is not the station licensee, who is responsible for the proper operation of the station?
\begin{enumerate}[label=\Alph*),noitemsep]
    \item All licensed amateurs who are present at the operation
    \item Only the station licensee
    \item Only the control operator
    \item \textbf{The control operator and the station licensee}
\end{enumerate}
\end{tcolorbox}

\subsubsection{Intuitive Explanation}
Imagine you and your friend are playing with a remote-controlled car. Your friend is the one holding the remote (the control operator), but the car actually belongs to you (the station licensee). If something goes wrong, like the car crashes into a wall, both of you are responsible. Why? Because your friend was controlling it, but it’s your car! So, both of you need to make sure everything is working properly. In the same way, when the control operator is not the station licensee, both of them are responsible for the proper operation of the station.

\subsubsection{Advanced Explanation}
In amateur radio operations, the station licensee is the person who owns the station and is responsible for its overall compliance with regulations. The control operator is the person who is actually operating the station at any given time. According to FCC rules, both the control operator and the station licensee share responsibility for ensuring that the station operates within legal limits. This dual responsibility ensures that both parties are accountable for the station’s proper operation, even if the control operator is not the licensee. This is particularly important in maintaining the integrity and legality of amateur radio communications.

% Diagram prompt: A simple diagram showing two figures, one labeled Station Licensee and the other labeled Control Operator, with arrows pointing to a radio station labeled Responsibility for Proper Operation.