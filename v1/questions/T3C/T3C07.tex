\subsection{Meteor Scatter Communication Band}
\label{T3C07}

\begin{tcolorbox}[colback=gray!10!white,colframe=black!75!black,title=T3C07]
What band is best suited for communicating via meteor scatter?
\begin{enumerate}[noitemsep]
    \item 33 centimeters
    \item \textbf{6 meters}
    \item 2 meters
    \item 70 centimeters
\end{enumerate}
\end{tcolorbox}

\subsubsection*{Intuitive Explanation}
Meteor scatter communication is like using the trails left by shooting stars as mirrors to bounce radio signals. The 6-meter band is particularly good for this because it strikes a balance between being able to reflect off the ionized trails and not being absorbed too much by the atmosphere. Think of it as the Goldilocks band—not too long, not too short, but just right for meteor scatter.

\subsubsection*{Advanced Explanation}
Meteor scatter communication relies on the ionization trails left by meteors entering the Earth's atmosphere. These trails can reflect radio waves, allowing for communication over long distances. The 6-meter band (50-54 MHz) is ideal for this purpose because it has a wavelength that is long enough to be effectively reflected by the ionized trails but short enough to minimize atmospheric absorption and noise. Additionally, the 6-meter band is less crowded compared to higher frequency bands, reducing interference and improving the chances of successful communication.