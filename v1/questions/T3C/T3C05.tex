\subsection{Radio Signal Propagation Beyond Obstructions}
\label{T3C05}

\begin{tcolorbox}[colback=gray!10!white,colframe=black!75!black,title=T3C05]
Which of the following effects may allow radio signals to travel beyond obstructions between the transmitting and receiving stations?
\begin{enumerate}[noitemsep]
    \item \textbf{Knife-edge diffraction}
    \item Faraday rotation
    \item Quantum tunneling
    \item Doppler shift
\end{enumerate}
\end{tcolorbox}

\subsubsection*{Intuitive Explanation}
Imagine you're trying to throw a ball over a tall fence. If you throw it straight at the fence, it will hit the fence and bounce back. But if you throw it at an angle, the ball might just skim the top of the fence and land on the other side. Knife-edge diffraction is like that angled throw—it allows radio signals to bend around the edges of obstacles, like mountains or buildings, and reach the other side.

\subsubsection*{Advanced Explanation}
Knife-edge diffraction is a phenomenon where radio waves bend around the sharp edges of obstacles, such as hills or buildings, allowing the signal to propagate beyond the line of sight. This effect is particularly useful in radio communication when there are obstructions between the transmitting and receiving stations. The amount of diffraction depends on the wavelength of the signal and the size and shape of the obstacle. In contrast, Faraday rotation, quantum tunneling, and Doppler shift are not typically associated with overcoming physical obstructions in radio signal propagation.

% Diagram Prompt: Generate a diagram showing a radio wave bending around a sharp edge (knife-edge diffraction). Use Python with Matplotlib to create the diagram, showing the wavefronts bending around the obstacle.