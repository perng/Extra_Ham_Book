\subsection{Optimal Time for 10 Meter Band Propagation via F Region}
\label{T3C09}

\begin{tcolorbox}[colback=gray!10!white,colframe=black!75!black,title=T3C09]
What is generally the best time for long-distance 10 meter band propagation via the F region?
\begin{enumerate}[noitemsep]
    \item \textbf{From dawn to shortly after sunset during periods of high sunspot activity}
    \item From shortly after sunset to dawn during periods of high sunspot activity
    \item From dawn to shortly after sunset during periods of low sunspot activity
    \item From shortly after sunset to dawn during periods of low sunspot activity
\end{enumerate}
\end{tcolorbox}

The F region of the ionosphere is most effective for long-distance radio propagation during daylight hours when solar radiation ionizes the atmosphere. High sunspot activity increases ionization, enhancing propagation conditions. Therefore, the best time for 10 meter band propagation via the F region is from dawn to shortly after sunset during periods of high sunspot activity.