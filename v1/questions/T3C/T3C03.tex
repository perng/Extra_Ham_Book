\subsection{Characteristics of VHF Signals via Auroral Backscatter}
\label{T3C03}

\begin{tcolorbox}[colback=gray!10!white,colframe=black!75!black,title=T3C03]
What is a characteristic of VHF signals received via auroral backscatter?
\begin{enumerate}[noitemsep]
    \item They are often received from 10,000 miles or more
    \item \textbf{They are distorted and signal strength varies considerably}
    \item They occur only during winter nighttime hours
    \item They are generally strongest when your antenna is aimed west
\end{enumerate}
\end{tcolorbox}

\subsubsection*{Intuitive Explanation}
Imagine the aurora as a giant, shimmering curtain in the sky. When VHF signals hit this curtain, they bounce back, but not in a neat, orderly way. Instead, the signals get all jumbled up, like a reflection in a funhouse mirror. This means the signals you receive can be distorted, and their strength can go up and down unpredictably.

\subsubsection*{Advanced Explanation}
Auroral backscatter occurs when VHF signals are reflected by the ionized regions of the atmosphere associated with auroras. These ionized regions are highly dynamic and irregular, causing the reflected signals to be distorted. The signal strength can vary considerably due to the changing density and movement of the ionized particles. This phenomenon is most commonly observed during periods of high solar activity when auroras are more pronounced. The distortion and variability in signal strength are key characteristics of VHF signals received via auroral backscatter.