\subsection{Polarization of Radio Waves}
\label{T3B02}

\begin{tcolorbox}[colback=gray!10!white,colframe=black!75!black,title=T3B02]
What property of a radio wave defines its polarization?
\begin{enumerate}[noitemsep]
    \item \textbf{The orientation of the electric field}
    \item The orientation of the magnetic field
    \item The ratio of the energy in the magnetic field to the energy in the electric field
    \item The ratio of the velocity to the wavelength
\end{enumerate}
\end{tcolorbox}

\subsubsection*{Intuitive Explanation}
Imagine a radio wave as a wiggling rope. The way the rope wiggles up and down or side to side is similar to how the electric field of a radio wave oscillates. The direction of this wiggle is what we call the polarization of the wave. So, if the electric field is moving up and down, the wave is vertically polarized. If it's moving side to side, it's horizontally polarized.

\subsubsection*{Advanced Explanation}
Polarization refers to the orientation of the electric field vector of an electromagnetic wave as it propagates through space. In a radio wave, the electric field and magnetic field are perpendicular to each other and to the direction of propagation. The polarization is determined by the direction of the electric field vector. For example, if the electric field oscillates in a vertical plane, the wave is said to be vertically polarized. Conversely, if it oscillates in a horizontal plane, the wave is horizontally polarized. The magnetic field, while always perpendicular to the electric field, does not define the polarization. The other options, such as the ratio of energies or the ratio of velocity to wavelength, are not related to the concept of polarization.

% Diagram prompt: Generate a diagram showing the electric and magnetic fields of a radio wave propagating in space. Use Python with Matplotlib to create a 3D plot where the electric field is oscillating vertically and the magnetic field is oscillating horizontally, both perpendicular to the direction of propagation.