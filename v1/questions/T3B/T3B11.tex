\subsection{Velocity of a Radio Wave in Free Space}
\label{T3B11}

\begin{tcolorbox}[colback=gray!10!white,colframe=black!75!black,title=T3B11]
What is the approximate velocity of a radio wave in free space?
\begin{enumerate}[noitemsep]
    \item 150,000 meters per second
    \item \textbf{300,000,000 meters per second}
    \item 300,000,000 miles per hour
    \item 150,000 miles per hour
\end{enumerate}
\end{tcolorbox}

\subsubsection*{Intuitive Explanation}
Imagine you're sending a message using a flashlight in a completely empty, dark room. The light travels at a constant speed, and in the same way, radio waves travel at a constant speed in free space. This speed is incredibly fast—about 300 million meters per second! That's why when you turn on your radio, the music starts almost instantly, even if the station is far away.

\subsubsection*{Advanced Explanation}
The velocity of a radio wave in free space is a fundamental constant in physics, often denoted by the symbol \( c \). This speed is the same as the speed of light in a vacuum, which is approximately \( 3 \times 10^8 \) meters per second. This value is derived from Maxwell's equations, which describe how electric and magnetic fields propagate through space. The speed of radio waves is not affected by the medium of free space, as there is no material to slow them down. This is why radio waves can travel vast distances in space without losing their speed.

% Diagram prompt: Generate a simple line graph showing the speed of light (c) as a constant line across time using Python's Matplotlib. Label the y-axis as Speed (m/s) and the x-axis as Time (s). The line should be labeled Speed of Radio Waves in Free Space.