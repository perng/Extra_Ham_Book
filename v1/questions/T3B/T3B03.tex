\subsection{Components of a Radio Wave}
\label{T3B03}

\begin{tcolorbox}[colback=gray!10!white,colframe=black!75!black,title=T3B03]
What are the two components of a radio wave?
\begin{enumerate}[noitemsep]
    \item Impedance and reactance
    \item Voltage and current
    \item \textbf{Electric and magnetic fields}
    \item Ionizing and non-ionizing radiation
\end{enumerate}
\end{tcolorbox}

\subsubsection*{Intuitive Explanation}
Imagine a radio wave as a wave in the ocean. Just like the wave has both height and movement, a radio wave has two main parts: an electric field and a magnetic field. These fields work together to carry the wave through space, much like how the height and movement of an ocean wave work together to move water.

\subsubsection*{Advanced Explanation}
A radio wave is an electromagnetic wave, which means it consists of oscillating electric and magnetic fields that are perpendicular to each other and to the direction of wave propagation. The electric field (\textbf{E}) and the magnetic field (\textbf{B}) are interdependent; a changing electric field generates a magnetic field, and a changing magnetic field generates an electric field. This mutual generation allows the wave to propagate through space without the need for a medium. The relationship between these fields is described by Maxwell's equations, which form the foundation of classical electromagnetism.