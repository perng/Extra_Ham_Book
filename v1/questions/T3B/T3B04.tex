\subsection{Velocity of a Radio Wave in Free Space}
\label{T3B04}

\begin{tcolorbox}[colback=gray!10!white,colframe=black!75!black,title=T3B04]
What is the velocity of a radio wave traveling through free space?
\begin{enumerate}[noitemsep]
    \item \textbf{Speed of light}
    \item Speed of sound
    \item Speed inversely proportional to its wavelength
    \item Speed that increases as the frequency increases
\end{enumerate}
\end{tcolorbox}

\subsubsection*{Intuitive Explanation}
Imagine you're sending a message using a flashlight in a completely empty space. The light from the flashlight travels at the fastest speed possible in the universe, which is the speed of light. Similarly, radio waves, which are a type of electromagnetic wave, also travel at this speed when they move through free space. So, the velocity of a radio wave in free space is the same as the speed of light.

\subsubsection*{Advanced Explanation}
Radio waves are a form of electromagnetic radiation, and like all electromagnetic waves, they travel through free space at the speed of light, denoted by \( c \). The speed of light in a vacuum is approximately \( 3 \times 10^8 \) meters per second. This speed is a fundamental constant of nature and is the same for all electromagnetic waves, regardless of their frequency or wavelength. The relationship between the speed of light (\( c \)), frequency (\( f \)), and wavelength (\( \lambda \)) is given by the equation:
\[
c = f \lambda
\]
This equation shows that the speed of light is constant, and any changes in frequency or wavelength are inversely proportional to each other. Therefore, the velocity of a radio wave in free space is always the speed of light.