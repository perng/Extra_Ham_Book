\subsection{Scanning Function of an FM Transceiver}
\label{T4B05}

\begin{tcolorbox}[colback=gray!10!white,colframe=black!75!black,title=T4B05]
What does the scanning function of an FM transceiver do?
\begin{enumerate}[noitemsep]
    \item Checks incoming signal deviation
    \item Prevents interference to nearby repeaters
    \item \textbf{Tunes through a range of frequencies to check for activity}
    \item Checks for messages left on a digital bulletin board
\end{enumerate}
\end{tcolorbox}

The scanning function of an FM transceiver is designed to automatically tune through a range of frequencies to check for any activity. This is particularly useful for monitoring multiple channels or frequencies without manually adjusting the transceiver. The correct answer is \textbf{C}.