\subsection{Entering a Transceiver’s Operating Frequency}
\label{T4B02}

\begin{tcolorbox}[colback=gray!10!white,colframe=black!75!black,title=T4B02]
Which of the following can be used to enter a transceiver’s operating frequency?
\begin{enumerate}[noitemsep]
    \item \textbf{The keypad or VFO knob}
    \item The CTCSS or DTMF encoder
    \item The Automatic Frequency Control
    \item All these choices are correct
\end{enumerate}
\end{tcolorbox}

To set the operating frequency on a transceiver, you typically use the keypad or the Variable Frequency Oscillator (VFO) knob. These are the standard methods for manually entering or adjusting the frequency. The other options, such as CTCSS or DTMF encoders and Automatic Frequency Control, are not used for this purpose.