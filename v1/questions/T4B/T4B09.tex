\subsection{Selecting a Group of Stations on a Digital Voice Transceiver}
\label{T4B09}

\begin{tcolorbox}[colback=gray!10!white,colframe=black!75!black,title=T4B09]
How is a specific group of stations selected on a digital voice transceiver?
\begin{enumerate}[noitemsep]
    \item By retrieving the frequencies from transceiver memory
    \item By enabling the group’s CTCSS tone
    \item \textbf{By entering the group’s identification code}
    \item By activating automatic identification
\end{enumerate}
\end{tcolorbox}

\subsubsection*{Explanation}
On a digital voice transceiver, selecting a specific group of stations is typically done by entering the group’s identification code. This code allows the transceiver to connect to the correct group of stations within the digital network. The other options, such as retrieving frequencies or enabling CTCSS tones, are more relevant to analog systems or specific filtering techniques, but not the primary method for group selection in digital voice systems.