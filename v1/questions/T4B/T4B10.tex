\subsection{Optimal Receiver Filter Bandwidth for SSB Reception}
\label{T4B10}

\begin{tcolorbox}[colback=gray!10!white,colframe=black!75!black,title=T4B10]
Which of the following receiver filter bandwidths provides the best signal-to-noise ratio for SSB reception?
\begin{enumerate}[noitemsep]
    \item 500 Hz
    \item 1000 Hz
    \item \textbf{2400 Hz}
    \item 5000 Hz
\end{enumerate}
\end{tcolorbox}

\subsubsection*{Intuitive Explanation}
Think of the receiver filter bandwidth as a gatekeeper that decides which signals get through to your radio. If the gate is too narrow (like 500 Hz), it might block some parts of the SSB signal, making it harder to hear clearly. If the gate is too wide (like 5000 Hz), it lets in a lot of noise along with the signal, which also makes it harder to hear clearly. The best gate size (2400 Hz) is just right—it lets through the entire SSB signal without letting in too much noise.

\subsubsection*{Advanced Explanation}
The signal-to-noise ratio (SNR) is a measure of how much desired signal is present compared to unwanted noise. For SSB (Single Sideband) reception, the optimal filter bandwidth should be wide enough to pass the entire SSB signal but narrow enough to exclude as much noise as possible. The typical bandwidth of an SSB signal is around 2400 Hz, which includes the voice frequencies and some guard bands. A filter bandwidth of 2400 Hz is therefore ideal because it matches the signal bandwidth, maximizing the SNR. A narrower bandwidth would attenuate parts of the signal, reducing the SNR, while a wider bandwidth would include more noise, also reducing the SNR.