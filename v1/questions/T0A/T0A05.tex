\subsection{Fuse Replacement Safety}
\label{T0A05}

\begin{tcolorbox}[colback=gray!10!white,colframe=black!75!black,title=T0A05]
Why should a 5-ampere fuse never be replaced with a 20-ampere fuse?
\begin{enumerate}[noitemsep]
    \item The larger fuse would be likely to blow because it is rated for higher current
    \item The power supply ripple would greatly increase
    \item \textbf{Excessive current could cause a fire}
    \item All these choices are correct
\end{enumerate}
\end{tcolorbox}

\subsubsection*{Intuitive Explanation}
Imagine a fuse as a safety guard for your electrical devices. If the guard is too lenient (like a 20-ampere fuse replacing a 5-ampere one), it might let too much electricity pass through, which can overheat and potentially cause a fire. Always use the correct fuse to keep your devices safe!

\subsubsection*{Advanced Explanation}
A fuse is designed to protect electrical circuits by breaking the circuit when the current exceeds a specified value. A 5-ampere fuse will blow when the current exceeds 5 amperes, preventing damage or fire. Replacing it with a 20-ampere fuse means the circuit will not break until the current exceeds 20 amperes, which could allow excessive current to flow, leading to overheating and potentially causing a fire. Always match the fuse rating to the circuit's requirements to ensure safety.