\subsection{Hazard in Power Supply After Turning Off}
\label{T0A11}

\begin{tcolorbox}[colback=gray!10!white,colframe=black!75!black,title=T0A11]
What hazard exists in a power supply immediately after turning it off?
\begin{enumerate}[noitemsep]
    \item Circulating currents in the dc filter
    \item Leakage flux in the power transformer
    \item Voltage transients from kickback diodes
    \item \textbf{Charge stored in filter capacitors}
\end{enumerate}
\end{tcolorbox}

\subsubsection*{Intuitive Explanation}
Imagine you have a water balloon that's full of water. Even after you stop filling it, the water doesn't just disappear—it stays in the balloon until you let it out. Similarly, in a power supply, the filter capacitors store electrical charge. When you turn off the power supply, this charge doesn't instantly vanish. It remains stored in the capacitors, posing a potential hazard if not properly discharged.

\subsubsection*{Advanced Explanation}
Filter capacitors in a power supply are used to smooth out the voltage by storing electrical charge. When the power supply is turned off, these capacitors retain their charge due to their inherent property of storing energy. This stored charge can be dangerous because it can deliver a significant electrical shock if someone comes into contact with the capacitor terminals. To mitigate this risk, power supplies often include discharge resistors or other mechanisms to safely dissipate the stored charge after the power is turned off.