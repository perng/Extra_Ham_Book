\subsection{Formula for Calculating Current in a Circuit}
\label{T5D01}

\begin{tcolorbox}[colback=gray!10!white,colframe=black!75!black,title=T5D01]
What formula is used to calculate current in a circuit?
\begin{enumerate}[noitemsep]
    \item I = E R
    \item \textbf{I = E / R}
    \item I = E + R
    \item I = E - R
\end{enumerate}
\end{tcolorbox}

\subsubsection*{Intuitive Explanation}
Imagine you have a water pipe. The water flow (current) depends on the pressure (voltage) and the resistance of the pipe. If you increase the pressure, more water flows. If the pipe is narrower (more resistance), less water flows. The formula \( I = \frac{E}{R} \) simply tells us that current is voltage divided by resistance. Easy, right?

\subsubsection*{Advanced Explanation}
Ohm's Law is a fundamental principle in electrical engineering that relates voltage (\( E \)), current (\( I \)), and resistance (\( R \)) in a circuit. The law states that the current through a conductor between two points is directly proportional to the voltage across the two points and inversely proportional to the resistance between them. Mathematically, this is expressed as:
\[
I = \frac{E}{R}
\]
where:
\begin{itemize}
    \item \( I \) is the current in amperes (A),
    \item \( E \) is the voltage in volts (V),
    \item \( R \) is the resistance in ohms ($\Omega$).
\end{itemize}
This formula is essential for analyzing and designing electrical circuits.