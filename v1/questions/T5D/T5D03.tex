\subsection{Formula for Calculating Resistance}
\label{T5D03}

\begin{tcolorbox}[colback=gray!10!white,colframe=black!75!black,title=T5D03]
What formula is used to calculate resistance in a circuit?
\begin{enumerate}[noitemsep]
    \item R = E x I
    \item \textbf{R = E / I}
    \item R = E + I
    \item R = E - I
\end{enumerate}
\end{tcolorbox}

\subsubsection*{Intuitive Explanation}
Imagine you have a water pipe. The water pressure (E) is like the voltage, and the flow rate (I) is like the current. The resistance (R) is how much the pipe resists the flow of water. To find out how much the pipe resists, you divide the pressure by the flow rate. So, resistance is voltage divided by current.

\subsubsection*{Advanced Explanation}
In electrical circuits, resistance (R) is a measure of how much a material opposes the flow of electric current. According to Ohm's Law, the relationship between voltage (E), current (I), and resistance (R) is given by the formula:
\[
R = \frac{E}{I}
\]
This formula states that resistance is equal to the voltage across the circuit divided by the current flowing through it. This fundamental relationship is crucial for analyzing and designing electrical circuits.