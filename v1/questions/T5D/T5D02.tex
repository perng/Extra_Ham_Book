\subsection{Voltage Calculation Formula}
\label{T5D02}

\begin{tcolorbox}[colback=gray!10!white,colframe=black!75!black,title=T5D02]
What formula is used to calculate voltage in a circuit?
\begin{enumerate}[noitemsep]
    \item \textbf{E = I x R}
    \item E = I / R
    \item E = I + R
    \item E = I - R
\end{enumerate}
\end{tcolorbox}

\subsubsection{Intuitive Explanation}
Imagine you have a water hose. The water flow (current, I) is like the amount of water coming out of the hose. The resistance (R) is like the narrowness of the hose. The voltage (E) is the pressure pushing the water through the hose. To find out how much pressure is needed to push a certain amount of water through a hose of a certain narrowness, you multiply the water flow by the narrowness. That's why the formula is E = I x R.

\subsubsection{Advanced Explanation}
In electrical circuits, voltage (E) is the potential difference that drives the current (I) through a resistance (R). Ohm's Law, which is fundamental in electrical engineering, states that the voltage across a conductor is directly proportional to the current flowing through it, provided the temperature remains constant. The mathematical representation of Ohm's Law is:

\[
E = I \times R
\]

Where:
\begin{itemize}
    \item \( E \) is the voltage in volts (V),
    \item \( I \) is the current in amperes (A),
    \item \( R \) is the resistance in ohms ($\Omega$).
\end{itemize}

This formula is essential for calculating the voltage in a circuit when the current and resistance are known. It is widely used in designing and analyzing electrical circuits.