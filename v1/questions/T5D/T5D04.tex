\subsection{Resistance Calculation} \label{T5D04}

\begin{tcolorbox}[colback=gray!10!white,colframe=black!75!black,title=T5D04]
What is the resistance of a circuit in which a current of 3 amperes flows when connected to 90 volts?
\begin{enumerate}[noitemsep]
    \item 3 ohms
    \item \textbf{30 ohms}
    \item 93 ohms
    \item 270 ohms
\end{enumerate}
\end{tcolorbox}

\subsubsection*{Intuitive Explanation}
Imagine you have a water hose connected to a pump. The pump is like the voltage, pushing water (current) through the hose. If the pump is pushing 90 units of pressure and you get 3 units of water flow, the hose's resistance is like how much it's resisting the water flow. In this case, the resistance is 30 ohms, meaning the hose is not too tight or too loose—just right for the flow.

\subsubsection*{Advanced Explanation}
To find the resistance in a circuit, we use Ohm's Law, which states that \( V = I \times R \), where \( V \) is voltage, \( I \) is current, and \( R \) is resistance. Rearranging the formula to solve for resistance, we get \( R = \frac{V}{I} \). Given \( V = 90 \) volts and \( I = 3 \) amperes, the resistance \( R \) is calculated as follows:

\[
R = \frac{V}{I} = \frac{90}{3} = 30 \, \text{ohms}
\]

Thus, the correct answer is 30 ohms.