\subsection{Voltage Across a Resistor}
\label{T5D11}

\begin{tcolorbox}[colback=gray!10!white,colframe=black!75!black,title=T5D11]
What is the voltage across a 10-ohm resistor if a current of 1 ampere flows through it?
\begin{enumerate}[noitemsep]
    \item 1 volt
    \item \textbf{10 volts}
    \item 11 volts
    \item 9 volts
\end{enumerate}
\end{tcolorbox}

\subsubsection*{Intuitive Explanation}
Imagine the resistor as a narrow pipe and the current as water flowing through it. The voltage is like the pressure pushing the water through the pipe. If the pipe is narrow (high resistance) and the water flow is steady (current), the pressure (voltage) needed to push the water through the pipe is directly related to how narrow the pipe is. In this case, a 10-ohm resistor with 1 ampere of current means the voltage is 10 volts.

\subsubsection*{Advanced Explanation}
According to Ohm's Law, the voltage \( V \) across a resistor is given by the product of the current \( I \) flowing through it and the resistance \( R \) of the resistor. Mathematically, this is expressed as:
\[
V = I \times R
\]
Given:
\[
I = 1 \text{ ampere}, \quad R = 10 \text{ ohms}
\]
Substituting the values:
\[
V = 1 \text{ A} \times 10 \text{ }\Omega = 10 \text{ volts}
\]
Thus, the voltage across the resistor is 10 volts.