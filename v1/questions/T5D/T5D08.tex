\subsection{Current Through a Resistor}
\label{T5D08}

\begin{tcolorbox}[colback=gray!10!white,colframe=black!75!black,title=T5D08]
What is the current through a 100-ohm resistor connected across 200 volts?
\begin{enumerate}[noitemsep]
    \item 20,000 amperes
    \item 0.5 amperes
    \item \textbf{2 amperes}
    \item 100 amperes
\end{enumerate}
\end{tcolorbox}

\subsubsection*{Intuitive Explanation}
Imagine you have a garden hose (the resistor) and you're trying to push water (the current) through it. The voltage is like the pressure you're applying to the water. If you increase the pressure, more water flows through the hose. Similarly, if you increase the voltage, more current flows through the resistor. In this case, you have a 100-ohm resistor and 200 volts of pressure. Using Ohm's Law, you can calculate the current.

\subsubsection*{Advanced Explanation}
Ohm's Law states that the current \( I \) through a resistor is equal to the voltage \( V \) across the resistor divided by the resistance \( R \). Mathematically, this is expressed as:
\[
I = \frac{V}{R}
\]
Given:
\[
V = 200 \text{ volts}, \quad R = 100 \text{ ohms}
\]
Substituting the values into Ohm's Law:
\[
I = \frac{200}{100} = 2 \text{ amperes}
\]
Therefore, the current through the resistor is 2 amperes.

% Diagram prompt: Generate a simple circuit diagram showing a 100-ohm resistor connected across a 200-volt battery. Use Circuitikz in LaTeX for the diagram.