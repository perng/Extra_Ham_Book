\subsection{Voltage Across Circuit Components}
\label{T5D14}

\begin{tcolorbox}[colback=gray!10!white,colframe=black!75!black,title=T5D14]
In which type of circuit is voltage the same across all components?
\begin{enumerate}[noitemsep]
    \item Series
    \item \textbf{Parallel}
    \item Resonant
    \item Branch
\end{enumerate}
\end{tcolorbox}

\subsubsection*{Intuitive Explanation}
Imagine you have a bunch of light bulbs connected in a circuit. If they are all connected side by side (like in a parallel circuit), each bulb gets the same amount of voltage from the power source. It's like giving each bulb its own direct line to the battery. In a series circuit, the bulbs are connected one after the other, so the voltage gets divided among them. So, in a parallel circuit, the voltage is the same across all components.

\subsubsection*{Advanced Explanation}
In a parallel circuit, all components are connected across the same two points, effectively creating multiple paths for the current to flow. This means that each component experiences the same potential difference (voltage) as the power source. The total current in the circuit is the sum of the currents through each component, but the voltage remains constant across all of them. This is in contrast to a series circuit, where the voltage is divided among the components. Therefore, the correct answer is \textbf{Parallel}.