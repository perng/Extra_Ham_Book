\subsection{DC Current in Circuit Types}
\label{T5D13}

\begin{tcolorbox}[colback=gray!10!white,colframe=black!75!black,title=T5D13]
In which type of circuit is DC current the same through all components?
\begin{enumerate}[noitemsep]
    \item \textbf{Series}
    \item Parallel
    \item Resonant
    \item Branch
\end{enumerate}
\end{tcolorbox}

\subsubsection*{Intuitive Explanation}
Imagine a group of friends walking in a single file line. Each friend represents a component in a circuit, and the line represents the path of the current. In a series circuit, the same current flows through each friend because there's only one path for the current to take. If one friend stops, the whole line stops, just like how a break in a series circuit stops the current flow.

\subsubsection*{Advanced Explanation}
In a series circuit, components are connected end-to-end, forming a single path for the current to flow. According to Kirchhoff's Current Law (KCL), the current entering a node must equal the current leaving the node. Since there are no branches in a series circuit, the current remains constant throughout all components. This is why the DC current is the same through all components in a series circuit. In contrast, parallel circuits have multiple paths for current to flow, resulting in different currents through each branch. Resonant circuits and branch circuits involve more complex behaviors and are not characterized by a constant current through all components.