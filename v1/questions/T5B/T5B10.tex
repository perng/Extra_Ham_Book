\subsection{Power Decrease in Decibels}
\label{T5B10}

\begin{tcolorbox}[colback=gray!10!white,colframe=black!75!black,title=T5B10]
Which decibel value most closely represents a power decrease from 12 watts to 3 watts?
\begin{enumerate}[noitemsep]
    \item -1 dB
    \item -3 dB
    \item \textbf{-6 dB}
    \item -9 dB
\end{enumerate}
\end{tcolorbox}

\subsubsection*{Intuitive Explanation}
Imagine you have a light bulb that's shining at 12 watts. If you dim it down to 3 watts, it's like turning down the brightness by a certain amount. Decibels (dB) are a way to measure this change in power. A decrease of 6 dB means the power has been reduced to one-fourth of its original value, which is exactly what happened here (12 watts to 3 watts).

\subsubsection*{Advanced Explanation}
The decibel (dB) is a logarithmic unit used to express the ratio of two power levels. The formula to calculate the power ratio in decibels is:

\[
\text{dB} = 10 \log_{10}\left(\frac{P_2}{P_1}\right)
\]

Where \(P_1\) is the initial power and \(P_2\) is the final power. In this case, \(P_1 = 12\) watts and \(P_2 = 3\) watts. Plugging these values into the formula:

\[
\text{dB} = 10 \log_{10}\left(\frac{3}{12}\right) = 10 \log_{10}\left(\frac{1}{4}\right) = 10 \times (-0.602) \approx -6 \text{ dB}
\]

Thus, a power decrease from 12 watts to 3 watts corresponds to a decrease of approximately -6 dB.