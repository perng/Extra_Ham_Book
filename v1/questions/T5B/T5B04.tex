\subsection{Understanding Microvolts}
\label{T5B04}

\begin{tcolorbox}[colback=gray!10!white,colframe=black!75!black,title=T5B04]
Which is equal to one microvolt?
\begin{enumerate}[noitemsep]
    \item \textbf{One one-millionth of a volt}
    \item One million volts
    \item One thousand kilovolts
    \item One one-thousandth of a volt
\end{enumerate}
\end{tcolorbox}

\subsubsection*{Intuitive Explanation}
Imagine a volt as a big pizza. If you cut that pizza into one million tiny slices, each slice would be a microvolt. So, a microvolt is just a very, very small piece of a volt—specifically, one one-millionth of it. It's like comparing a single grain of sand to a whole beach!

\subsubsection*{Advanced Explanation}
In the International System of Units (SI), the prefix micro denotes a factor of \(10^{-6}\). Therefore, one microvolt (\(\mu V\)) is equal to \(10^{-6}\) volts. This means that one microvolt is one one-millionth of a volt. Understanding these prefixes is crucial in electronics and radio technology, as they help in expressing very small or very large quantities in a more manageable form.