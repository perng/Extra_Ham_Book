\subsection{Milliamperes Conversion}
\label{T5B01}

\begin{tcolorbox}[colback=gray!10!white,colframe=black!75!black,title=T5B01]
How many milliamperes is 1.5 amperes?
\begin{enumerate}[noitemsep]
    \item 15 milliamperes
    \item 150 milliamperes
    \item \textbf{1500 milliamperes}
    \item 15,000 milliamperes
\end{enumerate}
\end{tcolorbox}

\subsubsection*{Intuitive Explanation}
Think of amperes (A) and milliamperes (mA) like dollars and cents. Just as 1 dollar is equal to 100 cents, 1 ampere is equal to 1000 milliamperes. So, if you have 1.5 dollars, you have 150 cents. Similarly, 1.5 amperes is 1500 milliamperes.

\subsubsection*{Advanced Explanation}
The prefix milli- denotes one-thousandth of a unit. Therefore, 1 ampere (A) is equivalent to 1000 milliamperes (mA). To convert amperes to milliamperes, you multiply the value in amperes by 1000. 

\[
1.5 \, \text{A} \times 1000 = 1500 \, \text{mA}
\]

Thus, 1.5 amperes is equal to 1500 milliamperes.