\subsection{Unit Conversion: MHz to kHz}
\label{T5B07}

\begin{tcolorbox}[colback=gray!10!white,colframe=black!75!black,title=T5B07]
Which is equal to 3.525 MHz?
\begin{enumerate}[noitemsep]
    \item 0.003525 kHz
    \item 35.25 kHz
    \item \textbf{3525 kHz}
    \item 3,525,000 kHz
\end{enumerate}
\end{tcolorbox}

\subsubsection*{Intuitive Explanation}
Think of MHz (Megahertz) and kHz (kilohertz) as different units for measuring frequency, just like miles and kilometers are different units for measuring distance. 1 MHz is equal to 1000 kHz. So, to convert 3.525 MHz to kHz, you simply multiply by 1000. It's like converting 3.525 miles to kilometers by multiplying by 1.609, but in this case, the conversion factor is 1000.

\subsubsection*{Advanced Explanation}
The prefix Mega (M) in MHz stands for $10^6$, and kilo (k) in kHz stands for $10^3$. Therefore, to convert from MHz to kHz, you multiply by $10^3$ (or 1000). Mathematically, this is represented as:
\[
3.525 \, \text{MHz} = 3.525 \times 10^3 \, \text{kHz} = 3525 \, \text{kHz}
\]
This conversion is straightforward and does not require complex calculations. The key is understanding the relationship between the prefixes and applying the correct conversion factor.