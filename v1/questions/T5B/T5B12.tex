\subsection{Frequency Conversion}
\label{T5B12}

\begin{tcolorbox}[colback=gray!10!white,colframe=black!75!black,title=T5B12]
Which is equal to 28400 kHz?
\begin{enumerate}[noitemsep]
    \item 28.400 kHz
    \item 2.800 MHz
    \item 284.00 MHz
    \item \textbf{28.400 MHz}
\end{enumerate}
\end{tcolorbox}

\subsubsection{Intuitive Explanation}
Imagine you have a big number like 28400 kHz, and you want to make it easier to read by converting it to a smaller unit. Just like converting 1000 meters to 1 kilometer, we can convert kHz to MHz. Here, 28400 kHz is the same as 28.400 MHz because 1000 kHz equals 1 MHz.

\subsubsection{Advanced Explanation}
Frequency units can be converted using the relationship between kilohertz (kHz) and megahertz (MHz). Specifically, 1 MHz is equal to 1000 kHz. To convert 28400 kHz to MHz, divide by 1000:

\[
28400 \text{ kHz} \div 1000 = 28.400 \text{ MHz}
\]

Thus, 28400 kHz is equivalent to 28.400 MHz. This conversion is straightforward and is commonly used in radio frequency measurements to simplify large numbers.