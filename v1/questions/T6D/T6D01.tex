\subsection{Rectification of Alternating Current}
\label{T6D01}

\begin{tcolorbox}[colback=gray!10!white,colframe=black!75!black,title=T6D01]
Which of the following devices or circuits changes an alternating current into a varying direct current signal?
\begin{enumerate}[noitemsep]
    \item Transformer
    \item \textbf{Rectifier}
    \item Amplifier
    \item Reflector
\end{enumerate}
\end{tcolorbox}

\subsubsection*{Intuitive Explanation}
Imagine you have a river that flows back and forth (alternating current). You want to make it flow in just one direction (direct current). A rectifier is like a one-way valve for electricity—it lets the current flow in only one direction, turning the back-and-forth flow into a steady stream.

\subsubsection*{Advanced Explanation}
A rectifier is an electrical device that converts alternating current (AC), which periodically reverses direction, to direct current (DC), which flows in only one direction. This process is known as rectification. Rectifiers are typically made using diodes, which allow current to flow in one direction only. There are different types of rectifiers, such as half-wave and full-wave rectifiers, each with its own method of converting AC to DC. The output of a rectifier is not perfectly smooth DC but rather a pulsating DC, which can be further smoothed using capacitors or other filtering techniques.