\subsection{Component for Voltage Reduction}
\label{T6D06}

\begin{tcolorbox}[colback=gray!10!white,colframe=black!75!black,title=T6D06]
What component changes 120 V AC power to a lower AC voltage for other uses?
\begin{enumerate}[noitemsep]
    \item Variable capacitor
    \item \textbf{Transformer}
    \item Transistor
    \item Diode
\end{enumerate}
\end{tcolorbox}

\subsubsection*{Intuitive Explanation}
Imagine you have a big water pipe with high pressure, and you need to reduce the pressure to use it in your garden hose. A transformer is like a magical device that can take the high pressure (voltage) from the big pipe and reduce it to a lower pressure suitable for your hose. In electrical terms, it takes the high voltage from your wall outlet and steps it down to a safer, lower voltage for your devices.

\subsubsection*{Advanced Explanation}
A transformer is an electrical device that transfers electrical energy between two or more circuits through electromagnetic induction. It consists of two coils of wire, known as the primary and secondary windings, which are wound around a core made of ferromagnetic material. When an alternating current (AC) flows through the primary winding, it creates a varying magnetic field in the core, which induces a voltage in the secondary winding. The ratio of the number of turns in the primary winding to the number of turns in the secondary winding determines the voltage transformation ratio. For example, if the primary winding has 120 turns and the secondary winding has 12 turns, the transformer will step down the voltage by a factor of 10, converting 120 V AC to 12 V AC. This principle is fundamental in power distribution and in the design of many electronic devices.