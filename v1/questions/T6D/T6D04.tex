\subsection{Electrical Quantity Display}
\label{T6D04}

\begin{tcolorbox}[colback=gray!10!white,colframe=black!75!black,title=T6D04]
Which of the following displays an electrical quantity as a numeric value?
\begin{enumerate}[noitemsep]
    \item Potentiometer
    \item Transistor
    \item \textbf{Meter}
    \item Relay
\end{enumerate}
\end{tcolorbox}

\subsubsection*{Intuitive Explanation}
Imagine you want to know how much electricity is flowing through a wire. You wouldn't use a knob (potentiometer), a switch (transistor), or a switch that turns things on and off (relay). Instead, you'd use a meter, which is like a tiny computer that shows you the exact number of electricity units.

\subsubsection*{Advanced Explanation}
A meter is an instrument designed to measure and display electrical quantities such as voltage, current, or resistance in numeric form. Unlike a potentiometer, which is a variable resistor used to adjust levels, or a transistor, which is a semiconductor device used to amplify or switch electronic signals, a meter provides a direct readout of the measured value. Relays, on the other hand, are electromechanical switches used to control circuits. Therefore, the correct answer is the meter, as it is specifically designed to display electrical quantities numerically.