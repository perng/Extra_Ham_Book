\subsection{Understanding FT8}
\label{T8D13}

\begin{tcolorbox}[colback=gray!10!white,colframe=black!75!black,title=T8D13]
What is FT8?
\begin{enumerate}[noitemsep]
    \item A wideband FM voice mode
    \item \textbf{A digital mode capable of low signal-to-noise operation}
    \item An eight channel multiplex mode for FM repeaters
    \item A digital slow-scan TV mode with forward error correction and automatic color compensation
\end{enumerate}
\end{tcolorbox}

FT8 is a digital communication mode designed for amateur radio operators. It is particularly effective in low signal-to-noise ratio (SNR) conditions, making it ideal for weak signal communication. Unlike traditional voice modes or wideband FM, FT8 uses a highly efficient digital protocol to transmit data quickly and reliably. This mode is not related to multiplexing or slow-scan TV, but rather focuses on maximizing the efficiency of data transmission in challenging conditions.