\subsection{Mesh Network in Amateur Radio}
\label{T8D12}

\begin{tcolorbox}[colback=gray!10!white,colframe=black!75!black,title=T8D12]
Which of the following best describes an amateur radio mesh network?
\begin{enumerate}[noitemsep]
    \item \textbf{An amateur-radio based data network using commercial Wi-Fi equipment with modified firmware}
    \item A wide-bandwidth digital voice mode employing DMR protocols
    \item A satellite communications network using modified commercial satellite TV hardware
    \item An internet linking protocol used to network repeaters
\end{enumerate}
\end{tcolorbox}

An amateur radio mesh network is essentially a data network that leverages modified commercial Wi-Fi equipment to facilitate communication among amateur radio operators. This setup allows for the creation of a robust and flexible network that can operate independently of traditional internet infrastructure. The correct answer is \textbf{A}, as it accurately describes the use of modified Wi-Fi equipment in an amateur radio context.