\subsection{Using a VHF/UHF Transceiver's Reverse Function}
\label{T2B01}

\begin{tcolorbox}[colback=gray!10!white,colframe=black!75!black,title=T2B01]
How is a VHF/UHF transceiver’s “reverse” function used?
\begin{enumerate}[noitemsep]
    \item To reduce power output
    \item To increase power output
    \item \textbf{To listen on a repeater’s input frequency}
    \item To listen on a repeater’s output frequency
\end{enumerate}
\end{tcolorbox}

\subsubsection*{Explanation}
The reverse function on a VHF/UHF transceiver is used to switch the listening frequency from the repeater's output frequency to its input frequency. This allows the operator to hear the signals being transmitted directly to the repeater, rather than the signals being retransmitted by the repeater. This can be useful for monitoring the quality of the signal being sent to the repeater or for troubleshooting communication issues.