\subsection{Signaling with Audio Tone Pairs}\label{T2B06}

\begin{tcolorbox}[colback=gray!10!white,colframe=black!75!black,title=T2B06]
What type of signaling uses pairs of audio tones?
\begin{enumerate}[noitemsep]
    \item \textbf{DTMF}
    \item CTCSS
    \item GPRS
    \item D-STAR
\end{enumerate}
\end{tcolorbox}

\subsubsection*{Explanation}
DTMF (Dual-Tone Multi-Frequency) signaling uses pairs of audio tones to represent different digits or commands. This method is commonly used in telephone systems for dialing numbers and in some radio systems for remote control operations. Each key press generates a unique combination of two tones, one from a low-frequency group and one from a high-frequency group. This allows for reliable and distinct signaling even over noisy channels.