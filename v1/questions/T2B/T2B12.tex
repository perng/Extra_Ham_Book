\subsection{Color Code in DMR Repeater Systems}
\label{T2B12}

\begin{tcolorbox}[colback=gray!10!white,colframe=black!75!black,title=T2B12]
What is the purpose of the color code used on DMR repeater systems?
\begin{enumerate}[noitemsep]
    \item \textbf{Must match the repeater color code for access}
    \item Defines the frequency pair to use
    \item Identifies the codec used
    \item Defines the minimum signal level required for access
\end{enumerate}
\end{tcolorbox}

The color code in DMR (Digital Mobile Radio) repeater systems is a simple yet essential feature. It ensures that only radios with the correct color code can communicate through the repeater. This helps in preventing interference from other nearby DMR systems operating on the same frequency. The correct answer is \textbf{A}, as the color code must match the repeater's color code for access.