\subsection{Common Repeater Frequency Offset in the 2 Meter Band}
\label{T2A01}

\begin{tcolorbox}[colback=gray!10!white,colframe=black!75!black,title=T2A01]
What is a common repeater frequency offset in the 2 meter band?
\begin{enumerate}[noitemsep]
    \item Plus or minus 5 MHz
    \item \textbf{Plus or minus 600 kHz}
    \item Plus or minus 500 kHz
    \item Plus or minus 1 MHz
\end{enumerate}
\end{tcolorbox}

In the 2 meter band, repeaters typically use a frequency offset to separate the transmit and receive frequencies. This offset helps prevent interference and allows for simultaneous transmission and reception. The most common offset in this band is \textbf{plus or minus 600 kHz}. This means that if a repeater is receiving on a certain frequency, it will transmit on a frequency that is 600 kHz higher or lower than the receive frequency.