\subsection{Amateur Station Transmission and Reception on Same Frequency}
\label{T2A11}

\begin{tcolorbox}[colback=gray!10!white,colframe=black!75!black,title=T2A11]
What term describes an amateur station that is transmitting and receiving on the same frequency?
\begin{enumerate}[noitemsep]
    \item Full duplex
    \item Diplex
    \item \textbf{Simplex}
    \item Multiplex
\end{enumerate}
\end{tcolorbox}

\subsubsection*{Intuitive Explanation}
Imagine you're talking on a walkie-talkie. When you press the button to talk, you can't hear the other person, and when you release the button to listen, you can't talk. This is because both talking and listening happen on the same frequency. This is called \textbf{simplex} communication. It's like a one-lane road where traffic can only go one way at a time.

\subsubsection*{Advanced Explanation}
In radio communication, \textbf{simplex} refers to a mode where transmission and reception occur on the same frequency, but not simultaneously. This is different from \textbf{full duplex}, where transmission and reception can happen at the same time on different frequencies, and \textbf{half duplex}, where transmission and reception can alternate on the same frequency but not simultaneously. Simplex is commonly used in amateur radio for its simplicity and efficiency in certain scenarios, such as direct communication between two stations without the need for complex frequency management.