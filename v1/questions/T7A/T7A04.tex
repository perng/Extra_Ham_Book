\subsection{Receiver Signal Discrimination Ability}
\label{T7A04}

\begin{tcolorbox}[colback=gray!10!white,colframe=black!75!black,title=T7A04]
Which term describes the ability of a receiver to discriminate between multiple signals?
\begin{enumerate}[noitemsep]
    \item Discrimination ratio
    \item Sensitivity
    \item \textbf{Selectivity}
    \item Harmonic distortion
\end{enumerate}
\end{tcolorbox}

\subsubsection*{Intuitive Explanation}
Imagine you're at a party with multiple conversations happening at once. Your ability to focus on one conversation while ignoring the others is similar to a receiver's ability to pick out one signal from many. This ability is called \textbf{selectivity}.

\subsubsection*{Advanced Explanation}
Selectivity in radio receivers refers to the ability to distinguish between signals of different frequencies. It is a crucial parameter because it determines how well a receiver can isolate a desired signal from other signals that may be present on nearby frequencies. Selectivity is often achieved through the use of filters that allow the desired frequency to pass while attenuating others. High selectivity is essential in crowded frequency bands to avoid interference from adjacent channels.