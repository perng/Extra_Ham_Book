\subsection{Function of a Transceiver’s PTT Input}
\label{T7A07}

\begin{tcolorbox}[colback=gray!10!white,colframe=black!75!black,title=T7A07]
What is the function of a transceiver’s PTT input?
\begin{enumerate}[noitemsep]
    \item Input for a key used to send CW
    \item \textbf{Switches transceiver from receive to transmit when grounded}
    \item Provides a transmit tuning tone when grounded
    \item Input for a preamplifier tuning tone
\end{enumerate}
\end{tcolorbox}

The PTT (Push-To-Talk) input on a transceiver is a simple yet crucial feature. When grounded, it switches the transceiver from receive mode to transmit mode, allowing the operator to communicate. This is a fundamental function in radio communication, ensuring that the transceiver only transmits when the operator intends to do so.