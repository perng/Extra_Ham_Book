\subsection{Frequency Conversion Devices}
\label{T7A03}

\begin{tcolorbox}[colback=gray!10!white,colframe=black!75!black,title=T7A03]
Which of the following is used to convert a signal from one frequency to another?
\begin{enumerate}[noitemsep]
    \item Phase splitter
    \item \textbf{Mixer}
    \item Inverter
    \item Amplifier
\end{enumerate}
\end{tcolorbox}

\subsubsection*{Intuitive Explanation}
Imagine you have a radio that can only listen to one station at a time. To listen to a different station, you need to change the frequency. A mixer is like the magic button on your radio that helps you switch from one frequency to another. It takes the signal from one frequency and converts it to a different frequency so you can tune in to your favorite station.

\subsubsection*{Advanced Explanation}
A mixer is a crucial component in radio frequency (RF) systems that performs frequency conversion. It takes two input signals—typically a local oscillator (LO) signal and a radio frequency (RF) signal—and produces an output signal at a frequency that is the sum or difference of the input frequencies. This process is known as heterodyning. The mixer is essential in superheterodyne receivers, where it converts the incoming RF signal to an intermediate frequency (IF) for easier processing and filtering. The correct answer is \textbf{B}, as the mixer is specifically designed for frequency conversion.