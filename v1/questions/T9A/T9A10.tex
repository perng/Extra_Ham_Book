\subsection{Half-Wave Dipole Antenna Radiation Pattern}
\label{T9A10}

\begin{tcolorbox}[colback=gray!10!white,colframe=black!75!black,title=T9A10]
In which direction does a half-wave dipole antenna radiate the strongest signal?
\begin{enumerate}[noitemsep]
    \item Equally in all directions
    \item Off the ends of the antenna
    \item In the direction of the feed line
    \item \textbf{Broadside to the antenna}
\end{enumerate}
\end{tcolorbox}

\subsubsection*{Intuitive Explanation}
Imagine a half-wave dipole antenna as a straight stick that's waving in the air. When you send a signal through it, the stick doesn't radiate the signal equally in all directions. Instead, it's like the stick is shouting the loudest to the sides, not to the ends. So, the strongest signal is sent out to the sides, or broadside, of the antenna.

\subsubsection*{Advanced Explanation}
A half-wave dipole antenna is designed to radiate electromagnetic waves most effectively in a direction perpendicular to its axis. This is known as the broadside direction. The radiation pattern of a half-wave dipole is typically doughnut-shaped, with the antenna lying along the axis of the doughnut. The signal strength is maximum in the plane perpendicular to the antenna and diminishes as you move towards the ends of the antenna. This is due to the current distribution along the antenna, which is maximum at the center and decreases towards the ends, resulting in the strongest radiation broadside to the antenna.