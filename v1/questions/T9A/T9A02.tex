\subsection{Antenna Loading Types}
\label{T9A02}

\begin{tcolorbox}[colback=gray!10!white,colframe=black!75!black,title=T9A02]
Which of the following describes a type of antenna loading?
\begin{enumerate}[noitemsep]
    \item \textbf{Electrically lengthening by inserting inductors in radiating elements}
    \item Inserting a resistor in the radiating portion of the antenna to make it resonant
    \item Installing a spring in the base of a mobile vertical antenna to make it more flexible
    \item Strengthening the radiating elements of a beam antenna to better resist wind damage
\end{enumerate}
\end{tcolorbox}

\subsubsection*{Brief Explanation}
Antenna loading refers to techniques used to modify the electrical characteristics of an antenna. In this case, the correct answer involves electrically lengthening the antenna by inserting inductors into the radiating elements. This method effectively increases the antenna's electrical length without physically altering its size, which can be useful for tuning and optimizing performance.