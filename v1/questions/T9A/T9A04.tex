\subsection{Disadvantage of Short, Flexible Antennas}
\label{T9A04}

\begin{tcolorbox}[colback=gray!10!white,colframe=black!75!black,title=T9A04]
What is a disadvantage of the short, flexible antenna supplied with most handheld radio transceivers, compared to a full-sized quarter-wave antenna?
\begin{enumerate}[noitemsep]
    \item \textbf{It has low efficiency}
    \item It transmits only circularly polarized signals
    \item It is mechanically fragile
    \item All these choices are correct
\end{enumerate}
\end{tcolorbox}

\subsubsection*{Intuitive Explanation}
Imagine you have a tiny, bendy antenna on your walkie-talkie. It's convenient because it doesn't get in the way, but it's not as good at sending and receiving signals as a bigger, full-sized antenna. This is because the short antenna isn't as efficient—it doesn't convert as much of the radio's power into actual radio waves that can travel far.

\subsubsection*{Advanced Explanation}
A full-sized quarter-wave antenna is designed to be resonant at the operating frequency, which means it efficiently radiates the signal. In contrast, the short, flexible antenna is often much shorter than a quarter-wavelength, leading to a mismatch in impedance and lower radiation efficiency. This inefficiency results in more of the transmitter's power being lost as heat rather than being radiated as useful signal. While the short antenna is convenient and durable, its reduced efficiency is a significant drawback compared to a full-sized antenna.