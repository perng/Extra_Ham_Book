\subsection{Frequency Ranges for Technician Licensees}
\label{T1B01}

\begin{tcolorbox}[colback=gray!10!white,colframe=black!75!black,title=T1B01]
Which of the following frequency ranges are available for phone operation by Technician licensees?
\begin{enumerate}[label=\Alph*),noitemsep]
    \item 28.050 MHz to 28.150 MHz
    \item 28.100 MHz to 28.300 MHz
    \item \textbf{28.300 MHz to 28.500 MHz}
    \item 28.500 MHz to 28.600 MHz
\end{enumerate}
\end{tcolorbox}

\subsubsection*{Intuitive Explanation}
Think of the radio frequency spectrum as a big highway with different lanes. Each lane is reserved for specific types of communication. For Technician licensees, the lane marked 28.300 MHz to 28.500 MHz is where you can make phone calls. The other lanes are either for different types of communication or are reserved for other license classes.

\subsubsection*{Advanced Explanation}
The frequency range from 28.300 MHz to 28.500 MHz is part of the 10-meter band, which is allocated for amateur radio use. Technician licensees are permitted to use this range for phone (voice) communication. The other options provided either fall outside the authorized frequency range for Technician licensees or are designated for other modes of communication such as data or Morse code. Understanding the frequency allocations for different license classes is crucial for operating within the legal limits of amateur radio.