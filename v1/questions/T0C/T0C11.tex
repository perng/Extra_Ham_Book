\subsection{Duty Cycle Definition for RF Exposure}
\label{T0C11}

\begin{tcolorbox}[colback=gray!10!white,colframe=black!75!black,title=T0C11]
What is the definition of duty cycle during the averaging time for RF exposure?
\begin{enumerate}[noitemsep]
    \item The difference between the lowest power output and the highest power output of a transmitter
    \item The difference between the PEP and average power output of a transmitter
    \item \textbf{The percentage of time that a transmitter is transmitting}
    \item The percentage of time that a transmitter is not transmitting
\end{enumerate}
\end{tcolorbox}

The duty cycle during the averaging time for RF exposure refers to the proportion of time that a transmitter is actively transmitting signals. This is important for calculating the average power output and ensuring compliance with RF exposure limits. The correct answer is \textbf{C}, which accurately defines the duty cycle as the percentage of time that a transmitter is transmitting.