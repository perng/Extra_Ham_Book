\subsection{Maximum Permissible Exposure and Frequency}
\label{T0C02}

\begin{tcolorbox}[colback=gray!10!white,colframe=black!75!black,title=T0C02]
At which of the following frequencies does maximum permissible exposure have the lowest value?
\begin{enumerate}[noitemsep]
    \item 3.5 MHz
    \item \textbf{50 MHz}
    \item 440 MHz
    \item 1296 MHz
\end{enumerate}
\end{tcolorbox}

\subsubsection*{Intuitive Explanation}
Think of maximum permissible exposure (MPE) as a safety limit for how much radio frequency energy your body can handle. Just like how you need to wear sunscreen to protect your skin from too much sun, MPE protects you from too much radio energy. Interestingly, the safety limit changes depending on the frequency of the radio waves. Lower frequencies (like 3.5 MHz) are generally safer, while higher frequencies (like 1296 MHz) can be more harmful. However, there's a sweet spot around 50 MHz where the MPE is the lowest, meaning you need to be extra careful at this frequency.

\subsubsection*{Advanced Explanation}
Maximum Permissible Exposure (MPE) limits are defined by regulatory bodies to ensure that human exposure to radio frequency (RF) energy remains within safe levels. These limits vary with frequency due to the way RF energy interacts with biological tissues. At lower frequencies (e.g., 3.5 MHz), the energy is less likely to be absorbed by the body, resulting in higher MPE limits. As the frequency increases, the energy absorption increases, leading to lower MPE limits. However, around 50 MHz, the MPE reaches its lowest point due to the resonant absorption characteristics of human tissues at this frequency. Beyond this point, as frequencies continue to increase (e.g., 440 MHz and 1296 MHz), the MPE limits start to rise again because the energy penetration depth decreases, reducing the overall absorption. Therefore, the frequency with the lowest MPE value is 50 MHz.