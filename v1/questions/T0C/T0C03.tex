\subsection{RF Safety Power Density and Duty Cycle}
\label{T0C03}

\begin{tcolorbox}[colback=gray!10!white,colframe=black!75!black,title=T0C03]
How does the allowable power density for RF safety change if duty cycle changes from 100 percent to 50 percent?
\begin{enumerate}[noitemsep]
    \item It increases by a factor of 3
    \item It decreases by 50 percent
    \item \textbf{It increases by a factor of 2}
    \item There is no adjustment allowed for lower duty cycle
\end{enumerate}
\end{tcolorbox}

\subsubsection*{Intuitive Explanation}
Think of the duty cycle as how often you're using a microwave oven. If you use it all the time (100\% duty cycle), you need to be careful not to overheat it. But if you only use it half the time (50\% duty cycle), you can afford to crank up the power a bit more without causing harm. This is because the average power over time is lower, so the allowable power density can be higher.

\subsubsection*{Advanced Explanation}
The allowable power density for RF safety is determined by the average power over time. When the duty cycle decreases from 100\% to 50\%, the transmitter is only active half the time. This means the average power is halved, allowing the peak power density to be doubled without exceeding safety limits. This adjustment ensures that the exposure to RF energy remains within safe levels, even when the transmitter is operating at higher peak powers for shorter durations.