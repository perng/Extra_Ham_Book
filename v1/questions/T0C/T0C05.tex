\subsection{Exposure Limits and Frequency}
\label{T0C05}

\begin{tcolorbox}[colback=gray!10!white,colframe=black!75!black,title=T0C05]
Why do exposure limits vary with frequency?
\begin{enumerate}[noitemsep]
    \item Lower frequency RF fields have more energy than higher frequency fields
    \item Lower frequency RF fields do not penetrate the human body
    \item Higher frequency RF fields are transient in nature
    \item \textbf{The human body absorbs more RF energy at some frequencies than at others}
\end{enumerate}
\end{tcolorbox}

\subsubsection*{Intuitive Explanation}
Think of your body as a sponge and radio frequencies (RF) as water. Just like a sponge absorbs water differently depending on the type of water (hot, cold, salty, etc.), your body absorbs RF energy differently depending on the frequency. Some frequencies are like hot water—they get absorbed more easily, while others are like cold water—they don't get absorbed as much. That's why exposure limits vary with frequency; to protect you from the frequencies that your body absorbs more easily.

\subsubsection*{Advanced Explanation}
The human body's absorption of RF energy is frequency-dependent due to the interaction between electromagnetic waves and biological tissues. At certain frequencies, the body's tissues resonate, leading to higher absorption rates. This phenomenon is known as the specific absorption rate (SAR). For example, frequencies around 30 MHz to 300 MHz are particularly effective at penetrating and being absorbed by the body, which is why exposure limits are stricter in this range. The International Commission on Non-Ionizing Radiation Protection (ICNIRP) and other regulatory bodies set exposure limits based on extensive research into how different frequencies interact with human tissues, ensuring safety across the RF spectrum.

% Diagram prompt: Generate a graph showing the Specific Absorption Rate (SAR) versus frequency for human tissues. Use Python with Matplotlib for the graph. The x-axis should represent frequency (Hz) on a logarithmic scale, and the y-axis should represent SAR (W/kg). Highlight the frequency range where SAR peaks.