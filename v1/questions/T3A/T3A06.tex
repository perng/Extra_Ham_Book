\subsection{Picket Fencing in Radio Communications}
\label{T3A06}

\begin{tcolorbox}[colback=gray!10!white,colframe=black!75!black,title=T3A06]
What is the meaning of the term “picket fencing”?
\begin{enumerate}[noitemsep]
    \item Alternating transmissions during a net operation
    \item \textbf{Rapid flutter on mobile signals due to multipath propagation}
    \item A type of ground system used with vertical antennas
    \item Local vs long-distance communications
\end{enumerate}
\end{tcolorbox}

The term picket fencing refers to the rapid flutter or distortion experienced in mobile radio signals due to multipath propagation. This phenomenon occurs when radio waves reflect off various surfaces, such as buildings or terrain, causing multiple signal paths to arrive at the receiver at slightly different times. The result is a signal that fluctuates rapidly, resembling the appearance of a picket fence. This is a common issue in mobile communications, especially in urban environments with many reflective surfaces.