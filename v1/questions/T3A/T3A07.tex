\subsection{Weather Impact on Microwave Range}
\label{T3A07}

\begin{tcolorbox}[colback=gray!10!white,colframe=black!75!black,title=T3A07]
What weather condition might decrease range at microwave frequencies?
\begin{enumerate}[noitemsep]
    \item High winds
    \item Low barometric pressure
    \item \textbf{Precipitation}
    \item Colder temperatures
\end{enumerate}
\end{tcolorbox}

\subsubsection*{Intuitive Explanation}
Imagine you're trying to shout across a field. If it starts raining heavily, your voice won't carry as far because the rain absorbs and scatters the sound waves. Similarly, at microwave frequencies, precipitation like rain or snow can absorb and scatter the radio waves, reducing their range.

\subsubsection*{Advanced Explanation}
Microwave frequencies are particularly susceptible to attenuation caused by precipitation. When radio waves encounter rain, snow, or other forms of precipitation, the water droplets or ice particles absorb and scatter the energy of the waves. This phenomenon is known as rain fade and is a significant factor in the degradation of signal strength over distance. The higher the frequency, the more pronounced this effect becomes. Therefore, precipitation is the weather condition most likely to decrease the range at microwave frequencies.