\subsection{Effects of Elliptical Polarization in Ionospheric Signals}
\label{T3A09}

\begin{tcolorbox}[colback=gray!10!white,colframe=black!75!black,title=T3A09]
Which of the following results from the fact that signals propagated by the ionosphere are elliptically polarized?
\begin{enumerate}[noitemsep]
    \item Digital modes are unusable
    \item \textbf{Either vertically or horizontally polarized antennas may be used for transmission or reception}
    \item FM voice is unusable
    \item Both the transmitting and receiving antennas must be of the same polarization
\end{enumerate}
\end{tcolorbox}

\subsubsection*{Explanation}
When signals propagate through the ionosphere, they often become elliptically polarized. This means that the electric field of the signal rotates as it travels, creating an elliptical pattern. As a result, the polarization of the signal at the receiving end can vary. This allows for flexibility in the choice of antennas: either vertically or horizontally polarized antennas can be used for transmission or reception, as the elliptical polarization ensures that the signal can be received regardless of the antenna's orientation. This is why option B is correct.