\subsection{Computer-Radio Interface Signals for Digital Mode Operation}
\label{T4A06}

\begin{tcolorbox}[colback=gray!10!white,colframe=black!75!black,title=T4A06]
What signals are used in a computer-radio interface for digital mode operation?
\begin{enumerate}[noitemsep]
    \item Receive and transmit mode, status, and location
    \item Antenna and RF power
    \item \textbf{Receive audio, transmit audio, and transmitter keying}
    \item NMEA GPS location and DC power
\end{enumerate}
\end{tcolorbox}

\subsubsection*{Explanation}
In digital mode operation, the computer-radio interface primarily handles audio signals and control signals. The receive audio signal carries the incoming digital data from the radio to the computer, while the transmit audio signal sends the outgoing digital data from the computer to the radio. The transmitter keying signal is used to control when the radio should transmit the data. These three signals are essential for effective digital communication between the computer and the radio.