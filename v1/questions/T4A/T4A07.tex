\subsection{Computer and Transceiver Connection for Digital Modes}
\label{T4A07}

\begin{tcolorbox}[colback=gray!10!white,colframe=black!75!black,title=T4A07]
Which of the following connections is made between a computer and a transceiver to use computer software when operating digital modes?
\begin{enumerate}[noitemsep]
    \item Computer “line out” to transceiver push-to-talk
    \item Computer “line in” to transceiver push-to-talk
    \item \textbf{Computer “line in” to transceiver speaker connector}
    \item Computer “line out” to transceiver speaker connector
\end{enumerate}
\end{tcolorbox}

To operate digital modes using computer software, the computer needs to receive audio signals from the transceiver. This is typically done by connecting the computer's “line in” to the transceiver's speaker connector. This allows the computer to process the audio signals for digital communication.