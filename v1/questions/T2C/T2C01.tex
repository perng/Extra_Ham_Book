\subsection{FCC Rules and Amateur Station Operation}
\label{T2C01}

\begin{tcolorbox}[colback=gray!10!white,colframe=black!75!black,title=T2C01]
When do FCC rules NOT apply to the operation of an amateur station?
\begin{enumerate}[noitemsep]
    \item When operating a RACES station
    \item When operating under special FEMA rules
    \item When operating under special ARES rules
    \item \textbf{FCC rules always apply}
\end{enumerate}
\end{tcolorbox}

\subsubsection*{Intuitive Explanation}
Think of the FCC (Federal Communications Commission) as the traffic police for radio waves. Just like how traffic rules apply to all vehicles on the road, FCC rules apply to all amateur radio stations. No matter what special club or emergency group you're part of, the FCC's rules are always in effect. So, there's no off-duty time for these rules when you're operating an amateur station.

\subsubsection*{Advanced Explanation}
The FCC regulates all radio communications in the United States, including amateur radio stations. This regulation ensures that radio frequencies are used efficiently and without causing harmful interference. Whether you're operating under the Radio Amateur Civil Emergency Service (RACES), Federal Emergency Management Agency (FEMA) rules, or Amateur Radio Emergency Service (ARES) rules, the FCC's overarching regulations still apply. This is because these special services operate within the broader framework of amateur radio, which is governed by the FCC. Therefore, the correct answer is that FCC rules always apply.