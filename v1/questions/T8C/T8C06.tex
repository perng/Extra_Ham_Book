\subsection{Accessing IRLP Nodes Over the Air}
\label{T8C06}

\begin{tcolorbox}[colback=gray!10!white,colframe=black!75!black,title=T8C06]
How is over the air access to IRLP nodes accomplished?
\begin{enumerate}[noitemsep]
    \item By obtaining a password that is sent via voice to the node
    \item \textbf{By using DTMF signals}
    \item By entering the proper internet password
    \item By using CTCSS tone codes
\end{enumerate}
\end{tcolorbox}

\subsubsection*{Intuitive Explanation}
Think of IRLP nodes as a special kind of radio that can connect to other radios over the internet. To access these nodes over the air, you need a way to knock on the door and say, Hey, let me in! Instead of using a key or a password, you use DTMF signals, which are like musical tones that the node recognizes as your way of asking for access.

\subsubsection*{Advanced Explanation}
IRLP (Internet Radio Linking Project) nodes are devices that allow amateur radio operators to connect their radios over the internet. To access these nodes over the air, operators use DTMF (Dual-Tone Multi-Frequency) signals. These signals are combinations of two specific frequencies that correspond to the buttons on a telephone keypad. When you send the correct sequence of DTMF tones, the IRLP node recognizes it as a valid access request and allows you to connect. This method is secure and efficient, as it doesn't require voice passwords or internet credentials, which could be intercepted or misused.