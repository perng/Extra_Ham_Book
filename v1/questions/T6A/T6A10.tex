\subsection{Battery Chemistries and Rechargeability}
\label{T6A10}

\begin{tcolorbox}[colback=gray!10!white,colframe=black!75!black,title=T6A10]
Which of the following battery chemistries is rechargeable?
\begin{enumerate}[noitemsep]
    \item Nickel-metal hydride
    \item Lithium-ion
    \item Lead-acid
    \item \textbf{All these choices are correct}
\end{enumerate}
\end{tcolorbox}

\subsubsection*{Intuitive Explanation}
Think of batteries as tiny energy storage units. Some batteries, like the ones in your TV remote, are single-use and can't be recharged. Others, like the ones in your phone or car, can be recharged and used over and over again. The question is asking which types of batteries fall into the rechargeable category. The answer is simple: all of them! Nickel-metal hydride, lithium-ion, and lead-acid batteries are all rechargeable.

\subsubsection*{Advanced Explanation}
Rechargeable batteries, also known as secondary cells, can be recharged by applying an electric current, which reverses the chemical reactions that occur during discharge. 

\begin{itemize}
    \item \textbf{Nickel-metal hydride (NiMH)}: These batteries use a nickel oxide hydroxide cathode and a hydrogen-absorbing alloy anode. They are commonly used in portable electronics and hybrid vehicles due to their high energy density and relatively low cost.
    
    \item \textbf{Lithium-ion (Li-ion)}: These batteries use lithium ions moving between the anode and cathode to store and release energy. They are widely used in consumer electronics, electric vehicles, and renewable energy storage due to their high energy density and long cycle life.
    
    \item \textbf{Lead-acid}: These are one of the oldest types of rechargeable batteries, using lead dioxide as the positive electrode and metallic lead as the negative electrode. They are commonly used in automotive applications and backup power systems due to their reliability and low cost.
\end{itemize}

All three chemistries are designed to be recharged, making them suitable for applications where long-term use is required.