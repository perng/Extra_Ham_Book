\subsection{Exploring the Great EME Distance: How Far Apart Can We Connect?}

\begin{tcolorbox}[colback=gray!10!white,colframe=black!75!black,title=E3A01] What is the approximate maximum separation measured along the surface of the Earth between two stations communicating by EME?
    \begin{enumerate}[label=\Alph*),noitemsep]
        \item 2,000 miles, if the moon is at perigee
        \item 2,000 miles, if the moon is at apogee
        \item 5,000 miles, if the moon is at perigee
        \item \textbf{12,000 miles, if the moon is “visible” by both stations}
    \end{enumerate}
\end{tcolorbox}

\subsubsection{Intuitive Explanation}
Imagine you and a friend are trying to talk to each other using the moon as a giant mirror. The moon is really far away, so you can both see it from different places on Earth. The farthest apart you can be and still both see the moon is about 12,000 miles. This is because the Earth is round, and the moon is high enough in the sky that it can be seen from two points on opposite sides of the Earth. So, if you and your friend are 12,000 miles apart and both can see the moon, you can use it to send messages to each other!

\subsubsection{Advanced Explanation}
EME (Earth-Moon-Earth) communication involves using the moon as a passive reflector for radio signals. The maximum separation between two stations communicating via EME is determined by the geometry of the Earth and the moon's position in the sky. 

The Earth's circumference is approximately 24,901 miles. Since the moon is approximately 238,900 miles away from Earth, it subtends an angle that allows it to be visible from two points on Earth that are nearly opposite each other. The maximum separation between these two points is approximately half the Earth's circumference, which is about 12,000 miles. This is the maximum distance at which both stations can have the moon in their line of sight simultaneously, allowing for EME communication.

Mathematically, the maximum separation \( d \) can be approximated as:
\[
d \approx \frac{C}{2}
\]
where \( C \) is the Earth's circumference. Substituting the value of \( C \):
\[
d \approx \frac{24,901 \text{ miles}}{2} \approx 12,000 \text{ miles}
\]

This calculation assumes that the moon is visible to both stations, meaning it is above the horizon for both locations. The moon's position at perigee or apogee does not significantly affect this maximum separation, as the primary factor is the Earth's curvature and the moon's altitude.

% Diagram Prompt: Generate a diagram showing the Earth, the moon, and two stations on opposite sides of the Earth, with the moon visible to both stations. Label the maximum separation distance of 12,000 miles.