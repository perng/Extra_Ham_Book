\subsection{Exploring the Magic of Microwave Ducting!}

\begin{tcolorbox}[colback=blue!5!white,colframe=blue!75!black]
    \textbf{E3A11} What is a typical range for tropospheric duct propagation of microwave signals?
    \begin{enumerate}[label=\Alph*),noitemsep]
        \item 10 miles to 50 miles
        \item \textbf{100 miles to 300 miles}
        \item 1,200 miles
        \item 2,500 miles
    \end{enumerate}
\end{tcolorbox}

\subsubsection{Intuitive Explanation}
Imagine you are throwing a ball in a long, narrow tunnel. The ball can travel much farther than it would in open air because the walls of the tunnel keep it from flying off in different directions. Similarly, microwave signals can travel much farther when they are trapped in a tunnel in the atmosphere called a tropospheric duct. This duct acts like a guide, keeping the signals from spreading out and allowing them to travel distances between 100 and 300 miles!

\subsubsection{Advanced Explanation}
Tropospheric ducting occurs when there is a temperature inversion in the troposphere, creating a layer of air with a higher refractive index. This layer acts as a waveguide, trapping microwave signals and allowing them to propagate over long distances with minimal attenuation. The typical range for this phenomenon is between 100 and 300 miles, as the signals are confined within the duct and can travel much farther than they would in free space.

The refractive index \( n \) of the atmosphere is given by:
\[
n = 1 + \frac{77.6}{T} \left( P + \frac{4810 e}{T} \right) \times 10^{-6}
\]
where \( T \) is the temperature in Kelvin, \( P \) is the atmospheric pressure in millibars, and \( e \) is the partial pressure of water vapor in millibars. When a temperature inversion occurs, the refractive index gradient changes, creating the ducting effect.

This phenomenon is particularly useful in long-distance communication, as it allows microwave signals to travel beyond the horizon, which would otherwise be impossible due to the curvature of the Earth.

% [Prompt for diagram: A diagram showing the tropospheric duct with microwave signals propagating within it, illustrating the temperature inversion and the refractive index gradient.]