\subsection{Microwave Magic: Ducts Over Land!}

\begin{tcolorbox}[colback=gray!10!white,colframe=black!75!black,title=\textbf{E3A07}]
\textbf{Question:} Atmospheric ducts capable of propagating microwave signals often form over what geographic feature?
\begin{enumerate}[label=\Alph*),noitemsep]
    \item Mountain ranges
    \item Stratocumulus clouds
    \item \textbf{Large bodies of water}
    \item Nimbus clouds
\end{enumerate}
\end{tcolorbox}

\subsubsection{Intuitive Explanation}
Imagine you're at the beach, and you see the ocean stretching out as far as the eye can see. Now, think about how the air above the water behaves. When the sun heats the water, it also warms the air just above it. This warm air can create a kind of invisible tunnel in the atmosphere, called an atmospheric duct. These ducts can carry microwave signals over long distances, almost like a superhighway for radio waves. So, when you're wondering where these ducts form, think of large bodies of water like oceans or big lakes. They're the perfect places for these magical microwave tunnels to appear!

\subsubsection{Advanced Explanation}
Atmospheric ducts are layers in the atmosphere where the refractive index changes in such a way that it traps microwave signals, allowing them to propagate over long distances with minimal loss. These ducts often form over large bodies of water due to the specific temperature and humidity conditions that prevail there.

When the sun heats the surface of a large body of water, it warms the air directly above it. This warm air can create a temperature inversion, where the air temperature increases with altitude instead of decreasing. This inversion layer can trap microwave signals, effectively creating a duct. The refractive index \( n \) of the air in this layer is modified by the temperature and humidity gradients, which can be described by the following relationship:

\[
n = 1 + \frac{77.6}{T} \left( P + \frac{4810 e}{T} \right) \times 10^{-6}
\]

where \( T \) is the temperature in Kelvin, \( P \) is the atmospheric pressure in millibars, and \( e \) is the partial pressure of water vapor in millibars.

The formation of these ducts is more pronounced over large bodies of water because the water's high heat capacity allows for more stable and prolonged temperature inversions. This stability is crucial for the formation and maintenance of atmospheric ducts, making large bodies of water the most likely geographic feature for their occurrence.

% Diagram Prompt: Generate a diagram showing the formation of an atmospheric duct over a large body of water, illustrating the temperature inversion and the propagation of microwave signals within the duct.