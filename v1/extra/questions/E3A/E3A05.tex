\subsection{Unraveling the Dance of Electromagnetic Waves!}

\begin{tcolorbox}[colback=gray!10!white,colframe=black!75!black,title=Multiple Choice Question]
\textbf{E3A05} How are the component fields of an electromagnetic wave oriented?

\begin{enumerate}[label=\Alph*.]
    \item They are parallel
    \item They are tangential
    \item \textbf{They are at right angles}
    \item They are 90 degrees out of phase
\end{enumerate}
\end{tcolorbox}

\subsubsection*{Intuitive Explanation}
Imagine you are holding a jump rope and shaking it up and down. The rope moves up and down, but the wave travels forward. Now, think of an electromagnetic wave like this jump rope, but with two parts: one part is like the up-and-down movement (the electric field), and the other part is like a side-to-side movement (the magnetic field). These two parts are always at right angles to each other, just like the up-and-down and side-to-side movements of the rope. So, the electric and magnetic fields in an electromagnetic wave are always at right angles to each other.

\subsubsection*{Advanced Explanation}
An electromagnetic wave consists of two oscillating fields: the electric field (\(\mathbf{E}\)) and the magnetic field (\(\mathbf{B}\)). These fields are perpendicular to each other and to the direction of wave propagation. Mathematically, this can be described using Maxwell's equations, which govern the behavior of electromagnetic fields. 

The relationship between the electric and magnetic fields in a plane electromagnetic wave can be expressed as:

\[
\mathbf{E} \times \mathbf{B} = \mathbf{k}
\]

where \(\mathbf{k}\) is the wave vector pointing in the direction of wave propagation. This cross product indicates that \(\mathbf{E}\) and \(\mathbf{B}\) are perpendicular to each other and to \(\mathbf{k}\).

The orientation of these fields is crucial for understanding how electromagnetic waves propagate through space. The electric field oscillates in one plane, while the magnetic field oscillates in a plane perpendicular to it, and both are perpendicular to the direction of the wave's travel.

% Prompt for generating a diagram:
% Diagram showing an electromagnetic wave with the electric field (E) and magnetic field (B) oriented at right angles to each other and to the direction of wave propagation (k).