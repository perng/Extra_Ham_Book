\subsection{Unraveling the Speed of Light: What Influences Electromagnetic Waves?}

\begin{tcolorbox}[colback=gray!10!white,colframe=black!75!black,title=Multiple Choice Question]
    \textbf{E3A10} What determines the speed of electromagnetic waves through a medium?
    \begin{enumerate}[label=\Alph*.]
        \item Resistance and reactance
        \item Evanescence
        \item Birefringence
        \item \textbf{The index of refraction}
    \end{enumerate}
\end{tcolorbox}

\subsubsection{Intuitive Explanation}
Imagine you are running through a field. If the field is flat and clear, you can run very fast. But if the field is filled with tall grass or mud, you will slow down because it’s harder to move through. Similarly, electromagnetic waves, like light, travel at different speeds depending on what they are moving through. The index of refraction is like a measure of how thick or dense the medium is for the wave. A higher index of refraction means the wave slows down more, just like you would slow down in thick mud.

\subsubsection{Advanced Explanation}
The speed of electromagnetic waves in a medium is determined by the medium's index of refraction, denoted as \( n \). The index of refraction is defined as the ratio of the speed of light in a vacuum (\( c \)) to the speed of light in the medium (\( v \)):

\[
n = \frac{c}{v}
\]

For example, if the index of refraction of a medium is 1.5, the speed of light in that medium is:

\[
v = \frac{c}{1.5}
\]

This means the light travels slower in the medium compared to a vacuum. The index of refraction depends on the material's properties, such as its density and how it interacts with electromagnetic fields. Other factors like resistance, reactance, evanescence, and birefringence do not directly determine the speed of electromagnetic waves in a medium.

% Prompt for diagram: A diagram showing light passing through different media (e.g., air, water, glass) with varying speeds, labeled with their respective indices of refraction.