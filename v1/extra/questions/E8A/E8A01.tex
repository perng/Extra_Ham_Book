\subsection{Harmonizing Waves: Unraveling the Square Wave!}

\begin{tcolorbox}
\textbf{Question ID: E8A01} 

What technique shows that a square wave is made up of a sine wave and its odd harmonics?

\begin{enumerate}[label=\Alph*.]
    \item \textbf{Fourier analysis}
    \item Vector analysis
    \item Numerical analysis
    \item Differential analysis
\end{enumerate}
\end{tcolorbox}

\subsubsection{Intuitive Explanation}
Imagine you are listening to music, and you hear a powerful sound that seems a bit rough or sharp. This type of sound is called a square wave, and it has a unique feature: it can be created by combining different simpler sounds, like sine waves. Think of sine waves as smooth, flowing sounds. By mixing a sine wave with certain other sounds that are a bit louder or sharper, you can create that strong, sharp square wave sound. The technique to understand how this combination happens is called Fourier analysis. It's like a magic recipe for sounds!

\subsubsection{Advanced Explanation}
In mathematics and signal processing, a square wave can be expressed as a sum of sine waves of different frequencies and amplitudes. This is known as Fourier series representation. Specifically, a square wave can be expressed in terms of its odd harmonics, which means it contains only the sine waves corresponding to odd integer multiples of the fundamental frequency.

The Fourier series representation of a square wave can be mathematically stated as follows:

\[
f(t) = \frac{4}{\pi} \sum_{n=0}^{\infty} \frac{1}{2n+1} \sin((2n+1) \omega_0 t)
\]

where \( \omega_0 \) is the fundamental angular frequency of the wave. This sum shows that the square wave is made up of sine waves of frequencies \( \omega_0, 3\omega_0, 5\omega_0, \ldots \) plus their corresponding amplitudes that decrease as we go to higher frequencies.

To derive the function for the square wave, consider the following:

1. Identify the fundamental frequency \( \omega_0 \).
2. Recognize that the square wave contains only the sine waves at odd multiples of this frequency.
3. Calculate the contributions from each odd harmonic.

Hence, the correct answer is A: Fourier analysis, as it is the mathematical method to decompose complex waveforms into simple sinusoidal components.

% Prompt for generating a diagram: Create a diagram showing a square wave alongside its corresponding Fourier series representation using odd harmonics.