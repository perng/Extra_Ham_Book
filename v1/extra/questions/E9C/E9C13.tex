\subsection{E9C13: Rising Waves: The Cheerful Dance of Antenna Patterns!}

\begin{tcolorbox}[colback=gray!10!white,colframe=black!75!black,title=Question E9C13]
\textbf{E9C13} How does the radiation pattern of a horizontally polarized antenna vary with increasing height above ground?
\begin{enumerate}[label=\Alph*),noitemsep]
    \item The takeoff angle of the lowest elevation lobe increases
    \item \textbf{The takeoff angle of the lowest elevation lobe decreases}
    \item The horizontal beamwidth increases
    \item The horizontal beamwidth decreases
\end{enumerate}
\end{tcolorbox}

\subsubsection*{Intuitive Explanation}
Imagine you're at a dance party, and the DJ is your antenna. The music (radio waves) is being blasted out in all directions, but the way it reaches the dancers (the ground) changes depending on how high the DJ is. If the DJ is on the ground, the music spreads out more horizontally, and the dancers close to the DJ get the most beats. But if the DJ climbs up on a stage, the music starts to shoot out more downward, and the dancers further away start to feel the rhythm more. Similarly, when a horizontally polarized antenna is raised higher above the ground, the radio waves it sends out tend to point more downward, making the lowest angle of the wave (the takeoff angle) smaller. So, the takeoff angle of the lowest elevation lobe decreases as the antenna goes higher!

\subsubsection*{Advanced Explanation}
The radiation pattern of a horizontally polarized antenna is influenced by the height above the ground due to the interaction between the direct wave and the ground-reflected wave. When the antenna is placed at a height \( h \) above the ground, the phase difference between the direct and reflected waves changes, altering the radiation pattern. The takeoff angle \( \theta \) of the lowest elevation lobe can be approximated using the following relationship:

\[
\theta \approx \arcsin\left(\frac{\lambda}{4h}\right)
\]

where \( \lambda \) is the wavelength of the transmitted signal. As the height \( h \) increases, the argument of the arcsine function decreases, leading to a smaller takeoff angle \( \theta \). This means that the lowest elevation lobe points more downward as the antenna is raised higher above the ground.

Additionally, the horizontal beamwidth is primarily determined by the antenna's physical dimensions and design, and it is less affected by the height above the ground. Therefore, the horizontal beamwidth remains relatively unchanged as the antenna height increases.

In summary, increasing the height of a horizontally polarized antenna above the ground decreases the takeoff angle of the lowest elevation lobe, while the horizontal beamwidth remains largely unaffected.

% [Prompt for diagram: A diagram showing the radiation pattern of a horizontally polarized antenna at different heights above the ground, illustrating the change in the takeoff angle of the lowest elevation lobe.]