\subsection{Let's Tune In: What Makes FM Phone Signals?}

\begin{tcolorbox}[colback=gray!10!white,colframe=black!75!black,title=E7E01] Which of the following can be used to generate FM phone signals?
    \begin{enumerate}[label=\Alph*),noitemsep]
        \item Balanced modulation of the audio amplifier
        \item \textbf{Reactance modulation of a local oscillator}
        \item Reactance modulation of the final amplifier
        \item Balanced modulation of a local oscillator
    \end{enumerate}
\end{tcolorbox}

\subsubsection{Intuitive Explanation}
Imagine you have a radio station, and you want to send your voice over the airwaves using FM (Frequency Modulation). To do this, you need to change the frequency of the radio wave based on your voice. Think of it like a singer changing the pitch of their voice to match a song. The key is to use a special method called reactance modulation on a part of the radio called the local oscillator. This oscillator is like the heart of the radio, and by tweaking it, you can make the frequency change smoothly, just like your voice. This is why option B is the correct choice.

\subsubsection{Advanced Explanation}
Frequency Modulation (FM) involves varying the frequency of a carrier wave in proportion to the amplitude of the input signal (e.g., voice). To generate FM signals, one common method is to use reactance modulation of a local oscillator. The local oscillator generates a stable frequency, and by applying reactance modulation, we can vary this frequency in response to the input signal.

The reactance modulator essentially changes the effective capacitance or inductance in the oscillator circuit, thereby altering its frequency. This is mathematically represented as:

\[
f(t) = f_c + k_f \cdot m(t)
\]

where:
\begin{itemize}
    \item \( f(t) \) is the instantaneous frequency,
    \item \( f_c \) is the carrier frequency,
    \item \( k_f \) is the frequency deviation constant,
    \item \( m(t) \) is the modulating signal.
\end{itemize}

Balanced modulation, on the other hand, is typically used in Amplitude Modulation (AM) and not suitable for FM. Reactance modulation of the final amplifier is also not practical because it would distort the signal. Therefore, the correct method is reactance modulation of a local oscillator, making option B the correct answer.

% Diagram Prompt: Generate a diagram showing the block diagram of an FM transmitter with reactance modulation of a local oscillator.