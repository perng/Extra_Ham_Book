\subsection{Baseband Basics: Unraveling Radio Communications!}

\begin{tcolorbox}[colback=gray!10!white,colframe=black!75!black,title=Multiple Choice Question]
\textbf{E7E07} What is meant by the term “baseband” in radio communications?

\begin{enumerate}[label=\Alph*.]
    \item The lowest frequency band that the transmitter or receiver covers
    \item \textbf{The frequency range occupied by a message signal prior to modulation}
    \item The unmodulated bandwidth of the transmitted signal
    \item The basic oscillator frequency in an FM transmitter that is multiplied to increase the deviation and carrier frequency
\end{enumerate}
\end{tcolorbox}

\subsubsection*{Intuitive Explanation}
Imagine you have a message, like a song or a voice recording, that you want to send over the radio. Before you can send it, you need to prepare it in a way that the radio can handle. This preparation is called modulation. The original message, before it gets modulated, is called the baseband. Think of it like the raw ingredients before you cook a meal. The baseband is the raw message that hasn't been changed yet to be sent over the radio.

\subsubsection*{Advanced Explanation}
In radio communications, the term baseband refers to the original frequency range of a signal before it undergoes modulation. Modulation is the process of altering a carrier wave to encode information for transmission. The baseband signal typically contains the information to be transmitted, such as audio or data, and exists at frequencies much lower than the carrier frequency used for transmission.

For example, consider an audio signal with a frequency range of 20 Hz to 20 kHz. This is the baseband signal. When this signal is modulated onto a carrier wave, say at 1 MHz, the resulting modulated signal will have a bandwidth centered around 1 MHz, but the original baseband signal remains the 20 Hz to 20 kHz range.

Mathematically, if \( m(t) \) represents the baseband signal and \( c(t) = A_c \cos(2\pi f_c t) \) represents the carrier wave, the modulated signal \( s(t) \) can be expressed as:
\[ s(t) = A_c \cos(2\pi f_c t) \cdot m(t) \]
Here, \( m(t) \) is the baseband signal, and \( s(t) \) is the modulated signal.

Understanding baseband is crucial because it represents the original information that needs to be transmitted. The modulation process shifts this baseband signal to a higher frequency range suitable for transmission over the airwaves.

% Diagram Prompt: Generate a diagram showing the baseband signal, the carrier wave, and the resulting modulated signal.