\subsection{Unlocking the Mystery of Vestigial Sideband Modulation!}

\begin{tcolorbox}[colback=gray!10!white,colframe=black!75!black,title=E2B06] What is vestigial sideband modulation?
    \begin{enumerate}[label=\Alph*),noitemsep]
        \item \textbf{Amplitude modulation in which one complete sideband and a portion of the other are transmitted}
        \item A type of modulation in which one sideband is inverted
        \item Narrow-band FM modulation achieved by filtering one sideband from the audio before frequency modulating the carrier
        \item Spread spectrum modulation achieved by applying FM modulation following single sideband amplitude modulation
    \end{enumerate}
\end{tcolorbox}

\subsubsection{Intuitive Explanation}
Imagine you are sending a message using a radio signal. Normally, when you use amplitude modulation (AM), you send two copies of your message—one on each side of the carrier frequency. However, sending both copies takes up a lot of space on the radio spectrum. Vestigial sideband modulation (VSB) is like sending just one full copy of the message and a tiny piece of the other copy. This way, you save space but still keep enough information so the receiver can understand the message clearly.

\subsubsection{Advanced Explanation}
Vestigial sideband modulation (VSB) is a type of amplitude modulation where one complete sideband and a portion of the other sideband are transmitted. This technique is particularly useful in television broadcasting and digital communication systems because it efficiently uses bandwidth while maintaining signal integrity.

Mathematically, in AM, the modulated signal can be represented as:
\[
s(t) = A_c \left[1 + m(t)\right] \cos(2\pi f_c t)
\]
where \( A_c \) is the carrier amplitude, \( m(t) \) is the message signal, and \( f_c \) is the carrier frequency. In VSB, one sideband is fully transmitted, and a vestige (a small portion) of the other sideband is also transmitted. This is achieved by filtering the AM signal in such a way that one sideband is mostly preserved, and a small part of the other sideband is retained.

The filtering process can be represented as:
\[
S_{\text{VSB}}(f) = S_{\text{AM}}(f) \cdot H(f)
\]
where \( S_{\text{AM}}(f) \) is the spectrum of the AM signal, and \( H(f) \) is the transfer function of the filter designed to retain one full sideband and a portion of the other.

VSB modulation is advantageous because it reduces the required bandwidth compared to double sideband (DSB) modulation while still allowing for easy demodulation at the receiver. This makes it a popular choice in applications where bandwidth efficiency is crucial, such as in television broadcasting.

% Diagram prompt: Generate a diagram showing the spectrum of a VSB signal compared to a standard AM signal, highlighting the retained sideband and the vestige of the other sideband.