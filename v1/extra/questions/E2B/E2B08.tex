\subsection{Unlocking Fast-Scan TV: The Magic of Commercial Analog Receivers!}

\begin{tcolorbox}[colback=gray!10!white,colframe=black!75!black,title=E2B08] What technique allows commercial analog TV receivers to be used for fast-scan TV operations on the 70-centimeter band?
    \begin{enumerate}[label=\Alph*),noitemsep]
        \item \textbf{Transmitting on channels shared with cable TV}
        \item Using converted satellite TV dishes
        \item Transmitting on the abandoned TV channel 2
        \item Using USB and demodulating the signal with a computer sound card
    \end{enumerate}
\end{tcolorbox}

\subsubsection{Intuitive Explanation}
Imagine you have an old TV that can only pick up certain channels, like the ones you used to watch cable TV on. Now, if someone wants to send a fast-scan TV signal (like a video) on a different frequency, they can use one of those old cable TV channels. This way, your old TV can still pick up the signal because it’s tuned to the same channel it already knows how to receive. It’s like using a familiar key to unlock a new door!

\subsubsection{Advanced Explanation}
Commercial analog TV receivers are designed to operate on specific frequency bands, such as those used for cable TV. Fast-scan TV (FSTV) operations on the 70-centimeter band (420-450 MHz) can be made compatible with these receivers by transmitting on channels that overlap with the cable TV frequency spectrum. This is because the 70-centimeter band includes frequencies that are also used by certain cable TV channels.

For example, if a fast-scan TV signal is transmitted on a frequency that corresponds to a cable TV channel, the analog TV receiver can demodulate the signal as if it were a standard TV broadcast. This eliminates the need for specialized equipment, as the receiver is already tuned to the appropriate frequency range.

Mathematically, the frequency of the transmitted signal \( f_{tx} \) must satisfy:
\[
f_{tx} \in [f_{cable\_min}, f_{cable\_max}]
\]
where \( f_{cable\_min} \) and \( f_{cable\_max} \) are the minimum and maximum frequencies of the cable TV channels that the receiver can decode.

This technique leverages the existing infrastructure of analog TV receivers, making it a cost-effective solution for fast-scan TV operations. It also avoids the need for additional hardware modifications, as the receiver’s tuning circuitry is already optimized for the relevant frequency range.

% Diagram prompt: Generate a diagram showing the frequency spectrum of the 70-centimeter band overlapping with cable TV channels, highlighting the shared frequencies used for fast-scan TV operations.