\subsection{Unlocking the Bands: What Are L and S?}

\begin{tcolorbox}[colback=blue!5!white,colframe=blue!75!black]
    \textbf{E2A09} What do the terms “L band” and “S band” specify?
    \begin{enumerate}[label=\Alph*.]
        \item \textbf{The 23- and 13-centimeter bands}
        \item The 2-meter and 70-centimeter bands
        \item FM and digital store-and-forward systems
        \item Which sideband to use
    \end{enumerate}
\end{tcolorbox}

\subsubsection{Intuitive Explanation}
Think of radio waves like different types of light. Just like how you have different colors of light, radio waves come in different bands or groups. The L band and S band are just names for specific groups of radio waves. The L band includes waves that are about 23 centimeters long, and the S band includes waves that are about 13 centimeters long. These bands are used for different types of communication, like satellite TV or weather radar.

\subsubsection{Advanced Explanation}
The terms L band and S band refer to specific frequency ranges within the electromagnetic spectrum. The L band typically covers frequencies from 1 to 2 GHz, corresponding to wavelengths around 30 to 15 centimeters. The S band covers frequencies from 2 to 4 GHz, corresponding to wavelengths around 15 to 7.5 centimeters. These bands are crucial for various applications, including satellite communications, radar systems, and wireless networking.

To calculate the wavelength (\(\lambda\)) from the frequency (\(f\)), we use the formula:
\[
\lambda = \frac{c}{f}
\]
where \(c\) is the speed of light (\(3 \times 10^8\) meters per second). For example, for a frequency of 1.5 GHz (L band):
\[
\lambda = \frac{3 \times 10^8}{1.5 \times 10^9} = 0.2 \text{ meters} = 20 \text{ centimeters}
\]

Understanding these bands is essential for designing and operating communication systems that utilize these specific frequency ranges.

% [Prompt for diagram: A diagram showing the electromagnetic spectrum with L band and S band highlighted, along with their corresponding frequency ranges and wavelengths.]