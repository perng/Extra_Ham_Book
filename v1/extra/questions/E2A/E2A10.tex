\subsection{Spotting a Sky-Stationary Satellite!}

\begin{tcolorbox}[colback=gray!10!white,colframe=black!75!black,title=E2A10] What type of satellite appears to stay in one position in the sky?
    \begin{enumerate}[label=\Alph*),noitemsep]
        \item HEO
        \item \textbf{Geostationary}
        \item Geomagnetic
        \item LEO
    \end{enumerate}
\end{tcolorbox}

\subsubsection*{Intuitive Explanation}
Imagine you are looking up at the sky and you see a star that never moves. It stays in the exact same spot all night, every night. A geostationary satellite is like that star! It orbits the Earth at the same speed that the Earth rotates, so it always stays above the same place on the ground. This is super useful for things like TV broadcasts and weather monitoring because the satellite can always see the same area.

\subsubsection*{Advanced Explanation}
A geostationary satellite orbits the Earth at an altitude of approximately 35,786 kilometers (22,236 miles) above the equator. This specific altitude is chosen because it allows the satellite to match the Earth's rotational period of 24 hours. The orbital velocity \( v \) of a satellite can be calculated using the formula:

\[
v = \sqrt{\frac{G M}{r}}
\]

where:
\begin{itemize}
    \item \( G \) is the gravitational constant,
    \item \( M \) is the mass of the Earth,
    \item \( r \) is the distance from the center of the Earth to the satellite.
\end{itemize}

For a geostationary orbit, \( r \) is the sum of the Earth's radius and the satellite's altitude. The satellite's orbital period \( T \) is given by:

\[
T = 2\pi \sqrt{\frac{r^3}{G M}}
\]

When \( T \) equals 24 hours, the satellite is geostationary. This synchronization ensures that the satellite remains fixed relative to a point on the Earth's surface, making it ideal for continuous communication and observation.

\subsubsection*{Related Concepts}
\begin{itemize}
    \item \textbf{Orbital Mechanics}: Understanding how objects move in space under the influence of gravity.
    \item \textbf{Kepler's Laws}: Describing the motion of planets and satellites.
    \item \textbf{Communication Satellites}: How geostationary satellites are used for global communication.
\end{itemize}

% Diagram Prompt: Generate a diagram showing the Earth with a geostationary satellite orbiting above the equator, illustrating how it remains fixed relative to a point on the Earth's surface.