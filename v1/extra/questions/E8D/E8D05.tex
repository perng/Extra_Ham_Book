\subsection{Quieting the Clicks: Top Tips to Reduce Key Noise!}

\begin{tcolorbox}
    \textbf{Question ID: E8D05} \\
    What is the most common method of reducing key clicks? \\
    \begin{enumerate}[label=\Alph*.]
        \item \textbf{Increase keying waveform rise and fall times}
        \item Insert low-pass filters at the transmitter output
        \item Reduce keying waveform rise and fall times
        \item Insert high-pass filters at the transmitter output
    \end{enumerate}
\end{tcolorbox}

\subsubsection{Intuitive Explanation}
Imagine you are typing on a keyboard, and every time you press a key, it makes a loud click sound. This can be distracting, especially when you are trying to focus or work in a quiet place. 

To make the clicks quieter, one of the ways is to change how quickly the sound appears when you press the key. If the sound takes a bit longer to get to its full volume (which is what we mean by increasing the rise and fall times), it won’t be as startling and will sound softer. Think of it like a car accelerating slowly instead of quickly; the noise it makes gets to its maximum in a smoother way, leading to a quieter experience.

\subsubsection{Advanced Explanation}
In the context of key clicks generated by a transmitter, key clicks refer to the unwanted noise that occurs when a signal is turned on and off rapidly. This noise is primarily influenced by the characteristics of the keying waveform, which is the signal shape that represents the push and release of a key.

Increasing the rise and fall times of the keying waveform means that the transition from off to on and from on to off occurs more gradually. The rise time is the duration it takes for the waveform to change from a low voltage (0) to a high voltage (1), while the fall time is the time it takes to go back from high to low. By increasing these times, we smooth out the transitions, which helps reduce the sharpness of the clicks.

Let's define rise time \( t_r \) and fall time \( t_f \) mathematically. If the original rise time is \( t_{r0} \) and the fall time is \( t_{f0} \), the new rise time can be expressed as:
\[
t_{r} = k \cdot t_{r0}
\]
\[
t_{f} = k \cdot t_{f0}
\]
where \( k > 1 \) is some constant that describes how much we are increasing the times. As a practical application, if the original rise time was \( 10 \, \text{ms} \) and we want to double that, then \( t_{r} = 2 \cdot 10 = 20 \, \text{ms} \).

Thus, when we increase these parameters, the transitions of the keying waveform become more gradual, reducing the harshness of key clicks. The changes can be verified through signal analysis in frequency domain techniques, where lower-frequency components are emphasized while higher-frequency components are minimized, leading to a more pleasant sound profile.

% Diagram prompt: A waveform graph showing the difference between a sharp rise/fall time and a gradual rise/fall time in a keying waveform.