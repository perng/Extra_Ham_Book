\subsection{Unraveling AFSK Distortion: The Key Parameter!}

\begin{tcolorbox}
    \textbf{Question ID: E8D08}\\
    What parameter evaluates distortion of an AFSK signal caused by excessive input audio levels?\\
    \begin{enumerate}[label=\Alph*.]
        \item Signal-to-noise ratio
        \item Baud error rate
        \item Repeat Request Rate (RRR)
        \item \textbf{Intermodulation Distortion (IMD)}
    \end{enumerate}
\end{tcolorbox}

\subsubsection{Intuitive Explanation}
Imagine you are listening to your favorite song, but the volume is turned up too high. As a result, some of the words get mixed up and sound fuzzy – that's a bit like distortion! In our question, we're talking about a special kind of signal called AFSK, which is used in radios to send information. When the sound is too loud, it can make the signal all jumbled. The parameter that helps us understand how messed up the signal can get because of the volume is called Intermodulation Distortion or IMD for short. It’s like a warning sign that tells us when the music is too loud and it’s hurting the sound quality.

\subsubsection{Advanced Explanation}
In communications theory, particularly in frequency modulation schemes such as AFSK (Audio Frequency Shift Keying), distortion can significantly affect the integrity of the transmitted signal. Excessive input audio levels can lead to a phenomenon known as intermodulation distortion (IMD). IMD occurs when two or more signals interact and create additional unwanted signals at frequencies that are the sum and difference of the original frequencies. 

To quantify this effect, we can analyze the signal and measure the distortion as follows:

1. Define the input signal: Let \( x(t) \) be the input audio signal fed into the AFSK modulator.
2. Identify the modulation frequencies: Assume \( f_1 \) and \( f_2 \) are the two frequencies used for the AFSK symbols.
3. Observe the output: The output signal can be expressed in terms of its components, including \( x(t) \) and the generated sidebands.

The distortion can be calculated using the formula for IMD:

\[
IMD = \frac{P_{IMD}}{P_{Signal}}
\]

where \( P_{IMD} \) is the power of the intermodulation products and \( P_{Signal} \) is the power of the desired signal. The higher the IMD value, the greater the distortion in the signal.

Understanding IMD is crucial for ensuring good signal quality and maintaining clarity in data communication systems. It reveals how excessive audio input can lead to signal degradation, impacting data integrity.

% Diagram prompt: Illustration of AFSK signal with input audio levels and its relationship with intermodulation distortion.