\subsection{Baudot vs. ASCII: The Joyful Journey of Digital Codes!}

\begin{tcolorbox}
    \textbf{Question ID: E8D10} \\
    What are some of the differences between the Baudot digital code and ASCII? \\
    \begin{enumerate}[label=\Alph*.]
        \item Baudot uses 4 data bits per character, ASCII uses 7 or 8; Baudot uses 1 character as a letters/figures shift code, ASCII has no letters/figures code.
        \item \textbf{Baudot uses 5 data bits per character, ASCII uses 7 or 8; Baudot uses 2 characters as letters/figures shift codes, ASCII has no letters/figures shift code.}
        \item Baudot uses 6 data bits per character, ASCII uses 7 or 8; Baudot has no letters/figures shift code, ASCII uses 2 letters/figures shift codes.
        \item Baudot uses 7 data bits per character, ASCII uses 8; Baudot has no letters/figures shift code, ASCII uses 2 letters/figures shift codes.
    \end{enumerate}
\end{tcolorbox}

\subsubsection{Intuitive Explanation}
Imagine you have two different secret codes that you can use to send messages: one is the Baudot code, and the other is ASCII. Think of them as different ways to represent letters and numbers with a limited number of switches or lights. Baudot code can be thought of as having fewer switches, but it cleverly uses some special tricks to switch between letters and numbers. ASCII, on the other hand, has more switches, allowing it to represent more things at once without needing to switch. In this way, you can think of Baudot and ASCII as two different languages for your electronic devices, with Baudot being a simpler one and ASCII being a bit more complex.

\subsubsection{Advanced Explanation}
The Baudot code is a type of encoded signal used historically in telecommunications. It typically uses 5 data bits per character, which allows it to create a limited number of combinations (up to 32 unique characters). To differentiate between letters and figures (numbers), Baudot employs two shift characters, which alter the meaning of subsequent characters sent— this is effectively a way of switching between two different sets of characters.

ASCII, or the American Standard Code for Information Interchange, utilizes 7 or 8 data bits per character. This allows for the representation of up to 128 distinct characters (or 256 when using the 8-bit extended version). In the ASCII standard, there are no specific shift codes; each character directly represents a letter, number, or symbol. 

To illustrate the coding differences mathematically:
- Baudot provides a character set of size \( 2^5 = 32 \).
- ASCII offers a character set of size \( 2^7 = 128 \) (or \( 2^8 = 256 \) for the extended version).

Thus, while both coding systems are designed to facilitate communication via encoding characters, they differ fundamentally in their structure, complexity, and approach to representing letters and figures.

% Diagram Prompt: Show a comparison chart between Baudot and ASCII character encoding systems, indicating data bits, number of characters supported, and usage of shift codes.