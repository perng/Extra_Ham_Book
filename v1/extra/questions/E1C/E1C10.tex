\subsection{Maximizing Power: Spurious Emissions Under 30 MHz!}

\begin{tcolorbox}[colback=gray!10!white,colframe=black!75!black,title=E1C10]
\textbf{E1C10} What is the maximum mean power level for a spurious emission below 30 MHz with respect to the fundamental emission?
\begin{enumerate}[label=\Alph*),noitemsep]
    \item \textbf{- 43 dB}
    \item - 53 dB
    \item - 63 dB
    \item - 73 dB
\end{enumerate}
\end{tcolorbox}

\subsubsection{Intuitive Explanation}
Imagine you are listening to your favorite radio station. The station broadcasts at a specific frequency, which is called the fundamental emission. However, sometimes the radio equipment can accidentally produce extra signals at other frequencies, known as spurious emissions. These extra signals can interfere with other radio stations or devices. To prevent this, there are rules that limit how strong these extra signals can be. For frequencies below 30 MHz, the rule says that the spurious emissions must be at least 43 dB weaker than the main signal. This ensures that the extra signals are not strong enough to cause interference.

\subsubsection{Advanced Explanation}
In radio communications, spurious emissions are unwanted signals that occur at frequencies other than the intended fundamental frequency. These emissions can be caused by imperfections in the transmitter or other electronic components. To minimize interference, regulatory bodies such as the FCC (Federal Communications Commission) set limits on the power levels of these spurious emissions.

For frequencies below 30 MHz, the maximum mean power level for a spurious emission is specified relative to the fundamental emission. The correct answer is \textbf{-43 dB}, meaning that the spurious emission must be at least 43 decibels below the power level of the fundamental emission.

Mathematically, if \( P_{\text{fundamental}} \) is the power of the fundamental emission, then the power of the spurious emission \( P_{\text{spurious}} \) must satisfy:
\[
P_{\text{spurious}} \leq P_{\text{fundamental}} - 43 \text{ dB}
\]

This ensures that the spurious emissions are sufficiently attenuated to prevent interference with other communications systems.

% Prompt for generating a diagram: A diagram showing the power levels of the fundamental emission and the spurious emission, with the spurious emission being 43 dB below the fundamental emission.