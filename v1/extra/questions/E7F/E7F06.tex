\subsection{Bits and Volts: Finding the Perfect Match!}
\label{sec:E7F06}

\begin{tcolorbox}[colback=blue!5!white,colframe=blue!75!black,title=E7F06]
\textbf{E7F06} What is the minimum number of bits required to sample a signal with a range of 1 volt at a resolution of 1 millivolt?
\begin{enumerate}[label=\Alph*),noitemsep]
    \item 4 bits
    \item 6 bits
    \item 8 bits
    \item \textbf{10 bits}
\end{enumerate}
\end{tcolorbox}

\subsubsection{Intuitive Explanation}
Imagine you have a ruler that measures from 0 to 1 volt, and you want to measure every 1 millivolt (which is 0.001 volts). To do this, you need to divide the ruler into very small parts. The more parts you have, the more precise your measurements will be. In this case, you need to divide 1 volt into 1000 parts (since 1 volt / 0.001 volts = 1000). Now, think of each part as a step that you can count. The number of bits you need is like the number of switches you need to turn on to count all these steps. The more bits you have, the more steps you can count. Here, you need 10 bits because \(2^{10} = 1024\), which is enough to count all 1000 steps.

\subsubsection{Advanced Explanation}
To determine the minimum number of bits required to sample a signal with a range of 1 volt at a resolution of 1 millivolt, we need to calculate the number of distinct levels that can be represented by the bits. The resolution is given as 1 millivolt (0.001 volts), and the range is 1 volt. The number of distinct levels \(N\) is given by:

\[
N = \frac{\text{Range}}{\text{Resolution}} = \frac{1 \text{ volt}}{0.001 \text{ volt}} = 1000
\]

The number of bits \(n\) required to represent \(N\) distinct levels is given by:

\[
2^n \geq N
\]

Substituting \(N = 1000\):

\[
2^n \geq 1000
\]

To find the minimum \(n\), we solve for \(n\):

\[
n \geq \log_2{1000} \approx 9.9658
\]

Since the number of bits must be an integer, we round up to the next whole number:

\[
n = 10
\]

Therefore, the minimum number of bits required is 10.

\subsubsection{Related Concepts}
\begin{itemize}
    \item \textbf{Resolution}: The smallest change in the input signal that can be detected by the system.
    \item \textbf{Bit Depth}: The number of bits used to represent each sample in a digital signal.
    \item \textbf{Quantization}: The process of mapping a large set of input values to a smaller set, such as rounding to a fixed number of bits.
\end{itemize}

% Prompt for generating a diagram:
% Diagram showing the relationship between the range of the signal, the resolution, and the number of bits required to represent the signal.