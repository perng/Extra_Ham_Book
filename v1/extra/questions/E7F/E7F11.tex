\subsection{Unlocking the Secrets of Signal Detection in Direct-Sampling Receivers!}

\begin{tcolorbox}[colback=gray!10!white,colframe=black!75!black,title=\textbf{E7F11}]
What sets the minimum detectable signal level for a direct-sampling software defined receiver in the absence of atmospheric or thermal noise?
\begin{enumerate}[label=\Alph*.]
    \item Sample clock phase noise
    \item \textbf{Reference voltage level and sample width in bits}
    \item Data storage transfer rate
    \item Missing codes and jitter
\end{enumerate}
\end{tcolorbox}

\subsubsection*{Intuitive Explanation}
Imagine you are trying to listen to a very quiet sound in a completely silent room. The quietest sound you can hear depends on how sensitive your ears are and how finely you can distinguish different volumes. In a direct-sampling software defined receiver, the ears are the electronic components that detect signals. The minimum detectable signal level is determined by two main things: the reference voltage level (which is like the baseline volume) and the sample width in bits (which is like how finely you can distinguish different volumes). If the reference voltage is too high or the sample width is too coarse, you might miss very quiet signals.

\subsubsection*{Advanced Explanation}
In a direct-sampling software defined receiver, the minimum detectable signal level is primarily influenced by the reference voltage level and the sample width in bits. The reference voltage level sets the maximum amplitude that can be accurately sampled, while the sample width in bits determines the resolution of the analog-to-digital converter (ADC). The resolution is given by the formula:

\[
\text{Resolution} = \frac{V_{\text{ref}}}{2^n}
\]

where \( V_{\text{ref}} \) is the reference voltage and \( n \) is the number of bits in the sample width. A higher reference voltage or a larger number of bits increases the resolution, allowing the receiver to detect smaller signals. 

Other factors like sample clock phase noise, data storage transfer rate, and missing codes and jitter can affect the performance of the receiver, but they do not directly set the minimum detectable signal level. These factors are more related to the accuracy and speed of the sampling process rather than the fundamental ability to detect weak signals.

% Prompt for diagram: A diagram showing the relationship between reference voltage, sample width, and the minimum detectable signal level in a direct-sampling receiver would be helpful here.