\subsection{Exploring Frequency: The Gain Secrets of Ideal Op-Amps!}

\begin{tcolorbox}[colback=gray!10!white,colframe=black!75!black,title=E7G08] How does the gain of an ideal operational amplifier vary with frequency?
    \begin{enumerate}[label=\Alph*),noitemsep]
        \item It increases linearly with increasing frequency
        \item It decreases linearly with increasing frequency
        \item It decreases logarithmically with increasing frequency
        \item \textbf{It does not vary with frequency}
    \end{enumerate}
\end{tcolorbox}

\subsubsection{Intuitive Explanation}
Imagine you have a magical volume knob that controls the loudness of your music. An ideal operational amplifier (op-amp) is like this magical knob, but for electrical signals. Now, you might think that if you change the speed of the music (which is like changing the frequency), the volume would change too. But with an ideal op-amp, no matter how fast or slow the music plays, the volume stays the same. This means the gain (which is like the volume) of an ideal op-amp doesn’t change with frequency. It’s always constant!

\subsubsection{Advanced Explanation}
An ideal operational amplifier is characterized by infinite gain, infinite input impedance, zero output impedance, and infinite bandwidth. The gain of an ideal op-amp is defined as the ratio of the output voltage to the input voltage, and it is typically represented by the symbol \( A \). In an ideal scenario, the gain \( A \) is constant and does not depend on the frequency of the input signal. This can be mathematically expressed as:

\[
A = \frac{V_{\text{out}}}{V_{\text{in}}}
\]

where \( V_{\text{out}} \) is the output voltage and \( V_{\text{in}} \) is the input voltage. Since the gain is infinite and constant, it does not vary with frequency. This is a key characteristic of an ideal op-amp, distinguishing it from real-world op-amps where the gain may decrease at higher frequencies due to limitations in the device's bandwidth.

In practical terms, the ideal op-amp model assumes that the device can amplify signals of any frequency without any loss in gain. This is a simplification used in theoretical analysis and circuit design, allowing engineers to focus on other aspects of the circuit without worrying about frequency-dependent gain variations.

% Prompt for generating a diagram: A graph showing the gain (A) on the y-axis and frequency (f) on the x-axis, with a horizontal line representing the constant gain of an ideal op-amp.