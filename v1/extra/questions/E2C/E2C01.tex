\subsection{Remote Control Joy: What Indicator Do US Operators Need?}

\begin{tcolorbox}[colback=blue!5!white,colframe=blue!75!black]
    \textbf{E2C01} What indicator is required to be used by US-licensed operators when operating a station via remote control and the remote transmitter is located in the US?
    \begin{enumerate}[label=\Alph*),noitemsep]
        \item / followed by the USPS two-letter abbreviation for the state in which the remote station is located
        \item /R\# where \# is the district of the remote station
        \item / followed by the ARRL Section of the remote station
        \item \textbf{No additional indicator is required}
    \end{enumerate}
\end{tcolorbox}

\subsubsection{Intuitive Explanation}
Imagine you have a remote-controlled car, and you're driving it from your house. You don't need to put a special sticker on the car to show where you're controlling it from, right? Similarly, if you're operating a radio station from a remote location within the United States, you don't need to add any extra signs or indicators to show where the remote transmitter is. It's just like driving your remote-controlled car from home—no extra labels needed!

\subsubsection{Advanced Explanation}
In the context of amateur radio operations in the United States, the Federal Communications Commission (FCC) regulates the use of remote control stations. According to FCC rules, when a licensed operator controls a station remotely and the remote transmitter is located within the US, no additional indicator is required. This means that the operator does not need to append any special suffix or identifier to their call sign to denote the remote operation.

This rule simplifies the process for operators, as it eliminates the need for additional administrative steps when operating remotely within the country. It is important to note that this rule applies specifically to remote control operations within the US. If the remote transmitter were located outside the US, different regulations might apply, and additional indicators could be required.

In summary, the correct answer is that no additional indicator is required when operating a station via remote control within the US. This is in line with FCC regulations, which aim to streamline the process for amateur radio operators while maintaining compliance with licensing requirements.

% [Prompt for generating a diagram: A simple illustration showing a person operating a radio station remotely from their home, with the remote transmitter located within the US, and no additional indicators needed.]