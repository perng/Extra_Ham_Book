\subsection{E9D02: Turning Waves: Crafting Circular Polarization with Yagi Antennas!}

\begin{tcolorbox}[colback=blue!5!white,colframe=blue!75!black]
    \textbf{E9D02} How can two linearly polarized Yagi antennas be used to produce circular polarization?
    \begin{enumerate}[label=\Alph*),noitemsep]
        \item Stack two Yagis to form an array with the respective elements in parallel planes fed 90 degrees out of phase
        \item Stack two Yagis to form an array with the respective elements in parallel planes fed in phase
        \item \textbf{Arrange two Yagis on the same axis and perpendicular to each other with the driven elements at the same point on the boom and fed 90 degrees out of phase}
        \item Arrange two Yagis collinear to each other with the driven elements fed 180 degrees out of phase
    \end{enumerate}
\end{tcolorbox}

\subsubsection{Intuitive Explanation}
Imagine you have two Yagi antennas, like the ones you see on rooftops for TV signals. Now, think of these antennas as two dancers. If both dancers move in the same direction at the same time, their movements are in sync, and you get a straight line dance. But if one dancer moves up while the other moves to the side, and they do this with a slight delay (like a quarter of a beat), their combined movements create a circle! This is exactly what happens when you arrange two Yagi antennas perpendicular to each other and feed them 90 degrees out of phase. The result is a beautiful circular dance of radio waves, known as circular polarization.

\subsubsection{Advanced Explanation}
To produce circular polarization using two linearly polarized Yagi antennas, the antennas must be arranged such that their electric fields are perpendicular to each other and fed with a 90-degree phase difference. This setup ensures that the combined electric field vector rotates in a circular pattern over time.

Mathematically, if the electric fields of the two antennas are represented as:
\[
E_1(t) = E_0 \cos(\omega t)
\]
\[
E_2(t) = E_0 \cos\left(\omega t + \frac{\pi}{2}\right)
\]
where \(E_0\) is the amplitude, \(\omega\) is the angular frequency, and \(t\) is time. The combined electric field \(E(t)\) is:
\[
E(t) = E_1(t) \hat{x} + E_2(t) \hat{y}
\]
\[
E(t) = E_0 \cos(\omega t) \hat{x} + E_0 \cos\left(\omega t + \frac{\pi}{2}\right) \hat{y}
\]
Using the trigonometric identity \(\cos\left(\omega t + \frac{\pi}{2}\right) = -\sin(\omega t)\), we get:
\[
E(t) = E_0 \cos(\omega t) \hat{x} - E_0 \sin(\omega t) \hat{y}
\]
This represents a vector that rotates in a circular pattern with angular frequency \(\omega\), thus producing circular polarization.

The key concepts involved here are:
\begin{itemize}
    \item \textbf{Linear Polarization}: The electric field oscillates in a single plane.
    \item \textbf{Circular Polarization}: The electric field vector rotates in a circular pattern.
    \item \textbf{Phase Difference}: A 90-degree phase shift between the two antennas ensures the rotation of the combined electric field vector.
\end{itemize}

% Diagram prompt: Generate a diagram showing two Yagi antennas arranged perpendicular to each other on the same axis, with the driven elements at the same point on the boom, and indicate the 90-degree phase difference in the feeding lines.