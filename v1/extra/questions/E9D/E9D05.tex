\subsection{E9D05: Yagi Antenna Fun: How Long is the Driven Element?}

\begin{tcolorbox}[colback=gray!10!white,colframe=black!75!black]
    \textbf{E9D05} Approximately how long is a Yagi’s driven element?
    \begin{enumerate}[label=\Alph*),noitemsep]
        \item 234 divided by frequency in MHz
        \item 1005 divided by frequency in MHz
        \item 1/4 wavelength
        \item \textbf{1/2 wavelength}
    \end{enumerate}
\end{tcolorbox}

\subsubsection{Intuitive Explanation}
Alright, imagine you’re building a Yagi antenna, which is like a fancy TV antenna with a bunch of sticks. The driven element is the main stick that actually does the talking and listening. Now, how long should this stick be? Well, it’s not just any random length—it’s specifically designed to match the radio waves it’s dealing with. Think of it like tuning a guitar string to the right pitch. For the Yagi’s driven element, the magic length is half the wavelength of the radio wave. So, if the radio wave is like a big wave in the ocean, the driven element is half the size of that wave. Easy peasy!

\subsubsection{Advanced Explanation}
The driven element of a Yagi antenna is a critical component that determines the antenna’s resonant frequency. The length of the driven element is directly related to the wavelength (\(\lambda\)) of the operating frequency. The wavelength can be calculated using the formula:

\[
\lambda = \frac{c}{f}
\]

where \(c\) is the speed of light (\(3 \times 10^8\) meters per second) and \(f\) is the frequency in Hertz. For practical purposes, the length of the driven element is typically half the wavelength (\(\frac{\lambda}{2}\)). This is because a half-wavelength dipole is the most efficient and commonly used configuration for resonant antennas.

Given the choices:
\begin{itemize}
    \item \textbf{Option A}: \( \frac{234}{f_{\text{MHz}}} \) — This formula is often used to calculate the length of a quarter-wave vertical antenna, not the driven element of a Yagi.
    \item \textbf{Option B}: \( \frac{1005}{f_{\text{MHz}}} \) — This formula does not correspond to any standard antenna length calculation.
    \item \textbf{Option C}: \( \frac{\lambda}{4} \) — This is the length of a quarter-wave element, not the driven element of a Yagi.
    \item \textbf{Option D}: \( \frac{\lambda}{2} \) — This is the correct length for the driven element of a Yagi antenna.
\end{itemize}

Thus, the correct answer is \textbf{D}.

% Prompt for diagram: A diagram showing a Yagi antenna with labeled elements, highlighting the driven element and its length relative to the wavelength.