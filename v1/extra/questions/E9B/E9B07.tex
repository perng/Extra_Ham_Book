\subsection{E9B07: Power Play: Antenna Gains Unleashed!}

\begin{tcolorbox}[colback=gray!10!white,colframe=black!75!black,title=Multiple Choice Question]
\textbf{E9B07} What is the difference in radiated power between a lossless antenna with gain and an isotropic radiator driven by the same power?

\begin{enumerate}[label=\Alph*),noitemsep]
    \item The power radiated from the directional antenna is increased by the gain of the antenna
    \item The power radiated from the directional antenna is stronger by its front-to-back ratio
    \item \textbf{They are the same}
    \item The power radiated from the isotropic radiator is 2.15 dB greater than that from the directional antenna
\end{enumerate}
\end{tcolorbox}

\subsubsection*{Intuitive Explanation}
Imagine you have two flashlights: one is a regular flashlight that shines light in all directions (isotropic radiator), and the other is a super flashlight that focuses its light in one direction (directional antenna). If both flashlights use the same amount of battery power, the total amount of light they produce is the same. The super flashlight just makes the light brighter in one direction, but it doesn't create more light overall. So, the total radiated power is the same for both!

\subsubsection*{Advanced Explanation}
An isotropic radiator is a theoretical antenna that radiates power uniformly in all directions. A directional antenna, on the other hand, focuses its radiation in specific directions, which is quantified by its gain. However, the gain of an antenna does not imply that it radiates more power; it simply means that the power is concentrated in certain directions.

Mathematically, the total radiated power \( P_{\text{rad}} \) for both antennas is the same when driven by the same input power \( P_{\text{in}} \). The gain \( G \) of the directional antenna affects the power density in a particular direction but does not change the total radiated power. Therefore, the correct answer is that the radiated power is the same for both antennas.

\[ P_{\text{rad, isotropic}} = P_{\text{rad, directional}} \]

This concept is crucial in understanding antenna theory, as it highlights the difference between power concentration and total power radiated.

% [Prompt for diagram: A diagram showing an isotropic radiator radiating power uniformly in all directions, and a directional antenna focusing power in a specific direction, with both having the same total radiated power.]