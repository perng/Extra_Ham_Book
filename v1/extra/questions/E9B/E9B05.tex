\subsection{Exciting Antenna Patterns: What's in Figure E9-2?}

\begin{tcolorbox}[colback=gray!10!white,colframe=black!75!black]
    \textbf{E9B05} What type of antenna pattern is shown in Figure E9-2?
    \begin{enumerate}[label=\Alph*),noitemsep]
        \item \textbf{Elevation}
        \item Azimuth
        \item Near field
        \item Polarization
    \end{enumerate}
\end{tcolorbox}

\subsubsection{Intuitive Explanation}
Imagine you’re standing on a hill with a flashlight. If you point the flashlight up and down, you’re changing the elevation of the beam. Now, think of the antenna in Figure E9-2 as that flashlight. The pattern it’s showing is like the beam going up and down, not side to side. So, the correct answer is \textbf{Elevation}! It’s all about the up and down movement, not the left and right.

\subsubsection{Advanced Explanation}
Antenna patterns are graphical representations of the radiation properties of an antenna as a function of space coordinates. The elevation pattern specifically describes the radiation intensity of the antenna in the vertical plane. This is crucial for understanding how the antenna will perform in terms of coverage and signal strength at different heights.

In Figure E9-2, the pattern is depicted in a way that shows the variation of the antenna's radiation in the vertical plane. This is distinct from the azimuth pattern, which would show the radiation in the horizontal plane. The near field and polarization are different concepts altogether; the near field refers to the region close to the antenna where the electromagnetic field is not fully formed, and polarization refers to the orientation of the electric field of the electromagnetic wave.

To determine the correct answer, one must recognize that the pattern in Figure E9-2 is showing the vertical variation, hence the correct answer is \textbf{Elevation}.

% [Prompt for generating a diagram: Create a diagram showing an antenna with its radiation pattern in the vertical plane, labeled as Elevation Pattern.]