\subsection{FET vs. Bipolar: A Cheerful Impedance Showdown!}

\begin{tcolorbox}[colback=gray!10!white,colframe=black!75!black,title=E6A05] How does DC input impedance at the gate of a field-effect transistor (FET) compare with that of a bipolar transistor?
    \begin{enumerate}[label=\Alph*),noitemsep]
        \item They are both low impedance
        \item An FET has lower input impedance
        \item \textbf{An FET has higher input impedance}
        \item They are both high impedance
    \end{enumerate}
\end{tcolorbox}

\subsubsection{Intuitive Explanation}
Imagine you have two doors: one is a heavy, solid door (like the gate of an FET), and the other is a light, easy-to-open door (like the base of a bipolar transistor). The heavy door is harder to push open, which means it has a higher resistance to being opened. In electronics, this resistance is called impedance. The gate of an FET is like the heavy door—it has a higher impedance compared to the base of a bipolar transistor, which is like the light door. So, an FET has a higher input impedance than a bipolar transistor.

\subsubsection{Advanced Explanation}
The DC input impedance at the gate of a Field-Effect Transistor (FET) is significantly higher than that of a Bipolar Junction Transistor (BJT). This is due to the fundamental differences in their operation principles.

In an FET, the gate is insulated from the channel by a thin layer of oxide (in MOSFETs) or a reverse-biased PN junction (in JFETs). This insulation results in a very high input impedance, typically in the order of \(10^9\) to \(10^{12}\) ohms. The gate current is negligible, making the FET an excellent choice for applications requiring high input impedance.

On the other hand, a BJT operates based on the injection of minority carriers across a forward-biased PN junction (the base-emitter junction). This requires a small but significant base current, resulting in a much lower input impedance, typically in the range of \(10^3\) to \(10^5\) ohms.

Mathematically, the input impedance \(Z_{in}\) of a transistor can be expressed as:
\[
Z_{in} = \frac{V_{in}}{I_{in}}
\]
For an FET, \(I_{in}\) is extremely small, leading to a high \(Z_{in}\). For a BJT, \(I_{in}\) is larger, resulting in a lower \(Z_{in}\).

Therefore, the correct answer is that an FET has a higher input impedance compared to a bipolar transistor.

% [Diagram Prompt: A comparison diagram showing the input impedance of FET and BJT, with labels indicating the gate/base and the corresponding impedance values.]