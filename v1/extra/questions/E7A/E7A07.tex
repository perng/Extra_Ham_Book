\subsection{NAND Gate Magic: Unveiling Logical Operations!}

\begin{tcolorbox}[colback=gray!10!white,colframe=black!75!black,title=E7A07] What logical operation does a NAND gate perform?
    \begin{enumerate}[label=\Alph*),noitemsep]
        \item It produces a 0 at its output only if all inputs are 0
        \item It produces a 1 at its output only if all inputs are 1
        \item It produces a 0 at its output if some but not all inputs are 1
        \item \textbf{It produces a 0 at its output only if all inputs are 1}
    \end{enumerate}
\end{tcolorbox}

\subsubsection{Intuitive Explanation}
Imagine you have a NAND gate as a magical box with two switches (inputs). The box will only turn off (output 0) if both switches are turned on (input 1). If even one switch is off, the box stays on (output 1). So, the NAND gate is like saying, I will only turn off if both switches are on; otherwise, I stay on.

\subsubsection{Advanced Explanation}
A NAND gate is a digital logic gate that performs the logical NAND (NOT AND) operation. The NAND operation is the complement of the AND operation. The truth table for a NAND gate with two inputs \(A\) and \(B\) is as follows:

\[
\begin{array}{|c|c|c|}
\hline
A & B & \text{Output} \\
\hline
0 & 0 & 1 \\
0 & 1 & 1 \\
1 & 0 & 1 \\
1 & 1 & 0 \\
\hline
\end{array}
\]

Mathematically, the output \(Y\) of a NAND gate can be expressed as:
\[
Y = \overline{A \cdot B}
\]
where \(\overline{A \cdot B}\) represents the NOT of the AND operation between \(A\) and \(B\).

The NAND gate is a universal gate, meaning that any other logical operation (AND, OR, NOT, etc.) can be constructed using only NAND gates. This property makes it extremely useful in digital circuit design.

% Prompt for generating a diagram:
% Diagram showing a NAND gate with two inputs (A and B) and one output (Y). The gate should be labeled as NAND and the truth table should be displayed alongside it for clarity.