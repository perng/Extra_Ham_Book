\subsection{Understanding USB Signal Regulations!}

\begin{tcolorbox}[colback=gray!10!white,colframe=black!75!black,title=E1A01]   
\textbf{E1A01} Why is it not legal to transmit a 3 kHz bandwidth USB signal with a carrier frequency of 14.348 MHz?
\begin{enumerate}[label=\Alph*),noitemsep]
    \item USB is not used on 20-meter phone
    \item The lower 1 kHz of the signal is outside the 20-meter band
    \item 14.348 MHz is outside the 20-meter band
    \item \textbf{The upper 1 kHz of the signal is outside the 20-meter band}
\end{enumerate}
\end{tcolorbox}

\subsubsection{Intuitive Explanation}
Imagine you have a radio signal that you want to send out. This signal has a certain width, like a piece of paper. The 20-meter band is like a specific shelf where you can place this paper. If the paper is too wide, part of it will hang off the shelf, and that's not allowed. In this case, the signal is 3 kHz wide, and when you place it at 14.348 MHz, the top part of the signal (the upper 1 kHz) hangs off the shelf, making it illegal to transmit.

\subsubsection{Advanced Explanation}
The 20-meter amateur radio band spans from 14.000 MHz to 14.350 MHz. A USB (Upper Sideband) signal with a carrier frequency of 14.348 MHz and a bandwidth of 3 kHz will extend from 14.348 MHz to 14.351 MHz. 

To determine if the signal is within the band, we calculate the upper limit of the signal:
\[
\text{Upper Limit} = \text{Carrier Frequency} + \frac{\text{Bandwidth}}{2} = 14.348\,\text{MHz} + \frac{3\,\text{kHz}}{2} = 14.348\,\text{MHz} + 1.5\,\text{kHz} = 14.3495\,\text{MHz}
\]
However, since the bandwidth is 3 kHz, the signal extends up to:
\[
14.348\,\text{MHz} + 3\,\text{kHz} = 14.351\,\text{MHz}
\]
The 20-meter band ends at 14.350 MHz, so the upper 1 kHz of the signal (from 14.350 MHz to 14.351 MHz) is outside the allowed frequency range. This makes the transmission illegal.

Related concepts include:
\begin{itemize}
    \item \textbf{Bandwidth}: The range of frequencies a signal occupies.
    \item \textbf{Carrier Frequency}: The center frequency of a signal.
    \item \textbf{Upper Sideband (USB)}: A method of transmitting information by modulating the upper sideband of a carrier wave.
    \item \textbf{Frequency Allocation}: Specific frequency ranges allocated for different types of communication.
\end{itemize}

% Diagram Prompt: Generate a diagram showing the 20-meter band (14.000 MHz to 14.350 MHz) and the USB signal (14.348 MHz to 14.351 MHz) to visually illustrate why the upper 1 kHz is outside the band.