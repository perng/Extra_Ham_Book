\subsection{Finding the Fun in FM: What's the Modulation Index?}

\begin{tcolorbox}
    \textbf{Question ID: E8B03} \\
    What is the modulation index of an FM phone signal having a maximum frequency deviation of 3000 Hz either side of the carrier frequency if the highest modulating frequency is 1000 Hz? \\

    \begin{enumerate}[label=\Alph*.]
        \item \textbf{3}
        \item 0.3
        \item 6
        \item 0.6
    \end{enumerate}
\end{tcolorbox}

\subsubsection{Intuitive Explanation}
Imagine you are riding on a swing, swinging back and forth. The maximum distance you swing away from the center is like how much the radio signal can change from its original frequency – this is called frequency deviation. Now, think of how quickly you can move back and forth; that is like the modulating frequency, how often you make your swing go. The modulation index tells us how far out you swing relative to how fast you are swinging. If you swing out very far (big deviation) but swing back and forth slowly (small frequency), the modulation index is larger. In this question, we are seeing how this applies to a radio signal.

\subsubsection{Advanced Explanation}
In frequency modulation (FM), the modulation index (h) is defined as the ratio of the frequency deviation ($\Delta f$) to the modulating frequency (fm). The formula for the modulation index is given by:

\[
h = \frac{\Delta f}{f_m}
\]

In this problem, the maximum frequency deviation is given as 3000 Hz, which means:

\[
\Delta f = 3000 \text{ Hz}
\]

The highest modulating frequency is provided as 1000 Hz, thus:

\[
f_m = 1000 \text{ Hz}
\]

Now, replacing the values in the formula for modulation index:

\[
h = \frac{3000}{1000} = 3
\]

This calculation indicates that the modulation index is 3. Therefore, the correct answer to the question is option A.

Now, elaborating on related concepts, frequency modulation varies the frequency of a carrier signal, rather than its amplitude. This has several advantages in terms of signal quality and noise resistance, which is particularly important in telecommunications, such as radio broadcasts and mobile phone signals.

% Diagram prompt comment: Generate a diagram illustrating the concept of frequency modulation showing a carrier wave and modulating wave with indications of frequency deviation and modulation index.