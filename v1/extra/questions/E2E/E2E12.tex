\subsection{Connecting with ALE Stations: How It's Done!}
\label{subsec:E2E12}

\begin{tcolorbox}[colback=gray!10!white,colframe=black!75!black,title=\textbf{E2E12}]
\textbf{How do ALE stations establish contact?}
\begin{enumerate}[label=\Alph*.]
    \item \textbf{ALE constantly scans a list of frequencies, activating the radio when the designated call sign is received}
    \item ALE radios monitor an internet site for the frequency they are being paged on
    \item ALE radios send a constant tone code to establish a frequency for future use
    \item ALE radios activate when they hear their signal echoed by back scatter
\end{enumerate}
\end{tcolorbox}

\subsubsection{Intuitive Explanation}
Imagine you have a walkie-talkie that can listen to many different channels at once. ALE stations work like this walkie-talkie. They keep switching between different frequencies (channels) to see if someone is calling them. When they hear their special name (call sign) on one of these channels, they stop switching and start talking on that channel. This way, they can always be ready to communicate without missing any important messages.

\subsubsection{Advanced Explanation}
ALE (Automatic Link Establishment) stations use a sophisticated method to establish communication. They operate by continuously scanning a predefined list of frequencies. This scanning process is known as frequency hopping. When an ALE station detects its designated call sign on one of these frequencies, it halts the scanning process and locks onto that frequency to initiate communication. 

Mathematically, the process can be described as follows:
\begin{itemize}
    \item Let \( F = \{f_1, f_2, \dots, f_n\} \) be the set of frequencies that the ALE station scans.
    \item The station sequentially checks each frequency \( f_i \) for the presence of its call sign \( C \).
    \item Once \( C \) is detected on frequency \( f_k \), the station stops scanning and establishes a communication link on \( f_k \).
\end{itemize}

This method ensures that ALE stations can reliably establish communication links even in environments with varying propagation conditions. The use of frequency hopping also enhances security and reduces the likelihood of interference.

% Prompt for generating a diagram: A diagram showing an ALE station scanning multiple frequencies and locking onto one when the call sign is detected would be helpful here.