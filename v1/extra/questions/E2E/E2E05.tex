\subsection{Discovering Keyboard Modes: Which One's Left Out?}

\begin{tcolorbox}[colback=gray!10!white,colframe=black!75!black,title=\textbf{E2E05}]
\textbf{Which of these digital modes does not support keyboard-to-keyboard operation?}
\begin{enumerate}[label=\Alph*),noitemsep]
    \item \textbf{WSPR}
    \item RTTY
    \item PSK31
    \item MFSK16
\end{enumerate}
\end{tcolorbox}

\subsubsection{Intuitive Explanation}
Imagine you have different ways to send messages using your computer. Some of these ways let you type directly on your keyboard and send the message to someone else's keyboard, almost like chatting. However, not all of them work this way. WSPR is like a special tool that sends out signals to tell others where you are, but it doesn't let you type messages back and forth. So, WSPR is the one that doesn't support keyboard-to-keyboard chatting.

\subsubsection{Advanced Explanation}
Digital modes in radio communication allow for the transmission of data over radio waves. Keyboard-to-keyboard operation refers to the ability to send and receive text messages in real-time using a keyboard. 

\begin{itemize}
    \item \textbf{WSPR (Weak Signal Propagation Reporter)}: This mode is primarily used for weak signal communication and propagation reporting. It is not designed for real-time text communication and does not support keyboard-to-keyboard operation.
    \item \textbf{RTTY (Radio Teletype)}: This mode is a traditional digital mode that supports real-time text communication, allowing for keyboard-to-keyboard operation.
    \item \textbf{PSK31}: This mode is a popular digital mode for real-time text communication, supporting keyboard-to-keyboard operation.
    \item \textbf{MFSK16}: This mode is another digital mode that supports real-time text communication, allowing for keyboard-to-keyboard operation.
\end{itemize}

Therefore, the correct answer is \textbf{WSPR}, as it does not support keyboard-to-keyboard operation.

% Prompt for generating a diagram:
% A diagram showing the different digital modes and their support for keyboard-to-keyboard operation could be helpful. The diagram could include icons or labels for WSPR, RTTY, PSK31, and MFSK16, with a clear indication that WSPR does not support keyboard-to-keyboard operation.