\subsection{Understanding Susceptance: A Bright Inquiry!}

\begin{tcolorbox}[colback=gray!10!white,colframe=black!75!black,title=E5B06] What is susceptance?
    \begin{enumerate}[label=\Alph*),noitemsep]
        \item The magnetic impedance of a circuit
        \item The ratio of magnetic field to electric field
        \item \textbf{The imaginary part of admittance}
        \item A measure of the efficiency of a transformer
    \end{enumerate}
\end{tcolorbox}

\subsubsection*{Intuitive Explanation}
Imagine you have a water pipe that can carry water (electricity) through it. Now, think of a valve that can either let the water flow easily or make it harder for the water to pass. Susceptance is like how easily the valve lets the water flow, but in the world of electricity. It’s a part of something called admittance, which tells us how easily electricity can flow through a circuit. Susceptance is the imaginary part of admittance, which means it deals with how the circuit reacts to changes in the flow of electricity, especially when there are things like capacitors or inductors in the circuit.

\subsubsection*{Advanced Explanation}
In electrical engineering, admittance (\(Y\)) is a measure of how easily a circuit allows the flow of alternating current (AC). It is the reciprocal of impedance (\(Z\)) and is given by:
\[
Y = \frac{1}{Z}
\]
Admittance is a complex quantity, consisting of a real part called conductance (\(G\)) and an imaginary part called susceptance (\(B\)). Mathematically, it is expressed as:
\[
Y = G + jB
\]
where \(j\) is the imaginary unit. Susceptance (\(B\)) represents the reactive component of admittance and is associated with the energy storage elements in the circuit, such as capacitors and inductors. For a purely capacitive circuit, susceptance is positive, while for a purely inductive circuit, it is negative. The unit of susceptance is the siemens (S).

To calculate susceptance, we use the following formulas:
\[
B_C = \omega C \quad \text{(for a capacitor)}
\]
\[
B_L = -\frac{1}{\omega L} \quad \text{(for an inductor)}
\]
where \(\omega\) is the angular frequency, \(C\) is the capacitance, and \(L\) is the inductance.

Understanding susceptance is crucial for analyzing AC circuits, especially when dealing with resonance, power factor correction, and impedance matching.

% Diagram Prompt: Generate a diagram showing a simple AC circuit with a capacitor and an inductor, labeling the susceptance for each component.