\subsection{Susceptance Symbol: What’s the Letter?}

\begin{tcolorbox}[colback=gray!10!white,colframe=black!75!black,title=E5B02] What letter is commonly used to represent susceptance?
    \begin{enumerate}[label=\Alph*.]
        \item G
        \item X
        \item Y
        \item \textbf{B}
    \end{enumerate}
\end{tcolorbox}

\subsubsection*{Intuitive Explanation}
Susceptance is a term used in electrical engineering to describe how easily alternating current (AC) can flow through a circuit that has capacitors or inductors. Think of it like a measure of how open or closed a pathway is for AC electricity. Just like we use letters to represent different things in math (like V for voltage or I for current), engineers use the letter B to represent susceptance. It’s like a special code that everyone agrees on to make things easier to understand.

\subsubsection*{Advanced Explanation}
In electrical engineering, susceptance (\(B\)) is the imaginary part of admittance (\(Y\)), which describes how easily an alternating current (AC) can flow through a circuit. Admittance is the reciprocal of impedance (\(Z\)), and it is represented as:
\[
Y = G + jB
\]
where:
\begin{itemize}
    \item \(Y\) is the admittance,
    \item \(G\) is the conductance (the real part of admittance),
    \item \(B\) is the susceptance (the imaginary part of admittance),
    \item \(j\) is the imaginary unit.
\end{itemize}
Susceptance is specifically associated with reactive components like capacitors and inductors. For a capacitor, the susceptance is given by:
\[
B_C = \omega C
\]
where:
\begin{itemize}
    \item \(\omega\) is the angular frequency of the AC signal,
    \item \(C\) is the capacitance.
\end{itemize}
For an inductor, the susceptance is:
\[
B_L = -\frac{1}{\omega L}
\]
where:
\begin{itemize}
    \item \(L\) is the inductance.
\end{itemize}
The letter B is universally used to denote susceptance in these equations, making it a standard symbol in electrical engineering literature.

% [Prompt for diagram: A simple circuit diagram showing a capacitor and inductor in parallel with labels for susceptance (B) and admittance (Y).]