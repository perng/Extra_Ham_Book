\subsection{Converting Impedance to Admittance: A Cheerful Guide!}

\begin{tcolorbox}[colback=gray!10!white,colframe=black!75!black,title=Multiple Choice Question]
    \textbf{E5B03} How is impedance in polar form converted to an equivalent admittance?
    \begin{enumerate}[label=\Alph*),noitemsep]
        \item Take the reciprocal of the angle and change the sign of the magnitude
        \item \textbf{Take the reciprocal of the magnitude and change the sign of the angle}
        \item Take the square root of the magnitude and add 180 degrees to the angle
        \item Square the magnitude and subtract 90 degrees from the angle
    \end{enumerate}
\end{tcolorbox}

\subsubsection{Intuitive Explanation}
Imagine you have a special kind of resistance called impedance, which is like a combination of regular resistance and something called reactance. Now, admittance is just the opposite of impedance—it tells you how easily electricity can flow through a circuit. To change impedance into admittance, you flip the size (take the reciprocal) and reverse the direction (change the sign of the angle). It's like turning a big, hard-to-push boulder into a small, easy-to-roll pebble!

\subsubsection{Advanced Explanation}
Impedance \( Z \) in polar form is represented as:
\[
Z = |Z| \angle \theta
\]
where \( |Z| \) is the magnitude and \( \theta \) is the phase angle. Admittance \( Y \) is the reciprocal of impedance:
\[
Y = \frac{1}{Z}
\]
To convert impedance to admittance in polar form, follow these steps:
1. Take the reciprocal of the magnitude:
\[
|Y| = \frac{1}{|Z|}
\]
2. Change the sign of the angle:
\[
\theta_Y = -\theta
\]
Thus, the admittance in polar form is:
\[
Y = |Y| \angle \theta_Y = \frac{1}{|Z|} \angle -\theta
\]
This conversion is essential in analyzing AC circuits, where impedance and admittance are used to describe the opposition and ease of current flow, respectively.

% Prompt for generating a diagram: A diagram showing the relationship between impedance and admittance in polar form, with arrows indicating the reciprocal of magnitude and the change in angle sign.