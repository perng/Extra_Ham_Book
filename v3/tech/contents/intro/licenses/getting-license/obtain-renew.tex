\subsection{Obtaining and Renewing}
\label{subsec:obtain-renew}

So, you've decided to dive into the world of amateur radio! Congratulations! But before you can start chatting with fellow hams across the globe, you'll need to get your hands on a license. Let's break down the process, step by step.

\subsubsection{License Exam}
First things first, you need to pass an exam. You can take the exam in-person or online. 
I can share my experience that the online exam is much easier to take as long as you are comfortable with video conferencing.  

The first step is to register at the FCC Commission Registration System\footnote{https://apps.fcc.gov/cores/}. This is to obtain your FCC FRN\footnote{An FRN is a Federal Communications Commission Registration Number} number. 

Then you need to register for an exam session. The most user friendly portal to find an exam session would be on HamStudy\footnote{https://hamstudy.org/sessions/remote}. You can find the exam session that is the most convenient for you. You'll need to use your FRN number. The fee for the exam is typically \$15, you have to pay it online before the exam. 

Before the exam, the exam team will send you an email with the exam instructions with a Zoom link. The email has the tendency to go to your spam folder, so make sure to check it. 

During the exam, the examiners will ask you to show both your physical and computer environments. So I'd suggest you to be alone in a simple room with a clean desk and a closed door. Remove all smart devices from your desk and from your body. That means you should put away your smart watch, smart glasses, smart phone, smart shoes, and smart anything. This doesn't mean your dumb brother can stay in the room with you. You should also close all programs on your computer except for a web browser. The examiners would allow you to use a physical calculator or the calculator app on your computer for the exam but you should be able to do the technician exam without it. 

You have to show a photo ID to the examiners before the exam. 

The exam is a multiple choice exam. You will have 90 minutes to complete the exam. For Technician and General class, you need to get 26  out of 35 questions correct to pass. For Extra class, you need to get 37 out of 50 questions correct to pass. 

The online exam is exactly the same format as the practice exams on the HamStudy website. You should definitely get their mobile app to practice the exam questions while you are waiting in a cashier line. 

The examiners, from all 3 exams I took, were the most friendly and helpful you can imagine. But they are also very strict. There will be at least 3 examiners there  overseeing you taking the exam. They will be watching your every move. I had the pleasure to have 14(!) examiners watching me as the lone examinee taking the General exam, and 9 for the Extra exam, no pressure at all. 


\subsubsection{License Grant}
Once you've aced tje, the FCC (Federal Communications Commission) will issue you an operator/primary station license grant, and a call sign. But you need to pay a \$35 fee to the FCC to get your license. 

Now, here's the kicker: you can only hold one of these at a time. That's right, no hoarding licenses for different bands or locations. Just one per person, as per FCC rules.


\subsubsection{FCC ULS Database}
Once the FCC grants your license, it will appear in the FCC ULS (Universal Licensing System) database. This is your golden ticket. If your license is in the ULS, you're good to go. No need to wait for a physical copy in the mail or an email notification. The ULS is the ultimate authority on your licensing status.

\subsubsection*{License Term and Grace Period}
Your shiny new license is valid for ten years. Yes, a whole decade of radio fun! But what happens if you forget to renew it? Don't worry, the FCC has a grace period of two years. During this time, you can still renew your license without having to retake the exam. However, and this is important, you cannot transmit during the grace period. You must wait until your license is officially renewed.

\subsubsection*{Transmission Authorization}
Once your license appears in the ULS database, you're officially authorized to transmit. No need to wait for a physical copy or any other confirmation. The ULS is your go-to source for all things licensing.


\begin{table}[h]
    \centering
    \begin{tabular}{|l|l|}
        \hline
        \textbf{Key Point} & \textbf{Details} \\ \hline
        License Grant & One per person, issued by the FCC \\ \hline
        FCC ULS Database & Official record of license status \\ \hline
        License Term & 10 years \\ \hline
        Grace Period & 2 years, no transmission allowed \\ \hline
        Transmission Authorization & Upon ULS database entry \\ \hline
    \end{tabular}
    \caption{Summary of License Obtaining and Renewing}
    \label{tab:license-summary}
\end{table}

\subsubsection*{Questions}

\begin{tcolorbox}[colback=gray!10!white,colframe=black!75!black,title={T1A04}]
    How many operator/primary station license grants may be held by any one person?
    \begin{enumerate}[label=\Alph*),noitemsep]
        \item \textbf{One}
        \item No more than two
        \item One for each band on which the person plans to operate
        \item One for each permanent station location from which the person plans to operate
    \end{enumerate}
\end{tcolorbox}
The FCC allows only one operator/primary station license grant per person. This ensures that each licensee is responsible for their own station and operations.

\begin{tcolorbox}[colback=gray!10!white,colframe=black!75!black,title={T1A05}]
    What proves that the FCC has issued an operator/primary license grant?
    \begin{enumerate}[label=\Alph*),noitemsep]
        \item A printed copy of the certificate of successful completion of examination
        \item An email notification from the NCVEC granting the license
        \item \textbf{The license appears in the FCC ULS database}
        \item All these choices are correct
    \end{enumerate}
\end{tcolorbox}
The FCC ULS database is the official record of your license status. If your license is listed there, it's official.

\begin{tcolorbox}[colback=gray!10!white,colframe=black!75!black,title={T1C08}]
    What is the normal term for an FCC-issued amateur radio license?
    \begin{enumerate}[label=\Alph*),noitemsep]
        \item Five years
        \item Life
        \item \textbf{Ten years}
        \item Eight years
    \end{enumerate}
\end{tcolorbox}
An amateur radio license is valid for ten years. After that, you'll need to renew it to continue operating.

\begin{tcolorbox}[colback=gray!10!white,colframe=black!75!black,title={T1C09}]
    What is the grace period for renewal if an amateur license expires?
    \begin{enumerate}[label=\Alph*),noitemsep]
        \item \textbf{Two years}
        \item Three years
        \item Five years
        \item Ten years
    \end{enumerate}
\end{tcolorbox}
The grace period for renewing an expired license is two years. During this time, you can renew without retaking the exam, but you cannot transmit.

\begin{tcolorbox}[colback=gray!10!white,colframe=black!75!black,title={T1C10}]
    How soon after passing the examination for your first amateur radio license may you transmit on the amateur radio bands?
    \begin{enumerate}[label=\Alph*),noitemsep]
        \item Immediately on receiving your Certificate of Successful Completion of Examination (CSCE)
        \item As soon as your operator/station license grant appears on the ARRL website
        \item \textbf{As soon as your operator/station license grant appears in the FCC’s license database}
        \item As soon as you receive your license in the mail from the FCC
    \end{enumerate}
\end{tcolorbox}
You can start transmitting as soon as your license appears in the FCC ULS database. No need to wait for physical copies or other confirmations.

\begin{tcolorbox}[colback=gray!10!white,colframe=black!75!black,title={T1C11}]
    If your license has expired and is still within the allowable grace period, may you continue to transmit on the amateur radio bands?
    \begin{enumerate}[label=\Alph*),noitemsep]
        \item Yes, for up to two years
        \item Yes, as soon as you apply for renewal
        \item Yes, for up to one year
        \item \textbf{No, you must wait until the license has been renewed}
    \end{enumerate}
\end{tcolorbox}
Even if your license is within the grace period, you cannot transmit until it has been officially renewed. This ensures that all operators are properly licensed and compliant with FCC regulations.
