\subsection{License Maintenance, Contact Information}
\label{subsec:maint-contact}

Maintaining accurate contact information with the FCC is not just a bureaucratic formality—it’s a critical part of being a responsible amateur radio operator. Imagine the FCC trying to reach you about an important update or issue, only to find that your email address is outdated. Spoiler alert: it doesn’t end well. Let’s dive into why this matters and what could happen if you drop the ball.

\subsubsection*{Why Accurate Contact Information Matters}
The FCC requires you to provide and maintain a correct email address. Why? Because email is their primary method of communication with licensees. If they can’t reach you, they can’t notify you about important updates, rule changes, or even issues with your station. And trust me, you don’t want to be in the dark about those things.

\subsubsection*{Consequences of Failing to Maintain Contact Information}
If the FCC is unable to reach you by email, they have the authority to take action. This could include revoking your station license or suspending your operator license. Yes, you read that right—your license could be on the line. This isn’t just a slap on the wrist; it’s a serious consequence for what might seem like a minor oversight.


\begin{table}[h!]
    \centering
    \begin{tabular}{|l|l|}
        \hline
        \textbf{Requirement} & \textbf{Consequence} \\
        \hline
        Provide a correct email address & Revocation of station license or suspension of operator license \\
        \hline
        Update contact information promptly & Avoid fines and license issues \\
        \hline
    \end{tabular}
    \caption{Summary of License Maintenance and Contact Information Requirements}
    \label{tab:license-maintenance-summary}
\end{table}

\subsubsection*{Questions}

\begin{tcolorbox}[colback=gray!10!white,colframe=black!75!black,title={T1C04}]
What may happen if the FCC is unable to reach you by email?
\begin{enumerate}[label=\Alph*),noitemsep]
    \item Fine and suspension of operator license
    \item \textbf{Revocation of the station license or suspension of the operator license}
    \item Revocation of access to the license record in the FCC system
    \item Nothing; there is no such requirement
\end{enumerate}
\end{tcolorbox}

If the FCC can’t reach you by email, they have the authority to revoke your station license or suspend your operator license. This is because email is their primary method of communication, and they need to be able to contact you for important updates or issues. Options A and C are incorrect because the FCC’s primary concern is communication, not fines or access to records. Option D is incorrect because there is indeed a requirement to maintain a correct email address.

\begin{tcolorbox}[colback=gray!10!white,colframe=black!75!black,title={T1C07}]
Which of the following can result in revocation of the station license or suspension of the operator license?
\begin{enumerate}[label=\Alph*),noitemsep]
    \item Failure to inform the FCC of any changes in the amateur station following performance of an RF safety environmental evaluation
    \item \textbf{Failure to provide and maintain a correct email address with the FCC}
    \item Failure to obtain FCC type acceptance prior to using a home-built transmitter
    \item Failure to have a copy of your license available at your station
\end{enumerate}
\end{tcolorbox}

Failure to provide and maintain a correct email address with the FCC can result in revocation of your station license or suspension of your operator license. This is because the FCC relies on email to communicate with licensees. Option A is incorrect because while RF safety evaluations are important, they don’t directly relate to contact information. Option C is incorrect because type acceptance is about equipment, not contact information. Option D is incorrect because while having a copy of your license at your station is a good practice, it’s not directly related to contact information.
