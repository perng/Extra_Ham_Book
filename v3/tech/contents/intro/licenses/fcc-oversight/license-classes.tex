\subsection{Classes of Licenses}
\label{subsec:license-classes-overview}

If you've ever wondered why amateur radio operators seem to have different levels of "radio superpowers," it all comes down to their license class. The Federal Communications Commission (FCC) offers three main classes of amateur radio licenses: Technician, General, and Amateur Extra. Each class comes with its own set of privileges, requirements, and bragging rights. Let's dive into what makes each of these licenses unique.

\subsubsection*{Technician Class}
The Technician license is the entry-level ticket to the world of amateur radio. It’s like getting your driver's license, but instead of a car, you get to operate on the VHF and UHF bands. Technicians can also access some portions of the HF bands, but their privileges are more limited compared to the higher classes. Think of it as the "starter pack" of amateur radio.

\subsubsection*{General Class}
The General license is the next step up, offering more privileges, especially on the HF bands. With a General license, you can operate on most amateur radio frequencies, which means you can talk to people across the globe. It’s like upgrading from a bicycle to a motorcycle—more power, more range, and more fun.

\subsubsection*{Amateur Extra Class}
The Amateur Extra license is the top tier, the "black belt" of amateur radio. It grants access to all amateur radio frequencies and modes, including some exclusive bands. If you’re the kind of person who wants to explore every nook and cranny of the radio spectrum, this is the license for you.

\subsubsection*{Historical Context}
The amateur radio license classes have evolved over time. In the past, there were additional classes like Novice and Advanced, but these have been phased out but there are still some operators who hold these licenses. This created an interesting situation where "Technician, General, or Amateur Extra" are not equal to "all licenses". There is a question in Amateur Extra question pool that asks about this.


\subsubsection*{Comparison of Privileges}
To help you visualize the differences, take a look at Table~\ref{tab:license_privileges}, which compares the privileges of the Technician, General, and Amateur Extra license classes. You’ll notice that as you move up the license classes, the frequency access and modes of operation expand significantly.

\begin{table}[h!]
    \centering
    \footnotesize
    \caption{Comparison of U.S. Amateur Radio License Class Privileges}
    \label{tab:license_privileges}
    \begin{tabular}{|p{2.5cm}|p{3.5cm}|p{3.5cm}|p{3.5cm}|}
    \hline
    \textbf{Frequency Band (Partial List)} & \textbf{Technician} & \textbf{General} & \textbf{Amateur Extra} \\
    \hline
    \textbf{HF Bands}   \\
    \hline
    80 Meters (3.5-4.0 MHz) &  Novice privileges on CW: 3.525-3.600 MHz & Most privileges, except certain segments & All privileges \\
    \hline
    40 Meters (7.0-7.3 MHz) & Novice privileges on CW: 7.025-7.125 MHz & Most privileges, except certain segments & All privileges \\
    \hline
    20 Meters (14.0-14.35 MHz) &  & Most privileges, except certain segments & All privileges \\
    \hline
    15 Meters (21.0-21.45 MHz) & Novice privileges on CW: 21.100-21.200 MHz and RTTY/data: 21.025-21.200 MHz  & Most privileges, except certain segments & All privileges \\
    \hline
    10 Meters (28.0-29.7 MHz) & CW, RTTY/data: 28.000-28.300 MHz, SSB: 28.300-28.500 MHz & All privileges & All privileges \\
    \hline
    \textbf{Maximum HF Power} & 200W PEP for CW in authorized segments & 1500W PEP (except where limited by band/mode) & 1500W PEP (except where limited by band/mode) \\
    \hline
    \textbf{VHF/UHF Bands (Examples)}  \\
    \hline
    6 Meters (50-54 MHz) & All privileges & All privileges & All privileges \\
    \hline
    2 Meters (144-148 MHz) & All privileges & All privileges & All privileges \\
    \hline
    1.25 Meters (222-225 MHz) & All privileges & All privileges & All privileges \\
    \hline
    70 cm (420-450 MHz) & All privileges & All privileges & All privileges \\
    \hline
    33 cm (902-928 MHz) & All privileges & All privileges & All privileges \\
    \hline
    23 cm (1240-1300 MHz) & All privileges & All privileges & All privileges \\
    \hline
    \textbf{Other Privileges} &  &  &  \\
    \hline
    Maximum Transmit Power & 1500 Watts PEP (Except where limited by band or mode) & 1500 Watts PEP (Except where limited by band or mode) & 1500 Watts PEP (Except where limited by band or mode)\\
    \hline
    Exam Elements Required & Element 2 & Elements 2 and 3 & Elements 2, 3, and 4 \\
    \hline
    
    \end{tabular}
    \end{table}

\subsubsection*{Questions}
\begin{tcolorbox}[colback=gray!10!white,colframe=black!75!black,title={T1C01}]
For which license classes are new licenses currently available from the FCC?
\begin{enumerate}[label=\Alph*),noitemsep]
    \item Novice, Technician, General, Amateur Extra
    \item Technician, Technician Plus, General, Amateur Extra
    \item Novice, Technician Plus, General, Advanced
    \item \textbf{Technician, General, Amateur Extra}
\end{enumerate}
\end{tcolorbox}

 The FCC currently offers new licenses for the Technician, General, and Amateur Extra classes. The Novice and Advanced classes, along with the Technician Plus, are no longer available for new licenses. This reflects the FCC's streamlined approach to amateur radio licensing, focusing on a clear progression from Technician to General to Amateur Extra.

