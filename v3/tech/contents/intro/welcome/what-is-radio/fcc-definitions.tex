\subsection{FCC Part 97 Definitions \& Concepts}
\label{subsec:fcc-definitions}

In this section, we'll dive into some key definitions and concepts from FCC Part 97\footnote{https://www.ecfr.gov/current/title-47/chapter-I/subchapter-D/part-97}. These definitions are crucial for understanding the rules and regulations that govern amateur radio operations. So, grab a cup of tea, and let's get started!

\subsubsection*{Beacon}
According to FCC Part 97, a \textbf{beacon} is defined as an amateur station transmitting communications for the purposes of observing propagation or related experimental activities. Think of it as a lighthouse in the vast ocean of radio waves, helping us understand how signals propagate under different conditions. Beacons are essential for experiments and can provide valuable data on signal strength, frequency stability, and more.

The International Amateur Radio Union (IARU) Beacon Project is a globally coordinated network of HF radio beacons designed to help amateur radio operators assess real-time propagation conditions on the 14, 18, 21, 24, and 28 MHz bands.  Eighteen time-shared beacons on five continents transmit sequentially on each band, allowing listeners to quickly determine which areas of the world are reachable on which frequencies. While the IARU coordinates the project and sets standards, the beacons themselves are built and maintained by volunteer member societies and radio clubs around the world, demonstrating international cooperation within the amateur radio community. The NCDXF/IARU International Beacon Project also supports this endeavor.

\subsubsection*{Space Station}
A \textbf{space station}, as per FCC Part 97, is an amateur station located more than 50 km above Earth's surface. This definition is particularly exciting because it opens up the possibility for amateur radio operators to communicate with satellites and even the International Space Station (ISS). Imagine chatting with an astronaut orbiting Earth—how cool is that?

\subsubsection*{Frequency Coordinator}
The role of a \textbf{Frequency Coordinator} is to recommend transmit/receive channels and other parameters for auxiliary and repeater stations. These coordinators are volunteers recognized by local amateur operators. They play a crucial role in ensuring that frequencies are used efficiently and without causing interference. The selection of a Frequency Coordinator is done by amateur operators in a local or regional area whose stations are eligible to be repeater or auxiliary stations.

\subsubsection*{Radio Amateur Civil Emergency Service (RACES)}
The \textbf{Radio Amateur Civil Emergency Service (RACES)} is a radio service using amateur stations for emergency management or civil defense communications. In times of crisis, RACES operators can provide critical communication links when other systems fail. It's like having a superhero team of radio operators ready to save the day!


\subsubsection*{Willful Interference}
Willful interference to other amateur radio stations is strictly prohibited under FCC Part 97. This means that no amateur operator is allowed to intentionally disrupt the communications of another station. The rule is clear: play nice, or face the consequences.

\begin{table}[h]
    \centering
    \caption{Summary of key definitions from FCC Part 97.}
    \label{tab:fcc-definitions-summary}
    \begin{tabular}{|l|p{8cm}|}
        \hline
        \textbf{Term} & \textbf{Definition} \\
        \hline
        Beacon & An amateur station transmitting communications for the purposes of observing propagation or related experimental activities. \\
        \hline
        Space Station & An amateur station located more than 50 km above Earth's surface. \\
        \hline
        Frequency Coordinator & A volunteer who recommends transmit/receive channels and other parameters for auxiliary and repeater stations. \\
        \hline
        RACES & A radio service using amateur stations for emergency management or civil defense communications. \\
        \hline
        Willful Interference & Intentional disruption of other amateur radio stations, which is strictly prohibited. \\
        \hline
    \end{tabular}
\end{table}


\subsubsection{Questions}

\begin{tcolorbox}[colback=gray!10!white,colframe=black!75!black,title={T1A06}]
    What is the FCC Part 97 definition of a beacon?
    \begin{enumerate}[label=\Alph*),noitemsep]
        \item A government transmitter marking the amateur radio band edges
        \item A bulletin sent by the FCC to announce a national emergency
        \item A continuous transmission of weather information authorized in the amateur bands by the National Weather Service
        \item \textbf{An amateur station transmitting communications for the purposes of observing propagation or related experimental activities}
    \end{enumerate}
\end{tcolorbox}
The correct definition of a beacon, as per FCC Part 97, is an amateur station transmitting communications for the purposes of observing propagation or related experimental activities. This aligns with the role of beacons in providing valuable data on signal propagation. Option A is usually called "band-edge markers" or "boundary markers". Option B is the Emergency Alert System (EAS). Option C doesn't exist as described. The National Weather Service (NWS) does not authorize continuous weather information transmissions within the amateur radio bands.

\begin{tcolorbox}[colback=gray!10!white,colframe=black!75!black,title={T1A07}]
    What is the FCC Part 97 definition of a space station?
    \begin{enumerate}[label=\Alph*),noitemsep]
        \item Any satellite orbiting Earth
        \item A manned satellite orbiting Earth
        \item \textbf{An amateur station located more than 50 km above Earth's surface}
        \item An amateur station using amateur radio satellites for relay of signals
    \end{enumerate}
\end{tcolorbox}
A space station, according to FCC Part 97, is specifically an amateur station located more than 50 km above Earth's surface. Satellites in Low Earth Orbit (LEO) typically orbit at altitudes between 160 km (roughly 100 miles) and 2,000 km (roughly 1,200 miles). There is no satellite orbiting under 50 km above Earth's surface.

\begin{tcolorbox}[colback=gray!10!white,colframe=black!75!black,title={T1A08}]
    Which of the following entities recommends transmit/receive channels and other parameters for auxiliary and repeater stations?
    \begin{enumerate}[label=\Alph*),noitemsep]
        \item Frequency Spectrum Manager appointed by the FCC
        \item \textbf{Volunteer Frequency Coordinator recognized by local amateurs}
        \item FCC Regional Field Office
        \item International Telecommunication Union
    \end{enumerate}
\end{tcolorbox}
The correct answer is the Volunteer Frequency Coordinator recognized by local amateurs. These coordinators play a vital role in managing frequencies to prevent interference.

\begin{tcolorbox}[colback=gray!10!white,colframe=black!75!black,title={T1A09}]
    Who selects a Frequency Coordinator?
    \begin{enumerate}[label=\Alph*),noitemsep]
        \item The FCC Office of Spectrum Management and Coordination Policy
        \item The local chapter of the Office of National Council of Independent Frequency Coordinators
        \item \textbf{Amateur operators in a local or regional area whose stations are eligible to be repeater or auxiliary stations}
        \item FCC Regional Field Office
    \end{enumerate}
\end{tcolorbox}
Frequency Coordinators are selected by amateur operators in a local or regional area whose stations are eligible to be repeater or auxiliary stations. This ensures that the coordinator is familiar with the specific needs of the local amateur community. Option A is incorrect because the FCC Office of Spectrum Management and Coordination Policy does not select Frequency Coordinators. Option B is incorrect because the Office of National Council of Independent Frequency Coordinators does not exist. Option D is incorrect because the FCC Regional Field Office does not select Frequency Coordinators.

\begin{tcolorbox}[colback=gray!10!white,colframe=black!75!black,title={T1A10}]
    What is the Radio Amateur Civil Emergency Service (RACES)?
    \begin{enumerate}[label=\Alph*),noitemsep]
        \item A radio service using amateur frequencies for emergency management or civil defense communications
        \item A radio service using amateur stations for emergency management or civil defense communications
        \item An emergency service using amateur operators certified by a civil defense organization as being enrolled in that organization
        \item \textbf{All these choices are correct}
    \end{enumerate}
\end{tcolorbox}
RACES encompasses all the options listed, making it a comprehensive service for emergency communications using amateur radio. Option A is incorrect because it doesn't mention the amateur radio service. Option B is incorrect because it doesn't mention the amateur radio service. Option C is incorrect because it doesn't mention the amateur radio service. Option D is correct.

\begin{tcolorbox}[colback=gray!10!white,colframe=black!75!black,title={T1A11}]
    When is willful interference to other amateur radio stations permitted?
    \begin{enumerate}[label=\Alph*),noitemsep]
        \item To stop another amateur station that is breaking the FCC rules
        \item \textbf{At no time}
        \item When making short test transmissions
        \item At any time, stations in the Amateur Radio Service are not protected from willful interference
    \end{enumerate}
\end{tcolorbox}
Willful interference is never permitted under FCC Part 97. This rule ensures that all amateur operators can communicate without disruption.
You might think option A is correct. But interestingly, the FCC does not have the authority to stop another amateur station that is breaking the rules. The FCC can only issue fines and penalties for breaking the rules. The FCC does not have the authority to stop another amateur station from breaking the rules. The FCC can only issue fines and penalties for breaking the rules.  And if you attempt to stop another amateur station from breaking the rules, you are breaking the rules yourself (D'oh!).


