\subsection{Units for Frequency and Power}
\label{subsec:minimal-metric}

When working with radio technology, understanding metric prefixes is essential for dealing with frequencies and power levels. These prefixes help us express very large or very small numbers in a more manageable way. For example, instead of saying "1,000,000 hertz," we can say "1 megahertz" (1 MHz).

\subsubsection*{Frequency Units in Radio Communications}
Frequency is the heartbeat of radio communications. Understanding units like hertz (Hz), kilohertz (kHz), and megahertz (MHz) is crucial because they define the range of frequencies that radios can transmit and receive. For example, the AM radio band operates in the range of 530 to 1700 kHz, while FM radio operates between 88 and 108 MHz.

\subsubsection*{Power Units in Radio}
Power in radio equipment is measured in watts (W). Common prefixes include milliwatts (mW) for very low power devices, watts (W) for typical handheld radios, and kilowatts (kW) for more powerful stations. For instance, 5 watts (5 W) is a common power level for handheld transceivers, while home stations might run 100 watts or more.

\begin{table}[htbp]
    \centering
    \caption{Common Metric Prefixes and Electrical Units}
    \label{tab:power-freq-metric-prefixes}
    \scriptsize
    \begin{tabular}{|c|c|c|c|c|c|c|c|c|c|}
        \hline
        Prefix & Symbol & Scientific & Frequency & Power \\
        & & Notation &  (Hertz) & (Watts) \\
        \hline
        Tera & T & $10^{12}$ & THz & TW \\
        Giga & G & $10^9$ & GHz & GW \\
        Mega & M & $10^6$ & MHz & MW \\
        Kilo & k & $10^3$ & kHz & kW \\
        \hline
        (unit) & - & $10^0$ & Hz & W \\
        \hline
        milli & m & $10^{-3}$ & mHz & mW \\
        micro & $\mu$ & $10^{-6}$ & $\mu$Hz & $\mu$W \\
        \hline
    \end{tabular}
\end{table}

\begin{tcolorbox}[colback=gray!10!white,colframe=black!75!black,title={Common Usage}]
Most frequently used combinations in amateur radio:
\begin{itemize}[noitemsep]
    \item Frequency: Hz, kHz, MHz, GHz
    \item Power: mW, W, kW
\end{itemize}
\end{tcolorbox}
% \begin{table}[htbp]
%     \centering
%     \caption{Unit Conversion Factors}
%     \label{tab:unit-conversion-factors}
%     \begin{tabular}{|c|c|}
%         \hline
%         Conversion & Factor \\
%         \hline
%         1 A to mA & 1,000 \\
%         1 V to kV & 0.001 \\
%         1 W to mW & 1,000 \\
%         1 F to pF & $10^{12}$ \\
%         \hline
%     \end{tabular}
% \end{table}

% \begin{table}[htbp]
%     \centering
%     \caption{Frequency Unit Conversions}
%     \label{tab:frequency-conversions}
%     \begin{tabular}{|c|c|}
%         \hline
%         Conversion & Factor \\
%         \hline
%         1 Hz to kHz & 0.001 \\
%         1 kHz to MHz & 0.001 \\
%         1 MHz to GHz & 0.001 \\
%         \hline
%     \end{tabular}
% \end{table}

\textbf{Questions}


\begin{tcolorbox}[colback=gray!10!white,colframe=black!75!black,title={T5B02}]
    Which is equal to 1,500,000 hertz?
    \begin{enumerate}[label=\Alph*),noitemsep]
        \item \textbf{1500 kHz}
        \item 1500 MHz
        \item 15 GHz
        \item 150 kHz
    \end{enumerate}
\end{tcolorbox}
1,500,000 hertz is equal to 1,500 kHz. The other options are incorrect because they either over or underestimate the conversion factor.


\begin{tcolorbox}[colback=gray!10!white,colframe=black!75!black,title={T5B05}]
    Which is equal to 500 milliwatts?
    \begin{enumerate}[label=\Alph*),noitemsep]
        \item 0.02 watts
        \item \textbf{0.5 watts}
        \item 5 watts
        \item 50 watts
    \end{enumerate}
\end{tcolorbox}
500 milliwatts is equal to 0.5 watts. The other options are incorrect because they either under or overestimate the conversion factor.

\begin{tcolorbox}[colback=gray!10!white,colframe=black!75!black,title={T5B07}]
    Which is equal to 3.525 MHz?
    \begin{enumerate}[label=\Alph*),noitemsep]
        \item 0.003525 kHz
        \item 35.25 kHz
        \item \textbf{3525 kHz}
        \item 3,525,000 kHz
    \end{enumerate}
\end{tcolorbox}
3.525 MHz is equal to 3,525 kHz. The other options are incorrect because they either under or overestimate the conversion factor.

\begin{tcolorbox}[colback=gray!10!white,colframe=black!75!black,title={T5B12}]
    Which is equal to 28400 kHz?
    \begin{enumerate}[label=\Alph*),noitemsep]
        \item 28.400 kHz
        \item 2.800 MHz
        \item 284.00 MHz
        \item \textbf{28.400 MHz}
    \end{enumerate}
\end{tcolorbox}
28,400 kHz is equal to 28.4 MHz. The other options are incorrect because they either under or overestimate the conversion factor.

\begin{tcolorbox}[colback=gray!10!white,colframe=black!75!black,title={T5B13}]
    Which is equal to 2425 MHz?
    \begin{enumerate}[label=\Alph*),noitemsep]
        \item 0.002425 GHz
        \item 24.25 GHz
        \item \textbf{2.425 GHz}
        \item 2425 GHz
    \end{enumerate}
\end{tcolorbox}
2,425 MHz is equal to 2.425 GHz. The other options are incorrect because they either under or overestimate the conversion factor.

\begin{tcolorbox}[colback=gray!10!white,colframe=black!75!black,title={T5C13}]
    What is the abbreviation for kilohertz?
    \begin{enumerate}[label=\Alph*),noitemsep]
        \item KHZ
        \item khz
        \item khZ
        \item \textbf{kHz}
    \end{enumerate}
\end{tcolorbox}
The correct abbreviation for kilohertz is kHz. The lowercase 'k' stands for kilo, and the uppercase 'Hz' stands for hertz. The other options are incorrect due to improper capitalization or letter order.

