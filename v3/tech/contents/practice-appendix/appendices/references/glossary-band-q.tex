\subsection{Glossary, Band Plans, Q Signals}
\label{subsec:glossary-band-q}

\subsubsection*{Glossary: Your Radio Dictionary}

\subsubsection*{Band Plans: The Roadmap of the Airwaves}

\subsubsection*{Q Signals: The Universal Language of Radio}
Q signals are a set of three-letter codes that originated in Morse code communication but are now widely used in voice and digital modes. They’re like the emojis of the radio world—short, efficient, and universally understood. For example, "QTH" means "location," and "QRM" means "interference."

These signals are especially useful when you’re dealing with weak signals or noisy conditions. Instead of saying, "I’m experiencing interference from another station," you can simply say, "QRM," and everyone will know what you mean. It’s a quick and efficient way to communicate, especially when every second counts.

\begin{figure}[h!]
    \centering
    % \includegraphics[width=0.8\textwidth]{q-signals}
    \caption{Common Q signals used in amateur radio.}
    \label{fig:q-signals}
    % Image prompt: Illustration of common Q signals used in amateur radio. The figure should include a list of Q signals with their meanings, such as "QTH: Location," "QRM: Interference," etc.
\end{figure}

\begin{table}[h!]
    \centering
    \begin{tabular}{|c|c|}
        \hline
        \textbf{Q Signal} & \textbf{Meaning} \\
        \hline
        QTH & Location \\
        QRM & Interference \\
        QSL & Acknowledgment \\
        QSY & Change frequency \\
        \hline
    \end{tabular}
    \caption{Common Q signals and their meanings.}
    \label{tab:q-signals}
\end{table}

