\subsection{Phonetic Alphabet, Formulas}
\label{subsec:phonetic-formulas}

When it comes to radio communication, clarity is king. Imagine trying to spell out your call sign over a crackling, static-filled channel. Without a standardized system, "B" could easily be mistaken for "D," and "M" for "N." Enter the \textbf{phonetic alphabet}, a lifesaver in the world of radio communication. By using words like "Bravo" for "B" and "Delta" for "D," we ensure that every letter is understood, no matter the interference. It's like having a universal translator for the alphabet, and trust me, it makes life a whole lot easier.

Now, let's shift gears and talk about something equally important but a bit more mathematical: \textbf{formulas}. If the phonetic alphabet is the language of radio communication, then formulas are the grammar of radio circuit design. They help us understand how signals behave, how to amplify them, and how to filter out the noise. Whether you're calculating the resonant frequency of a circuit or figuring out the impedance of an antenna, formulas are your best friends. They might look intimidating at first, but once you get the hang of them, they’re like the cheat codes to mastering radio technology.

\begin{figure}[htbp]
    \centering
    % \includegraphics[width=0.8\textwidth]{phonetic-alphabet.svg}
    % Diagram showing the phonetic alphabet used in radio communication.
    % The figure should include a clear representation of the NATO phonetic alphabet, with each letter paired with its corresponding word (e.g., A - Alpha, B - Bravo, etc.).
    \caption{Phonetic alphabet used in radio communication.}
    \label{fig:phonetic-alphabet}
\end{figure}

\begin{figure}[htbp]
    \centering
    % \includegraphics[width=0.8\textwidth]{radio-formulas.png}
    % Visual representation of key formulas used in radio circuit design.
    % The figure should include a clean, well-labeled plot or diagram showing formulas such as Ohm's Law, the resonant frequency formula, and the Friis transmission equation.
    \caption{Key formulas used in radio circuit design.}
    \label{fig:radio-formulas}
\end{figure}

\begin{table}[htbp]
    \centering
    \begin{tabular}{|c|c|}
        \hline
        \textbf{Formula} & \textbf{Description} \\
        \hline
        \( V = IR \) & Ohm's Law: Voltage equals current times resistance. \\
        \hline
        \( f_r = \frac{1}{2\pi\sqrt{LC}} \) & Resonant frequency of an LC circuit. \\
        \hline
        \( P_r = P_t \frac{G_t G_r \lambda^2}{(4\pi d)^2} \) & Friis transmission equation for received power. \\
        \hline
    \end{tabular}
    \caption{Key formulas used in radio circuit design.}
    \label{tab:radio-formulas}
\end{table}
