\subsection{Offsets, National Calling Frequencies}
\label{subsec:offsets-national}

In this section, we’ll dive into the fascinating world of repeater frequency offsets and national calling frequencies. These concepts are fundamental to understanding how repeaters work and how operators communicate effectively on amateur radio bands. Let’s break it down step by step, with a sprinkle of humor to keep things light—because radio waves are already serious enough!

\subsubsection*{Repeater Frequency Offset}
Repeaters are like the helpful middlemen of the radio world. They receive a signal on one frequency and retransmit it on another. The difference between these two frequencies is called the \textit{repeater frequency offset}. This offset ensures that the repeater doesn’t interfere with its own transmissions. For example, in the 2 meter band, a common offset is \textbf{±600 kHz}, while in the 70 cm band, it’s typically \textbf{±5 MHz}. These values are like the repeater’s personal space—necessary to avoid chaos. Check out Figure~\ref{fig:repeater-offset} for a visual explanation.

\subsubsection*{National Calling Frequency}
When you’re itching to make a contact and don’t have a specific frequency in mind, the national calling frequency is your go-to spot. For FM simplex operations in the 2 meter band, this is \textbf{146.520 MHz}. Think of it as the radio equivalent of a town square—everyone knows to check there first. Figure~\ref{fig:2m-calling-frequency} highlights this frequency in the 2 meter band.

\subsubsection*{Calling a Station on a Repeater}
If you know the call sign of the station you want to contact, the proper procedure is to say their call sign first, followed by your own. For example, “W1ABC, this is W2XYZ.” This is like saying, “Hey, W1ABC, it’s me, W2XYZ!” Figure~\ref{fig:repeater-calling-procedure} provides a handy flowchart for this process.

\subsubsection*{Responding to a CQ Call}
When you hear someone calling “CQ,” they’re essentially shouting, “Hey, is anyone out there?” To respond, you should say their call sign followed by yours. For example, “W1ABC, this is W2XYZ.” It’s like raising your hand in a crowded room—polite and to the point.

\subsubsection*{On-the-Air Test Transmissions}
Before you start testing your equipment on the air, remember to identify your station. This is a requirement under FCC rules. It’s like saying, “This is W2XYZ, testing, testing!” before you start broadcasting.

\subsubsection*{The Meaning of “CQ”}
The procedural signal “CQ” means “Calling any station.” It’s the universal way to say, “Hey, anyone want to chat?” It’s not a test signal or a call for a specific station—it’s an open invitation.

\subsubsection*{Listening on a Repeater}
If a station is listening on a repeater and looking for a contact, they’ll often say their call sign followed by the word “monitoring.” For example, “W1ABC monitoring.” It’s like saying, “I’m here and ready to talk!”

\subsubsection*{Band Plans}
A band plan is a voluntary guideline for using different modes or activities within an amateur band. It’s like a traffic map for radio frequencies, helping operators avoid collisions and stay organized.

\subsubsection*{Simplex Communication}
Simplex communication means transmitting and receiving on the same frequency. It’s like a walkie-talkie—you can’t talk and listen at the same time. This contrasts with duplex communication, where you can do both simultaneously.

\subsubsection*{Before Calling CQ}
Before you call “CQ,” make sure the frequency is clear, ask if it’s in use, and ensure you’re authorized to use it. It’s like knocking on a door before entering—basic etiquette!

\begin{table}[h]
\centering
\caption{Common Repeater Frequency Offsets}
\label{tab:repeater-offsets}
\begin{tabular}{|l|l|}
\hline
\textbf{Band} & \textbf{Common Offset} \\ \hline
2 meter       & ±600 kHz               \\ \hline
70 cm         & ±5 MHz                 \\ \hline
\end{tabular}
\end{table}

\begin{table}[h]
\centering
\caption{National Calling Frequencies}
\label{tab:national-calling-frequencies}
\begin{tabular}{|l|l|}
\hline
\textbf{Band} & \textbf{National Calling Frequency} \\ \hline
2 meter       & 146.520 MHz                        \\ \hline
70 cm         & 446.000 MHz                        \\ \hline
\end{tabular}
\end{table}

\begin{figure}[h]
\centering
% \includegraphics{repeater-offset-diagram.svg} % Placeholder for the actual image
\caption{Repeater Frequency Offset Diagram}
\label{fig:repeater-offset}
% Prompt: Diagram showing the relationship between repeater transmit and receive frequencies with offset.
\end{figure}

\begin{figure}[h]
\centering
% \includegraphics{2m-calling-frequency.svg} % Placeholder for the actual image
\caption{2 Meter Band National Calling Frequency}
\label{fig:2m-calling-frequency}
% Prompt: Illustration of the 2 meter band with the national calling frequency highlighted.
\end{figure}

\begin{figure}[h]
\centering
% \includegraphics{repeater-calling-procedure.svg} % Placeholder for the actual image
\caption{Procedure for Calling a Station on a Repeater}
\label{fig:repeater-calling-procedure}
% Prompt: Flowchart showing the steps for calling another station on a repeater.
\end{figure}

\subsubsection*{Questions}

\begin{tcolorbox}[colback=gray!10!white,colframe=black!75!black,title={T2A01}]
What is a common repeater frequency offset in the 2 meter band?
\begin{enumerate}[label=\Alph*),noitemsep]
    \item Plus or minus 5 MHz
    \item \textbf{Plus or minus 600 kHz}
    \item Plus or minus 500 kHz
    \item Plus or minus 1 MHz
\end{enumerate}
\end{tcolorbox}
The common repeater frequency offset in the 2 meter band is ±600 kHz. This ensures that the repeater’s transmit and receive frequencies don’t interfere with each other.

\begin{tcolorbox}[colback=gray!10!white,colframe=black!75!black,title={T2A02}]
What is the national calling frequency for FM simplex operations in the 2 meter band?
\begin{enumerate}[label=\Alph*),noitemsep]
    \item \textbf{146.520 MHz}
    \item 145.000 MHz
    \item 432.100 MHz
    \item 446.000 MHz
\end{enumerate}
\end{tcolorbox}
The national calling frequency for FM simplex operations in the 2 meter band is 146.520 MHz. This is the go-to frequency for making general calls.

\begin{tcolorbox}[colback=gray!10!white,colframe=black!75!black,title={T2A03}]
What is a common repeater frequency offset in the 70 cm band?
\begin{enumerate}[label=\Alph*),noitemsep]
    \item \textbf{Plus or minus 5 MHz}
    \item Plus or minus 600 kHz
    \item Plus or minus 500 kHz
    \item Plus or minus 1 MHz
\end{enumerate}
\end{tcolorbox}
In the 70 cm band, the common repeater frequency offset is ±5 MHz. This larger offset is necessary due to the higher frequencies involved.

\begin{tcolorbox}[colback=gray!10!white,colframe=black!75!black,title={T2A04}]
What is an appropriate way to call another station on a repeater if you know the other station's call sign?
\begin{enumerate}[label=\Alph*),noitemsep]
    \item Say “break, break,” then say the station's call sign
    \item \textbf{Say the station's call sign, then identify with your call sign}
    \item Say “CQ” three times, then the other station's call sign
    \item Wait for the station to call CQ, then answer
\end{enumerate}
\end{tcolorbox}
The correct procedure is to say the other station’s call sign first, followed by your own. This ensures clarity and proper identification.

\begin{tcolorbox}[colback=gray!10!white,colframe=black!75!black,title={T2A05}]
How should you respond to a station calling CQ?
\begin{enumerate}[label=\Alph*),noitemsep]
    \item Transmit “CQ” followed by the other station’s call sign
    \item Transmit your call sign followed by the other station’s call sign
    \item \textbf{Transmit the other station’s call sign followed by your call sign}
    \item Transmit a signal report followed by your call sign
\end{enumerate}
\end{tcolorbox}
When responding to a CQ call, you should say the other station’s call sign first, followed by your own. This is the standard protocol.

\begin{tcolorbox}[colback=gray!10!white,colframe=black!75!black,title={T2A06}]
Which of the following is required when making on-the-air test transmissions?
\begin{enumerate}[label=\Alph*),noitemsep]
    \item \textbf{Identify the transmitting station}
    \item Conduct tests only between 10 p.m. and 6 a.m. local time
    \item Notify the FCC of the transmissions
    \item All these choices are correct
\end{enumerate}
\end{tcolorbox}
The only requirement is to identify the transmitting station. The other options are not necessary.

\begin{tcolorbox}[colback=gray!10!white,colframe=black!75!black,title={T2A07}]
What is meant by “repeater offset”?
\begin{enumerate}[label=\Alph*),noitemsep]
    \item \textbf{The difference between a repeater’s transmit and receive frequencies}
    \item The repeater has a time delay to prevent interference
    \item The repeater station identification is done on a separate frequency
    \item The number of simultaneous transmit frequencies used by a repeater
\end{enumerate}
\end{tcolorbox}
Repeater offset refers to the difference between a repeater’s transmit and receive frequencies. This prevents self-interference.

\begin{tcolorbox}[colback=gray!10!white,colframe=black!75!black,title={T2A08}]
What is the meaning of the procedural signal “CQ”?
\begin{enumerate}[label=\Alph*),noitemsep]
    \item Call on the quarter hour
    \item Test transmission, no reply expected
    \item Only the called station should transmit
    \item \textbf{Calling any station}
\end{enumerate}
\end{tcolorbox}
“CQ” means “Calling any station.” It’s a general call to anyone listening.

\begin{tcolorbox}[colback=gray!10!white,colframe=black!75!black,title={T2A09}]
Which of the following indicates that a station is listening on a repeater and looking for a contact?
\begin{enumerate}[label=\Alph*),noitemsep]
    \item “CQ CQ” followed by the repeater’s call sign
    \item \textbf{The station’s call sign followed by the word “monitoring”}
    \item The repeater call sign followed by the station’s call sign
    \item “QSY” followed by your call sign
\end{enumerate}
\end{tcolorbox}
A station listening on a repeater will often say their call sign followed by “monitoring.”

\begin{tcolorbox}[colback=gray!10!white,colframe=black!75!black,title={T2A10}]
What is a band plan, beyond the privileges established by the FCC?
\begin{enumerate}[label=\Alph*),noitemsep]
    \item \textbf{A voluntary guideline for using different modes or activities within an amateur band}
    \item A list of operating schedules
    \item A list of available net frequencies
    \item A plan devised by a club to indicate frequency band usage
\end{enumerate}
\end{tcolorbox}
A band plan is a voluntary guideline for using different modes or activities within an amateur band. It helps operators stay organized.

\begin{tcolorbox}[colback=gray!10!white,colframe=black!75!black,title={T2A11}]
What term describes an amateur station that is transmitting and receiving on the same frequency?
\begin{enumerate}[label=\Alph*),noitemsep]
    \item Full duplex
    \item Diplex
    \item \textbf{Simplex}
    \item Multiplex
\end{enumerate}
\end{tcolorbox}
Simplex communication involves transmitting and receiving on the same frequency.

\begin{tcolorbox}[colback=gray!10!white,colframe=black!75!black,title={T2A12}]
What should you do before calling CQ?
\begin{enumerate}[label=\Alph*),noitemsep]
    \item Listen first to be sure that no one else is using the frequency
    \item Ask if the frequency is in use
    \item Make sure you are authorized to use that frequency
    \item \textbf{All these choices are correct}
\end{enumerate}
\end{tcolorbox}
Before calling CQ, you should listen to ensure the frequency is clear, ask if it’s in use, and confirm you’re authorized to use it.
