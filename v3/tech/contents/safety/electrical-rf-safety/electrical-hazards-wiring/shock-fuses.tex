\subsection{Shock Hazards, Fuses, Circuit Breakers}
\label{subsec:shock-fuses}

\subsubsection*{Shock Hazards}
When dealing with electrical circuits, one of the most immediate dangers is the risk of electrical shock. Electrical current flowing through the human body can cause a variety of injuries, from minor discomfort to severe tissue damage or even death. The effects depend on the amount of current, the path it takes through the body, and the duration of exposure. For example, even a small current can disrupt the electrical functions of cells, leading to involuntary muscle contractions or, in severe cases, cardiac arrest. Heating of tissue is another concern, as high currents can cause burns. Always remember: electricity is a powerful tool, but it demands respect!

\subsubsection*{Fuses}
Fuses are vital safety devices in electrical circuits. They are designed to protect circuits by breaking the flow of current when it exceeds a safe level. Inside a fuse, a thin wire melts when too much current passes through it, effectively stopping the flow of electricity. This prevents overheating, which could otherwise lead to fires or damage to equipment. Think of a fuse as a sacrificial component—it gives its life to save your circuit!

\subsubsection*{Circuit Breakers}
Circuit breakers serve a similar purpose to fuses but with a key difference: they can be reset. When a circuit breaker detects an overload, it trips and interrupts the current flow. Unlike a fuse, which must be replaced after it blows, a circuit breaker can simply be reset once the issue is resolved. This makes circuit breakers more convenient for protecting circuits in homes and businesses. However, both fuses and circuit breakers are essential for maintaining electrical safety.

\subsubsection*{Installing Fuses and Circuit Breakers}
In a 120V AC power circuit, fuses and circuit breakers must be installed in series with the hot conductor. This ensures that the circuit is interrupted if an overload occurs. Installing them in parallel or on the neutral conductor would defeat their purpose, as the circuit would remain live even during an overload. Proper installation is critical to ensuring the safety of both the circuit and the people using it.

\subsubsection*{Measuring High Voltages}
When measuring high voltages with a voltmeter, it’s crucial to use equipment rated for the voltages you’re working with. Using a voltmeter or leads that aren’t rated for high voltages can result in dangerous situations, including electrical shocks or equipment failure. Always double-check the ratings before taking measurements—safety first!

% Figure: Effects of Electrical Current on the Human Body
\begin{figure}[h!]
    \centering
    % \includegraphics[width=0.8\textwidth]{current-effects.svg} % Placeholder for the image
    \caption{Effects of Electrical Current on the Human Body. The diagram illustrates how electrical current flows through tissues and cells, causing heating, muscle contractions, and potential disruption of cellular functions.}
    \label{fig:current-effects}
\end{figure}

% Figure: Fuse in an Electrical Circuit
\begin{figure}[h!]
    \centering
    % \includegraphics[width=0.8\textwidth]{fuse-circuit.svg} % Placeholder for the image
    \caption{Fuse in an Electrical Circuit. The schematic shows how a fuse interrupts current flow during an overload, protecting the circuit from damage.}
    \label{fig:fuse-circuit}
\end{figure}

% Figure: Circuit Breaker Installation in a 120V AC Circuit
\begin{figure}[h!]
    \centering
    % \includegraphics[width=0.8\textwidth]{circuit-breaker.svg} % Placeholder for the image
    \caption{Circuit Breaker Installation in a 120V AC Circuit. The diagram demonstrates the correct placement of a circuit breaker in series with the hot conductor.}
    \label{fig:circuit-breaker}
\end{figure}

% Table: Comparison of Fuses and Circuit Breakers
\begin{table}[h!]
    \centering
    \begin{tabular}{|l|l|l|}
        \hline
        \textbf{Feature} & \textbf{Fuse} & \textbf{Circuit Breaker} \\
        \hline
        Response Time & Fast & Slightly slower \\
        Resetability & Must be replaced & Can be reset \\
        Typical Applications & Small electronics, automotive & Homes, businesses \\
        \hline
    \end{tabular}
    \caption{Comparison of Fuses and Circuit Breakers. This table highlights the key differences between fuses and circuit breakers, including their response time, resetability, and typical applications.}
    \label{tab:fuse-vs-breaker}
\end{table}

\subsubsection*{Questions}
\begin{tcolorbox}[colback=gray!10!white,colframe=black!75!black,title={T0A01}]
    Which of the following is a safety hazard of a 12-volt storage battery?
    \begin{enumerate}[label=\Alph*),noitemsep]
        \item Touching both terminals with the hands can cause electrical shock
        \item \textbf{Shorting the terminals can cause burns, fire, or an explosion}
        \item RF emissions from a nearby transmitter can cause the electrolyte to emit poison gas
        \item All these choices are correct
    \end{enumerate}
\end{tcolorbox}
Shorting the terminals of a 12-volt battery can generate a large current, leading to overheating, burns, or even explosions. While touching the terminals with bare hands is unlikely to cause a shock due to the low voltage, RF emissions causing poison gas is not a realistic hazard.

\begin{tcolorbox}[colback=gray!10!white,colframe=black!75!black,title={T0A02}]
    What health hazard is presented by electrical current flowing through the body?
    \begin{enumerate}[label=\Alph*),noitemsep]
        \item It may cause injury by heating tissue
        \item It may disrupt the electrical functions of cells
        \item It may cause involuntary muscle contractions
        \item \textbf{All these choices are correct}
    \end{enumerate}
\end{tcolorbox}
Electrical current flowing through the body can cause heating of tissues, disrupt cellular functions, and induce involuntary muscle contractions. All these effects are hazardous and can lead to serious injury or death.

\begin{tcolorbox}[colback=gray!10!white,colframe=black!75!black,title={T0A04}]
    What is the purpose of a fuse in an electrical circuit?
    \begin{enumerate}[label=\Alph*),noitemsep]
        \item To prevent power supply ripple from damaging a component
        \item \textbf{To remove power in case of overload}
        \item To limit current to prevent shocks
        \item All these choices are correct
    \end{enumerate}
\end{tcolorbox}
A fuse is designed to interrupt the flow of current in case of an overload, protecting the circuit from damage. It does not prevent power supply ripple or limit current to prevent shocks.

\begin{tcolorbox}[colback=gray!10!white,colframe=black!75!black,title={T0A05}]
    Why should a 5-ampere fuse never be replaced with a 20-ampere fuse?
    \begin{enumerate}[label=\Alph*),noitemsep]
        \item The larger fuse would be likely to blow because it is rated for higher current
        \item The power supply ripple would greatly increase
        \item \textbf{Excessive current could cause a fire}
        \item All these choices are correct
    \end{enumerate}
\end{tcolorbox}
Replacing a 5-ampere fuse with a 20-ampere fuse allows excessive current to flow, which can overheat the circuit and potentially cause a fire. The fuse rating must match the circuit's requirements for safety.

\begin{tcolorbox}[colback=gray!10!white,colframe=black!75!black,title={T0A08}]
    Where should a fuse or circuit breaker be installed in a 120V AC power circuit?
    \begin{enumerate}[label=\Alph*),noitemsep]
        \item \textbf{In series with the hot conductor only}
        \item In series with the hot and neutral conductors
        \item In parallel with the hot conductor only
        \item In parallel with the hot and neutral conductors
    \end{enumerate}
\end{tcolorbox}
Fuses and circuit breakers must be installed in series with the hot conductor to ensure the circuit is interrupted during an overload. Installing them in parallel or on the neutral conductor would not provide proper protection.

\begin{tcolorbox}[colback=gray!10!white,colframe=black!75!black,title={T0A09}]
    What should be done to all external ground rods or earth connections?
    \begin{enumerate}[label=\Alph*),noitemsep]
        \item Waterproof them with silicone caulk or electrical tape
        \item Keep them as far apart as possible
        \item \textbf{Bond them together with heavy wire or conductive strap}
        \item Tune them for resonance on the lowest frequency of operation
    \end{enumerate}
\end{tcolorbox}
Ground rods and earth connections should be bonded together to ensure a low-resistance path to ground. This helps maintain electrical safety and proper grounding.

\begin{tcolorbox}[colback=gray!10!white,colframe=black!75!black,title={T0A12}]
    Which of the following precautions should be taken when measuring high voltages with a voltmeter?
    \begin{enumerate}[label=\Alph*),noitemsep]
        \item Ensure that the voltmeter has very low impedance
        \item \textbf{Ensure that the voltmeter and leads are rated for use at the voltages to be measured}
        \item Ensure that the circuit is grounded through the voltmeter
        \item Ensure that the voltmeter is set to the correct frequency
    \end{enumerate}
\end{tcolorbox}
When measuring high voltages, it is essential to use a voltmeter and leads rated for the voltage levels being measured. Using underrated equipment can result in dangerous situations, including electrical shocks or equipment failure.
