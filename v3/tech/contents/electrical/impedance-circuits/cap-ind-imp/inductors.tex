\subsection{Inductors}
\label{subsec:inductors}

Inductors are fascinating little components that play a crucial role in electrical circuits. At their core, inductors are all about storing energy in a magnetic field. This ability is known as \textbf{inductance}. When current flows through an inductor, it generates a magnetic field around it. The energy stored in this magnetic field can be released back into the circuit when the current changes, making inductors essential in applications like filtering, tuning, and energy storage.

The unit of inductance is the \textbf{henry} (H), named after the American scientist Joseph Henry. One henry is defined as the inductance of a circuit in which an electromotive force of one volt is produced when the current in the circuit changes at a rate of one ampere per second. In simpler terms, the henry tells us how much magnetic field an inductor can generate for a given current. It's like the inductor's way of saying, "Hey, I can store this much energy in my magnetic field!"

% \begin{figure}[h]
%     \centering
%     % \includegraphics[width=0.8\textwidth]{inductor-circuit.svg}
%     \caption{Inductor in a circuit with magnetic field representation.}
%     \label{fig:inductor-circuit}
%     % Diagram showing an inductor in a circuit and its magnetic field. The inductor is represented as a coil, and the magnetic field lines are shown surrounding it, illustrating the energy storage mechanism.
% \end{figure}

\begin{table}[h]
    \centering
    \begin{tabular}{|c|c|}
        \hline
        \textbf{Inductor Type} & \textbf{Inductance (H)} \\
        \hline
        Air-core & 0.001 - 0.1 \\
        Iron-core & 0.1 - 10 \\
        Ferrite-core & 0.01 - 1 \\
        Toroidal & 0.1 - 100 \\
        \hline
    \end{tabular}
    \caption{Comparison of inductance values for various inductor types.}
    \label{tab:inductance-values}
\end{table}

\subsubsection*{Questions}

\begin{tcolorbox}[colback=gray!10!white,colframe=black!75!black,title={T5C03}]
    What describes the ability to store energy in a magnetic field?
    \begin{enumerate}[label=\Alph*),noitemsep]
        \item Admittance
        \item Capacitance
        \item Resistance
        \item \textbf{Inductance}
    \end{enumerate}
\end{tcolorbox}

Inductance is the property that allows an inductor to store energy in a magnetic field. Capacitance, on the other hand, stores energy in an electric field, while resistance dissipates energy as heat. Admittance is a measure of how easily a circuit allows current to flow, but it doesn't directly relate to energy storage in a magnetic field.

\begin{tcolorbox}[colback=gray!10!white,colframe=black!75!black,title={T5C04}]
    What is the unit of inductance?
    \begin{enumerate}[label=\Alph*),noitemsep]
        \item The coulomb
        \item The farad
        \item \textbf{The henry}
        \item The ohm
    \end{enumerate}
\end{tcolorbox}

The henry is the unit of inductance, named after Joseph Henry. The coulomb is the unit of electric charge, the farad is the unit of capacitance, and the ohm is the unit of resistance. So, if you're talking about inductance, you're talking about henries!
