\subsection{Capacitors}
\label{subsec:capacitors}

In this section, we'll dive into the fascinating world of capacitors. If you've ever wondered how energy can be stored in an electric field, you're in the right place. Let's start by understanding the concept of capacitance and its role in storing energy.

\subsubsection*{Capacitance: Storing Energy in an Electric Field}

Capacitance is the ability of a system to store energy in an electric field. Think of it as a tiny energy reservoir that can hold onto electrical energy until it's needed. This is particularly useful in circuits where you might need a quick burst of energy, like in a camera flash or a defibrillator.

The basic idea is that when you apply a voltage across a capacitor, it stores energy in the form of an electric field between its plates. The amount of energy stored depends on the capacitance of the capacitor, which is measured in farads (F). The higher the capacitance, the more energy it can store.

Mathematically, the energy \( E \) stored in a capacitor is given by:
\begin{equation}
    E = \frac{1}{2} C V^2
\end{equation}
where \( C \) is the capacitance and \( V \) is the voltage across the capacitor.

\subsubsection*{The Farad: Unit of Capacitance}

The unit of capacitance is the farad, named after the English physicist Michael Faraday. One farad is defined as the capacitance of a capacitor that stores one coulomb of charge when one volt is applied across it. In practical terms, a farad is a pretty large unit, so you'll often see capacitors measured in microfarads (\( \mu F \)), nanofarads (\( nF \)), or picofarads (\( pF \)).

The farad is crucial in electrical circuits because it determines how much energy a capacitor can store and how quickly it can release that energy. For example, in a timing circuit, the capacitance value will directly affect the time constant of the circuit.


\begin{table}[h]
    \centering
    \begin{tabular}{|c|c|}
        \hline
        \textbf{Capacitor Type} & \textbf{Capacitance Range} \\
        \hline
        Ceramic & 1 pF to 100 \(\mu F\) \\
        Electrolytic & 1 \(\mu F\) to 1 F \\
        Tantalum & 1 \(\mu F\) to 100 \(\mu F\) \\
        Film & 1 pF to 100 \(\mu F\) \\
        \hline
    \end{tabular}
    \caption{Comparison of capacitance values for various capacitor types.}
    \label{tab:capacitance-values}
\end{table}

\subsubsection*{Questions}

\begin{tcolorbox}[colback=gray!10!white,colframe=black!75!black,title={T5C01}]
    What describes the ability to store energy in an electric field?
    \begin{enumerate}[label=\Alph*),noitemsep]
        \item Inductance
        \item Resistance
        \item Tolerance
        \item \textbf{Capacitance}
    \end{enumerate}
\end{tcolorbox}

Capacitance is the correct answer because it directly relates to the ability to store energy in an electric field. Inductance (A) is related to magnetic fields, resistance (B) is about opposing current flow, and tolerance (C) refers to the allowable variation in a component's value.

\begin{tcolorbox}[colback=gray!10!white,colframe=black!75!black,title={T5C02}]
    What is the unit of capacitance?
    \begin{enumerate}[label=\Alph*),noitemsep]
        \item \textbf{The farad}
        \item The ohm
        \item The volt
        \item The henry
    \end{enumerate}
\end{tcolorbox}

The farad (A) is the unit of capacitance, named after Michael Faraday. The ohm (B) is the unit of resistance, the volt (C) is the unit of voltage, and the henry (D) is the unit of inductance.

That's it for capacitors! Now you know how they store energy and why the farad is such an important unit in electrical circuits. Next time you see a capacitor, you'll appreciate the tiny electric field powerhouse it truly is.
