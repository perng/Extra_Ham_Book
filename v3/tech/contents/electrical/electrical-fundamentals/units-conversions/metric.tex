\subsection{Metric Units (k, M, etc.)}
\label{subsec:metric}

When working with radio technology, understanding metric prefixes is like knowing the alphabet before you can read. These prefixes help us express very large or very small numbers in a more manageable way. For example, instead of saying "1,000,000 hertz," we can say "1 megahertz" (1 MHz). This not only saves time but also makes communication clearer, especially when dealing with frequencies, voltages, and currents.

\subsubsection*{Metric Prefixes and Their Significance}
Metric prefixes are like the shorthand of the engineering world. They represent powers of ten, making it easier to work with numbers that would otherwise be unwieldy. For instance, "kilo" (k) stands for $10^3$, "mega" (M) for $10^6$, and "milli" (m) for $10^{-3}$. These prefixes are used across various units, such as volts, amperes, and hertz, to simplify calculations and discussions.

\subsubsection*{Unit Conversions}
Converting between units is a fundamental skill in radio technology. For example, converting amperes to milliamperes involves multiplying by 1,000, since 1 ampere (A) equals 1,000 milliamperes (mA). Similarly, converting hertz to kilohertz involves dividing by 1,000. These conversions are essential when working with different components and systems, ensuring that everything is compatible and functions as intended.

\subsubsection*{Frequency Units in Radio Communications}
Frequency is the heartbeat of radio communications. Understanding units like hertz (Hz), kilohertz (kHz), and megahertz (MHz) is crucial because they define the range of frequencies that radios can transmit and receive. For example, the AM radio band operates in the range of 530 to 1700 kHz, while FM radio operates between 88 and 108 MHz. Knowing how to convert between these units helps in tuning radios and understanding signal propagation.

\subsubsection*{Voltage and Power Conversions}
Voltage and power are also expressed using metric prefixes. For instance, 1 kilovolt (kV) is equal to 1,000 volts (V), and 1 milliwatt (mW) is equal to 0.001 watts (W). These conversions are important when designing circuits or selecting components, as they ensure that the correct voltage and power levels are maintained.

\subsubsection*{Capacitance and Picofarads}
Capacitance, measured in farads (F), often involves very small values, such as picofarads (pF). One picofarad is equal to $10^{-12}$ farads. Understanding these units is essential when working with capacitors, which are used in tuning circuits and filtering signals.

\subsubsection*{Practical Applications of Frequency Conversions}
Frequency conversions are not just theoretical; they have practical applications in radio technology. For example, converting MHz to kHz is necessary when working with different radio bands. Similarly, converting GHz to MHz is important in microwave communications. These conversions ensure that signals are transmitted and received at the correct frequencies, avoiding interference and ensuring clear communication.

% \begin{figure}[htbp]
%     \centering
%     % \includegraphics[width=0.8\textwidth]{metric-prefixes}
%     \caption{Metric Prefixes and Their Values}
%     \label{fig:metric-prefixes}
%     % Diagram showing the relationship between different metric prefixes (kilo, mega, milli, micro, etc.) and their corresponding values.
% \end{figure}

% \begin{figure}[htbp]
%     \centering
%     % \includegraphics[width=0.8\textwidth]{unit-conversion-flowchart}
%     \caption{Unit Conversion Flowchart}
%     \label{fig:unit-conversion-flowchart}
%     % Flowchart illustrating the process of converting between different units of electrical measurements.
% \end{figure}

% \begin{figure}[htbp]
%     \centering
%     % \includegraphics[width=0.8\textwidth]{frequency-units}
%     \caption{Frequency Unit Relationships}
%     \label{fig:frequency-units}
%     % Graph showing the relationship between frequency units (Hz, kHz, MHz, GHz).
% \end{figure}

\begin{table}[htbp]
    \centering
    \caption{Common Metric Prefixes and Electrical Units}
    \label{tab:metric-prefixes}
    \footnotesize
    \begin{tabular}{|c|c|c|c|c|c|c|c|c|}
        \hline
        Prefix & Symbol & Scientific & Current & Voltage & Resistance & Capacitance & Inductance & Frequency \\
        & & Notation & (Amperes) & (Volts) & (Ohms) & (Farads) & (Henries) & (Hertz) \\
        \hline
        Tera & T & $10^{12}$ & TA & TV & T$\Omega$ & TF & TH & THz \\
        Giga & G & $10^9$ & GA & GV & G$\Omega$ & GF & GH & GHz \\
        Mega & M & $10^6$ & MA & MV & M$\Omega$ & MF & MH & MHz \\
        Kilo & k & $10^3$ & kA & kV & k$\Omega$ & kF & kH & kHz \\
        \hline
        (unit) & - & $10^0$ & A & V & $\Omega$ & F & H & Hz \\
        \hline
        milli & m & $10^{-3}$ & mA & mV & m$\Omega$ & mF & mH & mHz \\
        micro & $\mu$ & $10^{-6}$ & $\mu$A & $\mu$V & $\mu\Omega$ & $\mu$F & $\mu$H & $\mu$Hz \\
        nano & n & $10^{-9}$ & nA & nV & n$\Omega$ & nF & nH & nHz \\
        pico & p & $10^{-12}$ & pA & pV & p$\Omega$ & pF & pH & pHz \\
        femto & f & $10^{-15}$ & fA & fV & f$\Omega$ & fF & fH & fHz \\
        \hline
    \end{tabular}
\end{table}

\begin{tcolorbox}[colback=gray!10!white,colframe=black!75!black,title={Common Usage}]
Most frequently used combinations in amateur radio:
\begin{itemize}
    \item Current: mA, A
    \item Voltage: mV, V, kV
    \item Resistance: $\Omega$, k$\Omega$, M$\Omega$
    \item Capacitance: pF, nF, $\mu$F
    \item Inductance: $\mu$H, mH, H
    \item Frequency: Hz, kHz, MHz, GHz
\end{itemize}
\end{tcolorbox}

% \begin{table}[htbp]
%     \centering
%     \caption{Unit Conversion Factors}
%     \label{tab:unit-conversion-factors}
%     \begin{tabular}{|c|c|}
%         \hline
%         Conversion & Factor \\
%         \hline
%         1 A to mA & 1,000 \\
%         1 V to kV & 0.001 \\
%         1 W to mW & 1,000 \\
%         1 F to pF & $10^{12}$ \\
%         \hline
%     \end{tabular}
% \end{table}

% \begin{table}[htbp]
%     \centering
%     \caption{Frequency Unit Conversions}
%     \label{tab:frequency-conversions}
%     \begin{tabular}{|c|c|}
%         \hline
%         Conversion & Factor \\
%         \hline
%         1 Hz to kHz & 0.001 \\
%         1 kHz to MHz & 0.001 \\
%         1 MHz to GHz & 0.001 \\
%         \hline
%     \end{tabular}
% \end{table}

\textbf{Questions}

\begin{tcolorbox}[colback=gray!10!white,colframe=black!75!black,title={T5B01}]
    How many milliamperes is 1.5 amperes?
    \begin{enumerate}[label=\Alph*),noitemsep]
        \item 15 milliamperes
        \item 150 milliamperes
        \item \textbf{1500 milliamperes}
        \item 15,000 milliamperes
    \end{enumerate}
\end{tcolorbox}
To convert amperes to milliamperes, multiply by 1,000. Therefore, 1.5 amperes is equal to 1,500 milliamperes. The other options are incorrect because they either under or overestimate the conversion factor.

\begin{tcolorbox}[colback=gray!10!white,colframe=black!75!black,title={T5B02}]
    Which is equal to 1,500,000 hertz?
    \begin{enumerate}[label=\Alph*),noitemsep]
        \item \textbf{1500 kHz}
        \item 1500 MHz
        \item 15 GHz
        \item 150 kHz
    \end{enumerate}
\end{tcolorbox}
1,500,000 hertz is equal to 1,500 kHz. The other options are incorrect because they either over or underestimate the conversion factor.

\begin{tcolorbox}[colback=gray!10!white,colframe=black!75!black,title={T5B03}]
    Which is equal to one kilovolt?
    \begin{enumerate}[label=\Alph*),noitemsep]
        \item One one-thousandth of a volt
        \item One hundred volts
        \item \textbf{One thousand volts}
        \item One million volts
    \end{enumerate}
\end{tcolorbox}
One kilovolt is equal to 1,000 volts. The other options are incorrect because they either under or overestimate the conversion factor.

\begin{tcolorbox}[colback=gray!10!white,colframe=black!75!black,title={T5B04}]
    Which is equal to one microvolt?
    \begin{enumerate}[label=\Alph*),noitemsep]
        \item \textbf{One one-millionth of a volt}
        \item One million volts
        \item One thousand kilovolts
        \item One one-thousandth of a volt
    \end{enumerate}
\end{tcolorbox}
One microvolt is equal to one one-millionth of a volt. The other options are incorrect because they either over or underestimate the conversion factor.

\begin{tcolorbox}[colback=gray!10!white,colframe=black!75!black,title={T5B05}]
    Which is equal to 500 milliwatts?
    \begin{enumerate}[label=\Alph*),noitemsep]
        \item 0.02 watts
        \item \textbf{0.5 watts}
        \item 5 watts
        \item 50 watts
    \end{enumerate}
\end{tcolorbox}
500 milliwatts is equal to 0.5 watts. The other options are incorrect because they either under or overestimate the conversion factor.

\begin{tcolorbox}[colback=gray!10!white,colframe=black!75!black,title={T5B06}]
    Which is equal to 3000 milliamperes?
    \begin{enumerate}[label=\Alph*),noitemsep]
        \item 0.003 amperes
        \item 0.3 amperes
        \item 3,000,000 amperes
        \item \textbf{3 amperes}
    \end{enumerate}
\end{tcolorbox}
3000 milliamperes is equal to 3 amperes. The other options are incorrect because they either under or overestimate the conversion factor.

\begin{tcolorbox}[colback=gray!10!white,colframe=black!75!black,title={T5B07}]
    Which is equal to 3.525 MHz?
    \begin{enumerate}[label=\Alph*),noitemsep]
        \item 0.003525 kHz
        \item 35.25 kHz
        \item \textbf{3525 kHz}
        \item 3,525,000 kHz
    \end{enumerate}
\end{tcolorbox}
3.525 MHz is equal to 3,525 kHz. The other options are incorrect because they either under or overestimate the conversion factor.

\begin{tcolorbox}[colback=gray!10!white,colframe=black!75!black,title={T5B08}]
    Which is equal to 1,000,000 picofarads?
    \begin{enumerate}[label=\Alph*),noitemsep]
        \item 0.001 microfarads
        \item \textbf{1 microfarad}
        \item 1000 microfarads
        \item 1,000,000,000 microfarads
    \end{enumerate}
\end{tcolorbox}
1,000,000 picofarads is equal to 1 microfarad. The other options are incorrect because they either under or overestimate the conversion factor.

\begin{tcolorbox}[colback=gray!10!white,colframe=black!75!black,title={T5B12}]
    Which is equal to 28400 kHz?
    \begin{enumerate}[label=\Alph*),noitemsep]
        \item 28.400 kHz
        \item 2.800 MHz
        \item 284.00 MHz
        \item \textbf{28.400 MHz}
    \end{enumerate}
\end{tcolorbox}
28,400 kHz is equal to 28.4 MHz. The other options are incorrect because they either under or overestimate the conversion factor.

\begin{tcolorbox}[colback=gray!10!white,colframe=black!75!black,title={T5B13}]
    Which is equal to 2425 MHz?
    \begin{enumerate}[label=\Alph*),noitemsep]
        \item 0.002425 GHz
        \item 24.25 GHz
        \item \textbf{2.425 GHz}
        \item 2425 GHz
    \end{enumerate}
\end{tcolorbox}
2,425 MHz is equal to 2.425 GHz. The other options are incorrect because they either under or overestimate the conversion factor.
