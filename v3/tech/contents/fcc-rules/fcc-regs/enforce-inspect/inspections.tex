\subsection{Inspections}
\label{subsec:inspections}

Alright, let's dive into the world of FCC inspections! If you're an amateur radio operator, you might be wondering, "When can the FCC show up at my door and ask to see my station?" Well, the answer is pretty straightforward, but let's break it down.

\subsubsection*{FCC Inspection Requirements}
The Federal Communications Commission (FCC) has the authority to inspect your amateur radio station and its records. According to FCC regulations, your station and its records must be available for inspection \textbf{at any time upon request by an FCC representative}. That's right—no ten-day notice, no written notification, and definitely no need for a warrant. If an FCC representative knocks on your door and asks to see your station, you better be ready to show them around.

\subsubsection*{Station Records}
Now, what exactly are these "records" that the FCC might want to see? Well, as an amateur radio operator, you're required to maintain certain records, such as your station log, which includes details of your transmissions, frequencies used, and any contacts made. These records should be organized and easily accessible, so when the FCC comes knocking, you can quickly pull them up without breaking a sweat.

\subsubsection*{Consequences of Non-Compliance}
Failing to comply with an FCC inspection request can have serious consequences. If you refuse to allow an inspection or fail to provide the necessary records, you could face fines, license revocation, or even criminal charges. So, it's in your best interest to keep your station and records in tip-top shape and ready for inspection at any time.

% \begin{figure}[h]
%     \centering
%     % \includegraphics[width=0.8\textwidth]{fcc-inspection-process}
%     \caption{Process of FCC Inspection}
%     \label{fig:fcc-inspection-process}
%     % Diagram showing the process of an FCC inspection of an amateur radio station. The diagram should include steps such as FCC representative arrival, station inspection, record review, and conclusion of inspection.
% \end{figure}

\begin{table}[h]
    \footnotesize
    \centering
    \begin{tabular}{|l|l|}
        \hline
        \textbf{Requirement} & \textbf{Description} \\
        \hline
        Availability & Station and records must be available at any time upon request by an FCC representative. \\
        \hline
        Records & Maintain a station log with details of transmissions, frequencies used, and contacts made. \\
        \hline
        Consequences & Non-compliance can result in fines, license revocation, or criminal charges. \\
        \hline
    \end{tabular}
    \caption{FCC Inspection Requirements Summary}
    \label{tab:fcc-inspection-summary}
\end{table}

\subsubsection{Questions}

\begin{tcolorbox}[colback=gray!10!white,colframe=black!75!black,title={T1F01}]
    When must the station and its records be available for FCC inspection?
    \begin{enumerate}[label=\Alph*),noitemsep]
        \item At any time ten days after notification by the FCC of such an inspection
        \item \textbf{At any time upon request by an FCC representative}
        \item At any time after written notification by the FCC of such inspection
        \item Only when presented with a valid warrant by an FCC official or government agent
    \end{enumerate}
\end{tcolorbox}

The FCC can request to inspect your station and its records at any time, without prior notice. This is a key requirement under FCC regulations to ensure compliance with amateur radio operating rules. Options A, C, and D are incorrect because they suggest that the FCC needs to provide some form of notice or warrant, which is not the case. The FCC has the authority to conduct inspections without any prior notification or warrant.

So, keep your station ready, and remember, the FCC could show up at any time!
