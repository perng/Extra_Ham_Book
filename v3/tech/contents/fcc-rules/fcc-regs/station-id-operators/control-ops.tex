\subsection{Control Operator Duties}
\label{subsec:control-ops}

The control operator is the backbone of any amateur radio station. Think of them as the captain of a ship—without them, the station would be adrift in a sea of frequencies. The control operator is responsible for ensuring that the station operates within the rules and regulations set by the FCC. This includes making sure that transmissions are made on the correct frequencies, that the station is properly identified, and that all communications are conducted in a manner consistent with the amateur radio service.

\subsubsection*{Automatic Control}
Automatic control is like having a robot co-pilot. It allows the station to operate without a human control operator present, but only under specific conditions. For example, repeaters often operate under automatic control, retransmitting signals without the need for constant human oversight. However, even in automatic control, the control operator is still responsible for the station's operation.

\subsubsection*{Remote Control}
Remote control is the next level of convenience. It allows the control operator to manage the station from a distance, often over the internet. This is particularly useful for operators who want to control their station from home while the equipment is located elsewhere. However, the control operator must still be present at the control point, even if that point is miles away from the actual transmitting equipment.

\subsubsection*{Station Control Point}
The station control point is where the magic happens. It's the location where the control operator performs their duties. This could be a physical location, like a shack filled with radios and antennas, or a virtual one, like a computer connected to the station via the internet. The control point is crucial because it's where the control operator ensures that everything is running smoothly.

\subsubsection*{License Class Privileges}
The control operator's license class is like a key that unlocks certain frequency privileges. The higher the license class, the more frequencies the operator can access. For example, a Technician class licensee has limited frequency privileges compared to an Amateur Extra class licensee. This means that the control operator's license class directly determines the station's transmitting frequency privileges.

\subsubsection*{Station Licensee vs. Control Operator}
The station licensee and the control operator are like the owner and the captain of a ship. The licensee owns the station, but the control operator is responsible for its day-to-day operation. Both have shared responsibilities, especially when it comes to ensuring that the station complies with FCC regulations. If the control operator is not the licensee, both parties are responsible for the station's proper operation.

% \begin{figure}[h]
%     \centering
%     % \includegraphics[width=0.8\textwidth]{figures/licensee-control-operator}
%     \caption{Relationship Between Licensee and Control Operator}
%     \label{fig:licensee-control-operator}
%     % Diagram illustrating the relationship between the station licensee and the control operator.
% \end{figure}

% \begin{figure}[h]
%     \centering
%     % \includegraphics[width=0.8\textwidth]{figures/control-operator-flowchart}
%     \caption{Control Operator Decision Flowchart}
%     \label{fig:control-operator-flowchart}
%     % Flowchart showing the decision-making process for determining control operator responsibilities.
% \end{figure}

\begin{table}[h]
    \centering
    \begin{tabular}{|l|l|}
        \hline
        \textbf{Duty} & \textbf{Responsibility} \\
        \hline
        Frequency Compliance & Ensure transmissions are on authorized frequencies \\
        Station Identification & Properly identify the station during transmissions \\
        Operational Oversight & Monitor and manage station operations \\
        \hline
    \end{tabular}
    \caption{Control Operator Duties and Responsibilities}
    \label{tab:control-operator-duties}
\end{table}

\subsubsection*{Questions}

\begin{tcolorbox}[colback=gray!10!white,colframe=black!75!black,title={T1E01}]
    When may an amateur station transmit without a control operator?
    \begin{enumerate}[label=\Alph*),noitemsep]
        \item When using automatic control, such as in the case of a repeater
        \item When the station licensee is away and another licensed amateur is using the station
        \item When the transmitting station is an auxiliary station
        \item \textbf{Never}
    \end{enumerate}
\end{tcolorbox}
An amateur station must always have a control operator present, either physically or through automatic control. The control operator is responsible for ensuring that the station operates within FCC regulations, and this responsibility cannot be delegated away.

\begin{tcolorbox}[colback=gray!10!white,colframe=black!75!black,title={T1E02}]
    Who may be the control operator of a station communicating through an amateur satellite or space station?
    \begin{enumerate}[label=\Alph*),noitemsep]
        \item Only an Amateur Extra Class operator
        \item A General class or higher licensee with a satellite operator certification
        \item Only an Amateur Extra Class operator who is also an AMSAT member
        \item \textbf{Any amateur allowed to transmit on the satellite uplink frequency}
    \end{enumerate}
\end{tcolorbox}
Any licensed amateur who is authorized to transmit on the satellite's uplink frequency can be the control operator. There are no additional certifications or memberships required beyond the standard amateur radio license.

\begin{tcolorbox}[colback=gray!10!white,colframe=black!75!black,title={T1E03}]
    Who must designate the station control operator?
    \begin{enumerate}[label=\Alph*),noitemsep]
        \item \textbf{The station licensee}
        \item The FCC
        \item The frequency coordinator
        \item Any licensed operator
    \end{enumerate}
\end{tcolorbox}
The station licensee is responsible for designating the control operator. This ensures that the person operating the station is qualified and authorized to do so.

\begin{tcolorbox}[colback=gray!10!white,colframe=black!75!black,title={T1E04}]
    What determines the transmitting frequency privileges of an amateur station?
    \begin{enumerate}[label=\Alph*),noitemsep]
        \item The frequency authorized by the frequency coordinator
        \item The frequencies printed on the license grant
        \item The highest class of operator license held by anyone on the premises
        \item \textbf{The class of operator license held by the control operator}
    \end{enumerate}
\end{tcolorbox}
The control operator's license class determines the station's transmitting frequency privileges. This ensures that the station operates within the limits of the control operator's authorization.

\begin{tcolorbox}[colback=gray!10!white,colframe=black!75!black,title={T1E05}]
    What is an amateur station’s control point?
    \begin{enumerate}[label=\Alph*),noitemsep]
        \item The location of the station’s transmitting antenna
        \item The location of the station’s transmitting apparatus
        \item \textbf{The location at which the control operator function is performed}
        \item The mailing address of the station licensee
    \end{enumerate}
\end{tcolorbox}
The control point is where the control operator performs their duties. This could be a physical location or a virtual one, depending on how the station is set up.

\begin{tcolorbox}[colback=gray!10!white,colframe=black!75!black,title={T1E06}]
    When, under normal circumstances, may a Technician class licensee be the control operator of a station operating in an Amateur Extra Class band segment?
    \begin{enumerate}[label=\Alph*),noitemsep]
        \item \textbf{At no time}
        \item When designated as the control operator by an Amateur Extra Class licensee
        \item As part of a multi-operator contest team
        \item When using a club station whose trustee holds an Amateur Extra Class license
    \end{enumerate}
\end{tcolorbox}
A Technician class licensee does not have the privileges to operate in Amateur Extra Class band segments, regardless of the circumstances.

\begin{tcolorbox}[colback=gray!10!white,colframe=black!75!black,title={T1E07}]
    When the control operator is not the station licensee, who is responsible for the proper operation of the station?
    \begin{enumerate}[label=\Alph*),noitemsep]
        \item All licensed amateurs who are present at the operation
        \item Only the station licensee
        \item Only the control operator
        \item \textbf{The control operator and the station licensee}
    \end{enumerate}
\end{tcolorbox}
Both the control operator and the station licensee share responsibility for the station's proper operation. This ensures that both parties are accountable for compliance with FCC regulations.

\begin{tcolorbox}[colback=gray!10!white,colframe=black!75!black,title={T1E08}]
    Which of the following is an example of automatic control?
    \begin{enumerate}[label=\Alph*),noitemsep]
        \item \textbf{Repeater operation}
        \item Controlling a station over the internet
        \item Using a computer or other device to send CW automatically
        \item Using a computer or other device to identify automatically
    \end{enumerate}
\end{tcolorbox}
Repeater operation is a classic example of automatic control, where the station operates without direct human intervention.

\begin{tcolorbox}[colback=gray!10!white,colframe=black!75!black,title={T1E09}]
    Which of the following are required for remote control operation?
    \begin{enumerate}[label=\Alph*),noitemsep]
        \item The control operator must be at the control point
        \item A control operator is required at all times
        \item The control operator must indirectly manipulate the controls
        \item \textbf{All these choices are correct}
    \end{enumerate}
\end{tcolorbox}
All of the listed requirements are necessary for remote control operation. The control operator must be present at the control point, even if that point is remote from the transmitting equipment.

\begin{tcolorbox}[colback=gray!10!white,colframe=black!75!black,title={T1E10}]
    Which of the following is an example of remote control as defined in Part 97?
    \begin{enumerate}[label=\Alph*),noitemsep]
        \item Repeater operation
        \item \textbf{Operating the station over the internet}
        \item Controlling a model aircraft, boat, or car by amateur radio
        \item All these choices are correct
    \end{enumerate}
\end{tcolorbox}
Operating the station over the internet is a clear example of remote control, where the control operator manages the station from a distance.

\begin{tcolorbox}[colback=gray!10!white,colframe=black!75!black,title={T1E11}]
    Who does the FCC presume to be the control operator of an amateur station, unless documentation to the contrary is in the station records?
    \begin{enumerate}[label=\Alph*),noitemsep]
        \item The station custodian
        \item The third party participant
        \item The person operating the station equipment
        \item \textbf{The station licensee}
    \end{enumerate}
\end{tcolorbox}
The FCC presumes that the station licensee is the control operator unless there is documentation indicating otherwise. This places the responsibility for the station's operation squarely on the licensee.
