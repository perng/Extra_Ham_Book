\subsection{Station ID Requirements}
\label{subsec:id-requirements}

In amateur radio communications, identifying your station is not just a formality—it's a legal requirement. The FCC has specific rules about how and when you must transmit your call sign, and understanding these rules is crucial for staying compliant. Let's dive into the details.

\subsubsection*{Tactical Call Signs}
Tactical call signs, like "Race Headquarters," are often used in events or emergencies to simplify communication. However, they don't replace your FCC-assigned call sign. You still need to identify with your official call sign at specific intervals. Think of it as wearing a name tag at a party—it helps everyone know who's talking!

\subsubsection*{Call Sign Identification Frequency}
The FCC requires you to transmit your call sign at least every 10 minutes during a communication and at the end of each transmission. This ensures that your station is easily identifiable, even if you're using tactical call signs. For example, if you're chatting away about the latest antenna design, don't forget to drop your call sign every 10 minutes—like a friendly reminder of who's on the air.

\subsubsection*{Language for Identification}
When operating in a phone sub-band, the FCC mandates that you use English for station identification. This rule helps maintain clarity and consistency across the airwaves. So, while you might be fluent in Elvish, stick to English when identifying your station.  A dōr ammen?

\subsubsection*{Phone Emission Identification}
For phone transmissions, you can identify your station using either voice (phone emission) or Morse code (CW). This flexibility allows you to choose the method that works best for you. Just make sure your call sign is clear and unmistakable.

\subsubsection*{Self-Assigned Indicators}
Self-assigned indicators, like "KL7CC/W3," are acceptable for phone transmissions. These indicators can provide additional information about your location or operating conditions. The FCC allows various formats, so feel free to get creative—just don't forget the basics!

\subsubsection*{Phonetic Alphabet Usage}
While not mandatory, the FCC encourages the use of the phonetic alphabet for station identification. It can make your call sign easier to understand, especially in noisy or crowded bands. So, if you're feeling fancy, go ahead and say "Alpha Bravo Charlie" instead of "ABC."The NATO phonetic alphabet is widely used in amateur radio.

\begin{table}[h]
    \centering
    \footnotesize
    \begin{tabular}{|c|l||c|l||c|l|}
        \hline
        \textbf{Letter} & \textbf{Word} & \textbf{Letter} & \textbf{Word} & \textbf{Letter} & \textbf{Word} \\
        \hline
        A & Alpha & J & Juliet & S & Sierra \\
        B & Bravo & K & Kilo & T & Tango \\
        C & Charlie & L & Lima & U & Uniform \\
        D & Delta & M & Mike & V & Victor \\
        E & Echo & N & November & W & Whiskey \\
        F & Foxtrot & O & Oscar & X & X-ray \\
        G & Golf & P & Papa & Y & Yankee \\
        H & Hotel & Q & Quebec & Z & Zulu \\
        I & India & R & Romeo & & \\
        \hline
    \end{tabular}
    \caption{NATO Phonetic Alphabet}
    \label{tab:phonetic-alphabet}
\end{table}

For example, the call sign "K1ABC" would be pronounced as "Kilo One Alpha Bravo Charlie". This standardized system helps prevent confusion between similar-sounding letters like B/D/E or M/N.



\begin{figure}[h]
    \centering
    % \includegraphics[width=0.8\textwidth]{figures/call-sign-timing.svg}
    \caption{Timing of Call Sign Identification}
    \label{fig:call-sign-timing}
    % Diagram showing the timing of call sign identification during a transmission.
    % The figure should include a timeline with markers at 10-minute intervals, indicating when the call sign should be transmitted.
\end{figure}

\begin{table}[h]
    \centering
    \begin{tabular}{|l|l|}
        \hline
        \textbf{Rule} & \textbf{Description} \\
        \hline
        Identification Frequency & Every 10 minutes and at the end of each transmission \\
        Language & English for phone sub-bands \\
        Emission Types & Phone or CW \\
        Self-Assigned Indicators & Acceptable in various formats \\
        Phonetic Alphabet & Encouraged but not required \\
        \hline
    \end{tabular}
    \caption{FCC Rules for Station Identification}
    \label{tab:station-id-rules}
\end{table}

\subsection*{Questions}
\begin{tcolorbox}[colback=gray!10!white,colframe=black!75!black,title={T1F02}]
    How often must you identify with your FCC-assigned call sign when using tactical call signs such as “Race Headquarters”?
    \begin{enumerate}[label=\Alph*),noitemsep]
        \item Never, the tactical call is sufficient
        \item Once during every hour
        \item \textbf{At the end of each communication and every ten minutes during a communication}
        \item At the end of every transmission
    \end{enumerate}
\end{tcolorbox}
The FCC requires you to identify with your call sign at the end of each communication and every ten minutes during a communication, even when using tactical call signs. This ensures that your station is properly identified.

\begin{tcolorbox}[colback=gray!10!white,colframe=black!75!black,title={T1F03}]
    When are you required to transmit your assigned call sign?
    \begin{enumerate}[label=\Alph*),noitemsep]
        \item At the beginning of each contact, and every 10 minutes thereafter
        \item At least once during each transmission
        \item At least every 15 minutes during and at the end of a communication
        \item \textbf{At least every 10 minutes during and at the end of a communication}
    \end{enumerate}
\end{tcolorbox}
You must transmit your call sign at least every 10 minutes during a communication and at the end of the communication. This rule ensures that your station is identifiable throughout the transmission.

\begin{tcolorbox}[colback=gray!10!white,colframe=black!75!black,title={T1F04}]
    What language may you use for identification when operating in a phone sub-band?
    \begin{enumerate}[label=\Alph*),noitemsep]
        \item Any language recognized by the United Nations
        \item Any language recognized by the ITU
        \item \textbf{English}
        \item English, French, or Spanish
    \end{enumerate}
\end{tcolorbox}
When operating in a phone sub-band, you must use English for station identification. This rule helps maintain clarity and consistency in communications.

\begin{tcolorbox}[colback=gray!10!white,colframe=black!75!black,title={T1F05}]
    What method of call sign identification is required for a station transmitting phone signals?
    \begin{enumerate}[label=\Alph*),noitemsep]
        \item Send the call sign followed by the indicator RPT
        \item \textbf{Send the call sign using a CW or phone emission}
        \item Send the call sign followed by the indicator R
        \item Send the call sign using only a phone emission
    \end{enumerate}
\end{tcolorbox}
For phone transmissions, you can identify your station using either voice (phone emission) or Morse code (CW). This flexibility allows you to choose the method that works best for you.

\begin{tcolorbox}[colback=gray!10!white,colframe=black!75!black,title={T1F06}]
    Which of the following self-assigned indicators are acceptable when using a phone transmission?
    \begin{enumerate}[label=\Alph*),noitemsep]
        \item KL7CC stroke W3
        \item KL7CC slant W3
        \item KL7CC slash W3
        \item \textbf{All these choices are correct}
    \end{enumerate}
\end{tcolorbox}
All the listed self-assigned indicators are acceptable for phone transmissions. The FCC allows various formats, so you can choose the one that best suits your needs.

\begin{tcolorbox}[colback=gray!10!white,colframe=black!75!black,title={T1A03}]
    What do the FCC rules state regarding the use of a phonetic alphabet for station identification in the Amateur Radio Service?
    \begin{enumerate}[label=\Alph*),noitemsep]
        \item It is required when transmitting emergency messages
        \item \textbf{It is encouraged}
        \item It is required when in contact with foreign stations
        \item All these choices are correct
    \end{enumerate}
\end{tcolorbox}
The FCC encourages the use of the phonetic alphabet for station identification, as it can make your call sign easier to understand. However, it is not mandatory.
