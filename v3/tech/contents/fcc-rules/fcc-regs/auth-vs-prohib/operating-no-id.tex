\subsection{Operating Without Identification}
\label{subsec:operating-no-id}

In the world of amateur radio, there are specific rules and regulations that govern how and when a station must identify itself. However, there is a particular scenario under FCC regulations where an amateur station may transmit \emph{without} identifying on the air: telecommand of certain model craft. Let's examine this exemption and also clarify why other commonly mentioned cases still require identification under normal circumstances.

\subsubsection*{When Identification is Not Required}
According to FCC Part 97.215, an amateur station transmitting signals to \emph{telecommand a model craft} (e.g., model airplanes or boats) \textbf{is not} required to identify, as long as:
\begin{itemize}[noitemsep]    \item The station controls the model craft on frequencies specifically authorized for telecommand.
    \item The transmitter power does not exceed 1~watt.
\end{itemize}
Under these conditions, the station does not need to transmit its callsign.

\subsubsection*{Commonly Misunderstood Cases}
While the following situations are often cited as ``ID exemptions,'' they generally remain subject to the normal FCC requirement to identify every 10 minutes during a communication and at the end of the final transmission:

\begin{itemize}[noitemsep]    \item \textbf{Brief Transmissions for Station Adjustments}: Even when making quick adjustments or tests, you must still provide station identification if you transmit beyond the short, permissible interval or if your transmissions include actual voice/audio.  
    \item \textbf{Unmodulated Transmissions}: When testing or calibrating with a carrier, the station must identify if transmissions continue for longer than the permitted interval or if you modulate at any point.  
    \item \textbf{Low-Power Transmissions}: Running below 1~watt does not, by itself, eliminate the ID requirement unless it meets the telecommand exemption for model craft.
\end{itemize}

\begin{tcolorbox}[colback=gray!10!white,colframe=black!75!black,title={T1D11}]
    When may an amateur station transmit without identifying on the air?
    \begin{enumerate}[label=\Alph*),noitemsep]
        \item When the transmissions are of a brief nature to make station adjustments
        \item When the transmissions are unmodulated
        \item When the transmitted power level is below 1 watt
        \item \textbf{When transmitting signals to control model craft}
    \end{enumerate}
\end{tcolorbox}

As reflected in the current FCC regulations, transmitting signals to control model craft (within the specified power limit and frequency allocations) is the primary case where identification is explicitly exempted. The other cases listed (brief transmissions, unmodulated signals, or low power) generally still require station identification unless they specifically qualify under the telecommand-of-model-craft rules.
