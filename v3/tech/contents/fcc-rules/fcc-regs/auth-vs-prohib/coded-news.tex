\subsection{Coded Messages, News Gathering}
\label{subsec:coded-news}

In this section, we'll dive into the fascinating world of coded messages and news gathering in amateur radio. Don't worry, we won't be cracking any secret codes or breaking news stories, but we will explore the rules and conditions under which these activities are permitted. So, grab your decoder rings and let's get started!

\subsubsection*{Coded Messages}
Coded messages in amateur radio are like the secret handshakes of the radio world—they're not for everyone, and there are strict rules about when and how they can be used. According to FCC regulations, transmitting messages encoded to obscure their meaning is generally a big no-no. However, there are a few exceptions to this rule. For example, you can use coded messages when transmitting control commands to space stations or radio control craft. This makes sense, right? You wouldn't want your drone to start doing loop-de-loops because someone accidentally sent it the wrong signal!

\subsubsection*{News Gathering}
Now, let's talk about news gathering. Amateur radio operators are not typically in the business of breaking news, but there are times when they can lend a hand. For instance, if there's an emergency situation where human life or property is at immediate risk, and no other means of communication is available, amateur stations can transmit information in support of broadcasting, program production, or news gathering. This is a great example of how amateur radio can step up in times of crisis.

% \begin{figure}[h]
%     \centering
%     % \includegraphics[width=0.8\textwidth]{figures/coded-news-flow.png}
%     \caption{Conditions for Coded Messages and News Gathering in Amateur Radio}
%     \label{fig:coded-news-flow}
%     % Flowchart showing the conditions under which coded messages and news gathering are permitted in amateur radio. The flowchart should include decision points for coded messages (e.g., "Is the message a control command for a space station or radio control craft?") and news gathering (e.g., "Is there an immediate safety concern?").
% \end{figure}

% \begin{table}[h]
%     \centering
%     \begin{tabular}{|l|l|}
%         \hline
%         \textbf{Activity} & \textbf{Permitted Conditions} \\
%         \hline
%         Coded Messages & Only when transmitting control commands to space stations or radio control craft \\
%         \hline
%         News Gathering & Only when directly related to the immediate safety of human life or protection of property, and no other means is available \\
%         \hline
%     \end{tabular}
%     \caption{Rules for Coded Messages and News Gathering}
%     \label{tab:coded-news-rules}
% \end{table}

\subsubsection{Questions}

\begin{tcolorbox}[colback=gray!10!white,colframe=black!75!black,title={T1D03}]
    When is it permissible to transmit messages encoded to obscure their meaning?
    \begin{enumerate}[label=\Alph*),noitemsep]
        \item Only during contests
        \item Only when transmitting certain approved digital codes
        \item \textbf{Only when transmitting control commands to space stations or radio control craft}
        \item Never
    \end{enumerate}
\end{tcolorbox}

According to FCC regulations, coded messages are only allowed when transmitting control commands to space stations or radio control craft. This ensures that the use of coded messages is limited to specific, necessary scenarios.

\begin{tcolorbox}[colback=gray!10!white,colframe=black!75!black,title={T1D09}]
    When may amateur stations transmit information in support of broadcasting, program production, or news gathering, assuming no other means is available?
    \begin{enumerate}[label=\Alph*),noitemsep]
        \item \textbf{When such communications are directly related to the immediate safety of human life or protection of property}
        \item When broadcasting communications to or from the space shuttle
        \item Where noncommercial programming is gathered and supplied exclusively to the National Public Radio network
        \item Never
    \end{enumerate}
\end{tcolorbox}

Amateur stations can transmit information in support of broadcasting, program production, or news gathering only when it is directly related to the immediate safety of human life or protection of property, and no other means of communication is available. This ensures that amateur radio is used responsibly in emergency situations.

And there you have it! A quick rundown of the rules and regulations surrounding coded messages and news gathering in amateur radio. Remember, these rules are in place to ensure that amateur radio remains a safe and effective means of communication for everyone. Happy transmitting!
