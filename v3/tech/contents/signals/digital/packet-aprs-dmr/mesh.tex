\subsection{Mesh Networks}
\label{subsec:mesh}

Amateur radio mesh networks are a fascinating evolution in the world of radio communication. Unlike traditional repeater networks, which rely on a central hub to relay signals, mesh networks operate in a decentralized manner. Each node in the network can communicate directly with other nodes, creating a web-like structure that is both resilient and scalable. This decentralized approach allows for more flexible and robust communication, especially in scenarios where traditional infrastructure might be compromised, such as during natural disasters.

\subsubsection*{The Role of Wi-Fi Equipment in Mesh Networks}

One of the key components of amateur radio mesh networks is the use of commercial Wi-Fi equipment that has been modified with custom firmware. This modification allows the equipment to operate on amateur radio frequencies, enabling the creation of a data network that is both cost-effective and highly adaptable. The modified firmware often includes features that are specifically tailored for amateur radio use, such as enhanced error correction and the ability to operate in a mesh topology.

\subsubsection*{Advantages of Mesh Networks in Emergency Communication}

Mesh networks offer several advantages over traditional repeater networks, particularly in emergency communication scenarios. Because each node in the network can act as a relay, the network can continue to function even if some nodes are damaged or destroyed. This makes mesh networks particularly well-suited for use in disaster zones, where traditional communication infrastructure may be unavailable. Additionally, the decentralized nature of mesh networks allows for greater scalability, as new nodes can be added to the network without the need for significant infrastructure changes.

\begin{figure}[h]
    \centering
    % \includegraphics[width=0.8\textwidth]{mesh-topology}
    \caption{Amateur Radio Mesh Network Topology}
    \label{fig:mesh-topology}
    % Diagram showing the topology of an amateur radio mesh network, including nodes and data paths.
\end{figure}

\begin{figure}[h]
    \centering
    % \includegraphics[width=0.8\textwidth]{mesh-wifi}
    \caption{Modified Wi-Fi Equipment for Mesh Networks}
    \label{fig:mesh-wifi}
    % Illustration of modified Wi-Fi equipment used in amateur radio mesh networks.
\end{figure}

\begin{table}[h]
    \centering
    \begin{tabular}{|l|l|l|}
        \hline
        \textbf{Feature} & \textbf{Mesh Networks} & \textbf{Repeater Networks} \\
        \hline
        Scalability & High & Limited \\
        Reliability & High (due to decentralized nature) & Moderate (dependent on central hub) \\
        Use Cases & Emergency communication, rural areas & Urban areas, fixed infrastructure \\
        \hline
    \end{tabular}
    \caption{Comparison of Mesh Networks and Repeater Networks}
    \label{tab:mesh-repeater-comparison}
\end{table}

\subsubsection*{Questions}

\begin{tcolorbox}[colback=gray!10!white,colframe=black!75!black,title={T8D12}]
    Which of the following best describes an amateur radio mesh network?
    \begin{enumerate}[label=\Alph*),noitemsep]
        \item \textbf{An amateur-radio based data network using commercial Wi-Fi equipment with modified firmware}
        \item A wide-bandwidth digital voice mode employing DMR protocols
        \item A satellite communications network using modified commercial satellite TV hardware
        \item An internet linking protocol used to network repeaters
    \end{enumerate}
\end{tcolorbox}

\subsubsection*{Explanation}

An amateur radio mesh network is best described as an amateur-radio based data network that utilizes commercial Wi-Fi equipment with modified firmware. This setup allows for the creation of a decentralized network where each node can communicate directly with others, enhancing both scalability and reliability. Option B is incorrect because it describes a digital voice mode, not a data network. Option C is also incorrect as it refers to satellite communications, which is a different technology altogether. Option D is incorrect because it describes a protocol used for linking repeaters, not a mesh network.
