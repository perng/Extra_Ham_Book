\subsection{ARQ Systems}
\label{subsec:arq-systems}

Automatic Repeat Request (ARQ) is a fascinating and highly effective method for ensuring data integrity during transmission. Imagine you're sending a message to a friend, but the signal gets a bit garbled along the way. ARQ is like having a built-in spell checker that not only spots the errors but also asks for a do-over when things go wrong. In technical terms, ARQ is an error correction method where the receiving station detects errors and sends a request for retransmission. This process ensures that the data you receive is as accurate as possible, even in less-than-ideal transmission conditions.

\subsubsection*{How ARQ Works}
The ARQ process can be broken down into a few key steps:
\begin{enumerate}
    \item \textbf{Transmission}: The sender transmits a block of data to the receiver.
    \item \textbf{Error Detection}: The receiver checks the data for errors using various error-detection techniques, such as cyclic redundancy checks (CRC).
    \item \textbf{Request for Retransmission}: If errors are detected, the receiver sends a request back to the sender to retransmit the erroneous data.
    \item \textbf{Retransmission}: The sender retransmits the data, and the process repeats until the data is received error-free.
\end{enumerate}

This back-and-forth might sound a bit tedious, but it's incredibly effective. ARQ systems are particularly useful in environments where the transmission medium is prone to errors, such as in wireless communication.

\subsubsection*{Advantages and Limitations}
ARQ systems have several advantages:
\begin{itemize}
    \item \textbf{High Reliability}: By ensuring that only error-free data is accepted, ARQ systems provide a high level of reliability.
    \item \textbf{Simplicity}: The concept is straightforward and doesn't require complex algorithms for error correction.
\end{itemize}

However, there are some limitations:
\begin{itemize}
    \item \textbf{Latency}: The need for retransmissions can introduce delays, especially in high-error environments.
    \item \textbf{Bandwidth Usage}: Retransmissions consume additional bandwidth, which can be a concern in bandwidth-limited systems.
\end{itemize}

\subsubsection*{Real-World Applications}
ARQ systems are widely used in various real-world applications, including:
\begin{itemize}
    \item \textbf{Wireless Communication}: Ensuring reliable data transmission over cellular networks.
    \item \textbf{Satellite Communication}: Maintaining data integrity over long distances where signal degradation is common.
    \item \textbf{Internet Protocols}: Protocols like TCP use ARQ-like mechanisms to ensure data packets are received correctly.
\end{itemize}

\begin{figure}[h!]
    \centering
    % \includegraphics[width=0.8\textwidth]{arq-flowchart}
    \caption{Flowchart of the ARQ process.}
    \label{fig:arq-flowchart}
    % Prompt: Flowchart illustrating the ARQ process, including error detection and retransmission request.
    % Software: Graphviz
\end{figure}

\begin{figure}[h!]
    \centering
    % \includegraphics[width=0.8\textwidth]{arq-interaction}
    \caption{Interaction between transmitter and receiver in an ARQ system.}
    \label{fig:arq-interaction}
    % Prompt: Diagram showing the interaction between transmitter and receiver in an ARQ system.
    % Software: Matplotlib
\end{figure}

\begin{table}[h!]
    \centering
    \begin{tabular}{|l|c|c|c|}
        \hline
        \textbf{ARQ Type} & \textbf{Efficiency} & \textbf{Complexity} & \textbf{Error Handling} \\
        \hline
        Stop-and-Wait ARQ & Low & Low & Basic \\
        Go-Back-N ARQ & Medium & Medium & Moderate \\
        Selective Repeat ARQ & High & High & Advanced \\
        \hline
    \end{tabular}
    \caption{Comparison of different ARQ systems.}
    \label{tab:arq-comparison}
\end{table}

\begin{tcolorbox}[colback=gray!10!white,colframe=black!75!black,title={T8D11}]
    \textbf{What is an ARQ transmission system?}
    \begin{enumerate}[label=\Alph*),noitemsep]
        \item A special transmission format limited to video signals
        \item A system used to encrypt command signals to an amateur radio satellite
        \item \textbf{An error correction method in which the receiving station detects errors and sends a request for retransmission}
        \item A method of compressing data using autonomous reiterative Q codes prior to final encoding
    \end{enumerate}
\end{tcolorbox}

ARQ stands for Automatic Repeat Request, and it's a method used to ensure data integrity during transmission. When the receiver detects an error in the received data, it sends a request back to the transmitter to resend the data. This process continues until the data is received without errors. Option C is correct because it accurately describes this process. Option A is incorrect because ARQ is not limited to video signals. Option B is incorrect because ARQ is not related to encryption. Option D is incorrect because ARQ does not involve data compression or Q codes.
