\subsection{EchoLink}
\label{subsec:echolink}

EchoLink is a fascinating system that allows amateur radio stations to connect to the internet, bridging the gap between traditional radio communication and modern digital networks. Imagine being able to chat with fellow ham radio operators across the globe without needing to rely solely on atmospheric conditions or expensive equipment. That's the magic of EchoLink!

At the heart of EchoLink is the concept of a \textbf{gateway}. A gateway in EchoLink acts as a bridge between amateur radio stations and the internet. It takes the radio signals from one station, converts them into digital data, and sends them over the internet to another station. This process is reversed when the other station responds, allowing for seamless communication between stations that might be thousands of miles apart.

To use EchoLink, you'll need to go through a registration process. This process ensures that only licensed amateur radio operators can access the system. You'll need to provide proof of your license, which helps maintain the integrity and security of the network. Once registered, you can start connecting with other operators and exploring the vast possibilities that EchoLink offers.

\subsubsection*{EchoLink System Architecture}
% Diagram illustrating the EchoLink system and its gateway functionality.
% Software: Graphviz
% Caption: EchoLink System Architecture
% Label: fig:echolink-architecture
\begin{figure}[h]
    \centering
    % \includegraphics[width=0.8\textwidth]{path/to/echolink-architecture.png}
    \caption{EchoLink System Architecture}
    \label{fig:echolink-architecture}
\end{figure}

\subsubsection*{EchoLink Features and Use Cases}
\begin{table}[h]
    \centering
    \begin{tabular}{|l|l|}
        \hline
        \textbf{Feature} & \textbf{Use Case} \\
        \hline
        Internet Connectivity & Connect with operators worldwide \\
        Gateway Functionality & Bridge between radio and internet \\
        Registration Process & Ensure licensed operators only \\
        \hline
    \end{tabular}
    \caption{EchoLink Features and Use Cases}
    \label{tab:echolink-features}
\end{table}

\begin{tcolorbox}[colback=gray!10!white,colframe=black!75!black,title={T8C11}]
    What is an amateur radio station that connects other amateur stations to the internet?
    \begin{enumerate}[label=\Alph*),noitemsep]
        \item \textbf{A gateway}
        \item A repeater
        \item A digipeater
        \item A beacon
    \end{enumerate}
\end{tcolorbox}

A gateway is the correct answer because it serves as the bridge between amateur radio stations and the internet, enabling communication over long distances. A repeater (B) simply retransmits signals, while a digipeater (C) is used in packet radio networks. A beacon (D) is used to transmit signals for propagation testing, not for internet connectivity.
