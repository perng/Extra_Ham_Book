\subsection{CW (Morse Code)}
\label{subsec:cw}

Continuous Wave (CW) transmission is one of the oldest and simplest forms of radio communication. It involves transmitting a pure, unmodulated carrier wave that is turned on and off to convey information. Historically, CW was the backbone of early radio communication, especially in maritime and military applications. Its simplicity made it reliable and effective, even with the limited technology of the time. Today, CW is still used in amateur radio, particularly for long-distance communication, due to its efficiency and narrow bandwidth requirements.

\subsubsection*{Bandwidth of CW Signals}
One of the key advantages of CW is its extremely narrow bandwidth. Unlike other modulation techniques like FM or SSB, which require a wider range of frequencies to transmit voice or data, CW signals occupy only a tiny slice of the spectrum. This is because CW is essentially just a single frequency being turned on and off. The bandwidth of a CW signal is typically around \textbf{150 Hz}, making it the narrowest among common modulation types. This narrow bandwidth allows CW signals to travel long distances with minimal interference and noise, which is why it remains popular in amateur radio.

\subsubsection*{Morse Code in CW Transmissions}
Morse Code is the language of CW. It uses a series of dots (short signals) and dashes (long signals) to represent letters, numbers, and punctuation. While Morse Code may seem archaic, it has a unique charm and efficiency. It requires no special equipment beyond a simple key and a receiver, making it accessible to amateur radio operators. Even in the age of digital communication, Morse Code remains a valuable skill, especially in emergency situations where other forms of communication may fail.

\begin{figure}[h]
    \centering
    % \includegraphics[width=0.8\textwidth]{cw-spectrum}
    \caption{Frequency Spectrum of CW Signals}
    \label{fig:cw-spectrum}
    % Diagram showing the frequency spectrum of a CW signal compared to SSB and FM signals.
    % The figure should clearly illustrate the narrow bandwidth of CW compared to the wider bandwidths of SSB and FM.
\end{figure}

\begin{table}[h]
    \centering
    \begin{tabular}{|l|c|c|}
        \hline
        \textbf{Modulation Type} & \textbf{Bandwidth} & \textbf{Typical Usage} \\
        \hline
        CW & 150 Hz & Long-distance communication, amateur radio \\
        SSB & 2.4 kHz & Voice communication, amateur radio \\
        FM & 15 kHz & Broadcast radio, two-way radio \\
        \hline
    \end{tabular}
    \caption{Comparison of CW, SSB, and FM Modulation Techniques}
    \label{tab:cw-comparison}
\end{table}

\subsubsection*{Questions}

\begin{tcolorbox}[colback=gray!10!white,colframe=black!75!black,title={T8A05}]
    Which of the following types of signal has the narrowest bandwidth?
    \begin{enumerate}[label=\Alph*),noitemsep]
        \item FM voice
        \item SSB voice
        \item \textbf{CW}
        \item Slow-scan TV
    \end{enumerate}
\end{tcolorbox}
CW signals have the narrowest bandwidth, typically around 150 Hz. This is because CW is a simple on-off keying of a carrier wave, unlike FM or SSB, which require more bandwidth to transmit voice or data.

\begin{tcolorbox}[colback=gray!10!white,colframe=black!75!black,title={T8A11}]
    What is the approximate bandwidth required to transmit a CW signal?
    \begin{enumerate}[label=\Alph*),noitemsep]
        \item 2.4 kHz
        \item \textbf{150 Hz}
        \item 1000 Hz
        \item 15 kHz
    \end{enumerate}
\end{tcolorbox}
The bandwidth required for a CW signal is approximately 150 Hz. This narrow bandwidth is one of the reasons CW is so efficient for long-distance communication, as it minimizes interference and noise.
