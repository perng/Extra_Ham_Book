\subsection{Decibel Calculations}
\label{subsec:power-db}

Now, let's talk about decibels (dB). Decibels are a logarithmic unit used to express the ratio of two values of a physical quantity, often power or intensity. In radio systems, decibels are used to describe power changes in a way that's easy to understand and compare.

The formula to calculate the decibel value for a power change is:

\begin{equation}
\text{dB} = 10 \log_{10}\left(\frac{P_2}{P_1}\right)
\label{eq:db-calc}
\end{equation}

where \(P_1\) is the initial power and \(P_2\) is the final power. This formula is your best friend when dealing with power changes in radio systems.

\subsubsection*{Examples of Power Changes in Decibels}
Let's look at some examples to make this clearer. Suppose you increase the power from 5 watts to 10 watts. Using equation \ref{eq:db-calc}, we get:

\[
\text{dB} = 10 \log_{10}\left(\frac{10}{5}\right) = 10 \log_{10}(2) \approx 3 \text{ dB}
\]

So, a power increase from 5 watts to 10 watts is approximately 3 dB. Similarly, if you decrease the power from 12 watts to 3 watts, the calculation would be:

\[
\text{dB} = 10 \log_{10}\left(\frac{3}{12}\right) = 10 \log_{10}(0.25) \approx -6 \text{ dB}
\]

This means a power decrease from 12 watts to 3 watts is approximately -6 dB.

\begin{figure}[h]
    \centering
    \begin{minipage}{0.48\textwidth}
        \centering
        \scriptsize
        \begin{tabular}{|c|c||c|c|}
            \hline
            \textbf{Ratio} & \textbf{dB} & \textbf{Ratio} & \textbf{dB} \\
            \hline
            10000 & +40.0 & 1/10000 & -40.0 \\
            1000 & +30.0 & 1/1000 & -30.0 \\
            100 & +20.0 & 1/100 & -20.0 \\
            90 & +19.5 & 1/90 & -19.5 \\
            80 & +19.0 & 1/80 & -19.0 \\
            70 & +18.5 & 1/70 & -18.5 \\
            60 & +17.8 & 1/60 & -17.8 \\
            50 & +17.0 & 1/50 & -17.0 \\
            40 & +16.0 & 1/40 & -16.0 \\
            30 & +14.8 & 1/30 & -14.8 \\
            20 & +13.0 & 1/20 & -13.0 \\
            10 & +10.0 & 1/10 & -10.0 \\
            9 & +9.5 & 1/9 & -9.5 \\
            8 & +9.0 & 1/8 & -9.0 \\
            7 & +8.5 & 1/7 & -8.5 \\
            6 & +7.8 & 1/6 & -7.8 \\
            5 & +7.0 & 1/5 & -7.0 \\
            4 & +6.0 & 1/4 & -6.0 \\
            3 & +4.8 & 1/3 & -4.8 \\
            2 & +3.0 & 1/2 & -3.0 \\
            1 & 0.0 & 1 & 0.0 \\
            \hline
        \end{tabular}
        \captionof{table}{Power Ratios to dB}
        \label{tab:power-db-ratios}
    \end{minipage}%
    \begin{minipage}{0.48\textwidth}
        \centering
        \scriptsize
        \begin{tabular}{|c|c||c|c|}
            \hline
            \textbf{dB} & \textbf{Ratio} & \textbf{dB} & \textbf{Ratio} \\
            \hline
            +20 & 100.0 & -20 & 0.010 \\
            +19 & 79.4 & -19 & 0.013 \\
            +18 & 63.1 & -18 & 0.016 \\
            +17 & 50.1 & -17 & 0.020 \\
            +16 & 39.8 & -16 & 0.025 \\
            +15 & 31.6 & -15 & 0.032 \\
            +14 & 25.1 & -14 & 0.040 \\
            +13 & 20.0 & -13 & 0.050 \\
            +12 & 15.8 & -12 & 0.063 \\
            +11 & 12.6 & -11 & 0.079 \\
            +10 & 10.0 & -10 & 0.100 \\
            +9 & 7.94 & -9 & 0.126 \\
            +8 & 6.31 & -8 & 0.158 \\
            +7 & 5.01 & -7 & 0.200 \\
            +6 & 3.98 & -6 & 0.251 \\
            +5 & 3.16 & -5 & 0.316 \\
            +4 & 2.51 & -4 & 0.398 \\
            +3 & 2.00 & -3 & 0.500 \\
            +2 & 1.58 & -2 & 0.631 \\
            +1 & 1.26 & -1 & 0.794 \\
            0 & 1.00 & 0 & 1.000 \\
            \hline
        \end{tabular}
        \captionof{table}{dB to Power Ratios}
        \label{tab:db-power-ratios}
    \end{minipage}
    \begin{tcolorbox}[colback=gray!10!white,colframe=black!75!black,title={Key Points}]
        Common values to remember:
        \begin{itemize}[noitemsep]            \item Double power (+3 dB) / Half power (-3 dB)
            \item 10 times power (+10 dB) / One-tenth power (-10 dB)
            \item 100 times power (+20 dB) / One-hundredth power (-20 dB)
        \end{itemize}
        \end{tcolorbox}
        \end{figure}

Understanding decibel values is crucial in radio communication. It helps you determine how much power you need to transmit a signal over a certain distance or how much power you can expect to lose due to obstacles. It's like knowing the fuel efficiency of your car—it helps you plan your journey better.


\subsubsection{Questions}

\begin{tcolorbox}[colback=gray!10!white,colframe=black!75!black,title={T5B09}]
Which decibel value most closely represents a power increase from 5 watts to 10 watts?
\begin{enumerate}[label=\Alph*),noitemsep]
    \item 2 dB
    \item \textbf{3 dB}
    \item 5 dB
    \item 10 dB
\end{enumerate}
\end{tcolorbox}

Using $\text{dB} = 10 \log_{10}(\frac{P_2}{P_1})$:
$\text{dB} = 10 \log_{10}(\frac{10\text{ W}}{5\text{ W}}) = 10 \log_{10}(2) \approx 3\text{ dB}$

\begin{tcolorbox}[colback=gray!10!white,colframe=black!75!black,title={T5B10}]
Which decibel value most closely represents a power decrease from 12 watts to 3 watts?
\begin{enumerate}[label=\Alph*),noitemsep]
    \item -1 dB
    \item -3 dB
    \item \textbf{-6 dB}
    \item -9 dB
\end{enumerate}
\end{tcolorbox}

Using $\text{dB} = 10 \log_{10}(\frac{P_2}{P_1})$:
$\text{dB} = 10 \log_{10}(\frac{3\text{ W}}{12\text{ W}}) = 10 \log_{10}(0.25) \approx -6\text{ dB}$

\begin{tcolorbox}[colback=gray!10!white,colframe=black!75!black,title={T5B11}]
Which decibel value represents a power increase from 20 watts to 200 watts?
\begin{enumerate}[label=\Alph*),noitemsep]
    \item \textbf{10 dB}
    \item 12 dB
    \item 18 dB
    \item 28 dB
\end{enumerate}
\end{tcolorbox}

Using $\text{dB} = 10 \log_{10}(\frac{P_2}{P_1})$:
$\text{dB} = 10 \log_{10}(\frac{200\text{ W}}{20\text{ W}}) = 10 \log_{10}(10) = 10\text{ dB}$
