\subsection{RF and Frequency Abbreviations}
\label{subsec:rf-abbrev}

In this section, we'll dive into some of the most common abbreviations you'll encounter in the world of radio technology. Don't worry, we'll keep it light and fun—no need to panic if you're not a math whiz or a physics guru. We'll start with the basics and build up from there.

\subsubsection*{What is RF?}

First up, let's talk about \textbf{RF}. No, it's not short for "Really Fun" (although radio can be pretty fun). RF stands for \textbf{Radio Frequency}, and it refers to the range of electromagnetic signals used for wireless communication. These signals can be anything from the low-frequency waves used in AM radio to the high-frequency waves used in satellite communications. Essentially, RF is the backbone of all wireless communication, from your Wi-Fi router to your favorite FM radio station.

\subsubsection*{Megahertz and Kilohertz}

Next, let's tackle the abbreviations for frequency measurements. You've probably heard of \textbf{MHz} and \textbf{kHz}, but what do they actually mean? 

- \textbf{MHz} stands for \textbf{Megahertz}, which is a unit of frequency equal to one million hertz. It's commonly used to describe the frequency of radio waves, computer processors, and even some types of light.
  
- \textbf{kHz} stands for \textbf{Kilohertz}, which is a unit of frequency equal to one thousand hertz. This is often used to describe lower-frequency signals, like those used in AM radio.

Both of these units are crucial for understanding how different types of signals are transmitted and received. For example, when you tune your radio to 98.5 FM, you're actually tuning it to 98.5 MHz.

\begin{figure}[h]
    \centering
    % \includegraphics[width=0.8\textwidth]{frequency-spectrum}
    \caption{Frequency spectrum with RF, MHz, and kHz labeled.}
    \label{fig:frequency-spectrum}
    % Prompt: Diagram showing the frequency spectrum with RF, MHz, and kHz labeled.
\end{figure}

\begin{table}[h]
    \centering
    \begin{tabular}{|c|c|}
        \hline
        \textbf{Abbreviation} & \textbf{Meaning} \\
        \hline
        RF & Radio Frequency \\
        MHz & Megahertz \\
        kHz & Kilohertz \\
        \hline
    \end{tabular}
    \caption{Common RF and frequency abbreviations.}
    \label{tab:rf-abbreviations}
\end{table}

\subsubsection*{Questions}

\begin{tcolorbox}[colback=gray!10!white,colframe=black!75!black,title={T5C06}]
    What does the abbreviation “RF” mean?
    \begin{enumerate}[label=\Alph*),noitemsep]
        \item \textbf{Radio frequency signals of all types}
        \item The resonant frequency of a tuned circuit
        \item The real frequency transmitted as opposed to the apparent frequency
        \item Reflective force in antenna transmission lines
    \end{enumerate}
\end{tcolorbox}

RF stands for Radio Frequency, which encompasses all types of radio frequency signals used in wireless communication. The other options are either incorrect or unrelated to the abbreviation RF.

\begin{tcolorbox}[colback=gray!10!white,colframe=black!75!black,title={T5C07}]
    What is the abbreviation for megahertz?
    \begin{enumerate}[label=\Alph*),noitemsep]
        \item MH
        \item mh
        \item Mhz
        \item \textbf{MHz}
    \end{enumerate}
\end{tcolorbox}

The correct abbreviation for megahertz is \textbf{MHz}. The other options either use incorrect capitalization or incorrect lettering, which can lead to confusion in technical contexts.
