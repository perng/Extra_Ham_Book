\subsection{Measuring Voltage, Current, Resistance}
\label{subsec:voltage-current}

In this section, we'll dive into the tools and techniques used to measure voltage, current, and resistance. These are the bread and butter of any radio technician's toolkit, so let's get started!

\subsubsection*{Voltmeter: Measuring Electric Potential}

A voltmeter is your go-to instrument for measuring electric potential, or voltage. Think of it as the "pressure gauge" for your electrical circuits. When you want to know how much "push" is driving the electrons through a circuit, you use a voltmeter. It measures the potential difference between two points in a circuit, and it does this by being connected in parallel to the component you're measuring. This means the voltmeter sits alongside the component, not in the middle of the current flow.

\begin{figure}[h]
    \centering
    % \includegraphics{voltmeter-parallel.svg} % Placeholder for the actual image
    \caption{Voltmeter connected in parallel to measure voltage.}
    \label{fig:voltmeter-parallel}
    % Image prompt: Diagram showing the correct connection of a voltmeter in parallel to measure voltage.
\end{figure}

\subsubsection*{Ammeter: Measuring Electric Current}

Now, if you want to measure the flow of electrons—the current—you'll need an ammeter. Unlike the voltmeter, the ammeter is connected in series with the component you're measuring. This means the current flows through the ammeter itself, allowing it to measure the amount of current passing through the circuit.

\begin{figure}[h]
    \centering
    % \includegraphics{ammeter-series.svg} % Placeholder for the actual image
    \caption{Ammeter connected in series to measure current.}
    \label{fig:ammeter-series}
    % Image prompt: Diagram showing the correct connection of an ammeter in series to measure current.
\end{figure}

\subsubsection*{Ohmmeter: Measuring Resistance}

An ohmmeter is used to measure resistance, which is the opposition to the flow of current in a circuit. It's a bit like measuring how narrow a pipe is—the narrower the pipe, the harder it is for water to flow through. Similarly, the higher the resistance, the harder it is for current to flow. Ohmmeters are typically used when the circuit is powered off, as they send a small current through the component to measure its resistance.

\subsubsection*{Connection Methods: Voltmeter vs. Ammeter}

To summarize, here's a quick comparison of how voltmeters and ammeters are connected:

\begin{table}[h]
    \centering
    \begin{tabular}{|l|l|}
        \hline
        \textbf{Instrument} & \textbf{Connection Method} \\
        \hline
        Voltmeter & Parallel \\
        \hline
        Ammeter & Series \\
        \hline
    \end{tabular}
    \caption{Comparison of voltmeter and ammeter connection methods.}
    \label{tab:voltmeter-ammeter-comparison}
\end{table}

\subsubsection*{Questions}

\begin{tcolorbox}[colback=gray!10!white,colframe=black!75!black,title={T7D01}]
    Which instrument would you use to measure electric potential?
    \begin{enumerate}[label=\Alph*),noitemsep]
        \item An ammeter
        \item \textbf{A voltmeter}
        \item A wavemeter
        \item An ohmmeter
    \end{enumerate}
\end{tcolorbox}

A voltmeter is used to measure electric potential, as it measures the voltage difference between two points in a circuit. An ammeter measures current, a wavemeter measures wavelength, and an ohmmeter measures resistance.

\begin{tcolorbox}[colback=gray!10!white,colframe=black!75!black,title={T7D02}]
    How is a voltmeter connected to a component to measure applied voltage?
    \begin{enumerate}[label=\Alph*),noitemsep]
        \item In series
        \item \textbf{In parallel}
        \item In quadrature
        \item In phase
    \end{enumerate}
\end{tcolorbox}

A voltmeter is connected in parallel to measure the voltage across a component. Connecting it in series would disrupt the current flow, and the other options are not relevant to voltage measurement.

\begin{tcolorbox}[colback=gray!10!white,colframe=black!75!black,title={T7D03}]
    When configured to measure current, how is a multimeter connected to a component?
    \begin{enumerate}[label=\Alph*),noitemsep]
        \item \textbf{In series}
        \item In parallel
        \item In quadrature
        \item In phase
    \end{enumerate}
\end{tcolorbox}

When measuring current, a multimeter is connected in series with the component. This allows the current to flow through the multimeter, enabling it to measure the current accurately.

\begin{tcolorbox}[colback=gray!10!white,colframe=black!75!black,title={T7D04}]
    Which instrument is used to measure electric current?
    \begin{enumerate}[label=\Alph*),noitemsep]
        \item An ohmmeter
        \item An electrometer
        \item A voltmeter
        \item \textbf{An ammeter}
    \end{enumerate}
\end{tcolorbox}

An ammeter is specifically designed to measure electric current. An ohmmeter measures resistance, an electrometer measures charge, and a voltmeter measures voltage.
