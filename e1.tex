\chapter{Rules and Regulations for Extra Class Operators}

\section{Frequency Privileges}

\subsection*{Introduction to Frequency Privileges}
Welcome to the frequency playground! As an Extra Class operator, you’ve unlocked the full spectrum of amateur radio bands. But with great power comes great responsibility—staying within the lines (or edges, in this case) is key. This section will help you understand how to operate without spilling over the boundaries, why certain frequencies are sacred, and how to maximize your privileges.

\subsubsection*{Staying Within the Band Edges}
Imagine you’re at a buffet—everything looks delicious, but you have to stay on your plate! Transmitting within the band works the same way. If your signal spills out of the assigned range, it’s like dropping food on someone else’s table—messy and definitely not allowed.

Here’s the breakdown:
\begin{itemize}
    \item \textbf{USB (Upper Sideband):} Common above 10 MHz. The carrier frequency needs to be low enough that the upper edge of your signal doesn’t exceed the band’s upper limit.
    \item \textbf{LSB (Lower Sideband):} Common below 10 MHz. The carrier frequency should be high enough to keep the lower edge of your signal within the band.
\end{itemize}

\textbf{Pro Tip:} Add a buffer zone! Keep your carrier a few kHz away from the band edge to avoid accidental spillage.

\begin{tcolorbox}[title=Fun Fact: Why USB and LSB?]
Did you know? USB is used above 10 MHz, and LSB is used below 10 MHz simply because that’s how radio operators standardized it during WWII. It kept equipment designs simpler at the time, and we’ve stuck with it ever since!
\end{tcolorbox}

\subsubsection*{Calculating Safe Frequencies}
Let’s say you’re operating on the 20-meter band (14.000–14.350 MHz):
\begin{itemize}
    \item For a USB signal with a 3 kHz bandwidth, the highest legal carrier frequency is 14.347 MHz. Why? The upper edge of your signal will then reach exactly 14.350 MHz.
    \item For LSB, if the lower band edge is 14.000 MHz, your carrier must be at least 3 kHz above it, or 14.003 MHz.
\end{itemize}

\subsubsection*{The Consequences of Overflowing}
What happens if your signal crosses the line? You risk:
\begin{itemize}
    \item Causing interference to adjacent bands.
    \item Receiving a stern warning—or worse—from the FCC.
\end{itemize}
So, stay in bounds and enjoy harmonious operation!

\subsection*{Questions for E1A: Frequency Privileges}
\begin{enumerate}
    \item \textbf{Why is it not legal to transmit a 3 kHz bandwidth USB signal with a carrier frequency of 14.348 MHz?}
    \begin{enumerate}
        \item USB is not used on 20-meter phone.\\
        \item The lower 1 kHz of the signal is outside the 20-meter band.\\
        \item 14.348 MHz is outside the 20-meter band.\\
        \item \textbf{The upper 1 kHz of the signal is outside the 20-meter band.}
    \end{enumerate}
    \textbf{Explanation:} A USB signal's bandwidth extends above its carrier. At 14.348 MHz, the upper edge reaches 14.351 MHz, exceeding the 20-meter band limit of 14.350 MHz. Options A and C are incorrect because USB is used on 20 meters, and 14.348 MHz is within the band. Option B incorrectly assumes the lower part of the signal causes the issue.

    \item \textbf{When using a transceiver that displays the carrier frequency of phone signals, which displayed frequency represents the lowest frequency at which a properly adjusted LSB emission will be totally within the band?}
    \begin{enumerate}
        \item The exact lower band edge.\\
        \item 300 Hz above the lower band edge.\\
        \item 1 kHz above the lower band edge.\\
        \item \textbf{3 kHz above the lower band edge.}
    \end{enumerate}
    \textbf{Explanation:} LSB signals extend below the carrier frequency. To ensure the entire 3 kHz signal is within the band, the carrier must be 3 kHz above the band edge. Options A, B, and C underestimate the bandwidth.

    \item \textbf{What is the highest legal carrier frequency on the 20-meter band for transmitting a 2.8 kHz wide USB data signal?}
    \begin{enumerate}
        \item 14.0708 MHz.\\
        \item 14.1002 MHz.\\
        \item \textbf{14.1472 MHz.}\\
        \item 14.3490 MHz.
    \end{enumerate}
    \textbf{Explanation:} For a USB signal, the carrier frequency must be low enough to keep the signal's upper edge within the band. At 14.1472 MHz, the upper edge of the 2.8 kHz signal reaches 14.150 MHz, the upper band limit for this part of the band. Other options either miscalculate the bandwidth or reference unrelated frequency limits.

    \item \textbf{May an Extra Class operator answer the CQ of a station on 3.601 MHz LSB phone?}
    \begin{enumerate}
        \item Yes, the entire signal will be inside the SSB allocation for Extra Class operators.\\
        \item Yes, the displayed frequency is within the 75-meter phone band segment.\\
        \item \textbf{No, the sideband components will extend beyond the edge of the phone band segment.}\\
        \item No, US stations are not permitted to use phone emissions below 3.610 MHz.
    \end{enumerate}
    \textbf{Explanation:} The lower band edge for 75-meter phone is 3.610 MHz. An LSB signal with a carrier at 3.601 MHz would extend 3 kHz below the carrier, violating the band edge. Options A and B ignore this overflow, and option D misrepresents the FCC rules.

\item \textbf{Who must be in physical control of the station apparatus of an amateur station aboard any vessel or craft registered in the United States?}
    \begin{enumerate}
        \item Only a person with an FCC Marine Radio license grant.\\
        \item Only a person named in an amateur station license grant.\\
        \item \textbf{Any person holding an FCC-issued amateur license or authorized for alien reciprocal operation.}\\
        \item Any person named in an amateur station license grant or a person holding an unrestricted Radiotelephone Operator Permit.
    \end{enumerate}
    \textbf{Explanation:} Any amateur operator with a valid FCC license or reciprocal authorization may control the station aboard a U.S.-registered vessel. Options A and D describe unrelated or unnecessary requirements. Option B restricts control unnecessarily to specific named individuals.

    \item \textbf{What is the required transmit frequency of a CW signal for channelized 60-meter operation?}
    \begin{enumerate}
        \item At the lowest frequency of the channel.\\
        \item \textbf{At the center frequency of the channel.}\\
        \item At the highest frequency of the channel.\\
        \item On any frequency where the signal's sidebands are within the channel.
    \end{enumerate}
    \textbf{Explanation:} CW operation on 60 meters must use the center frequency of the channel to keep the entire signal within the assigned band. Options A and C would place parts of the signal outside the channel, and D is ambiguous.

    \item \textbf{What is the maximum power permitted on the 2200-meter band?}
    \begin{enumerate}
        \item 50 watts PEP (peak envelope power).\\
        \item 100 watts PEP (peak envelope power).\\
        \item \textbf{1 watt EIRP (equivalent isotropic radiated power).}\\
        \item 5 watts EIRP (equivalent isotropic radiated power).
    \end{enumerate}
    \textbf{Explanation:} The 2200-meter band has a strict power limit of 1 watt EIRP to prevent interference with primary users. Options A and B describe power levels unsuitable for this band, while D applies to the 630-meter band.

    \item \textbf{If a station in a message forwarding system inadvertently forwards a message in violation of FCC rules, who is primarily accountable?}
    \begin{enumerate}
        \item The control operator of the packet bulletin board station.\\
        \item \textbf{The control operator of the originating station.}\\
        \item The control operators of all the stations in the system.\\
        \item The control operators of stations not authenticating the source from which they accept communications.
    \end{enumerate}
    \textbf{Explanation:} The originating station’s control operator is responsible for ensuring transmitted content complies with FCC rules. Options A and C misunderstand accountability, while D adds unnecessary complexity to the rule.

    \item \textbf{Except in some parts of Alaska, what is the maximum power permitted on the 630-meter band?}
    \begin{enumerate}
        \item 50 watts PEP (peak envelope power).\\
        \item 100 watts PEP (peak envelope power).\\
        \item 1 watt EIRP (equivalent isotropic radiated power).\\
        \item \textbf{5 watts EIRP (equivalent isotropic radiated power).}
    \end{enumerate}
    \textbf{Explanation:} The 630-meter band permits up to 5 watts EIRP in most locations. Options A and B reference PEP limits, not EIRP. Option C applies to the 2200-meter band.

    \item \textbf{If an amateur station is installed aboard a ship or aircraft, what condition must be met before it is operated?}
    \begin{enumerate}
        \item \textbf{Its operation must be approved by the master of the ship or the pilot in command of the aircraft.}\\
        \item The amateur station operator must agree not to transmit when the main radio is in use.\\
        \item The amateur station must have an independent power supply.\\
        \item The amateur station must operate only in specific amateur service HF and VHF bands.
    \end{enumerate}
    \textbf{Explanation:} The station’s operation must be authorized by the ship’s master or pilot. Other options impose additional restrictions not required by FCC regulations.

    \item \textbf{What licensing is required when operating an amateur station aboard a U.S.-registered vessel in international waters?}
    \begin{enumerate}
        \item Any amateur license with an FCC Marine or Aircraft endorsement.\\
        \item \textbf{Any FCC-issued amateur license.}\\
        \item Only General Class or higher amateur licenses.\\
        \item An unrestricted Radiotelephone Operator Permit.
    \end{enumerate}
    \textbf{Explanation:} Any FCC-issued amateur license is sufficient to operate in international waters. Options A and D impose unnecessary restrictions, while C limits access unfairly to certain license classes.
\end{enumerate}



%%%%%%%%%%%%%%%%%%%%%%%%%%%%%%%%%%%

\section{Station Restrictions and Special Operations}

\subsection*{Introduction to Station Restrictions}
Operating an amateur station is more than just flipping a switch and chatting. Certain restrictions and special rules ensure that operations are conducted responsibly, safely, and without interference. From spurious emissions to the rules for emergency communications, this section guides you through the maze of regulations.

\subsubsection*{Spurious Emissions}
Spurious emissions are the unwanted side effects of transmitting signals. These emissions can cause interference and must be minimized to acceptable levels.
\begin{itemize}
    \item \textbf{Definition:} A spurious emission is a signal outside the necessary bandwidth that can be reduced or eliminated without affecting the transmitted information.
    \item \textbf{Key Rule:} Spurious emissions must be at least 40 dB below the fundamental power level to prevent interference.
\end{itemize}

\subsubsection*{The National Radio Quiet Zone (NRQZ)}
The NRQZ surrounds the National Radio Astronomy Observatory and is designed to protect sensitive equipment from radio frequency interference. Operators in or near this zone must comply with strict rules and may need to obtain special permission for certain operations.

\subsubsection*{Antenna Restrictions Near Airports}
Erecting an antenna structure near a public airport involves additional requirements:
\begin{itemize}
    \item You may need to notify the Federal Aviation Administration (FAA).
    \item Registration with the FCC may be required under Part 17 of FCC rules.
\end{itemize}
These rules are designed to ensure safety and avoid interference with aviation communications.

\subsubsection*{PRB-1 and State and Local Regulations}
PRB-1 requires that state and local zoning regulations make reasonable accommodations for amateur radio antennas. While it doesn’t guarantee unlimited freedom, it provides a framework for balancing amateur radio needs with local ordinances.

\subsubsection*{RACES Operations}
The Radio Amateur Civil Emergency Service (RACES) is a special set of rules for amateur stations operating during emergencies. Key points include:
\begin{itemize}
    \item Only certified stations may operate under RACES.
    \item All amateur service frequencies authorized to the control operator may be used.
\end{itemize}

\subsection*{Questions for E1B: Station Restrictions and Special Operations}
\begin{enumerate}
    \item \textbf{Which of the following constitutes a spurious emission?}
    \begin{enumerate}
        \item An amateur station transmission made without the proper call sign identification.\\
        \item A signal transmitted to prevent its detection by any station other than the intended recipient.\\
        \item Any transmitted signal that unintentionally interferes with another licensed radio station and whose levels exceed 40 dB below the fundamental power level.\\
        \item \textbf{An emission outside the signal’s necessary bandwidth that can be reduced or eliminated without affecting the information transmitted.}
    \end{enumerate}
    \textbf{Explanation:} A spurious emission is any unnecessary signal that lies outside the bandwidth required for communication. Options A, B, and C describe other unrelated issues, not spurious emissions.

    \item \textbf{Which of the following is an acceptable bandwidth for digital voice or slow-scan TV transmissions made on the HF amateur bands?}
    \begin{enumerate}
        \item \textbf{3 kHz.}\\
        \item 10 kHz.\\
        \item 15 kHz.\\
        \item 20 kHz.
    \end{enumerate}
    \textbf{Explanation:} Digital voice and SSTV transmissions on HF bands must fit within the 3 kHz bandwidth commonly used for phone operations. Wider bandwidths (B, C, and D) are not allowed on HF.

    \item \textbf{Within what distance must an amateur station protect an FCC monitoring facility from harmful interference?}
    \begin{enumerate}
        \item \textbf{1 mile.}\\
        \item 3 miles.\\
        \item 10 miles.\\
        \item 30 miles.
    \end{enumerate}
    \textbf{Explanation:} Amateur stations within 1 mile of an FCC monitoring facility must avoid causing harmful interference to ensure accurate enforcement operations. Greater distances (B, C, D) are not required.

    \item \textbf{What must the control operator of a repeater operating in the 70-centimeter band do if a radiolocation system experiences interference from that repeater?}
    \begin{enumerate}
        \item Reduce the repeater antenna HAAT (Height Above Average Terrain).\\
        \item File an FAA NOTAM (Notice to Air Missions) with the repeater system's ERP, call sign, and six-character grid locator.\\
        \item \textbf{Cease operation or make changes to the repeater that mitigate the interference.}\\
        \item All these choices are correct.
    \end{enumerate}
    \textbf{Explanation:} The control operator must mitigate interference with radiolocation systems by ceasing operations or making technical adjustments. Options A and B are not appropriate for addressing interference directly.

    \item \textbf{What is the National Radio Quiet Zone?}
    \begin{enumerate}
        \item An area surrounding the FCC monitoring station in Laurel, Maryland.\\
        \item An area in New Mexico surrounding the White Sands Test Area.\\
        \item \textbf{An area surrounding the National Radio Astronomy Observatory.}\\
        \item An area in Florida surrounding Cape Canaveral.
    \end{enumerate}
    \textbf{Explanation:} The NRQZ protects the National Radio Astronomy Observatory. The other areas mentioned are unrelated to radio quiet zones.

    \item \textbf{What does PRB-1 require of state and local regulations affecting amateur radio antenna size and structures?}
    \begin{enumerate}
        \item No limitations may be placed on antenna size or placement.\\
        \item \textbf{Reasonable accommodations of amateur radio must be made.}\\
        \item Such structures must be permitted when use for emergency communications can be demonstrated.\\
        \item Such structures must be permitted if certified by a registered professional engineer.
    \end{enumerate}
    \textbf{Explanation:} PRB-1 mandates reasonable accommodations for amateur antennas but doesn’t guarantee unrestricted installation (A) or require emergency use (C) or professional certification (D).

\item \textbf{Which of the following additional rules apply if you are erecting an amateur station antenna structure at a site at or near a public use airport?}
\begin{enumerate}
    \item \textbf{You may have to notify the Federal Aviation Administration and register it with the FCC as required by Part 17 of the FCC rules.}\\
    \item You may have to enter the height above ground in meters, and the latitude and longitude in degrees, minutes, and seconds on the FAA website.\\
    \item You must file an Environmental Impact Statement with the EPA before construction begins.\\
    \item You must obtain a construction permit from the airport zoning authority per Part 119 of the FAA regulations.
\end{enumerate}
\textbf{Explanation:} The FAA and FCC require notification and registration for antenna structures that might impact aviation safety, especially near airports. Options B, C, and D are either unnecessary or apply to unrelated regulatory requirements.

\item \textbf{To what type of regulations does PRB-1 apply?}
\begin{enumerate}
    \item Homeowners associations.\\
    \item FAA tower height limits.\\
    \item \textbf{State and local zoning.}\\
    \item Use of wireless devices in vehicles.
\end{enumerate}
\textbf{Explanation:} PRB-1 ensures that state and local zoning authorities make reasonable accommodations for amateur radio antennas. Options A and D are unrelated, and B pertains to federal, not state or local, regulations.

\item \textbf{What limitations may the FCC place on an amateur station if its signal causes interference to domestic broadcast reception, assuming that the receivers involved are of good engineering design?}
\begin{enumerate}
    \item The amateur station must cease operation.\\
    \item The amateur station must cease operation on all frequencies below 30 MHz.\\
    \item The amateur station must cease operation on all frequencies above 30 MHz.\\
    \item \textbf{The amateur station must avoid transmitting during certain hours on frequencies that cause the interference.}
\end{enumerate}
\textbf{Explanation:} The FCC can impose time and frequency restrictions to mitigate interference. Ceasing all operations (options A, B, and C) is unnecessarily broad and not the standard approach.

\item \textbf{Which amateur stations may be operated under RACES rules?}
\begin{enumerate}
    \item Only those club stations licensed to Amateur Extra class operators.\\
    \item Any FCC-licensed amateur station except a Technician class.\\
    \item \textbf{Any FCC-licensed amateur station certified by the responsible civil defense organization for the area served.}\\
    \item Only stations meeting the FCC Part 97 technical standards for operation during an emergency.
\end{enumerate}
\textbf{Explanation:} RACES certification is required by the civil defense organization for a station to operate under these rules. Options A, B, and D impose unnecessary limitations or reference unrelated criteria.

\item \textbf{What frequencies are authorized to an amateur station operating under RACES rules?}
\begin{enumerate}
    \item \textbf{All amateur service frequencies authorized to the control operator.}\\
    \item Specific segments in the amateur service MF, HF, VHF, and UHF bands.\\
    \item Specific local government channels.\\
    \item All these choices are correct.
\end{enumerate}
\textbf{Explanation:} RACES rules allow use of all frequencies authorized to the operator’s license. Options B and C impose restrictions not required under RACES, and D is overly inclusive.

\end{enumerate}



%%%%%%%%%%%%%%%%%%%%%%%%%%%%%%%%%%%%%%%%%%%%%%%%%%%%%%%%

\section{Automatic and Remote Control}

\subsection*{Introduction to Automatic and Remote Control}
The magic of amateur radio expands with the ability to control your station remotely or automatically. These advanced features let you explore the airwaves from afar or keep your station active without direct interaction. However, these operations come with specific rules to ensure smooth and safe practices.

\subsubsection*{Remote Control}
Remote-controlled stations are a powerful tool, enabling operators to use their station even when physically distant. Here are the essentials:
\begin{itemize}
    \item \textbf{Control Link:} A secure and reliable control link must be maintained at all times to ensure the station operates correctly.
    \item \textbf{Failsafe Mechanism:} If the control link is lost, transmissions must cease within \textbf{three minutes}. This prevents unintentional interference or rogue transmissions.
    \item \textbf{Operator Responsibilities:} Even remotely, the operator retains full responsibility for complying with all FCC rules.
\end{itemize}

\subsubsection*{Automatic Control}
Automatic control takes the operator out of the immediate loop, allowing the station to operate unattended under specific conditions:
\begin{itemize}
    \item \textbf{Permissible Uses:} Automatic control is allowed for repeater stations, RTTY (Radio Teletype), and data emissions.
    \item \textbf{Failsafe Design:} Similar to remote control, automatic stations must incorporate safeguards to prevent unauthorized or harmful transmissions.
    \item \textbf{Message Forwarding Systems:} Stations involved in message forwarding systems (e.g., packet networks) can operate under automatic control but must ensure compliance with FCC regulations.
\end{itemize}

\subsubsection*{Why It Matters}
Automatic and remote control enable flexible and innovative uses of amateur radio. Imagine maintaining communication with your local repeater while you’re halfway across the globe, or running a beacon that informs others of propagation conditions—all without being tied to the operator’s chair.

\begin{tcolorbox}[title=Fun Fact: The First Remote-Controlled Stations]
Did you know? Remote-controlled amateur radio stations became more popular in the 1980s with advancements in computer interfaces. Early adopters used dial-up modems to control their equipment—a far cry from today’s seamless web interfaces!
\end{tcolorbox}

\subsubsection*{Key Considerations}
Remote and automatic operations are exciting but require adherence to strict rules:
\begin{itemize}
    \item Always monitor your station’s operation, even if automated, to quickly address issues.
    \item Understand and respect local band plans to ensure compatibility with other operators.
\end{itemize}


\subsection*{Questions for E1C: Automatic and Remote Control}

\begin{enumerate}
    \item \textbf{What is the maximum bandwidth for a data emission on 60 meters?}
    \begin{enumerate}
        \item 60 Hz\\
        \item 170 Hz\\
        \item 1.5 kHz\\
        \item \textbf{2.8 kHz}
    \end{enumerate}
    \textbf{Explanation:} The FCC specifies a maximum bandwidth of 2.8 kHz for data emissions on the 60-meter band to prevent interference with primary users. Other bandwidths listed are either too narrow or exceed the legal limit.

    \item \textbf{Which of the following apply to communications transmitted to amateur stations in foreign countries?}
    \begin{enumerate}
        \item Third-party traffic must be limited to that intended for the exclusive use of government and NGOs involved in emergency relief activities.\\
        \item All transmissions must be in English.\\
        \item \textbf{Communications must be limited to those incidental to the purpose of the amateur service and remarks of a personal nature.}\\
        \item All these choices are correct.
    \end{enumerate}
    \textbf{Explanation:} Transmissions to foreign amateur stations are restricted to those incidental to amateur service and personal remarks. Options A and B impose unnecessary restrictions, while D incorrectly suggests all options are valid.

    \item \textbf{How long must an operator wait after filing a notification with the Utilities Technology Council (UTC) before operating on the 2200-meter or 630-meter band?}
    \begin{enumerate}
        \item Operators must not operate until approval is received.\\
        \item \textbf{Operators may operate after 30 days, providing they have not been told that their station is within 1 kilometer of PLC systems using those frequencies.}\\
        \item Operators may not operate until a test signal has been transmitted in coordination with the local power company.\\
        \item Operations may commence immediately and may continue unless interference is reported by the UTC.
    \end{enumerate}
    \textbf{Explanation:} Operators must wait 30 days after notifying the UTC unless informed otherwise. This ensures compatibility with existing PLC systems. Other options either impose unnecessary requirements or incorrectly allow immediate operation.

    \item \textbf{What is an IARP?}
    \begin{enumerate}
        \item \textbf{A permit that allows US amateurs to operate in certain countries of the Americas.}\\
        \item The internal amateur radio practices policy of the FCC.\\
        \item An indication of increased antenna reflected power.\\
        \item A forecast of intermittent aurora radio propagation.
    \end{enumerate}
    \textbf{Explanation:} An Inter-American Amateur Radio Permit (IARP) allows licensed operators to operate in participating countries without additional licensing. The other options describe unrelated concepts.

    \item \textbf{Under what situation may a station transmit third-party communications while being automatically controlled?}
    \begin{enumerate}
        \item Never.\\
        \item \textbf{Only when transmitting RTTY or data emissions.}\\
        \item Only when transmitting SSB or CW.\\
        \item On any mode approved by the National Telecommunication and Information Administration.
    \end{enumerate}
    \textbf{Explanation:} Automatically controlled stations may transmit third-party communications only for RTTY and data emissions. Options A, C, and D either restrict or expand beyond permissible rules.

    \item \textbf{Which of the following is required in order to operate in accordance with CEPT rules in foreign countries where permitted?}
    \begin{enumerate}
        \item You must identify in the official language of the country in which you are operating.\\
        \item The US embassy must approve of your operation.\\
        \item \textbf{You must have a copy of FCC Public Notice DA 16-1048.}\\
        \item You must append "/CEPT" to your call sign.
    \end{enumerate}
    \textbf{Explanation:} CEPT rules require carrying documentation such as FCC Public Notice DA 16-1048. Other requirements listed are either incorrect or not mandatory.

    \item \textbf{What notifications must be given before transmitting on the 630- or 2200-meter bands?}
    \begin{enumerate}
        \item A special endorsement must be requested from the FCC.\\
        \item An environmental impact statement must be filed with the Department of the Interior.\\
        \item Operators must inform the FAA of their intent to operate, giving their call sign and distance to the nearest runway.\\
        \item \textbf{Operators must inform the Utilities Technology Council (UTC) of their call sign and coordinates of the station.}
    \end{enumerate}
    \textbf{Explanation:} Operators must notify the UTC to check for compatibility with PLC systems. Other notifications listed are unnecessary or unrelated.

    \item \textbf{What is the maximum permissible duration of a remotely controlled station’s transmissions if its control link malfunctions?}
    \begin{enumerate}
        \item 30 seconds.\\
        \item \textbf{3 minutes.}\\
        \item 5 minutes.\\
        \item 10 minutes.
    \end{enumerate}
    \textbf{Explanation:} Remotely controlled stations must stop transmitting within 3 minutes of losing control. Longer durations risk interference.

    \item \textbf{What is the highest modulation index permitted at the highest modulation frequency for angle modulation below 29.0 MHz?}
    \begin{enumerate}
        \item 0.5.\\
        \item \textbf{1.0.}\\
        \item 2.0.\\
        \item 3.0.
    \end{enumerate}
    \textbf{Explanation:} A modulation index of 1.0 ensures compliance with FCC standards. Other values listed are either too high or too low.

    \item \textbf{What is the maximum mean power level for a spurious emission below 30 MHz with respect to the fundamental emission?}
    \begin{enumerate}
        \item \textbf{-43 dB.}\\
        \item -53 dB.\\
        \item -63 dB.\\
        \item -73 dB.
    \end{enumerate}
    \textbf{Explanation:} Spurious emissions must be at least -43 dB below the fundamental level to comply with FCC regulations.

    \item \textbf{Which of the following operating arrangements allows an FCC-licensed US citizen to operate in many European countries, and amateurs from many European countries to operate in the US?}
    \begin{enumerate}
        \item \textbf{CEPT.}\\
        \item IARP.\\
        \item ITU reciprocal license.\\
        \item All these choices are correct.
    \end{enumerate}
    \textbf{Explanation:} CEPT agreements enable reciprocal operations across Europe and the US. Other options, while similar, do not provide the same wide access.

    \item \textbf{In what portion of the 630-meter band are phone emissions permitted?}
    \begin{enumerate}
        \item None.\\
        \item Only the top 3 kHz.\\
        \item Only the bottom 3 kHz.\\
        \item \textbf{The entire band.}
    \end{enumerate}
    \textbf{Explanation:} Phone emissions are permitted throughout the 630-meter band. Options A, B, and C incorrectly limit phone use to specific portions.
\end{enumerate}

%%%%%%%%%%% E1D %%%%%%%%%%%%%%%%%%%%%%%%%%
\section{Amateur Space and Earth Stations}

\subsection*{Exploring Space with Amateur Radio}
Amateur radio isn't just limited to Earth—it reaches into the vast expanse of space! This section covers the rules and concepts related to telemetry, telecommand, and operating amateur radio stations in space. Whether you're curious about satellites, balloons, or Earth stations, this is your guide to operating beyond our planet.

\subsubsection*{What is Telemetry?}
Telemetry involves the one-way transmission of measurement data from a device to a receiving station. For example:
\begin{itemize}
    \item A satellite sending temperature readings back to Earth.
    \item A weather balloon transmitting pressure and humidity data.
\end{itemize}
Telemetry is critical for monitoring devices in inaccessible locations, such as space stations or high-altitude balloons.

\subsubsection*{Encrypted Telecommand Signals}
Telecommand refers to sending signals to control a device remotely, such as modifying a satellite’s orbit or turning an instrument on or off. In amateur radio, encryption is allowed only for telecommand signals sent to space stations to ensure security and prevent unauthorized access.

\subsubsection*{Space Telecommand Stations}
A space telecommand station sends instructions to satellites or other space stations to initiate, modify, or terminate their functions. These stations are vital for managing satellite operations and ensuring their proper functioning.

\subsubsection*{Identification Requirements for Balloon Telemetry Stations}
Balloon-borne telemetry stations must identify themselves with:
\begin{itemize}
    \item The station’s call sign.
    \end{itemize}
This identification ensures compliance with FCC regulations and enables tracking of the station’s operations.

\subsubsection*{Posting Requirements for Telecommand Stations}
For telecommand stations located on or near Earth’s surface, the following must be posted at the station:
\begin{itemize}
    \item A photocopy of the station license.
    \item Contact information for the licensee and control operator.
\end{itemize}
This information provides transparency and facilitates accountability.

\subsubsection*{Operating Model Craft by Telecommand}
When using telecommand to operate model craft, such as drones or boats, the maximum transmitter output power is limited to 1 watt. This restriction minimizes interference with other radio services.

\subsubsection*{Amateur Radio Allocations for Space Stations}
Space stations can operate on various amateur radio bands. Here are the HF, VHF, and UHF allocations:
\begin{itemize}
    \item \textbf{HF Bands:} 40 meters, 20 meters, 15 meters, and 10 meters.
    \item \textbf{VHF Bands:} 2 meters.
    \item \textbf{UHF Bands:} 70 centimeters and 13 centimeters.
\end{itemize}
These allocations allow communication with satellites, space stations, and other amateur operators.

\subsubsection*{Telecommand and Earth Stations}
\begin{itemize}
    \item Any amateur station designated by the space station licensee may act as a telecommand station, provided it complies with the operator’s license privileges.
    \item Earth stations, which communicate with space stations, are open to all amateur licensees within their operating privileges.
\end{itemize}

\subsubsection*{One-Way Communications}
Certain amateur stations are allowed to transmit one-way communications, including:
\begin{itemize}
    \item Space stations.
    \item Beacon stations.
    \item Telecommand stations.
\end{itemize}
These transmissions are essential for providing information like propagation conditions or controlling devices remotely.

\begin{tcolorbox}[title=Fun Fact: The First Amateur Satellite]
Did you know? OSCAR-1 (Orbiting Satellite Carrying Amateur Radio), launched in 1961, was the first amateur satellite. It was built by volunteers and transmitted "HI" in Morse code to demonstrate amateur satellite communications.
\end{tcolorbox}

\subsection*{Key Takeaways}
Amateur radio operations in space open exciting opportunities for exploration and innovation. From telemetry and telecommand to one-way beacon transmissions, these operations extend the reach of amateur radio far beyond the atmosphere. Remember to follow FCC rules and enjoy your journey into the final frontier!


