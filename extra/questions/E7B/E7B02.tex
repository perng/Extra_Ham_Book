\subsection{Discovering Class D Amplifiers: The Future of Sound!}

\begin{tcolorbox}[colback=gray!10!white,colframe=black!75!black,title=Multiple Choice Question]
    \textbf{E7B02} What is a Class D amplifier?
    \begin{enumerate}[label=\Alph*,noitemsep]
        \item \textbf{An amplifier that uses switching technology to achieve high efficiency}
        \item A low power amplifier that uses a differential amplifier for improved linearity
        \item An amplifier that uses drift-mode FETs for high efficiency
        \item An amplifier biased to be relatively free from distortion
    \end{enumerate}
\end{tcolorbox}

\subsubsection{Intuitive Explanation}
Imagine you have a light switch. When you turn it on, the light is fully on, and when you turn it off, the light is fully off. A Class D amplifier works similarly but with sound. Instead of constantly adjusting the volume, it quickly switches the sound on and off. This switching happens so fast that it creates the sound we hear, and because it’s not always on, it uses less energy, making it very efficient.

\subsubsection{Advanced Explanation}
A Class D amplifier is a type of electronic amplifier that uses pulse-width modulation (PWM) or other switching techniques to amplify signals. Unlike traditional amplifiers (such as Class A, B, or AB) that operate in the linear region of their active components, Class D amplifiers operate in the switching region. This means the output transistors are either fully on or fully off, minimizing power loss and maximizing efficiency.

The input signal is converted into a series of pulses with varying widths (PWM), which are then amplified by the switching transistors. These pulses are filtered to reconstruct the original analog signal at the output. The efficiency of a Class D amplifier can exceed 90\%, compared to 50\% or less for traditional linear amplifiers.

Mathematically, the efficiency \(\eta\) of a Class D amplifier can be approximated by:
\[
\eta = \frac{P_{\text{out}}}{P_{\text{in}}}
\]
where \(P_{\text{out}}\) is the output power and \(P_{\text{in}}\) is the input power. Since the transistors are either fully on or off, the power dissipation is minimized, leading to high efficiency.

Class D amplifiers are widely used in applications where efficiency is critical, such as portable audio devices, subwoofers, and high-power audio systems.

% Diagram prompt: Generate a diagram showing the basic block diagram of a Class D amplifier, including the PWM modulator, switching transistors, and output filter.