\subsection{Class C Amplifier: Boosting Your SSB Signal!}

\begin{tcolorbox}[colback=gray!10!white,colframe=black!75!black,title=E7B07]
\textbf{E7B07} Which of the following is the likely result of using a Class C amplifier to amplify a single-sideband phone signal?
\begin{enumerate}[label=\Alph*.]
    \item Reduced intermodulation products
    \item Increased overall intelligibility
    \item Reduced third-order intermodulation
    \item \textbf{Signal distortion and excessive bandwidth}
\end{enumerate}
\end{tcolorbox}

\subsubsection{Intuitive Explanation}
Imagine you have a special machine that makes your voice louder, but this machine is designed to work best with simple, steady sounds like a single tone. Now, if you try to use this machine to make a more complex sound, like your voice, louder, it doesn’t work as well. Instead of making your voice clearer, it might make it sound weird and distorted. This is similar to what happens when you use a Class C amplifier with a single-sideband phone signal. The amplifier isn’t designed for this type of signal, so it ends up distorting the sound and making it harder to understand.

\subsubsection{Advanced Explanation}
A Class C amplifier is optimized for amplifying constant envelope signals, such as CW (Continuous Wave) or FM (Frequency Modulation) signals. However, single-sideband (SSB) signals are amplitude-modulated and have a varying envelope. When a Class C amplifier is used to amplify an SSB signal, it introduces significant distortion due to its non-linear operation. This distortion manifests as signal distortion and excessive bandwidth, which can degrade the quality of the transmitted signal.

Mathematically, the non-linear transfer function of a Class C amplifier can be represented as:
\[ V_{out} = a_1 V_{in} + a_2 V_{in}^2 + a_3 V_{in}^3 + \dots \]
where \( V_{in} \) is the input signal and \( V_{out} \) is the output signal. The higher-order terms (\( a_2 V_{in}^2, a_3 V_{in}^3, \dots \)) introduce harmonic distortion and intermodulation products, which are particularly problematic for SSB signals.

In summary, using a Class C amplifier for SSB signals is not recommended due to the inherent non-linearity of the amplifier, which leads to signal distortion and excessive bandwidth.

% Prompt for diagram: A diagram showing the input and output waveforms of a Class C amplifier when fed with an SSB signal, highlighting the distortion and bandwidth expansion.