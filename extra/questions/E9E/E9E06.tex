\subsection{Finding the Perfect Match: Q-Section for 100-Ohm to 50-Ohm!}

\begin{tcolorbox}[colback=gray!10!white,colframe=black!75!black,title=\textbf{E9E06}]
\textbf{Which of these transmission line impedances would be suitable for constructing a quarter-wave Q-section for matching a 100-ohm feed point impedance to a 50-ohm transmission line?}
\begin{enumerate}[label=\Alph*),noitemsep]
    \item 50 ohms
    \item 62 ohms
    \item \textbf{75 ohms}
    \item 90 ohms
\end{enumerate}
\end{tcolorbox}

\subsubsection{Intuitive Explanation}
Imagine you're trying to connect two pipes of different sizes. One pipe is big (100 ohms), and the other is small (50 ohms). You need a special adapter (the Q-section) to make sure water flows smoothly from the big pipe to the small pipe without any splashing or backflow. The adapter needs to be just the right size—not too big, not too small. In this case, the perfect size for the adapter is 75 ohms. It’s like Goldilocks finding the perfect porridge—just right!

\subsubsection{Advanced Explanation}
To match a 100-ohm feed point impedance to a 50-ohm transmission line using a quarter-wave Q-section, we use the formula for the characteristic impedance \( Z_0 \) of the Q-section:

\[
Z_0 = \sqrt{Z_{\text{in}} \cdot Z_{\text{out}}}
\]

Where:
\begin{itemize}
    \item \( Z_{\text{in}} = 100 \) ohms (the feed point impedance)
    \item \( Z_{\text{out}} = 50 \) ohms (the transmission line impedance)
\end{itemize}

Plugging in the values:

\[
Z_0 = \sqrt{100 \cdot 50} = \sqrt{5000} \approx 70.71 \text{ ohms}
\]

The closest standard impedance to 70.71 ohms is 75 ohms, which is why option C is the correct answer.

\subsubsection{Related Concepts}
\begin{itemize}
    \item \textbf{Quarter-Wave Transformer}: A transmission line of length \( \lambda/4 \) used to match impedances. The characteristic impedance of the transformer is the geometric mean of the two impedances to be matched.
    \item \textbf{Impedance Matching}: The process of making the impedance of a source equal to the impedance of the load to maximize power transfer and minimize reflections.
    \item \textbf{Transmission Line Theory}: The study of how electrical signals propagate along transmission lines, including the effects of impedance, reflection, and standing waves.
\end{itemize}

% Diagram Prompt: Generate a diagram showing a 100-ohm feed point connected to a 50-ohm transmission line via a 75-ohm quarter-wave Q-section.