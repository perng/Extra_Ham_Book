\subsection{Powering Up: The Savings of Switching Supplies!}

\begin{tcolorbox}[colback=gray!10!white,colframe=black!75!black,title=E7D10] Why is a switching type power supply less expensive and lighter than an equivalent linear power supply?
    \begin{enumerate}[label=\Alph*,noitemsep]
        \item The inverter design does not require an output filter circuit
        \item The control circuitry uses less current, therefore smaller heat sinks are required
        \item \textbf{The high frequency inverter design uses much smaller transformers and filter components for an equivalent power output}
        \item It recovers power from the unused portion of the AC cycle, thus using fewer components
    \end{enumerate}
\end{tcolorbox}

\subsubsection{Intuitive Explanation}
Imagine you have two types of power supplies: one is like a big, heavy truck, and the other is like a small, fast car. The big truck is the linear power supply, and the small car is the switching power supply. The linear power supply works by using a lot of energy to get the job done, which makes it heavy and expensive. On the other hand, the switching power supply works smarter, not harder. It uses high-frequency switching to do the same job but with much smaller and lighter parts. This makes it cheaper and easier to carry around.

\subsubsection{Advanced Explanation}
Switching power supplies operate at high frequencies, typically in the range of tens to hundreds of kilohertz. This high-frequency operation allows for the use of significantly smaller transformers and filter components compared to linear power supplies, which operate at the mains frequency (50 or 60 Hz). The size of a transformer is inversely proportional to the frequency of operation, as given by the equation:

\[
L = \frac{N^2 \mu A}{l}
\]

where \( L \) is the inductance, \( N \) is the number of turns, \( \mu \) is the permeability of the core, \( A \) is the cross-sectional area, and \( l \) is the magnetic path length. At higher frequencies, the required inductance \( L \) is smaller, allowing for smaller transformers.

Additionally, the efficiency of switching power supplies is higher because they minimize power loss through heat dissipation. Linear power supplies dissipate excess energy as heat, requiring larger heat sinks and more robust components. In contrast, switching power supplies use semiconductor switches (like MOSFETs) to rapidly turn the current on and off, reducing energy loss and allowing for smaller, lighter components.

The high-frequency operation also simplifies the design of the output filter circuit, as smaller capacitors and inductors can be used to achieve the same filtering effect. This further reduces the overall size, weight, and cost of the power supply.

% Diagram Prompt: Generate a diagram comparing the size and components of a linear power supply versus a switching power supply, highlighting the smaller transformers and filter components in the switching supply.