\subsection{Cabrillo Format: Unlocking Contest Logging Fun!}

\begin{tcolorbox}[colback=gray!10!white,colframe=black!75!black,title=Multiple Choice Question]
\textbf{E2C07} What is the Cabrillo format?

\begin{enumerate}[label=\Alph*.]
    \item \textbf{A standard for submission of electronic contest logs}
    \item A method of exchanging information during a contest QSO
    \item The most common set of contest rules
    \item A digital protocol specifically designed for rapid contest exchanges
\end{enumerate}
\end{tcolorbox}

\subsubsection{Intuitive Explanation}
Imagine you’re participating in a big radio contest where you need to log all your contacts. Instead of writing everything down on paper, you use a special format called Cabrillo. It’s like a digital notebook that helps you organize all your contest information in a way that’s easy to share with the contest organizers. This way, they can quickly check your logs and see how well you did!

\subsubsection{Advanced Explanation}
The Cabrillo format is a standardized text-based format used for submitting electronic logs in amateur radio contests. It was developed to streamline the process of logging and submitting contest results. The format includes specific fields for essential information such as the call sign, contest name, date, time, frequency, mode, and the exchanged information (e.g., signal report, serial number).

Here’s a brief example of how a Cabrillo log entry might look:

\begin{verbatim}
START-OF-LOG: 3.0
CALLSIGN: W1AW
CONTEST: ARRL-DX
QSO: 14000 CW 2023-10-01 1200 W1AW 599 001 K1ABC 599 001
END-OF-LOG:
\end{verbatim}

In this example, the log entry records a QSO (contact) made on 14 MHz using CW (Morse code) on October 1, 2023, at 12:00 UTC. The exchanged information includes signal reports (599) and serial numbers (001).

The Cabrillo format ensures consistency and compatibility across different logging software and contest management systems, making it easier for contest organizers to process and verify logs.

% Prompt for generating a diagram: 
% A diagram showing a sample Cabrillo log entry with labeled fields could help visualize the structure of the format.