\subsection{Discover the Through-Hole Treasure!}

\begin{tcolorbox}[colback=gray!10!white,colframe=black!75!black,title=Multiple Choice Question]
\textbf{E6E02} Which of the following device packages is a through-hole type?
\begin{enumerate}[label=\Alph*),noitemsep]
    \item \textbf{DIP}
    \item PLCC
    \item BGA
    \item SOT
\end{enumerate}
\end{tcolorbox}

\subsubsection*{Intuitive Explanation}
Imagine you have a piece of paper with holes in it, and you want to stick something through those holes so it stays in place. A through-hole device package is like that! It has pins that go through holes on a circuit board, making it easy to attach and secure. Out of the options, DIP (Dual In-line Package) is the one that uses this method. The other options, like PLCC, BGA, and SOT, are different types of packages that don’t use through-holes.

\subsubsection*{Advanced Explanation}
Through-hole technology (THT) involves mounting electronic components by inserting their leads (pins) into holes drilled in a printed circuit board (PCB) and then soldering them in place. This method provides strong mechanical bonds and is often used for components that require high reliability or are subject to mechanical stress.

Among the given options:
\begin{itemize}
    \item \textbf{DIP (Dual In-line Package)}: A through-hole package with two parallel rows of pins. It is widely used for integrated circuits (ICs) and is known for its ease of use in prototyping and repair.
    \item PLCC (Plastic Leaded Chip Carrier): A surface-mount package with leads on all four sides, bent under the package. It does not use through-holes.
    \item BGA (Ball Grid Array): A surface-mount package that uses an array of solder balls for connection. It is not a through-hole type.
    \item SOT (Small Outline Transistor): A surface-mount package for small components like transistors. It does not use through-holes.
\end{itemize}

Thus, the correct answer is \textbf{DIP}, as it is the only through-hole package listed.

% Prompt for diagram: A diagram showing a DIP package with pins inserted through a PCB, compared to surface-mount packages like PLCC, BGA, and SOT.