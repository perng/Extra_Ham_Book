\subsection{E9A08: Exploring the Tiny Wonders: The Smallest First Fresnel Zone!}

\begin{tcolorbox}[colback=gray!10!white,colframe=black!75!black]
    \textbf{E9A08} Which frequency band has the smallest first Fresnel zone?
    \begin{enumerate}[label=\Alph*,noitemsep]
        \item \textbf{5.8 GHz}
        \item 3.4 GHz
        \item 2.4 GHz
        \item 900 MHz
    \end{enumerate}
\end{tcolorbox}

\subsubsection{Intuitive Explanation}
Imagine you’re throwing a ball through a hula hoop. The size of the hula hoop depends on how fast you throw the ball. If you throw it really fast (like a high frequency), the hula hoop is small. If you throw it slowly (like a low frequency), the hula hoop is big. The first Fresnel zone is like that hula hoop for radio waves. The higher the frequency, the smaller the Fresnel zone. So, 5.8 GHz, being the highest frequency here, has the smallest Fresnel zone. Easy peasy!

\subsubsection{Advanced Explanation}
The first Fresnel zone is a critical concept in radio wave propagation, representing the area around the direct line of sight between two antennas where the signal is most concentrated. The radius \( r \) of the first Fresnel zone at a point along the path is given by:

\[
r = \sqrt{\frac{n \lambda d_1 d_2}{d_1 + d_2}}
\]

where:
\begin{itemize}
    \item \( n \) is the Fresnel zone number (1 for the first Fresnel zone),
    \item \( \lambda \) is the wavelength of the signal,
    \item \( d_1 \) and \( d_2 \) are the distances from the point to the two antennas.
\end{itemize}

The wavelength \( \lambda \) is inversely proportional to the frequency \( f \):

\[
\lambda = \frac{c}{f}
\]

where \( c \) is the speed of light (\( 3 \times 10^8 \) m/s). Therefore, as the frequency increases, the wavelength decreases, leading to a smaller Fresnel zone. Among the given options, 5.8 GHz has the highest frequency, resulting in the smallest first Fresnel zone.

% Prompt for diagram: A diagram showing the Fresnel zone with different frequencies and their corresponding zone sizes would be helpful here.