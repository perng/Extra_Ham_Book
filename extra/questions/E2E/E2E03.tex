\subsection{Unpacking the Mystery of FT4!}

\begin{tcolorbox}[colback=gray!10!white,colframe=black!75!black,title=E2E03] To what does the 4 in FT4 refer?
    \begin{enumerate}[label=\Alph*,noitemsep]
        \item Multiples of 4 bits of user information
        \item \textbf{Four-tone continuous-phase frequency shift keying}
        \item Four transmit/receive cycles per minute
        \item All these choices are correct
    \end{enumerate}
\end{tcolorbox}

\subsubsection{Intuitive Explanation}
Imagine you are sending a message using different musical notes. In FT4, the 4 refers to the fact that there are four different notes (or tones) used to send the message. These notes are carefully chosen so that they smoothly transition from one to the next, making it easier for the receiver to understand the message. This method is called continuous-phase frequency shift keying, which is just a fancy way of saying that the notes change in a smooth, continuous way.

\subsubsection{Advanced Explanation}
FT4 is a digital communication mode used in amateur radio. The 4 in FT4 specifically refers to the modulation technique employed, which is \textbf{four-tone continuous-phase frequency shift keying (CPFSK)}. In this method, four distinct tones are used to represent different symbols in the digital signal. The continuous-phase aspect ensures that the phase of the signal does not abruptly change between symbols, which minimizes spectral bandwidth and reduces the likelihood of interference.

Mathematically, the signal can be represented as:
\[ s(t) = A \cos\left(2\pi f_c t + 2\pi h \int_{-\infty}^{t} m(\tau) \, d\tau\right) \]
where:
\begin{itemize}
    \item \( A \) is the amplitude of the signal,
    \item \( f_c \) is the carrier frequency,
    \item \( h \) is the modulation index,
    \item \( m(t) \) is the message signal.
\end{itemize}

In FT4, the message signal \( m(t) \) is encoded using four different tones, each representing a specific symbol. This allows for efficient and reliable communication, especially in noisy environments.

% [Prompt for generating a diagram: A diagram showing the four tones used in FT4 and how they transition smoothly from one to the next would be helpful here.]