\subsection{Discovering the FST4 Mode: What's Special?}

\begin{tcolorbox}[colback=gray!10!white,colframe=black!75!black,title=E2E04] Which of the following is characteristic of the FST4 mode?
    \begin{enumerate}[label=\Alph*)]
        \item Four-tone Gaussian frequency shift keying
        \item Variable transmit/receive periods
        \item Seven different tone spacings
        \item \textbf{All these choices are correct}
    \end{enumerate}
\end{tcolorbox}

\subsubsection{Intuitive Explanation}
The FST4 mode is a special way of sending and receiving radio signals. It has some unique features that make it stand out. First, it uses a method called Four-tone Gaussian frequency shift keying, which is a fancy way of saying it changes the frequency of the signal in a smooth and controlled manner. Second, it has Variable transmit/receive periods, meaning it can adjust how long it sends and listens for signals. Lastly, it offers Seven different tone spacings, which means it can use different distances between the tones it sends. All these features together make the FST4 mode very versatile and useful.

\subsubsection{Advanced Explanation}
The FST4 mode is a digital modulation scheme used in amateur radio. It employs Four-tone Gaussian Frequency Shift Keying (GFSK), which is a type of frequency modulation where the signal shifts between four different frequencies in a Gaussian-filtered manner. This results in a smoother transition between frequencies, reducing spectral splatter and improving signal clarity.

Additionally, FST4 supports variable transmit/receive periods, allowing operators to adjust the timing of their transmissions and receptions based on the specific requirements of their communication. This flexibility can be particularly useful in optimizing performance under varying propagation conditions.

Furthermore, FST4 offers seven different tone spacings, which refer to the frequency separation between the tones used in the modulation. This variety allows for different data rates and bandwidths, making FST4 adaptable to different communication needs.

In summary, the FST4 mode combines these three characteristics—Four-tone GFSK, variable transmit/receive periods, and seven different tone spacings—to provide a robust and flexible communication method for amateur radio operators.

% [Prompt for generating a diagram: A diagram showing the frequency spectrum of FST4 mode with four tones, variable periods, and different tone spacings would be helpful for visualization.]