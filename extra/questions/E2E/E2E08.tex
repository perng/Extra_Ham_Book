\subsection{Unlocking Digital Fun: Which HF Modes Transfer Files?}

\begin{tcolorbox}[colback=gray!10!white,colframe=black!75!black,title=E2E08] Which of the following HF digital modes can be used to transfer binary files?
    \begin{enumerate}[label=\Alph*),noitemsep]
        \item PSK31
        \item \textbf{PACTOR}
        \item RTTY
        \item AMTOR
    \end{enumerate}
\end{tcolorbox}

\subsubsection{Intuitive Explanation}
Imagine you want to send a picture or a document over the radio. Not all radio modes can handle this kind of data. Some modes are like sending a letter—they can only handle text. But others are like sending a package—they can handle more complex things like pictures or documents. PACTOR is one of those modes that can send packages, which means it can transfer binary files like pictures or documents. PSK31, RTTY, and AMTOR are more like sending letters—they’re great for text but not for files.

\subsubsection{Advanced Explanation}
In the context of High Frequency (HF) digital modes, the ability to transfer binary files depends on the protocol's design and error correction capabilities. PACTOR (Packet Teleprinting Over Radio) is a robust digital mode specifically designed for data transfer, including binary files. It employs advanced error correction and data compression techniques, making it suitable for file transfers over HF bands.

PSK31 (Phase Shift Keying, 31 Baud) is primarily a text-based mode optimized for low bandwidth and efficient text communication. It lacks the necessary protocols for binary file transfer. Similarly, RTTY (Radio Teletype) and AMTOR (Amateur Teleprinting Over Radio) are also text-oriented modes. RTTY uses frequency-shift keying (FSK) for text transmission, while AMTOR adds error detection and correction but is still limited to text.

To summarize, PACTOR is the only mode among the options that supports binary file transfer due to its advanced data handling capabilities.

% Diagram Prompt: Generate a diagram comparing the data handling capabilities of PACTOR, PSK31, RTTY, and AMTOR.