\subsection{Bandwidth Showdown: Which Digital Mode Wins?}

\begin{tcolorbox}[colback=blue!5!white,colframe=blue!75!black]
    \textbf{E2E10} Which of these digital modes has the narrowest bandwidth?
    \begin{enumerate}[label=\Alph*,noitemsep]
        \item MFSK16
        \item 170 Hz shift, 45-baud RTTY
        \item \textbf{FT8}
        \item PACTOR IV
    \end{enumerate}
\end{tcolorbox}

\subsubsection{Intuitive Explanation}
Imagine you are trying to send a message through a narrow tunnel. The narrower the tunnel, the less space you have to send your message. In the world of radio, bandwidth is like the width of that tunnel. The narrower the bandwidth, the less space the signal takes up. Among the options given, FT8 is like the narrowest tunnel, allowing it to send messages using the least amount of space.

\subsubsection{Advanced Explanation}
Bandwidth in digital modes refers to the range of frequencies occupied by a signal. The narrower the bandwidth, the more efficient the mode is in terms of spectrum usage. 

- \textbf{MFSK16}: This mode uses multiple frequency shifts and typically has a bandwidth of around 316 Hz.
- \textbf{170 Hz shift, 45-baud RTTY}: This mode uses a frequency shift of 170 Hz and operates at 45 baud, resulting in a bandwidth of approximately 250 Hz.
- \textbf{FT8}: This mode is designed for weak signal communication and has a bandwidth of only 50 Hz, making it the narrowest among the options.
- \textbf{PACTOR IV}: This mode is more complex and typically has a bandwidth of around 2.4 kHz.

The calculation for bandwidth can be complex, but for FT8, it is specifically designed to operate within a very narrow frequency range, which is why it has the narrowest bandwidth.

% Diagram Prompt: Generate a diagram showing the bandwidth comparison of MFSK16, 170 Hz shift RTTY, FT8, and PACTOR IV on a frequency spectrum.