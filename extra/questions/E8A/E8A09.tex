\subsection{Unlocking the Magic of 8-Bit: How Many Levels Can We Encode?}

\begin{tcolorbox}[colback=yellow!10!white, colframe=black, title=Question ID: \textbf{E8A09}]
How many different input levels can be encoded by an analog-to-digital converter with 8-bit resolution?
\begin{enumerate}[label=\Alph*),noitemsep]
    \item 8
    \item 8 multiplied by the gain of the input amplifier
    \item 256 divided by the gain of the input amplifier
    \item \textbf{256}
\end{enumerate}
\end{tcolorbox}

\subsubsection{Intuitive Explanation}
Imagine you have a box of crayons, each crayon representing a different color. If you have 8 crayons, you can make 8 different pictures, each with its own unique color. Now, think about an analog-to-digital converter (ADC) like a really smart camera that can take pictures of colors. If this camera can remember 256 different colors, it means it has a capability to make 256 different pictures based on the colors it sees. So, when we talk about 8-bit resolution, we are saying that this smart camera can have a total of 256 different colors (or levels) that it can use to capture and remember what it sees!

\subsubsection{Advanced Explanation}
An analog-to-digital converter (ADC) with an 8-bit resolution can represent a specific number of discrete levels or values in its output. The main principle to understand here is that the number of different levels an ADC can encode is calculated based on the total combinations of bits it has. 

Since 1 bit can represent 2 values (0 or 1), for \( n \) bits, the total number of distinct levels (or combinations) will be given by the formula:
\[
\text{Number of Levels} = 2^n
\]
For an ADC with 8 bits, the calculation would be:
\[
\text{Number of Levels} = 2^8 = 256
\]
Thus, an ADC with 8-bit resolution can encode a total of 256 different input levels.

In the context of signal processing or digital communications, these 256 discrete levels allow the conversion of an analog signal (which can take any value within a range) into a binary representation that can be processed digitally. Each level corresponds to a unique combination of bits.

\noindent Additional concepts related to this topic include quantization and sampling, which are fundamental in digital signal processing. Quantization refers to the approximation of the analog signal to the nearest available level, while sampling refers to the rate at which the analog signal is measured or sampled over time.

% Diagram Prompt: Generate a diagram showing a 1-8 bits binary representation with the corresponding decimal values and levels encoded by an 8-bit ADC, highlighting the range of 0-255.