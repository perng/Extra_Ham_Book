\subsection{Creating SSB Signals with Digital Magic!}
\label{sec:E7F04}

\begin{tcolorbox}[colback=gray!10!white,colframe=black!75!black,title=\textbf{E7F04}]
Which method generates an SSB signal using digital signal processing?
\begin{enumerate}[label=\Alph*.]
    \item Mixing products are converted to voltages and subtracted by adder circuits
    \item A frequency synthesizer removes unwanted sidebands
    \item Varying quartz crystal characteristics are emulated in digital form
    \item \textbf{Signals are combined in quadrature phase relationship}
\end{enumerate}
\end{tcolorbox}

\subsubsection{Intuitive Explanation}
Imagine you have two friends who are singing the same song, but one starts singing a little later than the other. If you combine their voices in a special way, you can create a new sound that only has one part of the song, not both. This is similar to how Single Sideband (SSB) signals are created using digital signal processing. By combining signals in a specific timing (called quadrature phase relationship), we can keep only one side of the signal and remove the other, making the signal cleaner and more efficient.

\subsubsection{Advanced Explanation}
In digital signal processing, generating an SSB signal involves the use of quadrature phase relationships. This method leverages the Hilbert transform to create a phase-shifted version of the original signal. When the original signal and its Hilbert transform are combined, one of the sidebands is canceled out, leaving only the desired sideband.

Mathematically, if we have a signal \( x(t) \), its Hilbert transform \( \hat{x}(t) \) is given by:
\[
\hat{x}(t) = \frac{1}{\pi} \int_{-\infty}^{\infty} \frac{x(\tau)}{t - \tau} d\tau
\]
The SSB signal \( s(t) \) can then be generated by:
\[
s(t) = x(t) \cos(\omega_c t) \mp \hat{x}(t) \sin(\omega_c t)
\]
where \( \omega_c \) is the carrier frequency, and the sign determines which sideband is retained.

This method is efficient and widely used in digital communication systems to reduce bandwidth and improve signal clarity.

% [Diagram Prompt: A diagram showing the process of combining signals in quadrature phase relationship to generate an SSB signal would be helpful here.]