\subsection{Unlocking the Magic of Fast Fourier Transform!}

\begin{tcolorbox}[colback=gray!10!white,colframe=black!75!black,title=\textbf{E7F07}]
\textbf{What function is performed by a Fast Fourier Transform?}
\begin{enumerate}[label=\Alph*)]
    \item Converting analog signals to digital form
    \item Converting digital signals to analog form
    \item \textbf{Converting signals from the time domain to the frequency domain}
    \item Converting signals from the frequency domain to the time domain
\end{enumerate}
\end{tcolorbox}

\subsubsection{Intuitive Explanation}
Imagine you have a song playing on your radio. The song is a mix of different sounds and frequencies, like bass, drums, and vocals. The Fast Fourier Transform (FFT) is like a magical tool that helps us break down this song into its individual parts. Instead of hearing the song as one continuous sound, the FFT shows us the different frequencies that make up the song. It’s like turning a smoothie back into its original fruits and vegetables!

\subsubsection{Advanced Explanation}
The Fast Fourier Transform (FFT) is an algorithm that efficiently computes the Discrete Fourier Transform (DFT) of a sequence. The DFT converts a finite sequence of equally-spaced samples of a function into a sequence of coefficients of a finite combination of complex sinusoids, ordered by their frequencies. Mathematically, the DFT of a sequence \( x(n) \) of length \( N \) is given by:

\[
X(k) = \sum_{n=0}^{N-1} x(n) \cdot e^{-i 2 \pi k n / N}
\]

where \( X(k) \) represents the frequency domain representation of the signal \( x(n) \). The FFT reduces the computational complexity of the DFT from \( O(N^2) \) to \( O(N \log N) \), making it practical for real-time signal processing applications.

The FFT is widely used in various fields such as audio processing, image processing, and telecommunications to analyze the frequency components of signals. It allows us to transform a signal from the time domain, where we see how the signal changes over time, to the frequency domain, where we can see the different frequencies that make up the signal.

% Prompt for diagram: Generate a diagram showing a time-domain signal being transformed into its frequency-domain representation using the FFT.