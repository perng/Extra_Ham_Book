\subsection{E9H03: Exploring the Joy of Receiving Directivity Factor (RDF)!}

\begin{tcolorbox}[colback=blue!5!white,colframe=blue!75!black]
    \textbf{E9H03} What is receiving directivity factor (RDF)? 
    \begin{enumerate}[label=\Alph*),noitemsep]
        \item Forward gain compared to the gain in the reverse direction
        \item Relative directivity compared to isotropic
        \item Relative directivity compared to a dipole
        \item \textbf{Peak antenna gain compared to average gain over the hemisphere around and above the antenna}
    \end{enumerate}
\end{tcolorbox}

\subsubsection*{Intuitive Explanation}
Imagine you have a flashlight. The receiving directivity factor (RDF) is like comparing how bright the flashlight is at its brightest spot to how bright it is on average when you shine it all around in a half-circle. If the flashlight is super bright in one direction but not so bright in others, it has a high RDF. In radio terms, it’s about how good an antenna is at picking up signals from one direction compared to all directions around it. So, RDF is like the antenna’s superpower to focus on one spot!

\subsubsection*{Advanced Explanation}
The receiving directivity factor (RDF) is a measure of an antenna's ability to receive signals from a specific direction compared to the average reception over a hemisphere. Mathematically, it is defined as the ratio of the peak antenna gain \( G_{\text{peak}} \) to the average gain \( G_{\text{avg}} \) over the hemisphere surrounding the antenna:

\[
\text{RDF} = \frac{G_{\text{peak}}}{G_{\text{avg}}}
\]

Here, \( G_{\text{avg}} \) is calculated by integrating the gain over the hemisphere and dividing by the solid angle of the hemisphere (\( 2\pi \) steradians):

\[
G_{\text{avg}} = \frac{1}{2\pi} \int_{0}^{2\pi} \int_{0}^{\pi/2} G(\theta, \phi) \sin\theta \, d\theta \, d\phi
\]

Where \( G(\theta, \phi) \) is the gain in the direction specified by the angles \( \theta \) and \( \phi \). The RDF is a crucial parameter in antenna design, as it quantifies how effectively an antenna can focus its reception in a particular direction, minimizing interference from other directions.

% Prompt for diagram: A diagram showing an antenna with a hemisphere around it, highlighting the peak gain direction and the average gain over the hemisphere would be helpful here.