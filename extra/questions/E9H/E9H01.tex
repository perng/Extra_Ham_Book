\subsection{E9H01: Designing a Beverage Antenna for Success!}

\begin{tcolorbox}[colback=gray!10!white,colframe=black!75!black,title=Multiple Choice Question]
    \textbf{E9H01} When constructing a Beverage antenna, which of the following factors should be included in the design to achieve good performance at the desired frequency?
    \begin{enumerate}[label=\Alph*.]
        \item Its overall length must not exceed 1/4 wavelength
        \item It must be mounted more than 1 wavelength above ground
        \item It should be configured as a four-sided loop
        \item \textbf{It should be at least one wavelength long}
    \end{enumerate}
\end{tcolorbox}

\subsubsection{Intuitive Explanation}
Imagine you're trying to catch a wave in the ocean. If your surfboard is too short, you'll miss the wave entirely. Similarly, a Beverage antenna needs to be long enough to catch the radio waves effectively. Think of it as a giant fishing net for radio signals. If the net is too short, the fish (or in this case, the radio waves) will just swim right past it. So, to make sure your antenna works well, it needs to be at least one wavelength long. This way, it can catch the radio waves properly and give you a strong signal.

\subsubsection{Advanced Explanation}
The Beverage antenna is a type of long-wire antenna that is primarily used for receiving low-frequency signals. The key to its performance lies in its length relative to the wavelength of the desired frequency. The wavelength (\(\lambda\)) of a signal is given by the formula:

\[
\lambda = \frac{c}{f}
\]

where \(c\) is the speed of light (\(3 \times 10^8\) m/s) and \(f\) is the frequency of the signal.

For optimal performance, the Beverage antenna should be at least one wavelength long. This ensures that the antenna can effectively capture the electromagnetic waves at the desired frequency. If the antenna is shorter than one wavelength, it will not be able to fully interact with the incoming waves, leading to poor reception.

Additionally, the Beverage antenna is typically mounted close to the ground, which helps in reducing noise and improving signal clarity. The antenna's length and its proximity to the ground are critical factors in its design, ensuring that it can efficiently receive signals over long distances.

% Diagram Prompt: Generate a diagram showing a Beverage antenna mounted close to the ground, with its length labeled as at least one wavelength.