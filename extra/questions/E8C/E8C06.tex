\subsection{FT8 Signal Bandwidth: What to Know!}

\begin{tcolorbox}
\textbf{Question ID: E8C06} \\
What is the bandwidth of an FT8 signal? \\
\begin{enumerate}[label=\Alph*.]
    \item 10 Hz
    \item \textbf{50 Hz}
    \item 600 Hz
    \item 2.4 kHz
\end{enumerate}
\end{tcolorbox}

\subsubsection{Intuitive Explanation}
Imagine you are trying to listen to music on the radio. Each radio station has its own frequency, and sometimes, multiple stations can be close together. The bandwidth of a signal tells us how much space it takes up in the radio spectrum. 

In the case of an FT8 signal, which is a digital communication mode used by amateur radio operators, the bandwidth is like the width of a road that allows cars (the signals) to travel. If the road is very wide, many cars can pass at the same time. For FT8, this width is 50 Hz. This means that the FT8 signal doesn’t take up too much space on the radio, allowing other signals to share the spectrum.

\subsubsection{Advanced Explanation}
The bandwidth of a communication signal is defined as the difference between the upper and lower frequencies of the signal's transmission. For FT8, an advanced digital mode used for weak signal communication in amateur radio, the calculated bandwidth is 50 Hz.

To understand how we arrive at this, let's look deeper into modulation schemes. FT8 operates within a narrow band, meaning it can transmit and receive data in limited frequency ranges. Each communication mode has a preferred bandwidth, which is generally defined by the modulation technique used.

FT8 employs a phase shift keying technique that allows for efficient use of the available bandwidth. If we consider transmission specifications, we may analyze the actual signal and its spectral representation. The operational frequency range contributes to determining the bandwidth defined as:

\[
\text{Bandwidth} = f_{\text{high}} - f_{\text{low}}
\]

In the case of FT8:

\[
\text{Bandwidth} = 50 \text{ Hz} - 0 \text{ Hz} = 50 \text{ Hz}
\]

This efficient modulation scheme allows FT8 to effectively communicate even in extremely low signal conditions, making it popular among amateur radio operators.

% Generate diagram: Diagram showing the frequency spectrum with FT8 bandwidth highlighted