\subsection{Exploring Node Addresses in Mesh Networks!}

\begin{tcolorbox}
\textbf{Question ID: E8C14} \\
What type of addresses do nodes have in a mesh network? 
\begin{enumerate}[label=\Alph*.]
    \item Email
    \item Trust server
    \item \textbf{Internet Protocol (IP)}
    \item Talk group
\end{enumerate}
\end{tcolorbox}

\subsubsection{Intuitive Explanation}
In a mesh network, the devices that connect to each other use special addresses to find one another, just like you might use someone's home address to send them a letter. In this case, the type of address they use is called an Internet Protocol (IP) address. This is similar to how every house has a unique address so that mail can be delivered correctly. So, when we talk about addresses in a mesh network, think of it as the unique identifiers that help each device communicate effectively with other devices.

\subsubsection{Advanced Explanation}
A mesh network is a type of network topology where each node (device) relays data for the network. In this architecture, each node has a unique Internet Protocol (IP) address that allows it to send and receive data. The IP address is a numerical label assigned to each device connected to a computer network that uses the Internet Protocol for communication.

To understand what an IP address is, consider the following:

1. \textbf(IPv4 and IPv6:) 
   - An IP address can be of two versions, IPv4 or IPv6. IPv4 addresses are in the format of four decimal numbers separated by dots (e.g., 192.168.1.1), while IPv6 addresses are longer and written in hexadecimal (e.g., 2001:0db8:85a3:0000:0000:8a2e:0370:7334). Mesh nodes, like all devices on a network, must have IP addresses to be part of the network.

2. \textbf(Datagram Delivery:)
   - Each packet of data sent across a network contains the destination IP address, allowing for the correct routing of the information. Routers and switches in the network use these addresses to forward packets to their intended destinations.

3. \textbf(Addressing Scheme:)
   - In a mesh network, nodes can communicate directly with one another without depending on a central network controller. Each node maintains a routing table with the addresses of its neighbors, making it efficient for data transmission.

To summarize, in a mesh network, nodes utilize Internet Protocol (IP) addresses to identify themselves and communicate effectively within the network.

% Diagram prompt: Generate a diagram showing a mesh network with nodes connected to each other, labeled with IP addresses to demonstrate how they communicate.