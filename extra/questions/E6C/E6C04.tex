\subsection{Discover the Perks of BiCMOS Logic!}

\begin{tcolorbox}[colback=gray!10!white,colframe=black!75!black,title=E6C04] Which of the following is an advantage of BiCMOS logic?
    \begin{enumerate}[label=\Alph*,noitemsep]
        \item Its simplicity results in much less expensive devices than standard CMOS
        \item It is immune to electrostatic damage
        \item \textbf{It has the high input impedance of CMOS and the low output impedance of bipolar transistors}
        \item All these choices are correct
    \end{enumerate}
\end{tcolorbox}

\subsubsection{Intuitive Explanation}
Imagine you have a device that combines the best features of two different technologies. BiCMOS logic is like that! It takes the high input impedance from CMOS, which means it doesn't need much power to control the input, and the low output impedance from bipolar transistors, which means it can drive strong signals without losing much energy. This combination makes BiCMOS logic very efficient and powerful, like having a car that is both fuel-efficient and fast.

\subsubsection{Advanced Explanation}
BiCMOS (Bipolar Complementary Metal-Oxide-Semiconductor) logic integrates the advantages of both CMOS and bipolar transistor technologies. CMOS technology is known for its high input impedance, which means it requires very little input current to control the gate. This results in low power consumption, especially in static conditions. On the other hand, bipolar transistors offer low output impedance, allowing them to drive large currents with minimal voltage drop, which is beneficial for high-speed and high-current applications.

The combination of these two technologies in BiCMOS logic results in circuits that have the high input impedance of CMOS and the low output impedance of bipolar transistors. This hybrid approach provides several advantages:
\begin{itemize}
    \item \textbf{High Input Impedance}: Reduces power consumption and allows for easier interfacing with other circuits.
    \item \textbf{Low Output Impedance}: Enables the driving of large loads with minimal signal degradation, which is crucial for high-speed and high-power applications.
\end{itemize}

Mathematically, the input impedance \( Z_{in} \) of a CMOS gate is typically very high, often in the range of \( 10^{12} \) ohms, while the output impedance \( Z_{out} \) of a bipolar transistor can be as low as a few ohms. This combination allows BiCMOS circuits to efficiently manage both input and output signals, making them highly versatile in various electronic applications.

% Diagram Prompt: Generate a diagram showing the integration of CMOS and bipolar transistors in a BiCMOS circuit, highlighting the high input impedance and low output impedance.