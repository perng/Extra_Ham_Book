\subsection{Spot the NAND Gate: A Fun Symbol Challenge!}

\begin{tcolorbox}[colback=gray!10!white,colframe=black!75!black,title=E6C08] In Figure E6-3, which is the schematic symbol for a NAND gate? 
    \begin{enumerate}[label=\Alph*)]
        \item 1
        \item \textbf{2}
        \item 3
        \item 4
    \end{enumerate}
\end{tcolorbox}

\subsubsection*{Intuitive Explanation}
Imagine you have a magic box that says NO to everything unless both of its two friends say YES at the same time. This magic box is called a NAND gate. In Figure E6-3, you need to find the symbol that represents this magic box. The correct symbol is the one that looks like a combination of an AND gate (which says YES only if both friends say YES) with a little circle at the end that flips the answer to NO. So, the correct answer is the symbol labeled B.

\subsubsection*{Advanced Explanation}
A NAND gate is a digital logic gate that performs the logical NAND operation. The NAND operation is a combination of the AND operation followed by a NOT operation. Mathematically, the output \( Y \) of a NAND gate with inputs \( A \) and \( B \) is given by:
\[
Y = \overline{A \cdot B}
\]
where \( \overline{A \cdot B} \) represents the negation of the AND operation.

In schematic diagrams, the NAND gate is represented by the symbol of an AND gate with a small circle (also known as a bubble) at the output. This bubble indicates the negation operation. Therefore, the correct symbol for a NAND gate in Figure E6-3 is the one labeled B, which shows an AND gate with a bubble at its output.

The NAND gate is significant in digital electronics because it is a universal gate, meaning that any other logic gate (AND, OR, NOT, etc.) can be constructed using only NAND gates. This property makes NAND gates fundamental in the design of digital circuits.

% Prompt for generating the diagram:
% Include Figure E6-3 showing the schematic symbols for various logic gates, with labels A, B, C, and D corresponding to the choices. Highlight the NAND gate symbol (B) for clarity.