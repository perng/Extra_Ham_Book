\subsection{Unraveling the Noise Resistance of CMOS Circuits!}

\begin{tcolorbox}[colback=gray!10!white,colframe=black!75!black,title=E6C06] Why do CMOS digital integrated circuits have high immunity to noise on the input signal or power supply?
    \begin{enumerate}[label=\Alph*,noitemsep]
        \item Large bypass capacitance is inherent
        \item The input switching threshold is about twice the power supply voltage
        \item \textbf{The input switching threshold is about half the power supply voltage}
        \item Bandwidth is very limited
    \end{enumerate}
\end{tcolorbox}

\subsubsection{Intuitive Explanation}
Imagine you have a light switch that turns on when you push it halfway. If someone accidentally bumps the switch a little, it won’t turn on or off unless the bump is strong enough to push it past the halfway point. CMOS circuits work similarly. They have a switching threshold at about half the power supply voltage. This means that small noises or bumps in the input signal or power supply won’t accidentally turn the circuit on or off unless the noise is strong enough to cross this halfway point. This makes CMOS circuits very resistant to noise.

\subsubsection{Advanced Explanation}
CMOS (Complementary Metal-Oxide-Semiconductor) digital integrated circuits exhibit high noise immunity due to their input switching threshold, which is typically around half the power supply voltage (\(V_{DD}/2\)). This threshold is crucial because it defines the voltage level at which the input signal is recognized as a logic high or low. 

Mathematically, the noise margin (\(NM\)) is defined as the difference between the input switching threshold and the noise level. For CMOS circuits, the noise margin is relatively large because the switching threshold is centered around \(V_{DD}/2\). This can be expressed as:

\[
NM = \frac{V_{DD}}{2} - V_{noise}
\]

where \(V_{noise}\) is the noise voltage. Since the switching threshold is at \(V_{DD}/2\), small fluctuations in the input signal or power supply (i.e., noise) are unlikely to cause the circuit to misinterpret the logic state unless the noise exceeds \(V_{DD}/2\).

Additionally, CMOS circuits have complementary pairs of MOSFETs (Metal-Oxide-Semiconductor Field-Effect Transistors) that ensure only one transistor is on at any given time, minimizing power consumption and further enhancing noise immunity. The inherent design of CMOS circuits, including their high input impedance and low output impedance, also contributes to their robustness against noise.

% Diagram Prompt: Generate a diagram showing the input switching threshold at VDD/2 and how noise below this threshold does not affect the logic state.