\subsection{Unraveling the NOT! Discover the Inversion Symbol!}

\begin{tcolorbox}[colback=gray!10!white,colframe=black!75!black,title=E6C11] In Figure E6-3, which is the schematic symbol for the NOT operation (inversion)?
    \begin{enumerate}[label=\Alph*)]
        \item 2
        \item 4
        \item \textbf{5}
        \item 6
    \end{enumerate}
\end{tcolorbox}

\subsubsection{Intuitive Explanation}
Imagine you have a light switch. Normally, when you flip the switch up, the light turns on, and when you flip it down, the light turns off. The NOT operation is like flipping the switch in the opposite way. If the switch is on, the NOT operation turns it off, and if it's off, the NOT operation turns it on. In electronics, we use a special symbol to represent this flipping action. In Figure E6-3, the symbol that does this is number 5.

\subsubsection{Advanced Explanation}
The NOT operation, also known as logical inversion, is a fundamental operation in digital logic. It takes a single binary input and produces the opposite binary output. Mathematically, if the input is \( A \), the output \( Y \) is given by:
\[
Y = \overline{A}
\]
where \( \overline{A} \) denotes the logical NOT of \( A \). In digital circuits, the NOT operation is typically represented by a triangle followed by a small circle (also known as a bubble) at the output. This symbol indicates that the input signal is inverted. In Figure E6-3, the symbol labeled as 5 corresponds to this representation.

The NOT gate is one of the basic building blocks of digital electronics and is used in various applications, including logic gates, flip-flops, and microprocessors. Understanding the NOT operation is crucial for designing and analyzing digital circuits.

% Prompt for generating the diagram: Include Figure E6-3 showing the schematic symbols for various logic operations, with the NOT operation clearly labeled as symbol 5.