\subsection{Unlocking the Secrets of Keplerian Elements!}

\begin{tcolorbox}[colback=gray!10!white,colframe=black!75!black,title=E2A06] What are Keplerian elements? \\
    \begin{enumerate}[label=\Alph*)]
        \item \textbf{Parameters that define the orbit of a satellite}
        \item Phase reversing elements in a Yagi antenna
        \item High-emission heater filaments used in magnetron tubes
        \item Encrypting codes used for spread spectrum modulation
    \end{enumerate}
\end{tcolorbox}

\subsubsection{Intuitive Explanation}
Imagine you are trying to describe the path of a satellite as it moves around the Earth. Just like you might use a map to describe where a car is driving, scientists use something called Keplerian elements to describe the satellite's orbit. These elements are like a set of instructions that tell us exactly where the satellite is and how it is moving. They include things like how oval-shaped the orbit is, how tilted it is, and where the satellite is in its path around the Earth.

\subsubsection{Advanced Explanation}
Keplerian elements, also known as orbital elements, are a set of six parameters that uniquely define the orbit of a satellite or any celestial body around a primary body (like the Earth). These elements are derived from Kepler's laws of planetary motion and are essential for predicting the position and velocity of the satellite at any given time. The six Keplerian elements are:

\begin{enumerate}
    \item \textbf{Semi-major axis (a)}: Half the length of the longest diameter of the elliptical orbit.
    \item \textbf{Eccentricity (e)}: A measure of how much the orbit deviates from a perfect circle.
    \item \textbf{Inclination (i)}: The tilt of the orbit relative to the reference plane (usually the Earth's equator).
    \item \textbf{Longitude of the ascending node ($\Omega$)}: The angle where the orbit crosses the reference plane from south to north.
    \item \textbf{Argument of periapsis ($\omega$)}: The angle from the ascending node to the point of closest approach to the primary body.
    \item \textbf{True anomaly ($\nu$)}: The current position of the satellite in its orbit, measured from the periapsis.
\end{enumerate}

These elements are crucial for satellite tracking, orbital mechanics, and space mission planning. By knowing these parameters, we can calculate the satellite's position and velocity using Kepler's equations and Newton's laws of motion.

% Prompt for generating a diagram: 
% Diagram showing an elliptical orbit with labeled Keplerian elements: semi-major axis, eccentricity, inclination, longitude of the ascending node, argument of periapsis, and true anomaly.