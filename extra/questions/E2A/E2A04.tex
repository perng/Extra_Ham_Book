\subsection{Discovering the Mode of Amateur Radio Satellites!}

\begin{tcolorbox}[colback=gray!10!white,colframe=black!75!black,title=E2A04] What is meant by the “mode” of an amateur radio satellite?
    \begin{enumerate}[label=\Alph*),noitemsep]
        \item Whether the satellite is in a low earth or geostationary orbit
        \item \textbf{The satellite’s uplink and downlink frequency bands}
        \item The satellite’s orientation with respect to the Earth
        \item Whether the satellite is in a polar or equatorial orbit
    \end{enumerate}
\end{tcolorbox}

\subsubsection{Intuitive Explanation}
Imagine you have a walkie-talkie that you use to talk to your friend. Now, think of an amateur radio satellite as a super-powered walkie-talkie in space. The mode of the satellite is like the specific channels or frequencies it uses to send and receive messages. Just like you and your friend need to be on the same channel to talk, the satellite and the people on Earth need to use the same frequencies to communicate. So, the mode tells us which frequencies the satellite is using for sending (uplink) and receiving (downlink) signals.

\subsubsection{Advanced Explanation}
In the context of amateur radio satellites, the mode refers to the specific frequency bands allocated for the uplink and downlink communications. The uplink is the frequency band used to transmit signals from Earth to the satellite, while the downlink is the frequency band used to transmit signals from the satellite back to Earth. These frequency bands are crucial for ensuring that the communication between the satellite and the ground station is effective and interference-free.

For example, a common mode for amateur radio satellites is Mode A, which typically uses a 2-meter band (144-146 MHz) for the uplink and a 10-meter band (28-30 MHz) for the downlink. Another example is Mode B, which uses a 70-centimeter band (430-440 MHz) for the uplink and a 2-meter band for the downlink.

Understanding the mode of a satellite is essential for amateur radio operators because it determines the equipment and antennas they need to use to communicate with the satellite. It also helps in planning the communication session, as different modes may require different propagation conditions and antenna setups.

% Diagram prompt: Generate a diagram showing the uplink and downlink frequency bands for a typical amateur radio satellite.