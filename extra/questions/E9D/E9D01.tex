\subsection{E9D01: Doubling the Frequency: A Cheerful Boost for Your Antenna Gain!}

\begin{tcolorbox}[colback=gray!10!white,colframe=black!75!black]
\textbf{E9D01} How much does the gain of an ideal parabolic reflector antenna increase when the operating frequency is doubled?

\begin{enumerate}[label=\Alph*,noitemsep]
    \item 2 dB
    \item 3 dB
    \item 4 dB
    \item \textbf{6 dB}
\end{enumerate}
\end{tcolorbox}

\subsubsection{Intuitive Explanation}
Imagine your antenna is like a giant ear that listens to radio waves. When you double the frequency, it's like turning up the volume on your favorite song—your antenna becomes twice as good at picking up those waves! But here's the fun part: the gain doesn't just double; it actually increases by 6 dB. That's like going from a whisper to a shout! So, doubling the frequency gives your antenna a cheerful boost, making it much more powerful.

\subsubsection{Advanced Explanation}
The gain of an ideal parabolic reflector antenna is directly related to the operating frequency. The gain \( G \) of such an antenna can be expressed as:

\[
G = \left( \frac{4 \pi A}{\lambda^2} \right) \eta
\]

where:
\begin{itemize}
    \item \( A \) is the area of the parabolic reflector,
    \item \( \lambda \) is the wavelength of the operating frequency,
    \item \( \eta \) is the efficiency of the antenna.
\end{itemize}

When the operating frequency is doubled, the wavelength \( \lambda \) is halved. Substituting \( \lambda' = \frac{\lambda}{2} \) into the gain equation:

\[
G' = \left( \frac{4 \pi A}{\left( \frac{\lambda}{2} \right)^2} \right) \eta = \left( \frac{4 \pi A}{\frac{\lambda^2}{4}} \right) \eta = 4 \left( \frac{4 \pi A}{\lambda^2} \right) \eta = 4G
\]

The gain increases by a factor of 4, which corresponds to a 6 dB increase in logarithmic scale:

\[
10 \log_{10}(4) \approx 6 \text{ dB}
\]

Thus, doubling the frequency results in a 6 dB increase in the gain of an ideal parabolic reflector antenna.

% Diagram prompt: A diagram showing the relationship between frequency, wavelength, and antenna gain could be helpful here.