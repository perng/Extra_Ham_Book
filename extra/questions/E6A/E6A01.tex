\subsection{Gallium Arsenide: Unleashing Exciting Applications!}

\begin{tcolorbox}[colback=gray!10!white,colframe=black!75!black,title=E6A01] In what application is gallium arsenide used as a semiconductor material?
    \begin{enumerate}[label=\Alph*,noitemsep]
        \item In high-current rectifier circuits
        \item In high-power audio circuits
        \item \textbf{In microwave circuits}
        \item In very low-frequency RF circuits
    \end{enumerate}
\end{tcolorbox}

\subsubsection{Intuitive Explanation}
Gallium arsenide (GaAs) is a special material that is used in devices that need to work really fast, like in microwave ovens or satellite communications. Think of it like this: if regular semiconductors are like bicycles, gallium arsenide is like a sports car—it can go much faster! This makes it perfect for applications where speed is crucial, such as in microwave circuits, which handle very high-frequency signals.

\subsubsection{Advanced Explanation}
Gallium arsenide (GaAs) is a compound semiconductor material that exhibits superior electron mobility compared to silicon (Si). This property allows GaAs-based devices to operate at higher frequencies, making them ideal for microwave and radio frequency (RF) applications. The high electron mobility in GaAs results in faster electron transit times, which is critical for high-frequency operation.

Mathematically, the electron mobility (\(\mu\)) is given by:
\[
\mu = \frac{v_d}{E}
\]
where \(v_d\) is the drift velocity of electrons and \(E\) is the electric field. GaAs has a higher \(\mu\) compared to Si, enabling it to support higher frequencies.

Additionally, GaAs has a direct bandgap, which makes it efficient for optoelectronic applications, although this is not directly relevant to the question. The primary advantage in microwave circuits is the ability to handle high-frequency signals with minimal loss, making GaAs the material of choice for such applications.

% Diagram Prompt: Generate a diagram comparing the electron mobility of GaAs and Si, showing how GaAs supports higher frequencies.