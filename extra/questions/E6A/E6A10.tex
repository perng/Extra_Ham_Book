\subsection{Spot the N-Channel Dual-Gate MOSFET!}

\begin{tcolorbox}[colback=gray!10!white,colframe=black!75!black,title=E6A10]
\textbf{E6A10} In Figure E6-1, which is the schematic symbol for an N-channel dual-gate MOSFET?
\begin{enumerate}[label=\Alph*]
    \item 2
    \item \textbf{4}
    \item 5
    \item 6
\end{enumerate}
\end{tcolorbox}

\subsubsection*{Intuitive Explanation}
Imagine you have a special kind of switch called a MOSFET. This switch has two gates instead of one, and it’s called a dual-gate MOSFET. The gates are like doors that control the flow of electricity. In this question, you’re looking at a picture (Figure E6-1) with different symbols, and you need to find the one that represents this special N-channel dual-gate MOSFET. The correct symbol is the one labeled 4. Think of it as finding the right key that fits the lock!

\subsubsection*{Advanced Explanation}
A MOSFET (Metal-Oxide-Semiconductor Field-Effect Transistor) is a type of transistor used for amplifying or switching electronic signals. An N-channel MOSFET uses electrons as the primary charge carriers. A dual-gate MOSFET has two gates, which allows for more control over the current flow and is often used in RF (radio frequency) applications for better performance.

In schematic diagrams, the symbol for an N-channel dual-gate MOSFET typically includes two gate terminals, a source, and a drain. The correct symbol in Figure E6-1 is labeled 4, which correctly represents the dual-gate structure. The other symbols (2, 5, and 6) represent different components or single-gate MOSFETs, which do not match the dual-gate configuration.

% Prompt for generating the diagram:
% Include Figure E6-1 here, showing the schematic symbols for various components, with the N-channel dual-gate MOSFET labeled as 4.