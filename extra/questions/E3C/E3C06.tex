\subsection{Exploring the Radio Horizon vs. the Geographic Horizon!}

\begin{tcolorbox}[colback=gray!10!white,colframe=black!75!black,title=Multiple Choice Question]
    \textbf{E3C06} How does the VHF/UHF radio horizon compare to the geographic horizon?
    \begin{enumerate}[label=\Alph*)]
        \item \textbf{It is approximately 15 percent farther}
        \item It is approximately 20 percent nearer
        \item It is approximately 50 percent farther
        \item They are approximately the same
    \end{enumerate}
\end{tcolorbox}

\subsubsection{Intuitive Explanation}
Imagine you are standing on a beach looking out at the ocean. The farthest point you can see where the sky meets the water is called the geographic horizon. Now, think about using a walkie-talkie to talk to someone far away. The radio waves from your walkie-talkie can travel a bit farther than what you can see with your eyes. This is because the radio waves can bend slightly around the Earth's surface. So, the radio horizon is a little bit farther than the geographic horizon—about 15 percent farther!

\subsubsection{Advanced Explanation}
The radio horizon for VHF/UHF frequencies is influenced by the Earth's curvature and the refractive properties of the atmosphere. Radio waves tend to bend slightly around the Earth due to atmospheric refraction, which effectively extends the horizon. The formula to calculate the radio horizon distance \( d \) is given by:

\[
d = \sqrt{2hR}
\]

where:
\begin{itemize}
    \item \( h \) is the height of the antenna above the Earth's surface,
    \item \( R \) is the Earth's radius (approximately 6,371 km).
\end{itemize}

Due to atmospheric refraction, the effective Earth radius \( R' \) is often considered to be \( \frac{4}{3} \) times the actual radius \( R \). This increases the radio horizon distance by approximately 15 percent compared to the geographic horizon. Thus, the radio horizon is about 15 percent farther than the geographic horizon.

% Diagram Prompt: Generate a diagram showing the Earth's curvature, the geographic horizon, and the radio horizon with atmospheric refraction.