\subsection{Exploring the Wonders of Space Weather: What’s an Extreme Geomagnetic Storm?}

\begin{tcolorbox}[colback=gray!10!white,colframe=black!75!black,title=Multiple Choice Question]
\textbf{E3C08} Which of the following is the space-weather term for an extreme geomagnetic storm?
\begin{enumerate}[label=\Alph*]
    \item B9
    \item X5
    \item M9
    \item \textbf{G5}
\end{enumerate}
\end{tcolorbox}

\subsubsection*{Intuitive Explanation}
Imagine the Sun is like a giant ball of energy that sometimes sends out bursts of energy and particles into space. When these bursts reach Earth, they can mess with our planet's magnetic field, causing what we call a geomagnetic storm. Just like we have different levels of storms on Earth, like light rain or heavy thunderstorms, geomagnetic storms also have different levels. The term G5 is used to describe the most extreme geomagnetic storm, kind of like a superstorm in space weather.

\subsubsection*{Advanced Explanation}
Geomagnetic storms are classified using the G-scale, which ranges from G1 (minor) to G5 (extreme). The G-scale is based on the Kp index, a measure of geomagnetic activity derived from magnetometer data. A G5 storm corresponds to a Kp index of 9, indicating severe disturbances in Earth's magnetosphere. These disturbances can lead to widespread power grid fluctuations, satellite disruptions, and enhanced auroral activity. The other options, B9, X5, and M9, are classifications related to solar flares, not geomagnetic storms. Solar flares are categorized by their X-ray brightness, with X-class being the most intense, followed by M-class and B-class. However, these classifications do not directly describe geomagnetic storms.

% Prompt for generating a diagram:
% Diagram showing the classification scales for geomagnetic storms (G-scale) and solar flares (X, M, B classes) with examples of each level.