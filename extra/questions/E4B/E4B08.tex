\subsection{Explore the Q Factor: Unveiling Series-Tuned Circuits!}

\begin{tcolorbox}[colback=gray!10, colframe=black, title=E4B08] Which of the following can be used to determine the Q of a series-tuned circuit?
\begin{enumerate}[label=\Alph*.]
    \item The ratio of inductive reactance to capacitive reactance
    \item The frequency shift
    \item \textbf{The bandwidth of the circuit's frequency response}
    \item The resonant frequency of the circuit
\end{enumerate} \end{tcolorbox}

\subsubsection{Related Concepts}

To understand how to determine the Q factor of a series-tuned circuit, it is essential to grasp several key concepts:

1. \textbf{Q Factor}: The Q factor (Quality factor) of a resonant circuit quantifies its bandwidth relative to its center frequency. A higher Q factor indicates a narrower bandwidth and implies that a circuit can resonate at a specific frequency with less energy loss.

2. \textbf{Bandwidth}: This concept refers to the range of frequencies over which the circuit can effectively resonate. For a series-tuned circuit, the bandwidth is influenced by resistance and reactance.

3. \textbf{Frequency Response}: The frequency response of a circuit shows how the output amplitude varies with frequency. The Q factor can be directly related to the bandwidth and resonant frequency observed in this response.

4. \textbf{Resonant Frequency}: This is the frequency at which the inductive reactance equals the capacitive reactance, resulting in a purely resistive impedance and maximum circuit current.

\subsubsection{Calculating the Q Factor}

The Q factor can be determined with the following formula:

\[
Q = \frac{f_0}{\Delta f}
\]

where \( f_0 \) is the resonant frequency and \( \Delta f \) is the bandwidth (the difference between the upper and lower cutoff frequencies).
 
To summarize:
- The Q factor is a measure of how selective a circuit is, which can be understood through its bandwidth.
- The correct answer to the question is \textbf{C}, as the bandwidth of the circuit's frequency response is what specifically measures the circuit's quality factor, influencing its performance.

\subsubsection{Diagram of a Series-Tuned Circuit}

A diagram illustrating a series-tuned circuit can enhance understanding. Below is a simple representation created using TikZ:

\begin{center}
\begin{tikzpicture}[scale=0.8]
    \draw (0,0) to[R, l=R, v<={$\text{Voltage } V_s$}] (0,-2)
          to[L, l=L] (2,-2) -- (2,0) to[C, l=C] (0,0);
    \draw[->] (2,-2) -- (3,-2) node[right] {To Output};
    \node at (1,-1.5) {Series-Tuned Circuit};
    \draw[->] (-1,0) -- (1,0);
    \node at (-1,0.5) {Input};
\end{tikzpicture}
\end{center}
