\subsection{Spotting the Magic of Transequatorial Propagation!}

\begin{tcolorbox}[colback=gray!10!white,colframe=black!75!black,title=E3B01] Where is transequatorial propagation (TEP) most likely to occur?
    \begin{enumerate}[label=\Alph*),noitemsep]
        \item \textbf{Between points separated by 2,000 miles to 3,000 miles over a path perpendicular to the geomagnetic equator}
        \item Between points located 1,500 miles to 2,000 miles apart on the geomagnetic equator
        \item Between points located at each other’s antipode
        \item Through the region where the terminator crosses the geographic equator
    \end{enumerate}
\end{tcolorbox}

\subsubsection{Intuitive Explanation}
Imagine the Earth has an invisible belt called the geomagnetic equator. Transequatorial propagation (TEP) is like a magical radio bridge that connects two points on opposite sides of this belt, but only if they are about 2,000 to 3,000 miles apart and directly across from each other. It’s like throwing a ball straight across a river—the ball (or radio signal) travels best when you aim directly across, not sideways or at an angle.

\subsubsection{Advanced Explanation}
Transequatorial propagation (TEP) occurs due to the interaction of radio waves with the ionosphere, particularly the F-layer, which is influenced by the Earth’s geomagnetic field. The geomagnetic equator is a region where the Earth’s magnetic field lines are horizontal. For TEP to occur, the radio signals must travel perpendicular to these field lines, typically between points separated by 2,000 to 3,000 miles. This perpendicular path allows the signals to be refracted efficiently by the ionosphere, enabling long-distance communication.

The ionosphere’s F-layer is most effective at refracting radio signals during periods of high solar activity, which increases ionization. The specific distance range (2,000 to 3,000 miles) ensures that the signals are refracted back to Earth rather than being lost into space. This phenomenon is less likely to occur along the geomagnetic equator itself or at antipodal points, as the geometry of the ionosphere and the Earth’s magnetic field does not support efficient refraction in those configurations.

% [Prompt for diagram: A diagram showing the Earth with the geomagnetic equator, two points separated by 2,000 to 3,000 miles, and the path of radio signals perpendicular to the geomagnetic equator.]