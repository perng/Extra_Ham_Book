\subsection{Sunny Signals: Best Times for Transequatorial Propagation!}

\begin{tcolorbox}[colback=gray!10!white,colframe=black!75!black,title=Multiple Choice Question]
\textbf{E3B03} At what time of day is transequatorial propagation most likely to occur?
\begin{enumerate}[label=\Alph*)]
    \item Morning
    \item Noon
    \item \textbf{Afternoon or early evening}
    \item Late at night
\end{enumerate}
\end{tcolorbox}

\subsubsection*{Intuitive Explanation}
Imagine the Earth is like a giant playground, and the Sun is a bright light shining on it. Transequatorial propagation is like a game where radio signals bounce between the Earth and the ionosphere (a layer of the atmosphere) to travel long distances. The best time for this game is when the Sun is shining directly on the equator, which happens in the afternoon or early evening. This is because the Sun's energy makes the ionosphere more active, helping the radio signals bounce better.

\subsubsection*{Advanced Explanation}
Transequatorial propagation (TEP) is a phenomenon where radio waves are refracted by the ionosphere to travel between points on opposite sides of the Earth's equator. The ionosphere's ionization is influenced by solar radiation, which is most intense when the Sun is directly overhead. This occurs around local noon at the equator, but the ionosphere takes time to respond to the solar radiation, reaching peak ionization in the afternoon or early evening. Therefore, TEP is most likely to occur during these times.

Mathematically, the ionization level \( N_e \) of the ionosphere can be approximated by:
\[ N_e \propto \frac{I_{\text{solar}}}{\cos(\theta)} \]
where \( I_{\text{solar}} \) is the solar radiation intensity and \( \theta \) is the solar zenith angle. The maximum ionization occurs when \( \theta \) is minimized, which is around local noon, but the ionosphere's response time delays the peak propagation conditions to the afternoon or early evening.

Related concepts include the ionospheric layers (D, E, F1, and F2), solar radiation, and the critical frequency of the ionosphere, which determines the maximum usable frequency (MUF) for radio propagation.

% Diagram prompt: Generate a diagram showing the Earth, the Sun, and the ionosphere with radio waves bouncing between the equator during the afternoon or early evening.