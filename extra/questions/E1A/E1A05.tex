\subsection{Steering the Waves: Who's in Charge of Amateur Radio on Your Boat?}

\begin{tcolorbox}[colback=gray!10!white,colframe=black!75!black,title=E1A05] Who must be in physical control of the station apparatus of an amateur station aboard any vessel or craft that is documented or registered in the United States?
    \begin{enumerate}[label=\Alph*)]
        \item Only a person with an FCC Marine Radio license grant
        \item Only a person named in an amateur station license grant
        \item \textbf{Any person holding an FCC issued amateur license or who is authorized for alien reciprocal operation}
        \item Any person named in an amateur station license grant or a person holding an unrestricted Radiotelephone Operator Permit
    \end{enumerate}
\end{tcolorbox}

\subsubsection*{Intuitive Explanation}
Imagine you're on a boat, and you have a radio to communicate with others. The rules say that someone needs to be in charge of the radio to make sure it's used correctly. This person doesn't have to be the boat's owner or someone special—just anyone who has a license to use amateur radios. This license is like a permission slip that says you know how to use the radio properly. So, as long as someone on the boat has this permission slip, they can be in charge of the radio.

\subsubsection*{Advanced Explanation}
According to FCC regulations, the physical control of an amateur station aboard any vessel or craft documented or registered in the United States must be maintained by a person holding an FCC-issued amateur license or someone authorized for alien reciprocal operation. This ensures that the operator is knowledgeable about the rules and procedures governing amateur radio operations, thereby minimizing the risk of interference and ensuring effective communication.

The key points here are:
\begin{itemize}
    \item \textbf{FCC Issued Amateur License}: This license is granted by the Federal Communications Commission (FCC) to individuals who have passed the necessary examinations, demonstrating their understanding of radio operation, regulations, and technical knowledge.
    \item \textbf{Alien Reciprocal Operation}: This allows foreign amateur radio operators to operate in the United States under certain conditions, provided their home country has a reciprocal agreement with the U.S.
\end{itemize}

The correct answer, \textbf{C}, reflects these regulatory requirements, ensuring that the operator is either licensed by the FCC or authorized under reciprocal agreements.

% Diagram Prompt: A flowchart showing the decision process for determining who can control the amateur radio station on a U.S. registered vessel.