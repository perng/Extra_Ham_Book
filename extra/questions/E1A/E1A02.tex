\subsection{Unlocking the LSB Frequency Fun!}

\begin{tcolorbox}[colback=gray!10!white,colframe=black!75!black,title=E1A02] When using a transceiver that displays the carrier frequency of phone signals, which of the following displayed frequencies represents the lowest frequency at which a properly adjusted LSB emission will be totally within the band?
    \begin{enumerate}[label=\Alph*.]
        \item The exact lower band edge
        \item 300 Hz above the lower band edge
        \item 1 kHz above the lower band edge
        \item \textbf{3 kHz above the lower band edge}
    \end{enumerate}
\end{tcolorbox}

\subsubsection{Intuitive Explanation}
Imagine you are tuning a radio to listen to a conversation. The radio shows you the frequency where the conversation is happening. Now, if you want to make sure that the entire conversation is within the allowed frequency range, you need to set the radio slightly above the lowest allowed frequency. This is because the conversation (or signal) takes up some space in the frequency range. If you set it too low, part of the conversation might fall outside the allowed range. Setting it 3 kHz above the lowest frequency ensures that the entire conversation stays within the allowed range.

\subsubsection{Advanced Explanation}
In radio communications, LSB (Lower Sideband) is a type of modulation where the signal is transmitted using the lower sideband of the carrier frequency. The bandwidth of a typical voice signal in LSB mode is approximately 3 kHz. To ensure that the entire signal is within the allocated band, the carrier frequency must be set such that the lower sideband does not extend below the lower band edge.

Let’s denote the lower band edge as \( f_{\text{edge}} \). The carrier frequency \( f_{\text{carrier}} \) must be set such that:
\[ f_{\text{carrier}} - 3 \text{ kHz} \geq f_{\text{edge}} \]
Rearranging the inequality, we get:
\[ f_{\text{carrier}} \geq f_{\text{edge}} + 3 \text{ kHz} \]

Therefore, the carrier frequency must be at least 3 kHz above the lower band edge to ensure that the entire LSB emission is within the band. This is why the correct answer is \textbf{D: 3 kHz above the lower band edge}.

\subsubsection{Related Concepts}
\begin{itemize}
    \item \textbf{Carrier Frequency}: The central frequency of a radio signal that is modulated to carry information.
    \item \textbf{Lower Sideband (LSB)}: A type of amplitude modulation where only the lower sideband is transmitted, reducing bandwidth usage.
    \item \textbf{Bandwidth}: The range of frequencies occupied by a signal.
    \item \textbf{Band Edge}: The boundary frequencies that define the limits of a frequency band.
\end{itemize}

% Prompt for generating a diagram: A diagram showing the relationship between the carrier frequency, lower sideband, and the lower band edge would be helpful here. The diagram should illustrate how setting the carrier frequency 3 kHz above the lower band edge ensures the entire LSB emission is within the band.