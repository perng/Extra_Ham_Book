\subsection{Finding the Perfect Frequency for 60 Meter Fun!}

\begin{tcolorbox}[colback=gray!10!white,colframe=black!75!black,title=E1A06] What is the required transmit frequency of a CW signal for channelized 60 meter operation?
    \begin{enumerate}[label=\Alph*),noitemsep]
        \item At the lowest frequency of the channel
        \item \textbf{At the center frequency of the channel}
        \item At the highest frequency of the channel
        \item On any frequency where the signal’s sidebands are within the channel
    \end{enumerate}
\end{tcolorbox}

\subsubsection{Intuitive Explanation}
Imagine you have a radio channel that is like a narrow road. To drive safely, you need to stay in the middle of the road. Similarly, when you transmit a CW (Continuous Wave) signal on the 60-meter band, you need to transmit at the center frequency of the channel. This ensures that your signal stays within the allowed range and doesn’t interfere with other signals.

\subsubsection{Advanced Explanation}
In channelized operation, the 60-meter band is divided into specific frequency channels. Each channel has a defined bandwidth, and the center frequency is the midpoint of this bandwidth. For CW signals, which are narrowband, transmitting at the center frequency ensures that the entire signal, including its sidebands, remains within the channel's allocated bandwidth. This is crucial for compliance with regulatory requirements and to avoid interference with adjacent channels.

Mathematically, if a channel spans from \( f_{\text{low}} \) to \( f_{\text{high}} \), the center frequency \( f_{\text{center}} \) is calculated as:
\[
f_{\text{center}} = \frac{f_{\text{low}} + f_{\text{high}}}{2}
\]
For example, if a channel ranges from 5.330 MHz to 5.350 MHz, the center frequency would be:
\[
f_{\text{center}} = \frac{5.330 + 5.350}{2} = 5.340 \text{ MHz}
\]
Transmitting at this frequency ensures that the CW signal remains within the channel's limits.

% Diagram Prompt: Generate a diagram showing a frequency spectrum with a channel marked from f_low to f_high, and the center frequency f_center highlighted.