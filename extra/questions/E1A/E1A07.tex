\subsection{Maxing Out on the 2200-Meter Band!}

\begin{tcolorbox}[colback=gray!10!white,colframe=black!75!black,title=E1A07]
\textbf{E1A07} What is the maximum power permitted on the 2200-meter band?
\begin{enumerate}[label=\Alph*,noitemsep]
    \item 50 watts PEP (peak envelope power)
    \item 100 watts PEP (peak envelope power)
    \item \textbf{1 watt EIRP (equivalent isotropic radiated power)}
    \item 5 watts EIRP (equivalent isotropic radiated power)
\end{enumerate}
\end{tcolorbox}

\subsubsection{Intuitive Explanation}
Imagine you have a flashlight, and you want to shine it as far as possible. But there's a rule: you can only use a very small amount of energy to do this. On the 2200-meter band, which is a very long wavelength used in radio communication, the rule is similar. You can only use a tiny amount of power to send your signal, just like using a small flashlight. The maximum power allowed is 1 watt, which is like using a very dim light bulb. This ensures that everyone can use the band without causing too much interference.

\subsubsection{Advanced Explanation}
The 2200-meter band (135.7–137.8 kHz) is part of the Low Frequency (LF) spectrum. The Federal Communications Commission (FCC) and other regulatory bodies impose strict power limits to minimize interference and ensure efficient use of the spectrum. The maximum permitted power on this band is 1 watt EIRP (Equivalent Isotropic Radiated Power). EIRP is a measure of the power that would be radiated by an ideal isotropic antenna (a theoretical antenna that radiates equally in all directions) to produce the same power density as the actual antenna in the direction of its maximum radiation.

The calculation of EIRP involves the following formula:
\[
\text{EIRP} = P_{\text{transmitter}} \times G_{\text{antenna}}
\]
where:
\begin{itemize}
    \item \( P_{\text{transmitter}} \) is the power output of the transmitter.
    \item \( G_{\text{antenna}} \) is the gain of the antenna relative to an isotropic radiator.
\end{itemize}

For the 2200-meter band, the regulatory limit is set at 1 watt EIRP, meaning that the product of the transmitter power and the antenna gain must not exceed 1 watt. This ensures that the radiated power remains low, reducing the risk of interference with other users of the band.

Related concepts include:
\begin{itemize}
    \item \textbf{Peak Envelope Power (PEP)}: The maximum power level of a signal during one complete cycle of modulation.
    \item \textbf{Equivalent Isotropic Radiated Power (EIRP)}: The total power that would be radiated by an isotropic antenna to achieve the same power density as the actual antenna in the direction of its maximum radiation.
    \item \textbf{Low Frequency (LF) Spectrum}: The range of radio frequencies from 30 kHz to 300 kHz, known for their long wavelengths and ability to propagate over long distances.
\end{itemize}

% Prompt for generating a diagram: A diagram showing the relationship between transmitter power, antenna gain, and EIRP would be helpful here.