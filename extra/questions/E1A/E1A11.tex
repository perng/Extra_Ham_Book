\subsection{Amateur Radio Adventures: Licensing for Your Sea Voyage!}

\begin{tcolorbox}[colback=gray!10!white,colframe=black!75!black,title=E1A11] What licensing is required when operating an amateur station aboard a US-registered vessel in international waters?
    \begin{enumerate}[label=\Alph*,noitemsep]
        \item Any amateur license with an FCC Marine or Aircraft endorsement
        \item \textbf{Any FCC-issued amateur license}
        \item Only General class or higher amateur licenses
        \item An unrestricted Radiotelephone Operator Permit
    \end{enumerate}
\end{tcolorbox}

\subsubsection{Intuitive Explanation}
Imagine you're on a big boat in the middle of the ocean, and you want to use your amateur radio to talk to people far away. The question is asking what kind of permission (license) you need to do this. The answer is simple: as long as you have any amateur radio license from the FCC (the Federal Communications Commission), you're good to go! You don't need any special extra permissions or higher-level licenses. Just your regular amateur radio license is enough to operate your radio on a US-registered boat in international waters.

\subsubsection{Advanced Explanation}
When operating an amateur radio station aboard a US-registered vessel in international waters, the licensing requirements are governed by the Federal Communications Commission (FCC). According to FCC regulations, any FCC-issued amateur radio license is sufficient for such operations. This means that whether you hold a Technician, General, or Amateur Extra class license, you are permitted to operate your amateur station on a US-registered vessel in international waters.

The key point here is that the license must be issued by the FCC, and there is no requirement for additional endorsements or higher-class licenses. This regulation ensures that amateur radio operators can communicate effectively while adhering to international maritime laws and FCC guidelines.

\noindent\textbf{Calculation:} No specific calculations are required for this question.

\noindent\textbf{Related Concepts:}
\begin{itemize}
    \item \textbf{FCC Regulations:} The FCC sets the rules for amateur radio operations in the United States, including on vessels registered in the US.
    \item \textbf{International Waters:} These are areas of the ocean that are not under the jurisdiction of any single country, but certain regulations, such as those from the vessel's flag state (in this case, the US), still apply.
    \item \textbf{Amateur Radio Licenses:} The FCC issues different classes of amateur radio licenses, each with varying privileges and requirements.
\end{itemize}

% Prompt for generating a diagram: A diagram showing a US-registered vessel in international waters with an amateur radio operator using their equipment, highlighting the FCC license requirement.