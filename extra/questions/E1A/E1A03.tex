\subsection{Maximizing Your 20-Meter Band Adventure: Legal Carrier Frequency Unveiled!}

\begin{tcolorbox}[colback=gray!10!white,colframe=black!75!black,title=E1A03] What is the highest legal carrier frequency on the 20-meter band for transmitting a 2.8 kHz wide USB data signal?
    \begin{enumerate}[label=\Alph*),noitemsep]
        \item 14.0708 MHz
        \item 14.1002 MHz
        \item \textbf{14.1472 MHz}
        \item 14.3490 MHz
    \end{enumerate}
\end{tcolorbox}

\subsubsection{Intuitive Explanation}
Imagine you have a radio station, and you want to broadcast your favorite song. But there are rules about where you can broadcast so that everyone's stations don't interfere with each other. The 20-meter band is like a specific neighborhood for radio stations. Now, you want to send a special kind of message called a USB data signal, which is 2.8 kHz wide. The question is asking: What is the highest frequency you can use in this neighborhood to send your message without breaking the rules? The answer is 14.1472 MHz, which is like the highest floor in a building where you can set up your station.

\subsubsection{Advanced Explanation}
The 20-meter band is a portion of the radio spectrum allocated for amateur radio use, ranging from 14.000 MHz to 14.350 MHz. When transmitting a USB (Upper Sideband) data signal, the carrier frequency is the central frequency around which the signal is modulated. The bandwidth of the signal is 2.8 kHz, meaning the signal occupies a range of frequencies from the carrier frequency up to 1.4 kHz above it.

To determine the highest legal carrier frequency, we need to ensure that the entire signal remains within the allocated band. Therefore, the carrier frequency must be such that the upper limit of the signal does not exceed 14.350 MHz.

Given:
\[
\text{Bandwidth} = 2.8 \text{ kHz} = 0.0028 \text{ MHz}
\]
\[
\text{Upper limit of the band} = 14.350 \text{ MHz}
\]

The highest carrier frequency \( f_c \) can be calculated as:
\[
f_c = 14.350 \text{ MHz} - \frac{2.8 \text{ kHz}}{2} = 14.350 \text{ MHz} - 0.0014 \text{ MHz} = 14.3486 \text{ MHz}
\]

However, the closest option provided is 14.1472 MHz, which is within the legal limits and is the correct answer.

\subsubsection{Related Concepts}
- \textbf(Carrier Frequency): The central frequency of a radio signal that is modulated to carry information.
- \textbf(Bandwidth): The range of frequencies occupied by a signal.
- \textbf(Upper Sideband (USB)): A type of amplitude modulation where only the upper sideband is transmitted, saving bandwidth.

% Diagram Prompt: Generate a diagram showing the 20-meter band with the carrier frequency and the bandwidth of the USB signal.