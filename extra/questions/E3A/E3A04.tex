\subsection{Chasing Waves: The Direction of Electromagnetic Travels!}

\begin{tcolorbox}[colback=gray!10!white,colframe=black!75!black,title=Multiple Choice Question]
\textbf{E3A04} In what direction does an electromagnetic wave travel?
\begin{enumerate}[label=\Alph*,noitemsep]
    \item It depends on the phase angle of the magnetic field
    \item It travels parallel to the electric and magnetic fields
    \item It depends on the phase angle of the electric field
    \item \textbf{It travels at a right angle to the electric and magnetic fields}
\end{enumerate}
\end{tcolorbox}

\subsubsection{Intuitive Explanation}
Imagine you are standing on a beach, watching the waves roll in. The water moves up and down, but the wave itself moves forward. Similarly, an electromagnetic wave is like a wave on the beach. The electric and magnetic fields wiggle up and down, but the wave itself travels in a straight line, perpendicular to the direction of the wiggles. So, the wave moves at a right angle to both the electric and magnetic fields.

\subsubsection{Advanced Explanation}
An electromagnetic wave is a transverse wave, meaning that the oscillations of the electric field \(\mathbf{E}\) and the magnetic field \(\mathbf{B}\) are perpendicular to the direction of wave propagation \(\mathbf{k}\). Mathematically, this can be described using Maxwell's equations, which govern the behavior of electromagnetic fields. The Poynting vector \(\mathbf{S}\), which represents the direction of energy flow in an electromagnetic wave, is given by:

\[
\mathbf{S} = \frac{1}{\mu_0} (\mathbf{E} \times \mathbf{B})
\]

Here, \(\mu_0\) is the permeability of free space. The cross product \(\mathbf{E} \times \mathbf{B}\) indicates that the direction of wave propagation is perpendicular to both the electric and magnetic fields. Therefore, the electromagnetic wave travels at a right angle to the electric and magnetic fields.

% Diagram Prompt: Generate a diagram showing an electromagnetic wave with the electric field (E), magnetic field (B), and direction of propagation (k) labeled, illustrating their perpendicular relationship.