\subsection{Brightening Up: Best Emission Modes for Auroral Adventure!}

\begin{tcolorbox}[colback=gray!10!white,colframe=black!75!black,title=E3A13] Which of these emission modes is best for auroral propagation?
    \begin{enumerate}[label=\Alph*),noitemsep]
        \item \textbf{CW}
        \item SSB
        \item FM
        \item RTTY
    \end{enumerate}
\end{tcolorbox}

\subsubsection{Intuitive Explanation}
Imagine you're trying to send a message through a flickering, colorful curtain of light in the sky—this is what auroral propagation is like. The best way to send a message through this curtain is to use a simple and steady signal, like a flashlight that stays on continuously. This is what CW (Continuous Wave) does. It’s like a steady beam of light that can easily pass through the aurora, making it the best choice for this kind of propagation.

\subsubsection{Advanced Explanation}
Auroral propagation involves the reflection of radio waves off the ionized layers of the atmosphere, particularly during auroral activity. The ionosphere becomes highly irregular and dynamic, causing rapid fluctuations in signal strength and phase. 

CW (Continuous Wave) is the most effective emission mode for auroral propagation due to its simplicity and narrow bandwidth. The narrow bandwidth of CW allows it to be less affected by the rapid phase and amplitude changes caused by the aurora. Additionally, CW signals are easier to detect and decode under these conditions compared to more complex modulation schemes like SSB, FM, or RTTY.

Mathematically, the signal-to-noise ratio (SNR) for CW can be expressed as:

\[
\text{SNR}_{\text{CW}} = \frac{P_{\text{signal}}}{P_{\text{noise}}}
\]

where \( P_{\text{signal}} \) is the power of the CW signal and \( P_{\text{noise}} \) is the power of the noise. The narrow bandwidth of CW minimizes \( P_{\text{noise}} \), thereby maximizing the SNR.

In contrast, SSB, FM, and RTTY have wider bandwidths and are more susceptible to distortion and noise in the auroral environment. Therefore, CW is the optimal choice for reliable communication during auroral propagation.

% Prompt for diagram: A diagram showing the reflection of CW signals off the auroral ionosphere compared to other emission modes would be helpful here.