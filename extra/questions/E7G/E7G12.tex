\subsection{Unleashing the Power of Operational Amplifiers!}

\begin{tcolorbox}[colback=gray!10!white,colframe=black!75!black,title=E7G12] What is an operational amplifier? \\
    \begin{enumerate}[label=\Alph*),noitemsep]
        \item \textbf{A high-gain, direct-coupled differential amplifier with very high input impedance and very low output impedance}
        \item A digital audio amplifier whose characteristics are determined by components external to the amplifier
        \item An amplifier used to increase the average output of frequency modulated amateur signals to the legal limit
        \item A RF amplifier used in the UHF and microwave regions
    \end{enumerate}
\end{tcolorbox}

\subsubsection{Intuitive Explanation}
An operational amplifier, often called an op-amp, is like a super-powered magnifying glass for electrical signals. Imagine you have a tiny sound or signal that you can barely hear or detect. An op-amp can take that tiny signal and make it much louder or stronger, so it’s easier to work with. It’s also very good at not interfering with the original signal, meaning it doesn’t change what the signal is trying to say. Think of it as a helpful assistant that makes your job easier without getting in the way.

\subsubsection{Advanced Explanation}
An operational amplifier (op-amp) is a high-gain electronic voltage amplifier with a differential input and, usually, a single-ended output. The gain of an op-amp is typically very high, often in the range of \(10^5\) to \(10^6\). The input impedance is also very high, often in the order of \(10^6\) to \(10^{12}\) ohms, which means it draws very little current from the input source. The output impedance is very low, typically less than 100 ohms, allowing it to drive a load without significant loss of signal strength.

The op-amp is characterized by the following equation:
\[
V_{\text{out}} = A_{\text{OL}} \times (V_+ - V_-)
\]
where \(V_{\text{out}}\) is the output voltage, \(A_{\text{OL}}\) is the open-loop gain, and \(V_+\) and \(V_-\) are the voltages at the non-inverting and inverting inputs, respectively.

Operational amplifiers are widely used in various applications, including signal conditioning, filtering, and mathematical operations such as addition, subtraction, integration, and differentiation. They are fundamental components in analog electronics and are often used in feedback configurations to achieve desired circuit behaviors.

% Diagram Prompt: Generate a diagram showing the basic structure of an operational amplifier, including the inverting and non-inverting inputs, the output, and the power supply connections.