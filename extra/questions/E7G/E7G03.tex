\subsection{Understanding Op-Amp Input Impedance!}

\begin{tcolorbox}[colback=gray!10!white,colframe=black!75!black,title=E7G03] What is the typical input impedance of an op-amp?
    \begin{enumerate}[label=\Alph*),noitemsep]
        \item 100 ohms
        \item 10,000 ohms
        \item Very low
        \item \textbf{Very high}
    \end{enumerate}
\end{tcolorbox}

\subsubsection{Intuitive Explanation}
Imagine an op-amp as a super-sensitive microphone that listens to electrical signals. The input impedance is like how hard it is for the signal to get into the microphone. If the input impedance is very high, it means the microphone is very easy to talk to—it doesn’t take much effort for the signal to enter. This is good because it doesn’t disturb the original signal. So, op-amps have very high input impedance to make sure they don’t interfere with the signals they are measuring.

\subsubsection{Advanced Explanation}
The input impedance of an operational amplifier (op-amp) is a measure of how much the op-amp resists the flow of current into its input terminals. A high input impedance is desirable because it minimizes the loading effect on the source circuit. Mathematically, input impedance \( Z_{in} \) is defined as the ratio of the input voltage \( V_{in} \) to the input current \( I_{in} \):

\[
Z_{in} = \frac{V_{in}}{I_{in}}
\]

For an ideal op-amp, the input impedance is infinite, meaning no current flows into the input terminals. In practical op-amps, the input impedance is very high, typically in the range of megaohms (M$\Omega$) to gigaohms (G$\Omega$). This high impedance ensures that the op-amp draws negligible current from the source, preserving the integrity of the input signal.

The correct answer is \textbf{D: Very high}, as this is the characteristic input impedance of a typical op-amp.

% Prompt for diagram: A diagram showing an op-amp with labeled input and output terminals, and arrows indicating the high input impedance.