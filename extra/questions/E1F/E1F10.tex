\subsection{Who Can Be the Cheerful Captain of an Auxiliary Station?}

\begin{tcolorbox}[colback=gray!10!white,colframe=black!75!black,title=\textbf{E1F10}]
\textbf{Question:} Who may be the control operator of an auxiliary station?

\begin{enumerate}[label=\Alph*),noitemsep]
    \item Any licensed amateur operator
    \item \textbf{Only Technician, General, Advanced, or Amateur Extra class operators}
    \item Only General, Advanced, or Amateur Extra class operators
    \item Only Amateur Extra class operators
\end{enumerate}
\end{tcolorbox}

\subsubsection{Intuitive Explanation}
Imagine you have a special walkie-talkie that helps other walkie-talkies work better. This special walkie-talkie is called an auxiliary station. Now, not just anyone can be in charge of this special walkie-talkie. Only people who have passed certain tests and have special licenses can be the boss of it. These people are called Technician, General, Advanced, or Amateur Extra class operators. So, if you want to be the cheerful captain of this special walkie-talkie, you need to have one of these licenses.

\subsubsection{Advanced Explanation}
An auxiliary station in amateur radio is a station that is used to retransmit communications automatically. The control operator of such a station must have the necessary qualifications to ensure proper operation and compliance with regulations. According to the Federal Communications Commission (FCC) rules, only licensed amateur operators who hold a Technician, General, Advanced, or Amateur Extra class license are permitted to be the control operator of an auxiliary station. This requirement ensures that the operator has the requisite knowledge and skills to manage the station effectively and adhere to the legal and technical standards.

The Technician class license is the entry-level license, which grants limited privileges on certain bands. The General class license provides more extensive privileges, including access to additional frequency bands. The Advanced class license offers even broader privileges, and the Amateur Extra class license provides the most extensive privileges available to amateur radio operators. Therefore, only operators with these specific licenses are authorized to control an auxiliary station.

% Prompt for generating a diagram:
% Diagram showing the hierarchy of amateur radio licenses (Technician, General, Advanced, Amateur Extra) with an arrow pointing to Control Operator of Auxiliary Station to visually represent the concept.