\subsection{What's Off the Airwaves?}

\begin{tcolorbox}[colback=gray!10!white,colframe=black!75!black,title=E1F09]
\textbf{E1F09} Which of the following cannot be transmitted over an amateur radio mesh network?
\begin{enumerate}[label=\Alph*),noitemsep]
    \item Third party traffic
    \item Email
    \item \textbf{Messages encoded to obscure their meaning}
    \item All these choices are correct
\end{enumerate}
\end{tcolorbox}

\subsubsection*{Intuitive Explanation}
Imagine you and your friends are using walkie-talkies to send messages to each other. You can send regular messages, like Let's meet at the park, or even emails. However, if you try to send a secret code that no one else can understand, that's not allowed. In amateur radio mesh networks, you can send most types of messages, but you can't send messages that are encoded to hide their meaning. This rule helps keep the communication clear and open for everyone.

\subsubsection*{Advanced Explanation}
Amateur radio mesh networks operate under specific regulations that ensure the transparency and legality of the communications. According to the Federal Communications Commission (FCC) rules, messages transmitted over amateur radio must not be encoded to obscure their meaning. This is to prevent the misuse of amateur radio for clandestine or illegal activities.

The other options, such as third-party traffic and email, are permissible under certain conditions. Third-party traffic refers to messages sent on behalf of someone else, which is allowed as long as it complies with the rules. Email can also be transmitted over amateur radio mesh networks, provided it adheres to the same regulations.

In summary, while amateur radio mesh networks are versatile and can handle various types of communications, they must always operate within the legal framework that prohibits the transmission of encoded messages intended to obscure their meaning.

% Prompt for generating a diagram: A diagram showing the flow of different types of messages (third-party traffic, email, and encoded messages) through an amateur radio mesh network, with a clear indication of which types are allowed and which are not.