\subsection{Get Amped: Understanding FCC Certification Standards!}
\label{sec:get-amped}

\begin{tcolorbox}[colback=gray!10!white,colframe=black!75!black,title=Multiple Choice Question]
    \textbf{E1F11} Which of the following best describes one of the standards that must be met by an external RF power amplifier if it is to qualify for a grant of FCC certification?
    \begin{enumerate}[label=\Alph*,noitemsep]
        \item It must produce full legal output when driven by not more than 5 watts of mean RF input power.
        \item It must have received an Underwriters Laboratory certification for electrical safety as well as having met IEEE standard 14.101(B).
        \item It must exhibit a gain of less than 23 dB when driven by 10 watts or less.
        \item \textbf{It must satisfy the FCC’s spurious emission standards when operated at the lesser of 1500 watts or its full output power.}
    \end{enumerate}
\end{tcolorbox}

\subsubsection{Intuitive Explanation}
Imagine you have a big speaker that makes a lot of noise. The government (in this case, the FCC) wants to make sure that this speaker doesn’t make too much extra noise that could bother other people. So, they set some rules. One of the rules is that when the speaker is turned on, it shouldn’t make any weird, unwanted sounds (called spurious emissions) that could interfere with other devices. This rule applies even if the speaker is turned up really loud, but only up to a certain point (1500 watts or its maximum power, whichever is less). This way, everyone can enjoy their music without causing problems for others.

\subsubsection{Advanced Explanation}
The Federal Communications Commission (FCC) sets stringent standards for RF power amplifiers to ensure they do not emit unwanted signals, known as spurious emissions, which could interfere with other communication systems. One of the key requirements for FCC certification is that the amplifier must comply with the FCC’s spurious emission standards when operated at the lesser of 1500 watts or its full output power. This means that the amplifier must be tested to ensure that any unintended emissions are below the specified limits, even when the amplifier is operating at its maximum capacity or up to 1500 watts, whichever is lower.

To understand this mathematically, consider the spurious emission limits set by the FCC. These limits are typically expressed in terms of the power level of the spurious emissions relative to the carrier signal. For example, the FCC might require that spurious emissions be at least 43 + 10 log(P) dB below the carrier power, where P is the power in watts. This ensures that the spurious emissions are sufficiently attenuated and do not cause interference.

In summary, the correct answer is \textbf{D}, as it directly addresses the FCC’s requirement for spurious emission compliance, which is a critical aspect of RF power amplifier certification.

% Prompt for diagram: A diagram showing the relationship between the carrier signal and spurious emissions, with labels indicating the FCC’s spurious emission limits.