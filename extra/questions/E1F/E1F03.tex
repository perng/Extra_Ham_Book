\subsection{Cheers to Circumstances: Selling RF Power Amplifiers!}

\begin{tcolorbox}[colback=gray!10!white,colframe=black!75!black,title=E1F03] Under what circumstances may a dealer sell an external RF power amplifier capable of operation below 144 MHz if it has not been granted FCC certification?
    \begin{enumerate}[label=\Alph*,noitemsep]
        \item Gain is less than 23 dB when driven by power of 10 watts or less
        \item The equipment dealer assembled it from a kit
        \item It was manufactured and certificated in a country which has a reciprocal certification agreement with the FCC
        \item \textbf{The amplifier is constructed or modified by an amateur radio operator for use at an amateur station}
    \end{enumerate}
\end{tcolorbox}

\subsubsection{Intuitive Explanation}
Imagine you have a toy that makes your voice louder when you talk into it. Now, let's say someone wants to sell this toy, but it hasn't been checked by the people who make sure toys are safe and work properly. Normally, they can't sell it. But if you, as someone who loves playing with these toys, make or change one yourself to use at your own play station, then it's okay to sell it. This is like the rule for selling RF power amplifiers without FCC certification.

\subsubsection{Advanced Explanation}
The Federal Communications Commission (FCC) regulates the sale of RF power amplifiers to ensure they comply with specific standards and do not cause harmful interference. According to FCC rules, a dealer may sell an external RF power amplifier capable of operation below 144 MHz without FCC certification if it is constructed or modified by an amateur radio operator for use at an amateur station. This exception is provided under Part 97 of the FCC rules, which governs amateur radio operations.

The rationale behind this exception is that amateur radio operators are expected to have the technical expertise to ensure their equipment operates within legal limits and does not cause interference. This rule allows for innovation and experimentation within the amateur radio community while maintaining regulatory oversight.

% Prompt for generating a diagram: A flowchart showing the conditions under which an RF power amplifier can be sold without FCC certification, highlighting the exception for amateur radio operators.