\subsection{Cheerful Channels: Exploring Spread Spectrum Frequencies!}

\begin{tcolorbox}[colback=gray!10!white,colframe=black!75!black,title=E1F01] On what frequencies are spread spectrum transmissions permitted?
    \begin{enumerate}[label=\Alph*,noitemsep]
        \item Only on amateur frequencies above 50 MHz
        \item \textbf{Only on amateur frequencies above 222 MHz}
        \item Only on amateur frequencies above 420 MHz
        \item Only on amateur frequencies above 144 MHz
    \end{enumerate}
\end{tcolorbox}

\subsubsection{Intuitive Explanation}
Imagine you have a special type of radio that can send messages in a way that spreads them out over a wide range of frequencies. This is called spread spectrum transmission. But just like you can't play loud music in a library, there are rules about where you can use this special radio. The rule is that you can only use it on certain radio frequencies that are higher than 222 MHz. Think of it like a playground where you can only play certain games in certain areas.

\subsubsection{Advanced Explanation}
Spread spectrum transmissions are a method of transmitting radio signals by spreading the signal over a wide range of frequencies. This technique is used to increase the signal's resistance to interference and to make it harder to intercept. According to the Federal Communications Commission (FCC) regulations, spread spectrum transmissions are permitted only on amateur frequencies above 222 MHz. This is to ensure that these transmissions do not interfere with other communication services that operate on lower frequencies.

The specific frequency bands above 222 MHz include the 1.25-meter band (222-225 MHz), the 70-centimeter band (420-450 MHz), and higher bands. These bands are allocated for amateur radio use, and the higher frequencies provide more bandwidth and less interference from other services.

To summarize, the correct answer is \textbf{B}: Only on amateur frequencies above 222 MHz.

% [Prompt for generating a diagram: A frequency spectrum chart showing the amateur radio bands above 222 MHz, highlighting the 1.25-meter and 70-centimeter bands.]