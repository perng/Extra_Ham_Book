\subsection{Understanding Impedance: Unveiling the Joy of 50 - j25 Ohms!}

\begin{tcolorbox}[colback=gray!10!white,colframe=black!75!black,title=E5C06]
\textbf{E5C06} What does the impedance 50 - j25 ohms represent?
\begin{enumerate}[label=\Alph*,noitemsep]
    \item 50 ohms resistance in series with 25 ohms inductive reactance
    \item \textbf{50 ohms resistance in series with 25 ohms capacitive reactance}
    \item 25 ohms resistance in series with 50 ohms inductive reactance
    \item 25 ohms resistance in series with 50 ohms capacitive reactance
\end{enumerate}
\end{tcolorbox}

\subsubsection{Intuitive Explanation}
Imagine you have a water pipe with a certain amount of resistance to the flow of water. Now, think of a capacitor as a device that can store and release water quickly, creating a kind of back-and-forth effect. The impedance \(50 - j25\) ohms tells us that there is a resistance of 50 ohms (like the pipe's resistance) and a capacitive reactance of 25 ohms (like the capacitor's effect). The negative sign in front of the \(j25\) indicates that the reactance is capacitive, meaning it behaves like a capacitor, not an inductor.

\subsubsection{Advanced Explanation}
Impedance is a complex quantity that combines resistance (\(R\)) and reactance (\(X\)) in a circuit. It is represented as \(Z = R + jX\), where \(j\) is the imaginary unit. In this case, the impedance is given as \(50 - j25\) ohms. Here, the real part (50) represents the resistance, and the imaginary part (-25) represents the reactance. The negative sign indicates that the reactance is capacitive, meaning it is due to a capacitor. Therefore, the impedance \(50 - j25\) ohms represents a circuit with 50 ohms of resistance in series with 25 ohms of capacitive reactance.

To further understand, the impedance of a capacitor is given by:
\[
Z_C = \frac{1}{j\omega C} = -j\frac{1}{\omega C}
\]
where \(\omega\) is the angular frequency and \(C\) is the capacitance. The negative sign confirms that the reactance is capacitive.

\subsubsection{Related Concepts}
\begin{itemize}
    \item \textbf{Resistance (\(R\))}: The opposition to the flow of current in a circuit, measured in ohms ($\Omega$).
    \item \textbf{Reactance (\(X\))}: The opposition to the change in current due to inductance or capacitance, measured in ohms ($\Omega$).
    \item \textbf{Capacitive Reactance (\(X_C\))}: The reactance due to a capacitor, given by \(X_C = \frac{1}{\omega C}\).
    \item \textbf{Inductive Reactance (\(X_L\))}: The reactance due to an inductor, given by \(X_L = \omega L\).
    \item \textbf{Impedance (\(Z\))}: The total opposition to current in a circuit, combining resistance and reactance, represented as a complex number \(Z = R + jX\).
\end{itemize}

% Diagram Prompt: Generate a diagram showing a circuit with a resistor and a capacitor in series, labeled with their respective values (50 ohms and 25 ohms capacitive reactance).