\subsection{Capacitive Reactance Fun: Unraveling 100 Ohms in Rectangular Notation!}

\begin{tcolorbox}[colback=gray!10!white,colframe=black!75!black,title=E5C01] Which of the following represents pure capacitive reactance of 100 ohms in rectangular notation?
    \begin{enumerate}[label=\Alph*)]
        \item \textbf{0 - j100}
        \item 0 + j100
        \item 100 - j0
        \item 100 + j0
    \end{enumerate}
\end{tcolorbox}

\subsubsection{Intuitive Explanation}
Imagine you have a capacitor, which is like a tiny storage tank for electricity. When you send electricity through it, the capacitor resists the flow of electricity in a special way called capacitive reactance. This resistance is measured in ohms, just like regular resistance. Now, in the world of electronics, we sometimes use a special way to write down this resistance, called rectangular notation. In this notation, the number before the j tells us about the resistance, and the number after the j tells us about the reactance. Since we're talking about pure capacitive reactance, there's no regular resistance, so the number before the j is 0. The reactance is 100 ohms, so the correct way to write it is 0 - j100.

\subsubsection{Advanced Explanation}
In electrical engineering, impedance is a complex quantity that combines resistance (R) and reactance (X). Impedance in rectangular notation is expressed as:
\[
Z = R + jX
\]
where \( R \) is the real part (resistance) and \( X \) is the imaginary part (reactance). For a purely capacitive reactance, the resistance \( R \) is zero, and the reactance \( X \) is negative because capacitive reactance opposes the flow of current. Therefore, the impedance of a pure capacitive reactance of 100 ohms is:
\[
Z = 0 - j100
\]
This corresponds to option A.

The negative sign in front of the imaginary component indicates that the reactance is capacitive. If the reactance were inductive, the sign would be positive. In this case, since the question specifies pure capacitive reactance, the correct representation is \( 0 - j100 \).

% Prompt for diagram: A diagram showing the relationship between resistance, reactance, and impedance in the complex plane could be helpful here.