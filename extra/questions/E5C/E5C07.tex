\subsection{Plotting Pure Resistance: Where's the Impedance Magic?}

\begin{tcolorbox}[colback=gray!10!white,colframe=black!75!black,title=E5C07] Where is the impedance of a pure resistance plotted on rectangular coordinates?
    \begin{enumerate}[label=\Alph*,noitemsep]
        \item On the vertical axis
        \item On a line through the origin, slanted at 45 degrees
        \item On a horizontal line, offset vertically above the horizontal axis
        \item \textbf{On the horizontal axis}
    \end{enumerate}
\end{tcolorbox}

\subsubsection{Intuitive Explanation}
Imagine you have a simple resistor, which only resists the flow of electricity without storing any energy. When we plot its impedance (which is just its resistance in this case) on a graph with rectangular coordinates, we place it on the horizontal axis. This is because the impedance of a pure resistance doesn't have any imaginary part; it's purely real. So, it sits comfortably on the horizontal line, like a dot on a number line.

\subsubsection{Advanced Explanation}
In electrical engineering, impedance is a complex quantity that combines resistance (real part) and reactance (imaginary part). For a pure resistance, the reactance is zero, meaning the impedance \( Z \) is purely real and can be expressed as:
\[
Z = R + j0
\]
where \( R \) is the resistance and \( j \) is the imaginary unit. When plotting this on a rectangular coordinate system (also known as the complex plane), the real part (resistance) is plotted on the horizontal axis, and the imaginary part (reactance) is plotted on the vertical axis. Since the reactance is zero for a pure resistance, the impedance lies entirely on the horizontal axis.

\subsubsection{Related Concepts}
\begin{itemize}
    \item \textbf{Impedance}: A measure of opposition to alternating current (AC) in a circuit, combining resistance and reactance.
    \item \textbf{Resistance}: The opposition to the flow of electric current, measured in ohms ($\Omega$).
    \item \textbf{Reactance}: The opposition to the change in current due to inductance or capacitance, also measured in ohms ($\Omega$).
    \item \textbf{Complex Plane}: A graphical representation of complex numbers, where the horizontal axis represents the real part and the vertical axis represents the imaginary part.
\end{itemize}

% Prompt for generating a diagram: 
% Diagram showing a rectangular coordinate system with the horizontal axis labeled Real (Resistance) and the vertical axis labeled Imaginary (Reactance). A single point is plotted on the horizontal axis, labeled Pure Resistance.