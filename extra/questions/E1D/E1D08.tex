\subsection{Exploring VHF Bands for Space Station Fun!}

\begin{tcolorbox}[colback=gray!10!white,colframe=black!75!black,title=E1D08] Which VHF amateur bands have frequencies authorized for space stations?
    \begin{enumerate}[label=\Alph*,noitemsep]
        \item 6 meters and 2 meters
        \item 6 meters, 2 meters, and 1.25 meters
        \item 2 meters and 1.25 meters
        \item \textbf{2 meters}
    \end{enumerate}
\end{tcolorbox}

\subsubsection{Intuitive Explanation}
Imagine you have a walkie-talkie that can talk to astronauts in space. Not all walkie-talkie channels work for this, but there’s a special channel called the 2-meter band that does! This is the only channel in the VHF range that space stations are allowed to use. So, if you want to chat with a space station, you’ll need to tune your radio to the 2-meter band.

\subsubsection{Advanced Explanation}
In the context of amateur radio, the VHF (Very High Frequency) bands are segments of the radio spectrum allocated for various uses, including communication with space stations. The specific VHF bands authorized for space station operations are regulated by international agreements and national authorities.

The 2-meter band, which spans from 144 MHz to 148 MHz, is the only VHF band authorized for space station communications. This band is particularly suitable for space-to-Earth and Earth-to-space communications due to its propagation characteristics and the availability of equipment designed for this frequency range.

The other VHF bands, such as the 6-meter band (50-54 MHz) and the 1.25-meter band (222-225 MHz), are not authorized for space station operations. Therefore, the correct answer is \textbf{D: 2 meters}.

% Diagram Prompt: Generate a diagram showing the VHF frequency spectrum with the 2-meter band highlighted for space station communications.