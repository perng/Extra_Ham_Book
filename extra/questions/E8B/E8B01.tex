\subsection{Unwrapping the Magic of FM Signals: What's the Modulation Index?}

\begin{tcolorbox}
\textbf{Question ID: E8B01} \\
What is the modulation index of an FM signal? \\
\begin{enumerate}[label=\Alph*.]
    \item \textbf{The ratio of frequency deviation to modulating signal frequency}
    \item The ratio of modulating signal amplitude to frequency deviation
    \item The modulating signal frequency divided by the bandwidth of the transmitted signal
    \item The bandwidth of the transmitted signal divided by the modulating signal frequency
\end{enumerate}
\end{tcolorbox}

\subsubsection{Intuitive Explanation}
FM signals, or frequency modulation signals, are a type of radio wave where the frequency of the wave changes based on the information (like music or voice) being sent. Think about how a teacher's voice gets higher or lower when they're excited or calm. The modulation index is like a measure of how much the teacher's voice changes. It tells us how big the changes in frequency (like the ups and downs of the voice) are compared to how fast the original sound (modulating signal) is. So basically, it’s a way to understand how much we're twisting or bending the sound in our radio waves!

\subsubsection{Advanced Explanation}
The modulation index \( \beta \) in frequency modulation (FM) is defined mathematically as the ratio of the frequency deviation \( \Delta f \) to the frequency of the modulating signal \( f_m \). This can be represented by the equation:

\[
\beta = \frac{\Delta f}{f_m}
\]

Where:
- \( \Delta f \) is the maximum frequency deviation of the carrier signal from its unmodulated frequency due to modulation.
- \( f_m \) is the frequency of the modulating or baseband signal.

To elaborate, the modulation index is crucial in determining the bandwidth of the FM signal. According to Carson's Rule, the total bandwidth \( B \) of the FM signal is given by:

\[
B = 2(\Delta f + f_m)
\]

Thus, a larger modulation index indicates a more significant deviation, leading to a wider bandwidth, which can accommodate more complex signals and carry more information.

The concept of modulation index also ties into the performance of the transmitted signal, affecting factors like noise immunity and fidelity of reception. 

% If a diagram illustrating the modulation index vs. frequency deviation is needed, generate the diagram here.