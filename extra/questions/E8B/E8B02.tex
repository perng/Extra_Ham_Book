\subsection{Frequencies of Fun: Exploring Modulation Indices!}
\begin{tcolorbox}[colback=gray!10, colframe=gray!80, title={Question ID: \textbf{E8B02}}]
    How does the modulation index of a phase-modulated emission vary with RF carrier frequency?
    \begin{enumerate}[label=\Alph*.]
        \item It increases as the RF carrier frequency increases
        \item It decreases as the RF carrier frequency increases
        \item It varies with the square root of the RF carrier frequency
        \item \textbf{It does not depend on the RF carrier frequency}
    \end{enumerate}
\end{tcolorbox}

\subsubsection{Intuitive Explanation}
Imagine you are playing with a radio that can change the sounds you hear. When you change the frequency of the radio waves, think about how you can make the music sound different without changing the volume or type of music itself. The modulation index is like a secret setting on the radio that tells us how much we want to change the sounds in one way or another. If we change the frequency of the carrier wave, it does not change this special setting; it remains the same no matter how we adjust the carrier frequency. This means that the modulation index does not depend on how high or low the frequency is—it's like knowing that your favorite playlist stays the same no matter what device you use to play it.

\subsubsection{Advanced Explanation}
In phase modulation (PM), the modulation index is defined as the ratio of the peak phase deviation to the frequency of the modulating signal. The modulation index (\( \beta \)) can be expressed mathematically as:

\[
\beta = \frac{\Delta \phi}{\omega_m}
\]

where \( \Delta \phi \) is the peak phase deviation and \( \omega_m \) is the angular frequency of the modulating signal. The RF carrier frequency (\( f_c \)) is the frequency of the electromagnetic wave that carries the modulating signal. Importantly, the modulation index does not have a direct dependency on the carrier frequency \( f_c \). 

When considering the relationship between modulation index and RF carrier frequency, we note that the modulation index remains constant regardless of changes in \( f_c \). Thus, the correct choice in the multiple-choice question is that the modulation index does not depend on the RF carrier frequency.

To elaborate, the modulation index can influence the bandwidth of the signal produced, but this bandwidth depends on the modulating frequency and the modulation index itself, rather than the carrier frequency. The modulation index affects how effectively information is conveyed in the phase-modulated signal but remains a constant factor when only varying the RF carrier frequency.

% Prompt for generating diagram: Create a diagram illustrating the relationship between modulation index, phase deviation, and modulation frequency, showing how carrier frequency does not influence the modulation index.