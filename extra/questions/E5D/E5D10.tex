\subsection{Lengthening the Loop: How Diameter Influences Electrical Path!}

\begin{tcolorbox}[colback=gray!10!white,colframe=black!75!black,title=E5D10] As a conductor’s diameter increases, what is the effect on its electrical length?
    \begin{enumerate}[label=\Alph*.]
        \item Thickness has no effect on electrical length
        \item It varies randomly
        \item It decreases
        \item \textbf{It increases}
    \end{enumerate}
\end{tcolorbox}

\subsubsection{Intuitive Explanation}
Imagine you are walking around a circular path. If the circle gets bigger, the distance you have to walk around it also increases. Similarly, when the diameter of a conductor increases, the path that electricity has to travel around it becomes longer. This means the electrical length of the conductor increases as its diameter grows.

\subsubsection{Advanced Explanation}
The electrical length of a conductor is influenced by its physical dimensions, particularly its diameter. As the diameter of a conductor increases, the circumference of the conductor also increases. The circumference \( C \) of a conductor is given by the formula:

\[
C = \pi \times d
\]

where \( d \) is the diameter of the conductor. Since the electrical length is directly related to the physical length of the path that the current travels, an increase in diameter leads to an increase in the electrical length. This is because the current must travel a longer path around the conductor's circumference.

Additionally, the skin effect plays a role in high-frequency applications. The skin effect causes the current to flow more on the surface of the conductor rather than through its entire cross-section. As the diameter increases, the surface area available for current flow increases, further contributing to the increase in electrical length.

% Diagram Prompt: Generate a diagram showing a conductor with increasing diameter and the corresponding increase in electrical length.