\subsection{E9C01: Delightful Dual Antenna Patterns!}

\begin{tcolorbox}[colback=gray!10!white,colframe=black!75!black]
    \textbf{E9C01} What type of radiation pattern is created by two 1/4-wavelength vertical antennas spaced 1/2-wavelength apart and fed 180 degrees out of phase?
    \begin{enumerate}[label=\Alph*)]
        \item Cardioid
        \item Omni-directional
        \item A figure-eight broadside to the axis of the array
        \item \textbf{A figure-eight oriented along the axis of the array}
    \end{enumerate}
\end{tcolorbox}

\subsubsection{Intuitive Explanation}
Imagine you have two antennas standing side by side, like two friends holding hands but facing opposite directions. When they talk (or in this case, send out radio waves), they don’t just send waves in all directions like a single antenna would. Instead, because they’re out of sync by 180 degrees (like one saying hello while the other says goodbye), their waves cancel each other out in some directions and add up in others. This creates a pattern that looks like a figure-eight, with the antennas pointing along the waist of the eight. So, the correct answer is a figure-eight oriented along the axis of the array. It’s like a dance where the antennas are the dancers, and their moves create a cool pattern!

\subsubsection{Advanced Explanation}
When two 1/4-wavelength vertical antennas are spaced 1/2-wavelength apart and fed 180 degrees out of phase, the resulting radiation pattern is determined by the interference of the electromagnetic waves emitted by each antenna. The phase difference of 180 degrees means that the waves from the two antennas are out of phase by half a cycle, leading to destructive interference in certain directions and constructive interference in others.

The spacing of 1/2-wavelength ensures that the waves from the two antennas interfere in a specific manner. The resulting pattern is a figure-eight (also known as a dipole pattern) oriented along the axis of the array. This means that the maximum radiation occurs along the line connecting the two antennas, while the minimum radiation occurs perpendicular to this line.

Mathematically, the radiation pattern \( E(\theta) \) can be described by the following equation:

\[
E(\theta) = E_0 \left| \cos\left(\frac{\pi}{2} \cos\theta\right) \right|
\]

where \( E_0 \) is the maximum electric field strength and \( \theta \) is the angle relative to the axis of the array. This equation shows that the radiation pattern has nulls at \( \theta = 90^\circ \) and \( \theta = 270^\circ \), and maxima at \( \theta = 0^\circ \) and \( \theta = 180^\circ \), confirming the figure-eight pattern oriented along the axis of the array.

% Diagram prompt: Generate a diagram showing two vertical antennas spaced 1/2-wavelength apart, with arrows indicating the radiation pattern forming a figure-eight along the axis of the array.