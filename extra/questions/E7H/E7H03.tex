\subsection{Unlocking the Magic of Phase-Locked Loops!}

\begin{tcolorbox}[colback=gray!10!white,colframe=black!75!black,title=E7H03] What is a phase-locked loop?
    \begin{enumerate}[label=\Alph*.]
        \item An electronic servo loop consisting of a ratio detector, reactance modulator, and voltage-controlled oscillator
        \item An electronic circuit also known as a monostable multivibrator
        \item \textbf{An electronic servo loop consisting of a phase detector, a low-pass filter, a voltage-controlled oscillator, and a stable reference oscillator}
        \item An electronic circuit consisting of a precision push-pull amplifier with a differential phase input
    \end{enumerate}
\end{tcolorbox}

\subsubsection{Intuitive Explanation}
Imagine you have two friends who are trying to clap their hands at the same time. One friend is the leader, and the other is trying to match the leader's clapping speed. A phase-locked loop (PLL) is like a system that helps the second friend adjust their clapping speed to match the leader's exactly. It does this by checking the difference in timing (phase) between the two claps, smoothing out any mistakes (low-pass filter), and then adjusting the speed (voltage-controlled oscillator) until they are perfectly in sync.

\subsubsection{Advanced Explanation}
A phase-locked loop (PLL) is an electronic control system that generates an output signal whose phase is related to the phase of an input signal. It consists of four main components:

1. \textbf{Phase Detector (PD)}: Compares the phase of the input signal with the phase of the output signal and generates an error signal proportional to the phase difference.

2. \textbf{Low-Pass Filter (LPF)}: Filters out high-frequency components from the error signal, leaving a smooth DC voltage that represents the phase difference.

3. \textbf{Voltage-Controlled Oscillator (VCO)}: Generates an output signal whose frequency is controlled by the DC voltage from the LPF. The VCO adjusts its frequency to minimize the phase difference.

4. \textbf{Stable Reference Oscillator}: Provides a stable reference signal that the PLL tries to lock onto.

The PLL operates in a feedback loop where the phase detector continuously compares the input and output signals, and the VCO adjusts its frequency to minimize the phase difference. When the loop is locked, the output signal's phase is synchronized with the input signal's phase.

Mathematically, the phase difference $\phi_e$ between the input and output signals is given by:
\[
\phi_e = \phi_{in} - \phi_{out}
\]
The phase detector generates an error signal $V_e$ proportional to $\phi_e$:
\[
V_e = K_d \cdot \phi_e
\]
where $K_d$ is the phase detector gain. The low-pass filter smooths $V_e$ to produce a control voltage $V_c$ for the VCO:
\[
V_c = V_e \cdot H(s)
\]
where $H(s)$ is the transfer function of the low-pass filter. The VCO adjusts its frequency $\omega_{out}$ based on $V_c$:
\[
\omega_{out} = \omega_0 + K_v \cdot V_c
\]
where $\omega_0$ is the free-running frequency of the VCO and $K_v$ is the VCO gain. The loop continues to adjust until $\phi_e$ is minimized, achieving phase lock.

% Diagram Prompt: Generate a block diagram of a phase-locked loop showing the phase detector, low-pass filter, voltage-controlled oscillator, and stable reference oscillator.