\subsection{Boosting Vibes: Positive Feedback in Colpitts Oscillators!}

\begin{tcolorbox}[colback=gray!10!white,colframe=black!75!black,title=E7H04] How is positive feedback supplied in a Colpitts oscillator?
    \begin{enumerate}[label=\Alph*),noitemsep]
        \item Through a tapped coil
        \item Through link coupling
        \item \textbf{Through a capacitive divider}
        \item Through a neutralizing capacitor
    \end{enumerate}
\end{tcolorbox}

\subsubsection{Intuitive Explanation}
Imagine you have a swing, and every time you swing back, someone gives you a little push to keep you going. In a Colpitts oscillator, the push that keeps the circuit oscillating is called positive feedback. This feedback is created using something called a capacitive divider. Think of it like a seesaw with two capacitors that work together to send just the right amount of energy back into the circuit to keep it swinging, or in this case, oscillating.

\subsubsection{Advanced Explanation}
In a Colpitts oscillator, positive feedback is essential to sustain oscillations. This feedback is achieved through a capacitive voltage divider, which consists of two capacitors connected in series across the output of the oscillator. The junction between these capacitors is connected to the input of the amplifier stage, providing the necessary phase shift and amplitude to maintain oscillations.

Mathematically, the feedback fraction \(\beta\) is determined by the ratio of the capacitances:
\[
\beta = \frac{C_1}{C_1 + C_2}
\]
where \(C_1\) and \(C_2\) are the capacitances of the two capacitors in the divider. The oscillator will sustain oscillations if the loop gain \(A\beta\) is equal to or greater than 1, where \(A\) is the gain of the amplifier.

The Colpitts oscillator is a type of LC oscillator, where the frequency of oscillation \(f\) is given by:
\[
f = \frac{1}{2\pi \sqrt{L C_{eq}}}
\]
where \(L\) is the inductance of the coil, and \(C_{eq}\) is the equivalent capacitance of the series combination of \(C_1\) and \(C_2\):
\[
C_{eq} = \frac{C_1 C_2}{C_1 + C_2}
\]

This configuration ensures that the feedback is in phase with the input, maintaining the oscillations. The capacitive divider not only provides the necessary feedback but also helps in setting the frequency of oscillation.

% Diagram Prompt: Generate a diagram showing the Colpitts oscillator circuit with the capacitive divider highlighted.