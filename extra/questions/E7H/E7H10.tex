\subsection{Unlocking the Secrets of DDS Lookup Tables!}

\begin{tcolorbox}[colback=gray!10!white,colframe=black!75!black,title=E7H10] What information is contained in the lookup table of a direct digital synthesizer (DDS)?
    \begin{enumerate}[label=\Alph*),noitemsep]
        \item The phase relationship between a reference oscillator and the output waveform
        \item \textbf{Amplitude values that represent the desired waveform}
        \item The phase relationship between a voltage-controlled oscillator and the output waveform
        \item Frequently used receiver and transmitter frequencies
    \end{enumerate}
\end{tcolorbox}

\subsubsection{Intuitive Explanation}
Imagine you have a music box that plays a specific tune. The lookup table in a DDS is like the sheet music for that tune. Instead of notes, it contains a list of numbers that tell the DDS how to create a specific sound wave. These numbers represent the height (amplitude) of the wave at different points in time. By following this list, the DDS can produce the exact waveform you want, just like the music box plays the correct tune.

\subsubsection{Advanced Explanation}
A Direct Digital Synthesizer (DDS) generates waveforms by using a phase accumulator and a lookup table. The phase accumulator keeps track of the current phase of the waveform, and the lookup table contains precomputed amplitude values that correspond to specific phase angles. 

The process can be broken down as follows:
\begin{enumerate}
    \item The phase accumulator increments its value based on a frequency control word.
    \item The current phase value is used to index into the lookup table.
    \item The lookup table outputs the amplitude value corresponding to the current phase.
    \item This amplitude value is then converted to an analog signal using a digital-to-analog converter (DAC).
\end{enumerate}

Mathematically, the phase accumulator can be represented as:
\[
\phi[n] = (\phi[n-1] + \Delta\phi) \mod 2^N
\]
where \(\phi[n]\) is the current phase, \(\Delta\phi\) is the phase increment, and \(N\) is the number of bits in the phase accumulator.

The lookup table contains the amplitude values \(A(\phi)\) for each phase \(\phi\), which can be represented as:
\[
A(\phi) = \sin\left(\frac{2\pi\phi}{2^N}\right)
\]
where \(\sin\) is the sine function, and \(2^N\) is the total number of possible phase values.

The correct answer is \textbf{B}, as the lookup table contains amplitude values that represent the desired waveform. This is essential for the DDS to accurately generate the required waveform.

% Prompt for diagram: A diagram showing the block diagram of a DDS, including the phase accumulator, lookup table, and DAC, would be helpful here.