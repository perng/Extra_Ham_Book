\subsection{Exploring the T-Network: Unveiling Frequency Response Fun!}

\begin{tcolorbox}[colback=gray!10!white,colframe=black!75!black,title=E7C02] What is the frequency response of a T-network with series capacitors and a shunt inductor?
    \begin{enumerate}[label=\Alph*]
        \item Low-pass
        \item \textbf{High-pass}
        \item Band-pass
        \item Notch
    \end{enumerate}
\end{tcolorbox}

\subsubsection{Intuitive Explanation}
Imagine you have a T-shaped network made of two capacitors in the arms and an inductor in the middle. Capacitors are like gates that block low-frequency signals but let high-frequency signals pass through. Inductors, on the other hand, are like gates that block high-frequency signals but let low-frequency signals pass through. When you combine these components in a T-network, the capacitors in the arms allow high-frequency signals to pass through while the inductor in the middle blocks low-frequency signals. This means the network as a whole lets high-frequency signals pass through more easily, making it a high-pass filter.

\subsubsection{Advanced Explanation}
A T-network with series capacitors and a shunt inductor can be analyzed using impedance concepts. The impedance of a capacitor \(C\) is given by \(Z_C = \frac{1}{j\omega C}\), where \(\omega\) is the angular frequency. The impedance of an inductor \(L\) is \(Z_L = j\omega L\). 

At low frequencies, the impedance of the capacitors is very high, effectively blocking the signal, while the impedance of the inductor is low, allowing the signal to pass through the shunt path. However, at high frequencies, the impedance of the capacitors decreases, allowing the signal to pass through the series arms, while the impedance of the inductor increases, blocking the shunt path. 

The transfer function \(H(\omega)\) of the network can be derived as follows:

\[
H(\omega) = \frac{V_{out}}{V_{in}} = \frac{Z_C}{Z_C + Z_L}
\]

Substituting the impedances:

\[
H(\omega) = \frac{\frac{1}{j\omega C}}{\frac{1}{j\omega C} + j\omega L} = \frac{1}{1 - \omega^2 LC}
\]

For \(\omega \to 0\), \(H(\omega) \to 0\), indicating that low frequencies are attenuated. For \(\omega \to \infty\), \(H(\omega) \to 1\), indicating that high frequencies are passed. Therefore, the network exhibits a high-pass frequency response.

% Diagram prompt: Generate a diagram of a T-network with two series capacitors and a shunt inductor, showing the input and output terminals.