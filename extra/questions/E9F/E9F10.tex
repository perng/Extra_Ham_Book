\subsection{Impedance Insights: Unlocking the Wonders of 1/8-Wavelength Lines!}
\label{sec:E9F10}

\begin{tcolorbox}[colback=blue!5!white,colframe=blue!75!black,title=\textbf{Question E9F10}]
\textbf{What impedance does a 1/8-wavelength transmission line present to an RF generator when the line is shorted at the far end?}
\begin{enumerate}[label=\Alph*,noitemsep]
    \item A capacitive reactance
    \item The same as the characteristic impedance of the line
    \item \textbf{An inductive reactance}
    \item Zero
\end{enumerate}
\end{tcolorbox}

\subsubsection{Intuitive Explanation}
Imagine you have a jump rope that’s exactly 1/8th the length of a full wave. If you hold one end and your friend holds the other end but doesn’t move it (that’s the shorted part), the rope will behave in a certain way. When you wiggle your end, the rope will act like it’s trying to push back against your movement, almost like a spring. This push back is similar to what we call an inductive reactance. So, the transmission line, like the jump rope, presents an inductive reactance to the RF generator.

\subsubsection{Advanced Explanation}
When a transmission line is shorted at the far end, the impedance at the input end can be determined using the transmission line theory. For a lossless transmission line of length \( l \) and characteristic impedance \( Z_0 \), the input impedance \( Z_{in} \) is given by:

\[
Z_{in} = j Z_0 \tan(\beta l)
\]

where \( \beta = \frac{2\pi}{\lambda} \) is the phase constant, and \( \lambda \) is the wavelength. For a 1/8-wavelength line, \( l = \frac{\lambda}{8} \), so:

\[
\beta l = \frac{2\pi}{\lambda} \cdot \frac{\lambda}{8} = \frac{\pi}{4}
\]

Thus,

\[
Z_{in} = j Z_0 \tan\left(\frac{\pi}{4}\right) = j Z_0 \cdot 1 = j Z_0
\]

This result indicates that the input impedance is purely imaginary and positive, which corresponds to an inductive reactance. Therefore, the correct answer is \textbf{C}.

\subsubsection{Related Concepts}
\begin{itemize}
    \item \textbf{Transmission Line Theory}: Understanding how signals propagate along transmission lines and how impedance varies with line length and termination.
    \item \textbf{Impedance Matching}: The process of ensuring that the impedance of the transmission line matches the impedance of the load to minimize reflections.
    \item \textbf{Reactance}: The opposition to alternating current due to inductance (inductive reactance) or capacitance (capacitive reactance).
\end{itemize}

% Prompt for generating a diagram:
% Diagram showing a 1/8-wavelength transmission line shorted at the far end, with an RF generator connected at the input end. The diagram should illustrate the inductive reactance at the input.