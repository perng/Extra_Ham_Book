\subsection{Next Steps for VEs After a Successful Amateur Exam!}

\begin{tcolorbox}[colback=gray!10!white,colframe=black!75!black,title=E1E10] What must the administering VEs do after the administration of a successful examination for an amateur operator license?
    \begin{enumerate}[label=\Alph*.]
        \item They must collect and send the documents directly to the FCC
        \item They must collect and submit the documents to the coordinating VEC for grading
        \item \textbf{They must submit the application document to the coordinating VEC according to the coordinating VEC instructions}
        \item They must return the documents to the applicant for submission to the FCC according to the FCC instructions
    \end{enumerate}
\end{tcolorbox}

\subsubsection{Intuitive Explanation}
After someone passes their amateur radio license exam, the people who administered the test (called Volunteer Examiners or VEs) have a specific job to do. They don’t just hand the paperwork back to the person who took the test or send it directly to the government. Instead, they follow the instructions given by the group that organized the exam (called the coordinating VEC). This group knows exactly how to handle the paperwork to make sure everything is done correctly and the new license is issued properly.

\subsubsection{Advanced Explanation}
After a successful amateur operator license examination, the Volunteer Examiners (VEs) are responsible for ensuring that the application documents are processed correctly. The VEs must follow the specific instructions provided by the coordinating Volunteer Examiner Coordinator (VEC). The coordinating VEC acts as an intermediary between the VEs and the Federal Communications Commission (FCC). The VEs submit the application documents to the coordinating VEC, which then reviews and forwards the necessary paperwork to the FCC. This process ensures that all regulatory requirements are met and that the applicant’s license is issued in a timely manner. 

The correct answer, \textbf{C}, emphasizes the importance of adhering to the coordinating VEC’s instructions, which streamlines the submission process and ensures compliance with FCC regulations. This step is crucial for maintaining the integrity and efficiency of the amateur radio licensing system.

% [Prompt for generating a diagram: A flowchart showing the process from exam administration to license issuance, highlighting the roles of VEs, coordinating VEC, and FCC.]