\subsection{Who Can't Take the Exam? Let’s Find Out!}

\begin{tcolorbox}[colback=gray!10!white,colframe=black!75!black,title=Multiple Choice Question]
\textbf{E1E08} To which of the following examinees may a VE not administer an examination?
\begin{enumerate}[label=\Alph*),noitemsep]
    \item Employees of the VE
    \item Friends of the VE
    \item \textbf{Relatives of the VE as listed in the FCC rules}
    \item All these choices are correct
\end{enumerate}
\end{tcolorbox}

\subsubsection{Intuitive Explanation}
Imagine you are taking a test, and the person giving you the test is your uncle or aunt. That might not seem fair, right? The rules say that a Volunteer Examiner (VE) cannot give the test to their close relatives. This is to make sure everyone has a fair chance and there’s no favoritism. So, if the VE is related to you in a way that’s listed in the FCC rules, they can’t be the one to give you the test.

\subsubsection{Advanced Explanation}
The Federal Communications Commission (FCC) has specific rules to ensure the integrity of the examination process for amateur radio licenses. According to these rules, a Volunteer Examiner (VE) is prohibited from administering an examination to their relatives as defined by the FCC. This includes immediate family members such as spouses, children, parents, and siblings. The rationale behind this rule is to prevent any potential conflicts of interest or bias that could arise from personal relationships. 

The FCC rules are designed to maintain a fair and impartial examination environment. By excluding relatives from being examined by a VE, the FCC ensures that all candidates are evaluated based solely on their knowledge and skills, without any undue influence. This rule is part of a broader set of regulations that govern the conduct of VEs and the administration of amateur radio license examinations.

% Diagram prompt: A flowchart showing the relationships between VEs and examinees, highlighting the prohibited relationships as per FCC rules.