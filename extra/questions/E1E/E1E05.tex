\subsection{Next Steps for Success: Handling Application Forms After Exam Outcomes!}

\begin{tcolorbox}[colback=gray!10!white,colframe=black!75!black,title=E1E05] What must the VE team do with the application form if the examinee does not pass the exam?
    \begin{enumerate}[label=\Alph*,noitemsep]
        \item Maintain the application form with the VEC’s records
        \item \textbf{Return the application document to the examinee}
        \item Send the application form to the FCC and inform the FCC of the grade
        \item Destroy the application form
    \end{enumerate}
\end{tcolorbox}

\subsubsection{Intuitive Explanation}
Imagine you took a test, but unfortunately, you didn’t pass. What happens to the form you filled out to take the test? The people who gave you the test (the VE team) don’t need to keep it anymore. Instead, they give it back to you. This way, you have your own copy, and they don’t have to worry about storing it. It’s like returning a borrowed book to its owner after you’ve finished reading it.

\subsubsection{Advanced Explanation}
When an examinee does not pass the exam, the Volunteer Examiner (VE) team is responsible for handling the application form appropriately. According to standard procedures, the VE team must return the application document to the examinee. This ensures that the examinee retains their personal information and documentation, and the VE team does not retain unnecessary records. 

This process aligns with privacy and data management best practices, ensuring that personal information is not stored longer than necessary. Additionally, it simplifies record-keeping for the VE team, as they only maintain records for successful candidates who require further processing, such as certification issuance.

% Diagram Prompt: Consider generating a flowchart showing the process of handling application forms after an exam, including the decision point for passing or failing and the corresponding actions for each outcome.