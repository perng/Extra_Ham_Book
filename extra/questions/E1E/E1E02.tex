\subsection{Who's the Keeper of the Question Pools for Ham Radio Licenses?}

\begin{tcolorbox}[colback=gray!10!white,colframe=black!75!black,title=E1E02] Who is tasked by Part 97 with maintaining the pools of questions for all US amateur license examinations?
    \begin{enumerate}[label=\Alph*)]
        \item The VEs
        \item The FCC
        \item \textbf{The VECs}
        \item The ARRL
    \end{enumerate}
\end{tcolorbox}

\subsubsection{Intuitive Explanation}
Imagine you're playing a game where you need to answer questions to level up. Someone has to make sure those questions are fair and cover all the important topics. In the world of ham radio, the people who create and manage these questions are called the VECs (Volunteer Examiner Coordinators). They make sure the questions are up-to-date and relevant so that everyone who wants to get a ham radio license has a fair chance.

\subsubsection{Advanced Explanation}
Under Part 97 of the Federal Communications Commission (FCC) rules, the responsibility for maintaining the question pools for all US amateur radio license examinations is assigned to the Volunteer Examiner Coordinators (VECs). The VECs are organizations that coordinate the efforts of Volunteer Examiners (VEs) who administer the exams. The VECs ensure that the question pools are comprehensive, current, and in compliance with FCC regulations. This process involves periodic review and updating of the questions to reflect changes in technology and regulations. The FCC oversees the entire licensing process but delegates the specific task of question pool maintenance to the VECs. The ARRL (American Radio Relay League) is one of the VECs but not the sole entity responsible for this task.

% Prompt for diagram: A flowchart showing the relationship between the FCC, VECs, VEs, and the question pool maintenance process could be helpful here.