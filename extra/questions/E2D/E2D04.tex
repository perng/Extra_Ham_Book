\subsection{Balloon Tracking Magic: How Tech Keeps Tabs on Amateur Radio!}

\begin{tcolorbox}[colback=gray!10!white,colframe=black!75!black,title=E2D04] What technology is used for real-time tracking of balloons carrying amateur radio transmitters?
    \begin{enumerate}[label=\Alph*,noitemsep]
        \item FT8
        \item Bandwidth compressed LORAN
        \item \textbf{APRS}
        \item PACTOR III
    \end{enumerate}
\end{tcolorbox}

\subsubsection{Intuitive Explanation}
Imagine you have a balloon that’s floating high up in the sky, and it’s carrying a small radio transmitter. You want to know exactly where it is at any moment, just like how you can track a package you ordered online. The technology that makes this possible is called APRS, which stands for Automatic Packet Reporting System. It’s like a GPS for balloons! APRS sends little bits of information (called packets) from the balloon to the ground, telling people where the balloon is in real-time. So, if you’re curious about where the balloon is, you can just check the APRS system, and it will show you the balloon’s location on a map.

\subsubsection{Advanced Explanation}
APRS (Automatic Packet Reporting System) is a digital communication protocol used for real-time tracking and communication. It operates on the 2-meter amateur radio band (144.390 MHz in the United States) and uses packet radio to transmit data. The system encodes information such as GPS coordinates, altitude, speed, and other telemetry data into digital packets, which are then transmitted to APRS receivers on the ground. These receivers decode the packets and display the information on a map, allowing for real-time tracking of the balloon’s position.

APRS is particularly useful for tracking high-altitude balloons (HABs) because it provides a reliable and efficient way to monitor their location and status. The system can also be used for other applications, such as weather reporting, emergency communications, and vehicle tracking.

The correct answer to the question is \textbf{C: APRS}, as it is the technology specifically designed for real-time tracking of amateur radio transmitters, including those carried by balloons.

% Prompt for generating a diagram: A diagram showing a high-altitude balloon with an APRS transmitter sending data packets to a ground station, which then displays the balloon's location on a map.