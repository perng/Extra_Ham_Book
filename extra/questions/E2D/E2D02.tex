\subsection{Unlocking FT8/FT4: The New Metric for VHF Contests!}

\begin{tcolorbox}[colback=gray!10!white,colframe=black!75!black]
\textbf{E2D02} What information replaces signal-to-noise ratio when using the FT8 or FT4 modes in a VHF contest?

\begin{enumerate}[label=\Alph*,noitemsep]
    \item RST report
    \item State abbreviation
    \item Serial number
    \item \textbf{Grid square}
\end{enumerate}
\end{tcolorbox}

\subsubsection*{Intuitive Explanation}
When you're using FT8 or FT4 modes in a VHF contest, instead of talking about how strong your signal is compared to the noise (which is what signal-to-noise ratio means), you use something called a grid square. A grid square is a way to tell exactly where you are on the Earth. It's like giving your address but in a special code that everyone can understand. This helps people know where the signal is coming from, which is super important in contests!

\subsubsection*{Advanced Explanation}
In traditional radio communication, the signal-to-noise ratio (SNR) is a critical metric that quantifies the strength of the desired signal relative to the background noise. However, in the context of FT8 and FT4 digital modes, particularly during VHF contests, the focus shifts from SNR to the exchange of grid squares. 

A grid square, also known as a Maidenhead Locator, is a geographic coordinate system that divides the Earth into a grid of squares, each identified by a unique alphanumeric code. This system allows for precise location identification, which is essential in VHF contests where participants often operate from fixed locations and need to confirm their positions accurately.

The grid square is exchanged as part of the digital message in FT8/FT4 modes. This exchange replaces the traditional SNR metric because the digital nature of these modes inherently provides robust error correction and decoding capabilities, making SNR less critical. Instead, the grid square provides valuable information about the operator's location, which is a key element in contest scoring and verification.

\subsubsection*{Related Concepts}
- \textbf(Maidenhead Locator System): A grid-based system used to specify locations on Earth.
- \textbf(FT8/FT4 Modes): Digital communication modes designed for weak signal communication, often used in amateur radio contests.
- \textbf(VHF Contests): Competitions where amateur radio operators communicate over very high frequency (VHF) bands, often focusing on distance and location.

% Diagram Prompt: Generate a diagram showing the Earth divided into grid squares with a few examples of grid square codes labeled.