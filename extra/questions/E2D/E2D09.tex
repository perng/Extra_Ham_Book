\subsection{Discovering the Magic of JT65 Modulation!}

\begin{tcolorbox}[colback=gray!10!white,colframe=black!75!black,title=E2D09] What type of modulation is used by JT65?  
    \begin{enumerate}[label=\Alph*,noitemsep]
        \item \textbf{Multitone AFSK}
        \item PSK
        \item RTTY
        \item QAM
    \end{enumerate}
\end{tcolorbox}

\subsubsection{Intuitive Explanation}
Imagine you are sending a secret message to a friend using different musical notes. Each note represents a piece of your message. JT65 is like this musical message system, but instead of using musical notes, it uses different tones (sounds) to send information. This method is called Multitone AFSK, where multiple tones are used to encode the data. It’s like playing a melody that carries your message across the airwaves!

\subsubsection{Advanced Explanation}
JT65 employs a modulation technique known as \textit{Multitone AFSK} (Audio Frequency Shift Keying). In this method, the data is encoded using multiple audio tones, each representing a specific symbol or piece of information. The JT65 protocol uses 65 distinct tones, each separated by a fixed frequency interval. These tones are transmitted sequentially, and the receiver decodes them to reconstruct the original message.

Mathematically, each tone can be represented as:
\[
s(t) = A \cos(2\pi f_i t + \phi)
\]
where \(A\) is the amplitude, \(f_i\) is the frequency of the \(i\)-th tone, and \(\phi\) is the phase. The receiver detects these tones using a Fast Fourier Transform (FFT) to identify the frequency components present in the signal.

JT65 is particularly effective for weak signal communication because it allows for very narrow bandwidths and high sensitivity. The use of multiple tones ensures that even if some tones are lost due to noise or interference, the remaining tones can still be used to decode the message accurately.

% Diagram Prompt: Generate a diagram showing the frequency spectrum of JT65 with multiple tones spaced at fixed intervals.