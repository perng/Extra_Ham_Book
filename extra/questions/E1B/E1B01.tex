\subsection{Spotting Spurious Emissions: A Fun Quiz!}

\begin{tcolorbox}[colback=gray!10!white,colframe=black!75!black,title=Multiple Choice Question]
    \textbf{E1B01} Which of the following constitutes a spurious emission?
    \begin{enumerate}[label=\Alph*),noitemsep]
        \item An amateur station transmission made without the proper call sign identification
        \item A signal transmitted to prevent its detection by any station other than the intended recipient
        \item Any transmitted signal that unintentionally interferes with another licensed radio station and whose levels exceed 40 dB below the fundamental power level
        \item \textbf{An emission outside the signal’s necessary bandwidth that can be reduced or eliminated without affecting the information transmitted}
    \end{enumerate}
\end{tcolorbox}

\subsubsection*{Intuitive Explanation}
Imagine you're listening to your favorite radio station, and suddenly you hear some weird noises or static that don't belong to the music or the DJ's voice. These unwanted noises are like spurious emissions. They are extra signals that come out of a radio transmitter but aren't part of the main message or music. The correct answer, \textbf{D}, tells us that these are emissions outside the necessary bandwidth (the range of frequencies needed for the main signal) and can be removed without messing up the actual information being sent.

\subsubsection*{Advanced Explanation}
In radio communications, a spurious emission refers to any emission that occurs outside the necessary bandwidth of a signal. The necessary bandwidth is the range of frequencies required to transmit the information without distortion. Spurious emissions are typically unwanted and can interfere with other communications. According to the International Telecommunication Union (ITU), spurious emissions can be reduced or eliminated without affecting the information transmitted, as they are not part of the essential signal.

Mathematically, the power level of spurious emissions is often measured in decibels (dB) relative to the fundamental power level. For example, if a spurious emission is 40 dB below the fundamental, it means it is 10,000 times weaker. However, the key characteristic of a spurious emission is that it is outside the necessary bandwidth and can be minimized without impacting the transmitted information.

Related concepts include:
\begin{itemize}
    \item \textbf{Necessary Bandwidth}: The range of frequencies required to transmit the information without distortion.
    \item \textbf{Harmonics}: Frequencies that are integer multiples of the fundamental frequency, often a source of spurious emissions.
    \item \textbf{Intermodulation Products}: Unwanted frequencies generated when two or more signals mix in a non-linear device.
\end{itemize}

% Prompt for diagram: A diagram showing the frequency spectrum of a signal with spurious emissions outside the necessary bandwidth would be helpful here.