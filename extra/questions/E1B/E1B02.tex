\subsection{Exploring Acceptable Bandwidths for HF Voice \& TV Fun!}

\begin{tcolorbox}[colback=gray!10!white,colframe=black!75!black,title=E1B02] Which of the following is an acceptable bandwidth for digital voice or slow-scan TV transmissions made on the HF amateur bands?
    \begin{enumerate}[label=\Alph*),noitemsep]
        \item \textbf{3 kHz}
        \item 10 kHz
        \item 15 kHz
        \item 20 kHz
    \end{enumerate}
\end{tcolorbox}

\subsubsection{Intuitive Explanation}
Imagine you’re trying to send a voice message or a picture over a radio wave. The radio wave has a certain width called bandwidth, which determines how much information it can carry. For digital voice or slow-scan TV on HF bands, you don’t need a very wide bandwidth because these types of transmissions don’t carry a lot of data. Think of it like sending a small package through a narrow pipe—it’s just the right size! In this case, 3 kHz is the perfect width for these transmissions.

\subsubsection{Advanced Explanation}
The HF (High Frequency) amateur bands are typically used for long-distance communication. The bandwidth of a signal is crucial because it determines the amount of data that can be transmitted. For digital voice and slow-scan TV (SSTV) transmissions, the required bandwidth is relatively low. 

Digital voice signals are compressed and encoded, allowing them to fit within a narrow bandwidth. Similarly, SSTV transmits images by converting them into audio signals, which also do not require a wide bandwidth. The International Telecommunication Union (ITU) and amateur radio regulations specify that the bandwidth for such transmissions should be kept as narrow as possible to minimize interference with other users of the spectrum.

Mathematically, the bandwidth \( B \) is related to the data rate \( R \) and the modulation scheme used. For digital voice and SSTV, the data rate is low, and the modulation schemes (such as Single Sideband, SSB) are efficient, allowing the signal to fit within a 3 kHz bandwidth. This is why 3 kHz is the correct answer.

\subsubsection{Related Concepts}
\begin{itemize}
    \item \textbf{Bandwidth}: The range of frequencies occupied by a signal. It is a critical parameter in determining the capacity of a communication channel.
    \item \textbf{Digital Voice}: A method of transmitting voice signals by converting them into digital data, which can be compressed and transmitted efficiently.
    \item \textbf{Slow-Scan TV (SSTV)}: A method of transmitting still images over radio waves by converting them into audio signals.
    \item \textbf{Modulation Schemes}: Techniques used to encode information onto a carrier wave. Common schemes include AM, FM, and SSB.
\end{itemize}

% [Prompt for diagram: A diagram showing the frequency spectrum of an HF band with a 3 kHz bandwidth signal marked for digital voice or SSTV transmissions.]