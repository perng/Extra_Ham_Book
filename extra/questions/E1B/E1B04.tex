\subsection{Keeping Signals Clear: Handling Repeater Interference!}

\begin{tcolorbox}[colback=gray!10!white,colframe=black!75!black,title=\textbf{E1B04}]
What must the control operator of a repeater operating in the 70-centimeter band do if a radiolocation system experiences interference from that repeater?
\begin{enumerate}[label=\Alph*.]
    \item Reduce the repeater antenna HAAT (Height Above Average Terrain)
    \item File an FAA NOTAM (Notice to Air Missions) with the repeater system's ERP, call sign, and six-character grid locator
    \item \textbf{Cease operation or make changes to the repeater that mitigate the interference}
    \item All these choices are correct
\end{enumerate}
\end{tcolorbox}

\subsubsection*{Intuitive Explanation}
Imagine you have a walkie-talkie and you’re talking to your friend, but someone else’s walkie-talkie is causing static and making it hard to hear. If you’re the one causing the static, you need to either stop talking or adjust your walkie-talkie so it doesn’t interfere anymore. Similarly, if a repeater (which helps boost signals) is causing interference with a radiolocation system (like a radar), the person in charge of the repeater must either stop using it or make changes to prevent the interference.

\subsubsection*{Advanced Explanation}
In the context of radio communication, a repeater operating in the 70-centimeter band (420-450 MHz) can sometimes cause interference with radiolocation systems, which are used for navigation and positioning. When such interference occurs, the control operator of the repeater is legally obligated to take corrective action under FCC regulations. The correct action, as per the FCC rules, is to either cease operation of the repeater or implement technical modifications to mitigate the interference. This could involve adjusting the frequency, reducing power output, or altering the antenna configuration to minimize the impact on the radiolocation system.

The other options provided are not appropriate in this context:
\begin{itemize}
    \item Reducing the HAAT (Height Above Average Terrain) of the repeater antenna might help in some cases, but it is not the immediate required action.
    \item Filing an FAA NOTAM is not relevant to resolving interference issues between a repeater and a radiolocation system.
    \item The option All these choices are correct is incorrect because only ceasing operation or making changes to mitigate interference is the legally mandated action.
\end{itemize}

Therefore, the correct answer is \textbf{C}.

% Prompt for diagram: A diagram showing a repeater system and a radiolocation system with overlapping signals, illustrating the interference scenario.