\subsection{Global Ham Harmony: Bridging Amateurs Across Borders!}

\begin{tcolorbox}[colback=blue!5!white,colframe=blue!75!black]
    \textbf{E1C11} Which of the following operating arrangements allows an FCC-licensed US citizen to operate in many European countries, and amateurs from many European countries to operate in the US?
    \begin{enumerate}[label=\Alph*),noitemsep]
        \item \textbf{CEPT}
        \item IARP
        \item ITU reciprocal license
        \item All these choices are correct
    \end{enumerate}
\end{tcolorbox}

\subsubsection{Intuitive Explanation}
Imagine you have a special pass that lets you visit many countries without needing a new visa every time. Similarly, in the world of amateur radio, there is an agreement called CEPT that allows radio operators from the United States to operate in many European countries, and vice versa. This agreement makes it easier for amateur radio enthusiasts to communicate across borders without needing to get a new license for each country.

\subsubsection{Advanced Explanation}
The CEPT (European Conference of Postal and Telecommunications Administrations) agreement is a reciprocal licensing arrangement that facilitates the operation of amateur radio across participating countries. Under this agreement, an FCC-licensed US citizen can operate in CEPT member countries without needing to obtain a separate license, provided they adhere to the regulations of the host country. Similarly, amateur radio operators from CEPT countries can operate in the US under the same conditions.

The IARP (International Amateur Radio Permit) is another arrangement, but it is not as widely recognized as CEPT. The ITU (International Telecommunication Union) reciprocal license is a broader concept but does not specifically address the ease of operation between the US and Europe as CEPT does. Therefore, the correct answer is CEPT, as it is the most relevant and widely accepted arrangement for this scenario.

% Diagram Prompt: A diagram showing the flow of amateur radio signals between the US and European countries under the CEPT agreement could help visualize the concept.