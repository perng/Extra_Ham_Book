\subsection{Bright Ideas: Unpacking the Magic of TV Signals!}
\label{sec:E2B10}

\begin{tcolorbox}[colback=blue!5!white,colframe=blue!75!black]
    \textbf{Question E2B10}: What aspect of an analog slow-scan television signal encodes the brightness of the picture?
    \begin{enumerate}[label=\Alph*)]
        \item \textbf{Tone frequency}
        \item Tone amplitude
        \item Sync amplitude
        \item Sync frequency
    \end{enumerate}
\end{tcolorbox}

\subsubsection{Intuitive Explanation}
Imagine you are watching an old black-and-white TV. The brightness of the picture is like how light or dark each part of the image appears. In slow-scan television signals, the brightness is controlled by something called the tone frequency. Think of it like a musical note: the higher the note, the brighter the picture, and the lower the note, the darker the picture. So, the tone frequency is what tells the TV how bright or dark each part of the image should be.

\subsubsection{Advanced Explanation}
In analog slow-scan television (SSTV), the brightness of the picture is encoded using the frequency of the audio tone. This technique is known as frequency modulation (FM). The frequency of the tone varies in proportion to the brightness of the image. For example, a higher frequency corresponds to a brighter pixel, while a lower frequency corresponds to a darker pixel.

Mathematically, the relationship between the tone frequency \( f \) and the brightness \( B \) can be expressed as:
\[
f = f_0 + k \cdot B
\]
where:
\begin{itemize}
    \item \( f_0 \) is the base frequency,
    \item \( k \) is a constant of proportionality,
    \item \( B \) is the brightness level.
\end{itemize}

The sync amplitude and sync frequency are used to synchronize the image frames and lines, but they do not encode the brightness information. The tone amplitude, on the other hand, is related to the strength of the signal but not directly to the brightness of the image.

% Diagram prompt: Generate a diagram showing the relationship between tone frequency and brightness in an SSTV signal.