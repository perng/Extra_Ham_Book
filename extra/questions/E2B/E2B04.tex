\subsection{Bright Signals: Unraveling Color Transmission in Analog SSTV!}

\begin{tcolorbox}[colback=gray!10!white,colframe=black!75!black,title=Multiple Choice Question]
\textbf{E2B04} How is color information sent in analog SSTV?

\begin{enumerate}[label=\Alph*,noitemsep]
    \item \textbf{Color lines are sent sequentially}
    \item Color information is sent on a 2.8 kHz subcarrier
    \item Color is sent in a color burst at the end of each line
    \item Color is amplitude modulated on the frequency modulated intensity signal
\end{enumerate}
\end{tcolorbox}

\subsubsection{Intuitive Explanation}
Imagine you are watching a black-and-white movie, and someone wants to add color to it. Instead of mixing all the colors together at once, they decide to add one color at a time, line by line. This is similar to how color information is sent in analog Slow Scan Television (SSTV). The colors are sent one after the other, in a sequence, so that the final picture has all the colors in the right places.

\subsubsection{Advanced Explanation}
In analog SSTV, color information is transmitted using a method called sequential color transmission. This means that the color lines are sent one after another, rather than simultaneously. Each line of the image is assigned a specific color, and these lines are transmitted in a sequence. This method ensures that the color information is accurately represented in the final image.

The process involves dividing the image into lines, each corresponding to a specific color. These lines are then transmitted sequentially, allowing the receiver to reconstruct the image with the correct colors. This method is efficient and ensures that the color information is not lost or distorted during transmission.

Mathematically, this can be represented as:
\[
C(t) = \sum_{n=1}^{N} c_n(t)
\]
where \( C(t) \) is the total color signal, \( c_n(t) \) is the color signal for the \( n \)-th line, and \( N \) is the total number of lines in the image.

This sequential transmission method is crucial for maintaining the integrity of the color information in analog SSTV, ensuring that the final image is as close to the original as possible.

% Prompt for generating a diagram: 
% Diagram showing the sequential transmission of color lines in analog SSTV, with each line representing a different color and being transmitted one after another.