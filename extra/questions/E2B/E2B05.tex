\subsection{Unlocking the Secrets of Vestigial Sideband in Analog TV!}

\begin{tcolorbox}[colback=gray!10!white,colframe=black!75!black,title=E2B05] Which of the following describes the use of vestigial sideband in analog fast-scan TV transmissions?
    \begin{enumerate}[label=\Alph*,noitemsep]
        \item The vestigial sideband carries the audio information
        \item The vestigial sideband contains chroma information
        \item \textbf{Vestigial sideband reduces the bandwidth while increasing the fidelity of low frequency video components}
        \item Vestigial sideband provides high frequency emphasis to sharpen the picture
    \end{enumerate}
\end{tcolorbox}

\subsubsection*{Intuitive Explanation}
Imagine you are watching an old analog TV show. The TV signal needs to carry a lot of information, like the picture and the sound. To make sure the picture looks good and doesn't take up too much space, engineers use something called vestigial sideband. Think of it like a smart way to pack the information so that the TV can show the picture clearly, especially the parts that are not very detailed (like large areas of the same color), without using too much bandwidth. This helps the TV signal travel more efficiently and still look great on your screen.

\subsubsection*{Advanced Explanation}
In analog fast-scan TV transmissions, vestigial sideband (VSB) modulation is employed to optimize the use of bandwidth. The VSB technique involves transmitting one complete sideband and a portion of the other sideband. This approach reduces the total bandwidth required for transmission while maintaining the fidelity of the low-frequency video components. 

Mathematically, the bandwidth \( B \) of a VSB signal can be expressed as:
\[ B = f_c + f_m - f_v \]
where \( f_c \) is the carrier frequency, \( f_m \) is the maximum frequency of the modulating signal, and \( f_v \) is the frequency of the vestigial sideband. By carefully selecting \( f_v \), the system can achieve a balance between bandwidth efficiency and signal fidelity.

The primary advantage of VSB is its ability to preserve the low-frequency components of the video signal, which are crucial for maintaining image quality. This is particularly important in analog TV, where the low-frequency components correspond to large, uniform areas of the image. By reducing the bandwidth without compromising these components, VSB ensures that the transmitted video remains clear and detailed.

% Diagram Prompt: Generate a diagram showing the frequency spectrum of a vestigial sideband signal, highlighting the carrier frequency, the complete sideband, and the vestigial portion of the other sideband.