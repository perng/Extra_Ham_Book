\subsection{Unlocking Admittance: What Does It Mean?}

\begin{tcolorbox}[colback=gray!10!white,colframe=black!75!black,title=E5B12] What is admittance?  
    \begin{enumerate}[label=\Alph*),noitemsep]
        \item \textbf{The inverse of impedance}
        \item The term for the gain of a field effect transistor
        \item The inverse of reactance
        \item The term for the on-impedance of a field effect transistor
    \end{enumerate}
\end{tcolorbox}

\subsubsection{Intuitive Explanation}
Imagine you are trying to push water through a pipe. The resistance of the pipe to the flow of water is like impedance in an electrical circuit. Now, if you think about how easily the water can flow through the pipe, that's like admittance. Admittance is just a fancy way of saying how easily electricity can flow through a circuit. It's the opposite of impedance—the lower the impedance, the higher the admittance, and the easier it is for electricity to flow.

\subsubsection{Advanced Explanation}
Admittance, denoted by the symbol \( Y \), is a measure of how easily a circuit allows the flow of alternating current (AC). It is defined as the reciprocal of impedance \( Z \), which is the total opposition a circuit offers to the flow of AC. Mathematically, admittance is expressed as:

\[
Y = \frac{1}{Z}
\]

Impedance \( Z \) itself is a complex quantity, combining resistance \( R \) and reactance \( X \), and is given by:

\[
Z = R + jX
\]

where \( j \) is the imaginary unit. Therefore, admittance can also be expressed in terms of conductance \( G \) and susceptance \( B \):

\[
Y = G + jB
\]

Here, \( G \) is the real part of admittance, representing the ease with which current flows through the resistive part of the circuit, and \( B \) is the imaginary part, representing the ease with which current flows through the reactive part (inductors and capacitors).

In summary, admittance is a comprehensive measure of a circuit's ability to conduct AC, and it is directly related to the inverse of impedance.

% [Prompt for diagram: A diagram showing the relationship between impedance, resistance, reactance, admittance, conductance, and susceptance would be helpful here.]