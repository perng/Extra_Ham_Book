\subsection{Transforming Reactance: The Cheerful Shift to Susceptance!}

\begin{tcolorbox}[colback=gray!10!white,colframe=black!75!black,title=E5B05] What is the effect on the magnitude of pure reactance when it is converted to susceptance?
    \begin{enumerate}[label=\Alph*),noitemsep]
        \item It is unchanged
        \item The sign is reversed
        \item It is shifted by 90 degrees
        \item \textbf{It is replaced by its reciprocal}
    \end{enumerate}
\end{tcolorbox}

\subsubsection*{Intuitive Explanation}
Imagine you have a rubber band that represents reactance. When you convert this rubber band into susceptance, it’s like turning the rubber band inside out. The length of the rubber band doesn’t stay the same; instead, it changes to its reciprocal. So, if the rubber band was 2 units long, it becomes 1/2 units long. This is what happens when pure reactance is converted to susceptance—the magnitude becomes its reciprocal.

\subsubsection*{Advanced Explanation}
In electrical engineering, reactance (\(X\)) and susceptance (\(B\)) are related concepts in the analysis of AC circuits. Reactance is the opposition to the change in current due to inductance or capacitance, while susceptance is the reciprocal of reactance and represents the ease with which current can flow through a reactive component.

Mathematically, the relationship between reactance and susceptance is given by:
\[
B = \frac{1}{X}
\]
where:
\begin{itemize}
    \item \(B\) is the susceptance,
    \item \(X\) is the reactance.
\end{itemize}

When converting pure reactance to susceptance, the magnitude of the reactance is replaced by its reciprocal. For example, if the reactance \(X = 4 \, \Omega\), the susceptance \(B\) would be:
\[
B = \frac{1}{4} = 0.25 \, \text{Siemens (S)}
\]

This transformation is crucial in circuit analysis, especially when dealing with parallel circuits, where susceptance simplifies the calculations.

% Prompt for diagram: A diagram showing the relationship between reactance and susceptance, with an example calculation, would be helpful here.