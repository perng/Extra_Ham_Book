\subsection{Inductor Insights: AC Current Meets Voltage!}

\begin{tcolorbox}[colback=gray!10!white,colframe=black!75!black,title=E5B10] What is the relationship between the AC current through an inductor and the voltage across an inductor?
    \begin{enumerate}[label=\Alph*),noitemsep]
        \item \textbf{Voltage leads current by 90 degrees}
        \item Current leads voltage by 90 degrees
        \item Voltage and current are 180 degrees out of phase
        \item Voltage and current are in phase
    \end{enumerate}
\end{tcolorbox}

\subsubsection{Intuitive Explanation}
Imagine you have a water wheel in a stream. The water wheel is like an inductor, and the water flow is like the current. When the water starts to flow, the wheel doesn't immediately start spinning at full speed. It takes a little time to get going. Similarly, when you apply a voltage to an inductor, the current doesn't immediately reach its maximum value. Instead, the voltage pushes the current to start flowing, but the current lags behind. This means the voltage is ahead of the current by 90 degrees, just like the water flow is ahead of the wheel's movement.

\subsubsection{Advanced Explanation}
In an inductor, the relationship between voltage \( V \) and current \( I \) in an AC circuit is governed by the equation:
\[
V = L \frac{dI}{dt}
\]
where \( L \) is the inductance. This equation shows that the voltage across an inductor is proportional to the rate of change of the current. In an AC circuit, the current is sinusoidal, so its rate of change (derivative) is also sinusoidal but shifted by 90 degrees. Specifically, if the current is \( I(t) = I_0 \sin(\omega t) \), then:
\[
\frac{dI}{dt} = I_0 \omega \cos(\omega t) = I_0 \omega \sin\left(\omega t + \frac{\pi}{2}\right)
\]
Thus, the voltage \( V(t) = L I_0 \omega \sin\left(\omega t + \frac{\pi}{2}\right) \) leads the current by 90 degrees. This phase relationship is a fundamental property of inductors in AC circuits.

% Prompt for diagram: A diagram showing a sinusoidal waveform for current and voltage, with the voltage waveform shifted 90 degrees ahead of the current waveform, would be helpful for visualizing the phase relationship.