\subsection{Finding the Fun Frequency of an RLC Circuit!}

\begin{tcolorbox}[colback=gray!10!white,colframe=black!75!black,title=E5A02] What is the resonant frequency of an RLC circuit if R is 22 ohms, L is 50 microhenries, and C is 40 picofarads?
    \begin{enumerate}[label=\Alph*,noitemsep]
        \item 44.72 MHz
        \item 22.36 MHz
        \item \textbf{3.56 MHz}
        \item 1.78 MHz
    \end{enumerate}
\end{tcolorbox}

\subsubsection{Intuitive Explanation}
Imagine you have a swing. If you push the swing at just the right time, it goes higher and higher. This is called resonance. In an RLC circuit, resonance happens when the circuit swings at a special frequency called the resonant frequency. This frequency depends on the size of the inductor (L) and the capacitor (C). The resistor (R) doesn't change the resonant frequency, but it affects how strong the swing is. In this question, we are given the values of L and C, and we need to find the resonant frequency where the circuit swings the most.

\subsubsection{Advanced Explanation}
The resonant frequency \( f_0 \) of an RLC circuit is given by the formula:
\[
f_0 = \frac{1}{2\pi\sqrt{LC}}
\]
where:
\begin{itemize}
    \item \( L \) is the inductance in henries (H),
    \item \( C \) is the capacitance in farads (F).
\end{itemize}

Given:
\begin{itemize}
    \item \( L = 50 \, \mu\text{H} = 50 \times 10^{-6} \, \text{H} \),
    \item \( C = 40 \, \text{pF} = 40 \times 10^{-12} \, \text{F} \).
\end{itemize}

Plugging these values into the formula:
\[
f_0 = \frac{1}{2\pi\sqrt{(50 \times 10^{-6})(40 \times 10^{-12})}}
\]
First, calculate the product \( LC \):
\[
LC = (50 \times 10^{-6})(40 \times 10^{-12}) = 2000 \times 10^{-18} = 2 \times 10^{-15}
\]
Next, take the square root of \( LC \):
\[
\sqrt{LC} = \sqrt{2 \times 10^{-15}} = \sqrt{2} \times 10^{-7.5}
\]
Now, calculate the resonant frequency:
\[
f_0 = \frac{1}{2\pi \times \sqrt{2} \times 10^{-7.5}} \approx \frac{1}{6.28 \times 1.414 \times 10^{-7.5}} \approx \frac{1}{8.88 \times 10^{-7.5}} \approx 1.13 \times 10^{7.5} \, \text{Hz}
\]
Converting to MHz:
\[
f_0 \approx 3.56 \, \text{MHz}
\]
Thus, the correct answer is \textbf{C: 3.56 MHz}.

\subsubsection{Related Concepts}
\begin{itemize}
    \item \textbf{Resonance in RLC Circuits}: Resonance occurs when the inductive reactance \( X_L \) and capacitive reactance \( X_C \) are equal, causing the impedance of the circuit to be minimized.
    \item \textbf{Inductive Reactance}: \( X_L = 2\pi f L \)
    \item \textbf{Capacitive Reactance}: \( X_C = \frac{1}{2\pi f C} \)
    \item \textbf{Impedance}: \( Z = \sqrt{R^2 + (X_L - X_C)^2} \)
\end{itemize}

% Diagram Prompt: Generate a diagram showing an RLC circuit with labeled components (R, L, C) and the resonant frequency formula.