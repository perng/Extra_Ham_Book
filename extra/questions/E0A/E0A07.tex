\subsection{Understanding 100\% Tie-Off: A Key to Tower Safety!}

\begin{tcolorbox}[colback=gray!10!white,colframe=black!75!black,title=E0A07] What is meant by “100\% tie-off” regarding tower safety?
    \begin{enumerate}[label=\Alph*)]
        \item All loose ropes and guys secured to a fixed structure
        \item \textbf{At least one lanyard attached to the tower at all times}
        \item All tools secured to the climber’s harness
        \item All circuit breakers feeding power to the tower must be tied closed with tape, cable, or ties
    \end{enumerate}
\end{tcolorbox}

\subsubsection{Intuitive Explanation}
Imagine you’re climbing a tall tower. To make sure you don’t fall, you need to always have something holding you to the tower. This is called “100\% tie-off.” It means that no matter where you are on the tower, you should always have at least one safety rope (called a lanyard) attached to the tower. This way, if you slip, the lanyard will catch you and keep you safe. It’s like always holding onto a railing when you’re on a high staircase.

\subsubsection{Advanced Explanation}
In the context of tower safety, “100\% tie-off” is a critical safety protocol that ensures a climber is always secured to the tower structure. This is achieved by maintaining at least one lanyard attached to the tower at all times during the climb. A lanyard is a flexible line of rope, wire, or strap that connects the climber’s harness to an anchor point on the tower. 

The primary purpose of this protocol is to minimize the risk of a fall. When a climber is 100\% tied off, they are continuously protected by a fall arrest system. This system is designed to stop a fall within a short distance, thereby reducing the potential for injury. 

Mathematically, the force exerted on the climber during a fall can be calculated using the formula:
\[
F = m \cdot a
\]
where \( F \) is the force, \( m \) is the mass of the climber, and \( a \) is the acceleration due to gravity (approximately \( 9.81 \, \text{m/s}^2 \)). The lanyard and harness system must be able to withstand this force to ensure the climber’s safety.

Additionally, the concept of 100\% tie-off is closely related to the principles of fall protection, which include fall prevention, fall arrest, and fall containment. These principles are essential for ensuring the safety of workers who perform tasks at height.

% Diagram prompt: Generate a diagram showing a climber on a tower with a lanyard attached to an anchor point, illustrating the concept of 100% tie-off.