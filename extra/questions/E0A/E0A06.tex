\subsection{Unraveling the Magic of E and H Limits Below 300 MHz!}

\begin{tcolorbox}[colback=blue!5!white,colframe=blue!75!black]
    \textbf{E0A06} Why are there separate electric (E) and magnetic (H) MPE limits at frequencies below 300 MHz?
    \begin{enumerate}[label=\Alph*)]
        \item The body reacts to electromagnetic radiation from both the E and H fields
        \item Ground reflections and scattering cause the field strength to vary with location
        \item E field and H field radiation intensity peaks can occur at different locations
        \item \textbf{All these choices are correct}
    \end{enumerate}
\end{tcolorbox}

\subsubsection{Intuitive Explanation}
Imagine you are standing near a radio tower that is broadcasting at a frequency below 300 MHz. The tower sends out both electric (E) and magnetic (H) fields. These fields can affect your body in different ways. Sometimes, the ground can bounce these fields around, making them stronger or weaker depending on where you are standing. Also, the strongest points of the electric and magnetic fields might not be in the same place. Because of all these reasons, scientists have set separate safety limits for how much of each type of field you can be exposed to.

\subsubsection{Advanced Explanation}
At frequencies below 300 MHz, the interaction of electromagnetic fields with the human body is complex due to the varying nature of the E and H fields. The body's response to these fields is different because the electric field primarily induces currents in the body, while the magnetic field can induce eddy currents. 

Ground reflections and scattering can cause significant variations in field strength. For instance, the electric field can be enhanced or diminished depending on the reflective properties of the ground and surrounding objects. Similarly, the magnetic field can be influenced by nearby conductive materials.

Moreover, the spatial distribution of the E and H fields can differ. The electric field intensity might peak at a different location compared to the magnetic field intensity due to the wave's phase and interference patterns.

Therefore, separate Maximum Permissible Exposure (MPE) limits are established for the electric and magnetic fields to ensure comprehensive safety. This approach accounts for the distinct ways in which these fields interact with the human body and the environment.

% Diagram Prompt: Generate a diagram showing the spatial distribution of E and H fields around a radio tower, highlighting the peaks and variations due to ground reflections and scattering.