\subsection{Pulse Power: What Can Halve a Frequency?}

\begin{tcolorbox}[colback=gray!10!white,colframe=black!75!black,title=\textbf{E7A03}]
\textbf{Which of the following can divide the frequency of a pulse train by 2?}
\begin{enumerate}[label=\Alph*),noitemsep]
    \item An XOR gate
    \item \textbf{A flip-flop}
    \item An OR gate
    \item A multiplexer
\end{enumerate}
\end{tcolorbox}

\subsubsection{Intuitive Explanation}
Imagine you have a string of light bulbs that blink on and off very quickly. You want to make them blink half as fast. A flip-flop is like a special switch that can take every other blink and ignore it, making the lights blink slower. It’s like skipping every other step when you’re walking to make your steps slower. The other options, like XOR gates, OR gates, and multiplexers, don’t know how to skip steps like a flip-flop does.

\subsubsection{Advanced Explanation}
A flip-flop is a basic memory element in digital circuits that can store one bit of information. When used as a frequency divider, a flip-flop toggles its output state on every rising or falling edge of the input pulse train. This effectively halves the frequency of the input signal. 

For example, consider a T flip-flop (Toggle flip-flop). The output of a T flip-flop toggles its state (from 0 to 1 or from 1 to 0) on each clock pulse. If the input frequency is \( f \), the output frequency will be \( \frac{f}{2} \).

Mathematically, if the input signal has a period \( T \), the output signal will have a period \( 2T \), thus halving the frequency:
\[
f_{\text{out}} = \frac{1}{2T} = \frac{f_{\text{in}}}{2}
\]

Other components like XOR gates, OR gates, and multiplexers do not have the inherent ability to store state or toggle output in a way that would allow them to divide frequency. They are primarily used for logic operations and signal routing, not for frequency division.

% Diagram prompt: Generate a diagram showing a T flip-flop with an input pulse train and the resulting output pulse train with halved frequency.