\subsection{Decade Counter Delight: Unraveling Its Function!}

\begin{tcolorbox}[colback=gray!10!white,colframe=black!75!black,title=\textbf{E7A02}]
\textbf{What is the function of a decade counter?}
\begin{enumerate}[label=\Alph*),noitemsep]
    \item \textbf{It produces one output pulse for every 10 input pulses}
    \item It decodes a decimal number for display on a seven-segment LED display
    \item It produces 10 output pulses for every input pulse
    \item It decodes a binary number for display on a seven-segment LED display
\end{enumerate}
\end{tcolorbox}

\subsubsection{Intuitive Explanation}
Imagine you have a special counter that only counts up to 10. Every time you press a button (which is like an input pulse), the counter adds one to its count. When it reaches 10, it gives you a signal (an output pulse) and then starts counting from zero again. This is what a decade counter does—it counts up to 10 and then gives you a signal to let you know it has reached that number.

\subsubsection{Advanced Explanation}
A decade counter is a type of digital counter that counts in a sequence from 0 to 9 (which is 10 states, hence the name decade). It is typically implemented using flip-flops and logic gates. The counter increments its state by one for each input pulse it receives. When it reaches the count of 9, the next pulse resets it back to 0, and it generates an output pulse. This output pulse can be used to trigger other circuits or indicate that a full cycle of counting has been completed.

Mathematically, a decade counter can be represented as a finite state machine with 10 states. The state transition can be described as follows:

\[
\text{State}_{n+1} = (\text{State}_n + 1) \mod 10
\]

Where \(\text{State}_n\) is the current state of the counter, and \(\text{State}_{n+1}\) is the next state after receiving an input pulse. The modulo operation ensures that the counter resets to 0 after reaching 9.

Decade counters are commonly used in applications where counting in base-10 is required, such as in digital clocks, frequency dividers, and various timing circuits.

% Prompt for generating a diagram:
% Diagram showing a decade counter circuit with input pulses, flip-flops, and an output pulse. The diagram should illustrate the counting sequence from 0 to 9 and the reset to 0 after the 9th pulse.