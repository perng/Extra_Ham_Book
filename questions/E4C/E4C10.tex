\subsection{Maximizing Choices: The Joy of Receiver Bandwidth Variety!}

\begin{tcolorbox}[colback=gray!10, colframe=black, title=E4C10] What is an advantage of having a variety of receiver bandwidths from which to select?
\begin{enumerate}[label=\Alph*.]
    \item The noise figure of the RF amplifier can be adjusted to match the modulation type, thus increasing receiver sensitivity.
    \item Receiver power consumption can be reduced when wider bandwidth is not required.
    \item \textbf{Receive bandwidth can be set to match the modulation bandwidth, maximizing signal-to-noise ratio and minimizing interference.}
    \item Multiple frequencies can be received simultaneously if desired.
\end{enumerate} \end{tcolorbox}

\subsubsection*{Related Concepts}

When discussing receiver bandwidths in radio communication, it is essential to understand how bandwidth affects signal reception and overall system performance. The bandwidth of a receiver refers to the range of frequencies it can process, while the modulation bandwidth is the range of frequencies occupied by the signal that is transmitted.

1. \textbf{Signal-to-Noise Ratio (SNR)}: This is a critical factor in determining the quality of signal reception. A higher SNR indicates a cleaner signal with less noise. By matching the receiver's bandwidth to the modulation bandwidth of the signal, we can filter out noise that is outside this range, thus improving the SNR.

2. \textbf{Interference}: In many communication scenarios, multiple signals may overlap in frequency. If the receiver's bandwidth is too wide, it can pick up unwanted signals (or interference) along with the desired signal. By selecting a narrower bandwidth that matches the modulation of the desired signal, the receiver minimizes the chance of such interference affecting the quality of reception.

3. \textbf{Power Consumption}: The receiver’s power consumption can also depend on the bandwidth. Wider bandwidths tend to consume more power, as they need to process a larger range of frequencies. If the receiver can switch to narrower bandwidths when wider ones are not necessary, overall power efficiency can be improved.

4. \textbf{Flexibility of Receivers}: Having a range of selectable bandwidths provides flexibility. Depending on the situation, a user can choose the appropriate bandwidth to optimize for the best performance based on the received signal characteristics and the environment.

Ultimately, \textbf{the correct answer (C)} emphasizes the importance of tailoring the receiver bandwidth to match the modulation bandwidth, which results in maximizing the SNR while minimizing potential interference.

\subsubsection*{Example Calculation}

If a signal is modulated with a bandwidth of 10 kHz, and the receiver operates with a bandwidth of 15 kHz, the resulting SNR might be decreased due to interference. If we choose a receiver bandwidth of 10 kHz instead, we can filter out everything outside of the modulation range.

Let's consider a simple scenario:

- \textbf{Input Signal Power (P\_signal)}: 10 µW
- \textbf{Noise Power (P\_noise)}: 2 µW

The Signal-to-Noise Ratio (SNR) can be calculated using the formula:
\[
SNR = \frac{P\_signal}{P\_noise}
\]
For the wider bandwidth scenario:
\[
SNR_{wide} = \frac{10 \, \mu W}{2 \, \mu W} = 5
\]

For the narrow bandwidth scenario, assuming we reduce noise to 1 µW by filtering:
\[
SNR_{narrow} = \frac{10 \, \mu W}{1 \, \mu W} = 10
\]

Thus, by selecting an appropriate bandwidth, we can effectively double the SNR from 5 to 10.

% \subsubsection*{Diagram Representation}

% \begin{center}
% \begin{tikzpicture}
%     \draw[->] (0,0) -- (5,0) node[anchor=north] {Frequency};
%     \draw[pattern=checkerboard, pattern color=blue] (1,0.5) rectangle (3,-0.5);
%     \draw[pattern=checkerboard, pattern color=red] (1.5,-0.5) rectangle (2.5,-1.5);
%     \node at (2,-0.25) {Signal Bandwidth};
%     \node at (2,-1) {Noise Bandwidth};
%     \draw[dashed] (1, 0) -- (1, 1.5);
%     \draw[dashed] (3, 0) -- (3, 1.5);
%     \node at (0.5, 1) {Narrow Receiver Bandwidth};
%     \node at (4, 1) {Wider Receiver Bandwidth};
% \end{tikzpicture}
% \end{center}

% This diagram showcases the relationship between signal bandwidths and the receiver bandwidths, highlighting how selective bandwidths can help mitigate noise while receiving the desired signal.
