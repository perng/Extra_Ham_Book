\subsection{Zesty Circuit Secrets: Arranging a Pi-Network Low-Pass Filter!}

\begin{tcolorbox}[colback=gray!10!white,colframe=black!75!black,title=E7C01] How are the capacitors and inductors of a low-pass filter Pi-network arranged between the network’s input and output?
    \begin{enumerate}[label=\Alph*]
        \item Two inductors are in series between the input and output, and a capacitor is connected between the two inductors and ground
        \item Two capacitors are in series between the input and output, and an inductor is connected between the two capacitors and ground
        \item An inductor is connected between the input and ground, another inductor is connected between the output and ground, and a capacitor is connected between the input and output
        \item \textbf{A capacitor is connected between the input and ground, another capacitor is connected between the output and ground, and an inductor is connected between the input and output}
    \end{enumerate}
\end{tcolorbox}

\subsubsection{Intuitive Explanation}
Imagine you have a water pipe system where water represents the electrical signal. A low-pass filter is like a system that allows only slow-moving water (low-frequency signals) to pass through while blocking fast-moving water (high-frequency signals). In a Pi-network low-pass filter, think of the capacitors as small tanks that store water (electrical energy) and the inductor as a long, narrow pipe that slows down the water flow. The capacitors are placed at the input and output to ground, acting like small reservoirs that absorb and release water slowly. The inductor is placed directly between the input and output, acting like a long pipe that slows down the water flow. This arrangement ensures that only slow-moving water (low-frequency signals) can pass through the system.

\subsubsection{Advanced Explanation}
A Pi-network low-pass filter is a type of filter that allows low-frequency signals to pass through while attenuating high-frequency signals. The filter is named Pi because its circuit diagram resembles the Greek letter Pi ($\pi$). The arrangement of components in a Pi-network low-pass filter is as follows:

1. A capacitor is connected between the input and ground.
2. Another capacitor is connected between the output and ground.
3. An inductor is connected between the input and output.

Mathematically, the impedance of a capacitor is given by:
\[
Z_C = \frac{1}{j\omega C}
\]
where $C$ is the capacitance and $\omega$ is the angular frequency. The impedance of an inductor is given by:
\[
Z_L = j\omega L
\]
where $L$ is the inductance. For low frequencies ($\omega$ is small), the impedance of the capacitors is high, and the impedance of the inductor is low. This allows low-frequency signals to pass through the inductor with minimal attenuation. For high frequencies ($\omega$ is large), the impedance of the capacitors is low, and the impedance of the inductor is high. This causes high-frequency signals to be shunted to ground through the capacitors, effectively blocking them from reaching the output.

The correct arrangement of components in a Pi-network low-pass filter is option D: A capacitor is connected between the input and ground, another capacitor is connected between the output and ground, and an inductor is connected between the input and output.

% Diagram Prompt: Generate a diagram showing a Pi-network low-pass filter with capacitors connected between input and ground, output and ground, and an inductor connected between input and output.