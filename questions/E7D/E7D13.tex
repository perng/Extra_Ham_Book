\subsection{Powering Up: How to Calculate Power Dissipation in Voltage Regulators!}

\begin{tcolorbox}[colback=gray!10!white,colframe=black!75!black,title=E7D13] Which of the following calculates power dissipated by a series linear voltage regulator?
    \begin{enumerate}[label=\Alph*)]
        \item Input voltage multiplied by input current
        \item Input voltage divided by output current
        \item \textbf{Voltage difference from input to output multiplied by output current}
        \item Output voltage multiplied by output current
    \end{enumerate}
\end{tcolorbox}

\subsubsection*{Intuitive Explanation}
Imagine you have a water pipe with a valve that controls how much water flows through it. The valve is like a voltage regulator, which controls the voltage (or pressure) of electricity. The power dissipated by the regulator is like the energy lost as heat when the valve reduces the water pressure. To find this energy loss, you need to know the difference in pressure (voltage) before and after the valve and how much water (current) is flowing through it. Multiplying these two gives you the power dissipated.

\subsubsection*{Advanced Explanation}
A series linear voltage regulator dissipates power as heat due to the voltage drop across it. The power dissipation \( P \) can be calculated using the formula:
\[
P = (V_{\text{in}} - V_{\text{out}}) \times I_{\text{out}}
\]
where:
\begin{itemize}
    \item \( V_{\text{in}} \) is the input voltage,
    \item \( V_{\text{out}} \) is the output voltage,
    \item \( I_{\text{out}} \) is the output current.
\end{itemize}

This formula arises from the fact that the regulator must handle the voltage difference \( (V_{\text{in}} - V_{\text{out}}) \) while allowing the output current \( I_{\text{out}} \) to flow. The product of these two quantities gives the power dissipated as heat.

For example, if \( V_{\text{in}} = 12 \, \text{V} \), \( V_{\text{out}} = 5 \, \text{V} \), and \( I_{\text{out}} = 1 \, \text{A} \), the power dissipated would be:
\[
P = (12 - 5) \times 1 = 7 \, \text{W}
\]

This power dissipation is crucial for designing heat sinks and ensuring the regulator operates within its thermal limits.

% Diagram Prompt: A diagram showing a series linear voltage regulator with input voltage, output voltage, and output current labeled, along with the power dissipation formula.