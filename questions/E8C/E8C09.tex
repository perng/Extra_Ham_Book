\subsection{One Bit Wonder: The Magic of Gray Codes!}

\begin{tcolorbox}[colback=blue!5!white, colframe=blue!75!black, title=Question E8C09]
    \textbf{Question ID: E8C09} \\
    Which digital code allows only one bit to change between sequential code values?
    \begin{enumerate}[label=\Alph*.]
        \item Binary Coded Decimal Code
        \item Extended Binary Coded Decimal Interchange Code
        \item Extended ASCII
        \item \textbf{Gray code}
    \end{enumerate}
\end{tcolorbox}

\subsubsection{Intuitive Explanation}
Imagine you have a room full of light switches that can be either on or off. If you flip one switch at a time to represent different numbers, you want to change the least amount of switches to go from one number to the next. Gray code is like a special system that helps you do just that! In Gray code, when you go from one number to the next, only one light switch changes at a time. This is really useful because it helps prevent mistakes, especially in machines and electronics.

\subsubsection{Advanced Explanation}
To understand Gray code, we first need to explore binary representation. In binary, every number is represented as a combination of bits (0s and 1s). For example, the binary representation of the decimal number 3 is 11, and for 4, it is 100. 

In standard binary counting, moving from one number to the next can cause multiple bits to change. For instance, when going from 3 (11) to 4 (100), we see that two bits change. Gray code, however, was designed so that only a single bit changes with each incremental value. This property makes Gray code very useful in error correction in digital communications and in rotary encoders.

To generate the Gray code for a 3-bit binary number, you can use the following formula:
1. The most significant bit (MSB) of the Gray code is the same as the MSB of the binary number.
2. Each subsequent bit of the Gray code can be computed as: \( G_i = B_i \oplus B_{i-1} \) where \( G \) is the Gray code bit, \( B \) is the binary bit, and \( \oplus \) is the XOR operation.

For example, converting the binary representation of 0, 1, 2, and 3 to Gray code:
- Binary 0: 000 → Gray 0: 000
- Binary 1: 001 → Gray 1: 001 (0 \oplus 0 = 0, 0 \oplus 1 = 1)
- Binary 2: 010 → Gray 2: 011 (0 \oplus 0 = 0, 1 \oplus 0 = 1)
- Binary 3: 011 → Gray 3: 010 (0 \oplus 0 = 0, 1 \oplus 1 = 0)

In this way, the transition from binary to Gray code ensures that only one bit changes for every step.

% Diagram prompt: Generate a diagram showing the transition of binary values to Gray code, highlighting the single bits that change at each step.