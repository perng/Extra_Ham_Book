\subsection{Building Your Antenna: Fun Rules Near Airports!}

\begin{tcolorbox}[colback=gray!10!white,colframe=black!75!black,title=\textbf{E1B06}]
\textbf{Which of the following additional rules apply if you are erecting an amateur station antenna structure at a site at or near a public use airport?}
\begin{enumerate}[label=\Alph*.]
    \item \textbf{You may have to notify the Federal Aviation Administration and register it with the FCC as required by Part 17 of the FCC rules}
    \item You may have to enter the height above ground in meters, and the latitude and longitude in degrees, minutes, and seconds on the FAA website
    \item You must file an Environmental Impact Statement with the EPA before construction begins
    \item You must obtain a construction permit from the airport zoning authority per Part 119 of the FAA regulations
\end{enumerate}
\end{tcolorbox}

\subsubsection{Intuitive Explanation}
Imagine you’re building a tall tower for your ham radio antenna near an airport. Airplanes fly high in the sky, and your tower could be in their way! To keep everyone safe, there are special rules you need to follow. One of these rules is that you might have to tell the Federal Aviation Administration (FAA) about your tower and also register it with the Federal Communications Commission (FCC). This way, pilots and air traffic controllers know where your tower is and can avoid it. It’s like putting up a big sign that says, “Hey, there’s a tower here!” so no one accidentally bumps into it.

\subsubsection{Advanced Explanation}
When erecting an amateur station antenna structure near a public use airport, compliance with Part 17 of the FCC rules is mandatory. Part 17 outlines the requirements for antenna structures that could pose a hazard to air navigation. Specifically, you must notify the Federal Aviation Administration (FAA) and register the structure with the FCC. This ensures that the antenna’s height and location are documented in the national database, allowing pilots and air traffic controllers to be aware of potential obstructions.

The process involves submitting detailed information about the antenna structure, including its height, geographic coordinates, and any lighting or marking requirements. The FAA evaluates this information to determine if the structure poses a hazard to air navigation. If it does, additional measures, such as installing obstruction lights, may be required.

This regulation is crucial for maintaining aviation safety, as unregistered or improperly marked structures could lead to accidents. The FCC and FAA work together to ensure that all antenna structures near airports are properly documented and marked, minimizing the risk to air traffic.

% Prompt for generating a diagram: 
% Diagram showing an antenna structure near an airport with FAA and FCC registration process flow.