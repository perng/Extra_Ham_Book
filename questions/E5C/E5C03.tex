\subsection{Polar Perplexities: Unraveling Pure Inductive Reactance!}

\begin{tcolorbox}[colback=gray!10!white,colframe=black!75!black,title=E5C03] Which of the following represents a pure inductive reactance in polar coordinates?
    \begin{enumerate}[label=\Alph*)]
        \item A positive 45 degree phase angle
        \item A negative 45 degree phase angle
        \item \textbf{A positive 90 degree phase angle}
        \item A negative 90 degree phase angle
    \end{enumerate}
\end{tcolorbox}

\subsubsection{Intuitive Explanation}
Imagine you have a coil of wire, and you pass an electric current through it. This coil creates a magnetic field, which resists changes in the current. This resistance to changes is called inductive reactance. In polar coordinates, which are like a map for angles and distances, pure inductive reactance is represented by a specific angle. Think of it like a compass pointing directly north, but instead of north, it points at a 90-degree angle. This means the current is lagging behind the voltage by 90 degrees, which is a key characteristic of pure inductive reactance.

\subsubsection{Advanced Explanation}
In electrical engineering, inductive reactance (\(X_L\)) is the opposition that an inductor presents to alternating current (AC). It is given by the formula:
\[
X_L = \omega L
\]
where \(\omega\) is the angular frequency of the AC signal and \(L\) is the inductance of the coil.

In polar coordinates, impedance (\(Z\)) is represented as:
\[
Z = R + jX
\]
where \(R\) is the resistance, \(X\) is the reactance, and \(j\) is the imaginary unit. For a pure inductor, the resistance \(R\) is zero, so the impedance simplifies to:
\[
Z = jX_L
\]
This means the phase angle (\(\theta\)) of the impedance is:
\[
\theta = \arctan\left(\frac{X_L}{R}\right) = \arctan\left(\frac{X_L}{0}\right) = 90^\circ
\]
Thus, pure inductive reactance is represented by a positive 90-degree phase angle in polar coordinates.

% [Prompt for diagram: A polar plot showing the impedance vector at a 90-degree angle to represent pure inductive reactance.]