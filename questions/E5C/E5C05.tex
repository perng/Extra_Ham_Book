\subsection{Decoding Impedance: The Phase Diagram Delight!}

\begin{tcolorbox}[colback=gray!10!white,colframe=black!75!black,title=Multiple Choice Question]
    \textbf{E5C05} What kind of diagram is used to show the phase relationship between impedances at a given frequency?
    \begin{enumerate}[label=\Alph*)]
        \item Venn diagram
        \item Near field diagram
        \item \textbf{Phasor diagram}
        \item Far field diagram
    \end{enumerate}
\end{tcolorbox}

\subsubsection{Intuitive Explanation}
Imagine you are trying to understand how two dancers are moving together in a dance. One dancer might be a little ahead or behind the other, and you want to see how their movements are related. In the world of electricity, we have something similar called impedance, which is like the resistance to the flow of electricity. When we want to see how different impedances are related to each other in terms of their timing (or phase), we use a special kind of picture called a phasor diagram. This diagram helps us see how the impedances are moving together, just like watching the dancers.

\subsubsection{Advanced Explanation}
A phasor diagram is a graphical representation used to show the phase relationship between different impedances at a given frequency. Impedance, denoted as \( Z \), is a complex quantity that includes both resistance \( R \) and reactance \( X \), and can be expressed as:
\[
Z = R + jX
\]
where \( j \) is the imaginary unit. In a phasor diagram, each impedance is represented as a vector in the complex plane. The length of the vector corresponds to the magnitude of the impedance, and the angle it makes with the real axis represents the phase angle.

For example, consider two impedances \( Z_1 \) and \( Z_2 \) at a frequency \( f \). The phasor diagram would show \( Z_1 \) and \( Z_2 \) as vectors, and the angle between them would indicate the phase difference. This is particularly useful in analyzing AC circuits, where the phase relationship between voltage and current is crucial.

The correct answer is \textbf{C: Phasor diagram}, as it is specifically designed to illustrate the phase relationships between impedances.

% Prompt for generating a diagram: A phasor diagram showing two impedances \( Z_1 \) and \( Z_2 \) with their respective magnitudes and phase angles in the complex plane.