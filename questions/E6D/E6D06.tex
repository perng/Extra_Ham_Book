\subsection{Unlocking Inductance: The Magic of Core Materials!}

\begin{tcolorbox}[colback=gray!10!white,colframe=black!75!black,title=E6D06]
\textbf{E6D06} What core material property determines the inductance of an inductor?
\begin{enumerate}[label=\Alph*)]
    \item Permittivity
    \item Resistance
    \item Reactivity
    \item \textbf{Permeability}
\end{enumerate}
\end{tcolorbox}

\subsubsection{Intuitive Explanation}
Imagine you have a coil of wire, like a spring. When you pass electricity through it, it creates a magnetic field around it. Now, if you put a material inside the coil, like iron, it can make the magnetic field stronger. The property of the material that decides how much it can strengthen the magnetic field is called \textit{permeability}. The higher the permeability, the stronger the magnetic field, and the more inductance the coil will have. So, permeability is the key property that determines the inductance of an inductor.

\subsubsection{Advanced Explanation}
The inductance \( L \) of an inductor is directly influenced by the core material's permeability \( \mu \). The relationship is given by the formula:

\[
L = \frac{\mu N^2 A}{l}
\]

where:
\begin{itemize}
    \item \( \mu \) is the permeability of the core material,
    \item \( N \) is the number of turns in the coil,
    \item \( A \) is the cross-sectional area of the coil,
    \item \( l \) is the length of the coil.
\end{itemize}

Permeability \( \mu \) is a measure of how easily a material can support the formation of a magnetic field within itself. Materials with high permeability, such as iron or ferrite, significantly increase the inductance of the coil compared to air or non-magnetic materials. Permittivity, resistance, and reactivity are not directly related to the inductance of an inductor. Permittivity is related to the ability of a material to store electrical energy in an electric field, resistance is a measure of opposition to current flow, and reactivity is not a standard term in this context.

% Prompt for generating a diagram: A diagram showing a coil with different core materials (air, iron, ferrite) and the corresponding magnetic field strength would be helpful here.