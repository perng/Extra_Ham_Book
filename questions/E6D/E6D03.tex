\subsection{Discover the Wonders of the Piezoelectric Effect!}

\begin{tcolorbox}[colback=gray!10!white,colframe=black!75!black,title=E6D03] Which of the following is an aspect of the piezoelectric effect?
    \begin{enumerate}[label=\Alph*.]
        \item \textbf{Mechanical deformation of material due to the application of a voltage}
        \item Mechanical deformation of material due to the application of a magnetic field
        \item Generation of electrical energy in the presence of light
        \item Increased conductivity in the presence of light
    \end{enumerate}
\end{tcolorbox}

\subsubsection{Intuitive Explanation}
Imagine you have a special material, like a crystal or ceramic, that can change its shape when you apply electricity to it. This is called the piezoelectric effect. It’s like magic! When you give it a little electric push, it bends or squeezes. This is useful in things like speakers, where the material vibrates to make sound. So, the correct answer is the one that says the material changes shape because of electricity.

\subsubsection{Advanced Explanation}
The piezoelectric effect is a phenomenon where certain materials generate mechanical deformation (such as strain or stress) when an electric voltage is applied. This effect is reversible, meaning that applying mechanical stress to the material can also generate an electric voltage. The materials that exhibit this property are called piezoelectric materials, and they include certain crystals (like quartz) and ceramics.

Mathematically, the piezoelectric effect can be described by the relationship between the applied electric field \( E \) and the resulting mechanical strain \( S \):
\[
S = d \cdot E
\]
where \( d \) is the piezoelectric coefficient, a material-specific constant that quantifies the material's ability to convert electrical energy into mechanical energy.

In the context of the question, the correct answer is (A) because it correctly describes the piezoelectric effect as the mechanical deformation of a material due to the application of a voltage. The other options describe different phenomena: (B) refers to magnetostriction, (C) refers to the photovoltaic effect, and (D) refers to photoconductivity.

% Prompt for diagram: A diagram showing a piezoelectric material with an applied voltage causing mechanical deformation would be helpful here.