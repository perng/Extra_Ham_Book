\subsection{Exploring FCC's Strictest RF Limits for Human Safety!}

\begin{tcolorbox}[colback=gray!10!white,colframe=black!75!black,title=E0A03]
\textbf{E0A03} Over what range of frequencies are the FCC human body RF exposure limits most restrictive?
\begin{enumerate}[label=\Alph*)]
    \item 300 kHz - 3 MHz
    \item 3 - 30 MHz
    \item \textbf{30 - 300 MHz}
    \item 300 - 3000 MHz
\end{enumerate}
\end{tcolorbox}

\subsubsection{Intuitive Explanation}
Imagine you are playing with a radio that can tune into different stations. Some stations are closer together, and some are farther apart. The FCC (Federal Communications Commission) has rules to make sure that the radio waves don't harm people. These rules are especially strict for certain ranges of frequencies, like the ones between 30 and 300 MHz. Think of it like a speed limit on a road—some roads have lower speed limits to keep everyone safe, and this range of frequencies has stricter rules to protect us from too much radio wave exposure.

\subsubsection{Advanced Explanation}
The FCC sets specific limits on the amount of radio frequency (RF) energy that can safely be absorbed by the human body, known as the Specific Absorption Rate (SAR). These limits are most restrictive in the frequency range of 30 to 300 MHz. This is because the human body is more efficient at absorbing RF energy in this range, leading to a higher SAR. 

The SAR is calculated using the following formula:

\[
\text{SAR} = \frac{\sigma |E|^2}{\rho}
\]

where:
\begin{itemize}
    \item \(\sigma\) is the conductivity of the tissue,
    \item \(|E|^2\) is the square of the electric field strength,
    \item \(\rho\) is the mass density of the tissue.
\end{itemize}

In the 30 to 300 MHz range, the wavelength of the RF energy is such that it can penetrate the body more effectively, leading to higher absorption rates. This is why the FCC imposes stricter limits in this frequency range to ensure public safety.

% [Prompt for generating a diagram: A diagram showing the frequency spectrum with the 30-300 MHz range highlighted, along with a graph showing SAR levels across different frequency ranges.]