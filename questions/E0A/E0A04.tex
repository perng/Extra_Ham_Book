\subsection{Teamwork in Transmission: Who's Responsible for Safety?}

\begin{tcolorbox}[colback=gray!10!white,colframe=black!75!black,title=\textbf{E0A04}]
When evaluating a site with multiple transmitters operating at the same time, the operators and licensees of which transmitters are responsible for mitigating over-exposure situations?
\begin{enumerate}[label=\Alph*.]
    \item Each transmitter that produces 20 percent or more of its MPE limit in areas where the total MPE limit is exceeded
    \item Each transmitter operating with a duty cycle greater than 25 percent
    \item \textbf{Each transmitter that produces 5 percent or more of its MPE limit in areas where the total MPE limit is exceeded}
    \item Each transmitter operating with a duty cycle greater than 50 percent
\end{enumerate}
\end{tcolorbox}

\subsubsection{Intuitive Explanation}
Imagine you and your friends are all playing with flashlights in a dark room. If everyone shines their flashlight in the same spot, it might get too bright and hurt your eyes. Now, think of the flashlights as transmitters and the brightness as the Maximum Permissible Exposure (MPE) limit. If any flashlight (transmitter) is making the spot more than 5\% brighter than it should be, the person holding that flashlight (the operator) needs to turn it down a bit to keep everyone safe. This way, no one gets hurt by too much light (radiation).

\subsubsection{Advanced Explanation}
When multiple transmitters operate simultaneously, the combined electromagnetic field can exceed the Maximum Permissible Exposure (MPE) limits, which are set to ensure safety from harmful radiation. According to regulatory standards, each transmitter contributing 5\% or more of its MPE limit in areas where the total MPE limit is exceeded must take responsibility for mitigating over-exposure. This ensures that no single transmitter disproportionately contributes to the overall radiation levels, thereby maintaining a safe environment.

To calculate the contribution of each transmitter, the following steps are typically followed:
\begin{enumerate}
    \item Measure the electromagnetic field strength produced by each transmitter.
    \item Determine the percentage of the MPE limit that each transmitter contributes.
    \item Identify transmitters contributing 5\% or more of their MPE limit in areas where the total MPE limit is exceeded.
    \item Implement mitigation measures for these transmitters to reduce their contribution.
\end{enumerate}

This approach ensures that all operators and licensees are collectively responsible for maintaining safe radiation levels, rather than placing the burden solely on high-power transmitters.

% Prompt for generating a diagram:
% Create a diagram showing multiple transmitters emitting signals in a shared space, with labels indicating the contribution of each transmitter to the total MPE limit. Highlight the transmitters contributing 5% or more of their MPE limit.