\subsection{Unlocking the Joy of FM Signal: What's the Deviation Ratio?}

\begin{tcolorbox}
\textbf{Question ID: E8B06} \\
What is the deviation ratio of an FM phone signal having a maximum frequency swing of plus or minus 7.5 kHz if the highest modulation frequency is 3.5 kHz? 
\begin{enumerate}[label=\Alph*.]
    \item \textbf{2.14}
    \item 0.214
    \item 0.47
    \item 47
\end{enumerate}
\end{tcolorbox}

\subsubsection{Intuitive Explanation}
Imagine you are on a swing at the park. When you push yourself away from the center, that's like how much a radio signal can change when it carries information. This change is called the frequency swing, and in this case, it's like swinging really high up and down by 7.5 kHz. Now, we also need to think about how fast we can push back and forth – this speed is what we call the modulation frequency, which is 3.5 kHz. 

To find out how much our swing stretches compared to how fast we are swinging, we do a simple division. If we find that our swing stretches out about 2.14 times in relation to our swinging speed, that's our deviation ratio!

\subsubsection{Advanced Explanation}
The deviation ratio, also known as the modulation index in Frequency Modulation (FM), is defined as the ratio of the frequency deviation of the carrier signal to the highest frequency of the modulating signal. This is formulated as:

\[
\text{Deviation Ratio (m)} = \frac{\Delta f}{f_m}
\]

Where:
- \(\Delta f\) is the maximum frequency deviation (in this case, \(7.5 \text{ kHz}\)).
- \(f_m\) is the maximum frequency of modulation (in this case, \(3.5 \text{ kHz}\)).

Substituting the given values into the formula, we have:

\[
m = \frac{7.5 \text{ kHz}}{3.5 \text{ kHz}} = \frac{7.5}{3.5}
\]

To simplify the calculation:

1. Calculate \( \frac{7.5}{3.5} \):
   - Dividing both numbers by 3.5 gives:
   \[
   m = \frac{7.5 \div 3.5}{3.5 \div 3.5} = \frac{2.14285714285714}{1} \approx 2.14
   \]

Thus, the deviation ratio is approximately \(2.14\).

This modulation index is critical in understanding how well the FM signal can transmit information. A higher deviation ratio implies better resistance to noise, meaning that the information sent through the signal can be received more clearly.

% Generate diagram demonstrating frequency deviation and modulation frequency.