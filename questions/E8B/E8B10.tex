\subsection{Discovering Frequency Division Multiplexing (FDM)!}

\begin{tcolorbox}
    \textbf{Question ID: E8B10} \\[1em]
    What is frequency division multiplexing (FDM)? 
    \begin{enumerate}[label=\Alph*.]
        \item The transmitted signal jumps from band to band at a predetermined rate
        \item \textbf{Dividing the transmitted signal into separate frequency bands that each carry a different data stream}
        \item The transmitted signal is divided into packets of information
        \item Two or more information streams are merged into a digital combiner, which then pulse position modulates the transmitter
    \end{enumerate}
\end{tcolorbox}

\subsubsection{Intuitive Explanation}
Frequency Division Multiplexing (FDM) is a way to send different pieces of information over the same wire or airwaves at the same time. Imagine you have a busy road where many cars want to travel simultaneously. Instead of waiting for each car to pass one at a time, we can divide the road into different lanes (like frequency bands) for each car to travel in. This allows many cars (or streams of information) to share the same road without getting mixed up. So, each piece of information travels in its own lane, and they can all move at the same time!

\subsubsection{Advanced Explanation}
Frequency Division Multiplexing (FDM) is a technique used to enable multiple signals to be transmitted simultaneously over a single communication channel by dividing the total bandwidth into distinct, non-overlapping frequency bands. Each of these bands is dedicated to a specific signal or data stream.

In more technical terms, the process involves:
1. **Bandwidth Allocation**: The total bandwidth available for transmission is divided into several smaller frequency ranges. Each range is assigned to a different signal.
2. **Modulation**: Each signal is modulated to fit into its assigned frequency band. This ensures that the signals do not interfere with each other.
3. **Transmission**: The modulated signals are then combined and transmitted over the medium, which can be a coaxial cable, fiber optics, or free space (as in the case of radio waves).

Mathematically, if we denote the total bandwidth as \( B \), and we divide it into \( N \) frequency bands, the bandwidth allocated for each individual channel can be expressed as:
\[
B_{channel} = \frac{B}{N}
\]
Where \( B_{channel} \) is the bandwidth allocated for each data stream.

To further illustrate, consider that if we have a total bandwidth of \( 100 \) MHz and we want to send \( 4 \) different signals, each signal would be allocated \( 25 \) MHz.

% Diagram prompt: Create a diagram showcasing a communication channel divided into several frequency bands, each carrying a different signal.