\subsection{Calculating the Deviation Delight: FM Signal Insights!}

\begin{tcolorbox}
\textbf{Question ID: E8B05} \\
What is the deviation ratio of an FM phone signal having a maximum frequency swing of plus or minus 5 kHz if the highest modulation frequency is 3 kHz? \\

\begin{enumerate}[label=\Alph*.]
    \item 6
    \item 0.167
    \item 0.6
    \item \textbf{1.67}
\end{enumerate}
\end{tcolorbox}

\subsubsection{Intuitive Explanation}
Imagine you are at a playground, and you are swinging back and forth. The maximum frequency swing is like how far you can go from the center of the swing - in this case, plus or minus 5 kHz means you can swing up to 5 kHz in both directions from your starting point. The highest modulation frequency is like how fast you can swing back and forth - here, it’s 3 kHz, which means you can change your position or swing 3 times in one second.

The deviation ratio helps us understand how much more you can swing compared to how fast you're swinging back and forth. If you can swing 5 kHz and you're swinging back and forth 3 times in a second, you can find out the ratio by dividing your maximum swing by the speed of swing.

\subsubsection{Advanced Explanation}
To compute the deviation ratio in frequency modulation (FM), we use the formula:

\[
\text{Deviation Ratio} = \frac{\Delta f}{f_m}
\]

where:
- \(\Delta f\) is the maximum frequency deviation (5 kHz in this case),
- \(f_m\) is the highest modulation frequency (3 kHz here).

Substituting the values into the formula:

\[
\text{Deviation Ratio} = \frac{5 \text{ kHz}}{3 \text{ kHz}} = 1.67
\]

Thus, the deviation ratio for the given FM signal is 1.67.

The deviation ratio quantifies the relationship between the frequency deviation of the carrier signal and the modulation frequency. A higher deviation ratio can indicate a clearer and more robust signal, which is crucial in communications.

% Prompt for generating a diagram of an FM signal with its frequency deviation and highest modulation frequency for visualization purposes.