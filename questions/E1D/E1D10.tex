\subsection{Who Can Be the Space Commanders?}

\begin{tcolorbox}
\textbf{E1D10} Which amateur stations are eligible to be telecommand stations of space stations, subject to the privileges of the class of operator license held by the control operator of the station?

\begin{enumerate}[label=\Alph*]
    \item Any amateur station approved by AMSAT
    \item \textbf{Any amateur station so designated by the space station licensee}
    \item Any amateur station so designated by the ITU
    \item All these choices are correct
\end{enumerate}
\end{tcolorbox}

\subsubsection{Intuitive Explanation}
Imagine you have a toy spaceship, and you want to control it using a remote control. Not just anyone can control the spaceship; only the person who owns the spaceship or someone they choose can be the space commander. Similarly, in the real world, only the person or group that owns the space station can decide which amateur radio stations are allowed to send commands to it. This ensures that only trusted and authorized people can control the space station.

\subsubsection{Advanced Explanation}
In the context of amateur radio and space stations, the term telecommand station refers to a station that sends commands to a space station. The eligibility of an amateur station to act as a telecommand station is determined by the space station licensee. This means that the entity or individual who owns or operates the space station has the authority to designate which amateur stations can send commands to it. This designation is subject to the privileges granted by the operator's license class, ensuring that only qualified operators can control the space station.

The correct answer is \textbf{B}, as it aligns with the regulatory framework that grants the space station licensee the authority to designate telecommand stations. This ensures proper control and operation of the space station, adhering to international and national regulations.

% Prompt for generating a diagram:
% Diagram showing a space station with an amateur radio station sending commands, labeled as Telecommand Station, and the space station licensee authorizing the station.