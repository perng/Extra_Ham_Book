\subsection{Finding the Perfect Match: Q-Section for 100-Ohm to 50-Ohm!}

\begin{tcolorbox}[colback=gray!10, colframe=black, title=E9E06] 
Which of these transmission line impedances would be suitable for constructing a quarter-wave Q-section for matching a 100-ohm feed point impedance to a 50-ohm transmission line?

\begin{enumerate}[label=\Alph*.]
    \item 50 ohms
    \item 62 ohms
    \item \textbf{75 ohms}
    \item 90 ohms
\end{enumerate} \end{tcolorbox}

\subsubsection{Concepts Related to the Question}
To understand how to choose the appropriate transmission line impedance for a quarter-wave Q-section, we need to look into the matching principles in transmission line theory. A quarter-wave transformer is used to match two different impedances at a particular frequency, utilizing the principle of impedance transformation.

The transformation of impedance through a quarter-wave (\(\lambda/4\)) transmission line is given by the formula:

\[
Z_L = \frac{Z_0^2}{Z_{in}}
\]

where \(Z_L\) is the load impedance, \(Z_0\) is the characteristic impedance of the quarter-wave transformer, and \(Z_{in}\) is the input impedance seen looking into the transformer from the transmission line.

Given:
- \(Z_{in} = 100 \, \Omega\) (the feed point impedance)
- \(Z_L = 50 \, \Omega\) (the impedance of the transmission line)

Rearranging the impedance transformation formula, we have:

\[
Z_0 = \sqrt{Z_L \cdot Z_{in}} = \sqrt{50 \, \Omega \cdot 100 \, \Omega}
\]

Calculating this step by step:

1. First calculate the product:
   \[
   50 \times 100 = 5000
   \]
   
2. Then take the square root:
   \[
   Z_0 = \sqrt{5000} \approx 70.71 \, \Omega
   \]

The closest standard impedance values are typically 75 ohms and 50 ohms. However, 75 ohms is notably closer to the calculated value of 70.71 ohms, making it the most suitable choice for the quarter-wave transformer in this scenario.

\subsubsection{Conclusion}
In conclusion, the suitable transmission line impedance for constructing a quarter-wave Q-section to match a 100-ohm feed point impedance to a 50-ohm transmission line is \textbf{75 ohms:}.

% \begin{center}
% \begin{tikzpicture}
% \draw[->] (0,0) -- (1,0) node[midway, below] {50 ohms (Z_L)};
% \draw[->] (1.5,0) -- (2.5,0) node[midway, below] {75 ohms (Z_0)};
% \draw[->] (3,0) -- (4,0) node[midway, below] {100 ohms (Z_{in})};
% \draw[thick] (0,0) -- (0,1.5);
% \draw[thick] (1.5,0) -- (1.5,1.5);
% \draw[thick] (3,0) -- (3,1.5);
% \draw[dotted] (-0.2,0) -- (0,0.2) node[above left] {Feed Point};
% \draw[dotted] (4,0) -- (4.2,0.2) node[above right] {Line};
% \end{tikzpicture}
% \end{center}
