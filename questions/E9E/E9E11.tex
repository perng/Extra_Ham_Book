\subsection{Boosting Signal Power: The Magic of Phased Driven Elements!}
\label{sec:E9E11}

\begin{tcolorbox}[colback=gray!10!white,colframe=black!75!black,title=\textbf{E9E11}]
\textbf{What is the purpose of using multiple driven elements connected through phasing lines?}
\begin{enumerate}[label=\Alph*.]
    \item \textbf{To control the antenna’s radiation pattern}
    \item To prevent harmonic radiation from the transmitter
    \item To allow single-band antennas to operate on other bands
    \item To create a low-angle radiation pattern
\end{enumerate}
\end{tcolorbox}

\subsubsection{Intuitive Explanation}
Imagine you’re at a concert, and the band is playing. If all the speakers are pointing in different directions, the sound will be all over the place, and you might not hear it clearly. But if the speakers are all pointing in the same direction, the sound will be much louder and clearer where you’re standing. That’s kind of what happens with antennas! When we use multiple driven elements connected through phasing lines, it’s like pointing all the speakers in the same direction. This helps control where the radio waves go, making the signal stronger in the direction we want. So, the answer is A: To control the antenna’s radiation pattern.

\subsubsection{Advanced Explanation}
In antenna theory, the radiation pattern is a graphical representation of the distribution of radiated power as a function of direction. By using multiple driven elements connected through phasing lines, we can manipulate the phase and amplitude of the currents in each element. This allows us to control the constructive and destructive interference of the electromagnetic waves, thereby shaping the antenna’s radiation pattern.

Mathematically, the far-field radiation pattern \( E(\theta, \phi) \) of an array of \( N \) driven elements can be expressed as:

\[
E(\theta, \phi) = \sum_{n=1}^{N} I_n e^{j(k \cdot \mathbf{r}_n + \phi_n)}
\]

where:
\begin{itemize}
    \item \( I_n \) is the current in the \( n \)-th element,
    \item \( k \) is the wave number,
    \item \( \mathbf{r}_n \) is the position vector of the \( n \)-th element,
    \item \( \phi_n \) is the phase shift introduced by the phasing lines.
\end{itemize}

By carefully adjusting the phase shifts \( \phi_n \), we can steer the main lobe of the radiation pattern in the desired direction, enhancing the antenna’s performance in that direction. This technique is fundamental in applications such as beamforming and phased array antennas.

% Prompt for diagram: 
% Generate a diagram showing an array of driven elements connected through phasing lines, with arrows indicating the direction of the main radiation lobe.