\subsection{Exploring the Exciting Data from Amateur Radio Propagation Networks!}

\begin{tcolorbox}[colback=gray!10!white,colframe=black!75!black,title=E3C09] What type of data is reported by amateur radio propagation reporting networks?
    \begin{enumerate}[label=\Alph*.]
        \item Solar flux
        \item Electric field intensity
        \item Magnetic declination
        \item \textbf{Digital-mode and CW signals}
    \end{enumerate}
\end{tcolorbox}

\subsubsection{Intuitive Explanation}
Amateur radio propagation reporting networks are like a big team of radio enthusiasts who share information about how well their radios are working. They don’t talk about things like how bright the sun is (solar flux) or how strong the electric or magnetic fields are. Instead, they focus on the signals they send and receive, especially those using digital modes and Morse code (CW signals). These signals help them understand how well their messages are traveling through the air.

\subsubsection{Advanced Explanation}
Amateur radio propagation reporting networks primarily collect and report data related to the reception of digital-mode and continuous wave (CW) signals. These networks, such as the Reverse Beacon Network (RBN) and WSPRnet, rely on automated systems to detect and decode signals transmitted by amateur radio operators. The data includes information such as signal strength, frequency, and the location of the transmitting and receiving stations.

The correct answer, \textbf{D}, highlights that these networks focus on digital-mode and CW signals rather than other types of data like solar flux (A), which measures solar radiation at a specific frequency, electric field intensity (B), which relates to the strength of an electric field, or magnetic declination (C), which is the angle between magnetic north and true north. These networks are crucial for understanding propagation conditions, which are influenced by factors such as ionospheric conditions, solar activity, and atmospheric conditions.

% Diagram Prompt: Generate a diagram showing the flow of data in an amateur radio propagation reporting network, including transmitting and receiving stations, and the network's data collection process.