\subsection{Becoming a Certified Volunteer Examiner: Your Guide to Accreditation!}

\begin{tcolorbox}[colback=gray!10!white,colframe=black!75!black,title=E1E04]
\textbf{E1E04} What is required to be accredited as a Volunteer Examiner?
\begin{enumerate}[label=\Alph*]
    \item Each General, Advanced and Amateur Extra class operator is automatically accredited as a VE when the license is granted
    \item The amateur operator applying must pass a VE examination administered by the FCC Enforcement Bureau
    \item The prospective VE must obtain accreditation from the FCC
    \item \textbf{A VEC must confirm that the VE applicant meets FCC requirements to serve as an examiner}
\end{enumerate}
\end{tcolorbox}

\subsubsection{Intuitive Explanation}
Becoming a Volunteer Examiner (VE) is like getting a special badge that allows you to help others become licensed radio operators. You don’t automatically get this badge just because you have a higher-level license. Instead, a special group called a Volunteer Examiner Coordinator (VEC) checks to make sure you meet all the rules set by the Federal Communications Commission (FCC). Think of the VEC as a teacher who makes sure you’re ready to help others take their tests.

\subsubsection{Advanced Explanation}
To become a Volunteer Examiner (VE), an individual must meet specific criteria outlined by the Federal Communications Commission (FCC). The process involves the following steps:

1. **Eligibility**: The applicant must hold an Amateur Extra, Advanced, or General class license. These licenses indicate a higher level of knowledge and experience in amateur radio operations.

2. **Accreditation**: The applicant must be accredited by a Volunteer Examiner Coordinator (VEC). The VEC is responsible for ensuring that the applicant meets all FCC requirements, including:
   - Being at least 18 years old.
   - Not being a representative of a telecommunications company.
   - Not having any conflicts of interest that could compromise the integrity of the examination process.

3. **Confirmation**: The VEC confirms that the applicant meets these requirements and accredits them as a VE. This accreditation allows the individual to administer amateur radio license examinations.

The correct answer to the question is \textbf{D}, as it accurately describes the role of the VEC in confirming that the VE applicant meets FCC requirements.

% Prompt for generating a diagram: A flowchart showing the steps from obtaining a higher-level license to becoming a Volunteer Examiner, with the VEC's role highlighted.