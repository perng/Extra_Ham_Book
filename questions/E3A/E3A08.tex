\subsection{Where Meteors Light Up the Ionosphere!}

\begin{tcolorbox}[colback=gray!10!white,colframe=black!75!black,title=E3A08] When a meteor strikes the Earth’s atmosphere, a linear ionized region is formed at what region of the ionosphere?
    \begin{enumerate}[label=\Alph*)]
        \item \textbf{The E region}
        \item The F1 region
        \item The F2 region
        \item The D region
    \end{enumerate}
\end{tcolorbox}

\subsubsection{Intuitive Explanation}
Imagine the Earth’s atmosphere as a layered cake. When a meteor zips through the sky, it creates a glowing trail, much like a sparkler on a dark night. This glowing trail happens in a specific layer of the atmosphere called the \textit{E region}. Think of the E region as the middle layer of the cake, where meteors leave their mark by ionizing the air, making it glow briefly.

\subsubsection{Advanced Explanation}
The ionosphere is divided into several layers based on altitude and ionization characteristics. The E region is located approximately between 90 km and 150 km above the Earth’s surface. When a meteor enters the Earth’s atmosphere at high speed, it collides with air molecules, causing ionization. This ionization creates a linear trail of charged particles, primarily in the E region. 

The E region is particularly suitable for this phenomenon because:
\begin{itemize}
    \item It has a sufficient density of air molecules to cause ionization upon collision.
    \item The altitude is high enough to allow the meteor to travel a significant distance before disintegrating.
\end{itemize}

Mathematically, the ionization process can be described by the energy transfer during collisions:
\[
E = \frac{1}{2} m v^2
\]
where \(E\) is the energy transferred, \(m\) is the mass of the meteor, and \(v\) is its velocity. This energy ionizes the air molecules, creating the visible trail.

Other regions of the ionosphere, such as the D, F1, and F2 regions, are either too low or too high in altitude to produce the same effect. The D region is too dense, causing meteors to disintegrate quickly, while the F regions are too sparse for significant ionization trails to form.

% Prompt for diagram: A diagram showing the layers of the ionosphere (D, E, F1, F2) with a meteor trail highlighted in the E region would be helpful.