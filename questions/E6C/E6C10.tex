\subsection{Spot the NOR Gate: A Fun Challenge!}

\begin{tcolorbox}[colback=gray!10!white,colframe=black!75!black,title=Multiple Choice Question]
\textbf{E6C10} In Figure E6-3, which is the schematic symbol for a NOR gate?
\begin{enumerate}[label=\Alph*.]
    \item 1
    \item 2
    \item 3
    \item \textbf{4}
\end{enumerate}
\end{tcolorbox}

\subsubsection*{Intuitive Explanation}
Imagine you have a gate that only lets you through if you are not wearing a red shirt \textbf{and} not wearing blue shoes. A NOR gate is like that gate, but for electricity! It only lets the signal through if both inputs are off. In Figure E6-3, the symbol labeled 4 is the NOR gate because it has a special shape that tells us it works this way.

\subsubsection*{Advanced Explanation}
A NOR gate is a digital logic gate that implements logical NOR operation. The NOR operation is a combination of the NOT and OR operations. The output of a NOR gate is true only when all of its inputs are false. Mathematically, the NOR operation can be represented as:

\[
Y = \overline{A + B}
\]

where \( A \) and \( B \) are the inputs, and \( Y \) is the output. The overline represents the NOT operation, and the plus sign represents the OR operation.

In schematic diagrams, the NOR gate is typically represented by a specific symbol that includes an OR gate followed by a small circle (denoting the NOT operation). In Figure E6-3, the symbol labeled 4 correctly represents this combination, making it the NOR gate.

% Prompt for generating the diagram:
% Include a diagram showing the schematic symbols for various logic gates, highlighting the NOR gate with label 4.