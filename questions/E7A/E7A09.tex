\subsection{Exploring the Magic of XOR: What Does an Exclusive NOR Gate Do?}

\begin{tcolorbox}[colback=gray!10!white,colframe=black!75!black,title=E7A09] What logical operation is performed by a two-input exclusive NOR gate?
    \begin{enumerate}[label=\Alph*]
        \item It produces a 0 at its output only if all inputs are 0
        \item It produces a 1 at its output only if all inputs are 1
        \item \textbf{It produces a 0 at its output if one and only one of its inputs is 1}
        \item It produces a 1 at its output if one and only one input is 1
    \end{enumerate}
\end{tcolorbox}

\subsubsection{Intuitive Explanation}
Imagine you have a special gate that checks two switches. If both switches are in the same position (both on or both off), the gate will give you a green light (which we can think of as a 1). But if one switch is on and the other is off, the gate will give you a red light (which we can think of as a 0). This special gate is called an Exclusive NOR (XNOR) gate. It’s like a fairness checker—it only gives a green light when both switches agree.

\subsubsection{Advanced Explanation}
An Exclusive NOR (XNOR) gate is a digital logic gate that outputs true (1) only when both inputs are the same. Mathematically, the XNOR operation can be represented as:

\[
\text{XNOR}(A, B) = A \odot B = \overline{A \oplus B}
\]

Where:
- \( A \) and \( B \) are the input signals,
- \( \oplus \) represents the XOR operation,
- \( \odot \) represents the XNOR operation,
- \( \overline{A \oplus B} \) denotes the negation of the XOR operation.

The truth table for a two-input XNOR gate is as follows:

\[
\begin{array}{|c|c|c|}
\hline
A & B & \text{XNOR}(A, B) \\
\hline
0 & 0 & 1 \\
0 & 1 & 0 \\
1 & 0 & 0 \\
1 & 1 & 1 \\
\hline
\end{array}
\]

From the truth table, it is clear that the XNOR gate outputs 1 when both inputs are the same (either both 0 or both 1) and outputs 0 when the inputs are different. This behavior is the inverse of the XOR gate, which outputs 1 when the inputs are different.

The XNOR gate is commonly used in digital circuits for equality checking and parity generation. It is also a fundamental component in more complex logic circuits, such as adders and comparators.

% Diagram prompt: Generate a diagram showing the logic symbol of a two-input XNOR gate and its truth table.