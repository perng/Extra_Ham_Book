\subsection{Power Play: Unveiling the Load's Absorption!}

\begin{tcolorbox}[colback=gray!10, colframe=black, title=E4B06]
How much power is being absorbed by the load when a directional power meter connected between a transmitter and a terminating load reads 100 watts forward power and 25 watts reflected power?

\begin{enumerate}[label=\Alph*.]
    \item 100 watts
    \item 125 watts
    \item 112.5 watts
    \item \textbf{75 watts}
\end{enumerate} \end{tcolorbox}

\subsubsection{Concepts Involved}

To answer this question, we need to understand the principles of power measurement in transmission lines, specifically in the context of forward and reflected power readings from a directional power meter.

\subsubsection{Power Calculation}

The power absorbed by the load, denoted as \( P_{\text{load}} \), can be calculated using the following formula:

\[
P_{\text{load}} = P_{\text{forward}} - P_{\text{reflected}}
\]

Where:
- \( P_{\text{forward}} \) is the forward power measured by the meter (the power being transmitted towards the load).
- \( P_{\text{reflected}} \) is the reflected power (the power that bounces back due to impedance mismatch).

In this scenario, we have:

\[
P_{\text{forward}} = 100 \text{ watts}
\]
\[
P_{\text{reflected}} = 25 \text{ watts}
\]

Substituting these values into the formula, we get:

\[
P_{\text{load}} = 100 \text{ watts} - 25 \text{ watts} = 75 \text{ watts}
\]

Thus, the power absorbed by the load is 75 watts.

\subsubsection{Conclusion}

In summary, when a directional power meter shows a forward power of 100 watts and a reflected power of 25 watts, the power absorbed by the load is 75 watts, making option D the correct choice. This principle is crucial for understanding the efficiency and performance of RF systems, ensuring proper load matching and minimizing losses due to reflections.
