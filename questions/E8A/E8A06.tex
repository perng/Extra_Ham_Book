\subsection{Decoding Signal Magic: PEP vs. Average Power!}

\begin{tcolorbox}[colback=yellow!10!white,colframe=yellow!80!black,title=\textbf{Question ID: E8A06}]
What is the approximate ratio of PEP-to-average power in an unprocessed single-sideband phone signal? 
\begin{enumerate}[label=\Alph*.]
    \item \textbf{2.5 to 1}
    \item 25 to 1
    \item 1 to 1
    \item 13 to 1
\end{enumerate}
\end{tcolorbox}

\subsubsection{Intuitive Explanation}
Imagine you are talking to a friend using a toy walkie-talkie. When you talk, the sounds are turned into signals that travel through the air. Some signals can carry more energy and sound clearer than others. The PEP, or Peak Envelope Power, is like the loudest part of your voice. The average power is like the average noise level of your entire conversation. In this case, we want to compare the loudest part of the signal to the quiet parts over time. A signal that has a PEP-to-average power ratio of about 2.5 to 1 means that the loud parts are more powerful than the average parts, which helps the message come through clearer.

\subsubsection{Advanced Explanation}
The Peak Envelope Power (PEP) is a crucial measurement in telecommunications, especially in single-sideband (SSB) modulation which is commonly used in voice transmissions. PEP measures the maximum power output of the signal, while average power provides an overall measure of how much power is being used over time.

In SSB transmissions, the PEP-to-average power ratio gives us insight into the efficiency and effectiveness of the signal. A higher ratio indicates that the signal can peak significantly higher than the average, which can improve the intelligibility and clarity of the communication.

To calculate this ratio accurately, one must understand the balance between PEP and the average power in the context of the signal characteristics. The appropriate ratio in this case is approximately 2.5 to 1. 

This means that, at its peak, the power output is about 2.5 times greater than the average. The importance of this ratio is evident in systems where signal clarity is particularly important, such as in radio communications where background noise can interfere with the transmission.

\begin{tikzpicture}
% Prompt for diagram: Peak Envelope Power vs Average Power ratio illustration
\end{tikzpicture}