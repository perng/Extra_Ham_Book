\subsection{Understanding Conduction in Class AB Amplifiers!}

\begin{tcolorbox}[colback=blue!5!white,colframe=blue!75!black]
    \textbf{E7B01} For what portion of the signal cycle does each active element in a push-pull, Class AB amplifier conduct?
    \begin{enumerate}[label=\Alph*)]
        \item \textbf{More than 180 degrees but less than 360 degrees}
        \item Exactly 180 degrees
        \item The entire cycle
        \item Less than 180 degrees
    \end{enumerate}
\end{tcolorbox}

\subsubsection{Intuitive Explanation}
Imagine you have two people taking turns pushing a swing. In a Class AB amplifier, each person (or active element) doesn't just push for half the time (180 degrees) or the whole time (360 degrees). Instead, they push for a bit more than half the time but not the entire time. This way, the swing (or the signal) keeps moving smoothly without any gaps or overlaps. So, each active element conducts for more than 180 degrees but less than 360 degrees of the signal cycle.

\subsubsection{Advanced Explanation}
In a push-pull Class AB amplifier, the active elements (usually transistors) are biased such that they conduct for more than half but less than the full cycle of the input signal. This is achieved by setting the bias point slightly above the cutoff point, ensuring that each transistor conducts for more than 180 degrees but less than 360 degrees of the signal cycle. This configuration reduces crossover distortion, which occurs when the signal transitions from one transistor to the other.

Mathematically, the conduction angle $\theta$ for each transistor in a Class AB amplifier satisfies:
\[
180^\circ < \theta < 360^\circ
\]
This ensures that there is always at least one transistor conducting, minimizing distortion and improving efficiency.

The related concepts include:
\begin{itemize}
    \item \textbf{Bias Point}: The DC voltage applied to the transistor to set its operating point.
    \item \textbf{Crossover Distortion}: Distortion that occurs when the signal transitions from one transistor to the other in a push-pull amplifier.
    \item \textbf{Conduction Angle}: The portion of the signal cycle during which the transistor is conducting.
\end{itemize}

% Prompt for generating a diagram:
% Diagram showing the conduction angles of the two transistors in a push-pull Class AB amplifier, with one transistor conducting for more than 180 degrees but less than 360 degrees of the signal cycle.