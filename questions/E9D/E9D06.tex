\subsection{Boosting Bandwidth: The Magic of Loading Coils!}

\begin{tcolorbox}[colback=gray!10, colframe=black, title=E9D06]
What happens to SWR bandwidth when one or more loading coils are used to resonate an electrically short antenna? 
\begin{enumerate}[label=\Alph*.]
    \item It is increased
    \item \textbf{It is decreased}
    \item It is unchanged if the loading coil is located at the feed point
    \item It is unchanged if the loading coil is located at a voltage maximum point
\end{enumerate} \end{tcolorbox}

\subsubsection{Relevant Concepts}
To understand the implications of using loading coils on SWR (Standing Wave Ratio) bandwidth in electrically short antennas, we need to consider the following concepts:

\begin{itemize}
    \item \textbf{Electrically Short Antennas:} These are antennas whose length is significantly shorter than the wavelength of the signal they are intended to transmit or receive. An electrically short antenna often exhibits high reactance and low radiation resistance.
    
    \item \textbf{Loading Coils:} These are inductive components added to antennas to effectively increase their electrical length. By doing so, they can help resonate an electrically short antenna at a desired frequency. However, the addition of loading coils alters the antenna's impedance characteristics.
    
    \item \textbf{SWR Bandwidth:} The bandwidth of an antenna is defined as the range of frequencies over which the antenna operates efficiently. SWR bandwidth specifically refers to the frequency range over which the SWR remains below a certain level (often 2:1).
\end{itemize}

\subsubsection{Effect of Loading Coils on SWR Bandwidth}
When loading coils are introduced to an electrically short antenna, they create additional reactive components in the antenna system. This alteration typically leads to:

\begin{itemize}
    \item A decrease in the bandwidth due to the increased reactance associated with the coils.
    \item Tighter coupling around the resonant frequency, which can result in higher Q (quality factor) values, thereby reducing the bandwidth. In essence, a higher Q means that the antenna is more selective to a narrower frequency range.
\end{itemize}

Therefore, the correct answer to the question is that the SWR bandwidth is decreased when loading coils are utilized to resonate an electrically short antenna.

\subsubsection{Summary of Calculation and Principles}
No complex calculations are needed to arrive at the conclusion regarding SWR bandwidth and loading coils since it is predominantly a matter of understanding the interaction between these components. However, if the antenna's impedance needs to be calculated, we could apply:

\[
Z_{total} = R + jX
\]

Where \( R \) is the resistance and \( jX \) represents the reactance introduced by the loading coils.

% \begin{tikzpicture}
%     \draw[->] (0,0) -- (5,0) node[right] {Frequency};
%     \draw[->] (0,0) -- (0,5) node[above] {SWR};
%     \draw[domain=0:5, samples=100] plot (\x, {4/(\x-3)^2});
%     \draw[dashed] (3,0) -- (3,5);
%     \node at (3,-0.5) {Resonant Frequency};
%     \node at (1,-0.5) {Lower Bandwidth Limit};
%     \node at (5,-0.5) {Upper Bandwidth Limit};
% \end{tikzpicture}
