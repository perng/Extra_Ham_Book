\subsection{Maximizing Frequencies: The Best Materials for MMICs!}

\begin{tcolorbox}[colback=gray!10!white,colframe=black!75!black,title=E6E03] Which of the following materials supports the highest frequency of operation when used in MMICs?
    \begin{enumerate}[label=\Alph*]
        \item Silicon
        \item Silicon nitride
        \item Silicon dioxide
        \item \textbf{Gallium nitride}
    \end{enumerate}
\end{tcolorbox}

\subsubsection{Intuitive Explanation}
Imagine you are trying to send a very fast message using a material. Some materials are like slow walkers, while others are like speedy runners. In the world of tiny electronic chips called MMICs (Monolithic Microwave Integrated Circuits), Gallium nitride is the fastest runner. It can handle very high frequencies, which means it can send messages much quicker than the other materials listed, like Silicon, Silicon nitride, and Silicon dioxide. So, if you want your chip to work at the highest possible speed, Gallium nitride is the best choice!

\subsubsection{Advanced Explanation}
The frequency of operation in MMICs is largely determined by the material's electronic properties, particularly its bandgap and electron mobility. Gallium nitride (GaN) has a wide bandgap (approximately 3.4 eV) and high electron mobility, which allows it to operate at much higher frequencies compared to Silicon (Si), Silicon nitride (Si3N4), and Silicon dioxide (SiO2). 

The wide bandgap of GaN enables it to sustain higher electric fields without breaking down, which is crucial for high-frequency operation. Additionally, the high electron mobility in GaN allows electrons to move quickly through the material, reducing the time it takes for signals to propagate. This combination of properties makes GaN the optimal material for MMICs designed to operate at the highest frequencies.

To illustrate, consider the following simplified calculation of the cutoff frequency \( f_T \) for a semiconductor material:

\[
f_T = \frac{v_{sat}}{2\pi L}
\]

where \( v_{sat} \) is the saturation velocity of electrons and \( L \) is the gate length. For GaN, \( v_{sat} \) is significantly higher than for Si, leading to a higher \( f_T \). This means GaN-based devices can operate at higher frequencies compared to Si-based devices.

In summary, Gallium nitride's superior electronic properties make it the best material for MMICs that need to support the highest frequencies.

% [Prompt for diagram: A comparison chart showing the bandgap and electron mobility of Silicon, Silicon nitride, Silicon dioxide, and Gallium nitride, highlighting why Gallium nitride supports the highest frequency of operation.]