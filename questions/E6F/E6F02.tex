\subsection{Bright Reactions: The Magic of Light on Photoconductive Materials!}

\begin{tcolorbox}[colback=gray!10!white,colframe=black!75!black,title=E6F02] What happens to photoconductive material when light shines on it?
    \begin{enumerate}[label=\Alph*)]
        \item \textbf{Resistance decreases}
        \item Resistance increases
        \item Reflectivity increases
        \item Reflectivity decreases
    \end{enumerate}
\end{tcolorbox}

\subsubsection{Intuitive Explanation}
Imagine you have a special material that changes its behavior when light shines on it. This material is called photoconductive material. When light hits it, the material becomes more open to letting electricity flow through it. Think of it like a gate that opens wider when the sun comes out. Because the gate is wider, it’s easier for electricity to pass through, which means the material’s resistance (how much it resists the flow of electricity) decreases. So, when light shines on photoconductive material, its resistance goes down.

\subsubsection{Advanced Explanation}
Photoconductive materials are semiconductors whose electrical conductivity increases when exposed to light. This phenomenon occurs because photons from the light provide enough energy to excite electrons from the valence band to the conduction band, creating electron-hole pairs. These free charge carriers increase the material's conductivity, which is inversely proportional to its resistance. Mathematically, conductivity (\(\sigma\)) is given by:

\[
\sigma = n e \mu
\]

where:
\begin{itemize}
    \item \(n\) is the number of charge carriers,
    \item \(e\) is the electron charge,
    \item \(\mu\) is the mobility of the charge carriers.
\end{itemize}

When light shines on the material, \(n\) increases, leading to an increase in \(\sigma\) and a corresponding decrease in resistance (\(R\)), as:

\[
R = \frac{1}{\sigma}
\]

This principle is widely used in devices like photoresistors and photodetectors, where the change in resistance is used to detect or measure light intensity.

% [Prompt for diagram: A diagram showing the energy bands of a semiconductor before and after light exposure, illustrating the creation of electron-hole pairs and the resulting increase in conductivity.]