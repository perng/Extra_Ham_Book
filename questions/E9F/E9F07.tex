\subsection{Electric Pathways: Parallel Conductors vs. Coaxial Cables!}
\label{sec:E9F07}

\begin{tcolorbox}[colback=gray!10!white,colframe=black!75!black,title=Question E9F07]
\textbf{E9F07} How does parallel conductor transmission line compare to coaxial cable with a plastic dielectric?

\begin{enumerate}[label=\Alph*)]
    \item \textbf{Lower loss}
    \item Higher SWR
    \item Smaller reflection coefficient
    \item Lower velocity factor
\end{enumerate}
\end{tcolorbox}

\subsubsection{Intuitive Explanation}
Imagine you're trying to send a message through two different types of pipes: one is a pair of straight, parallel pipes, and the other is a fancy, insulated tube (like a coaxial cable). The parallel pipes are like a simple, direct path—less stuff in the way means less energy gets lost along the way. On the other hand, the fancy tube has more layers and insulation, which can slow things down and cause more energy to be lost. So, the parallel conductors are like the express lane for your message, with lower loss compared to the coaxial cable.

\subsubsection{Advanced Explanation}
In transmission line theory, the loss in a transmission line is primarily due to conductor loss and dielectric loss. Parallel conductor transmission lines, such as twin-lead or ladder line, typically have lower dielectric loss compared to coaxial cables with plastic dielectrics. This is because the dielectric material in coaxial cables (e.g., polyethylene) introduces additional loss due to its inherent properties.

The characteristic impedance \( Z_0 \) of a transmission line is given by:
\[
Z_0 = \sqrt{\frac{R + j\omega L}{G + j\omega C}}
\]
where \( R \) is the resistance per unit length, \( L \) is the inductance per unit length, \( G \) is the conductance per unit length, and \( C \) is the capacitance per unit length. For parallel conductors, the dielectric loss is minimal, leading to a lower overall loss compared to coaxial cables.

Additionally, the velocity factor \( v_f \) of a transmission line is given by:
\[
v_f = \frac{1}{\sqrt{\mu_r \epsilon_r}}
\]
where \( \mu_r \) is the relative permeability and \( \epsilon_r \) is the relative permittivity of the dielectric. Coaxial cables with plastic dielectrics typically have a lower velocity factor due to the higher permittivity of the dielectric material.

In summary, parallel conductor transmission lines generally exhibit lower loss compared to coaxial cables with plastic dielectrics due to reduced dielectric loss and simpler construction.

% Diagram Prompt: A comparison diagram showing the cross-section of a parallel conductor transmission line and a coaxial cable, highlighting the dielectric material and conductor arrangement.