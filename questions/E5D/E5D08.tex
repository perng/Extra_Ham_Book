\subsection{Unveiling the Mystery: What's Behind Film Capacitor Loss at RF?}

\begin{tcolorbox}[colback=gray!10!white,colframe=black!75!black,title=E5D08] What is the primary cause of loss in film capacitors at RF?
    \begin{enumerate}[label=\Alph*)]
        \item Inductance
        \item Dielectric loss
        \item Self-discharge
        \item \textbf{Skin effect}
    \end{enumerate}
\end{tcolorbox}

\subsubsection*{Intuitive Explanation}
Imagine you have a thin sheet of plastic (the film) with some metal on it, and you roll it up to make a capacitor. When you use this capacitor at very high frequencies (like in radio signals), the electricity doesn't flow evenly through the metal. Instead, it tends to stay on the surface, like water skimming the top of a pond. This is called the skin effect. Because the electricity isn't using the whole metal, the capacitor doesn't work as well, and you lose some of the signal. This is the main reason film capacitors don't perform perfectly at RF frequencies.

\subsubsection*{Advanced Explanation}
At radio frequencies (RF), the skin effect becomes a significant factor in the performance of film capacitors. The skin effect is a phenomenon where alternating current (AC) tends to flow near the surface of a conductor rather than through its entire cross-section. This effect is quantified by the skin depth (\(\delta\)), which is given by:

\[
\delta = \sqrt{\frac{2\rho}{\omega\mu}}
\]

where:
\begin{itemize}
    \item \(\rho\) is the resistivity of the conductor,
    \item \(\omega\) is the angular frequency of the AC signal,
    \item \(\mu\) is the permeability of the conductor.
\end{itemize}

As the frequency increases, the skin depth decreases, causing the effective resistance of the conductor to increase. This increased resistance leads to higher losses in the capacitor. In film capacitors, the metal layers are thin, and the skin effect can significantly reduce the effective cross-sectional area available for current flow, leading to increased resistive losses.

Dielectric loss, inductance, and self-discharge are also factors that can affect capacitor performance, but at RF frequencies, the skin effect is the primary cause of loss in film capacitors. The dielectric loss is more relevant at lower frequencies, while inductance and self-discharge are generally less significant compared to the skin effect in this context.

% Diagram Prompt: Generate a diagram showing the skin effect in a film capacitor, illustrating how current density decreases with depth in the conductor at high frequencies.