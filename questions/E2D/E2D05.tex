\subsection{Discovering the Joy of JT65 Mode!}

\begin{tcolorbox}[colback=gray!10!white,colframe=black!75!black,title=E2D05] What is the characteristic of the JT65 mode?
    \begin{enumerate}[label=\Alph*)]
        \item Uses only a 65 Hz bandwidth
        \item \textbf{Decodes signals with a very low signal-to-noise ratio}
        \item Symbol rate is 65 baud
        \item Permits fast-scan TV transmissions over narrow bandwidth
    \end{enumerate}
\end{tcolorbox}

\subsubsection{Intuitive Explanation}
Imagine you are trying to talk to a friend in a very noisy room. It’s hard to hear each other, right? Now, think of JT65 as a special way of talking that allows you to understand your friend even when the noise is almost drowning out their voice. JT65 is like a super-sensitive ear that can pick up very faint signals, even when there’s a lot of noise around. This makes it really useful for communicating over long distances or in challenging conditions.

\subsubsection{Advanced Explanation}
The JT65 mode is a digital communication protocol designed for weak signal communication. It operates by transmitting a series of 65 tones, each lasting for a specific duration. The key characteristic of JT65 is its ability to decode signals with a very low signal-to-noise ratio (SNR). This is achieved through sophisticated error correction algorithms and the use of multiple frequency shifts to encode data.

Mathematically, the SNR is defined as:
\[
\text{SNR} = \frac{P_{\text{signal}}}{P_{\text{noise}}}
\]
where \( P_{\text{signal}} \) is the power of the signal and \( P_{\text{noise}} \) is the power of the noise. JT65 can decode signals with an SNR as low as -24 dB, which is significantly lower than many other communication modes.

The mode uses a bandwidth of approximately 250 Hz, which is wider than the 65 Hz mentioned in option A. The symbol rate is not 65 baud (option C), and it is not designed for fast-scan TV transmissions (option D). Instead, JT65 is optimized for reliable communication under weak signal conditions, making option B the correct answer.

% Diagram prompt: Generate a diagram showing the frequency spectrum of JT65 signals, highlighting the 65 tones and the bandwidth used.