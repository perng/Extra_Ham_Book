\subsection{Unlocking the Mystery of WIDE3-1!}

\begin{tcolorbox}[colback=gray!10!white,colframe=black!75!black,title=E2D10] What does the packet path WIDE3-1 designate?
    \begin{enumerate}[label=\Alph*]
        \item Three stations are allowed on frequency, one transmitting at a time
        \item Three subcarriers are permitted, subcarrier one is being used
        \item \textbf{Three digipeater hops are requested with one remaining}
        \item Three internet gateway stations may receive one transmission
    \end{enumerate}
\end{tcolorbox}

\subsubsection{Intuitive Explanation}
Imagine you're sending a message through a series of relay stations, like passing a note through a line of friends. The term WIDE3-1 is like saying, Hey, I want this message to go through three relay stations, and after it passes through two, there's still one more to go. It's a way to control how far your message travels and how many times it gets passed along.

\subsubsection{Advanced Explanation}
In packet radio communication, the term WIDE3-1 refers to a specific path designation used in the AX.25 protocol. The WIDE part indicates that the packet is intended to be forwarded by digipeaters (digital repeaters). The number 3 specifies the total number of digipeater hops requested, and the -1 indicates that one hop has already been used, leaving two more hops available. This path designation helps in controlling the propagation of the packet through the network, ensuring it reaches the intended destination without unnecessary retransmissions.

Mathematically, if we denote the total number of requested hops as \( H \) and the number of hops already used as \( h \), then the remaining hops \( R \) can be calculated as:
\[
R = H - h
\]
For WIDE3-1, \( H = 3 \) and \( h = 1 \), so:
\[
R = 3 - 1 = 2
\]
This means there are two more digipeater hops available for the packet to traverse.

% Diagram Prompt: Generate a diagram showing a packet being transmitted through three digipeater hops, with one hop already completed and two remaining.