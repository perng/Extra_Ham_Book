\subsection{E9A12: Unlocking Antenna Power: Comparing Gain to a Half-Wavelength Dipole!}

\begin{tcolorbox}[colback=gray!10!white,colframe=black!75!black,title=\textbf{Question E9A12}]
How much gain does an antenna have compared to a half-wavelength dipole if it has 6 dB gain over an isotropic radiator?
\begin{enumerate}[label=\Alph*.]
    \item \textbf{3.85 dB}
    \item 6.0 dB
    \item 8.15 dB
    \item 2.79 dB
\end{enumerate}
\end{tcolorbox}

\subsubsection*{Intuitive Explanation}
Imagine you have two flashlights. One is a regular flashlight (isotropic radiator) that shines light equally in all directions. The other is a super flashlight (your antenna) that shines light 6 times brighter than the regular one. Now, think of a half-wavelength dipole as a flashlight that’s already a bit better than the regular one—it shines light 2.15 times brighter. So, if your super flashlight is 6 times brighter than the regular one, how much brighter is it compared to the already better flashlight? The answer is about 3.85 times brighter! That’s why the gain compared to the half-wavelength dipole is 3.85 dB.

\subsubsection*{Advanced Explanation}
To solve this problem, we need to understand the relationship between the gain of an antenna over an isotropic radiator and its gain over a half-wavelength dipole. 

1. **Gain Over Isotropic Radiator (G\_iso):** The antenna has a gain of 6 dB over an isotropic radiator. In linear terms, this is:
   \[
   G_{\text{iso}} = 10^{6/10} = 4
   \]

2. **Gain of Half-Wavelength Dipole (G\_dipole):** A half-wavelength dipole has a gain of approximately 2.15 dBi (decibels over isotropic). In linear terms:
   \[
   G_{\text{dipole}} = 10^{2.15/10} \approx 1.64
   \]

3. **Gain Over Half-Wavelength Dipole (G\_over\_dipole):** To find the gain of the antenna over the half-wavelength dipole, we divide the gain over isotropic by the gain of the dipole:
   \[
   G_{\text{over\_dipole}} = \frac{G_{\text{iso}}}{G_{\text{dipole}}} = \frac{4}{1.64} \approx 2.44
   \]

4. **Convert to Decibels:** Finally, we convert this linear gain back to decibels:
   \[
   G_{\text{over\_dipole (dB)}} = 10 \log_{10}(2.44) \approx 3.85 \text{ dB}
   \]

Thus, the antenna has a gain of approximately 3.85 dB over a half-wavelength dipole.

% Diagram Prompt: A diagram showing the comparison of gain between an isotropic radiator, a half-wavelength dipole, and the given antenna would be helpful here.