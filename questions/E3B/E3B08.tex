\subsection{Riding the Waves: How Frequency Boosts Ground-Wave Reach!}

\begin{tcolorbox}[colback=gray!10!white,colframe=black!75!black,title=E3B08] How does the maximum range of ground-wave propagation change when the signal frequency is increased?
    \begin{enumerate}[label=\Alph*.]
        \item It stays the same
        \item It increases
        \item \textbf{It decreases}
        \item It peaks at roughly 8 MHz
    \end{enumerate}
\end{tcolorbox}

\subsubsection*{Intuitive Explanation}
Imagine you are throwing a ball. If you throw it gently (low frequency), it will travel a short distance but stay close to the ground. If you throw it harder (high frequency), it will go higher but not as far along the ground. Similarly, in ground-wave propagation, lower frequency signals can travel farther along the Earth's surface, while higher frequency signals tend to lose energy more quickly and don't travel as far.

\subsubsection*{Advanced Explanation}
Ground-wave propagation is influenced by the frequency of the signal and the conductivity of the Earth's surface. The attenuation (loss of signal strength) of ground waves increases with frequency due to the skin effect, where higher frequencies penetrate less deeply into the ground and experience greater losses. Mathematically, the attenuation constant \(\alpha\) is proportional to the square root of the frequency \(f\):

\[
\alpha \propto \sqrt{f}
\]

As the frequency increases, \(\alpha\) increases, leading to higher signal loss and a shorter maximum range. This relationship explains why lower frequencies (e.g., LF and MF bands) are preferred for long-distance ground-wave communication, while higher frequencies (e.g., HF and above) are less effective for this purpose.

Additionally, the Earth's curvature and conductivity play significant roles in ground-wave propagation. Lower frequencies can diffract around the Earth's curvature more effectively, further extending their range. In contrast, higher frequencies are more likely to be absorbed or scattered, reducing their effective range.

% [Prompt for diagram: A graph showing the relationship between signal frequency and ground-wave range, with frequency on the x-axis and range on the y-axis, illustrating the decreasing trend as frequency increases.]