\subsection{Zapping the Noise: Solutions for Radio Frequency Interference!}

\begin{tcolorbox}[colback=gray!10, colframe=black, title=E4E05`]
What is used to suppress radio frequency interference from a line-driven AC motor? \\
\begin{enumerate}[label=\Alph*)]
    \item A high-pass filter in series with the motor’s power leads
    \item \textbf{A brute-force AC-line filter in series with the motor’s power leads}
    \item A bypass capacitor in series with the motor’s field winding
    \item A bypass choke in parallel with the motor’s field winding
\end{enumerate} \end{tcolorbox}

\subsubsection{Related Concepts}

Radio frequency interference (RFI) can be a significant issue in electronic systems, especially when such systems are being operated near devices that have high current demands, like line-driven AC motors. The interference can distort the performance of nearby radio communications and may even disrupt the functionality of other sensitive electronic devices nearby.

In this context, a brute-force AC-line filter (the correct answer) is designed to reduce this interference. Such filters work by attenuating unwanted high-frequency signals that are generated by the motor during its operation. These filters usually consist of a combination of capacitors and inductors strategically placed to form low-pass filter circuits, which allow the fundamental operating frequencies of the AC line to pass through while blocking higher frequencies associated with RFI.

\subsubsection{Calculation and Example}

For a practical implementation, let's consider a simple calculation to understand the impact of a brute-force AC-line filter:

Assume that our motor operates at 60 Hz and generates interference at 10 kHz due to commutation and switching actions. To minimize this interference, we may use a low-pass filter design that has a cutoff frequency lower than 10 kHz but allows 60 Hz to pass.

The cutoff frequency \( f_c \) of a RC (resistor-capacitor) low-pass filter can be calculated with the following formula:

\[
f_c = \frac{1}{2\pi RC}
\]

Where:
- \( R \) is the resistance in ohms.
- \( C \) is the capacitance in farads.

If we want to set \( f_c \) to around 1 kHz, we can rearrange the formula:

\[
RC = \frac{1}{2\pi f_c}
\]

Substituting \( f_c = 1000 \) Hz,

\[
RC = \frac{1}{2\pi(1000)} \approx 0.1592 \text{ seconds}
\]

Now, you can choose different values of \( R \) and \( C \) that maintain this product. For example, let’s choose \( R = 1 \text{ k}\Omega \), then:

\[
C = \frac{0.1592}{1000} \approx 159.2 \mu F
\]

Thus, this RC network should be designed to filter out frequencies above about 1 kHz, effectively reducing the RFI emitted by the AC motor.

% \subsubsection{Diagram}

% We can also visualize the low-pass filter using a simple TikZ diagram illustrating the arrangement of components.

% \begin{center}
% \begin{tikzpicture}
%     \draw (0,0) to[R, l=R] (0,2) -- (2,2) to[C, l=C] (2,0) -- cycle;
%     \node at (0,-0.5) {AC In};
%     \node at (2,-0.5) {AC Out};
%     \draw[->] (0.5, 0) -- (0.5, 1.5);
%     \draw[->] (1.5, 2) -- (1.5, -0.5);
% \end{tikzpicture}
% \end{center}

% This LaTeX code provides a detailed treatment of the question regarding suppressing RFI in AC motors, catering to readers with basic knowledge in electronics and good mathematical skills.