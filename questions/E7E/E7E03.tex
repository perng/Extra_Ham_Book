\subsection{Unlocking the Mystery: What’s a Frequency Discriminator?}

\begin{tcolorbox}[colback=gray!10!white,colframe=black!75!black,title=E7E03] What is a frequency discriminator?
    \begin{enumerate}[label=\Alph*.]
        \item An FM generator circuit
        \item A circuit for filtering closely adjacent signals
        \item An automatic band-switching circuit
        \item \textbf{A circuit for detecting FM signals}
    \end{enumerate}
\end{tcolorbox}

\subsubsection{Intuitive Explanation}
Imagine you have a radio that can pick up different stations. Each station sends out signals at a specific frequency, like a unique musical note. A frequency discriminator is like a special listener inside your radio that can tell which station you’re tuned into by recognizing the unique frequency of that station. It helps your radio understand and play the music or talk from the station you want to hear.

\subsubsection{Advanced Explanation}
A frequency discriminator is a crucial component in FM (Frequency Modulation) receivers. Its primary function is to demodulate the FM signal, converting the frequency variations in the received signal back into the original audio signal. 

Mathematically, an FM signal can be represented as:
\[ s(t) = A_c \cos\left(2\pi f_c t + 2\pi k_f \int_0^t m(\tau) \, d\tau\right) \]
where:
\begin{itemize}
    \item \( A_c \) is the amplitude of the carrier signal,
    \item \( f_c \) is the carrier frequency,
    \item \( k_f \) is the frequency deviation constant,
    \item \( m(t) \) is the modulating signal.
\end{itemize}

The frequency discriminator detects the instantaneous frequency of the FM signal, which is proportional to the derivative of the phase of the signal:
\[ f_i(t) = \frac{1}{2\pi} \frac{d\phi(t)}{dt} \]
where \( \phi(t) = 2\pi f_c t + 2\pi k_f \int_0^t m(\tau) \, d\tau \).

By differentiating the phase, the discriminator extracts the original modulating signal \( m(t) \). This process is essential for accurately recovering the audio information from the FM signal.

Related concepts include:
\begin{itemize}
    \item \textbf{Frequency Modulation (FM)}: A method of encoding information in a carrier wave by varying its frequency.
    \item \textbf{Demodulation}: The process of extracting the original information-bearing signal from a modulated carrier wave.
    \item \textbf{Phase-Locked Loop (PLL)}: Often used in conjunction with frequency discriminators to improve demodulation accuracy.
\end{itemize}

% Diagram prompt: Generate a block diagram of an FM receiver showing the position and function of the frequency discriminator in the signal processing chain.