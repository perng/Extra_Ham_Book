\subsection{Delightful Dual Antenna Patterns!}

\begin{tcolorbox}[colback=gray!10, colframe=black, title=E9C01] 
What type of radiation pattern is created by two 1/4-wavelength vertical antennas spaced 1/2-wavelength apart and fed 180 degrees out of phase?

\begin{enumerate}[label=\Alph*.]
    \item Cardioid
    \item Omni-directional
    \item A figure-eight broadside to the axis of the array
    \item \textbf{A figure-eight oriented along the axis of the array}
\end{enumerate} \end{tcolorbox}

\subsubsection{Related Concepts}

To understand the radiation pattern produced by two antennas, it is essential to grasp the following concepts: 

1. \textbf{Antenna Theory:}: Antennas radiate electromagnetic waves, and their radiation pattern depends on their geometry, spacing, and feeding phase.

2. \textbf{Feed Phase:}: Feeding antennas out of phase (in this case, 180 degrees) affects the superposition of the waves generated by the antennas, significantly affecting the resultant radiation pattern.

3. \textbf{Antenna Spacing and Wavelength:}: The concepts of spacing in terms of wavelengths are crucial; here, the antennas are spaced at a distance of \( \frac{1}{2} \) wavelength.

4. \textbf{Radiation Patterns:}: The description of radiation patterns—such as omni-directional, cardioid, and figure-eight patterns—helps us visualize how the radiation occurs in different directions.

\subsubsection{Calculations and Diagram}

Given that we have two 1/4-wavelength antennas, the following relations hold:
- Each antenna radiates a pattern that generally is a dipole-like radiation.
- Antenna 1 and Antenna 2 are spaced \( \frac{1}{2} \lambda \) (half a wavelength) apart. 

When these antennas are fed 180 degrees out of phase:
- The waves from the antennas will constructively interfere at certain angles and destructively interfere at others. 

Considering the geometry, the resulting radiation pattern forms a figure-eight pattern oriented along the axis of the array. This can be visualized with a simple diagram using TikZ.

\begin{tikzpicture}
    \draw[->] (-3,0) -- (3,0) node[below right] {Direction};
    \draw[->] (0,-2) -- (0,2) node[left] {Radiation};
    \draw[scale=1,domain=-3:3,smooth,variable=\x,blue] 
    plot ({\x},{\x*\x - 1})
    node[right] {Radiation pattern};
    
    \foreach \x in {-1,1} {
        \filldraw[red] (\x,0) circle (2pt) 
        node[below right] {Antenna};
    }
\end{tikzpicture}
