\subsection{Remote Control in Part 97}
\label{T1E10}

\begin{tcolorbox}[colback=gray!10!white,colframe=black!75!black,title=T1E10]
Which of the following is an example of remote control as defined in Part 97?
\begin{enumerate}[label=\Alph*]
    \item Repeater operation
    \item \textbf{Operating the station over the internet}
    \item Controlling a model aircraft, boat, or car by amateur radio
    \item All these choices are correct
\end{enumerate}
\end{tcolorbox}

\subsubsection{Intuitive Explanation}
Imagine you have a super cool radio station, but you’re not at home to play with it. No worries! You can still control it using the internet, just like how you can control your smart lights from your phone. This is called remote control. It’s like having a magic wand that lets you operate your radio station from anywhere in the world, as long as you have an internet connection. So, the correct answer is operating the station over the internet. Easy peasy!

\subsubsection{Advanced Explanation}
In the context of Part 97 of the FCC rules, remote control refers to the operation of an amateur radio station from a location other than where the station is physically located. This is typically achieved through the use of internet-based control systems. 

Repeater operation (Choice A) involves retransmitting signals to extend the range of communication, but it does not inherently involve remote control. Controlling a model aircraft, boat, or car by amateur radio (Choice C) is a form of remote control, but it is not the type of remote control defined in Part 97. 

The correct answer is Choice B, operating the station over the internet, because it directly aligns with the definition of remote control as per Part 97. This method allows operators to control their stations from any location with internet access, providing flexibility and convenience.

% Diagram Prompt: A diagram showing a person using a computer to remotely control a radio station over the internet could be helpful here.