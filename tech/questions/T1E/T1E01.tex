\subsection{Transmission Without a Control Operator}
\label{T1E01}

\begin{tcolorbox}[colback=gray!10!white,colframe=black!75!black,title=T1E01]
When may an amateur station transmit without a control operator?
\begin{enumerate}[label=\Alph*,noitemsep]
    \item When using automatic control, such as in the case of a repeater
    \item When the station licensee is away and another licensed amateur is using the station
    \item When the transmitting station is an auxiliary station
    \item \textbf{Never}
\end{enumerate}
\end{tcolorbox}

\subsubsection{Intuitive Explanation}
Imagine you’re driving a car. You wouldn’t let the car drive itself without someone in the driver’s seat, right? Similarly, an amateur radio station always needs a control operator to make sure everything is running smoothly and legally. Even if the station is automated or someone else is using it, there must always be a responsible person in charge. So, the answer is simple: \textbf{Never} can a station transmit without a control operator.

\subsubsection{Advanced Explanation}
In amateur radio operations, the control operator is the person responsible for ensuring that the station complies with all applicable rules and regulations. According to the FCC rules, a control operator must always be present when the station is transmitting. This is true even in cases where the station is operating under automatic control, such as a repeater, or when another licensed amateur is using the station. The control operator does not necessarily have to be physically present at the station but must be able to take control if necessary. Therefore, the correct answer is \textbf{D: Never}, as there is no scenario where an amateur station can legally transmit without a control operator.

% Prompt for diagram: A diagram showing a control operator managing a radio station with various automated systems could help visualize the concept.