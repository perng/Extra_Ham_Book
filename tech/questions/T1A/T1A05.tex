\subsection{FCC License Grant Verification}
\label{T1A05}

\begin{tcolorbox}[colback=gray!10!white,colframe=black!75!black,title=T1A05]
What proves that the FCC has issued an operator/primary license grant?
\begin{enumerate}[label=\Alph*),noitemsep]
    \item A printed copy of the certificate of successful completion of examination
    \item An email notification from the NCVEC granting the license
    \item \textbf{The license appears in the FCC ULS database}
    \item All these choices are correct
\end{enumerate}
\end{tcolorbox}

\subsubsection*{Intuitive Explanation}
Think of the FCC ULS database as the ultimate hall of fame for radio operators. If your license is listed there, it's like getting your name on the scoreboard—official and undeniable. The other options? They're just practice rounds or friendly nods, but not the real deal.

\subsubsection*{Advanced Explanation}
The Federal Communications Commission (FCC) maintains the Universal Licensing System (ULS) database, which is the authoritative source for all issued licenses. When the FCC grants a license, it is recorded in this database, making it the definitive proof of licensure. While other documents or notifications may indicate progress or completion of steps toward licensure, only the presence of the license in the ULS database confirms that the FCC has officially issued the license. This ensures transparency and accessibility for verification purposes.