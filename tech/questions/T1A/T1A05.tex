\subsection{FCC License Grant Verification}
\label{T1A05}

\begin{tcolorbox}[colback=gray!10!white,colframe=black!75!black,title=T1A05]
What proves that the FCC has issued an operator/primary license grant?
\begin{enumerate}[label=\Alph*)]
    \item A printed copy of the certificate of successful completion of examination
    \item An email notification from the NCVEC granting the license
    \item \textbf{The license appears in the FCC ULS database}
    \item All these choices are correct
\end{enumerate}
\end{tcolorbox}

\subsubsection{Intuitive Explanation}
Imagine you just got a shiny new toy, but you’re not sure if it’s really yours until you see your name on the box. Similarly, when the FCC gives you a license, the best way to know it’s official is to check their big list of licenses, called the ULS database. If your name is there, congrats! You’re officially licensed to operate.

\subsubsection{Advanced Explanation}
The Federal Communications Commission (FCC) maintains the Universal Licensing System (ULS) database, which is the authoritative source for all issued licenses. When an operator/primary license is granted, it is recorded in this database. The ULS database is publicly accessible and provides detailed information about each license, including the licensee’s name, call sign, and the status of the license. 

To verify the issuance of a license, one must query the ULS database using the licensee’s information. The presence of the license in the ULS database is definitive proof that the FCC has issued the license. This method ensures transparency and accuracy in the licensing process.

% Diagram Prompt: A flowchart showing the process of verifying a license in the FCC ULS database could be helpful here.