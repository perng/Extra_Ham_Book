\subsection{Willful Interference in Amateur Radio}
\label{T1A11}

\begin{tcolorbox}[colback=gray!10!white,colframe=black!75!black,title=T1A11]
When is willful interference to other amateur radio stations permitted?
\begin{enumerate}[label=\Alph*)]
    \item To stop another amateur station that is breaking the FCC rules
    \item \textbf{At no time}
    \item When making short test transmissions
    \item At any time, stations in the Amateur Radio Service are not protected from willful interference
\end{enumerate}
\end{tcolorbox}

\subsubsection{Intuitive Explanation}
Imagine you're playing a game with your friends, and everyone has to follow the rules. If someone starts cheating, you might feel like interrupting their game to stop them. But in amateur radio, even if someone is breaking the rules, you can't just barge in and mess with their signals. It's like saying, No matter what, you can't just start yelling over someone else's conversation, even if they're being rude. So, the correct answer is that you're never allowed to willfully interfere with other amateur radio stations.

\subsubsection{Advanced Explanation}
In the context of amateur radio, willful interference refers to the intentional disruption of communications between other licensed amateur radio operators. The Federal Communications Commission (FCC) strictly prohibits such actions under any circumstances. This is outlined in Part 97 of the FCC rules, which governs the Amateur Radio Service. 

The rationale behind this prohibition is to maintain the integrity and reliability of amateur radio communications. Even if another station is violating FCC rules, it is not the responsibility of individual operators to enforce these rules through interference. Instead, such violations should be reported to the FCC for appropriate action.

Mathematically, the concept can be understood in terms of signal integrity. If we consider the transmitted signal \( s(t) \) and the received signal \( r(t) \), any intentional interference \( i(t) \) would corrupt the received signal such that:
\[ r(t) = s(t) + i(t) \]
where \( i(t) \) represents the interference. The goal of amateur radio is to ensure that \( r(t) \) is as close as possible to \( s(t) \), which is compromised by any form of willful interference.

In summary, willful interference is never permitted in amateur radio, regardless of the circumstances. This ensures that all operators can communicate effectively and that the amateur radio spectrum remains a reliable resource for all users.

% Prompt for generating a diagram: A diagram showing the transmission and reception of signals with and without interference could help visualize the concept.