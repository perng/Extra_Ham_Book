\subsection{Willful Interference in Amateur Radio}
\label{T1A11}

\begin{tcolorbox}[colback=gray!10!white,colframe=black!75!black,title=T1A11]
When is willful interference to other amateur radio stations permitted?
\begin{enumerate}[label=\Alph*),noitemsep]
    \item To stop another amateur station that is breaking the FCC rules
    \item \textbf{At no time}
    \item When making short test transmissions
    \item At any time, stations in the Amateur Radio Service are not protected from willful interference
\end{enumerate}
\end{tcolorbox}

\subsubsection*{Intuitive Explanation}
Imagine you're in a classroom where everyone is supposed to take turns speaking. If someone starts talking out of turn, it would be rude to shout over them to make them stop. Similarly, in amateur radio, even if another station is breaking the rules, it's not okay to interfere with their transmission. The proper way to handle it is to report the issue to the authorities, not to take matters into your own hands.

\subsubsection*{Advanced Explanation}
Willful interference in amateur radio is strictly prohibited under FCC regulations. The Amateur Radio Service is designed to promote communication and experimentation, and interference disrupts these activities. Even if another station is violating FCC rules, it is not the responsibility of individual operators to enforce these rules through interference. Instead, operators should report violations to the FCC, which has the authority to take appropriate action. This ensures that the amateur radio community remains a cooperative and respectful environment for all participants.