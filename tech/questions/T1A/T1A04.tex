\subsection{Operator/Primary Station License Grants}
\label{T1A04}

\begin{tcolorbox}[colback=gray!10!white,colframe=black!75!black,title=T1A04]
How many operator/primary station license grants may be held by any one person?
\begin{enumerate}[label=\Alph*)]
    \item \textbf{One}
    \item No more than two
    \item One for each band on which the person plans to operate
    \item One for each permanent station location from which the person plans to operate
\end{enumerate}
\end{tcolorbox}

\subsubsection{Intuitive Explanation}
Imagine you have a driver's license. You can only have one driver's license, right? It doesn't matter if you drive a car, a truck, or a motorcycle—you still only need one license. Similarly, when it comes to radio operator licenses, you only need one license to operate your radio station, no matter how many different bands or locations you plan to use. It's like having one key that opens all the doors you need!

\subsubsection{Advanced Explanation}
In the context of radio operation, the Federal Communications Commission (FCC) regulates the licensing of operators and primary stations. According to FCC rules, an individual is allowed to hold only one operator/primary station license. This license grants the holder the authority to operate a radio station across various frequency bands and locations. The rationale behind this regulation is to streamline the licensing process and ensure that each operator is uniquely identifiable and accountable under a single license. 

% Prompt for diagram: A simple flowchart showing a person holding a single license that covers multiple bands and locations.