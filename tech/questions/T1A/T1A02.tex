\subsection{Regulation of Amateur Radio Service in the U.S.}
\label{T1A02}

\begin{tcolorbox}[colback=gray!10!white,colframe=black!75!black,title=T1A02]
Which agency regulates and enforces the rules for the Amateur Radio Service in the United States?
\begin{enumerate}[label=\Alph*)]
    \item FEMA
    \item Homeland Security
    \item \textbf{The FCC}
    \item All these choices are correct
\end{enumerate}
\end{tcolorbox}

\subsubsection{Intuitive Explanation}
Imagine you’re playing a game, and there’s a referee who makes sure everyone follows the rules. In the world of amateur radio, the Federal Communications Commission (FCC) is that referee. They’re the ones who decide what frequencies you can use, how powerful your radio can be, and even what kind of license you need. So, if you’re wondering who’s in charge of keeping the airwaves orderly, it’s the FCC!

\subsubsection{Advanced Explanation}
The Federal Communications Commission (FCC) is an independent agency of the United States government that regulates interstate and international communications by radio, television, wire, satellite, and cable. The FCC was established by the Communications Act of 1934 and is charged with regulating the use of the radio frequency spectrum, including the Amateur Radio Service. The FCC enforces rules that ensure the efficient and fair use of the radio spectrum, which is a limited resource. This includes issuing licenses to amateur radio operators, setting technical standards for equipment, and monitoring compliance with regulations. The FCC’s authority is derived from the Communications Act of 1934, which grants it the power to regulate all non-federal government use of the radio spectrum.

% Diagram prompt: A flowchart showing the hierarchy of regulatory bodies in the U.S. with the FCC highlighted as the regulator for amateur radio.