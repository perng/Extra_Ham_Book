\subsection{FCC Part 97 Definition of a Space Station}
\label{T1A07}

\begin{tcolorbox}[colback=gray!10!white,colframe=black!75!black,title=T1A07]
What is the FCC Part 97 definition of a space station?
\begin{enumerate}[label=\Alph*,noitemsep]
    \item Any satellite orbiting Earth
    \item A manned satellite orbiting Earth
    \item \textbf{An amateur station located more than 50 km above Earth's surface}
    \item An amateur station using amateur radio satellites for relay of signals
\end{enumerate}
\end{tcolorbox}

\subsubsection*{Intuitive Explanation}
Imagine you're playing a game where you have to define what counts as a space station. The FCC (Federal Communications Commission) has a rulebook called Part 97, and in this rulebook, they say a space station is any amateur radio station that's more than 50 km above the Earth's surface. So, it's not just any satellite or a manned one, but specifically an amateur station that's high up in the sky.

\subsubsection*{Advanced Explanation}
The FCC Part 97 regulations govern the amateur radio service in the United States. According to these regulations, a space station is defined as an amateur station located more than 50 km above the Earth's surface. This definition distinguishes space stations from other types of satellites or manned spacecraft. The 50 km threshold is significant because it marks the boundary of the Earth's atmosphere and the beginning of space, as recognized by the FCC. This definition ensures that amateur radio operators have clear guidelines for operating in space, which is crucial for coordination and compliance with international space regulations.