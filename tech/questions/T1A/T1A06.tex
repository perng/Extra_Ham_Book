\subsection{FCC Part 97 Definition of a Beacon}
\label{T1A06}

\begin{tcolorbox}[colback=gray!10!white,colframe=black!75!black,title=T1A06]
What is the FCC Part 97 definition of a beacon?
\begin{enumerate}[label=\Alph*)]
    \item A government transmitter marking the amateur radio band edges
    \item A bulletin sent by the FCC to announce a national emergency
    \item A continuous transmission of weather information authorized in the amateur bands by the National Weather Service
    \item \textbf{An amateur station transmitting communications for the purposes of observing propagation or related experimental activities}
\end{enumerate}
\end{tcolorbox}

\subsubsection{Explanation}
In the context of FCC Part 97, a beacon is defined as an amateur station that transmits signals specifically for the purpose of observing propagation characteristics or conducting related experimental activities. This is crucial for understanding how radio waves propagate through different mediums and under various atmospheric conditions. The beacon's continuous transmission allows operators to study signal strength, frequency stability, and other propagation phenomena. This experimental data is invaluable for advancing the science of radio communication.

% Diagram prompt: A diagram showing a beacon station transmitting signals and how they propagate through different atmospheric layers could be useful here.