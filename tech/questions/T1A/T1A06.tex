\subsection{Definition of a Beacon} \label{T1A06}

\begin{tcolorbox}[colback=gray!10!white,colframe=black!75!black,title=T1A06]
What is the FCC Part 97 definition of a beacon?
\begin{enumerate}[label=\Alph*),noitemsep]
    \item A government transmitter marking the amateur radio band edges
    \item A bulletin sent by the FCC to announce a national emergency
    \item A continuous transmission of weather information authorized in the amateur bands by the National Weather Service
    \item \textbf{An amateur station transmitting communications for the purposes of observing propagation or related experimental activities}
\end{enumerate}
\end{tcolorbox}

\subsubsection{Intuitive Explanation}
Imagine you're trying to figure out how far your voice can travel in a big, open field. You might shout and listen for echoes to see how far the sound goes. In the world of amateur radio, a beacon is like that shout. It's a special signal sent out by amateur radio operators to study how radio waves travel through the air. This helps them understand things like how weather or the time of day affects radio communication.

\subsubsection{Advanced Explanation}
In the context of FCC Part 97, a beacon is defined as an amateur station that transmits signals specifically for the purpose of observing propagation characteristics or conducting related experimental activities. This means that the primary function of a beacon is to provide data on how radio waves propagate under various conditions, which can be influenced by factors such as atmospheric conditions, frequency, and time of day. Beacons are crucial for amateur radio operators who are interested in understanding and experimenting with radio wave behavior, which can enhance their ability to communicate over long distances.