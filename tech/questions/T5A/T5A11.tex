\subsection{Type of Current Flow Opposed by Resistance}
\label{T5A11}

\begin{tcolorbox}[colback=gray!10!white,colframe=black!75!black,title=T5A11]
What type of current flow is opposed by resistance?
\begin{enumerate}[label=\Alph*]
    \item Direct current
    \item Alternating current
    \item RF current
    \item \textbf{All these choices are correct}
\end{enumerate}
\end{tcolorbox}

\subsubsection{Intuitive Explanation}
Imagine you're trying to push a shopping cart through a crowded store. Whether you're pushing it straight ahead (direct current), back and forth (alternating current), or even wiggling it around (RF current), the people in the store (resistance) are going to make it harder for you to move the cart. Resistance doesn't care which way you're pushing—it just makes everything more difficult!

\subsubsection{Advanced Explanation}
Resistance is a property of a material that opposes the flow of electric current, regardless of the type of current. Mathematically, resistance \( R \) is defined by Ohm's Law:
\[
V = IR
\]
where \( V \) is the voltage, \( I \) is the current, and \( R \) is the resistance. This relationship holds true for direct current (DC), alternating current (AC), and radio frequency (RF) current. 

In DC, the current flows in one direction, and resistance opposes this flow. In AC, the current changes direction periodically, but resistance still opposes the flow at every instant. RF current, which is a high-frequency AC, also experiences opposition from resistance. Therefore, resistance opposes all types of current flow.

% Prompt for diagram: A diagram showing a simple circuit with a resistor and different types of current (DC, AC, RF) flowing through it, illustrating how resistance opposes each type.