\subsection{Unit of Frequency}
\label{T5A06}

\begin{tcolorbox}[colback=gray!10!white,colframe=black!75!black,title=T5A06]
What is the unit of frequency?
\begin{enumerate}[label=\Alph*)]
    \item \textbf{Hertz}
    \item Henry
    \item Farad
    \item Tesla
\end{enumerate}
\end{tcolorbox}

\subsubsection{Intuitive Explanation}
Imagine you're at a concert, and the band is playing a song. The number of times the drummer hits the drum in one second is like the frequency of the sound. The unit we use to measure how many times something happens in a second is called Hertz (Hz). So, if the drummer hits the drum 5 times in a second, the frequency is 5 Hz. Easy, right?

\subsubsection{Advanced Explanation}
Frequency is a fundamental concept in physics and engineering, particularly in the study of waves and oscillations. It is defined as the number of cycles of a periodic event that occur in one second. The unit of frequency is the Hertz (Hz), named after the German physicist Heinrich Hertz. Mathematically, frequency \( f \) is related to the period \( T \) of the wave by the equation:

\[
f = \frac{1}{T}
\]

For example, if a wave has a period of 0.02 seconds, its frequency is:

\[
f = \frac{1}{0.02} = 50 \text{ Hz}
\]

Other units mentioned in the choices, such as Henry (H), Farad (F), and Tesla (T), are units of inductance, capacitance, and magnetic flux density, respectively, and are not related to frequency.

% Diagram prompt: A simple sine wave with labeled period and frequency could help visualize the concept.