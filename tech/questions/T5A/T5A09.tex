\subsection{Alternating Current Description}
\label{T5A09}

\begin{tcolorbox}[colback=gray!10!white,colframe=black!75!black,title=T5A09]
Which of the following describes alternating current?
\begin{enumerate}[label=\Alph*]
    \item Current that alternates between a positive direction and zero
    \item Current that alternates between a negative direction and zero
    \item \textbf{Current that alternates between positive and negative directions}
    \item All these answers are correct
\end{enumerate}
\end{tcolorbox}

\subsubsection{Intuitive Explanation}
Imagine you’re playing a game of tug-of-war, but instead of pulling the rope in one direction, you keep switching sides. First, you pull to the left, then to the right, and you keep doing this back and forth. That’s what alternating current (AC) does with electricity! It’s like the electricity is playing a never-ending game of tug-of-war, switching directions constantly. So, the correct answer is the one that says the current alternates between positive and negative directions—just like your tug-of-war game!

\subsubsection{Advanced Explanation}
Alternating current (AC) is a type of electrical current where the flow of electric charge periodically reverses direction. This is in contrast to direct current (DC), where the flow of charge is unidirectional. The mathematical representation of AC is typically a sinusoidal function:

\[
I(t) = I_{\text{max}} \sin(2\pi ft)
\]

where:
\begin{itemize}
    \item \( I(t) \) is the current at time \( t \),
    \item \( I_{\text{max}} \) is the maximum current,
    \item \( f \) is the frequency of the AC, and
    \item \( t \) is time.
\end{itemize}

The frequency \( f \) determines how many times the current changes direction per second. For example, in the United States, the standard frequency for AC is 60 Hz, meaning the current changes direction 120 times per second (60 positive and 60 negative cycles).

The key concept here is that AC alternates between positive and negative values, which is why the correct answer is \textbf{C}. This alternating nature allows AC to be easily transformed to different voltages using transformers, making it more efficient for long-distance power transmission compared to DC.

% Prompt for generating a diagram: A sine wave graph showing the alternating current over time, with positive and negative peaks labeled.