\subsection{Description of Alternating Current}
\label{T5A09}

\begin{tcolorbox}[colback=gray!10!white,colframe=black!75!black,title=T5A09]
Which of the following describes alternating current?
\begin{enumerate}[noitemsep]
    \item Current that alternates between a positive direction and zero
    \item Current that alternates between a negative direction and zero
    \item \textbf{Current that alternates between positive and negative directions}
    \item All these answers are correct
\end{enumerate}
\end{tcolorbox}

Alternating current (AC) is a type of electrical current that periodically reverses direction. Unlike direct current (DC), which flows in a single direction, AC alternates between positive and negative directions. This characteristic is essential for the efficient transmission of electrical power over long distances. The correct answer is \textbf{C}, as it accurately describes the nature of alternating current.