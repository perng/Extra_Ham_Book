\subsection{Conductivity of Metals}
\label{T5A07}

\begin{tcolorbox}[colback=gray!10!white,colframe=black!75!black,title=T5A07]
Why are metals generally good conductors of electricity?
\begin{enumerate}[label=\Alph*)]
    \item They have relatively high density
    \item \textbf{They have many free electrons}
    \item They have many free protons
    \item All these choices are correct
\end{enumerate}
\end{tcolorbox}

\subsubsection{Intuitive Explanation}
Imagine you're at a concert, and everyone is packed tightly together. If someone starts a wave, it travels quickly through the crowd because everyone is close and ready to pass it along. Metals are like that concert crowd, but instead of people, they have lots of free electrons. These electrons are like the wave, moving easily through the metal, which is why metals are great at conducting electricity. It's not about how heavy the metal is (density) or having free protons (which don't move around like electrons), it's all about those free electrons!

\subsubsection{Advanced Explanation}
Metals are characterized by their metallic bonding, where atoms are arranged in a lattice structure and share a sea of delocalized electrons. These electrons are not bound to any specific atom and are free to move throughout the metal. When an electric field is applied, these free electrons drift in the direction opposite to the field, creating an electric current. The high conductivity of metals is due to this abundance of free electrons, which can move with minimal resistance.

The density of a metal (Choice A) does not directly affect its conductivity. While higher density might imply more atoms and potentially more electrons, it is the mobility of these electrons that determines conductivity. Free protons (Choice C) do not exist in metals; protons are bound within atomic nuclei and do not contribute to electrical conduction. Therefore, the correct answer is B: metals have many free electrons.

% Prompt for diagram: A diagram showing the lattice structure of a metal with delocalized electrons moving freely through the lattice could help visualize the concept of metallic bonding and conductivity.