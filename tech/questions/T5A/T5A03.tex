\subsection{Flow of Electrons in a Circuit}
\label{T5A03}

\begin{tcolorbox}[colback=gray!10!white,colframe=black!75!black,title=T5A03]
What is the name for the flow of electrons in an electric circuit?
\begin{enumerate}[label=\Alph*)]
    \item Voltage
    \item Resistance
    \item Capacitance
    \item \textbf{Current}
\end{enumerate}
\end{tcolorbox}

\subsubsection{Intuitive Explanation}
Imagine you have a water pipe. The water flowing through the pipe is like the electrons moving in a wire. The flow of water is called current in the pipe, and similarly, the flow of electrons in a wire is called current in an electric circuit. So, when you turn on a light, it's like opening a tap, and the electrons (water) start flowing, making the light (the tap) work!

\subsubsection{Advanced Explanation}
In an electric circuit, the movement of electrons constitutes an electric current. The current \( I \) is defined as the rate of flow of electric charge \( Q \) through a cross-sectional area of a conductor. Mathematically, it is expressed as:

\[
I = \frac{dQ}{dt}
\]

where \( I \) is the current in amperes (A), \( Q \) is the charge in coulombs (C), and \( t \) is the time in seconds (s). 

The flow of electrons is driven by an electric potential difference, commonly referred to as voltage. Resistance \( R \) opposes the flow of current, and capacitance \( C \) stores electric charge. However, the term that specifically describes the flow of electrons is \textbf{current}.

% Diagram prompt: Generate a diagram showing a simple circuit with a battery, a resistor, and arrows indicating the flow of electrons.