\subsection{Rate of Electrical Energy Usage}
\label{T5A10}

\begin{tcolorbox}[colback=gray!10!white,colframe=black!75!black,title=T5A10]
Which term describes the rate at which electrical energy is used?
\begin{enumerate}[label=\Alph*)]
    \item Resistance
    \item Current
    \item \textbf{Power}
    \item Voltage
\end{enumerate}
\end{tcolorbox}

\subsubsection{Intuitive Explanation}
Imagine you have a light bulb in your room. When you turn it on, it starts to glow, right? But have you ever wondered how fast it's using the electricity to make that light? That's where the term power comes in. Power is like the speedometer for electricity—it tells you how quickly the electrical energy is being used. So, if you want to know how fast your light bulb is gobbling up electricity, you're talking about power!

\subsubsection{Advanced Explanation}
In electrical systems, power is defined as the rate at which electrical energy is transferred by an electric circuit. Mathematically, power \( P \) is given by the product of voltage \( V \) and current \( I \):

\[
P = V \times I
\]

Where:
\begin{itemize}
    \item \( P \) is the power in watts (W),
    \item \( V \) is the voltage in volts (V),
    \item \( I \) is the current in amperes (A).
\end{itemize}

For example, if a device operates at 120 volts and draws a current of 2 amperes, the power consumed would be:

\[
P = 120 \, \text{V} \times 2 \, \text{A} = 240 \, \text{W}
\]

This means the device is using electrical energy at a rate of 240 watts. Understanding power is crucial for designing and analyzing electrical systems, as it helps in determining the efficiency and capacity of various components.

% Diagram Prompt: Generate a diagram showing a simple circuit with a voltage source, a resistor, and labels for voltage, current, and power.