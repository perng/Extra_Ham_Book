\subsection{Good Electrical Insulator}
\label{T5A08}

\begin{tcolorbox}[colback=gray!10!white,colframe=black!75!black,title=T5A08]
Which of the following is a good electrical insulator?
\begin{enumerate}[label=\Alph*)]
    \item Copper
    \item \textbf{Glass}
    \item Aluminum
    \item Mercury
\end{enumerate}
\end{tcolorbox}

\subsubsection{Intuitive Explanation}
Imagine you're trying to stop a bunch of tiny, invisible electric ants from moving through different materials. If you use copper or aluminum, it's like giving them a super highway—they can zip right through! Mercury is like a slippery slide, still easy for them to move. But glass? That's like a giant wall. The ants can't get through at all! So, glass is the best at stopping these electric ants, making it a great insulator.

\subsubsection{Advanced Explanation}
Electrical insulators are materials that resist the flow of electric current. This resistance is due to the material's atomic structure, which does not allow free movement of electrons. In the case of glass, its molecular structure consists of tightly bound electrons that are not free to move, making it an excellent insulator. 

In contrast, materials like copper and aluminum are conductors because they have free electrons that can move easily, allowing electric current to flow. Mercury, although a liquid, is also a conductor due to its metallic nature and free electrons. 

The key concept here is the band gap in materials. Insulators have a large band gap, which means that electrons require a significant amount of energy to move from the valence band to the conduction band. In glass, this band gap is very large, preventing electron flow under normal conditions. Conductors, on the other hand, have overlapping or very small band gaps, facilitating electron movement.

% Prompt for diagram: A diagram showing the band gap of insulators, conductors, and semiconductors could be helpful here.