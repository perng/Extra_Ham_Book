\subsection{Units of Electrical Current}
\label{T5A01}

\begin{tcolorbox}[colback=gray!10!white,colframe=black!75!black,title=T5A01]
Electrical current is measured in which of the following units?
\begin{enumerate}[label=\Alph*)]
    \item Volts
    \item Watts
    \item Ohms
    \item \textbf{Amperes}
\end{enumerate}
\end{tcolorbox}

\subsubsection*{Intuitive Explanation}
Imagine electricity as water flowing through a pipe. The amount of water flowing through the pipe is like the electrical current. Just like we measure water flow in liters per second, we measure electrical current in Amperes (or Amps for short). So, when someone asks how much electricity is flowing, they're really asking how many Amperes are moving through the wire. Easy, right?

\subsubsection*{Advanced Explanation}
Electrical current, denoted by the symbol \( I \), is the rate at which electric charge flows through a conductor. The SI unit for electric current is the Ampere (A), named after the French physicist André-Marie Ampère. One Ampere is defined as one Coulomb of charge passing through a point in a circuit per second. Mathematically, this is expressed as:

\[
I = \frac{Q}{t}
\]

where:
\begin{itemize}
    \item \( I \) is the current in Amperes (A),
    \item \( Q \) is the charge in Coulombs (C),
    \item \( t \) is the time in seconds (s).
\end{itemize}

Volts (V) measure electrical potential difference, Watts (W) measure power, and Ohms (\(\Omega\)) measure resistance. These are all related but distinct concepts in electrical engineering. Understanding these units is fundamental to analyzing and designing electrical circuits.

% Prompt for generating a diagram: A simple circuit diagram showing a battery, a resistor, and an ammeter measuring the current in Amperes.