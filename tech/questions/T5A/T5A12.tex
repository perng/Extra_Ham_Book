\subsection{Frequency of Alternating Current}
\label{T5A12}

\begin{tcolorbox}[colback=gray!10!white,colframe=black!75!black,title=T5A12]
What describes the number of times per second that an alternating current makes a complete cycle?
\begin{enumerate}[label=\Alph*)]
    \item Pulse rate
    \item Speed
    \item Wavelength
    \item \textbf{Frequency}
\end{enumerate}
\end{tcolorbox}

\subsubsection{Intuitive Explanation}
Imagine you're on a swing, going back and forth. Each time you swing all the way forward and then all the way back, that's one complete cycle. Now, if you count how many times you do that in one second, that's like the frequency of your swinging. In the world of electricity, alternating current (AC) is like that swing, going back and forth really fast. The frequency tells us how many times the current completes this back-and-forth cycle in one second. So, the correct answer is \textbf{Frequency} because it’s all about counting those cycles per second!

\subsubsection{Advanced Explanation}
In alternating current (AC), the flow of electric charge periodically reverses direction. The frequency of an AC signal is defined as the number of complete cycles it completes per second, measured in Hertz (Hz). Mathematically, frequency \( f \) is related to the period \( T \) (the time it takes to complete one cycle) by the equation:

\[
f = \frac{1}{T}
\]

For example, if an AC signal completes one cycle in 0.02 seconds, its frequency is:

\[
f = \frac{1}{0.02} = 50 \text{ Hz}
\]

Frequency is a fundamental concept in AC circuits and is crucial for understanding how electrical systems operate, especially in power distribution and signal processing. The other options, such as pulse rate, speed, and wavelength, are not directly related to the number of cycles per second in AC.

% Diagram prompt: A simple sine wave with labeled cycles and time axis to illustrate frequency.