\subsection{Units of Electrical Power}
\label{T5A02}

\begin{tcolorbox}[colback=gray!10!white,colframe=black!75!black,title=T5A02]
Electrical power is measured in which of the following units?
\begin{enumerate}[label=\Alph*)]
    \item Volts
    \item \textbf{Watts}
    \item Watt-hours
    \item Amperes
\end{enumerate}
\end{tcolorbox}

\subsubsection{Intuitive Explanation}
Imagine you have a light bulb. When you turn it on, it uses electricity to produce light. The amount of electricity it uses is called power. Just like you measure how fast a car is going in miles per hour, we measure power in a unit called Watts. So, if someone asks you how much power your light bulb is using, you would say it's using so many Watts. Easy, right?

\subsubsection{Advanced Explanation}
Electrical power is the rate at which electrical energy is transferred by an electric circuit. The SI unit of power is the Watt (W), which is defined as one joule per second. Mathematically, power \( P \) can be expressed as:

\[
P = V \times I
\]

where \( V \) is the voltage in volts (V) and \( I \) is the current in amperes (A). 

For example, if a device operates at a voltage of 10 volts and draws a current of 2 amperes, the power consumed by the device is:

\[
P = 10 \, \text{V} \times 2 \, \text{A} = 20 \, \text{W}
\]

This means the device is using 20 Watts of power. Understanding this relationship is crucial for designing and analyzing electrical circuits.

% Prompt for generating a diagram: A simple circuit diagram showing a voltage source, a resistor, and the power calculation formula P = V * I.