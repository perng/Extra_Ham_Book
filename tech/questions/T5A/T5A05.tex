\subsection{Force Causing Electron Flow}
\label{T5A05}

\begin{tcolorbox}[colback=gray!10!white,colframe=black!75!black,title=T5A05]
What is the electrical term for the force that causes electron flow?
\begin{enumerate}[label=\Alph*)]
    \item \textbf{Voltage}
    \item Ampere-hours
    \item Capacitance
    \item Inductance
\end{enumerate}
\end{tcolorbox}

\subsubsection{Intuitive Explanation}
Imagine you have a water slide. The water (which is like the electrons) flows down the slide because of the height difference (which is like the force pushing the electrons). In electricity, this push that makes the electrons move is called \textbf{voltage}. It's like the oomph that gets the electrons going!

\subsubsection{Advanced Explanation}
In electrical terms, the force that causes electron flow is known as \textbf{voltage}, or more formally, electric potential difference. Voltage is measured in volts (V) and is defined as the work done per unit charge to move a charge between two points in an electric field. Mathematically, it is expressed as:

\[
V = \frac{W}{q}
\]

where \( V \) is the voltage, \( W \) is the work done, and \( q \) is the charge. 

Voltage is a fundamental concept in electricity and is essential for understanding how electrical circuits work. It is the driving force that causes electrons to move through a conductor, creating an electric current. Other terms like ampere-hours, capacitance, and inductance are related to different aspects of electrical systems but do not describe the force that causes electron flow.

% Diagram prompt: Generate a diagram showing a simple circuit with a battery (representing voltage) and a resistor, illustrating the flow of electrons due to the voltage.