\subsection{Units of Electrical Resistance}
\label{T5A04}

\begin{tcolorbox}[colback=gray!10!white,colframe=black!75!black,title=T5A04]
What are the units of electrical resistance?
\begin{enumerate}[label=\Alph*)]
    \item Siemens
    \item Mhos
    \item \textbf{Ohms}
    \item Coulombs
\end{enumerate}
\end{tcolorbox}

\subsubsection{Intuitive Explanation}
Imagine you're trying to push a toy car through a sandbox. The sand makes it harder to push the car, right? That's kind of like electrical resistance—it's what makes it harder for electricity to flow through a wire. The unit we use to measure this push-back is called the Ohm. So, when someone asks about the units of electrical resistance, they're asking, What do we call the measure of how much something resists electricity? And the answer is Ohms!

\subsubsection{Advanced Explanation}
Electrical resistance is a fundamental concept in electrical engineering and physics, defined as the opposition to the flow of electric current through a conductor. The unit of resistance is the Ohm, symbolized by the Greek letter $\Omega$. The relationship between voltage ($V$), current ($I$), and resistance ($R$) is given by Ohm's Law:

\[
V = I \times R
\]

From this equation, we can see that resistance is directly proportional to voltage and inversely proportional to current. The Ohm is defined as the resistance between two points of a conductor when a constant potential difference of 1 volt, applied to these points, produces in the conductor a current of 1 ampere.

Other units mentioned in the choices, such as Siemens and Mhos, are actually units of electrical conductance, which is the reciprocal of resistance. Coulombs, on the other hand, are units of electric charge, not resistance. Therefore, the correct answer is Ohms.

% Diagram prompt: A simple circuit diagram showing a resistor with the label R and the unit Ohms next to it, connected to a voltage source and an ammeter to illustrate Ohm's Law.