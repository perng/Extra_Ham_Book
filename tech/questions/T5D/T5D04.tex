\subsection{Resistance Calculation}
\label{T5D04}

\begin{tcolorbox}[colback=gray!10!white,colframe=black!75!black,title=T5D04]
What is the resistance of a circuit in which a current of 3 amperes flows when connected to 90 volts?
\begin{enumerate}[label=\Alph*)]
    \item 3 ohms
    \item \textbf{30 ohms}
    \item 93 ohms
    \item 270 ohms
\end{enumerate}
\end{tcolorbox}

\subsubsection{Intuitive Explanation}
Imagine you have a water hose connected to a pump. The pump is pushing water (voltage) through the hose, and the amount of water flowing (current) is 3 liters per second. Now, if the pump is pushing at 90 units of pressure, how much is the hose resisting the flow? Well, if you divide the pressure by the flow rate, you get the resistance. So, 90 divided by 3 is 30. The hose is resisting the flow by 30 units. Easy peasy!

\subsubsection{Advanced Explanation}
To determine the resistance in a circuit, we use Ohm's Law, which is given by the equation:
\[
V = I \times R
\]
where \( V \) is the voltage, \( I \) is the current, and \( R \) is the resistance. Rearranging the equation to solve for resistance, we get:
\[
R = \frac{V}{I}
\]
Given the values \( V = 90 \) volts and \( I = 3 \) amperes, we can substitute these into the equation:
\[
R = \frac{90}{3} = 30 \, \text{ohms}
\]
Thus, the resistance of the circuit is 30 ohms.

Ohm's Law is fundamental in understanding how voltage, current, and resistance interact in an electrical circuit. It is essential for designing and analyzing circuits in various applications, from simple electronics to complex power systems.

% Diagram Prompt: Generate a simple circuit diagram showing a voltage source, a resistor, and the current flow labeled with the given values (90V, 3A, 30Ω).