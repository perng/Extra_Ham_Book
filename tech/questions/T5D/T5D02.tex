\subsection{Voltage Calculation Formula}
\label{T5D02}

\begin{tcolorbox}[colback=gray!10!white,colframe=black!75!black,title=T5D02]
What formula is used to calculate voltage in a circuit?
\begin{enumerate}[label=\Alph*)]
    \item \textbf{E = I x R}
    \item E = I / R
    \item E = I + R
    \item E = I - R
\end{enumerate}
\end{tcolorbox}

\subsubsection{Intuitive Explanation}
Imagine you're trying to push a toy car across the floor. The harder you push (that's the current, I), and the more friction there is (that's the resistance, R), the more effort you need to use (that's the voltage, E). So, the voltage is like the effort you need to push the car, and it’s calculated by multiplying how hard you push by how much friction there is. That’s why the formula is E = I x R!

\subsubsection{Advanced Explanation}
In electrical circuits, voltage (E) is the potential difference that drives the flow of electric current (I) through a conductor. The relationship between voltage, current, and resistance (R) is described by Ohm's Law, which states:

\[
E = I \times R
\]

Here, \(E\) is the voltage in volts (V), \(I\) is the current in amperes (A), and \(R\) is the resistance in ohms (\(\Omega\)). This formula is fundamental in circuit analysis and is used to determine the voltage across a component when the current and resistance are known.

For example, if a circuit has a current of 2 A and a resistance of 3 \(\Omega\), the voltage can be calculated as:

\[
E = 2 \, \text{A} \times 3 \, \Omega = 6 \, \text{V}
\]

Ohm's Law is essential for understanding how electrical circuits behave and is widely used in designing and analyzing electronic systems.

% Prompt for generating a diagram: A simple circuit diagram showing a voltage source, a resistor, and the current flow, labeled with E, I, and R.