\subsection{Voltage in Circuit Types}
\label{T5D14}

\begin{tcolorbox}[colback=gray!10!white,colframe=black!75!black,title=T5D14]
In which type of circuit is voltage the same across all components?
\begin{enumerate}[label=\Alph*]
    \item Series
    \item \textbf{Parallel}
    \item Resonant
    \item Branch
\end{enumerate}
\end{tcolorbox}

\subsubsection{Intuitive Explanation}
Imagine you have a bunch of light bulbs connected to a battery. If you connect them in parallel, it's like giving each bulb its own direct connection to the battery. So, each bulb gets the same amount of push (voltage) from the battery, just like how each kid in a group gets their own slice of pizza. In a series circuit, it's more like passing the pizza around, and each kid gets a smaller piece. So, in parallel circuits, the voltage is the same across all components because everyone gets their own full slice!

\subsubsection{Advanced Explanation}
In a parallel circuit, all components are connected across the same two points, effectively creating multiple paths for the current to flow. According to Kirchhoff's Voltage Law (KVL), the voltage across each branch of a parallel circuit must be equal. This is because the potential difference between the two common points is the same for all branches.

Mathematically, if \( V \) is the voltage across the entire parallel circuit, then for each component \( i \), the voltage \( V_i \) is:
\[
V_i = V
\]
This equality holds true regardless of the resistance or impedance of each component. In contrast, in a series circuit, the voltage is divided among the components based on their resistances, leading to different voltages across each component.

Understanding this concept is crucial for designing circuits where consistent voltage across components is necessary, such as in household wiring or electronic devices.

% Diagram prompt: Generate a diagram showing a simple parallel circuit with a battery and three resistors, labeling the voltage across each resistor as equal.