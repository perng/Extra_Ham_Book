\subsection{Current Through a Resistor}
\label{T5D08}

\begin{tcolorbox}[colback=gray!10!white,colframe=black!75!black,title=T5D08]
What is the current through a 100-ohm resistor connected across 200 volts?
\begin{enumerate}[label=\Alph*)]
    \item 20,000 amperes
    \item 0.5 amperes
    \item \textbf{2 amperes}
    \item 100 amperes
\end{enumerate}
\end{tcolorbox}

\subsubsection{Intuitive Explanation}
Imagine you have a water pipe with a certain amount of water pressure (voltage) and a narrow section that resists the flow of water (resistor). The amount of water flowing through the pipe (current) depends on how much pressure you have and how narrow the pipe is. In this case, you have a lot of pressure (200 volts) and a moderately narrow pipe (100 ohms). The water flows at a rate of 2 liters per second (2 amperes). So, the current is 2 amperes!

\subsubsection{Advanced Explanation}
To determine the current through a resistor, we use Ohm's Law, which states:
\[
V = I \times R
\]
where \( V \) is the voltage, \( I \) is the current, and \( R \) is the resistance. Rearranging the formula to solve for current:
\[
I = \frac{V}{R}
\]
Given:
\[
V = 200 \text{ volts}, \quad R = 100 \text{ ohms}
\]
Substituting the values:
\[
I = \frac{200}{100} = 2 \text{ amperes}
\]
Thus, the current through the resistor is 2 amperes.

Ohm's Law is fundamental in understanding how voltage, current, and resistance interact in electrical circuits. It is essential for designing and analyzing circuits in various applications, from simple electronics to complex power systems.

% Diagram prompt: Generate a simple circuit diagram showing a 200V battery connected to a 100-ohm resistor with an ammeter in series to measure the current.