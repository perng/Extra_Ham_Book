\subsection{Voltage Across a Resistor}
\label{T5D11}

\begin{tcolorbox}[colback=gray!10!white,colframe=black!75!black,title=T5D11]
What is the voltage across a 10-ohm resistor if a current of 1 ampere flows through it?
\begin{enumerate}[label=\Alph*)]
    \item 1 volt
    \item \textbf{10 volts}
    \item 11 volts
    \item 9 volts
\end{enumerate}
\end{tcolorbox}

\subsubsection{Intuitive Explanation}
Imagine the resistor is like a narrow pipe, and the current is water flowing through it. The narrower the pipe (higher resistance), the more pressure (voltage) you need to push the water through. If you have a pipe that’s 10 units narrow (10 ohms) and you’re pushing 1 unit of water (1 ampere) through it, you’ll need 10 units of pressure (10 volts) to keep the water flowing. So, the voltage across the resistor is 10 volts!

\subsubsection{Advanced Explanation}
To determine the voltage across a resistor, we use Ohm's Law, which states:
\[
V = I \times R
\]
where:
\begin{itemize}
    \item \( V \) is the voltage in volts (V),
    \item \( I \) is the current in amperes (A),
    \item \( R \) is the resistance in ohms (\(\Omega\)).
\end{itemize}

Given:
\[
I = 1 \, \text{A}, \quad R = 10 \, \Omega
\]

Substituting the values into Ohm's Law:
\[
V = 1 \, \text{A} \times 10 \, \Omega = 10 \, \text{V}
\]

Thus, the voltage across the resistor is 10 volts. Ohm's Law is fundamental in understanding the relationship between voltage, current, and resistance in electrical circuits. It is essential for analyzing and designing circuits in various applications.

% Diagram Prompt: A simple circuit diagram showing a resistor with a current of 1 A and a voltage of 10 V across it.