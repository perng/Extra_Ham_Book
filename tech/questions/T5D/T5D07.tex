\subsection{Current in a Circuit with Given Voltage and Resistance}
\label{T5D07}

\begin{tcolorbox}[colback=gray!10!white,colframe=black!75!black,title=T5D07]
What is the current in a circuit with an applied voltage of 120 volts and a resistance of 80 ohms?
\begin{enumerate}[label=\Alph*)]
    \item 9600 amperes
    \item 200 amperes
    \item 0.667 amperes
    \item \textbf{1.5 amperes}
\end{enumerate}
\end{tcolorbox}

\subsubsection{Intuitive Explanation}
Imagine you have a water hose. The voltage is like the water pressure, and the resistance is like how narrow the hose is. If you have high pressure (120 volts) and a not-too-narrow hose (80 ohms), the water (current) will flow at a certain speed. In this case, the current is 1.5 amperes, which is like saying the water is flowing at a moderate speed. So, the correct answer is D, 1.5 amperes.

\subsubsection{Advanced Explanation}
To determine the current in a circuit, we use Ohm's Law, which is given by:
\[
I = \frac{V}{R}
\]
where \( I \) is the current in amperes, \( V \) is the voltage in volts, and \( R \) is the resistance in ohms.

Given:
\[
V = 120 \text{ volts}, \quad R = 80 \text{ ohms}
\]

Substitute the values into Ohm's Law:
\[
I = \frac{120}{80} = 1.5 \text{ amperes}
\]

Thus, the current in the circuit is 1.5 amperes, which corresponds to answer D.

Ohm's Law is fundamental in understanding how voltage, current, and resistance interact in an electrical circuit. It is essential for analyzing and designing circuits in various applications.

% Diagram prompt: A simple circuit diagram showing a voltage source (120V), a resistor (80 ohms), and the current (1.5A) flowing through the circuit.