\subsection{Voltage Across a Resistor}
\label{T5D12}

\begin{tcolorbox}[colback=gray!10!white,colframe=black!75!black,title=T5D12]
What is the voltage across a 10-ohm resistor if a current of 2 amperes flows through it?
\begin{enumerate}[noitemsep]
    \item 8 volts
    \item 0.2 volts
    \item 12 volts
    \item \textbf{20 volts}
\end{enumerate}
\end{tcolorbox}

\subsubsection*{Intuitive Explanation}
Imagine the resistor as a narrow pipe and the current as water flowing through it. The voltage is like the pressure pushing the water through the pipe. If you have a 10-ohm resistor (a specific type of pipe) and a current of 2 amperes (a certain amount of water flowing), the voltage (pressure) needed can be calculated using Ohm's Law, which is simply voltage equals current times resistance.

\subsubsection*{Advanced Explanation}
Ohm's Law states that the voltage \( V \) across a resistor is equal to the current \( I \) flowing through it multiplied by the resistance \( R \) of the resistor. Mathematically, this is expressed as:
\[
V = I \times R
\]
Given:
\[
I = 2 \, \text{A}, \quad R = 10 \, \Omega
\]
Substituting the values into Ohm's Law:
\[
V = 2 \, \text{A} \times 10 \, \Omega = 20 \, \text{V}
\]
Therefore, the voltage across the resistor is 20 volts.