\subsection{Voltage Across a Resistor}
\label{T5D12}

\begin{tcolorbox}[colback=gray!10!white,colframe=black!75!black,title=T5D12]
What is the voltage across a 10-ohm resistor if a current of 2 amperes flows through it?
\begin{enumerate}[label=\Alph*)]
    \item 8 volts
    \item 0.2 volts
    \item 12 volts
    \item \textbf{20 volts}
\end{enumerate}
\end{tcolorbox}

\subsubsection{Intuitive Explanation}
Imagine you have a water hose (the resistor) and water (the current) is flowing through it. The hose has a certain thickness (resistance) that makes it harder for the water to flow. If you know how much water is flowing and how thick the hose is, you can figure out how much pressure (voltage) is needed to push the water through. In this case, the hose is 10 ohms thick, and 2 amperes of water is flowing. So, the pressure needed is 20 volts. Easy peasy!

\subsubsection{Advanced Explanation}
To determine the voltage across a resistor, we use Ohm's Law, which states:
\[
V = I \times R
\]
where:
\begin{itemize}
    \item \( V \) is the voltage in volts (V),
    \item \( I \) is the current in amperes (A),
    \item \( R \) is the resistance in ohms (\(\Omega\)).
\end{itemize}

Given:
\[
I = 2 \, \text{A}, \quad R = 10 \, \Omega
\]

Substituting the values into Ohm's Law:
\[
V = 2 \, \text{A} \times 10 \, \Omega = 20 \, \text{V}
\]

Thus, the voltage across the resistor is 20 volts.

Ohm's Law is fundamental in electrical engineering and is used to relate voltage, current, and resistance in a circuit. Understanding this relationship is crucial for analyzing and designing electrical circuits.

% Diagram prompt: A simple circuit diagram showing a resistor with a current of 2A and a voltage of 20V across it.