\subsection{Voltage Across a Resistor}
\label{T5D10}

\begin{tcolorbox}[colback=gray!10!white,colframe=black!75!black,title=T5D10]
What is the voltage across a 2-ohm resistor if a current of 0.5 amperes flows through it?
\begin{enumerate}[label=\Alph*)]
    \item \textbf{1 volt}
    \item 0.25 volts
    \item 2.5 volts
    \item 1.5 volts
\end{enumerate}
\end{tcolorbox}

\subsubsection{Intuitive Explanation}
Imagine the resistor is like a narrow pipe, and the current is water flowing through it. The narrower the pipe (higher resistance), the harder it is for the water to flow. Now, if you have a pipe that's 2 units narrow (2 ohms) and water is flowing at 0.5 units per second (0.5 amperes), the pressure (voltage) needed to push the water through is just 1 unit (1 volt). So, the voltage across the resistor is 1 volt. Easy peasy!

\subsubsection{Advanced Explanation}
To determine the voltage across a resistor, we use Ohm's Law, which states:

\[
V = I \times R
\]

where:
\begin{itemize}
    \item \( V \) is the voltage in volts (V),
    \item \( I \) is the current in amperes (A),
    \item \( R \) is the resistance in ohms (\(\Omega\)).
\end{itemize}

Given:
\[
I = 0.5 \, \text{A}, \quad R = 2 \, \Omega
\]

Substituting the values into Ohm's Law:

\[
V = 0.5 \, \text{A} \times 2 \, \Omega = 1 \, \text{V}
\]

Thus, the voltage across the resistor is 1 volt.

Ohm's Law is fundamental in understanding the relationship between voltage, current, and resistance in electrical circuits. It is essential for analyzing and designing circuits in various applications.

% Diagram Prompt: A simple circuit diagram showing a resistor with a current of 0.5 A and a voltage of 1 V across it.