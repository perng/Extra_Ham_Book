\subsection{Current Through a Resistor}
\label{T5D09}

\begin{tcolorbox}[colback=gray!10!white,colframe=black!75!black,title=T5D09]
What is the current through a 24-ohm resistor connected across 240 volts?
\begin{enumerate}[noitemsep]
    \item 24,000 amperes
    \item 0.1 amperes
    \item \textbf{10 amperes}
    \item 216 amperes
\end{enumerate}
\end{tcolorbox}

\subsubsection*{Intuitive Explanation}
Imagine you have a water pipe with a certain resistance to the flow of water. If you increase the pressure (voltage) at one end, more water (current) will flow through the pipe. In this case, the pipe has a resistance of 24 ohms, and the pressure is 240 volts. Using Ohm's Law, we can calculate the current flowing through the resistor.

\subsubsection*{Advanced Explanation}
Ohm's Law states that the current \( I \) through a conductor between two points is directly proportional to the voltage \( V \) across the two points and inversely proportional to the resistance \( R \) of the conductor. Mathematically, Ohm's Law is expressed as:
\[
I = \frac{V}{R}
\]
Given:
\[
V = 240 \text{ volts}, \quad R = 24 \text{ ohms}
\]
Substituting the values into Ohm's Law:
\[
I = \frac{240}{24} = 10 \text{ amperes}
\]
Therefore, the current through the resistor is 10 amperes.