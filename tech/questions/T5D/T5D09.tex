\subsection{Current Through a Resistor}
\label{T5D09}

\begin{tcolorbox}[colback=gray!10!white,colframe=black!75!black,title=T5D09]
What is the current through a 24-ohm resistor connected across 240 volts?
\begin{enumerate}[label=\Alph*)]
    \item 24,000 amperes
    \item 0.1 amperes
    \item \textbf{10 amperes}
    \item 216 amperes
\end{enumerate}
\end{tcolorbox}

\subsubsection{Intuitive Explanation}
Imagine you have a water pipe with a certain amount of resistance (like a narrow section). If you push water (voltage) through it, the amount of water flowing (current) depends on how narrow the pipe is (resistance). In this case, you have a 24-ohm resistor (a moderately narrow pipe) and 240 volts (a strong push). The current (water flow) is 10 amperes, which is like a steady stream of water. So, the correct answer is 10 amperes.

\subsubsection{Advanced Explanation}
To determine the current through a resistor, we use Ohm's Law, which is given by:

\[
V = I \times R
\]

Where:
\begin{itemize}
    \item \( V \) is the voltage across the resistor (240 volts),
    \item \( I \) is the current through the resistor (unknown),
    \item \( R \) is the resistance of the resistor (24 ohms).
\end{itemize}

Rearranging the formula to solve for \( I \):

\[
I = \frac{V}{R}
\]

Substituting the given values:

\[
I = \frac{240 \text{ volts}}{24 \text{ ohms}} = 10 \text{ amperes}
\]

Thus, the current through the resistor is 10 amperes.

Ohm's Law is fundamental in understanding the relationship between voltage, current, and resistance in electrical circuits. It is essential for analyzing and designing circuits in various applications.

% Diagram prompt: A simple circuit diagram showing a 240V battery connected to a 24-ohm resistor with the current labeled as 10A.