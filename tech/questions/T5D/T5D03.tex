\subsection{Calculating Resistance in a Circuit}
\label{T5D03}

\begin{tcolorbox}[colback=gray!10!white,colframe=black!75!black,title=T5D03]
What formula is used to calculate resistance in a circuit?
\begin{enumerate}[label=\Alph*)]
    \item \( R = E \times I \)
    \item \textbf{\( R = E / I \)}
    \item \( R = E + I \)
    \item \( R = E - I \)
\end{enumerate}
\end{tcolorbox}

\subsubsection{Intuitive Explanation}
Imagine you're trying to push a shopping cart through a crowded store. The resistance you feel is like the resistance in a circuit. If you push harder (that's the voltage, \( E \)) and the cart moves faster (that's the current, \( I \)), the resistance \( R \) is how much the crowd is slowing you down. The formula \( R = E / I \) tells you that the resistance is the push divided by how fast the cart moves. So, if you push twice as hard and the cart moves twice as fast, the resistance stays the same. Cool, right?

\subsubsection{Advanced Explanation}
In electrical circuits, resistance \( R \) is a measure of how much a material opposes the flow of electric current. According to Ohm's Law, the relationship between voltage \( E \), current \( I \), and resistance \( R \) is given by:

\[
R = \frac{E}{I}
\]

Here, \( E \) is the voltage across the circuit, measured in volts (V), and \( I \) is the current flowing through the circuit, measured in amperes (A). The resistance \( R \) is then calculated in ohms (\(\Omega\)).

For example, if a circuit has a voltage of 12 volts and a current of 3 amperes, the resistance would be:

\[
R = \frac{12\, \text{V}}{3\, \text{A}} = 4\, \Omega
\]

This formula is fundamental in understanding how electrical circuits behave and is widely used in both theoretical and practical applications.

% Diagram Prompt: Generate a simple circuit diagram showing a voltage source, a resistor, and the current flow to visually represent Ohm's Law.