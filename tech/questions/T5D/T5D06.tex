\subsection{Resistance Calculation from Voltage and Current}
\label{T5D06}

\begin{tcolorbox}[colback=gray!10!white,colframe=black!75!black,title=T5D06]
What is the resistance of a circuit that draws 4 amperes from a 12-volt source?
\begin{enumerate}[label=\Alph*)]
    \item \textbf{3 ohms}
    \item 16 ohms
    \item 48 ohms
    \item 8 ohms
\end{enumerate}
\end{tcolorbox}

\subsubsection{Intuitive Explanation}
Imagine you have a water hose connected to a water tank. The water pressure (voltage) is 12 volts, and the water flow (current) is 4 amperes. The resistance is like the narrowness of the hose. If the hose is too narrow, it’s harder for the water to flow. In this case, the hose is just the right size to let 4 amperes flow with 12 volts of pressure, so the resistance is 3 ohms. It’s like saying, Hey, this hose is 3 ohms narrow!

\subsubsection{Advanced Explanation}
To calculate the resistance \( R \) of a circuit, we use Ohm's Law, which states:
\[
V = I \times R
\]
where \( V \) is the voltage, \( I \) is the current, and \( R \) is the resistance. Rearranging the formula to solve for \( R \):
\[
R = \frac{V}{I}
\]
Given:
\[
V = 12 \text{ volts}, \quad I = 4 \text{ amperes}
\]
Substituting the values:
\[
R = \frac{12 \text{ V}}{4 \text{ A}} = 3 \text{ ohms}
\]
Thus, the resistance of the circuit is 3 ohms.

Ohm's Law is fundamental in understanding how voltage, current, and resistance interact in electrical circuits. It is essential for designing and analyzing circuits in various applications, from simple household electronics to complex industrial systems.

% Diagram Prompt: Generate a simple circuit diagram showing a voltage source, a resistor, and the current flow labeled with the given values (12V, 4A, 3Ω).