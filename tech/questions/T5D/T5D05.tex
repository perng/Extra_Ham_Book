\subsection{Resistance Calculation in a Circuit}
\label{T5D05}

\begin{tcolorbox}[colback=gray!10!white,colframe=black!75!black,title=T5D05]
What is the resistance of a circuit for which the applied voltage is 12 volts and the current flow is 1.5 amperes?
\begin{enumerate}[label=\Alph*)]
    \item 18 ohms
    \item 0.125 ohms
    \item \textbf{8 ohms}
    \item 13.5 ohms
\end{enumerate}
\end{tcolorbox}

\subsubsection{Intuitive Explanation}
Imagine you're trying to push a cart through a hallway. The voltage is like how hard you're pushing, and the current is how fast the cart moves. The resistance is like how narrow or bumpy the hallway is. If you push with 12 units of force (volts) and the cart moves at 1.5 units of speed (amperes), the hallway must be 8 units of bumpiness (ohms). So, the resistance is 8 ohms!

\subsubsection{Advanced Explanation}
To find the resistance \( R \) in a circuit, we use Ohm's Law, which states:
\[
V = I \times R
\]
where \( V \) is the voltage, \( I \) is the current, and \( R \) is the resistance. Rearranging the formula to solve for \( R \):
\[
R = \frac{V}{I}
\]
Given \( V = 12 \) volts and \( I = 1.5 \) amperes, we can substitute these values into the equation:
\[
R = \frac{12}{1.5} = 8 \text{ ohms}
\]
Thus, the resistance of the circuit is 8 ohms. Ohm's Law is fundamental in understanding how voltage, current, and resistance interact in electrical circuits.

% Diagram Prompt: Generate a simple circuit diagram showing a voltage source, a resistor, and the current flow to visually represent Ohm's Law.