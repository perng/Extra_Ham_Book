\subsection{DC Current in Circuit Types}
\label{T5D13}

\begin{tcolorbox}[colback=gray!10!white,colframe=black!75!black,title=T5D13]
In which type of circuit is DC current the same through all components?
\begin{enumerate}[label=\Alph*)]
    \item \textbf{Series}
    \item Parallel
    \item Resonant
    \item Branch
\end{enumerate}
\end{tcolorbox}

\subsubsection{Intuitive Explanation}
Imagine you're in a line of people passing a ball. In a series circuit, the ball (which represents the current) has to go through each person (component) one by one. So, everyone gets the same ball, and no one can skip ahead. That's why the current is the same through all components in a series circuit. In a parallel circuit, it's like having multiple lines of people passing balls at the same time, so the current can split and take different paths.

\subsubsection{Advanced Explanation}
In a series circuit, components are connected end-to-end, forming a single path for the current to flow. According to Kirchhoff's Current Law (KCL), the current entering a junction must equal the current leaving it. Since there are no junctions in a series circuit, the current remains constant throughout. Mathematically, if \( I \) is the current, then for a series circuit:

\[ I_{\text{total}} = I_1 = I_2 = \dots = I_n \]

where \( I_1, I_2, \dots, I_n \) are the currents through each component.

In contrast, in a parallel circuit, components are connected across the same voltage source, creating multiple paths for the current. The total current is the sum of the currents through each branch, which can be different.

Understanding these principles is crucial for analyzing and designing electrical circuits, especially when dealing with DC (Direct Current) systems.

% Diagram Prompt: Generate a diagram showing a simple series circuit with three resistors and a battery, illustrating the same current flowing through each resistor.