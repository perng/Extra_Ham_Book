\subsection{Radio Horizon for VHF and UHF Signals}
\label{T3C11}

\begin{tcolorbox}[colback=gray!10!white,colframe=black!75!black,title=T3C11]
Why is the radio horizon for VHF and UHF signals more distant than the visual horizon?
\begin{enumerate}[noitemsep]
    \item Radio signals move somewhat faster than the speed of light
    \item Radio waves are not blocked by dust particles
    \item \textbf{The atmosphere refracts radio waves slightly}
    \item Radio waves are blocked by dust particles
\end{enumerate}
\end{tcolorbox}

\subsubsection*{Intuitive Explanation}
Imagine you're standing on a beach, looking out at the horizon. You can only see so far because the Earth curves away from you. Now, think of radio waves as a ball you're throwing. If you throw it straight, it will eventually hit the ground. But if you throw it at just the right angle, it can skip over the surface of the water and go further. The atmosphere acts like a gentle slope that helps the radio waves skip further than your eyes can see.

\subsubsection*{Advanced Explanation}
The Earth's atmosphere has a refractive index that decreases with altitude. This causes radio waves to bend slightly as they travel through the atmosphere, a phenomenon known as atmospheric refraction. For VHF (Very High Frequency) and UHF (Ultra High Frequency) signals, this bending effect extends the radio horizon beyond the visual horizon. The curvature of the Earth limits the visual horizon, but the refraction of radio waves allows them to follow the Earth's curvature more closely, effectively increasing the distance they can travel before being obstructed by the Earth's surface. This is why VHF and UHF signals can be received at greater distances than what the visual horizon would suggest.