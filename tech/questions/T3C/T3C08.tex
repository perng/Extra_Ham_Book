\subsection{Tropospheric Ducting Causes}
\label{T3C08}

\begin{tcolorbox}[colback=gray!10!white,colframe=black!75!black,title=T3C08]
What causes tropospheric ducting?
\begin{enumerate}[noitemsep]
    \item Discharges of lightning during electrical storms
    \item Sunspots and solar flares
    \item Updrafts from hurricanes and tornadoes
    \item \textbf{Temperature inversions in the atmosphere}
\end{enumerate}
\end{tcolorbox}

\subsubsection*{Intuitive Explanation}
Imagine the atmosphere as a giant sandwich. Normally, the temperature decreases as you go higher, like a sandwich with the top layer being cooler. But sometimes, a layer of warm air gets trapped above a layer of cooler air, like a warm slice of bread on top of a cool one. This weird sandwich can act like a tunnel, bending radio waves and making them travel much farther than usual. This phenomenon is called tropospheric ducting.

\subsubsection*{Advanced Explanation}
Tropospheric ducting occurs due to temperature inversions in the atmosphere. Normally, temperature decreases with altitude in the troposphere. However, during a temperature inversion, a layer of warm air is trapped above a layer of cooler air. This inversion layer can act as a waveguide, bending radio waves and allowing them to propagate over long distances. This effect is particularly noticeable in the VHF and UHF bands, where signals can travel hundreds of kilometers beyond the normal line-of-sight range. The inversion layer effectively ducts the radio waves, hence the term tropospheric ducting.