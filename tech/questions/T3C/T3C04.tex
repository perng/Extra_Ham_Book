\subsection{Propagation Types Beyond the Radio Horizon}
\label{T3C04}

\begin{tcolorbox}[colback=gray!10!white,colframe=black!75!black,title=T3C04]
Which of the following types of propagation is most commonly associated with occasional strong signals on the 10, 6, and 2 meter bands from beyond the radio horizon?
\begin{enumerate}[noitemsep]
    \item Backscatter
    \item \textbf{Sporadic E}
    \item D region absorption
    \item Gray-line propagation
\end{enumerate}
\end{tcolorbox}

\subsubsection*{Intuitive Explanation}
Imagine the ionosphere as a giant mirror in the sky that sometimes gets a bit wobbly. When it wobbles just right, it can bounce radio signals from far away back down to Earth, even if they’re beyond the normal radio horizon. This is called Sporadic E propagation, and it’s like catching a surprise signal from a distant station on your radio.

\subsubsection*{Advanced Explanation}
Sporadic E (Es) propagation occurs due to the formation of highly ionized patches in the E layer of the ionosphere, typically at altitudes of 90-120 km. These patches can reflect higher frequency signals (such as those on the 10, 6, and 2 meter bands) back to Earth, allowing for communication beyond the normal line-of-sight range. This phenomenon is sporadic and unpredictable, often resulting in strong, short-lived signals from distant stations. Other propagation types like backscatter, D region absorption, and gray-line propagation do not typically produce the same strong, occasional signals on these bands.