\subsection{Simplex UHF Signals and Radio Horizon}
\label{T3C01}

\begin{tcolorbox}[colback=gray!10!white,colframe=black!75!black,title=T3C01]
Why are simplex UHF signals rarely heard beyond their radio horizon?
\begin{enumerate}[label=\Alph*)]
    \item They are too weak to go very far
    \item FCC regulations prohibit them from going more than 50 miles
    \item \textbf{UHF signals are usually not propagated by the ionosphere}
    \item UHF signals are absorbed by the ionospheric D region
\end{enumerate}
\end{tcolorbox}

\subsubsection{Intuitive Explanation}
Imagine you're trying to throw a ball over a hill. If the hill is too high, the ball won't make it over, right? Now, think of UHF signals as that ball and the radio horizon as the hill. UHF signals are like lightweight balls that don't have the oomph to go over the hill. They usually stay close to the ground and don't bounce off the ionosphere (the sky's reflective layer) like some other signals do. So, they rarely go beyond the radio horizon.

\subsubsection{Advanced Explanation}
UHF (Ultra High Frequency) signals operate in the frequency range of 300 MHz to 3 GHz. These frequencies are generally too high to be effectively refracted by the ionosphere, which is a layer of the Earth's atmosphere ionized by solar radiation. The ionosphere can reflect lower frequency signals (like HF) back to Earth, allowing them to travel beyond the horizon. However, UHF signals typically propagate in a line-of-sight manner, meaning they travel in a straight line and are limited by the curvature of the Earth. 

The radio horizon is the maximum distance at which a radio signal can be received, determined by the height of the transmitting and receiving antennas and the Earth's curvature. For UHF signals, this distance is relatively short compared to lower frequency signals that can be reflected by the ionosphere. 

Mathematically, the radio horizon distance \( d \) can be approximated by:
\[ d \approx \sqrt{2h} \]
where \( h \) is the height of the antenna in meters, and \( d \) is in kilometers. For example, an antenna at 10 meters height would have a radio horizon of approximately 14.14 kilometers.

In summary, UHF signals are rarely heard beyond their radio horizon because they are not typically propagated by the ionosphere and are limited by line-of-sight propagation.

% Prompt for diagram: A diagram showing the difference between UHF signal propagation (line-of-sight) and HF signal propagation (ionospheric reflection) would be helpful here.