\subsection{Simplex UHF Signals and Radio Horizon}
\label{T3C01}

\begin{tcolorbox}[colback=gray!10!white,colframe=black!75!black,title=T3C01]
Why are simplex UHF signals rarely heard beyond their radio horizon?
\begin{enumerate}[noitemsep]
    \item They are too weak to go very far
    \item FCC regulations prohibit them from going more than 50 miles
    \item \textbf{UHF signals are usually not propagated by the ionosphere}
    \item UHF signals are absorbed by the ionospheric D region
\end{enumerate}
\end{tcolorbox}

\subsubsection*{Intuitive Explanation}
Imagine UHF signals as a flashlight beam. If you shine it straight ahead, it doesn't bend around corners or over hills. Similarly, UHF signals travel in a straight line and don't bounce off the ionosphere like some other radio waves. This means they can't reach beyond the horizon.

\subsubsection*{Advanced Explanation}
UHF (Ultra High Frequency) signals operate in the frequency range of 300 MHz to 3 GHz. These frequencies are generally too high to be refracted by the ionosphere, which is a layer of the Earth's atmosphere that can reflect lower frequency radio waves. Instead, UHF signals propagate primarily by line-of-sight, meaning they travel in a straight path from the transmitter to the receiver. This limits their range to the radio horizon, which is the distance at which the Earth's curvature blocks the direct path of the signal. Therefore, UHF signals are rarely heard beyond their radio horizon because they do not benefit from ionospheric propagation.