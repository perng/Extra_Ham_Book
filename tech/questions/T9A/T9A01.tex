\subsection{Beam Antenna Basics}
\label{T9A01}

\begin{tcolorbox}[colback=gray!10!white,colframe=black!75!black,title=T9A01]
What is a beam antenna?
\begin{enumerate}[label=\Alph*)]
    \item An antenna built from aluminum I-beams
    \item An omnidirectional antenna invented by Clarence Beam
    \item \textbf{An antenna that concentrates signals in one direction}
    \item An antenna that reverses the phase of received signals
\end{enumerate}
\end{tcolorbox}

\subsubsection{Intuitive Explanation}
Imagine you have a flashlight. If you point it in one direction, the light shines brightly in that direction, but not so much in others. A beam antenna works similarly, but instead of light, it focuses radio signals in one specific direction. This makes it great for sending or receiving signals from a particular place, like talking to a friend across the room with a flashlight instead of lighting up the whole room.

\subsubsection{Advanced Explanation}
A beam antenna, also known as a directional antenna, is designed to focus electromagnetic energy in a specific direction, thereby increasing the signal strength in that direction while reducing it in others. This is achieved through the use of elements such as reflectors and directors, which manipulate the phase and amplitude of the radio waves. The gain of a beam antenna is a measure of how effectively it can concentrate the signal, and it is typically expressed in decibels (dB). The directivity \( D \) of an antenna can be calculated using the formula:

\[
D = \frac{4\pi}{\int_0^{2\pi} \int_0^\pi P(\theta, \phi) \sin\theta \, d\theta \, d\phi}
\]

where \( P(\theta, \phi) \) is the power radiated in the direction \( (\theta, \phi) \). Beam antennas are commonly used in applications such as satellite communication, radar, and long-distance radio communication, where it is essential to maximize signal strength in a particular direction.

% Diagram prompt: Generate a diagram showing a beam antenna focusing radio waves in one direction, with elements like reflectors and directors labeled.