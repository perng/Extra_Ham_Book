\subsection{Quarter-Wavelength Antenna Length for 146 MHz}
\label{T9A08}

\begin{tcolorbox}[colback=gray!10!white,colframe=black!75!black,title=T9A08]
What is the approximate length, in inches, of a quarter-wavelength vertical antenna for 146 MHz?
\begin{enumerate}[label=\Alph*)]
    \item 112
    \item 50
    \item \textbf{19}
    \item 12
\end{enumerate}
\end{tcolorbox}

\subsubsection{Intuitive Explanation}
Imagine you’re trying to make a tiny antenna that’s just the right size to catch radio waves at 146 MHz. Think of it like tuning a guitar string—if the string is too long or too short, it won’t sound right. For this antenna, you need it to be about a quarter of the wavelength of the radio wave. It turns out that for 146 MHz, the magic length is around 19 inches. So, if you cut your antenna to about 19 inches, it’ll be just the right size to catch those waves!

\subsubsection{Advanced Explanation}
To determine the length of a quarter-wavelength antenna for a given frequency, we use the formula:

\[
\text{Length} = \frac{c}{4f}
\]

where \( c \) is the speed of light (\( 3 \times 10^8 \) meters per second) and \( f \) is the frequency in Hertz. For 146 MHz (\( 146 \times 10^6 \) Hz), the calculation is as follows:

\[
\text{Length} = \frac{3 \times 10^8}{4 \times 146 \times 10^6} \approx 0.5137 \text{ meters}
\]

Converting meters to inches (1 meter = 39.37 inches):

\[
0.5137 \text{ meters} \times 39.37 \approx 20.2 \text{ inches}
\]

The closest option to this calculated value is 19 inches, which is the correct answer. This calculation assumes a perfect conductor and ideal conditions, but in practice, the length might be slightly adjusted for optimal performance.

% Diagram prompt: Generate a diagram showing a quarter-wavelength vertical antenna with labeled dimensions.