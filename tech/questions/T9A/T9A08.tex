\subsection{Quarter-Wavelength Antenna Length for 146 MHz}
\label{T9A08}

\begin{tcolorbox}[colback=gray!10!white,colframe=black!75!black,title=T9A08]
What is the approximate length, in inches, of a quarter-wavelength vertical antenna for 146 MHz?
\begin{enumerate}[noitemsep]
    \item 112
    \item 50
    \item \textbf{19}
    \item 12
\end{enumerate}
\end{tcolorbox}

\subsubsection{Intuitive Explanation}
Imagine you're trying to build a tiny antenna that works perfectly with a radio signal at 146 MHz. The length of this antenna is crucial because it needs to match the wavelength of the signal. A quarter-wavelength antenna is one-fourth the length of the full wavelength of the signal. For 146 MHz, this turns out to be about 19 inches. Think of it like cutting a string to just the right length so it vibrates perfectly when you pluck it!

\subsubsection{Advanced Explanation}
To determine the length of a quarter-wavelength antenna, we first need to calculate the wavelength of the signal. The wavelength (\(\lambda\)) can be found using the formula:

\[
\lambda = \frac{c}{f}
\]

where \(c\) is the speed of light (\(3 \times 10^8\) meters per second) and \(f\) is the frequency (146 MHz or \(146 \times 10^6\) Hz). 

\[
\lambda = \frac{3 \times 10^8}{146 \times 10^6} \approx 2.05 \text{ meters}
\]

A quarter-wavelength is then:

\[
\frac{\lambda}{4} = \frac{2.05}{4} \approx 0.51 \text{ meters}
\]

Converting meters to inches (1 meter = 39.37 inches):

\[
0.51 \times 39.37 \approx 19.9 \text{ inches}
\]

Thus, the approximate length of a quarter-wavelength vertical antenna for 146 MHz is 19 inches.