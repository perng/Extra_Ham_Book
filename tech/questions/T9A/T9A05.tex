\subsection{Resonant Frequency of a Dipole Antenna}
\label{T9A05}

\begin{tcolorbox}[colback=gray!10!white,colframe=black!75!black,title=T9A05]
Which of the following increases the resonant frequency of a dipole antenna?
\begin{enumerate}[noitemsep]
    \item Lengthening it
    \item Inserting coils in series with radiating wires
    \item \textbf{Shortening it}
    \item Adding capacitive loading to the ends of the radiating wires
\end{enumerate}
\end{tcolorbox}

\subsubsection*{Intuitive Explanation}
Think of a dipole antenna like a swing. The longer the swing, the slower it moves back and forth (lower frequency). If you make the swing shorter, it moves faster (higher frequency). Similarly, shortening a dipole antenna increases its resonant frequency.

\subsubsection*{Advanced Explanation}
The resonant frequency of a dipole antenna is inversely proportional to its length. This relationship is derived from the fundamental principles of antenna theory, where the antenna's length is typically half the wavelength of the operating frequency. Mathematically, the resonant frequency \( f \) can be expressed as:

\[ f = \frac{c}{2L} \]

where \( c \) is the speed of light and \( L \) is the length of the antenna. Therefore, shortening the antenna \( L \) increases the resonant frequency \( f \). Adding coils or capacitive loading generally lowers the resonant frequency, as these components increase the effective electrical length of the antenna.