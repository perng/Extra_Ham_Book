\subsection{Resonant Frequency of a Dipole Antenna}
\label{T9A05}

\begin{tcolorbox}[colback=gray!10!white,colframe=black!75!black,title=T9A05]
Which of the following increases the resonant frequency of a dipole antenna?
\begin{enumerate}[label=\Alph*)]
    \item Lengthening it
    \item Inserting coils in series with radiating wires
    \item \textbf{Shortening it}
    \item Adding capacitive loading to the ends of the radiating wires
\end{enumerate}
\end{tcolorbox}

\subsubsection{Intuitive Explanation}
Imagine a dipole antenna as a jump rope. The longer the rope, the slower it swings back and forth. Similarly, a longer antenna has a lower resonant frequency. If you want the rope to swing faster (higher frequency), you need to make it shorter. So, shortening the dipole antenna increases its resonant frequency. It's like cutting the jump rope to make it swing faster!

\subsubsection{Advanced Explanation}
The resonant frequency \( f \) of a dipole antenna is inversely proportional to its length \( L \). The relationship can be expressed as:

\[
f \propto \frac{1}{L}
\]

This means that as the length of the antenna decreases, the resonant frequency increases. The exact resonant frequency can be calculated using the formula:

\[
f = \frac{c}{2L}
\]

where \( c \) is the speed of light (\( 3 \times 10^8 \) m/s). For example, if the length of the antenna is halved, the resonant frequency doubles. 

Adding coils in series with the radiating wires (choice B) increases the inductance, which lowers the resonant frequency. Adding capacitive loading to the ends of the radiating wires (choice D) increases the capacitance, which also lowers the resonant frequency. Therefore, the only way to increase the resonant frequency is to shorten the antenna (choice C).

% Diagram prompt: Generate a diagram showing the relationship between the length of a dipole antenna and its resonant frequency, with a shorter antenna labeled as having a higher frequency and a longer antenna labeled as having a lower frequency.