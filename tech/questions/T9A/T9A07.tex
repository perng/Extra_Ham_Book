\subsection{Disadvantage of Using Handheld VHF Transceiver in a Vehicle}
\label{T9A07}

\begin{tcolorbox}[colback=gray!10!white,colframe=black!75!black,title=T9A07]
What is a disadvantage of using a handheld VHF transceiver with a flexible antenna inside a vehicle?
\begin{enumerate}[label=\Alph*)]
    \item \textbf{Signal strength is reduced due to the shielding effect of the vehicle}
    \item The bandwidth of the antenna will decrease, increasing SWR
    \item The SWR might decrease, decreasing the signal strength
    \item All these choices are correct
\end{enumerate}
\end{tcolorbox}

\subsubsection{Intuitive Explanation}
Imagine you're trying to talk to your friend outside while you're inside a car. The car is like a big metal box that blocks your voice from reaching your friend easily. Similarly, when you use a handheld VHF transceiver inside a car, the metal body of the car acts like a shield, making it harder for the radio signals to get out. This means your signal strength is weaker, and your message might not get through as clearly.

\subsubsection{Advanced Explanation}
When a handheld VHF transceiver is used inside a vehicle, the metal structure of the vehicle acts as a Faraday cage. A Faraday cage is an enclosure used to block electromagnetic fields. The metal body of the vehicle reflects and absorbs the radio waves, reducing the effective signal strength that can be transmitted or received. This phenomenon is known as the shielding effect.

The shielding effect can be quantified by considering the attenuation of the radio signal. The attenuation \( A \) in decibels (dB) can be calculated using the formula:

\[ A = 10 \log_{10} \left( \frac{P_{\text{in}}}{P_{\text{out}}} \right) \]

where \( P_{\text{in}} \) is the power of the signal before it encounters the shielding, and \( P_{\text{out}} \) is the power of the signal after passing through the shielding. In the case of a vehicle, \( P_{\text{out}} \) is significantly lower than \( P_{\text{in}} \), leading to a high attenuation value and thus a reduced signal strength.

Additionally, the Standing Wave Ratio (SWR) is a measure of how well the antenna is matched to the transceiver. While the shielding effect primarily reduces signal strength, it does not directly affect the SWR or the bandwidth of the antenna. Therefore, the correct answer focuses on the reduction in signal strength due to the shielding effect of the vehicle.

% Diagram prompt: Generate a diagram showing a handheld VHF transceiver inside a car, with radio waves being reflected and absorbed by the car's metal body.