\subsection{Transmitting Encoded Messages}
\label{T1D03}

\begin{tcolorbox}[colback=gray!10!white,colframe=black!75!black,title=T1D03]
When is it permissible to transmit messages encoded to obscure their meaning?
\begin{enumerate}[label=\Alph*),noitemsep]
    \item Only during contests
    \item Only when transmitting certain approved digital codes
    \item \textbf{Only when transmitting control commands to space stations or radio control craft}
    \item Never
\end{enumerate}
\end{tcolorbox}

\subsubsection*{Explanation}
This question addresses the regulations surrounding the transmission of encoded messages in amateur radio. The correct answer is \textbf{C}, which specifies that encoded messages are only allowed when sending control commands to space stations or radio control craft. This ensures that the use of encoded messages is limited to specific, authorized purposes, maintaining the transparency and openness of amateur radio communications.