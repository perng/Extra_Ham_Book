\subsection{Circumstances Prohibiting One-Way Transmissions by an Amateur Station}
\label{T1D02}

\begin{tcolorbox}[colback=gray!10!white,colframe=black!75!black,title=T1D02]
Under which of the following circumstances are one-way transmissions by an amateur station prohibited?
\begin{enumerate}[label=\Alph*]
    \item In all circumstances
    \item \textbf{Broadcasting}
    \item International Morse Code Practice
    \item Telecommand or transmissions of telemetry
\end{enumerate}
\end{tcolorbox}

\subsubsection{Intuitive Explanation}
Imagine you're at a school assembly, and the principal is giving a speech. Everyone is listening, but no one is allowed to respond. That's kind of like a one-way transmission in radio. Now, think about a radio station playing music—it's sending out signals to everyone, but it's not expecting any replies. In the world of amateur radio, this kind of broadcasting is a big no-no! It's like the principal giving a speech but not letting anyone ask questions or make comments. So, when it comes to amateur radio, broadcasting is the one thing you can't do with one-way transmissions.

\subsubsection{Advanced Explanation}
In amateur radio, one-way transmissions refer to communications where only one station is transmitting, and no response or interaction is expected from other stations. The Federal Communications Commission (FCC) and other regulatory bodies have specific rules governing these transmissions to ensure they are used appropriately.

The correct answer, \textbf{Broadcasting}, is prohibited because broadcasting involves transmitting content intended for the general public, which is not the purpose of amateur radio. Amateur radio is meant for personal communication, experimentation, and emergency communication, not for disseminating information to a wide audience.

The other options are permissible under certain conditions:
\begin{itemize}
    \item \textbf{International Morse Code Practice}: This is allowed as it is considered a form of communication and training.
    \item \textbf{Telecommand or transmissions of telemetry}: These are also allowed as they are specific types of communication used for controlling remote devices or sending data.
\end{itemize}

Therefore, the only circumstance where one-way transmissions are prohibited is when they are used for broadcasting.

% Prompt for generating a diagram:
% Diagram showing a comparison between one-way transmissions (e.g., broadcasting) and two-way communications (e.g., amateur radio conversations).