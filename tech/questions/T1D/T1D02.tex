\subsection{Prohibition of One-Way Transmissions}
\label{T1D02}

\begin{tcolorbox}[colback=gray!10!white,colframe=black!75!black,title=T1D02]
Under which of the following circumstances are one-way transmissions by an amateur station prohibited?
\begin{enumerate}[label=\Alph*,noitemsep]
    \item In all circumstances
    \item \textbf{Broadcasting}
    \item International Morse Code Practice
    \item Telecommand or transmissions of telemetry
\end{enumerate}
\end{tcolorbox}

\subsubsection*{Explanation}
One-way transmissions in amateur radio are generally allowed for specific purposes such as telecommand, telemetry, and Morse code practice. However, broadcasting, which involves transmitting content intended for the general public, is prohibited. This regulation ensures that amateur radio remains a non-commercial service primarily for personal communication and experimentation.