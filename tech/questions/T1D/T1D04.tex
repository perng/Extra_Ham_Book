\subsection{Rules for Transmitting Music}
\label{T1D04}

\begin{tcolorbox}[colback=gray!10!white,colframe=black!75!black,title=T1D04]
Under what conditions is an amateur station authorized to transmit music using a phone emission?
\begin{enumerate}[label=\Alph*)]
    \item \textbf{When incidental to an authorized retransmission of manned spacecraft communications}
    \item When the music produces no spurious emissions
    \item When transmissions are limited to less than three minutes per hour
    \item When the music is transmitted above 1280 MHz
\end{enumerate}
\end{tcolorbox}

\subsubsection{Intuitive Explanation}
Imagine you're a radio operator, and you want to play some music over the airwaves. Normally, this is a big no-no because amateur radio is for communication, not for DJing. But there's one special exception: if you're helping to relay messages from astronauts in space, and the music just happens to be part of that transmission, then it's okay. Think of it like this: if the music is just tagging along with the important space talk, it's allowed. Otherwise, keep the tunes to yourself!

\subsubsection{Advanced Explanation}
In amateur radio, the transmission of music is generally prohibited under FCC regulations to prevent the misuse of amateur bands for entertainment purposes. However, there is an exception outlined in Part 97.113 of the FCC rules. This exception allows the transmission of music when it is incidental to an authorized retransmission of manned spacecraft communications. 

The key term here is incidental, meaning the music must be a minor and unintentional part of the primary communication. This rule ensures that the primary purpose of amateur radio—communication—is maintained, while allowing for the practicalities of retransmitting complex signals from manned spacecraft, which may include background music or other audio elements.


This regulation underscores the importance of maintaining the integrity of amateur radio bands for their intended purpose, while accommodating the unique requirements of space communication.

% Prompt for generating a diagram: A diagram showing the composition of the signal \( S(t) \) with \( C(t) \) as the primary communication signal and \( M(t) \) as the incidental music signal, with \( M(t) \) being much smaller in amplitude compared to \( C(t) \).