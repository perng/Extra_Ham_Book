\subsection{Authorized Transmission of Music in Amateur Stations}\label{T1D04}

\begin{tcolorbox}[colback=gray!10!white,colframe=black!75!black,title=T1D04]
Under what conditions is an amateur station authorized to transmit music using a phone emission?
\begin{enumerate}[label=\Alph*,noitemsep]
    \item \textbf{When incidental to an authorized retransmission of manned spacecraft communications}
    \item When the music produces no spurious emissions
    \item When transmissions are limited to less than three minutes per hour
    \item When the music is transmitted above 1280 MHz
\end{enumerate}
\end{tcolorbox}

Amateur radio stations are generally not allowed to transmit music, as it is considered a form of entertainment and not within the scope of amateur radio communications. However, there is an exception when the music is incidental to an authorized retransmission of manned spacecraft communications. This means that if the music is part of a broadcast from a manned spacecraft and the amateur station is retransmitting it, the transmission is allowed. The other options do not provide valid conditions for transmitting music in amateur radio.