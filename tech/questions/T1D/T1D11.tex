\subsection{Transmission Without Identification}
\label{T1D11}

\begin{tcolorbox}[colback=gray!10!white,colframe=black!75!black,title=T1D11]
When may an amateur station transmit without identifying on the air?
\begin{enumerate}[label=\Alph*,noitemsep]
    \item When the transmissions are of a brief nature to make station adjustments
    \item When the transmissions are unmodulated
    \item When the transmitted power level is below 1 watt
    \item \textbf{When transmitting signals to control model craft}
\end{enumerate}
\end{tcolorbox}

\subsubsection*{Explanation}
Amateur radio operators are generally required to identify their station at regular intervals during transmissions. However, there are specific exceptions to this rule. One such exception is when the transmissions are used to control model craft. In this case, the operator is allowed to transmit without identifying the station. This exception is in place to allow for the practical operation of model craft without the need for constant identification, which could interfere with the control signals.