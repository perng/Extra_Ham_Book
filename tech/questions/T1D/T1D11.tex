\subsection{Transmit Without Identifying?}
\label{T1D11}

\begin{tcolorbox}[colback=gray!10!white,colframe=black!75!black,title=T1D11]
When may an amateur station transmit without identifying on the air?
\begin{enumerate}[label=\Alph*]
    \item When the transmissions are of a brief nature to make station adjustments
    \item When the transmissions are unmodulated
    \item When the transmitted power level is below 1 watt
    \item \textbf{When transmitting signals to control model craft}
\end{enumerate}
\end{tcolorbox}

\subsubsection{Intuitive Explanation}
Imagine you're playing with a remote-controlled car or a drone. You don't need to shout your name every time you press a button to make it move, right? Similarly, when amateur radio operators are controlling model craft like planes, cars, or boats, they don't have to announce their call sign every time they send a signal. It's like a secret handshake between you and your toy—no need to tell the whole world about it!

\subsubsection{Advanced Explanation}
According to FCC regulations, amateur radio operators are generally required to identify their station by transmitting their call sign at regular intervals. However, there are specific exceptions to this rule. One such exception is when the transmissions are used to control model craft. This is outlined in Part 97.215 of the FCC rules, which states that an amateur station may transmit signals to control a model craft without identifying the station.

The rationale behind this exception is that the primary purpose of these transmissions is to control the model craft, and the identification requirement would be impractical and unnecessary in this context. The transmissions are typically short and frequent, making it cumbersome to include a call sign each time. Additionally, the power levels used for controlling model craft are usually low, minimizing the risk of interference with other communications.

In summary, the correct answer is \textbf{D}, as it aligns with the FCC regulations that permit amateur stations to transmit without identifying when controlling model craft.

% Diagram Prompt: Generate a diagram showing a remote-controlled model craft with a radio transmitter, illustrating the concept of transmitting signals without identification.