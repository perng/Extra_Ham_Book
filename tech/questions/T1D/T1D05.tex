\subsection{Use of Amateur Radio Stations for Equipment Sales}
\label{T1D05}

\begin{tcolorbox}[colback=gray!10!white,colframe=black!75!black,title=T1D05]
When may amateur radio operators use their stations to notify other amateurs of the availability of equipment for sale or trade?
\begin{enumerate}[label=\Alph*),noitemsep]
    \item Never
    \item When the equipment is not the personal property of either the station licensee, or the control operator, or their close relatives
    \item When no profit is made on the sale
    \item \textbf{When selling amateur radio equipment and not on a regular basis}
\end{enumerate}
\end{tcolorbox}

Amateur radio operators are allowed to use their stations to notify other amateurs about the availability of equipment for sale or trade, but only under specific conditions. The correct answer is \textbf{D}, which states that this is permissible when selling amateur radio equipment and not on a regular basis. This ensures that the primary purpose of amateur radio remains communication and not commercial activity.