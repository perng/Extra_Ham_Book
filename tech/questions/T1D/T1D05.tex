\subsection{Equipment for Sale or Trade?}
\label{T1D05}

\begin{tcolorbox}[colback=gray!10!white,colframe=black!75!black,title=T1D05]
When may amateur radio operators use their stations to notify other amateurs of the availability of equipment for sale or trade?
\begin{enumerate}[label=\Alph*)]
    \item Never
    \item When the equipment is not the personal property of either the station licensee, or the control operator, or their close relatives
    \item When no profit is made on the sale
    \item \textbf{When selling amateur radio equipment and not on a regular basis}
\end{enumerate}
\end{tcolorbox}

\subsubsection{Intuitive Explanation}
Imagine you have a cool toy that you don't play with anymore, and you want to let your friends know it's up for grabs. You can use your walkie-talkie to tell them about it, but only if it's a one-time thing and not something you do every day. If you start selling toys all the time, it's like turning your walkie-talkie into a shopping channel, and that's not what it's for! So, you can tell your friends about your toy, but don't make it a habit.

\subsubsection{Advanced Explanation}
Amateur radio operators are allowed to use their stations to notify other amateurs about the availability of equipment for sale or trade, but only under specific conditions. According to FCC regulations, this is permissible when the equipment being sold is amateur radio-related and the sale is not conducted on a regular basis. This means that the primary purpose of the amateur radio station should not be commercial activity. The key point here is that the sale should be incidental and not a regular business operation. This ensures that the amateur radio service remains primarily a non-commercial, personal communication service.

% Diagram Prompt: A flowchart showing the conditions under which amateur radio operators can notify others about equipment for sale or trade.