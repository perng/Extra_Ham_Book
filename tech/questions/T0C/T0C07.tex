\subsection{Hazard of Touching an Antenna During Transmission}
\label{T0C07}

\begin{tcolorbox}[colback=gray!10!white,colframe=black!75!black,title=T0C07]
What hazard is created by touching an antenna during a transmission?
\begin{enumerate}[label=\Alph*)]
    \item Electrocution
    \item \textbf{RF burn to skin}
    \item Radiation poisoning
    \item All these choices are correct
\end{enumerate}
\end{tcolorbox}

\subsubsection{Intuitive Explanation}
Imagine you’re holding a hot pan without an oven mitt—ouch, right? Now, think of an antenna during transmission as that hot pan, but instead of heat, it’s sending out radio waves. If you touch it, you might get a burn, but not from heat. It’s called an RF burn, and it’s like a tiny zap that can hurt your skin. So, just like you wouldn’t touch a hot pan, don’t touch a transmitting antenna!

\subsubsection{Advanced Explanation}
When an antenna is transmitting, it radiates electromagnetic waves, specifically radio frequency (RF) energy. The antenna is designed to efficiently radiate this energy into the surrounding space. However, if a person touches the antenna during transmission, the RF energy can be absorbed by the body, particularly the skin. This absorption can cause localized heating, leading to what is known as an RF burn.

The severity of the burn depends on the power of the transmission and the duration of contact. The skin acts as a conductor, and the RF energy induces currents within the tissue, causing heating. This is different from electrocution, which involves electric shock from a power source, and radiation poisoning, which results from exposure to ionizing radiation.

Mathematically, the power absorbed by the skin can be approximated using the formula:
\[
P = \sigma E^2
\]
where \( P \) is the power absorbed, \( \sigma \) is the conductivity of the skin, and \( E \) is the electric field strength at the point of contact. The resulting temperature rise can be calculated using the heat capacity of the tissue.

In summary, touching an antenna during transmission can cause an RF burn due to the absorption of RF energy by the skin, leading to localized heating. This is the primary hazard associated with such contact.

% Diagram Prompt: Generate a diagram showing an antenna radiating RF waves and a hand touching it, with labels indicating the RF energy absorption and the resulting RF burn.