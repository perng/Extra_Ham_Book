\subsection{Hazard of Touching an Antenna During Transmission}
\label{T0C07}

\begin{tcolorbox}[colback=gray!10!white,colframe=black!75!black,title=T0C07]
What hazard is created by touching an antenna during a transmission?
\begin{enumerate}[noitemsep]
    \item Electrocution
    \item \textbf{RF burn to skin}
    \item Radiation poisoning
    \item All these choices are correct
\end{enumerate}
\end{tcolorbox}

\subsubsection*{Intuitive Explanation}
Imagine the antenna is like a hot stove. If you touch it while it's on (transmitting), you're going to get burned, but not by heat—by radio frequency (RF) energy. This is called an RF burn, and it can hurt your skin just like a regular burn.

\subsubsection*{Advanced Explanation}
When an antenna is transmitting, it radiates electromagnetic waves at radio frequencies. These waves carry energy, and when you touch the antenna, your body can absorb this energy, causing localized heating in the skin. This is known as an RF burn. Unlike electrocution, which involves electric current passing through the body, an RF burn is caused by the absorption of RF energy. Radiation poisoning, on the other hand, is related to ionizing radiation, which is not emitted by typical radio antennas. Therefore, the correct answer is an RF burn to the skin.