\subsection{Duty Cycle Definition for RF Exposure}
\label{T0C11}

\begin{tcolorbox}[colback=gray!10!white,colframe=black!75!black,title=T0C11]
What is the definition of duty cycle during the averaging time for RF exposure?
\begin{enumerate}[label=\Alph*)]
    \item The difference between the lowest power output and the highest power output of a transmitter
    \item The difference between the PEP and average power output of a transmitter
    \item \textbf{The percentage of time that a transmitter is transmitting}
    \item The percentage of time that a transmitter is not transmitting
\end{enumerate}
\end{tcolorbox}

\subsubsection{Intuitive Explanation}
Imagine you have a light bulb that you turn on and off. The duty cycle is like telling you how much time the bulb is actually on compared to the total time you’re watching it. If it’s on half the time and off half the time, the duty cycle is 50\%. For RF exposure, it’s the same idea but with a transmitter—it’s the percentage of time the transmitter is actually sending out signals.

\subsubsection{Advanced Explanation}
The duty cycle (\(D\)) is a crucial parameter in RF exposure calculations, defined as the ratio of the time the transmitter is actively transmitting (\(T_{\text{on}}\)) to the total averaging time (\(T_{\text{total}}\)):

\[
D = \frac{T_{\text{on}}}{T_{\text{total}}} \times 100\%
\]

For example, if a transmitter is on for 2 seconds out of a 10-second period, the duty cycle is:

\[
D = \frac{2}{10} \times 100\% = 20\%
\]

This concept is essential for determining the average power output and ensuring compliance with RF exposure limits. The duty cycle helps in calculating the effective exposure level over time, which is critical for safety assessments.

% Diagram prompt: A simple timeline showing the on and off periods of a transmitter with labeled durations and the corresponding duty cycle calculation.