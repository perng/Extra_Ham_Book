\subsection{Reducing Exposure to RF Radiation}
\label{T0C08}

\begin{tcolorbox}[colback=gray!10!white,colframe=black!75!black,title=T0C08]
Which of the following actions can reduce exposure to RF radiation?
\begin{enumerate}[label=\Alph*)]
    \item \textbf{Relocate antennas}
    \item Relocate the transmitter
    \item Increase the duty cycle
    \item All these choices are correct
\end{enumerate}
\end{tcolorbox}

\subsubsection{Intuitive Explanation}
Imagine you're standing next to a loudspeaker at a concert. If you move away from the speaker, the music doesn't blast your ears as much. Similarly, if you move antennas away from people, the RF radiation they're exposed to decreases. Relocating the transmitter is like moving the DJ booth—it doesn't directly reduce the sound at your spot. Increasing the duty cycle is like turning up the volume, which actually makes things worse. So, the best way to reduce exposure is to move the antennas away from people.

\subsubsection{Advanced Explanation}
RF radiation exposure is influenced by the distance from the radiation source and the power density of the radiation. The power density \( S \) at a distance \( d \) from an isotropic antenna is given by:
\[
S = \frac{P_{\text{rad}}}{4 \pi d^2}
\]
where \( P_{\text{rad}} \) is the radiated power. Relocating antennas increases \( d \), thereby reducing \( S \). Relocating the transmitter does not necessarily change \( d \) for the antennas, and increasing the duty cycle increases \( P_{\text{rad}} \), which increases \( S \). Therefore, relocating antennas is the most effective method to reduce RF radiation exposure.

% Prompt for diagram: A diagram showing the relationship between distance from the antenna and power density, with a graph of \( S \) versus \( d \).