\subsection{Responsibility for RF Energy Exposure Limits}
\label{T0C13}

\begin{tcolorbox}[colback=gray!10!white,colframe=black!75!black,title=T0C13]
Who is responsible for ensuring that no person is exposed to RF energy above the FCC exposure limits?
\begin{enumerate}[label=\Alph*]
    \item The FCC
    \item \textbf{The station licensee}
    \item Anyone who is near an antenna
    \item The local zoning board
\end{enumerate}
\end{tcolorbox}

\subsubsection{Intuitive Explanation}
Imagine you have a super loud stereo system in your room. If you blast the music too loud, your neighbors might complain, right? Now, think of RF energy like the volume of that stereo. The FCC sets the rules for how loud (or strong) the RF energy can be, but it's up to you, the person with the stereo (or in this case, the radio station), to make sure you don't go over the limit. So, the station licensee is like the DJ who has to keep the volume in check!

\subsubsection{Advanced Explanation}
The Federal Communications Commission (FCC) establishes maximum permissible exposure (MPE) limits for RF energy to protect the public from potential health risks. These limits are based on extensive research and are designed to ensure safety. However, the FCC does not actively monitor every transmitter. Instead, the responsibility falls on the station licensee, who must ensure that their equipment operates within these limits. This involves calculating the RF exposure levels, conducting measurements if necessary, and implementing measures to reduce exposure if the limits are exceeded. The licensee must also maintain records and documentation to demonstrate compliance with FCC regulations.

% Prompt for diagram: A diagram showing the relationship between the FCC, the station licensee, and the RF exposure limits could be helpful here.