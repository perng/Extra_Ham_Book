\subsection{Type of Radiation in Radio Signals}
\label{T0C01}

\begin{tcolorbox}[colback=gray!10!white,colframe=black!75!black,title=T0C01]
What type of radiation are radio signals?
\begin{enumerate}[label=\Alph*)]
    \item Gamma radiation
    \item Ionizing radiation
    \item Alpha radiation
    \item \textbf{Non-ionizing radiation}
\end{enumerate}
\end{tcolorbox}

\subsubsection{Intuitive Explanation}
Imagine radio signals as friendly waves that travel through the air to bring you music, news, and your favorite podcasts. These waves are like the gentle ripples on a pond when you toss a small pebble into it. They don’t have enough energy to knock electrons off atoms, which means they can’t cause any harm to your body. That’s why we call them non-ionizing radiation. They’re the good guys of the radiation world, unlike their more energetic cousins like gamma rays, which can be harmful.

\subsubsection{Advanced Explanation}
Radio signals are a form of electromagnetic radiation, specifically in the radio frequency (RF) range of the electromagnetic spectrum. Electromagnetic radiation is classified based on its frequency and wavelength. Radio waves have relatively low frequencies (typically ranging from 3 kHz to 300 GHz) and long wavelengths (from 1 millimeter to 100 kilometers).

The key distinction between ionizing and non-ionizing radiation lies in the energy of the photons. Ionizing radiation, such as gamma rays, X-rays, and ultraviolet light, has enough energy to remove tightly bound electrons from atoms, leading to ionization. This process can cause damage to biological tissues and DNA.

In contrast, non-ionizing radiation, which includes radio waves, microwaves, and visible light, does not have sufficient energy to ionize atoms or molecules. The energy \(E\) of a photon is given by the equation:

\[
E = h \nu
\]

where \(h\) is Planck’s constant (\(6.626 \times 10^{-34} \, \text{Js}\)) and \(\nu\) is the frequency of the radiation. For radio waves, the frequency is relatively low, resulting in photon energies that are too weak to cause ionization.

Therefore, radio signals are classified as non-ionizing radiation, making them safe for everyday use in communication technologies.

% Prompt for generating a diagram: Illustrate the electromagnetic spectrum, highlighting the radio frequency range and its position relative to other types of radiation such as gamma rays, X-rays, ultraviolet, visible light, infrared, microwaves, and radio waves.