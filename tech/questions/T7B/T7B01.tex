\subsection{Handling Over-Deviation in FM Transceivers}
\label{T7B01}

\begin{tcolorbox}[colback=gray!10!white,colframe=black!75!black,title=T7B01]
What can you do if you are told your FM handheld or mobile transceiver is over-deviating?
\begin{enumerate}[label=\Alph*)]
    \item Talk louder into the microphone
    \item Let the transceiver cool off
    \item Change to a higher power level
    \item \textbf{Talk farther away from the microphone}
\end{enumerate}
\end{tcolorbox}

\subsubsection{Intuitive Explanation}
Imagine your FM transceiver is like a person shouting into a microphone. If they shout too loudly, the sound gets distorted, and it’s hard to understand what they’re saying. Over-deviation is like shouting too loudly—it messes up the signal. To fix this, you don’t need to shout louder, let the microphone cool off, or turn up the volume. Instead, just step back a bit from the microphone! This way, the signal stays clear, and everyone can hear you perfectly.

\subsubsection{Advanced Explanation}
In FM (Frequency Modulation) systems, deviation refers to the extent to which the carrier frequency varies from its center frequency in response to the modulating signal. Over-deviation occurs when the frequency deviation exceeds the allowed limit, causing distortion and potential interference with adjacent channels. 

To correct over-deviation, reducing the amplitude of the modulating signal is necessary. This can be achieved by increasing the distance between the microphone and the speaker, thereby decreasing the input signal strength. Mathematically, the frequency deviation $\Delta f$ is proportional to the amplitude of the modulating signal $A_m$:

\[
\Delta f = k_f \cdot A_m
\]

where $k_f$ is the frequency deviation constant. By reducing $A_m$, $\Delta f$ is also reduced, bringing the signal back within the acceptable range. This ensures proper signal transmission and minimizes distortion.

% Diagram Prompt: Generate a diagram showing the relationship between microphone distance and signal amplitude in an FM transceiver.