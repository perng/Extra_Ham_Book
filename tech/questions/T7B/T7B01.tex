\subsection{Handling Over-Deviation in FM Transceivers}
\label{T7B01}

\begin{tcolorbox}[colback=gray!10!white,colframe=black!75!black,title=T7B01]
What can you do if you are told your FM handheld or mobile transceiver is over-deviating?
\begin{enumerate}[noitemsep]
    \item Talk louder into the microphone
    \item Let the transceiver cool off
    \item Change to a higher power level
    \item \textbf{Talk farther away from the microphone}
\end{enumerate}
\end{tcolorbox}

\subsubsection*{Explanation}
Over-deviation in FM transceivers occurs when the signal's frequency deviation exceeds the allowed limit, which can cause interference and distortion. To address this, you should reduce the input signal's strength by talking farther away from the microphone. This decreases the modulation level, bringing it back within the acceptable range.