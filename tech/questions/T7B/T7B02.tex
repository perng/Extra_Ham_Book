\subsection{Unintentional Reception of Amateur Radio Signals}
\label{T7B02}

\begin{tcolorbox}[colback=gray!10!white,colframe=black!75!black,title=T7B02]
What would cause a broadcast AM or FM radio to receive an amateur radio transmission unintentionally?
\begin{enumerate}[label=\Alph*)]
    \item \textbf{The receiver is unable to reject strong signals outside the AM or FM band}
    \item The microphone gain of the transmitter is turned up too high
    \item The audio amplifier of the transmitter is overloaded
    \item The deviation of an FM transmitter is set too low
\end{enumerate}
\end{tcolorbox}

\subsubsection{Intuitive Explanation}
Imagine your radio is like a picky eater. It’s supposed to only eat AM or FM signals, but sometimes, if there’s a really strong snack (like an amateur radio signal) nearby, it can’t resist and ends up eating it too! This happens because the radio isn’t good at ignoring strong signals that are outside its usual menu.

\subsubsection{Advanced Explanation}
Broadcast AM and FM radios are designed to operate within specific frequency bands. However, they may unintentionally receive signals outside these bands due to insufficient filtering or the presence of very strong signals. This phenomenon is known as \textit{receiver desensitization} or \textit{intermodulation interference}. 

In technical terms, the receiver's front-end filters are not perfectly selective, allowing strong signals from adjacent or nearby frequencies to pass through. This can cause the receiver to demodulate and amplify these unintended signals, leading to the reception of amateur radio transmissions. The correct answer, \textbf{A}, highlights this issue by stating that the receiver is unable to reject strong signals outside the AM or FM band.

% Prompt for diagram: A diagram showing the frequency spectrum of AM/FM bands and how a strong amateur radio signal outside these bands can interfere with the receiver.