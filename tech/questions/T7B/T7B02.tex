\subsection{Unintentional Reception of Amateur Radio Signals}
\label{T7B02}

\begin{tcolorbox}[colback=gray!10!white,colframe=black!75!black,title=T7B02]
What would cause a broadcast AM or FM radio to receive an amateur radio transmission unintentionally?
\begin{enumerate}[noitemsep]
    \item \textbf{The receiver is unable to reject strong signals outside the AM or FM band}
    \item The microphone gain of the transmitter is turned up too high
    \item The audio amplifier of the transmitter is overloaded
    \item The deviation of an FM transmitter is set too low
\end{enumerate}
\end{tcolorbox}

\subsubsection*{Explanation}
Broadcast AM and FM radios are designed to receive signals within specific frequency bands. However, if an amateur radio transmission is strong enough and close in frequency to the broadcast band, the receiver might not be able to reject it effectively. This is because the receiver's filtering may not be sufficient to block out strong signals that are just outside the intended frequency range. The correct answer is \textbf{A}, as it directly addresses the receiver's inability to reject strong signals outside the AM or FM band.