\subsection{Fundamental Overload Reduction}
\label{T7B05}

\begin{tcolorbox}[colback=gray!10!white,colframe=black!75!black,title=T7B05]
How can fundamental overload of a non-amateur radio or TV receiver by an amateur signal be reduced or eliminated?
\begin{enumerate}[noitemsep]
    \item \textbf{Block the amateur signal with a filter at the antenna input of the affected receiver}
    \item Block the interfering signal with a filter on the amateur transmitter
    \item Switch the transmitter from FM to SSB
    \item Switch the transmitter to a narrow-band mode
\end{enumerate}
\end{tcolorbox}

\subsubsection*{Explanation}
Fundamental overload occurs when a strong amateur radio signal overwhelms a non-amateur receiver, causing interference. The most effective way to mitigate this is by placing a filter at the antenna input of the affected receiver. This filter blocks the amateur signal before it enters the receiver, preventing overload. Options B, C, and D involve modifying the transmitter, which does not directly address the issue at the receiver.