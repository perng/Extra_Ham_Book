\subsection{Resolving Non-Fiber Optic Cable TV Interference}
\label{T7B09}

\begin{tcolorbox}[colback=gray!10!white,colframe=black!75!black,title=T7B09]
What should be the first step to resolve non-fiber optic cable TV interference caused by your amateur radio transmission?
\begin{enumerate}[label=\Alph*)]
    \item Add a low-pass filter to the TV antenna input
    \item Add a high-pass filter to the TV antenna input
    \item Add a preamplifier to the TV antenna input
    \item \textbf{Be sure all TV feed line coaxial connectors are installed properly}
\end{enumerate}
\end{tcolorbox}

\subsubsection{Intuitive Explanation}
Imagine your TV is like a garden hose, and the signal is the water flowing through it. If there’s a leak (a bad connector), water (the signal) can escape, and you might get interference. The first thing you should do is check if the hose is properly connected before trying to add fancy filters or amplifiers. It’s like fixing the leak before buying a new sprinkler!

\subsubsection{Advanced Explanation}
Interference in non-fiber optic cable TV systems often arises due to improper shielding or grounding, which can be caused by poorly installed coaxial connectors. Coaxial cables are designed to carry signals with minimal interference, but if the connectors are not properly installed, the shielding can be compromised, allowing RF signals from your amateur radio transmission to leak into the TV system. 

The first step is to ensure all coaxial connectors are correctly installed and tightened. This ensures the integrity of the shielding, reducing the likelihood of RF interference. Only after verifying the connectors should you consider adding filters or amplifiers, as these are secondary measures to address any residual interference.

% Diagram Prompt: Generate a diagram showing a coaxial cable with properly installed connectors and another with improperly installed connectors, highlighting the potential points of RF leakage.