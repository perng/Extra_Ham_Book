\subsection{Common Transmission Modes in Amateur Radio Satellites}
\label{T8B04}

\begin{tcolorbox}[colback=gray!10!white,colframe=black!75!black,title=T8B04]
What mode of transmission is commonly used by amateur radio satellites?
\begin{enumerate}[label=\Alph*)]
    \item SSB
    \item FM
    \item CW/data
    \item \textbf{All these choices are correct}
\end{enumerate}
\end{tcolorbox}

\subsubsection{Intuitive Explanation}
Imagine you have a walkie-talkie that can talk to satellites in space. These satellites are like super cool space radios that can use different ways to send and receive messages. Some use a method called SSB (Single Sideband), which is like talking in a clear, crisp voice. Others use FM (Frequency Modulation), which is like talking in a smooth, steady voice. And some use CW/data, which is like sending Morse code or computer data. So, the satellites can use all these methods depending on what they need to do. That’s why the correct answer is All these choices are correct!

\subsubsection{Advanced Explanation}
Amateur radio satellites utilize various transmission modes to optimize communication based on the specific requirements of the mission and the capabilities of the equipment. 

1. \textbf{SSB (Single Sideband)}: This mode is efficient in terms of bandwidth and power usage. It is commonly used for voice communication, especially in HF bands, but it is also applicable in satellite communication due to its efficiency.

2. \textbf{FM (Frequency Modulation)}: FM is widely used for voice communication in VHF and UHF bands. It provides a robust signal that is less susceptible to noise, making it suitable for satellite communication where signal clarity is crucial.

3. \textbf{CW/data}: Continuous Wave (CW) and data modes are used for digital communication, including Morse code and various digital protocols. These modes are essential for sending precise information and are often used in satellite telemetry and control.

The versatility of amateur radio satellites allows them to employ all these transmission modes depending on the specific application. Therefore, the correct answer is that all these choices are correct.

% Diagram prompt: A diagram showing different transmission modes (SSB, FM, CW/data) used by amateur radio satellites with examples of their applications.