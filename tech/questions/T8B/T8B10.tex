\subsection{LEO Satellite}
\label{T8B10}

\begin{tcolorbox}[colback=gray!10!white,colframe=black!75!black,title=T8B10]
What is a LEO satellite?
\begin{enumerate}[label=\Alph*)]
    \item A sun synchronous satellite
    \item A highly elliptical orbit satellite
    \item A satellite in low energy operation mode
    \item \textbf{A satellite in low earth orbit}
\end{enumerate}
\end{tcolorbox}

\subsubsection{Intuitive Explanation}
Imagine you're playing a game of catch with a friend. If you throw the ball really high, it takes a long time to come back down. But if you throw it just a little bit above your head, it comes back quickly. A LEO satellite is like that low throw—it's a satellite that orbits the Earth really close, so it zips around the planet super fast! This makes it great for things like taking pictures of the Earth or helping with communication.

\subsubsection{Advanced Explanation}
A Low Earth Orbit (LEO) satellite is a satellite that orbits the Earth at an altitude typically between 160 to 2,000 kilometers. This is much closer than other types of satellites, such as geostationary satellites, which orbit at around 35,786 kilometers. The lower altitude means that LEO satellites have a shorter orbital period, usually ranging from 90 minutes to 2 hours. 

The velocity \( v \) of a satellite in LEO can be calculated using the formula:
\[
v = \sqrt{\frac{GM}{r}}
\]
where \( G \) is the gravitational constant, \( M \) is the mass of the Earth, and \( r \) is the distance from the center of the Earth to the satellite. 

LEO satellites are commonly used for Earth observation, scientific research, and communication because their proximity to the Earth allows for high-resolution imaging and low-latency communication. However, because they are so close to the Earth, they experience significant atmospheric drag, which can shorten their operational lifespan.

% Prompt for diagram: A diagram showing the orbit of a LEO satellite compared to other types of satellite orbits (e.g., geostationary orbit) would be helpful here.