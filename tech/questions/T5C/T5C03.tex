\subsection{Energy Storage in a Magnetic Field}
\label{T5C03}

\begin{tcolorbox}[colback=gray!10!white,colframe=black!75!black,title=T5C03]
What describes the ability to store energy in a magnetic field?
\begin{enumerate}[noitemsep]
    \item Admittance
    \item Capacitance
    \item Resistance
    \item \textbf{Inductance}
\end{enumerate}
\end{tcolorbox}

\subsubsection*{Intuitive Explanation}
Think of a magnetic field like a sponge that can soak up energy. When you pass an electric current through a coil of wire, it creates a magnetic field around it. This magnetic field can store energy, just like a sponge holds water. The ability of the coil to store this energy in its magnetic field is called \textbf{inductance}.

\subsubsection*{Advanced Explanation}
Inductance is a property of an electrical conductor, typically a coil, that quantifies its ability to store energy in a magnetic field when an electric current flows through it. The energy stored in the magnetic field is given by the formula:

\[
E = \frac{1}{2} L I^2
\]

where \( E \) is the energy stored, \( L \) is the inductance, and \( I \) is the current. Inductance is measured in henries (H). The other options, admittance, capacitance, and resistance, relate to different electrical properties and do not describe the storage of energy in a magnetic field.