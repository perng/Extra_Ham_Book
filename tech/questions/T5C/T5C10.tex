\subsection{Power Calculation in DC Circuits}
\label{T5C10}

\begin{tcolorbox}[colback=gray!10!white,colframe=black!75!black,title=T5C10]
How much power is delivered by a voltage of 12 volts DC and a current of 2.5 amperes?
\begin{enumerate}[label=\Alph*)]
    \item 4.8 watts
    \item \textbf{30 watts}
    \item 14.5 watts
    \item 0.208 watts
\end{enumerate}
\end{tcolorbox}

\subsubsection{Intuitive Explanation}
Imagine you have a water hose (the voltage) and you're pushing water (the current) through it. The power is like how much water you can push through the hose in a certain amount of time. If you have a big hose (12 volts) and you're pushing a lot of water (2.5 amperes), you're going to get a lot of power! In this case, it's 30 watts. So, the more voltage and current you have, the more power you get. Easy, right?

\subsubsection{Advanced Explanation}
Power in a DC circuit can be calculated using the formula:
\[
P = V \times I
\]
where \( P \) is the power in watts, \( V \) is the voltage in volts, and \( I \) is the current in amperes.

Given:
\[
V = 12 \, \text{volts}, \quad I = 2.5 \, \text{amperes}
\]

Substitute the values into the formula:
\[
P = 12 \, \text{V} \times 2.5 \, \text{A} = 30 \, \text{watts}
\]

This calculation shows that the power delivered by the circuit is 30 watts. The relationship between voltage, current, and power is fundamental in electrical engineering and is described by Ohm's Law and the power formula. Understanding these relationships is crucial for designing and analyzing electrical circuits.

% Diagram prompt: A simple circuit diagram showing a DC voltage source connected to a resistor with labeled voltage and current values.