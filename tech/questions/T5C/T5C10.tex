\subsection{Power Calculation in DC Circuits}
\label{T5C10}

\begin{tcolorbox}[colback=gray!10!white,colframe=black!75!black,title=T5C10]
How much power is delivered by a voltage of 12 volts DC and a current of 2.5 amperes?
\begin{enumerate}[noitemsep]
    \item 4.8 watts
    \item \textbf{30 watts}
    \item 14.5 watts
    \item 0.208 watts
\end{enumerate}
\end{tcolorbox}

\subsubsection*{Intuitive Explanation}
Imagine you have a water hose (the voltage) and you're pushing water through it (the current). The power is like how much water you can push through the hose in a certain amount of time. If you have a strong hose (12 volts) and you're pushing a lot of water (2.5 amperes), you're going to get a lot of power!

\subsubsection*{Advanced Explanation}
In electrical circuits, power \( P \) is calculated using the formula:
\[
P = V \times I
\]
where \( V \) is the voltage in volts and \( I \) is the current in amperes. Given a voltage of 12 volts and a current of 2.5 amperes, the power can be calculated as follows:
\[
P = 12 \, \text{V} \times 2.5 \, \text{A} = 30 \, \text{watts}
\]
Thus, the correct answer is 30 watts.