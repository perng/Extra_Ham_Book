\subsection{Abbreviation for Kilohertz}
\label{T5C13}

\begin{tcolorbox}[colback=gray!10!white,colframe=black!75!black,title=T5C13]
What is the abbreviation for kilohertz?
\begin{enumerate}[label=\Alph*)]
    \item KHZ
    \item khz
    \item khZ
    \item \textbf{kHz}
\end{enumerate}
\end{tcolorbox}

\subsubsection{Intuitive Explanation}
Imagine you're talking about how fast something is vibrating, like a guitar string. If it's vibrating really fast, we say it's at a high frequency. Now, kilohertz is just a fancy way of saying a thousand vibrations per second. The abbreviation kHz is like a nickname for this term. The k stands for kilo, which means a thousand, and Hz stands for hertz, which is the unit for frequency. So, kHz is the correct way to write it, with a lowercase k and an uppercase Hz.

\subsubsection{Advanced Explanation}
In the International System of Units (SI), kilo is a prefix that denotes a factor of \(10^3\) or 1,000. The unit hertz (Hz) is the SI unit for frequency, defined as one cycle per second. When combining these, the correct abbreviation for kilohertz is kHz. The lowercase k is used for the prefix kilo, and the uppercase Hz is used for the unit hertz. This follows the standard SI convention for unit abbreviations, where prefixes are lowercase and units are capitalized if they are derived from a proper name (in this case, Heinrich Hertz).

% Diagram prompt: A simple diagram showing the relationship between frequency, hertz, and kilohertz could be helpful here.