\subsection{Current Calculation for Power Delivery}
\label{T5C11}

\begin{tcolorbox}[colback=gray!10!white,colframe=black!75!black,title=T5C11]
How much current is required to deliver 120 watts at a voltage of 12 volts DC?
\begin{enumerate}[label=\Alph*)]
    \item 0.1 amperes
    \item \textbf{10 amperes}
    \item 12 amperes
    \item 132 amperes
\end{enumerate}
\end{tcolorbox}

\subsubsection{Intuitive Explanation}
Imagine you have a water hose. The voltage is like the water pressure, and the current is how much water is flowing through the hose. If you want to water your garden with a certain amount of water (power), you need to figure out how much water (current) is flowing if the pressure (voltage) is 12. In this case, to get 120 units of water, you need 10 units of water flowing through the hose. So, the answer is 10 amperes!

\subsubsection{Advanced Explanation}
To determine the current required to deliver a specific power at a given voltage, we use the formula:

\[
P = V \times I
\]

where:
\begin{itemize}
    \item \( P \) is the power in watts (W),
    \item \( V \) is the voltage in volts (V),
    \item \( I \) is the current in amperes (A).
\end{itemize}

Given:
\[
P = 120 \, \text{W}, \quad V = 12 \, \text{V}
\]

We need to solve for \( I \):

\[
I = \frac{P}{V} = \frac{120 \, \text{W}}{12 \, \text{V}} = 10 \, \text{A}
\]

Thus, the current required is 10 amperes.

This formula is fundamental in electrical engineering and is derived from Ohm's Law, which relates voltage, current, and resistance in an electrical circuit. Understanding this relationship is crucial for designing and analyzing electrical systems.

% Prompt for generating a diagram: A simple circuit diagram showing a voltage source (12V), a resistor, and the current (10A) flowing through the circuit.