\subsection{Unit of Capacitance}
\label{T5C02}

\begin{tcolorbox}[colback=gray!10!white,colframe=black!75!black,title=T5C02]
What is the unit of capacitance?
\begin{enumerate}[label=\Alph*)]
    \item \textbf{The farad}
    \item The ohm
    \item The volt
    \item The henry
\end{enumerate}
\end{tcolorbox}

\subsubsection{Intuitive Explanation}
Imagine you have a water tank. The bigger the tank, the more water it can hold. Capacitance is like the size of a capacitor, which is a device that stores electrical energy. The unit of capacitance, the farad, tells us how much electrical water (charge) the capacitor can hold. So, just like you measure water in liters, we measure capacitance in farads!

\subsubsection{Advanced Explanation}
Capacitance (\(C\)) is defined as the ratio of the electric charge (\(Q\)) stored on a conductor to the electric potential (\(V\)) across it. Mathematically, this is expressed as:
\[
C = \frac{Q}{V}
\]
The unit of capacitance is the farad (F), named after the English physicist Michael Faraday. One farad is defined as one coulomb of charge stored per volt of potential difference. 

In practical circuits, capacitors often have values in microfarads (\(\mu F\)), nanofarads (\(nF\)), or picofarads (\(pF\)) because one farad is a very large unit. Understanding capacitance is crucial in designing circuits that require energy storage, filtering, or timing functions.

% Prompt for diagram: A simple diagram showing a capacitor connected to a battery, with labels for charge (Q), voltage (V), and capacitance (C).