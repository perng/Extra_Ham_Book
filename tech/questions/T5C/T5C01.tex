\subsection{Energy Storage in Electric Fields}
\label{T5C01}

\begin{tcolorbox}[colback=gray!10!white,colframe=black!75!black,title=T5C01]
What describes the ability to store energy in an electric field?
\begin{enumerate}[label=\Alph*)]
    \item Inductance
    \item Resistance
    \item Tolerance
    \item \textbf{Capacitance}
\end{enumerate}
\end{tcolorbox}

\subsubsection{Intuitive Explanation}
Imagine you have a water balloon. When you fill it with water, it stretches and stores the water inside. Now, think of an electric field as the balloon and energy as the water. Capacitance is like the balloon's ability to stretch and hold that water. So, capacitance is the ability to store energy in an electric field, just like the balloon stores water.

\subsubsection{Advanced Explanation}
Capacitance, denoted by \( C \), is a measure of a capacitor's ability to store electric charge and, consequently, energy in an electric field. The energy stored in a capacitor can be calculated using the formula:

\[
E = \frac{1}{2} C V^2
\]

where \( E \) is the energy stored, \( C \) is the capacitance, and \( V \) is the voltage across the capacitor. 

Capacitance is defined as the ratio of the electric charge \( Q \) on each conductor to the potential difference \( V \) between them:

\[
C = \frac{Q}{V}
\]

The unit of capacitance is the farad (F), named after the English physicist Michael Faraday. Capacitors are widely used in electronic circuits for energy storage, filtering, and timing applications.

% Diagram Prompt: Generate a diagram showing a simple capacitor with electric field lines between the plates.