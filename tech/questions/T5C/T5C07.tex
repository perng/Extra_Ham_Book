\subsection{Megahertz Abbreviation}
\label{T5C07}

\begin{tcolorbox}[colback=gray!10!white,colframe=black!75!black,title=T5C07]
What is the abbreviation for megahertz?
\begin{enumerate}[label=\Alph*)]
    \item MH
    \item mh
    \item Mhz
    \item \textbf{MHz}
\end{enumerate}
\end{tcolorbox}

\subsubsection{Intuitive Explanation}
Imagine you're talking about how fast your favorite radio station is broadcasting. Instead of saying megahertz every time, which is a mouthful, you can just say MHz. It's like calling your best friend Buddy instead of their full name. The correct abbreviation is MHz, where M stands for mega (which means a lot) and Hz stands for hertz (which is how we measure how fast something is happening). So, MHz is the cool, short way to say megahertz.

\subsubsection{Advanced Explanation}
The term megahertz (MHz) is a unit of frequency in the International System of Units (SI). The prefix mega (M) denotes a factor of \(10^6\), and hertz (Hz) is the unit of frequency, defined as one cycle per second. Therefore, 1 MHz equals \(10^6\) Hz. The correct abbreviation MHz follows the SI convention, where the prefix is capitalized (M) and the unit is in lowercase (Hz). This standardization ensures clarity and consistency in scientific and technical communications.

% Prompt for generating a diagram: A simple diagram showing the relationship between hertz, kilohertz, megahertz, and gigahertz on a frequency scale could be helpful here.