\subsection{Power Calculation with Voltage and Current}
\label{T5C09}

\begin{tcolorbox}[colback=gray!10!white,colframe=black!75!black,title=T5C09]
How much power is delivered by a voltage of 13.8 volts DC and a current of 10 amperes?
\begin{enumerate}[label=\Alph*)]
    \item \textbf{138 watts}
    \item 0.7 watts
    \item 23.8 watts
    \item 3.8 watts
\end{enumerate}
\end{tcolorbox}

\subsubsection{Intuitive Explanation}
Imagine you have a water hose. The voltage is like the water pressure, and the current is like the amount of water flowing through the hose. Power is how much work the water can do, like turning a water wheel. If you have a lot of pressure (13.8 volts) and a lot of water flowing (10 amperes), you can do a lot of work! So, the power is 138 watts, which is like saying the water wheel is spinning really fast.

\subsubsection{Advanced Explanation}
Power in an electrical circuit is calculated using the formula:
\[
P = V \times I
\]
where \( P \) is power in watts, \( V \) is voltage in volts, and \( I \) is current in amperes.

Given:
\[
V = 13.8 \, \text{volts}, \quad I = 10 \, \text{amperes}
\]
Substitute the values into the formula:
\[
P = 13.8 \, \text{volts} \times 10 \, \text{amperes} = 138 \, \text{watts}
\]
Thus, the power delivered is 138 watts.

This formula is derived from the basic principles of electrical power, where power is the rate at which electrical energy is transferred by an electric circuit. The unit of power, the watt, is named after James Watt, who contributed significantly to the development of the steam engine.

% Diagram prompt: A simple circuit diagram showing a voltage source of 13.8 volts connected to a resistor with a current of 10 amperes flowing through it, labeled with the power calculation.