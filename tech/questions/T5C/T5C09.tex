\subsection{Power Calculation in DC Circuits}
\label{T5C09}

\begin{tcolorbox}[colback=gray!10!white,colframe=black!75!black,title=T5C09]
How much power is delivered by a voltage of 13.8 volts DC and a current of 10 amperes?
\begin{enumerate}[noitemsep]
    \item \textbf{138 watts}
    \item 0.7 watts
    \item 23.8 watts
    \item 3.8 watts
\end{enumerate}
\end{tcolorbox}

\subsubsection*{Intuitive Explanation}
Imagine you have a water hose. The voltage is like the water pressure, and the current is like the amount of water flowing through the hose. Power is how much work the water can do, like turning a water wheel. If you have a lot of pressure (voltage) and a lot of water (current), you can do a lot of work (power). In this case, 13.8 volts and 10 amperes give you 138 watts of power.

\subsubsection*{Advanced Explanation}
In electrical circuits, power \( P \) is calculated using the formula:
\[
P = V \times I
\]
where \( V \) is the voltage in volts and \( I \) is the current in amperes. For this question, the voltage \( V \) is 13.8 volts and the current \( I \) is 10 amperes. Plugging these values into the formula:
\[
P = 13.8 \, \text{volts} \times 10 \, \text{amperes} = 138 \, \text{watts}
\]
Thus, the power delivered is 138 watts.