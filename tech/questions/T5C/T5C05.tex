\subsection{Unit of Impedance}
\label{T5C05}

\begin{tcolorbox}[colback=gray!10!white,colframe=black!75!black,title=T5C05]
What is the unit of impedance?
\begin{enumerate}[label=\Alph*]
    \item The volt
    \item The ampere
    \item The coulomb
    \item \textbf{The ohm}
\end{enumerate}
\end{tcolorbox}

\subsubsection{Intuitive Explanation}
Imagine you're trying to push a shopping cart through a crowded store. The crowd is like resistance, making it harder for you to move. Now, if the cart also has a wobbly wheel, that's like reactance, adding another layer of difficulty. Impedance is the total push-back you feel from both the crowd and the wobbly wheel. Just like you measure how hard it is to push the cart in push units, we measure impedance in ohms. So, the unit of impedance is the ohm!

\subsubsection{Advanced Explanation}
Impedance (\(Z\)) is a complex quantity that combines resistance (\(R\)) and reactance (\(X\)) in an AC circuit. It is defined as:
\[
Z = R + jX
\]
where \(j\) is the imaginary unit. The unit of resistance is the ohm (\(\Omega\)), and since reactance is also measured in ohms, the unit of impedance is naturally the ohm. 

In more detail, resistance is the opposition to the flow of current due to the material's properties, while reactance is the opposition due to the circuit's inductance (\(L\)) and capacitance (\(C\)). The total impedance is the vector sum of resistance and reactance, and its magnitude is given by:
\[
|Z| = \sqrt{R^2 + X^2}
\]
This equation shows that impedance is a combination of both resistive and reactive components, and its unit remains the ohm.

% Diagram Prompt: Generate a diagram showing a simple AC circuit with a resistor, inductor, and capacitor, labeled with their respective impedance components.