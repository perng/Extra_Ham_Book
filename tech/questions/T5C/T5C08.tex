\subsection{Formula for Electrical Power in a DC Circuit}
\label{T5C08}

\begin{tcolorbox}[colback=gray!10!white,colframe=black!75!black,title=T5C08]
What is the formula used to calculate electrical power (P) in a DC circuit?
\begin{enumerate}[label=\Alph*)]
    \item \textbf{P = I E}
    \item P = E / I
    \item P = E – I
    \item P = I + E
\end{enumerate}
\end{tcolorbox}

\subsubsection{Intuitive Explanation}
Imagine you have a water hose. The water flowing through the hose is like the current (I) in a circuit, and the pressure pushing the water is like the voltage (E). The power (P) is how much work the water can do, like turning a water wheel. The more water and the more pressure, the more work it can do. So, power is just the current multiplied by the voltage: \( P = I \times E \). Easy, right?

\subsubsection{Advanced Explanation}
In a DC circuit, electrical power \( P \) is the rate at which electrical energy is transferred by the circuit. It is calculated using the formula:
\[
P = I \times E
\]
where:
\begin{itemize}
    \item \( P \) is the power in watts (W),
    \item \( I \) is the current in amperes (A),
    \item \( E \) is the voltage in volts (V).
\end{itemize}

This formula is derived from the basic definition of power, which is the product of voltage and current. Voltage represents the potential energy per unit charge, and current represents the flow of charge. When these two are multiplied, the result is the rate of energy transfer, or power.

For example, if a circuit has a current of 2 A and a voltage of 12 V, the power would be:
\[
P = 2 \, \text{A} \times 12 \, \text{V} = 24 \, \text{W}
\]

This formula is fundamental in understanding how electrical devices consume energy and is widely used in electrical engineering and physics.

% Diagram prompt: Generate a simple DC circuit diagram with a battery (voltage source), a resistor, and an ammeter to illustrate the relationship between voltage, current, and power.