\subsection{Unit of Inductance}
\label{T5C04}

\begin{tcolorbox}[colback=gray!10!white,colframe=black!75!black,title=T5C04]
What is the unit of inductance?
\begin{enumerate}[label=\Alph*)]
    \item The coulomb
    \item The farad
    \item \textbf{The henry}
    \item The ohm
\end{enumerate}
\end{tcolorbox}

\subsubsection{Intuitive Explanation}
Imagine you have a coil of wire, like a slinky. When you push electricity through it, it doesn't just go straight through like water in a pipe. Instead, the coil kind of holds onto the electricity for a bit, like a lazy cat holding onto a toy. The unit that measures how much the coil can hold onto the electricity is called the henry. So, if someone asks you what the unit of inductance is, you can say, It's the henry, like the lazy cat's unit!

\subsubsection{Advanced Explanation}
Inductance is a property of an electrical conductor (often a coil) that quantifies its ability to store energy in a magnetic field when an electric current flows through it. The unit of inductance is the henry (H), named after the American scientist Joseph Henry. Mathematically, inductance \( L \) is defined by the relationship:

\[
V = L \frac{dI}{dt}
\]

where \( V \) is the voltage across the inductor, and \( \frac{dI}{dt} \) is the rate of change of current with respect to time. 

The henry is a derived unit in the International System of Units (SI) and can be expressed in terms of other SI base units as:

\[
1 \, \text{H} = 1 \, \frac{\text{kg} \cdot \text{m}^2}{\text{s}^2 \cdot \text{A}^2}
\]

This means that one henry is equivalent to one kilogram meter squared per second squared per ampere squared. Understanding inductance is crucial in designing circuits, especially in applications involving transformers, inductors, and AC circuits.

% Prompt for generating a diagram: A diagram showing a simple inductor (coil) with current flowing through it, and the magnetic field lines around the coil.