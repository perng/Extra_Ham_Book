\subsection{License Expiration and Transmission on Amateur Radio Bands}\label{T1C11}

\begin{tcolorbox}[colback=gray!10!white,colframe=black!75!black,title=T1C11]
If your license has expired and is still within the allowable grace period, may you continue to transmit on the amateur radio bands?
\begin{enumerate}[label=\Alph*)]
    \item Yes, for up to two years
    \item Yes, as soon as you apply for renewal
    \item Yes, for up to one year
    \item \textbf{No, you must wait until the license has been renewed}
\end{enumerate}
\end{tcolorbox}

\subsubsection{Explanation}
In the context of amateur radio regulations, the Federal Communications Commission (FCC) in the United States mandates that operators must have a valid license to transmit on the amateur radio bands. Even if the license is within the grace period (typically 2 years for the FCC), the operator is not legally permitted to transmit until the license is officially renewed. This is to ensure that all operators are compliant with current regulations and standards.

The grace period is designed to allow operators to renew their licenses without penalty, but it does not grant them the authority to continue transmitting. The correct answer, therefore, is that you must wait until the license has been renewed before resuming transmission. This is a critical aspect of regulatory compliance in amateur radio operations.

% Prompt for diagram: A flowchart showing the process of license expiration, grace period, and renewal, with a clear indication that transmission is not allowed during the grace period.