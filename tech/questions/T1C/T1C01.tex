\subsection{Available License Classes from the FCC}
\label{T1C01}

\begin{tcolorbox}[colback=gray!10!white,colframe=black!75!black,title=T1C01]
For which license classes are new licenses currently available from the FCC?
\begin{enumerate}[label=\Alph*)]
    \item Novice, Technician, General, Amateur Extra
    \item Technician, Technician Plus, General, Amateur Extra
    \item Novice, Technician Plus, General, Advanced
    \item \textbf{Technician, General, Amateur Extra}
\end{enumerate}
\end{tcolorbox}

\subsubsection{Explanation}
The Federal Communications Commission (FCC) regulates amateur radio licenses in the United States. Over time, the FCC has phased out certain license classes, such as the Novice and Advanced licenses, due to changes in regulatory requirements and the evolving needs of the amateur radio community. Currently, the FCC issues new licenses for three classes:

\begin{itemize}
    \item \textbf{Technician Class}: This is the entry-level license, allowing access to all amateur radio frequencies above 30 MHz and limited privileges on HF (High Frequency) bands.
    \item \textbf{General Class}: This intermediate license grants additional HF band privileges, enabling more extensive communication capabilities.
    \item \textbf{Amateur Extra Class}: This is the highest level of amateur radio licensing, providing access to all amateur radio frequencies and modes with full privileges.
\end{itemize}

The Novice, Technician Plus, and Advanced licenses are no longer issued, making option D the correct answer. The FCC's decision to streamline the licensing process reflects the goal of simplifying the amateur radio licensing structure while maintaining a robust and skilled operator base.

% Prompt for generating a diagram: A flowchart showing the hierarchy of FCC amateur radio license classes, with the currently available licenses highlighted.