\subsection{Permitted International Communications for FCC-Licensed Amateur Radio Stations}
\label{T1C03}

\begin{tcolorbox}[colback=gray!10!white,colframe=black!75!black,title=T1C03]
What types of international communications are an FCC-licensed amateur radio station permitted to make?
\begin{enumerate}[label=\Alph*]
    \item \textbf{Communications incidental to the purposes of the Amateur Radio Service and remarks of a personal character}
    \item Communications incidental to conducting business or remarks of a personal nature
    \item Only communications incidental to contest exchanges; all other communications are prohibited
    \item Any communications that would be permitted by an international broadcast station
\end{enumerate}
\end{tcolorbox}

\subsubsection{Explanation}
The Federal Communications Commission (FCC) regulates amateur radio operations in the United States. According to FCC rules, amateur radio stations are permitted to engage in international communications that are incidental to the purposes of the Amateur Radio Service. This includes technical discussions, experimentation, and personal remarks. However, communications must not be used for business purposes or commercial gain. The Amateur Radio Service is intended for personal use and the advancement of radio technology, not for profit-making activities. Therefore, option A is correct as it aligns with the FCC's regulations on permissible communications for amateur radio operators.

% Prompt for generating a diagram: A flowchart showing the permitted and prohibited types of communications for FCC-licensed amateur radio stations.