\subsection{Who May Select a Desired Call Sign Under the Vanity Call Sign Rules?}
\label{T1C02}

\begin{tcolorbox}[colback=gray!10!white,colframe=black!75!black,title=T1C02]
Who may select a desired call sign under the vanity call sign rules?
\begin{enumerate}[label=\Alph*]
    \item Only a licensed amateur with a General or Amateur Extra Class license
    \item Only a licensed amateur with an Amateur Extra Class license
    \item Only a licensed amateur who has been licensed continuously for more than 10 years
    \item \textbf{Any licensed amateur}
\end{enumerate}
\end{tcolorbox}

\subsubsection{Explanation}
Under the Federal Communications Commission (FCC) rules, the vanity call sign program allows any licensed amateur radio operator to apply for a specific call sign of their choice, provided it is available and meets certain regulatory requirements. This program is designed to give operators the flexibility to personalize their call signs, which can be particularly useful for branding, memorability, or personal preference.

The key point here is that the eligibility for selecting a vanity call sign is not restricted by the class of license or the duration of licensure. Whether an operator holds a Technician, General, or Amateur Extra Class license, they are equally eligible to apply for a vanity call sign. This inclusivity ensures that all licensed amateurs, regardless of their experience or license class, have the opportunity to choose a call sign that resonates with them.

In summary, the correct answer is \textbf{D: Any licensed amateur}, as the vanity call sign rules do not impose restrictions based on license class or tenure.

% [Prompt for generating a diagram: A flowchart showing the process of applying for a vanity call sign, starting from eligibility check to final approval.]