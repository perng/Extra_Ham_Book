\subsection{Revocation of Station License or Suspension of Operator License}\label{T1C07}

\begin{tcolorbox}[colback=gray!10!white,colframe=black!75!black,title=T1C07]
Which of the following can result in revocation of the station license or suspension of the operator license?
\begin{enumerate}[label=\Alph*]
    \item Failure to inform the FCC of any changes in the amateur station following performance of an RF safety environmental evaluation
    \item \textbf{Failure to provide and maintain a correct email address with the FCC}
    \item Failure to obtain FCC type acceptance prior to using a home-built transmitter
    \item Failure to have a copy of your license available at your station
\end{enumerate}
\end{tcolorbox}

\subsubsection{Explanation}
The FCC requires all licensed amateur radio operators to maintain accurate contact information, including a valid email address. This is crucial for communication regarding regulatory updates, license renewals, and other official matters. Failure to provide and maintain a correct email address with the FCC is a violation of their rules and can lead to serious consequences, including the revocation of the station license or suspension of the operator license. This requirement ensures that the FCC can effectively communicate with licensees and enforce regulations.

While other options, such as failure to inform the FCC of changes in the amateur station or failure to obtain FCC type acceptance for a home-built transmitter, are also important, they do not typically result in immediate revocation or suspension. The requirement to have a copy of your license available at your station is a minor administrative rule and does not carry the same weight as maintaining accurate contact information.

% Diagram prompt: A flowchart showing the consequences of failing to maintain a correct email address with the FCC, leading to license revocation or suspension.