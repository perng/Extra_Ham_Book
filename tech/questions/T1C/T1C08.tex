\subsection{FCC-Issued Amateur Radio License Term}
\label{T1C08}

\begin{tcolorbox}[colback=gray!10!white,colframe=black!75!black,title=T1C08]
What is the normal term for an FCC-issued amateur radio license?
\begin{enumerate}[label=\Alph*)]
    \item Five years
    \item Life
    \item \textbf{Ten years}
    \item Eight years
\end{enumerate}
\end{tcolorbox}

\subsubsection{Explanation}
The Federal Communications Commission (FCC) is the regulatory body in the United States responsible for issuing amateur radio licenses. These licenses authorize individuals to operate amateur radio stations. The standard term for an FCC-issued amateur radio license is ten years. This duration is set to ensure that license holders remain updated with the latest regulations and technological advancements in the field of amateur radio.

The term of the license is defined under the FCC rules, specifically in Part 97 of the Code of Federal Regulations (CFR). According to 47 CFR §97.25, The term of an amateur station license is ten years from the date of issuance. This regulation ensures that license holders periodically renew their licenses, thereby maintaining a current understanding of the rules and practices governing amateur radio operations.

Renewal of the license involves a straightforward process where the licensee submits a renewal application to the FCC. This process helps the FCC keep an accurate record of active amateur radio operators and ensures compliance with the established regulations.

% Prompt for generating a diagram: A timeline showing the 10-year term of an FCC-issued amateur radio license, with milestones such as issuance, renewal, and expiration.