\subsection{Lost Contact?}
\label{T1C04}

\begin{tcolorbox}[colback=gray!10!white,colframe=black!75!black,title=T1C04]
What may happen if the FCC is unable to reach you by email?
\begin{enumerate}[label=\Alph*)]
    \item Fine and suspension of operator license
    \item \textbf{Revocation of the station license or suspension of the operator license}
    \item Revocation of access to the license record in the FCC system
    \item Nothing; there is no such requirement
\end{enumerate}
\end{tcolorbox}

\subsubsection{Explanation}
The FCC requires that licensees maintain accurate and up-to-date contact information, including a valid email address. This is crucial for regulatory compliance and communication. If the FCC is unable to reach a licensee via email, it may result in administrative actions under the Code of Federal Regulations (CFR). Specifically, according to 47 CFR §1.65, failure to maintain current contact information can lead to the revocation of the station license or suspension of the operator license. This ensures that all licensees remain accessible for regulatory oversight and emergency communications.

The revocation or suspension process involves a formal notice and an opportunity for the licensee to respond. If no response is received, the FCC may proceed with the revocation or suspension. This underscores the importance of maintaining accurate contact information and promptly responding to FCC communications.

% Diagram Prompt: A flowchart showing the process from FCC attempting to contact a licensee via email to the potential outcomes of revocation or suspension.