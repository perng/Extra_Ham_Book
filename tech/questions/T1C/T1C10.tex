\subsection{Transmission Timing After Amateur Radio License Examination}
\label{T1C10}

\begin{tcolorbox}[colback=gray!10!white,colframe=black!75!black,title=T1C10]
How soon after passing the examination for your first amateur radio license may you transmit on the amateur radio bands?
\begin{enumerate}[label=\Alph*)]
    \item Immediately on receiving your Certificate of Successful Completion of Examination (CSCE)
    \item As soon as your operator/station license grant appears on the ARRL website
    \item \textbf{As soon as your operator/station license grant appears in the FCC’s license database}
    \item As soon as you receive your license in the mail from the FCC
\end{enumerate}
\end{tcolorbox}

\subsubsection{Explanation}
The Federal Communications Commission (FCC) is the governing body that regulates amateur radio operations in the United States. After passing the examination, your license grant must be processed and entered into the FCC’s Universal Licensing System (ULS) database. This database is the official record of all licensed amateur radio operators. Only after your license grant appears in this database are you legally authorized to transmit on the amateur radio bands. 

The process involves several steps:
\begin{enumerate}
    \item Passing the examination and receiving a Certificate of Successful Completion of Examination (CSCE).
    \item The exam results are submitted to the FCC by the Volunteer Examiner Coordinator (VEC).
    \item The FCC processes the application and updates the ULS database.
    \item Once your license grant appears in the ULS database, you are officially licensed and can begin transmitting.
\end{enumerate}

It is important to note that the ARRL website and the physical license document are not the official sources for determining your licensing status. The FCC’s ULS database is the definitive source.

% Prompt for diagram: A flowchart showing the steps from passing the exam to being able to transmit, highlighting the FCC’s ULS database as the critical point.