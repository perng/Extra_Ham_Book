\subsection{Grace Period for Renewal of an Amateur License}
\label{T1C09}

\begin{tcolorbox}[colback=gray!10!white,colframe=black!75!black,title=T1C09]
What is the grace period for renewal if an amateur license expires?
\begin{enumerate}[label=\Alph*)]
    \item \textbf{Two years}
    \item Three years
    \item Five years
    \item Ten years
\end{enumerate}
\end{tcolorbox}

\subsubsection{Explanation}
The grace period for renewing an expired amateur radio license is governed by the Federal Communications Commission (FCC) regulations. According to these regulations, if an amateur radio license expires, the licensee has a grace period of two years to renew the license without having to retake the examination. This grace period is designed to provide flexibility for license holders who may have overlooked the renewal deadline.

The process of renewal during the grace period involves submitting the appropriate forms and fees to the FCC. If the renewal is completed within the two-year grace period, the licensee retains their original call sign and privileges. However, if the license is not renewed within this period, the licensee must reapply and pass the necessary examinations to obtain a new license.

This regulation ensures that amateur radio operators have ample time to maintain their licenses while also encouraging timely renewals to keep the amateur radio community active and compliant with FCC rules.