\subsection{Location! Location! Location!}
\label{T1C06}

\begin{tcolorbox}[colback=gray!10!white,colframe=black!75!black,title=T1C06]
From which of the following locations may an FCC-licensed amateur station transmit?
\begin{enumerate}[label=\Alph*)]
    \item From within any country that belongs to the International Telecommunication Union
    \item From within any country that is a member of the United Nations
    \item From anywhere within International Telecommunication Union (ITU) Regions 2 and 3
    \item \textbf{From any vessel or craft located in international waters and documented or registered in the United States}
\end{enumerate}
\end{tcolorbox}

\subsubsection{Explanation}
The Federal Communications Commission (FCC) regulates amateur radio operations in the United States. According to FCC rules, an FCC-licensed amateur station may transmit from any vessel or craft located in international waters, provided that the vessel or craft is documented or registered in the United States. This is specified in Part 97 of the FCC rules, which governs amateur radio service.

International waters are areas of the ocean that are not under the jurisdiction of any single country. When a vessel is in international waters, it is subject to the laws of the country in which it is registered. Therefore, an FCC-licensed amateur station can operate from such a vessel, as long as it is registered in the United States.

The other options (A, B, and C) are incorrect because the FCC does not have jurisdiction over amateur radio operations in other countries, regardless of their membership in international organizations like the International Telecommunication Union (ITU) or the United Nations. The FCC's authority is limited to the United States and its registered vessels or crafts in international waters.

% Diagram Prompt: A diagram showing a map with international waters highlighted and a boat registered in the United States transmitting from those waters.