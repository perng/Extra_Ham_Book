\subsection{Over the Air Access to IRLP Nodes}
\label{T8C06}

\begin{tcolorbox}[colback=gray!10!white,colframe=black!75!black,title=T8C06]
How is over the air access to IRLP nodes accomplished?
\begin{enumerate}[label=\Alph*]
    \item By obtaining a password that is sent via voice to the node
    \item \textbf{By using DTMF signals}
    \item By entering the proper internet password
    \item By using CTCSS tone codes
\end{enumerate}
\end{tcolorbox}

\subsubsection{Intuitive Explanation}
Imagine you have a secret clubhouse, and you want to let your friends in without shouting the password out loud. Instead, you use a special knock or a sequence of beeps that only your friends know. In the world of radio, IRLP nodes (which are like clubhouses for radio operators) use something called DTMF signals. These are like the beeps you hear when you press numbers on a phone. By sending these beeps over the air, you can knock on the IRLP node and gain access without anyone else knowing the secret code.

\subsubsection{Advanced Explanation}
IRLP (Internet Radio Linking Project) nodes are connected via the internet, but over-the-air access is managed using DTMF (Dual-Tone Multi-Frequency) signals. DTMF signals are generated by combining two specific frequencies, one from a high-frequency group and one from a low-frequency group. Each button on a telephone keypad corresponds to a unique pair of frequencies. For example, the number '1' corresponds to 697 Hz and 1209 Hz.

When a radio operator wants to access an IRLP node, they transmit the appropriate DTMF sequence using their radio. The IRLP node decodes these tones and grants access if the sequence matches the expected code. This method is secure and efficient, as it does not rely on voice passwords or internet-based authentication, which could be intercepted or require additional infrastructure.

% Diagram Prompt: Generate a diagram showing the process of accessing an IRLP node using DTMF signals, including the transmission of tones from the radio to the node and the node's response.