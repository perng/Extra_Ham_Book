\subsection{Amateur Radio Station Connecting to the Internet}
\label{T8C11}

\begin{tcolorbox}[colback=gray!10!white,colframe=black!75!black,title=T8C11]
What is an amateur radio station that connects other amateur stations to the internet?
\begin{enumerate}[noitemsep]
    \item \textbf{A gateway}
    \item A repeater
    \item A digipeater
    \item A beacon
\end{enumerate}
\end{tcolorbox}

\subsubsection*{Intuitive Explanation}
Imagine you have a walkie-talkie, but you want to talk to someone who is far away, maybe even on the other side of the world. A gateway in amateur radio is like a magical bridge that connects your walkie-talkie to the internet, allowing you to communicate with people who are not within the range of your radio. It’s like turning your radio into a phone that can call anyone, anywhere!

\subsubsection*{Advanced Explanation}
In amateur radio, a gateway is a specialized station that interfaces between radio frequencies and the internet. It allows radio operators to connect to digital networks, such as the Internet Protocol (IP) networks, enabling communication beyond the typical range of radio waves. This is particularly useful for modes like D-STAR, DMR, and other digital protocols that rely on internet connectivity to extend their reach. The gateway acts as a translator, converting radio signals into data packets that can be transmitted over the internet and vice versa. This technology bridges the gap between traditional radio communication and modern digital networks, enhancing the capabilities of amateur radio operators.