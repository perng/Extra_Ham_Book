\subsection{FT8 Transceiver Audio Connections}
\label{T4A04}

\begin{tcolorbox}[colback=gray!10!white,colframe=black!75!black,title=T4A04]
How are the transceiver audio input and output connected in a station configured to operate using FT8?
\begin{enumerate}[noitemsep]
    \item To a computer running a terminal program and connected to a terminal node controller unit
    \item \textbf{To the audio input and output of a computer running WSJT-X software}
    \item To an FT8 conversion unit, a keyboard, and a computer monitor
    \item To a computer connected to the FT8converter.com website
\end{enumerate}
\end{tcolorbox}

\subsubsection*{Explanation}
In a station configured to operate using FT8, the transceiver's audio input and output are connected directly to the audio input and output of a computer running WSJT-X software. This setup allows the computer to process the digital signals used in FT8 communication, which is a popular mode for weak signal communication in amateur radio. The WSJT-X software handles the encoding and decoding of the FT8 signals, making it the central component in this configuration.