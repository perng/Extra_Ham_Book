\subsection{Battery Power Duration Calculation}
\label{T4A09}

\begin{tcolorbox}[colback=gray!10!white,colframe=black!75!black,title=T4A09]
How can you determine the length of time that equipment can be powered from a battery?
\begin{enumerate}[noitemsep]
    \item Divide the watt-hour rating of the battery by the peak power consumption of the equipment
    \item \textbf{Divide the battery ampere-hour rating by the average current draw of the equipment}
    \item Multiply the watts per hour consumed by the equipment by the battery power rating
    \item Multiply the square of the current rating of the battery by the input resistance of the equipment
\end{enumerate}
\end{tcolorbox}

\subsubsection*{Intuitive Explanation}
To figure out how long a battery will last, think of it like a water tank. The ampere-hour (Ah) rating of the battery is like the size of the tank, and the average current draw of your equipment is like the rate at which water is being used. To find out how long the water will last, you simply divide the size of the tank by the rate of water usage. Similarly, to find out how long the battery will last, you divide the ampere-hour rating by the average current draw.

\subsubsection*{Advanced Explanation}
The ampere-hour (Ah) rating of a battery indicates the total charge it can deliver over time. For example, a 10 Ah battery can deliver 10 amperes for 1 hour, or 1 ampere for 10 hours. The average current draw of the equipment is the steady current it consumes during operation. To determine the duration the battery can power the equipment, you use the formula:

\[
\text{Duration (hours)} = \frac{\text{Battery Ah Rating}}{\text{Average Current Draw (A)}}
\]

This formula assumes that the battery is fully charged and that the current draw remains constant. Variations in current draw or battery efficiency can affect the actual duration.