\subsection{Preferred Conductor for RF Bonding}
\label{T4A08}

\begin{tcolorbox}[colback=gray!10!white,colframe=black!75!black,title=T4A08]
Which of the following conductors is preferred for bonding at RF?
\begin{enumerate}[noitemsep]
    \item Copper braid removed from coaxial cable
    \item Steel wire
    \item Twisted-pair cable
    \item \textbf{Flat copper strap}
\end{enumerate}
\end{tcolorbox}

\subsubsection*{Intuitive Explanation}
When dealing with RF (Radio Frequency) bonding, you want a conductor that can handle high frequencies efficiently. Think of it like choosing the right hose for watering your garden. A flat, wide hose (like a flat copper strap) will allow water (or in this case, RF signals) to flow smoothly without any kinks or resistance. On the other hand, a twisted hose (like twisted-pair cable) or a rusty hose (like steel wire) would cause problems. Copper braid might seem like a good option, but it’s not as effective as a flat copper strap for RF bonding.

\subsubsection*{Advanced Explanation}
In RF systems, bonding conductors are used to ensure a low-impedance path for RF currents. The choice of conductor is critical to minimize losses and avoid interference. A flat copper strap is preferred because it provides a large surface area, which reduces skin effect—a phenomenon where RF currents tend to flow on the surface of the conductor. This large surface area also helps in reducing inductance, which is crucial for maintaining a low impedance at high frequencies. Copper braid, while conductive, does not offer the same surface area and can introduce additional inductance. Steel wire is not ideal due to its higher resistance and lower conductivity compared to copper. Twisted-pair cable is designed for differential signaling and is not suitable for RF bonding. Therefore, the flat copper strap is the best choice for effective RF bonding.