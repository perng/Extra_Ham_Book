\subsection{Transmission of Assigned Call Sign}\label{T1F03}

\begin{tcolorbox}[colback=gray!10!white,colframe=black!75!black,title=T1F03]
When are you required to transmit your assigned call sign?
\begin{enumerate}[label=\Alph*)]
    \item At the beginning of each contact, and every 10 minutes thereafter
    \item At least once during each transmission
    \item At least every 15 minutes during and at the end of a communication
    \item \textbf{At least every 10 minutes during and at the end of a communication}
\end{enumerate}
\end{tcolorbox}

\subsubsection{Intuitive Explanation}
Imagine you're playing a game of tag, and you need to shout your name every 10 minutes so everyone knows you're still in the game. In radio communication, it's kind of like that! You need to say your call sign (which is like your radio name) at least every 10 minutes and when you're done talking. This way, everyone knows who's talking and that you're still on the air. It's like saying, Hey, it's me, and I'm still here!

\subsubsection{Advanced Explanation}
In radio communication, transmitting your assigned call sign is a regulatory requirement to ensure proper identification of the station. According to the Federal Communications Commission (FCC) rules, you must transmit your call sign at least every 10 minutes during a communication and at the end of the communication. This rule helps in maintaining order and accountability in the airwaves.

The rationale behind this rule is to prevent confusion and ensure that all transmissions can be traced back to their source. This is particularly important in emergency situations or when interference occurs. The 10-minute interval is chosen to balance the need for frequent identification without being overly burdensome to the operator.

Mathematically, if you start a transmission at time \( t_0 \), you must transmit your call sign at \( t_0 + 10 \) minutes, \( t_0 + 20 \) minutes, and so on, until the end of the communication. This ensures compliance with the regulation.

Related concepts include the importance of call signs in amateur radio, the role of the FCC in regulating radio communications, and the technical aspects of transmitting signals. Understanding these concepts helps in appreciating the necessity of the rule and its implementation in practice.

% Prompt for generating a diagram: A timeline showing the transmission of a call sign at 10-minute intervals during a communication.