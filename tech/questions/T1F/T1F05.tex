\subsection{Call Sign Identification for Phone Signals}
\label{T1F05}

\begin{tcolorbox}[colback=gray!10!white,colframe=black!75!black,title=T1F05]
What method of call sign identification is required for a station transmitting phone signals?
\begin{enumerate}[label=\Alph*)]
    \item Send the call sign followed by the indicator RPT
    \item \textbf{Send the call sign using a CW or phone emission}
    \item Send the call sign followed by the indicator R
    \item Send the call sign using only a phone emission
\end{enumerate}
\end{tcolorbox}

\subsubsection{Intuitive Explanation}
Imagine you're at a party and you want to let everyone know who you are. You could shout your name, or you could write it on a piece of paper and pass it around. In radio terms, shouting your name is like using a phone signal, and writing it down is like using Morse code (CW). The rules say you can do either one to let people know who's talking. So, whether you shout or write, you're good to go!

\subsubsection{Advanced Explanation}
In radio communication, the identification of a station is crucial for regulatory compliance and operational clarity. The Federal Communications Commission (FCC) mandates that a station transmitting phone signals must identify itself by sending its call sign. This can be done using either Continuous Wave (CW) emissions, commonly known as Morse code, or phone emissions, which are voice transmissions. The key point is that the identification must be clear and unambiguous. The use of CW allows for identification even in noisy or weak signal conditions, while phone emissions are straightforward for voice communications. The correct method, therefore, is to send the call sign using either CW or phone emissions, ensuring that the station is properly identified regardless of the transmission method.

% Prompt for diagram: A diagram showing a radio operator using both a microphone (phone emission) and a Morse code key (CW emission) to send a call sign could be helpful here.