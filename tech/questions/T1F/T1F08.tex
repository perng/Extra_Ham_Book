\subsection{Definition of Third Party Communications}
\label{T1F08}

\begin{tcolorbox}[colback=gray!10!white,colframe=black!75!black,title=T1F08]
What is the definition of third party communications?
\begin{enumerate}[label=\Alph*,noitemsep]
    \item \textbf{A message from a control operator to another amateur station control operator on behalf of another person}
    \item Amateur radio communications where three stations are in communications with one another
    \item Operation when the transmitting equipment is licensed to a person other than the control operator
    \item Temporary authorization for an unlicensed person to transmit on the amateur bands for technical experiments
\end{enumerate}
\end{tcolorbox}

Third party communications in amateur radio refer to the scenario where a licensed control operator sends a message on behalf of another person to another amateur station. This is a common practice and is allowed under specific regulations to facilitate communication for non-licensed individuals through licensed operators.