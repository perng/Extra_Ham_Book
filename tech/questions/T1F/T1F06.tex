\subsection{Acceptable Self-Assigned Indicators in Phone Transmission}\label{T1F06}

\begin{tcolorbox}[colback=gray!10!white,colframe=black!75!black,title=T1F06]
Which of the following self-assigned indicators are acceptable when using a phone transmission?
\begin{enumerate}[label=\Alph*)]
    \item KL7CC stroke W3
    \item KL7CC slant W3
    \item KL7CC slash W3
    \item \textbf{All these choices are correct}
\end{enumerate}
\end{tcolorbox}

\subsubsection{Intuitive Explanation}
Imagine you're playing a game where you need to introduce yourself with a special code. You can use different symbols like a stroke, slant, or slash to make your code unique. The cool part? All of these symbols are allowed! So, whether you choose a stroke, slant, or slash, you're good to go. It's like picking your favorite flavor of ice cream—any choice is a winner!

\subsubsection{Advanced Explanation}
In radio communication, self-assigned indicators are used to uniquely identify a station when operating in a different location or under special conditions. The terms stroke, slant, and slash are all acceptable ways to denote this separation in the call sign. According to the International Telecommunication Union (ITU) and FCC regulations, these indicators are interchangeable and serve the same purpose. Therefore, any of these forms—KL7CC stroke W3, KL7CC slant W3, or KL7CC slash W3—are considered valid and acceptable.


% Prompt for diagram: No diagram is necessary for this question as it is purely conceptual.