\subsection{Cold Tin-Lead Solder Joint Appearance}
\label{T7D09}

\begin{tcolorbox}[colback=gray!10!white,colframe=black!75!black,title=T7D09]
What is the characteristic appearance of a cold tin-lead solder joint?
\begin{enumerate}[label=\Alph*]
    \item Dark black spots
    \item A bright or shiny surface
    \item \textbf{A rough or lumpy surface}
    \item Excessive solder
\end{enumerate}
\end{tcolorbox}

\subsubsection{Intuitive Explanation}
Imagine you're trying to glue two pieces of paper together, but you don't press them firmly enough. The glue doesn't spread evenly, and you end up with a bumpy, uneven surface. That's pretty much what happens with a cold solder joint! When the solder doesn't get hot enough, it doesn't flow smoothly, and you end up with a rough, lumpy mess instead of a nice, shiny connection.

\subsubsection{Advanced Explanation}
A cold solder joint occurs when the solder does not reach its optimal melting temperature, typically around 183$^\circ$C for tin-lead solder. This results in insufficient wetting of the surfaces to be joined, leading to poor adhesion and a rough, uneven surface. The solder does not flow properly, causing it to solidify in a non-uniform manner. This can be due to insufficient heat from the soldering iron, improper technique, or contamination on the surfaces being soldered. 

The rough or lumpy surface is a clear indicator of a cold joint, as the solder has not properly bonded with the metal surfaces. This type of joint is mechanically weak and can lead to electrical failures over time. Proper soldering technique involves ensuring the soldering iron is at the correct temperature, cleaning the surfaces to be soldered, and applying the solder evenly to achieve a smooth, shiny joint.

% Diagram prompt: A side-by-side comparison of a cold solder joint (rough and lumpy) and a proper solder joint (smooth and shiny).