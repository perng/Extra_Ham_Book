\subsection{Precautions for Measuring In-Circuit Resistance}
\label{T7D11}

\begin{tcolorbox}[colback=gray!10!white,colframe=black!75!black,title=T7D11]
Which of the following precautions should be taken when measuring in-circuit resistance with an ohmmeter?
\begin{enumerate}[label=\Alph*)]
    \item Ensure that the applied voltages are correct
    \item \textbf{Ensure that the circuit is not powered}
    \item Ensure that the circuit is grounded
    \item Ensure that the circuit is operating at the correct frequency
\end{enumerate}
\end{tcolorbox}

\subsubsection{Intuitive Explanation}
Imagine you're trying to measure how much a door resists being pushed open. If someone is already pushing the door while you're measuring, your measurement will be all wrong! Similarly, when you're using an ohmmeter to measure resistance in a circuit, you need to make sure the circuit isn't pushing or pulling any electricity. That means the circuit should be turned off. If it's powered on, the ohmmeter will get confused and give you a wrong reading. So, always remember: no power, proper measurement!

\subsubsection{Advanced Explanation}
When measuring resistance with an ohmmeter, the device sends a small known current through the circuit and measures the voltage drop to calculate the resistance using Ohm's Law:
\[
R = \frac{V}{I}
\]
where \( R \) is the resistance, \( V \) is the voltage, and \( I \) is the current. If the circuit is powered, external voltages and currents can interfere with the ohmmeter's measurements, leading to inaccurate results. Additionally, the internal components of the ohmmeter are designed to handle only the small current it generates, and external power sources could damage the device. Therefore, it is crucial to ensure that the circuit is not powered when measuring resistance in-circuit.

% Diagram prompt: Generate a diagram showing an ohmmeter connected to a circuit with a power source turned off.