\subsection{Ohmmeter Reading Across a Discharged Capacitor}
\label{T7D10}

\begin{tcolorbox}[colback=gray!10!white,colframe=black!75!black,title=T7D10]
What reading indicates that an ohmmeter is connected across a large, discharged capacitor?
\begin{enumerate}[noitemsep]
    \item \textbf{Increasing resistance with time}
    \item Decreasing resistance with time
    \item Steady full-scale reading
    \item Alternating between open and short circuit
\end{enumerate}
\end{tcolorbox}

\subsubsection*{Intuitive Explanation}
Imagine you have a big empty bucket (the capacitor) and you start filling it with water (charge). At first, the bucket is empty, so it’s easy to pour water in. As the bucket fills up, it becomes harder to add more water. Similarly, when you connect an ohmmeter to a discharged capacitor, it initially shows a low resistance because the capacitor is empty and ready to accept charge. As the capacitor charges up, the resistance increases because it’s getting full and less willing to accept more charge.

\subsubsection*{Advanced Explanation}
When an ohmmeter is connected across a discharged capacitor, it applies a small voltage to the capacitor and measures the resulting current. Initially, the capacitor acts like a short circuit because it is discharged, and the current is high, indicating low resistance. As the capacitor charges, the current decreases, and the resistance appears to increase. This is because the capacitor is storing charge, and the voltage across it builds up, opposing the applied voltage from the ohmmeter. Therefore, the correct reading is an increasing resistance with time.

% Diagram prompt: Generate a graph showing the resistance over time when an ohmmeter is connected to a discharged capacitor. Use Python with Matplotlib for the graph. The x-axis should represent time, and the y-axis should represent resistance. The graph should show an increasing curve, starting from a low resistance value and gradually increasing over time.