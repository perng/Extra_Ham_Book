\subsection{Measuring Electric Potential}
\label{T7D01}

\begin{tcolorbox}[colback=gray!10!white,colframe=black!75!black,title=T7D01]
Which instrument would you use to measure electric potential?  
\begin{enumerate}[label=\Alph*)]
    \item An ammeter
    \item \textbf{A voltmeter}
    \item A wavemeter
    \item An ohmmeter
\end{enumerate}
\end{tcolorbox}

\subsubsection{Intuitive Explanation}
Imagine you have a water tank, and you want to know how much pressure the water has at the bottom. You wouldn’t use a ruler or a thermometer, right? You’d use a pressure gauge! Similarly, in the world of electricity, if you want to measure how much pressure (or electric potential) is in a circuit, you use a voltmeter. It’s like the pressure gauge for electricity!

\subsubsection{Advanced Explanation}
Electric potential, measured in volts (V), represents the potential energy per unit charge in an electric field. To measure this, a voltmeter is used. A voltmeter is designed to have a very high internal resistance, ensuring it draws minimal current from the circuit, thus not affecting the measurement significantly. 

Mathematically, electric potential \( V \) is defined as:
\[
V = \frac{W}{q}
\]
where \( W \) is the work done to move a charge \( q \) from one point to another. A voltmeter measures this potential difference between two points in a circuit.

Other instruments mentioned in the question serve different purposes:
\begin{itemize}
    \item An ammeter measures current (in amperes).
    \item A wavemeter measures the wavelength of electromagnetic waves.
    \item An ohmmeter measures resistance (in ohms).
\end{itemize}

% Prompt for diagram: A simple circuit diagram showing a battery, a resistor, and a voltmeter connected in parallel to measure the potential difference across the resistor.