\subsection{Types of Solder for Radio and Electronic Applications}
\label{T7D08}

\begin{tcolorbox}[colback=gray!10!white,colframe=black!75!black,title=T7D08]
Which of the following types of solder should not be used for radio and electronic applications?
\begin{enumerate}[noitemsep]
    \item \textbf{Acid-core solder}
    \item Lead-tin solder
    \item Rosin-core solder
    \item Tin-copper solder
\end{enumerate}
\end{tcolorbox}

\subsubsection{Intuitive Explanation}
When working with electronics, you want to use solder that won't damage your components. Acid-core solder is like using a strong cleaning agent on a delicate piece of art—it’s too harsh and can cause corrosion. That’s why it’s a no-go for radio and electronic applications.

\subsubsection{Advanced Explanation}
Acid-core solder contains a flux that is highly corrosive, which is suitable for plumbing but detrimental to electronic components. The acid can cause long-term damage by corroding the metal traces on circuit boards and the leads of components. In contrast, rosin-core solder is specifically designed for electronics, as its flux is non-corrosive and helps ensure a clean, reliable connection. Lead-tin and tin-copper solders are also commonly used in electronics, depending on the specific requirements of the application.