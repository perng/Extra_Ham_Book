\subsection{Damage to a Multimeter}
\label{T7D06}

\begin{tcolorbox}[colback=gray!10!white,colframe=black!75!black,title=T7D06]
Which of the following can damage a multimeter?
\begin{enumerate}[noitemsep]
    \item Attempting to measure resistance using the voltage setting
    \item Failing to connect one of the probes to ground
    \item \textbf{Attempting to measure voltage when using the resistance setting}
    \item Not allowing it to warm up properly
\end{enumerate}
\end{tcolorbox}

\subsubsection*{Intuitive Explanation}
Think of a multimeter like a Swiss Army knife for electrical measurements. Each setting is like a different tool on the knife. If you try to use the wrong tool for the job, you might break it. For example, using the resistance setting to measure voltage is like trying to cut a steak with a screwdriver—it’s not going to end well for the screwdriver!

\subsubsection*{Advanced Explanation}
Multimeters are designed to handle specific types of measurements within certain ranges. The resistance setting on a multimeter typically applies a small voltage to the circuit to measure resistance. If you attempt to measure voltage while in the resistance setting, the multimeter may be exposed to a much higher voltage than it is designed to handle, potentially damaging the internal components. This is why it is crucial to ensure the multimeter is set to the correct measurement type before taking any readings.