\subsection{Instrument for Measuring Electric Current}
\label{T7D04}

\begin{tcolorbox}[colback=gray!10!white,colframe=black!75!black,title=T7D04]
Which instrument is used to measure electric current?  
\begin{enumerate}[label=\Alph*)]
    \item An ohmmeter
    \item An electrometer
    \item A voltmeter
    \item \textbf{An ammeter}
\end{enumerate}
\end{tcolorbox}

\subsubsection*{Intuitive Explanation}
Imagine you’re trying to figure out how much water is flowing through a pipe. You’d use a flow meter, right? Well, electric current is like the flow of water, but instead of water, it’s tiny particles called electrons moving through a wire. To measure this flow of electrons, you need a special tool called an ammeter. It’s like the flow meter for electricity! So, if you want to know how much electric current is zipping through a circuit, grab an ammeter—it’s your go-to gadget.

\subsubsection*{Advanced Explanation}
Electric current, denoted by the symbol \( I \), is the rate of flow of electric charge through a conductor. It is measured in amperes (A). To measure this current, an ammeter is used. An ammeter is designed to be connected in series with the circuit, allowing it to measure the current flowing through the circuit directly. 

The working principle of an ammeter is based on the magnetic effect of electric current. When current flows through a coil in the ammeter, it generates a magnetic field that causes a needle to move, indicating the current value on a calibrated scale. Modern digital ammeters use electronic circuits to measure current and display the value digitally.

In contrast:
\begin{itemize}
    \item An ohmmeter measures resistance (\( R \)) in ohms (\( \Omega \)).
    \item An electrometer measures electric charge or voltage with high sensitivity.
    \item A voltmeter measures the potential difference (\( V \)) between two points in a circuit.
\end{itemize}

Thus, the correct instrument for measuring electric current is the ammeter.

% [Prompt for diagram: A simple circuit diagram showing an ammeter connected in series to measure current.]