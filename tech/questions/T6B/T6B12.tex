\subsection{Electrodes of a Bipolar Junction Transistor}
\label{T6B12}

\begin{tcolorbox}[colback=gray!10!white,colframe=black!75!black,title=T6B12]
What are the names of the electrodes of a bipolar junction transistor?
\begin{enumerate}[label=\Alph*)]
    \item Signal, bias, power
    \item \textbf{Emitter, base, collector}
    \item Input, output, supply
    \item Pole one, pole two, output
\end{enumerate}
\end{tcolorbox}

\subsubsection{Intuitive Explanation}
Imagine a bipolar junction transistor (BJT) as a tiny traffic controller for electricity. It has three main parts, just like a traffic light has three colors. The first part is the \textbf{emitter}, which is like the green light—it lets the cars (electrons) go. The second part is the \textbf{base}, which is like the yellow light—it controls when the cars can start moving. The last part is the \textbf{collector}, which is like the red light—it stops the cars when needed. So, the three electrodes are the emitter, base, and collector, and they work together to manage the flow of electricity.

\subsubsection{Advanced Explanation}
A bipolar junction transistor (BJT) is a semiconductor device with three terminals: the \textbf{emitter}, \textbf{base}, and \textbf{collector}. These terminals correspond to the three doped regions of the transistor: the emitter region, the base region, and the collector region. The emitter is heavily doped and emits charge carriers (electrons or holes) into the base. The base is lightly doped and controls the flow of charge carriers from the emitter to the collector. The collector is moderately doped and collects the charge carriers that pass through the base.

The operation of a BJT can be understood through the following steps:
\begin{enumerate}
    \item When a small current is applied to the base-emitter junction, it allows a larger current to flow from the collector to the emitter.
    \item The base-emitter junction is forward-biased, while the base-collector junction is reverse-biased.
    \item The current amplification factor, denoted as $\beta$, is the ratio of the collector current ($I_C$) to the base current ($I_B$), i.e., $\beta = \frac{I_C}{I_B}$.
\end{enumerate}

The BJT can operate in three modes: active, saturation, and cutoff. In the active mode, the transistor acts as an amplifier. In the saturation mode, it acts as a closed switch, allowing maximum current flow. In the cutoff mode, it acts as an open switch, blocking current flow.

% Diagram prompt: Generate a diagram showing the structure of a bipolar junction transistor with labeled emitter, base, and collector regions.