\subsection{Component for Unidirectional Current Flow}
\label{T6B02}

\begin{tcolorbox}[colback=gray!10!white,colframe=black!75!black,title=T6B02]
What electronic component allows current to flow in only one direction?
\begin{enumerate}[noitemsep]
    \item Resistor
    \item Fuse
    \item \textbf{Diode}
    \item Driven element
\end{enumerate}
\end{tcolorbox}

\subsubsection*{Intuitive Explanation}
Imagine a one-way street for electricity. A diode is like a traffic cop that only lets cars (electrons) go in one direction. If they try to go the wrong way, the cop stops them!

\subsubsection*{Advanced Explanation}
A diode is a semiconductor device that allows current to flow in one direction (forward bias) while blocking it in the opposite direction (reverse bias). This is due to the formation of a p-n junction within the diode, which creates a potential barrier. When the diode is forward-biased, the barrier is lowered, allowing current to flow. In reverse bias, the barrier is high, preventing current flow. This property makes diodes essential in circuits where unidirectional current flow is required, such as in rectifiers.