\subsection{Transistor Types with Gate, Drain, and Source}
\label{T6B05}

\begin{tcolorbox}[colback=gray!10!white,colframe=black!75!black,title=T6B05]
What type of transistor has a gate, drain, and source?
\begin{enumerate}[noitemsep]
    \item Varistor
    \item \textbf{Field-effect}
    \item Tesla-effect
    \item Bipolar junction
\end{enumerate}
\end{tcolorbox}

\subsubsection*{Intuitive Explanation}
Think of a transistor as a tiny switch that controls the flow of electricity. In a Field-Effect Transistor (FET), the gate is like the handle of the switch. When you turn the handle (apply voltage to the gate), it opens or closes the path for electricity to flow between the drain and the source. It's like turning a faucet to control water flow!

\subsubsection*{Advanced Explanation}
A Field-Effect Transistor (FET) is a type of transistor that uses an electric field to control the flow of current. It has three terminals: the gate, drain, and source. The gate controls the conductivity between the drain and source by modulating the electric field within the device. FETs are widely used in electronics due to their high input impedance and low power consumption. Unlike Bipolar Junction Transistors (BJTs), which use both electrons and holes for conduction, FETs primarily use one type of charge carrier, making them more efficient in certain applications.