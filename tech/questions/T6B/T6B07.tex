\subsection{LED Light Emission Mechanism}
\label{T6B07}

\begin{tcolorbox}[colback=gray!10!white,colframe=black!75!black,title=T6B07]
What causes a light-emitting diode (LED) to emit light?
\begin{enumerate}[label=\Alph*)]
    \item \textbf{Forward current}
    \item Reverse current
    \item Capacitively-coupled RF signal
    \item Inductively-coupled RF signal
\end{enumerate}
\end{tcolorbox}

\subsubsection{Intuitive Explanation}
Imagine an LED as a tiny light bulb that only turns on when you push electricity through it the right way. If you try to push electricity the wrong way, it’s like trying to blow air into a balloon that’s already full—nothing happens! The LED needs a forward current, which is like giving it a gentle push in the right direction, to light up. So, the correct answer is forward current—it’s the magic switch that makes the LED glow!

\subsubsection{Advanced Explanation}
An LED (Light-Emitting Diode) is a semiconductor device that emits light when an electric current passes through it in the forward direction. This phenomenon is known as electroluminescence. When a forward bias voltage is applied across the LED, electrons from the n-type semiconductor recombine with holes in the p-type semiconductor. During this recombination process, energy is released in the form of photons, which is the light we see.

The relationship between the energy of the emitted photons and the band gap of the semiconductor material is given by the equation:
\[
E = h \nu
\]
where \(E\) is the energy of the photon, \(h\) is Planck's constant, and \(\nu\) is the frequency of the emitted light. The band gap energy \(E_g\) of the semiconductor determines the wavelength \(\lambda\) of the emitted light:
\[
\lambda = \frac{hc}{E_g}
\]
where \(c\) is the speed of light.

In summary, the forward current causes electrons and holes to recombine, releasing energy as light. This is why the correct answer is forward current.

% Diagram prompt: Generate a diagram showing the forward bias of an LED, illustrating electron-hole recombination and photon emission.