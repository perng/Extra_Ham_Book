\subsection{Diode Electrodes}
\label{T6B09}

\begin{tcolorbox}[colback=gray!10!white,colframe=black!75!black,title=T6B09]
What are the names for the electrodes of a diode?
\begin{enumerate}[noitemsep]
    \item Plus and minus
    \item Source and drain
    \item \textbf{Anode and cathode}
    \item Gate and base
\end{enumerate}
\end{tcolorbox}

The question is straightforward and requires knowledge of basic electronic components. A diode is a semiconductor device that allows current to flow in one direction only. The two electrodes of a diode are called the anode and the cathode. The anode is the positive terminal, and the cathode is the negative terminal. This naming convention is standard in electronics and is essential for understanding how diodes function in circuits.