\subsection{Long-Distance Voice Mode on VHF and UHF Bands}
\label{T8A03}

\begin{tcolorbox}[colback=gray!10!white,colframe=black!75!black,title=T8A03]
Which type of voice mode is often used for long-distance (weak signal) contacts on the VHF and UHF bands?
\begin{enumerate}[noitemsep]
    \item FM
    \item DRM
    \item \textbf{SSB}
    \item PM
\end{enumerate}
\end{tcolorbox}

\subsubsection*{Intuitive Explanation}
When you're trying to talk to someone far away on VHF or UHF bands, you need a mode that can cut through the noise and make your signal as clear as possible. Think of it like shouting across a noisy room—you want to use the most efficient way to get your message across. Single Sideband (SSB) is like a focused shout that uses less power and bandwidth, making it ideal for long-distance communication when signals are weak.

\subsubsection*{Advanced Explanation}
Single Sideband (SSB) is a modulation technique that transmits only one sideband of the carrier wave, either the upper or lower sideband, while suppressing the other sideband and the carrier itself. This results in a more efficient use of power and bandwidth compared to other modes like FM (Frequency Modulation) or PM (Phase Modulation). SSB is particularly effective for long-distance communication on VHF and UHF bands because it reduces the signal's susceptibility to noise and interference, allowing for clearer reception even when the signal is weak. Additionally, SSB requires less power to transmit over long distances, making it a preferred choice for weak signal contacts.