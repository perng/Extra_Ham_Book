\subsection{Common Modulation for VHF/UHF Voice Repeaters}
\label{T8A04}

\begin{tcolorbox}[colback=gray!10!white,colframe=black!75!black,title=T8A04]
Which type of modulation is commonly used for VHF and UHF voice repeaters?
\begin{enumerate}[noitemsep]
    \item AM
    \item SSB
    \item PSK
    \item \textbf{FM or PM}
\end{enumerate}
\end{tcolorbox}

\subsubsection*{Intuitive Explanation}
Imagine you're trying to talk to your friend through a walkie-talkie. You want your voice to be clear and easy to understand, even if there's some noise around. FM (Frequency Modulation) and PM (Phase Modulation) are like the superheroes of modulation for this job. They make sure your voice comes through loud and clear, even in the noisy world of VHF (Very High Frequency) and UHF (Ultra High Frequency) bands. That's why they're the go-to choice for voice repeaters in these frequency ranges.

\subsubsection*{Advanced Explanation}
FM and PM are preferred for VHF and UHF voice repeaters due to their superior noise immunity compared to AM (Amplitude Modulation) and SSB (Single Sideband). FM modulates the frequency of the carrier wave, while PM modulates the phase. Both techniques are less susceptible to amplitude noise, which is common in these frequency bands. Additionally, FM and PM provide better signal-to-noise ratio (SNR) and are more efficient for voice communication, making them ideal for repeaters that need to relay clear and reliable voice signals over long distances.