\subsection{Common Modulation for VHF/UHF Voice Repeaters}
\label{T8A04}

\begin{tcolorbox}[colback=gray!10!white,colframe=black!75!black,title=T8A04]
Which type of modulation is commonly used for VHF and UHF voice repeaters?
\begin{enumerate}[label=\Alph*)]
    \item AM
    \item SSB
    \item PSK
    \item \textbf{FM or PM}
\end{enumerate}
\end{tcolorbox}

\subsubsection{Intuitive Explanation}
Imagine you're trying to talk to your friend on a walkie-talkie. You want your voice to travel far and clear, right? Well, VHF and UHF voice repeaters are like super-powered walkie-talkies that help your voice travel even further. The trick they use is called FM or PM (Frequency Modulation or Phase Modulation). Think of it like this: instead of yelling louder (which is what AM does), FM and PM change the pitch or timing of your voice slightly to make it clearer and less noisy. It's like turning your voice into a smooth, easy-to-understand melody that can travel long distances without getting messed up.

\subsubsection{Advanced Explanation}
Frequency Modulation (FM) and Phase Modulation (PM) are both angle modulation techniques where the frequency or phase of the carrier wave is varied in proportion to the amplitude of the modulating signal. For VHF (Very High Frequency) and UHF (Ultra High Frequency) voice repeaters, FM is more commonly used due to its superior noise immunity and signal quality over AM (Amplitude Modulation). 

In FM, the instantaneous frequency of the carrier wave is altered based on the input signal. Mathematically, the FM signal can be represented as:
\[
s(t) = A_c \cos\left(2\pi f_c t + 2\pi k_f \int_0^t m(\tau) \, d\tau\right)
\]
where \(A_c\) is the amplitude of the carrier, \(f_c\) is the carrier frequency, \(k_f\) is the frequency deviation constant, and \(m(t)\) is the modulating signal.

PM, on the other hand, varies the phase of the carrier wave:
\[
s(t) = A_c \cos\left(2\pi f_c t + k_p m(t)\right)
\]
where \(k_p\) is the phase deviation constant.

Both FM and PM are preferred for voice communication in VHF and UHF bands because they are less susceptible to amplitude noise, which is common in these frequency ranges. This makes FM and PM ideal for repeaters, which are used to extend the range of communication by receiving and retransmitting signals.

% Diagram Prompt: Generate a diagram comparing AM, FM, and PM waveforms to visually explain the differences in modulation techniques.