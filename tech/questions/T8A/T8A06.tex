\subsection{Sideband Usage in 10 Meter HF, VHF, and UHF Communications}
\label{T8A06}

\begin{tcolorbox}[colback=gray!10!white,colframe=black!75!black,title=T8A06]
Which sideband is normally used for 10 meter HF, VHF, and UHF single-sideband communications?
\begin{enumerate}[label=\Alph*]
    \item \textbf{Upper sideband}
    \item Lower sideband
    \item Suppressed sideband
    \item Inverted sideband
\end{enumerate}
\end{tcolorbox}

\subsubsection{Intuitive Explanation}
Imagine you're at a concert, and the band is playing music. The music has different parts: the high notes (upper sideband) and the low notes (lower sideband). In the world of radio, especially for 10 meter HF, VHF, and UHF communications, we usually use the high notes, or the upper sideband. It's like choosing to listen to the lead guitar instead of the bass guitar. It’s just the way things are done in these types of radio communications!

\subsubsection{Advanced Explanation}
In single-sideband (SSB) modulation, the carrier wave and one of the sidebands are suppressed, leaving only the upper or lower sideband for transmission. For frequencies in the 10 meter HF band (28-29.7 MHz), VHF (30-300 MHz), and UHF (300-3000 MHz), the upper sideband (USB) is conventionally used. This is due to historical and practical reasons, including the need for consistency in communication standards and the efficient use of bandwidth.

Mathematically, the SSB signal can be represented as:
\[ s(t) = A_c \cos(2\pi f_c t) \pm A_m \cos(2\pi f_m t) \]
where \( A_c \) is the carrier amplitude, \( f_c \) is the carrier frequency, \( A_m \) is the modulating signal amplitude, and \( f_m \) is the modulating signal frequency. The + sign corresponds to the upper sideband, while the - sign corresponds to the lower sideband.

The choice of upper sideband for these frequency ranges ensures that the transmitted signal occupies a narrower bandwidth, which is crucial for efficient spectrum utilization and minimizing interference.

% Diagram prompt: Generate a diagram showing the frequency spectrum of a single-sideband signal, highlighting the upper sideband and suppressed carrier and lower sideband.