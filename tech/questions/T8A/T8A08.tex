\subsection{Bandwidth of a Single Sideband (SSB) Voice Signal}\label{T8A08}

\begin{tcolorbox}[colback=gray!10!white,colframe=black!75!black,title=T8A08]
What is the approximate bandwidth of a typical single sideband (SSB) voice signal?
\begin{enumerate}[label=\Alph*)]
    \item 1 kHz
    \item \textbf{3 kHz}
    \item 6 kHz
    \item 15 kHz
\end{enumerate}
\end{tcolorbox}

\subsubsection*{Intuitive Explanation}
Imagine you're talking to your friend on a walkie-talkie. The sound of your voice doesn't take up a lot of space in the airwaves. In fact, it only needs about 3 kHz of room to travel through. This is like saying your voice fits into a small lane on a highway, while other signals might need a bigger lane. Single Sideband (SSB) is a clever way to make sure your voice gets through clearly without taking up too much space.

\subsubsection*{Advanced Explanation}
The bandwidth of a signal is the range of frequencies it occupies. For a typical human voice, the frequency range is approximately 300 Hz to 3 kHz. In Single Sideband (SSB) modulation, only one sideband (either the upper or lower) is transmitted, along with the carrier frequency suppressed. This reduces the bandwidth required compared to other modulation techniques like AM (Amplitude Modulation).

The bandwidth of an SSB signal is approximately equal to the highest frequency component of the modulating signal. For a typical voice signal, this is around 3 kHz. Mathematically, if the voice signal has a frequency range from \( f_{\text{min}} \) to \( f_{\text{max}} \), the bandwidth \( B \) is given by:
\[
B = f_{\text{max}} - f_{\text{min}}
\]
For a typical voice signal:
\[
B = 3\, \text{kHz} - 300\, \text{Hz} = 2.7\, \text{kHz} \approx 3\, \text{kHz}
\]
Thus, the approximate bandwidth of a typical SSB voice signal is 3 kHz.

% Diagram prompt: Generate a diagram showing the frequency spectrum of an SSB signal compared to an AM signal, highlighting the bandwidth difference.