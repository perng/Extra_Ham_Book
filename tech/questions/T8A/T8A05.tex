\subsection{Signal Bandwidth Comparison}
\label{T8A05}

\begin{tcolorbox}[colback=gray!10!white,colframe=black!75!black,title=T8A05]
Which of the following types of signal has the narrowest bandwidth?
\begin{enumerate}[noitemsep]
    \item FM voice
    \item SSB voice
    \item \textbf{CW}
    \item Slow-scan TV
\end{enumerate}
\end{tcolorbox}

\subsubsection*{Intuitive Explanation}
Imagine you're at a concert. FM voice is like a full band playing, taking up a lot of space. SSB voice is like a solo singer, taking up less space. Slow-scan TV is like a slideshow, taking up a bit more space than the singer. CW (Continuous Wave) is like a single note being played on a flute—it takes up the least space of all. So, CW has the narrowest bandwidth.

\subsubsection*{Advanced Explanation}
Bandwidth refers to the range of frequencies a signal occupies. FM voice typically uses a bandwidth of about 15 kHz, SSB voice uses about 3 kHz, Slow-scan TV uses about 3 kHz to 6 kHz, and CW uses a very narrow bandwidth, often less than 1 kHz. CW signals are simple, unmodulated carrier waves, which is why they occupy the least bandwidth. This makes CW ideal for long-distance communication, especially in crowded frequency bands.