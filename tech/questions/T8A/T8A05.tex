\subsection{Signal Bandwidth Comparison}
\label{T8A05}

\begin{tcolorbox}[colback=gray!10!white,colframe=black!75!black,title=T8A05]
Which of the following types of signal has the narrowest bandwidth?
\begin{enumerate}[label=\Alph*)]
    \item FM voice
    \item SSB voice
    \item \textbf{CW}
    \item Slow-scan TV
\end{enumerate}
\end{tcolorbox}

\subsubsection{Intuitive Explanation}
Imagine you're at a concert. FM voice is like a loudspeaker blasting music everywhere, taking up a lot of space. SSB voice is like a singer with a microphone, still taking up some space but not as much. Slow-scan TV is like a big projector showing a movie, taking up even more space. CW, on the other hand, is like a tiny whistle—it’s just a single tone, so it takes up the least amount of space. That’s why CW has the narrowest bandwidth!

\subsubsection{Advanced Explanation}
Bandwidth refers to the range of frequencies a signal occupies. FM voice typically has a bandwidth of about 15 kHz due to its modulation technique. SSB voice, which is a form of amplitude modulation, has a bandwidth of approximately 3 kHz. Slow-scan TV, which transmits images, can have a bandwidth of several kHz depending on the resolution and frame rate. CW (Continuous Wave) is a simple on-off keying signal, often used in Morse code, and it has a bandwidth of just a few Hz, making it the narrowest among the options.

Mathematically, the bandwidth \( B \) of a CW signal can be approximated by:
\[ B \approx \frac{1}{T} \]
where \( T \) is the duration of the signal element. For CW, \( T \) is typically very short, resulting in a very small \( B \).

% Diagram prompt: Generate a diagram showing the frequency spectrum of FM voice, SSB voice, Slow-scan TV, and CW signals to visually compare their bandwidths.