\subsection{Forms of Amplitude Modulation}
\label{T8A01}

\begin{tcolorbox}[colback=gray!10!white,colframe=black!75!black,title=T8A01]
Which of the following is a form of amplitude modulation?
\begin{enumerate}[noitemsep]
    \item Spread spectrum
    \item Packet radio
    \item \textbf{Single sideband}
    \item Phase shift keying (PSK)
\end{enumerate}
\end{tcolorbox}

\subsubsection*{Intuitive Explanation}
Amplitude modulation (AM) is like turning the volume knob on your radio up and down to send a message. Single sideband (SSB) is a special way of doing this where we only use one side of the signal, making it more efficient. The other options are like different ways of sending messages, but they don't involve turning the volume knob up and down.

\subsubsection*{Advanced Explanation}
Amplitude modulation (AM) varies the amplitude of the carrier wave in proportion to the message signal. Single sideband (SSB) is a type of AM that suppresses one sideband and the carrier, leaving only one sideband. This makes SSB more efficient in terms of bandwidth and power. Spread spectrum, packet radio, and phase shift keying (PSK) are different modulation techniques that do not involve varying the amplitude of the carrier wave. Spread spectrum spreads the signal over a wide frequency band, packet radio transmits data in packets, and PSK varies the phase of the carrier wave to represent data.