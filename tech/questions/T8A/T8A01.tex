\subsection{Forms of Amplitude Modulation}
\label{T8A01}

\begin{tcolorbox}[colback=gray!10!white,colframe=black!75!black,title=T8A01]
Which of the following is a form of amplitude modulation?
\begin{enumerate}[label=\Alph*)]
    \item Spread spectrum
    \item Packet radio
    \item \textbf{Single sideband}
    \item Phase shift keying (PSK)
\end{enumerate}
\end{tcolorbox}

\subsubsection{Intuitive Explanation}
Imagine you’re at a concert, and the band is playing really loud. Now, think of amplitude modulation (AM) as a way to change how loud the music is. Single sideband (SSB) is like a special trick where you only send part of the music—just the loud parts or just the quiet parts—to save energy and space. So, SSB is a type of AM because it’s all about changing the loudness of the signal!

\subsubsection{Advanced Explanation}
Amplitude modulation (AM) is a technique where the amplitude (strength) of a carrier wave is varied in proportion to the waveform being transmitted. Single sideband (SSB) is a specific form of AM where one of the sidebands and the carrier are suppressed, leaving only one sideband. This results in a more efficient use of bandwidth and power.

Mathematically, a standard AM signal can be represented as:
\[
s(t) = A_c [1 + m(t)] \cos(2\pi f_c t)
\]
where \( A_c \) is the amplitude of the carrier, \( m(t) \) is the modulating signal, and \( f_c \) is the carrier frequency. In SSB, only one sideband is transmitted, which can be represented as:
\[
s_{\text{SSB}}(t) = A_c m(t) \cos(2\pi f_c t) \mp A_c \hat{m}(t) \sin(2\pi f_c t)
\]
where \( \hat{m}(t) \) is the Hilbert transform of \( m(t) \), and the sign depends on whether the upper or lower sideband is transmitted.

SSB is particularly useful in radio communications because it reduces bandwidth and power consumption, making it more efficient than full AM.

% Diagram prompt: Generate a diagram showing the spectrum of a standard AM signal and an SSB signal, highlighting the suppression of the carrier and one sideband in SSB.