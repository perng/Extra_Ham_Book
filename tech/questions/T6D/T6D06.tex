\subsection{Component for Voltage Reduction}
\label{T6D06}

\begin{tcolorbox}[colback=gray!10!white,colframe=black!75!black,title=T6D06]
What component changes 120 V AC power to a lower AC voltage for other uses?
\begin{enumerate}[label=\Alph*)]
    \item Variable capacitor
    \item \textbf{Transformer}
    \item Transistor
    \item Diode
\end{enumerate}
\end{tcolorbox}

\subsubsection{Intuitive Explanation}
Imagine you have a big water pipe with a lot of water pressure (that's your 120 V AC power). Now, you want to use this water to fill a small kiddie pool, but the pressure is too high! You need something to reduce the pressure so it doesn’t blow up the pool. A transformer is like a magical pressure reducer for electricity. It takes the high voltage and turns it into a lower voltage, just like a valve reduces water pressure. So, when you need to power something that doesn’t need as much oomph, a transformer is your go-to gadget!

\subsubsection{Advanced Explanation}
A transformer is an electrical device that transfers electrical energy between two or more circuits through electromagnetic induction. It consists of two coils of wire, known as the primary and secondary windings, which are wound around a common magnetic core. When an alternating current (AC) flows through the primary winding, it creates a varying magnetic field in the core. This varying magnetic field induces a voltage in the secondary winding. The ratio of the number of turns in the primary winding to the number of turns in the secondary winding determines the voltage transformation ratio.

Mathematically, the relationship between the primary voltage (\(V_p\)), secondary voltage (\(V_s\)), primary turns (\(N_p\)), and secondary turns (\(N_s\)) is given by:

\[
\frac{V_p}{V_s} = \frac{N_p}{N_s}
\]

For example, if the primary winding has 120 turns and the secondary winding has 12 turns, the transformer will reduce the voltage by a factor of 10:

\[
\frac{120\,V}{V_s} = \frac{120}{12} \implies V_s = 12\,V
\]

Transformers are essential in power distribution systems to step down high voltages from power lines to safer, lower voltages for household use. They are also used in various electronic devices to provide the appropriate voltage levels required by different components.

% Diagram Prompt: Generate a diagram showing a transformer with primary and secondary windings, labeled with \(V_p\), \(V_s\), \(N_p\), and \(N_s\).