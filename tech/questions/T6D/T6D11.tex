\subsection{Resonant or Tuned Circuit}
\label{T6D11}

\begin{tcolorbox}[colback=gray!10!white,colframe=black!75!black,title=T6D11]
Which of the following is a resonant or tuned circuit?
\begin{enumerate}[noitemsep]
    \item \textbf{An inductor and a capacitor in series or parallel}
    \item A linear voltage regulator
    \item A resistor circuit used for reducing standing wave ratio
    \item A circuit designed to provide high-fidelity audio
\end{enumerate}
\end{tcolorbox}

\subsubsection*{Intuitive Explanation}
A resonant or tuned circuit is like a musical instrument that vibrates at a specific frequency. Just as a guitar string vibrates at a particular pitch, a circuit with an inductor and a capacitor can resonate at a specific frequency. This is why option A is the correct answer.

\subsubsection*{Advanced Explanation}
A resonant circuit, also known as a tuned circuit, consists of an inductor (L) and a capacitor (C) connected either in series or parallel. The circuit resonates at a specific frequency, known as the resonant frequency, which is determined by the values of L and C. The resonant frequency \( f_0 \) is given by the formula:
\[
f_0 = \frac{1}{2\pi\sqrt{LC}}
\]
At this frequency, the impedance of the circuit is either minimized (in series) or maximized (in parallel), allowing the circuit to selectively pass or block signals at that frequency. This property is widely used in radio frequency (RF) applications for tuning and filtering signals.