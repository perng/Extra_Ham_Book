\subsection{Resonant Circuit Components}
\label{T6D08}

\begin{tcolorbox}[colback=gray!10!white,colframe=black!75!black,title=T6D08]
Which of the following is combined with an inductor to make a resonant circuit?
\begin{enumerate}[label=\Alph*)]
    \item Resistor
    \item Zener diode
    \item Potentiometer
    \item \textbf{Capacitor}
\end{enumerate}
\end{tcolorbox}

\subsubsection{Intuitive Explanation}
Imagine you have a swing. If you push it at just the right time, it goes higher and higher. That's like a resonant circuit! An inductor is like the swing, and the capacitor is like your perfectly timed push. Together, they make the circuit swing at a specific frequency. Resistors, Zener diodes, and potentiometers are like trying to push the swing at the wrong time—they just don't work the same way!

\subsubsection{Advanced Explanation}
A resonant circuit, also known as an LC circuit, consists of an inductor (L) and a capacitor (C). The resonant frequency \( f_0 \) of the circuit is given by the formula:

\[
f_0 = \frac{1}{2\pi\sqrt{LC}}
\]

Here, \( L \) is the inductance in henries (H), and \( C \) is the capacitance in farads (F). The inductor stores energy in its magnetic field, while the capacitor stores energy in its electric field. When combined, they exchange energy back and forth at the resonant frequency, creating oscillations.

Resistors, Zener diodes, and potentiometers do not contribute to this energy exchange in the same way. Resistors dissipate energy as heat, Zener diodes regulate voltage, and potentiometers adjust resistance. Therefore, none of these components can form a resonant circuit with an inductor.

% Diagram prompt: Generate a diagram showing an LC circuit with an inductor and capacitor connected in series, labeled with their respective symbols (L and C), and indicate the resonant frequency formula below the circuit.