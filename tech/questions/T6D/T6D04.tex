\subsection{Displaying Electrical Quantities Numerically}
\label{T6D04}

\begin{tcolorbox}[colback=gray!10!white,colframe=black!75!black,title=T6D04]
Which of the following displays an electrical quantity as a numeric value?
\begin{enumerate}[label=\Alph*)]
    \item Potentiometer
    \item Transistor
    \item \textbf{Meter}
    \item Relay
\end{enumerate}
\end{tcolorbox}

\subsubsection*{Intuitive Explanation}
Imagine you have a magical box that can tell you how much electricity is flowing through a wire. You don’t need to guess or use complicated tools; this box just shows you a number. That’s what a meter does! It’s like a speedometer for electricity. A potentiometer is more like a volume knob, a transistor is a tiny switch, and a relay is a bigger switch. None of these show you a number directly, but the meter does!

\subsubsection*{Advanced Explanation}
A meter is an instrument designed to measure and display electrical quantities such as voltage, current, or resistance in a numeric format. It typically uses a digital or analog display to present the measured value. 

- \textbf{Potentiometer}: A variable resistor used to adjust the level of voltage or current in a circuit. It does not display any numeric value.
- \textbf{Transistor}: A semiconductor device used to amplify or switch electronic signals. It does not display any numeric value.
- \textbf{Meter}: An instrument that measures and displays electrical quantities numerically. It can be analog (using a needle) or digital (using a numeric display).
- \textbf{Relay}: An electrically operated switch used to control a circuit. It does not display any numeric value.

The correct answer is \textbf{C}, as a meter is specifically designed to display electrical quantities as numeric values.

% Diagram prompt: Generate a diagram showing a simple circuit with a meter measuring voltage across a resistor.