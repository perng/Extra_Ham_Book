\subsection{What Causes FM Transmission Audio Distortion on Voice Peaks?}
\label{T2B05}

\begin{tcolorbox}[colback=gray!10!white,colframe=black!75!black,title=T2B05]
What would cause your FM transmission audio to be distorted on voice peaks?
\begin{enumerate}[label=\Alph*)]
    \item Your repeater offset is inverted
    \item You need to talk louder
    \item \textbf{You are talking too loudly}
    \item Your transmit power is too high
\end{enumerate}
\end{tcolorbox}

\subsubsection{Intuitive Explanation}
Imagine you're trying to shout into a microphone at a concert. If you shout too loudly, the sound gets all garbled and distorted, right? The same thing happens with FM radio transmissions. When you talk too loudly into the microphone, the signal gets overloaded, and the audio comes out all messed up. So, the key is to keep your voice at a nice, steady level—not too soft, not too loud. Think of it like Goldilocks and the Three Bears: you want your voice to be just right!

\subsubsection{Advanced Explanation}
In FM (Frequency Modulation) transmission, the audio signal modulates the carrier wave by varying its frequency. When the input audio signal is too strong (i.e., you are talking too loudly), it can cause overmodulation. Overmodulation occurs when the modulation index exceeds the maximum allowable value, leading to distortion in the transmitted signal. This distortion is particularly noticeable on voice peaks, where the amplitude of the audio signal is highest.

Mathematically, the modulation index \( \beta \) is given by:
\[
\beta = \frac{\Delta f}{f_m}
\]
where \( \Delta f \) is the frequency deviation and \( f_m \) is the frequency of the modulating signal. If \( \beta \) becomes too large, the signal will be distorted.

To avoid this, it is crucial to maintain the input audio signal within the optimal range, ensuring that the modulation index stays within acceptable limits. This can be achieved by using proper microphone techniques and, if necessary, employing automatic gain control (AGC) circuits to regulate the audio signal level.

% Diagram Prompt: Generate a diagram showing the relationship between audio signal amplitude, frequency deviation, and modulation index in FM transmission.