\subsection{Distortion in FM Transmission Audio}
\label{T2B05}

\begin{tcolorbox}[colback=gray!10!white,colframe=black!75!black,title=T2B05]
What would cause your FM transmission audio to be distorted on voice peaks?
\begin{enumerate}[noitemsep]
    \item Your repeater offset is inverted
    \item You need to talk louder
    \item \textbf{You are talking too loudly}
    \item Your transmit power is too high
\end{enumerate}
\end{tcolorbox}

\subsubsection*{Intuitive Explanation}
Imagine you're shouting into a microphone. If you shout too loudly, the microphone can't handle the volume, and the sound gets all garbled. Similarly, in FM (Frequency Modulation) transmission, if you speak too loudly into the microphone, the audio signal can get distorted, especially during the loudest parts (peaks) of your speech. This is because the FM system has a limit to how much it can handle before it starts to distort the sound.

\subsubsection*{Advanced Explanation}
In FM transmission, the audio signal modulates the frequency of the carrier wave. The modulation index, which is the ratio of the frequency deviation to the modulating frequency, plays a crucial role in determining the quality of the transmitted signal. When the audio signal is too strong (i.e., you are talking too loudly), it causes excessive frequency deviation. This can lead to overmodulation, where the signal exceeds the maximum allowable deviation, resulting in distortion. The distortion is most noticeable during voice peaks because that's when the signal is at its strongest. Therefore, speaking too loudly into the microphone is the primary cause of audio distortion in FM transmission.