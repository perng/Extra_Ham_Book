\subsection{Sub-audible Tone for Squelch Control}
\label{T2B02}

\begin{tcolorbox}[colback=gray!10!white,colframe=black!75!black,title=T2B02]
What term describes the use of a sub-audible tone transmitted along with normal voice audio to open the squelch of a receiver?
\begin{enumerate}[noitemsep]
    \item Carrier squelch
    \item Tone burst
    \item DTMF
    \item \textbf{CTCSS}
\end{enumerate}
\end{tcolorbox}

\subsubsection*{Intuitive Explanation}
Imagine you're at a party where everyone is talking at the same time. You only want to hear your friend, so you both agree to whisper a secret word before speaking. This way, you only listen when you hear the secret word. CTCSS works similarly in radios. It uses a low-frequency tone (like a secret word) to tell the receiver when to open the squelch and let the audio through.

\subsubsection*{Advanced Explanation}
CTCSS (Continuous Tone-Coded Squelch System) is a method used in radio communications to control the squelch of a receiver. The transmitter sends a sub-audible tone (typically between 67 Hz and 254.1 Hz) along with the voice signal. The receiver is set to only open the squelch when it detects this specific tone. This allows multiple users to share the same frequency without hearing each other's transmissions, as long as they use different CTCSS tones. This system is particularly useful in repeater operations and in environments with high levels of interference.