\subsection{Q Signal for Changing Frequency}\label{T2B11}

\begin{tcolorbox}[colback=gray!10!white,colframe=black!75!black,title=T2B11]
Which Q signal indicates that you are changing frequency?
\begin{enumerate}[label=\Alph*]
    \item QRU
    \item \textbf{QSY}
    \item QSL
    \item QRZ
\end{enumerate}
\end{tcolorbox}

\subsubsection{Intuitive Explanation}
Imagine you're playing a game of hide and seek with your friends, and you decide to move to a new hiding spot. You might shout, I'm moving to a new spot! to let everyone know. In the world of radio communication, when you want to change your frequency (like moving to a new hiding spot), you use a special code called a Q signal. The Q signal that tells everyone you're changing frequency is QSY. It's like saying, Hey, I'm switching to a new channel, so tune in if you want to keep talking!

\subsubsection{Advanced Explanation}
In radio communication, Q signals are a set of three-letter codes that represent common phrases or questions. These signals are used to save time and reduce the chance of miscommunication, especially in Morse code or voice transmissions. The Q signal QSY specifically means Change to another frequency or Shift to another frequency. 

When a radio operator sends QSY, they are informing the other party that they are moving to a different frequency, and the other party should adjust their receiver accordingly to continue the communication. This is particularly useful in situations where interference or congestion on the current frequency necessitates a switch to a clearer channel.

For example, if two operators are communicating on 14.200 MHz and one operator decides to move to 14.250 MHz due to interference, they would send QSY 14.250 to indicate the new frequency. The other operator would then tune their receiver to 14.250 MHz to continue the conversation.

% Diagram prompt: Generate a diagram showing two radio operators communicating on one frequency, then one operator sending QSY and both switching to a new frequency.