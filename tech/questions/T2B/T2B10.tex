\subsection{Q Signal for Interference}
\label{T2B10}

\begin{tcolorbox}[colback=gray!10!white,colframe=black!75!black,title=T2B10]
Which Q signal indicates that you are receiving interference from other stations?
\begin{enumerate}[noitemsep]
    \item \textbf{QRM}
    \item QRN
    \item QTH
    \item QSB
\end{enumerate}
\end{tcolorbox}

\subsubsection*{Intuitive Explanation}
Imagine you're trying to listen to your favorite radio station, but someone else is talking loudly on the same frequency. That annoying noise is what we call interference. In the world of radio communication, we use special codes called Q signals to quickly describe common situations. The Q signal \textbf{QRM} is like saying, Hey, I'm hearing interference from other stations!

\subsubsection*{Advanced Explanation}
In radio communication, Q signals are standardized codes used to convey common messages efficiently. The Q signal \textbf{QRM} specifically refers to interference caused by other radio transmissions. This can occur when multiple stations are operating on the same or nearby frequencies, leading to overlapping signals. Understanding Q signals is crucial for clear and concise communication, especially in situations where brevity is essential, such as in amateur radio operations or emergency communications.