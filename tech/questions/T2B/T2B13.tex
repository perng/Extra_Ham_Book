\subsection{Purpose of a Squelch Function}
\label{T2B13}

\begin{tcolorbox}[colback=gray!10!white,colframe=black!75!black,title=T2B13]
What is the purpose of a squelch function?
\begin{enumerate}[noitemsep]
    \item Reduce a CW transmitter's key clicks
    \item \textbf{Mute the receiver audio when a signal is not present}
    \item Eliminate parasitic oscillations in an RF amplifier
    \item Reduce interference from impulse noise
\end{enumerate}
\end{tcolorbox}

The squelch function is used to mute the audio output of a receiver when no signal is present. This prevents the listener from hearing constant background noise or static, making the listening experience more pleasant. When a signal is detected, the squelch opens, allowing the audio to be heard.