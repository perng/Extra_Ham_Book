\subsection{Linked Repeater Network Description}
\label{T2B03}

\begin{tcolorbox}[colback=gray!10!white,colframe=black!75!black,title=T2B03]
Which of the following describes a linked repeater network?
\begin{enumerate}[label=\Alph*)]
    \item \textbf{A network of repeaters in which signals received by one repeater are transmitted by all the repeaters in the network}
    \item A single repeater with more than one receiver
    \item Multiple repeaters with the same control operator
    \item A system of repeaters linked by APRS
\end{enumerate}
\end{tcolorbox}

\subsubsection{Intuitive Explanation}
Imagine you and your friends each have a walkie-talkie. If one of you hears something important, you all repeat it so everyone else can hear it too. That's what a linked repeater network does! It’s like a team of walkie-talkies working together to make sure no one misses the message. So, when one repeater gets a signal, it tells all the other repeaters to send it out too. That way, everyone stays in the loop!

\subsubsection{Advanced Explanation}
A linked repeater network is a sophisticated system where multiple repeaters are interconnected to enhance communication coverage. When a signal is received by one repeater, it is simultaneously retransmitted by all other repeaters in the network. This ensures that the signal is broadcast over a much larger area than a single repeater could cover alone.

Mathematically, the coverage area of a linked repeater network can be considered as the union of the coverage areas of all individual repeaters. If each repeater has a coverage radius \( r \), the total coverage area \( A \) can be approximated by:

\[ A = \bigcup_{i=1}^{n} \pi r_i^2 \]

where \( n \) is the number of repeaters in the network. This interconnected system is particularly useful in scenarios where geographical barriers or distance would otherwise limit communication.

% Prompt for generating a diagram: A diagram showing multiple repeaters interconnected, with signals being transmitted from one repeater to all others in the network.