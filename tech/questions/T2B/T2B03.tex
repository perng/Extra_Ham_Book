\subsection{Linked Repeater Network Description}
\label{T2B03}

\begin{tcolorbox}[colback=gray!10!white,colframe=black!75!black,title=T2B03]
Which of the following describes a linked repeater network?
\begin{enumerate}[noitemsep]
    \item \textbf{A network of repeaters in which signals received by one repeater are transmitted by all the repeaters in the network}
    \item A single repeater with more than one receiver
    \item Multiple repeaters with the same control operator
    \item A system of repeaters linked by APRS
\end{enumerate}
\end{tcolorbox}

A linked repeater network is a system where multiple repeaters are interconnected. When one repeater receives a signal, it is retransmitted by all the other repeaters in the network, ensuring wide coverage and consistent communication across a large area. This setup is particularly useful in regions where a single repeater cannot cover the entire area effectively.