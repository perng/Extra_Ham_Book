\subsection{Standard Practices When Participating in a Net}
\label{T2C07}

\begin{tcolorbox}[colback=gray!10!white,colframe=black!75!black,title=T2C07]
Which of the following is standard practice when you participate in a net?
\begin{enumerate}[label=\Alph*]
    \item When first responding to the net control station, transmit your call sign, name, and address as in the FCC database
    \item Record the time of each of your transmissions
    \item \textbf{Unless you are reporting an emergency, transmit only when directed by the net control station}
    \item All these choices are correct
\end{enumerate}
\end{tcolorbox}

\subsubsection{Intuitive Explanation}
Imagine you're in a classroom, and the teacher is leading a discussion. You wouldn't just start talking whenever you feel like it, right? You'd wait for the teacher to call on you. The same idea applies when you're participating in a net (a group communication session) in radio. The net control station is like the teacher, and unless you have an emergency to report, you should only transmit when they tell you to. This keeps everything organized and prevents chaos.

\subsubsection{Advanced Explanation}
In radio communication, a net is a structured group communication session managed by a net control station (NCS). The NCS coordinates the flow of information, ensuring that transmissions are orderly and efficient. Standard practices in a net include:

1. \textbf{Transmitting Only When Directed}: Unless reporting an emergency, participants should transmit only when instructed by the NCS. This minimizes interference and ensures that the net operates smoothly.

2. \textbf{Identification}: While it's important to identify yourself, transmitting your full name and address is not typically required unless specifically requested by the NCS.

3. \textbf{Recording Transmission Times}: While keeping a log of transmissions can be useful, it is not a standard practice required during a net.

The correct answer, therefore, is to transmit only when directed by the net control station, unless reporting an emergency. This practice aligns with the principles of effective net management and ensures that communication remains clear and organized.

% Prompt for diagram: A diagram showing the flow of communication in a net, with the net control station at the center and participants transmitting only when directed.