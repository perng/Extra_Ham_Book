\subsection{Typical Duties of a Net Control Station}\label{T2C02}

\begin{tcolorbox}[colback=gray!10!white,colframe=black!75!black,title=T2C02]
Which of the following are typical duties of a Net Control Station?
\begin{enumerate}[label=\Alph*]
    \item Choose the regular net meeting time and frequency
    \item Ensure that all stations checking into the net are properly licensed for operation on the net frequency
    \item \textbf{Call the net to order and direct communications between stations checking in}
    \item All these choices are correct
\end{enumerate}
\end{tcolorbox}

\subsubsection{Intuitive Explanation}
Imagine you're the captain of a spaceship, and you need to make sure everyone on board is talking in an orderly fashion. The Net Control Station is like that captain! Their main job is to start the conversation (call the net to order) and make sure everyone gets a turn to speak (direct communications). It's not their job to decide when and where the spaceship meets or to check if everyone has a pilot's license. So, the correct answer is C!

\subsubsection{Advanced Explanation}
The Net Control Station (NCS) plays a crucial role in managing communication during a radio net. The primary responsibilities of the NCS include:

1. \textbf{Calling the Net to Order}: The NCS initiates the net by announcing its start and setting the tone for orderly communication.
2. \textbf{Directing Communications}: The NCS ensures that stations check in and out of the net in an organized manner, facilitating smooth and efficient communication.

While choosing the meeting time and frequency (Option A) and ensuring proper licensing (Option B) are important aspects of net operation, these tasks are typically handled by the net organizers or regulatory bodies, not the NCS. Therefore, the correct answer is C.

% Diagram Prompt: Generate a diagram showing the flow of communication in a radio net, with the Net Control Station at the center directing traffic between multiple stations.