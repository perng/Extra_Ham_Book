\subsection{Three-Wire 120 V Cable Insulation}
\label{T0A03}

\begin{tcolorbox}[colback=gray!10!white,colframe=black!75!black,title=T0A03]
In the United States, what circuit does black wire insulation indicate in a three-wire 120 V cable?
\begin{enumerate}[label=\Alph*)]
    \item Neutral
    \item \textbf{Hot}
    \item Equipment ground
    \item Black insulation is never used
\end{enumerate}
\end{tcolorbox}

\subsubsection{Intuitive Explanation}
Imagine you have a three-wire cable for your home's electrical system. The black wire is like the energized wire that carries the electricity to your devices. Think of it as the hot wire that’s always ready to power up your gadgets. The other wires, like the white one, are more like the return path for the electricity, and the green or bare wire is the safety wire that keeps everything grounded. So, the black wire is the one that’s live and ready to go!

\subsubsection{Advanced Explanation}
In a standard three-wire 120 V AC electrical system in the United States, the black wire is designated as the hot wire. This wire carries the current from the power source to the load. The white wire is the neutral wire, which provides the return path for the current. The green or bare wire is the equipment ground, which is a safety feature to prevent electrical shock by providing a path to the ground in case of a fault.

The voltage between the black (hot) wire and the white (neutral) wire is typically 120 V. The ground wire is at the same potential as the earth and is used to protect against electrical faults. The black wire is insulated with black material to distinguish it from the other wires, ensuring proper identification during installation and maintenance.

% Diagram prompt: Generate a diagram showing a three-wire 120 V cable with black, white, and green wires labeled as hot, neutral, and ground, respectively.