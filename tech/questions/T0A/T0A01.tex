\subsection{Safety Hazards of a 12-Volt Storage Battery}
\label{T0A01}

\begin{tcolorbox}[colback=gray!10!white,colframe=black!75!black,title=T0A01]
Which of the following is a safety hazard of a 12-volt storage battery?
\begin{enumerate}[noitemsep]
    \item Touching both terminals with the hands can cause electrical shock
    \item \textbf{Shorting the terminals can cause burns, fire, or an explosion}
    \item RF emissions from a nearby transmitter can cause the electrolyte to emit poison gas
    \item All these choices are correct
\end{enumerate}
\end{tcolorbox}

A 12-volt storage battery, while low in voltage, can still pose significant safety risks if mishandled. The primary hazard comes from shorting the terminals, which can lead to high current flow, resulting in burns, fire, or even an explosion. Touching the terminals with bare hands is generally safe due to the low voltage, and RF emissions from a nearby transmitter do not affect the electrolyte in such a way as to emit poison gas. Therefore, the correct answer is \textbf{B}.