\subsection{Safety Hazards of a 12-Volt Storage Battery}
\label{T0A01}

\begin{tcolorbox}[colback=gray!10!white,colframe=black!75!black,title=T0A01]
Which of the following is a safety hazard of a 12-volt storage battery?
\begin{enumerate}[label=\Alph*]
    \item Touching both terminals with the hands can cause electrical shock
    \item \textbf{Shorting the terminals can cause burns, fire, or an explosion}
    \item RF emissions from a nearby transmitter can cause the electrolyte to emit poison gas
    \item All these choices are correct
\end{enumerate}
\end{tcolorbox}

\subsubsection{Intuitive Explanation}
Imagine a 12-volt battery as a tiny, sleepy dragon. Normally, it’s harmless, but if you poke it the wrong way—like by shorting its terminals—it wakes up angry and can breathe fire (or cause burns, fires, or even explosions). Touching the terminals with your hands won’t shock you because 12 volts isn’t enough to zap you. And don’t worry about nearby radios turning the battery into a poison gas factory—that’s just a myth!

\subsubsection{Advanced Explanation}
A 12-volt storage battery, such as a lead-acid battery, stores chemical energy that can be converted into electrical energy. The primary safety hazard arises from short-circuiting the terminals. When the terminals are shorted, a large current flows through the circuit, governed by Ohm's Law:

\[
I = \frac{V}{R}
\]

where \( I \) is the current, \( V \) is the voltage (12 volts), and \( R \) is the resistance. In a short circuit, \( R \) is very low, leading to a high current. This can cause rapid heating, potentially leading to burns, fires, or even explosions due to the release of hydrogen gas from the electrolyte. 

Touching both terminals with bare hands does not pose a significant risk of electrical shock because the human body's resistance is too high for 12 volts to cause harm. Additionally, RF emissions from a nearby transmitter do not interact with the battery's electrolyte to produce poison gas. This is a misconception, as the energy levels involved are insufficient to induce such a chemical reaction.

% Diagram prompt: A diagram showing a 12-volt battery with terminals labeled, and a short circuit path with a high current flow, illustrating the potential hazards.