\subsection{Hazard in Power Supply After Turning Off}
\label{T0A11}

\begin{tcolorbox}[colback=gray!10!white,colframe=black!75!black,title=T0A11]
What hazard exists in a power supply immediately after turning it off?
\begin{enumerate}[label=\Alph*)]
    \item Circulating currents in the dc filter
    \item Leakage flux in the power transformer
    \item Voltage transients from kickback diodes
    \item \textbf{Charge stored in filter capacitors}
\end{enumerate}
\end{tcolorbox}

\subsubsection{Intuitive Explanation}
Imagine you have a water balloon that's full of water. Even after you stop filling it, the water stays inside until you let it out. Similarly, in a power supply, there are components called capacitors that store electrical charge like a water balloon stores water. When you turn off the power supply, these capacitors still hold onto their charge. If you touch them or accidentally short them, it's like popping the water balloon—you'll get a sudden release of energy, which can be dangerous. So, the hazard is the leftover charge in these capacitors.

\subsubsection{Advanced Explanation}
In a power supply, filter capacitors are used to smooth out the voltage by storing electrical charge. These capacitors are typically large and can hold a significant amount of energy. When the power supply is turned off, the capacitors retain this stored charge due to their inherent property of maintaining a voltage across their terminals. The energy stored in a capacitor is given by the formula:

\[
E = \frac{1}{2} C V^2
\]

where \(E\) is the energy, \(C\) is the capacitance, and \(V\) is the voltage across the capacitor. Even after the power supply is turned off, \(V\) remains non-zero for a period of time, depending on the capacitor's discharge rate. This stored charge can pose a hazard if not properly discharged, as it can cause electric shock or damage to components if accidentally shorted.

To safely discharge the capacitors, a discharge resistor is often used in parallel with the capacitor. The time constant \(\tau\) for the discharge is given by:

\[
\tau = R C
\]

where \(R\) is the resistance of the discharge resistor. The capacitor will discharge to a safe voltage level after approximately \(5\tau\) seconds.

% Diagram Prompt: Generate a diagram showing a power supply circuit with filter capacitors and a discharge resistor.