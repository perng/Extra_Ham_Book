\subsection{Precautions for Measuring High Voltages}
\label{T0A12}

\begin{tcolorbox}[colback=gray!10!white,colframe=black!75!black,title=T0A12]
Which of the following precautions should be taken when measuring high voltages with a voltmeter?
\begin{enumerate}[noitemsep]
    \item Ensure that the voltmeter has very low impedance
    \item \textbf{Ensure that the voltmeter and leads are rated for use at the voltages to be measured}
    \item Ensure that the circuit is grounded through the voltmeter
    \item Ensure that the voltmeter is set to the correct frequency
\end{enumerate}
\end{tcolorbox}

When measuring high voltages, it is crucial to ensure that the voltmeter and its leads are rated for the voltages being measured. This prevents damage to the equipment and ensures accurate readings. The other options are either incorrect or irrelevant to the safety and accuracy of high voltage measurements.