\subsection{Power Loss in a Feed Line}
\label{T7C07}

\begin{tcolorbox}[colback=gray!10!white,colframe=black!75!black,title=T7C07]
What happens to power lost in a feed line?
\begin{enumerate}[label=\Alph*)]
    \item It increases the SWR
    \item It is radiated as harmonics
    \item \textbf{It is converted into heat}
    \item It distorts the signal
\end{enumerate}
\end{tcolorbox}

\subsubsection{Intuitive Explanation}
Imagine you have a garden hose, and you're trying to water your plants. If the hose has a leak, some of the water will escape before it reaches the plants. In a similar way, when you send power through a feed line (like a hose for electricity), some of the power can leak out. But instead of water, this lost power turns into heat. So, the power doesn't just disappear; it warms up the feed line!

\subsubsection{Advanced Explanation}
When electrical power is transmitted through a feed line, some of the power is lost due to the resistance of the conductor. This power loss can be calculated using the formula:

\[
P_{\text{loss}} = I^2 R
\]

where \( P_{\text{loss}} \) is the power lost, \( I \) is the current flowing through the feed line, and \( R \) is the resistance of the feed line. This lost power is dissipated as heat, which is why the feed line may warm up during operation.

The Standing Wave Ratio (SWR) is a measure of how well the feed line is matched to the load. While high SWR can lead to additional power loss, the primary cause of power loss in a feed line is the resistive heating due to the current flowing through it. Harmonics and signal distortion are not directly related to the power loss in the feed line but can be influenced by other factors such as impedance mismatches and non-linear components in the system.

% Diagram Prompt: Generate a diagram showing a feed line with power flowing through it, highlighting the areas where power is lost as heat.