\subsection{Dummy Load Components}
\label{T7C03}

\begin{tcolorbox}[colback=gray!10!white,colframe=black!75!black,title=T7C03]
What does a dummy load consist of?
\begin{enumerate}[label=\Alph*)]
    \item A high-gain amplifier and a TR switch
    \item \textbf{A non-inductive resistor mounted on a heat sink}
    \item A low-voltage power supply and a DC relay
    \item A 50-ohm reactance used to terminate a transmission line
\end{enumerate}
\end{tcolorbox}

\subsubsection{Intuitive Explanation}
Imagine you have a toy car that you want to test, but you don’t want it to zoom off across the room. So, you put it on a treadmill that absorbs all its energy, making it stay in place. A dummy load is like that treadmill for radios! It’s a special resistor that soaks up all the radio’s power without letting it escape into the air. This way, you can test your radio without bothering your neighbors or breaking any rules. The resistor is mounted on a heat sink to keep it cool, just like how you might use a fan to cool down after running on a treadmill.

\subsubsection{Advanced Explanation}
A dummy load is a device used to simulate an electrical load, typically for testing radio transmitters. It consists of a non-inductive resistor, which means it has negligible inductance, ensuring that it behaves purely as a resistive load. This resistor is mounted on a heat sink to dissipate the power generated during testing, preventing overheating. The resistor is usually designed to match the characteristic impedance of the transmission line, commonly 50 ohms in radio applications. This ensures that the transmitter sees a matched load, minimizing reflections and standing waves. The heat sink is crucial because it allows the dummy load to handle high power levels without damage, making it an essential tool for safe and effective transmitter testing.

% Diagram Prompt: Generate a diagram showing a dummy load connected to a radio transmitter, with the non-inductive resistor and heat sink clearly labeled.