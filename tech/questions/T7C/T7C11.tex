\subsection{Disadvantages of Air Core Coaxial Cable}
\label{T7C11}

\begin{tcolorbox}[colback=gray!10!white,colframe=black!75!black,title=T7C11]
What is a disadvantage of air core coaxial cable when compared to foam or solid dielectric types?
\begin{enumerate}[label=\Alph*)]
    \item It has more loss per foot
    \item It cannot be used for VHF or UHF antennas
    \item \textbf{It requires special techniques to prevent moisture in the cable}
    \item It cannot be used at below freezing temperatures
\end{enumerate}
\end{tcolorbox}

\subsubsection*{Intuitive Explanation}
Imagine you have a straw filled with air (air core coaxial cable) and another straw filled with foam (foam dielectric coaxial cable). The straw with air is lighter, but if you leave it outside, it might get water inside it, and that’s not good! The foam-filled straw, on the other hand, doesn’t let water in easily. So, the air-filled straw needs extra care to keep it dry, which is a bit of a hassle. That’s why air core coaxial cables need special techniques to prevent moisture from sneaking in.

\subsubsection*{Advanced Explanation}
Air core coaxial cables use air as the dielectric material between the inner conductor and the outer shield. While air has a lower dielectric constant, which can reduce signal loss, it also introduces a significant disadvantage: susceptibility to moisture ingress. Moisture can condense inside the cable, especially in humid environments, leading to increased signal attenuation and potential damage to the cable. To mitigate this, air core coaxial cables often require special sealing techniques, such as the use of desiccants or hermetic seals, to prevent moisture from entering the cable. This additional complexity and cost make air core cables less convenient compared to foam or solid dielectric types, which inherently resist moisture penetration.

% Diagram prompt: A comparison diagram showing air core coaxial cable vs. foam dielectric coaxial cable, highlighting the moisture prevention techniques for air core cables.