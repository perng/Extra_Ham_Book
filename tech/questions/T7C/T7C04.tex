\subsection{SWR Meter and Impedance Match}
\label{T7C04}

\begin{tcolorbox}[colback=gray!10!white,colframe=black!75!black,title=T7C04]
What reading on an SWR meter indicates a perfect impedance match between the antenna and the feed line?
\begin{enumerate}[label=\Alph*)]
    \item 50:50
    \item Zero
    \item \textbf{1:1}
    \item Full Scale
\end{enumerate}
\end{tcolorbox}

\subsubsection{Intuitive Explanation}
Imagine you're trying to pour water from one bottle to another. If the bottles are the same size, the water flows smoothly without any spills. This is like a perfect match between the antenna and the feed line. The SWR meter is like a referee that checks if the bottles are the same size. When it shows 1:1, it means everything is perfectly matched, and the signal flows smoothly without any loss or reflection.

\subsubsection{Advanced Explanation}
The Standing Wave Ratio (SWR) is a measure of how well the impedance of the antenna matches the impedance of the feed line. Impedance is a complex quantity that combines resistance and reactance, and it is measured in ohms ($\Omega$). When the impedance of the antenna ($Z_{\text{antenna}}$) matches the impedance of the feed line ($Z_{\text{feed line}}$), the SWR is given by:

\[
\text{SWR} = \frac{1 + \Gamma}{1 - \Gamma}
\]

where $\Gamma$ is the reflection coefficient, defined as:

\[
\Gamma = \frac{Z_{\text{antenna}} - Z_{\text{feed line}}}{Z_{\text{antenna}} + Z_{\text{feed line}}}
\]

When $Z_{\text{antenna}} = Z_{\text{feed line}}$, $\Gamma = 0$, and thus:

\[
\text{SWR} = \frac{1 + 0}{1 - 0} = 1
\]

This is why a reading of 1:1 on the SWR meter indicates a perfect impedance match. Any deviation from this ratio indicates a mismatch, which can lead to signal loss and potential damage to the transmitter.

% Diagram Prompt: Generate a diagram showing the relationship between impedance matching, SWR, and signal reflection.