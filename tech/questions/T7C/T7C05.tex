\subsection{Output Power Reduction in Solid-State Transmitters}
\label{T7C05}

\begin{tcolorbox}[colback=gray!10!white,colframe=black!75!black,title=T7C05]
Why do most solid-state transmitters reduce output power as SWR increases beyond a certain level?
\begin{enumerate}[noitemsep]
    \item \textbf{To protect the output amplifier transistors}
    \item To comply with FCC rules on spectral purity
    \item Because power supplies cannot supply enough current at high SWR
    \item To lower the SWR on the transmission line
\end{enumerate}
\end{tcolorbox}

\subsubsection*{Intuitive Explanation}
Imagine your transmitter is like a car engine. If the road gets too bumpy (high SWR), the engine (transmitter) might get damaged. To avoid this, the car (transmitter) slows down (reduces power) to protect the engine (transistor).

\subsubsection*{Advanced Explanation}
Solid-state transmitters are designed to protect their output amplifier transistors from damage caused by high Standing Wave Ratio (SWR). When SWR increases, it indicates a mismatch between the transmitter and the antenna, leading to reflected power. This reflected power can cause excessive heat and stress on the transistors, potentially damaging them. To prevent this, the transmitter automatically reduces its output power, thereby protecting the transistors from overheating and failure. This is a built-in safety mechanism to ensure the longevity and reliability of the transmitter.