\subsection{Common Impedance of Coaxial Cables}
\label{T9B02}

\begin{tcolorbox}[colback=gray!10!white,colframe=black!75!black,title=T9B02]
What is the most common impedance of coaxial cables used in amateur radio?
\begin{enumerate}[label=\Alph*)]
    \item 8 ohms
    \item \textbf{50 ohms}
    \item 600 ohms
    \item 12 ohms
\end{enumerate}
\end{tcolorbox}

\subsubsection{Intuitive Explanation}
Imagine you're trying to send a message through a hose. If the hose is just the right size, the message flows smoothly without any bumps or jams. In the world of amateur radio, coaxial cables are like those hoses, and the right size is called impedance. The most common right size for these cables is 50 ohms. It's like the Goldilocks of impedances—not too big, not too small, but just right for most radio signals!

\subsubsection{Advanced Explanation}
Impedance, denoted by \( Z \), is a measure of opposition to the flow of alternating current (AC) in a circuit. It is a complex quantity, combining resistance \( R \) and reactance \( X \), and is given by:
\[
Z = R + jX
\]
where \( j \) is the imaginary unit. In coaxial cables used in amateur radio, the characteristic impedance is determined by the physical dimensions and the dielectric material between the inner conductor and the outer shield. The most common impedance for these cables is 50 ohms, which is a compromise between power handling capability and signal loss. This value is derived from the following formula:
\[
Z_0 = \frac{138 \log_{10}(\frac{D}{d})}{\sqrt{\epsilon_r}}
\]
where \( D \) is the inner diameter of the outer conductor, \( d \) is the outer diameter of the inner conductor, and \( \epsilon_r \) is the relative permittivity of the dielectric material. For most coaxial cables used in amateur radio, this calculation results in an impedance close to 50 ohms.

% Diagram prompt: Generate a diagram showing the cross-section of a coaxial cable with labeled dimensions \( D \) and \( d \), and indicate the dielectric material.