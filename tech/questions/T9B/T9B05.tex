\subsection{Effect of Frequency Increase in Coaxial Cable}
\label{T9B05}

\begin{tcolorbox}[colback=gray!10!white,colframe=black!75!black,title=T9B05]
What happens as the frequency of a signal in coaxial cable is increased?
\begin{enumerate}[noitemsep]
    \item The characteristic impedance decreases
    \item The loss decreases
    \item The characteristic impedance increases
    \item \textbf{The loss increases}
\end{enumerate}
\end{tcolorbox}

\subsubsection*{Intuitive Explanation}
Imagine you're trying to send a message through a long, narrow tunnel. If you whisper (low frequency), the message might travel quite far before it fades away. But if you shout (high frequency), the sound waves bounce around more and lose energy faster, so the message doesn't travel as far. Similarly, in a coaxial cable, higher frequency signals lose more energy as they travel, leading to increased loss.

\subsubsection*{Advanced Explanation}
As the frequency of a signal in a coaxial cable increases, the skin effect becomes more pronounced. The skin effect causes the signal to travel more on the surface of the conductor rather than through its entire cross-section. This results in higher resistance and, consequently, higher loss. Additionally, dielectric losses in the insulating material between the conductors also increase with frequency. These combined effects lead to an overall increase in signal loss as the frequency increases. The characteristic impedance of the cable, however, remains relatively constant with frequency, assuming the cable is properly designed and terminated.