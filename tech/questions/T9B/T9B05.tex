\subsection{Frequency Effects in Coaxial Cable}
\label{T9B05}

\begin{tcolorbox}[colback=gray!10!white,colframe=black!75!black,title=T9B05]
What happens as the frequency of a signal in coaxial cable is increased?
\begin{enumerate}[label=\Alph*]
    \item The characteristic impedance decreases
    \item The loss decreases
    \item The characteristic impedance increases
    \item \textbf{The loss increases}
\end{enumerate}
\end{tcolorbox}

\subsubsection{Intuitive Explanation}
Imagine you're trying to send a message through a long, twisty tube (the coaxial cable). If you whisper (low frequency), the message might get through just fine. But if you start shouting (high frequency), the tube starts to absorb more of your energy, and your message gets weaker as it travels. So, the higher the frequency, the more energy you lose along the way. That's why the loss increases with frequency!

\subsubsection{Advanced Explanation}
In coaxial cables, the loss is primarily due to two factors: conductor loss and dielectric loss. As the frequency of the signal increases, both of these losses tend to increase. 

The conductor loss is given by:
\[
\alpha_c = \frac{R}{2Z_0}
\]
where \( R \) is the resistance per unit length and \( Z_0 \) is the characteristic impedance. At higher frequencies, the skin effect causes the resistance \( R \) to increase, leading to higher conductor loss.

The dielectric loss is given by:
\[
\alpha_d = \frac{G Z_0}{2}
\]
where \( G \) is the conductance per unit length. As frequency increases, the dielectric material's ability to store and release energy diminishes, increasing \( G \) and thus the dielectric loss.

Therefore, the total loss \( \alpha \) in the coaxial cable increases with frequency:
\[
\alpha = \alpha_c + \alpha_d
\]

This explains why the correct answer is that the loss increases as the frequency of the signal in coaxial cable is increased.

% Diagram prompt: A diagram showing the relationship between frequency and loss in a coaxial cable, illustrating the increase in loss with frequency.