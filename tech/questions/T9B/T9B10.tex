\subsection{Electrical Difference Between RG-58 and RG-213 Coaxial Cable}
\label{T9B10}

\begin{tcolorbox}[colback=gray!10!white,colframe=black!75!black,title=T9B10]
What is the electrical difference between RG-58 and RG-213 coaxial cable?
\begin{enumerate}[label=\Alph*)]
    \item There is no significant difference between the two types
    \item RG-58 cable has two shields
    \item \textbf{RG-213 cable has less loss at a given frequency}
    \item RG-58 cable can handle higher power levels
\end{enumerate}
\end{tcolorbox}

\subsubsection{Intuitive Explanation}
Imagine you have two water hoses: one is thin (RG-58) and the other is thick (RG-213). When you try to push water through them, the thin hose loses more water along the way because it’s harder for the water to flow smoothly. Similarly, RG-213 is like the thick hose—it lets the signal (instead of water) flow with less loss compared to RG-58. So, RG-213 is better at keeping the signal strong over long distances!

\subsubsection{Advanced Explanation}
The primary electrical difference between RG-58 and RG-213 coaxial cables lies in their attenuation characteristics. Attenuation, measured in decibels per unit length (dB/m), is the loss of signal strength as it travels through the cable. RG-213 has a lower attenuation compared to RG-58, especially at higher frequencies. This is due to its larger conductor size and better shielding, which reduces resistive losses and electromagnetic interference.

The attenuation \( \alpha \) of a coaxial cable can be approximated by the formula:
\[
\alpha = \frac{R}{2Z_0} + \frac{G Z_0}{2}
\]
where \( R \) is the resistance per unit length, \( Z_0 \) is the characteristic impedance, and \( G \) is the conductance per unit length. RG-213’s larger diameter reduces \( R \), leading to lower attenuation.

Additionally, RG-213 can handle higher power levels due to its thicker conductor and better heat dissipation properties. This makes it more suitable for applications requiring long-distance transmission or higher power handling, such as in amateur radio or broadcast systems.

% Prompt for diagram: A diagram comparing the cross-sections of RG-58 and RG-213 coaxial cables, highlighting the differences in conductor size and shielding, would be beneficial here.