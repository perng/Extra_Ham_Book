\subsection{Electrical Difference Between RG-58 and RG-213 Coaxial Cable}
\label{T9B10}

\begin{tcolorbox}[colback=gray!10!white,colframe=black!75!black,title=T9B10]
What is the electrical difference between RG-58 and RG-213 coaxial cable?
\begin{enumerate}[noitemsep]
    \item There is no significant difference between the two types
    \item RG-58 cable has two shields
    \item \textbf{RG-213 cable has less loss at a given frequency}
    \item RG-58 cable can handle higher power levels
\end{enumerate}
\end{tcolorbox}

\subsubsection{Intuitive Explanation}
Think of coaxial cables like water pipes. The thicker the pipe, the more water can flow through it with less resistance. Similarly, RG-213 is like a thicker pipe compared to RG-58, allowing signals to travel with less loss. So, RG-213 is better for longer distances or higher frequencies because it doesn't lose as much signal strength.

\subsubsection{Advanced Explanation}
The primary electrical difference between RG-58 and RG-213 coaxial cables lies in their attenuation characteristics. Attenuation refers to the loss of signal strength as it travels through the cable. RG-213 has a larger diameter and better shielding compared to RG-58, which results in lower attenuation at a given frequency. This makes RG-213 more suitable for applications requiring longer cable runs or higher frequencies, where minimizing signal loss is crucial. Additionally, RG-213 can handle higher power levels due to its thicker conductor and better insulation, but the key difference highlighted in this question is the lower loss at a given frequency.