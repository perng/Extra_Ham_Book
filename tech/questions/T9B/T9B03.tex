\subsection{Common Feed Line for Amateur Radio Antennas}
\label{T9B03}

\begin{tcolorbox}[colback=gray!10!white,colframe=black!75!black,title=T9B03]
Why is coaxial cable the most common feed line for amateur radio antenna systems?
\begin{enumerate}[label=\Alph*)]
    \item \textbf{It is easy to use and requires few special installation considerations}
    \item It has less loss than any other type of feed line
    \item It can handle more power than any other type of feed line
    \item It is less expensive than any other type of feed line
\end{enumerate}
\end{tcolorbox}

\subsubsection{Intuitive Explanation}
Imagine you’re setting up a radio antenna in your backyard. You need a cable to connect your radio to the antenna, and you want something that’s easy to work with, like a garden hose that doesn’t kink or tangle. Coaxial cable is like that garden hose—it’s straightforward to install and doesn’t need any fancy tricks to get it working. It’s the go-to choice because it’s simple and reliable, just like your favorite pair of sneakers!

\subsubsection{Advanced Explanation}
Coaxial cable is widely used in amateur radio systems due to its balanced combination of ease of use, moderate power handling, and acceptable signal loss characteristics. The cable consists of an inner conductor surrounded by a dielectric insulator, which is then enclosed by a conductive shield and an outer insulating layer. This design minimizes electromagnetic interference and signal leakage, making it suitable for a variety of installations without requiring complex shielding or grounding techniques.

While other types of feed lines, such as waveguide or open-wire lines, may offer lower loss or higher power handling, they often require more specialized installation practices and are less flexible in terms of physical deployment. Coaxial cable strikes a practical balance, offering sufficient performance for most amateur radio applications with minimal installation complexity.

% Diagram prompt: Generate a diagram showing the structure of a coaxial cable, including the inner conductor, dielectric insulator, conductive shield, and outer insulating layer.