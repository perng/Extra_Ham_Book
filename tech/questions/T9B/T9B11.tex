\subsection{Lowest Loss Feed Line at VHF and UHF}
\label{T9B11}

\begin{tcolorbox}[colback=gray!10!white,colframe=black!75!black,title=T9B11]
Which of the following types of feed line has the lowest loss at VHF and UHF?
\begin{enumerate}[noitemsep]
    \item 50-ohm flexible coax
    \item Multi-conductor unbalanced cable
    \item \textbf{Air-insulated hardline}
    \item 75-ohm flexible coax
\end{enumerate}
\end{tcolorbox}

\subsubsection*{Intuitive Explanation}
Think of feed lines as pipes that carry your radio signals. Just like water pipes, some pipes are better at keeping the water (or signal) from leaking out. At higher frequencies like VHF and UHF, the type of pipe (feed line) matters a lot. Air-insulated hardline is like a super-efficient pipe that doesn't let much signal escape, making it the best choice for these frequencies.

\subsubsection*{Advanced Explanation}
Feed line loss is primarily due to the dielectric material and conductor resistance. At VHF and UHF frequencies, the dielectric loss becomes significant. Air-insulated hardline uses air as the dielectric, which has a very low loss tangent compared to solid dielectrics used in flexible coax. Additionally, the rigid construction of hardline reduces conductor loss. Therefore, air-insulated hardline exhibits the lowest loss at VHF and UHF frequencies.