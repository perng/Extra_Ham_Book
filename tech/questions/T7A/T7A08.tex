\subsection{Combining Speech with an RF Carrier Signal}
\label{T7A08}

\begin{tcolorbox}[colback=gray!10!white,colframe=black!75!black,title=T7A08]
Which of the following describes combining speech with an RF carrier signal?
\begin{enumerate}[noitemsep]
    \item Impedance matching
    \item Oscillation
    \item \textbf{Modulation}
    \item Low-pass filtering
\end{enumerate}
\end{tcolorbox}

\subsubsection*{Intuitive Explanation}
Imagine you want to send your voice over a radio wave. The radio wave is like a fast-moving train, and your voice is like a passenger. To get your voice onto the train, you need to modulate it—essentially, you're combining your voice with the radio wave so it can travel long distances. This process is called modulation.

\subsubsection*{Advanced Explanation}
Modulation is the process of varying one or more properties of a periodic waveform, called the carrier signal, with a modulating signal that typically contains information to be transmitted. In this case, the carrier signal is the RF (Radio Frequency) signal, and the modulating signal is the speech. There are different types of modulation, such as Amplitude Modulation (AM) and Frequency Modulation (FM), each with its own advantages and disadvantages. The key concept here is that modulation allows the transmission of information (speech) over a carrier wave (RF signal) efficiently.