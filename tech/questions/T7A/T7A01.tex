\subsection{Receiver Signal Detection Ability}
\label{T7A01}

\begin{tcolorbox}[colback=gray!10!white,colframe=black!75!black,title=T7A01]
Which term describes the ability of a receiver to detect the presence of a signal?
\begin{enumerate}[label=\Alph*]
    \item Linearity
    \item \textbf{Sensitivity}
    \item Selectivity
    \item Total Harmonic Distortion
\end{enumerate}
\end{tcolorbox}

\subsubsection{Intuitive Explanation}
Imagine you're trying to hear a whisper in a noisy room. If you can hear the whisper even when it's very soft, you have good hearing sensitivity. Similarly, in radio terms, sensitivity is how well a receiver can pick up a weak signal. If your receiver is sensitive, it can detect even the faintest signals, just like your ears can hear that whisper!

\subsubsection{Advanced Explanation}
Sensitivity in radio receivers refers to the minimum signal strength that the receiver can detect and process effectively. It is typically measured in microvolts ($\mu$V) or decibels relative to one milliwatt (dBm). A more sensitive receiver can detect weaker signals, which is crucial for long-distance communication or in environments with high noise levels.

Mathematically, sensitivity can be expressed as:
\[
\text{Sensitivity} = \frac{V_{\text{min}}}{G}
\]
where \( V_{\text{min}} \) is the minimum detectable voltage and \( G \) is the gain of the receiver.

Related concepts include:
\begin{itemize}
    \item \textbf{Linearity}: The ability of the receiver to handle signals without distortion.
    \item \textbf{Selectivity}: The ability to distinguish between signals of different frequencies.
    \item \textbf{Total Harmonic Distortion (THD)}: A measure of the distortion introduced by the receiver.
\end{itemize}

% Prompt for diagram: A diagram showing a receiver with different signal strengths and how sensitivity affects the detection of weak signals.