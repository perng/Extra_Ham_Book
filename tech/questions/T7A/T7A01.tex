\subsection{Receiver Signal Detection Ability}
\label{T7A01}

\begin{tcolorbox}[colback=gray!10!white,colframe=black!75!black,title=T7A01]
Which term describes the ability of a receiver to detect the presence of a signal?
\begin{enumerate}[noitemsep]
    \item Linearity
    \item \textbf{Sensitivity}
    \item Selectivity
    \item Total Harmonic Distortion
\end{enumerate}
\end{tcolorbox}

\subsubsection*{Intuitive Explanation}
Imagine you're trying to hear a whisper in a noisy room. The better your ears are at picking up that whisper, the more sensitive they are. Similarly, in radio technology, \textbf{sensitivity} refers to how well a receiver can detect a weak signal amidst noise.

\subsubsection*{Advanced Explanation}
Sensitivity in radio receivers is a measure of the minimum signal strength that the receiver can detect. It is typically expressed in microvolts ($\mu$ V) or decibels relative to one milliwatt (dBm). A receiver with high sensitivity can detect weaker signals, which is crucial for long-distance communication or in environments with high noise levels. Sensitivity is influenced by factors such as the receiver's noise figure, bandwidth, and the quality of its components. 

% Diagram Prompt: Generate a diagram showing a receiver detecting a weak signal amidst noise. Use Python with Matplotlib to plot a signal with noise and highlight the receiver's sensitivity threshold.