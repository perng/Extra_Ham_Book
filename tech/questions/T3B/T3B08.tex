\subsection{VHF Frequency Range}
\label{T3B08}

\begin{tcolorbox}[colback=gray!10!white,colframe=black!75!black,title=T3B08]
What frequency range is referred to as VHF?
\begin{enumerate}[label=\Alph*)]
    \item 30 kHz to 300 kHz
    \item \textbf{30 MHz to 300 MHz}
    \item 300 kHz to 3000 kHz
    \item 300 MHz to 3000 MHz
\end{enumerate}
\end{tcolorbox}

\subsubsection*{Intuitive Explanation}
Imagine the radio spectrum as a giant highway with different lanes. Each lane is a different frequency range, and VHF is like the middle lane. It’s not too slow (like AM radio) and not too fast (like UHF). VHF stands for Very High Frequency, and it’s the range where you’ll find FM radio, TV channels, and even some walkie-talkies. So, if you’re tuning into your favorite FM station, you’re cruising in the VHF lane!

\subsubsection*{Advanced Explanation}
The VHF (Very High Frequency) band spans from 30 MHz to 300 MHz. This range is particularly useful for FM radio broadcasting (typically 88 MHz to 108 MHz), television broadcasting, and two-way land mobile radio systems. The wavelength of VHF signals ranges from 10 meters to 1 meter, which allows for relatively long-distance communication with minimal interference compared to lower frequency bands. The propagation characteristics of VHF make it suitable for both line-of-sight and slightly beyond line-of-sight communication, depending on the environment and antenna height.

Mathematically, the wavelength \(\lambda\) of a signal can be calculated using the formula:
\[
\lambda = \frac{c}{f}
\]
where \(c\) is the speed of light (\(3 \times 10^8\) m/s) and \(f\) is the frequency. For example, at 100 MHz:
\[
\lambda = \frac{3 \times 10^8}{100 \times 10^6} = 3 \text{ meters}
\]
This wavelength is ideal for many communication applications, balancing between coverage and signal clarity.

% Diagram Prompt: Generate a diagram showing the radio spectrum with VHF highlighted between 30 MHz and 300 MHz.