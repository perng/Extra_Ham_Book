\subsection{Polarization of Radio Waves}
\label{T3B02}

\begin{tcolorbox}[colback=gray!10!white,colframe=black!75!black,title=T3B02]
What property of a radio wave defines its polarization?
\begin{enumerate}[label=\Alph*)]
    \item \textbf{The orientation of the electric field}
    \item The orientation of the magnetic field
    \item The ratio of the energy in the magnetic field to the energy in the electric field
    \item The ratio of the velocity to the wavelength
\end{enumerate}
\end{tcolorbox}

\subsubsection{Intuitive Explanation}
Imagine a radio wave as a wiggly rope that you’re shaking up and down. The way you shake the rope—whether it’s up and down, side to side, or in a circle—determines its polarization. In radio waves, it’s the electric field that does the wiggling, and the direction it wiggles is what we call polarization. So, if you’re asked what defines the polarization of a radio wave, it’s all about which way the electric field is pointing as it travels through space.

\subsubsection{Advanced Explanation}
Polarization of a radio wave is determined by the orientation of the electric field vector \(\mathbf{E}\) in the plane perpendicular to the direction of wave propagation. The electric field \(\mathbf{E}\) and the magnetic field \(\mathbf{B}\) are both perpendicular to each other and to the direction of propagation. However, it is the electric field that defines the polarization. 

Mathematically, the electric field of a plane wave can be expressed as:
\[
\mathbf{E}(z, t) = \mathbf{E}_0 \cos(kz - \omega t + \phi)
\]
where \(\mathbf{E}_0\) is the amplitude vector, \(k\) is the wave number, \(\omega\) is the angular frequency, \(z\) is the direction of propagation, \(t\) is time, and \(\phi\) is the phase. The orientation of \(\mathbf{E}_0\) in the \(xy\)-plane (assuming propagation along the \(z\)-axis) determines the polarization. For example, if \(\mathbf{E}_0\) is aligned along the \(x\)-axis, the wave is said to be linearly polarized in the \(x\)-direction.

The magnetic field \(\mathbf{B}\) is related to the electric field by Maxwell's equations, but it does not define the polarization. The energy ratios or velocity-to-wavelength ratios are not relevant to the concept of polarization.

% Diagram prompt: A diagram showing the electric and magnetic fields of a radio wave, with the electric field vector highlighted to indicate polarization.