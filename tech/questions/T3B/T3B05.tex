\subsection{Relationship Between Wavelength and Frequency}
\label{T3B05}

\begin{tcolorbox}[colback=gray!10!white,colframe=black!75!black,title=T3B05]
What is the relationship between wavelength and frequency?
\begin{enumerate}[label=\Alph*)]
    \item Wavelength gets longer as frequency increases
    \item \textbf{Wavelength gets shorter as frequency increases}
    \item Wavelength and frequency are unrelated
    \item Wavelength and frequency increase as path length increases
\end{enumerate}
\end{tcolorbox}

\subsubsection{Intuitive Explanation}
Imagine you're at a concert, and the band is playing a really fast song. The beats are coming at you quickly, one after another. Now, think of each beat as a wave. If the beats are coming faster (higher frequency), the distance between each beat (wavelength) gets shorter. It's like when you're clapping really fast—your hands are closer together each time you clap. So, when the frequency goes up, the wavelength goes down. Easy, right?

\subsubsection{Advanced Explanation}
The relationship between wavelength ($\lambda$) and frequency ($f$) is governed by the equation:

\[
v = \lambda \cdot f
\]

where $v$ is the speed of the wave. For electromagnetic waves, such as radio waves, the speed $v$ is the speed of light ($c$), which is approximately $3 \times 10^8$ meters per second. Rearranging the equation to solve for wavelength gives:

\[
\lambda = \frac{c}{f}
\]

From this equation, it is clear that wavelength is inversely proportional to frequency. As frequency increases, wavelength decreases, and vice versa. This fundamental relationship is crucial in understanding how different frequencies of radio waves propagate and interact with their environment.

% Prompt for generating a diagram: A graph showing wavelength on the x-axis and frequency on the y-axis, with a hyperbolic curve illustrating the inverse relationship.