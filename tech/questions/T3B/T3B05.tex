\subsection{Wavelength and Frequency Relationship}
\label{T3B05}

\begin{tcolorbox}[colback=gray!10!white,colframe=black!75!black,title=T3B05]
What is the relationship between wavelength and frequency?
\begin{enumerate}[noitemsep]
    \item Wavelength gets longer as frequency increases
    \item \textbf{Wavelength gets shorter as frequency increases}
    \item Wavelength and frequency are unrelated
    \item Wavelength and frequency increase as path length increases
\end{enumerate}
\end{tcolorbox}

\subsubsection*{Intuitive Explanation}
Imagine you're at a concert, and the band is playing a song. The faster they play (higher frequency), the closer together the waves of sound are (shorter wavelength). Conversely, if they slow down (lower frequency), the waves spread out more (longer wavelength). So, wavelength and frequency are like dance partners—when one moves faster, the other gets closer.

\subsubsection*{Advanced Explanation}
The relationship between wavelength ($\lambda$) and frequency ($f$) is governed by the equation:
\[
v = \lambda \cdot f
\]
where $v$ is the velocity of the wave. For electromagnetic waves in a vacuum, $v$ is the speed of light ($c$), which is a constant. Therefore, as frequency increases, wavelength must decrease to keep the product $\lambda \cdot f$ equal to $c$. This inverse relationship is fundamental in understanding wave behavior in various media.