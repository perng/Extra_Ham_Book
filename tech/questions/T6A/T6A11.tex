\subsection{Non-Rechargeable Battery Chemistry}
\label{T6A11}

\begin{tcolorbox}[colback=gray!10!white,colframe=black!75!black,title=T6A11]
Which of the following battery chemistries is not rechargeable?
\begin{enumerate}[label=\Alph*)]
    \item Nickel-cadmium
    \item \textbf{Carbon-zinc}
    \item Lead-acid
    \item Lithium-ion
\end{enumerate}
\end{tcolorbox}

\subsubsection{Intuitive Explanation}
Imagine you have a toy that runs on batteries. Some batteries can be recharged, like the ones in your phone or laptop. You plug them in, and they get their energy back. But some batteries are like a one-time snack—once they’re used up, they’re done! Carbon-zinc batteries are like that. You can’t recharge them, so once they’re out of juice, you have to throw them away and get new ones. That’s why Carbon-zinc is the odd one out in this list!

\subsubsection{Advanced Explanation}
Battery chemistries can be broadly categorized into rechargeable and non-rechargeable types. Rechargeable batteries, such as Nickel-cadmium (NiCd), Lead-acid, and Lithium-ion (Li-ion), can undergo multiple charge-discharge cycles due to their reversible chemical reactions. 

On the other hand, non-rechargeable batteries, like Carbon-zinc, have irreversible chemical reactions. Once the reactants are consumed, the battery cannot be recharged. The chemical reaction in a Carbon-zinc battery is as follows:

\[
\text{Zn} + 2\text{MnO}_2 + \text{H}_2\text{O} \rightarrow \text{ZnO} + 2\text{MnOOH}
\]

This reaction is not easily reversible, making Carbon-zinc batteries unsuitable for recharging. In contrast, the reactions in rechargeable batteries are designed to be reversible, allowing them to be charged and discharged multiple times.

% Prompt for diagram: A diagram comparing the chemical reactions of rechargeable and non-rechargeable batteries would be helpful here.