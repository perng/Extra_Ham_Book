\subsection{Battery Chemistries and Rechargeability}
\label{T6A11}

\begin{tcolorbox}[colback=gray!10!white,colframe=black!75!black,title=T6A11]
Which of the following battery chemistries is not rechargeable?
\begin{enumerate}[noitemsep]
    \item Nickel-cadmium
    \item \textbf{Carbon-zinc}
    \item Lead-acid
    \item Lithium-ion
\end{enumerate}
\end{tcolorbox}

\subsubsection*{Intuitive Explanation}
Think of batteries like water bottles. Some bottles can be refilled (rechargeable), while others are single-use (non-rechargeable). In this question, we're identifying which bottle can't be refilled. Carbon-zinc batteries are like disposable water bottles—once they're empty, they're done!

\subsubsection*{Advanced Explanation}
Battery chemistries determine whether a battery can be recharged. Rechargeable batteries, such as Nickel-cadmium, Lead-acid, and Lithium-ion, can undergo reversible chemical reactions, allowing them to be recharged multiple times. Carbon-zinc batteries, however, rely on irreversible chemical reactions, making them non-rechargeable. This fundamental difference in chemistry is why Carbon-zinc batteries are single-use.