\subsection{Energy Storage in Magnetic Fields}
\label{T6A06}

\begin{tcolorbox}[colback=gray!10!white,colframe=black!75!black,title=T6A06]
What type of electrical component stores energy in a magnetic field?
\begin{enumerate}[label=\Alph*)]
    \item Varistor
    \item Capacitor
    \item \textbf{Inductor}
    \item Diode
\end{enumerate}
\end{tcolorbox}

\subsubsection{Intuitive Explanation}
Imagine you have a spring. When you compress it, you store energy in it, right? Now, think of an inductor as a kind of spring for electricity. When electricity flows through it, it stores energy in a magnetic field, just like the spring stores energy when you compress it. So, the inductor is the component that stores energy in a magnetic field, just like the spring stores energy when you push it.

\subsubsection{Advanced Explanation}
An inductor is a passive electrical component that stores energy in its magnetic field when an electric current passes through it. The energy stored in an inductor can be calculated using the formula:

\[
E = \frac{1}{2} L I^2
\]

where:
\begin{itemize}
    \item \( E \) is the energy stored in the inductor (in joules),
    \item \( L \) is the inductance of the inductor (in henries),
    \item \( I \) is the current flowing through the inductor (in amperes).
\end{itemize}

Inductors are typically made of a coil of wire, and the magnetic field is generated by the current flowing through this coil. The inductance \( L \) depends on the number of turns in the coil, the cross-sectional area of the coil, and the material of the core around which the coil is wound.

In contrast, a capacitor stores energy in an electric field, a varistor is used to protect circuits from excessive voltage, and a diode allows current to flow in only one direction. Therefore, the correct answer is the inductor, as it is the component that stores energy in a magnetic field.

% Prompt for generating a diagram: 
% Diagram showing a simple inductor with a magnetic field around it, illustrating the energy storage in the magnetic field.