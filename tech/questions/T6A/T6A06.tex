\subsection{Energy Storage in Magnetic Fields}
\label{T6A06}

\begin{tcolorbox}[colback=gray!10!white,colframe=black!75!black,title=T6A06]
What type of electrical component stores energy in a magnetic field?
\begin{enumerate}[noitemsep]
    \item Varistor
    \item Capacitor
    \item \textbf{Inductor}
    \item Diode
\end{enumerate}
\end{tcolorbox}

\subsubsection*{Intuitive Explanation}
Think of an inductor as a tiny electromagnet. When you send electricity through it, it creates a magnetic field, just like how a magnet works. The inductor stores energy in this magnetic field, kind of like how a spring stores energy when you compress it. So, when the electricity stops flowing, the magnetic field collapses, and the energy is released back into the circuit.

\subsubsection*{Advanced Explanation}
An inductor is a passive electrical component that stores energy in its magnetic field when an electric current passes through it. It typically consists of a coil of wire, and the energy stored is given by the formula:
\[
E = \frac{1}{2} L I^2
\]
where \( E \) is the energy stored, \( L \) is the inductance of the coil, and \( I \) is the current flowing through it. The inductor opposes changes in current, which is why it is used in circuits to smooth out current fluctuations. Unlike a capacitor, which stores energy in an electric field, an inductor stores energy in a magnetic field.