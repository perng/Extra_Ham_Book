\subsection{Switch Type in Figure T-2}
\label{T6A12(A)}

\begin{tcolorbox}[colback=gray!10!white,colframe=black!75!black,title=T6A12(A)]
What type of switch is represented by component 3 in figure T-2?
\begin{enumerate}[label=\Alph*)]
    \item \textbf{Single-pole single-throw}
    \item Single-pole double-throw
    \item Double-pole single-throw
    \item Double-pole double-throw
\end{enumerate}
\end{tcolorbox}

\subsubsection{Intuitive Explanation}
Imagine you have a light switch in your room. When you flip it, the light turns on or off. That's a simple switch, right? Now, think of component 3 in figure T-2 as that light switch. It’s just a basic switch that can either be on or off. It’s called a single-pole single-throw switch because it has one input (single-pole) and one output (single-throw). So, it’s like your light switch—simple and straightforward!

\subsubsection{Advanced Explanation}
In electrical engineering, switches are categorized based on their poles and throws. A pole refers to the number of separate circuits that the switch can control, while a throw refers to the number of positions each pole can connect to. 

- \textbf{Single-pole single-throw (SPST)}: This switch has one input (pole) and one output (throw). It can either be open (off) or closed (on). It’s the simplest type of switch, commonly used in basic circuits.

- \textbf{Single-pole double-throw (SPDT)}: This switch has one input and two outputs. It can connect the input to either of the two outputs, but not both at the same time.

- \textbf{Double-pole single-throw (DPST)}: This switch has two inputs and two outputs. It can control two separate circuits simultaneously, but each circuit has only one output.

- \textbf{Double-pole double-throw (DPDT)}: This switch has two inputs and four outputs (two for each input). It can control two separate circuits, and each circuit can be connected to one of two outputs.

In the context of figure T-2, component 3 is a single-pole single-throw (SPST) switch, as it has one input and one output, allowing it to simply turn a circuit on or off.

% Prompt for generating a diagram: 
% Create a diagram showing a simple SPST switch in a circuit, with one input and one output, labeled as component 3.