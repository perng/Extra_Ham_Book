\subsection{Component Opposing Current in DC Circuit}
\label{T6A01}

\begin{tcolorbox}[colback=gray!10!white,colframe=black!75!black,title=T6A01]
What electrical component opposes the flow of current in a DC circuit?
\begin{enumerate}[label=\Alph*]
    \item Inductor
    \item \textbf{Resistor}
    \item Inverter
    \item Transformer
\end{enumerate}
\end{tcolorbox}

\subsubsection{Intuitive Explanation}
Imagine you're trying to push a toy car through a narrow tunnel. The walls of the tunnel are like a resistor—they make it harder for the car (which is like the electric current) to move through. In a DC circuit, the resistor is the component that slows down the flow of electricity, just like the tunnel walls slow down the toy car.

\subsubsection{Advanced Explanation}
In a DC circuit, the component that opposes the flow of current is the resistor. According to Ohm's Law, the voltage \( V \) across a resistor is directly proportional to the current \( I \) flowing through it, and the constant of proportionality is the resistance \( R \). Mathematically, this is expressed as:

\[
V = I \times R
\]

The resistor dissipates electrical energy in the form of heat, thereby reducing the current flow. Unlike inductors, which oppose changes in current in AC circuits, resistors provide a constant opposition to current in both AC and DC circuits. Inverters and transformers, on the other hand, are used to change the voltage or current levels and do not inherently oppose the flow of current.

% Prompt for diagram: A simple DC circuit diagram with a battery, resistor, and connecting wires to illustrate the concept.