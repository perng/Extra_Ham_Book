\subsection{Electrical Parameter Controlled by a Potentiometer}
\label{T6A03}

\begin{tcolorbox}[colback=gray!10!white,colframe=black!75!black,title=T6A03]
What electrical parameter is controlled by a potentiometer?
\begin{enumerate}[label=\Alph*]
    \item Inductance
    \item \textbf{Resistance}
    \item Capacitance
    \item Field strength
\end{enumerate}
\end{tcolorbox}

\subsubsection{Intuitive Explanation}
Imagine a potentiometer as a volume knob on your stereo. When you turn the knob, you’re not changing the type of music or the speakers; you’re just adjusting how loud the music is. In the same way, a potentiometer doesn’t change the type of electrical component; it just adjusts the resistance. So, if you’re asked what a potentiometer controls, think of it as the “volume knob” for resistance in an electrical circuit!

\subsubsection{Advanced Explanation}
A potentiometer is a three-terminal resistor with a sliding or rotating contact that forms an adjustable voltage divider. The primary function of a potentiometer is to vary the resistance in a circuit. The resistance between the two fixed terminals is constant, but the resistance between the wiper (the moving contact) and either of the fixed terminals can be adjusted. 

Mathematically, the resistance \( R \) between the wiper and one of the terminals can be expressed as:
\[ R = R_{\text{total}} \times \frac{x}{L} \]
where \( R_{\text{total}} \) is the total resistance of the potentiometer, \( x \) is the position of the wiper, and \( L \) is the total length of the resistive element.

Potentiometers are commonly used in applications where precise control of resistance is required, such as in volume controls, dimmer switches, and tuning circuits. Understanding how a potentiometer works is fundamental in designing and analyzing circuits that require variable resistance.

% Prompt for generating a diagram: A diagram showing a potentiometer with labeled terminals (A, B, and the wiper) and an arrow indicating the adjustable resistance would be helpful here.