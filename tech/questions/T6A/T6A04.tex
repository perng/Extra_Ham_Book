\subsection{Energy Storage in Electric Fields}
\label{T6A04}

\begin{tcolorbox}[colback=gray!10!white,colframe=black!75!black,title=T6A04]
What electrical component stores energy in an electric field?
\begin{enumerate}[noitemsep]
    \item Varistor
    \item \textbf{Capacitor}
    \item Inductor
    \item Diode
\end{enumerate}
\end{tcolorbox}

\subsubsection*{Intuitive Explanation}
Think of a capacitor as a tiny energy storage box for electricity. When you charge it up, it holds onto that energy in an electric field, just like a balloon holds air. When you need that energy back, the capacitor can release it. It's like a quick snack for your electronic devices!

\subsubsection*{Advanced Explanation}
A capacitor is a passive electronic component that stores energy in an electric field. It consists of two conductive plates separated by an insulating material called a dielectric. When a voltage is applied across the plates, an electric field is established, and energy is stored in this field. The amount of energy stored can be calculated using the formula:

\[
E = \frac{1}{2}CV^2
\]

where \(E\) is the energy stored, \(C\) is the capacitance, and \(V\) is the voltage across the capacitor. Capacitors are widely used in electronic circuits for filtering, energy storage, and signal processing.