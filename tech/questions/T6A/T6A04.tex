\subsection{Energy Storage in Electric Fields}
\label{T6A04}

\begin{tcolorbox}[colback=gray!10!white,colframe=black!75!black,title=T6A04]
What electrical component stores energy in an electric field?
\begin{enumerate}[label=\Alph*)]
    \item Varistor
    \item \textbf{Capacitor}
    \item Inductor
    \item Diode
\end{enumerate}
\end{tcolorbox}

\subsubsection{Intuitive Explanation}
Imagine you have a tiny invisible balloon that can hold electricity. When you pump electricity into it, it stretches and stores the energy, just like a balloon stores air. This magical balloon is called a capacitor! It doesn’t let the electricity escape right away; instead, it holds onto it until you need it later. So, if you’re looking for something that stores energy in an electric field, think of the capacitor as your electricity balloon!

\subsubsection{Advanced Explanation}
A capacitor is an electrical component that stores energy in an electric field. It consists of two conductive plates separated by an insulating material called a dielectric. When a voltage is applied across the plates, an electric field is established, and energy is stored in this field. The amount of energy \( E \) stored in a capacitor can be calculated using the formula:

\[
E = \frac{1}{2} C V^2
\]

where:
\begin{itemize}
    \item \( C \) is the capacitance of the capacitor, measured in farads (F),
    \item \( V \) is the voltage across the capacitor, measured in volts (V).
\end{itemize}

Capacitors are widely used in electronic circuits for various purposes, such as filtering, energy storage, and signal coupling. Unlike inductors, which store energy in a magnetic field, capacitors store energy in an electric field, making them essential components in many electrical and electronic systems.

% Prompt for generating a diagram: A simple diagram showing a capacitor with two plates and a dielectric material between them, with an electric field represented by arrows between the plates.