\subsection{Electrical Component with Conductive Surfaces and Insulator}
\label{T6A05}

\begin{tcolorbox}[colback=gray!10!white,colframe=black!75!black,title=T6A05]
What type of electrical component consists of conductive surfaces separated by an insulator?
\begin{enumerate}[label=\Alph*)]
    \item Resistor
    \item Potentiometer
    \item Oscillator
    \item \textbf{Capacitor}
\end{enumerate}
\end{tcolorbox}

\subsubsection{Intuitive Explanation}
Imagine you have two metal plates (the conductive surfaces) and you put a piece of plastic (the insulator) between them. Now, if you try to push electricity through this setup, the plastic stops it from flowing directly. But here's the cool part: the plates can store the electricity for a little while, like a tiny battery. This setup is called a capacitor. It’s like a sandwich where the bread slices are the metal plates, and the filling is the plastic. Yum!

\subsubsection{Advanced Explanation}
A capacitor is an electrical component that stores energy in an electric field. It consists of two conductive plates separated by a dielectric material (the insulator). The capacitance \( C \) of a capacitor is given by the formula:

\[
C = \frac{\epsilon A}{d}
\]

where:
\begin{itemize}
    \item \( \epsilon \) is the permittivity of the dielectric material,
    \item \( A \) is the area of one plate,
    \item \( d \) is the distance between the two plates.
\end{itemize}

When a voltage \( V \) is applied across the plates, an electric field is established, and charge \( Q \) accumulates on the plates according to the relation:

\[
Q = C \cdot V
\]

Capacitors are widely used in electronic circuits for filtering, energy storage, and signal processing. The dielectric material can be air, ceramic, plastic, or other insulating materials, each affecting the capacitor's properties differently.

% Diagram Prompt: Generate a diagram showing two parallel conductive plates separated by a dielectric material, with labels for the plates, dielectric, and the electric field lines between them.