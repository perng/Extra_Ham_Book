\subsection{Electrical Component with Conductive Surfaces and Insulator}
\label{T6A05}

\begin{tcolorbox}[colback=gray!10!white,colframe=black!75!black,title=T6A05]
What type of electrical component consists of conductive surfaces separated by an insulator?
\begin{enumerate}[noitemsep]
    \item Resistor
    \item Potentiometer
    \item Oscillator
    \item \textbf{Capacitor}
\end{enumerate}
\end{tcolorbox}

\subsubsection*{Intuitive Explanation}
Imagine you have two metal plates and you put a piece of plastic between them. Now, if you connect these plates to a battery, the plates will store some electric charge. This setup is like a tiny battery that can store and release energy quickly. This is essentially what a capacitor does—it stores electrical energy using two conductive surfaces separated by an insulator.

\subsubsection*{Advanced Explanation}
A capacitor is an electrical component that stores energy in an electric field. It consists of two conductive plates separated by a dielectric (insulator). When a voltage is applied across the plates, an electric field develops across the dielectric, causing positive charge to accumulate on one plate and negative charge on the other. The capacitance \( C \) of a capacitor is given by the formula:
\[
C = \frac{\epsilon A}{d}
\]
where \( \epsilon \) is the permittivity of the dielectric, \( A \) is the area of the plates, and \( d \) is the distance between the plates. Capacitors are widely used in electronic circuits for filtering, energy storage, and signal processing.