\subsection{Function of an SPDT Switch}
\label{T6A08}

\begin{tcolorbox}[colback=gray!10!white,colframe=black!75!black,title=T6A08]
What is the function of an SPDT switch?
\begin{enumerate}[noitemsep]
    \item A single circuit is opened or closed
    \item Two circuits are opened or closed
    \item \textbf{A single circuit is switched between one of two other circuits}
    \item Two circuits are each switched between one of two other circuits
\end{enumerate}
\end{tcolorbox}

\subsubsection*{Intuitive Explanation}
An SPDT (Single Pole Double Throw) switch is like a railway switch that can direct a train onto one of two tracks. In electronics, it allows a single input to be connected to one of two outputs. Think of it as a simple way to choose between two options with a single switch.

\subsubsection*{Advanced Explanation}
An SPDT switch has three terminals: one common terminal (the pole) and two other terminals (the throws). The common terminal can be connected to either of the two throws, but not both at the same time. This makes it useful for applications where you need to switch a single signal between two different paths, such as in audio equipment or control circuits. The key feature is that it only switches one circuit at a time, but provides two possible destinations for that circuit.