\subsection{Function of an SPDT Switch}
\label{T6A08}

\begin{tcolorbox}[colback=gray!10!white,colframe=black!75!black,title=T6A08]
What is the function of an SPDT switch?
\begin{enumerate}[label=\Alph*)]
    \item A single circuit is opened or closed
    \item Two circuits are opened or closed
    \item \textbf{A single circuit is switched between one of two other circuits}
    \item Two circuits are each switched between one of two other circuits
\end{enumerate}
\end{tcolorbox}

\subsubsection{Intuitive Explanation}
Imagine you have a light switch that can turn on either a red light or a green light, but not both at the same time. That's what an SPDT (Single Pole Double Throw) switch does! It’s like a traffic cop for electricity, directing the flow to one of two paths. So, when you flip the switch, it chooses between two options, like picking between chocolate or vanilla ice cream, but never both at the same time.

\subsubsection{Advanced Explanation}
An SPDT (Single Pole Double Throw) switch is a type of electrical switch that has one input terminal (the pole) and two output terminals (the throws). The function of the SPDT switch is to connect the input terminal to one of the two output terminals, effectively switching the circuit between two different paths. 

Mathematically, this can be represented as a binary decision:
\[
\text{Output} = 
\begin{cases}
\text{Output}_1 & \text{if switch is in position 1} \\
\text{Output}_2 & \text{if switch is in position 2}
\end{cases}
\]

This type of switch is commonly used in applications where a single input needs to be routed to one of two possible outputs, such as in audio equipment to switch between speakers or in electronic circuits to select between different signal paths.

% Prompt for diagram: Generate a diagram showing an SPDT switch with one input terminal connected to two output terminals, illustrating the switching mechanism.