\subsection{Power Increase in Decibels}
\label{T5B11}

\begin{tcolorbox}[colback=gray!10!white,colframe=black!75!black,title=T5B11]
Which decibel value represents a power increase from 20 watts to 200 watts?
\begin{enumerate}[noitemsep]
    \item \textbf{10 dB}
    \item 12 dB
    \item 18 dB
    \item 28 dB
\end{enumerate}
\end{tcolorbox}

\subsubsection{Intuitive Explanation}
Imagine you have a small speaker that uses 20 watts of power. If you upgrade to a bigger speaker that uses 200 watts, how much louder is it? Decibels (dB) are used to measure this increase in power. A 10 dB increase means the power has increased by a factor of 10. So, going from 20 watts to 200 watts is a 10 dB increase.

\subsubsection{Advanced Explanation}
The decibel (dB) is a logarithmic unit used to express the ratio of two power levels. The formula to calculate the power ratio in decibels is:

\[
\text{dB} = 10 \log_{10}\left(\frac{P_2}{P_1}\right)
\]

Where \( P_1 \) is the initial power level and \( P_2 \) is the final power level. In this case, \( P_1 = 20 \) watts and \( P_2 = 200 \) watts. Plugging these values into the formula:

\[
\text{dB} = 10 \log_{10}\left(\frac{200}{20}\right) = 10 \log_{10}(10) = 10 \times 1 = 10 \text{ dB}
\]

Thus, the power increase from 20 watts to 200 watts is 10 dB.