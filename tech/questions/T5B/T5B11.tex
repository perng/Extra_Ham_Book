\subsection{Power Increase in Decibels}
\label{T5B11}

\begin{tcolorbox}[colback=gray!10!white,colframe=black!75!black,title=T5B11]
Which decibel value represents a power increase from 20 watts to 200 watts?
\begin{enumerate}[label=\Alph*)]
    \item \textbf{10 dB}
    \item 12 dB
    \item 18 dB
    \item 28 dB
\end{enumerate}
\end{tcolorbox}

\subsubsection{Intuitive Explanation}
Imagine you have a small flashlight that uses 20 watts of power. Now, you upgrade to a super-bright flashlight that uses 200 watts of power. That's 10 times more power! In the world of decibels (dB), which is a way to measure how much something increases or decreases, a 10 times increase in power is represented by 10 dB. So, the correct answer is 10 dB. It's like saying your new flashlight is 10 dB brighter than the old one!

\subsubsection{Advanced Explanation}
To calculate the power increase in decibels, we use the formula:

\[
\text{dB} = 10 \log_{10}\left(\frac{P_2}{P_1}\right)
\]

where \(P_1\) is the initial power and \(P_2\) is the final power. In this case, \(P_1 = 20\) watts and \(P_2 = 200\) watts. Plugging these values into the formula:

\[
\text{dB} = 10 \log_{10}\left(\frac{200}{20}\right) = 10 \log_{10}(10)
\]

Since \(\log_{10}(10) = 1\), the calculation simplifies to:

\[
\text{dB} = 10 \times 1 = 10 \text{ dB}
\]

Thus, the power increase from 20 watts to 200 watts is 10 dB. This formula is essential in radio technology for comparing power levels, especially when dealing with signal strength and amplification.

% Diagram prompt: Generate a diagram showing the power levels of 20 watts and 200 watts, with an arrow indicating the increase and the corresponding 10 dB label.