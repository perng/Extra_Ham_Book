\subsection{Frequency Conversion: 2425 MHz}
\label{T5B13}

\begin{tcolorbox}[colback=gray!10!white,colframe=black!75!black,title=T5B13]
Which is equal to 2425 MHz?
\begin{enumerate}[noitemsep]
    \item 0.002425 GHz
    \item 24.25 GHz
    \item \textbf{2.425 GHz}
    \item 2425 GHz
\end{enumerate}
\end{tcolorbox}

\subsubsection*{Intuitive Explanation}
Imagine you have a big number like 2425 MHz, and you want to make it easier to read by converting it to GHz. Think of MHz as millions and GHz as billions. To convert MHz to GHz, you just need to move the decimal point three places to the left. So, 2425 MHz becomes 2.425 GHz. Easy, right?

\subsubsection*{Advanced Explanation}
The question involves converting a frequency from megahertz (MHz) to gigahertz (GHz). The prefix mega denotes \(10^6\) and giga denotes \(10^9\). Therefore, to convert MHz to GHz, you divide by \(10^3\) (or move the decimal point three places to the left). 

Mathematically, the conversion is:
\[
2425 \, \text{MHz} = \frac{2425}{10^3} \, \text{GHz} = 2.425 \, \text{GHz}
\]
This confirms that the correct answer is 2.425 GHz.