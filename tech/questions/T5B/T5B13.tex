\subsection{Frequency Conversion}\label{T5B13}

\begin{tcolorbox}[colback=gray!10!white,colframe=black!75!black,title=T5B13]
Which is equal to 2425 MHz?
\begin{enumerate}[label=\Alph*)]
    \item 0.002425 GHz
    \item 24.25 GHz
    \item \textbf{2.425 GHz}
    \item 2425 GHz
\end{enumerate}
\end{tcolorbox}

\subsubsection*{Intuitive Explanation}
Imagine you have a giant pizza, and you want to share it with your friends. The pizza is cut into 1000 tiny slices, and each slice is called a megahertz (MHz). Now, if you have 2425 slices, how many whole pizzas do you have? Well, since 1000 slices make one pizza, 2425 slices would be 2 whole pizzas and 425 slices left over. In the world of radio frequencies, we call one whole pizza a gigahertz (GHz). So, 2425 MHz is the same as 2.425 GHz. Easy peasy, pizza squeezy!

\subsubsection*{Advanced Explanation}
To convert a frequency from megahertz (MHz) to gigahertz (GHz), we use the relationship:
\[
1 \text{ GHz} = 1000 \text{ MHz}
\]
Given the frequency \( f = 2425 \text{ MHz} \), we can convert it to GHz by dividing by 1000:
\[
f_{\text{GHz}} = \frac{2425 \text{ MHz}}{1000} = 2.425 \text{ GHz}
\]
This conversion is essential in radio technology, where frequencies are often expressed in different units depending on the context. Understanding these conversions allows engineers to work seamlessly across different frequency ranges and applications.

% Diagram Prompt: A simple diagram showing the conversion from MHz to GHz with a pizza analogy could be helpful here.