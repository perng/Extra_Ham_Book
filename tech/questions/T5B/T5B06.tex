\subsection{Current Unit Conversion}
\label{T5B06}

\begin{tcolorbox}[colback=gray!10!white,colframe=black!75!black,title=T5B06]
Which is equal to 3000 milliamperes?
\begin{enumerate}[label=\Alph*)]
    \item 0.003 amperes
    \item 0.3 amperes
    \item 3,000,000 amperes
    \item \textbf{3 amperes}
\end{enumerate}
\end{tcolorbox}

\subsubsection{Intuitive Explanation}
Imagine you have a big bottle of soda. The bottle holds 3000 milliliters of soda. Now, if you want to know how many liters that is, you divide by 1000 because there are 1000 milliliters in a liter. So, 3000 milliliters is the same as 3 liters. Similarly, 3000 milliamperes is the same as 3 amperes because there are 1000 milliamperes in one ampere. Easy peasy!

\subsubsection{Advanced Explanation}
The question involves converting milliamperes (mA) to amperes (A). The prefix milli- denotes a factor of \(10^{-3}\). Therefore, to convert milliamperes to amperes, you multiply by \(10^{-3}\):

\[
3000 \, \text{mA} = 3000 \times 10^{-3} \, \text{A} = 3 \, \text{A}
\]

This conversion is based on the International System of Units (SI) prefixes, which are used to denote multiples and submultiples of units. Understanding these prefixes is crucial for accurate unit conversions in various scientific and engineering contexts.

% Prompt for generating a diagram: A simple diagram showing the conversion from milliamperes to amperes with a scale or a step-by-step visual representation.