\subsection{Kilovolt Definition}\label{T5B03}

\begin{tcolorbox}[colback=gray!10!white,colframe=black!75!black,title=T5B03]
Which is equal to one kilovolt?
\begin{enumerate}[noitemsep]
    \item One one-thousandth of a volt
    \item One hundred volts
    \item \textbf{One thousand volts}
    \item One million volts
\end{enumerate}
\end{tcolorbox}

\subsubsection*{Intuitive Explanation}
A kilovolt is simply a way to say one thousand volts. Think of it like a kilometer being one thousand meters. It's just a bigger unit for measuring voltage.

\subsubsection*{Advanced Explanation}
The prefix kilo in the International System of Units (SI) denotes a factor of one thousand. Therefore, one kilovolt (kV) is equivalent to one thousand volts (V). This is a standard unit used in electrical engineering to measure high voltages, such as those found in power transmission lines.