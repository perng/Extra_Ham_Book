\subsection{Kilovolt Definition}
\label{T5B03}

\begin{tcolorbox}[colback=gray!10!white,colframe=black!75!black,title=T5B03]
Which is equal to one kilovolt?
\begin{enumerate}[label=\Alph*)]
    \item One one-thousandth of a volt
    \item One hundred volts
    \item \textbf{One thousand volts}
    \item One million volts
\end{enumerate}
\end{tcolorbox}

\subsubsection{Intuitive Explanation}
Imagine you have a bunch of volts, like a stack of coins. Now, if you have one kilovolt, it's like having a big pile of 1,000 volts! So, when someone asks, Which is equal to one kilovolt? you can confidently say, One thousand volts! It's like saying a kilo of apples is 1,000 apples. Easy peasy!

\subsubsection{Advanced Explanation}
In the International System of Units (SI), the prefix kilo- denotes a factor of \(10^3\). Therefore, one kilovolt (kV) is equivalent to \(1 \times 10^3\) volts (V). Mathematically, this can be expressed as:

\[
1 \text{ kV} = 1 \times 10^3 \text{ V} = 1000 \text{ V}
\]

This means that one kilovolt is exactly one thousand volts. The other options provided in the question are incorrect because:
\begin{itemize}
    \item One one-thousandth of a volt is a millivolt (mV), which is \(1 \times 10^{-3}\) V.
    \item One hundred volts is simply 100 V, which is \(1 \times 10^2\) V.
    \item One million volts is a megavolt (MV), which is \(1 \times 10^6\) V.
\end{itemize}

Understanding these prefixes is crucial in fields like electrical engineering and physics, where different magnitudes of voltage are commonly encountered.

% Diagram Prompt: Generate a diagram showing the relationship between volts, kilovolts, millivolts, and megavolts.