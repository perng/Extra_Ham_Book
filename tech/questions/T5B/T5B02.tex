\subsection{Understanding Hertz and Frequency Conversion}\label{T5B02}

\begin{tcolorbox}[colback=gray!10!white,colframe=black!75!black,title=T5B02]
Which is equal to 1,500,000 hertz?
\begin{enumerate}[label=\Alph*)]
    \item \textbf{1500 kHz}
    \item 1500 MHz
    \item 15 GHz
    \item 150 kHz
\end{enumerate}
\end{tcolorbox}

\subsubsection{Intuitive Explanation}
Imagine you have a big jar of jellybeans, and you want to count them. Instead of counting each one individually, you decide to group them into smaller jars. Each smaller jar holds 1,000 jellybeans. Now, if you have 1,500,000 jellybeans, how many smaller jars would you need? You'd need 1,500 jars, right? In the same way, 1,500,000 hertz is like the big jar of jellybeans, and 1,500 kHz is like the smaller jars. So, 1,500,000 hertz is equal to 1,500 kHz!

\subsubsection{Advanced Explanation}
To understand this question, we need to know the relationship between hertz (Hz), kilohertz (kHz), and megahertz (MHz). The prefix kilo means 1,000, and mega means 1,000,000. Therefore:
\[
1 \text{ kHz} = 1,000 \text{ Hz}
\]
\[
1 \text{ MHz} = 1,000,000 \text{ Hz}
\]
Given that, we can convert 1,500,000 Hz to kHz by dividing by 1,000:
\[
1,500,000 \text{ Hz} \div 1,000 = 1,500 \text{ kHz}
\]
Thus, 1,500,000 Hz is equal to 1,500 kHz. This conversion is essential in radio technology, where frequencies are often expressed in kHz or MHz for simplicity.

% Prompt for generating a diagram: A diagram showing the conversion from Hz to kHz with arrows and labels would be helpful here.