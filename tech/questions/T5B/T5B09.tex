\subsection{Power Increase in Decibels}
\label{T5B09}

\begin{tcolorbox}[colback=gray!10!white,colframe=black!75!black,title=T5B09]
Which decibel value most closely represents a power increase from 5 watts to 10 watts?
\begin{enumerate}[noitemsep]
    \item 2 dB
    \item \textbf{3 dB}
    \item 5 dB
    \item 10 dB
\end{enumerate}
\end{tcolorbox}

\subsubsection*{Intuitive Explanation}
Imagine you have a light bulb that uses 5 watts of power. If you double the power to 10 watts, the bulb will be brighter. In the world of decibels (dB), which measure changes in power, doubling the power corresponds to an increase of approximately 3 dB. So, going from 5 watts to 10 watts is like turning up the brightness by 3 dB.

\subsubsection*{Advanced Explanation}
The decibel (dB) is a logarithmic unit used to express the ratio of two power levels. The formula to calculate the power increase in decibels is:

\[
\text{dB} = 10 \log_{10}\left(\frac{P_2}{P_1}\right)
\]

Where \( P_1 \) is the initial power and \( P_2 \) is the final power. In this case, \( P_1 = 5 \) watts and \( P_2 = 10 \) watts. Plugging these values into the formula:

\[
\text{dB} = 10 \log_{10}\left(\frac{10}{5}\right) = 10 \log_{10}(2) \approx 10 \times 0.301 = 3.01 \text{ dB}
\]

Thus, the power increase from 5 watts to 10 watts is approximately 3 dB.