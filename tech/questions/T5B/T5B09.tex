\subsection{Power Increase in Decibels}
\label{T5B09}

\begin{tcolorbox}[colback=gray!10!white,colframe=black!75!black,title=T5B09]
Which decibel value most closely represents a power increase from 5 watts to 10 watts?
\begin{enumerate}[label=\Alph*)]
    \item 2 dB
    \item \textbf{3 dB}
    \item 5 dB
    \item 10 dB
\end{enumerate}
\end{tcolorbox}

\subsubsection{Intuitive Explanation}
Imagine you have a flashlight that uses 5 watts of power. If you upgrade to a flashlight that uses 10 watts, it’s like doubling the brightness! In the world of decibels, doubling the power is like adding 3 dB. So, the answer is 3 dB. It’s like saying, “Hey, my flashlight is now twice as bright, and that’s a 3 dB increase!”

\subsubsection{Advanced Explanation}
To calculate the power increase in decibels (dB), we use the formula:
\[
\text{dB} = 10 \log_{10}\left(\frac{P_2}{P_1}\right)
\]
where \(P_1\) is the initial power and \(P_2\) is the final power. 

Given:
\[
P_1 = 5 \text{ watts}, \quad P_2 = 10 \text{ watts}
\]
Substitute these values into the formula:
\[
\text{dB} = 10 \log_{10}\left(\frac{10}{5}\right) = 10 \log_{10}(2)
\]
We know that \(\log_{10}(2) \approx 0.3010\), so:
\[
\text{dB} = 10 \times 0.3010 = 3.01 \text{ dB}
\]
Rounding to the nearest whole number, the power increase is approximately 3 dB.

This calculation shows that a doubling of power results in a 3 dB increase, which is a fundamental concept in radio technology and signal processing.

% Diagram Prompt: Generate a diagram showing the relationship between power levels in watts and their corresponding decibel values, highlighting the 5W to 10W increase.