\subsection{Understanding Frequency Conversion}\label{T5B07}

\begin{tcolorbox}[colback=gray!10!white,colframe=black!75!black,title=T5B07]
Which is equal to 3.525 MHz?
\begin{enumerate}[label=\Alph*)]
    \item 0.003525 kHz
    \item 35.25 kHz
    \item \textbf{3525 kHz}
    \item 3,525,000 kHz
\end{enumerate}
\end{tcolorbox}

\subsubsection{Intuitive Explanation}
Imagine you have a big bag of candies, and you want to share them with your friends. If you have 3.525 million candies, it's the same as having 3,525 thousand candies. Just like that, 3.525 MHz (which is 3.525 million hertz) is the same as 3,525 kHz (which is 3,525 thousand hertz). So, the correct answer is C: 3525 kHz. Easy peasy!

\subsubsection{Advanced Explanation}
To convert megahertz (MHz) to kilohertz (kHz), we use the fact that 1 MHz is equal to 1,000 kHz. Therefore, to convert 3.525 MHz to kHz, we multiply by 1,000:

\[
3.525 \, \text{MHz} \times 1,000 = 3,525 \, \text{kHz}
\]

This calculation shows that 3.525 MHz is equivalent to 3,525 kHz. The other options either understate or overstate the conversion by incorrect factors of 1,000. Understanding the relationship between these units is crucial in radio technology, where frequencies are often expressed in different scales depending on the context.

% [Prompt for diagram: A simple diagram showing the conversion from MHz to kHz with arrows and labels could be helpful here.]