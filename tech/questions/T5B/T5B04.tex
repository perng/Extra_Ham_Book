\subsection{Understanding Microvolts}
\label{T5B04}

\begin{tcolorbox}[colback=gray!10!white,colframe=black!75!black,title=T5B04]
Which is equal to one microvolt?
\begin{enumerate}[label=\Alph*)]
    \item \textbf{One one-millionth of a volt}
    \item One million volts
    \item One thousand kilovolts
    \item One one-thousandth of a volt
\end{enumerate}
\end{tcolorbox}

\subsubsection{Intuitive Explanation}
Imagine you have a big pizza, and you cut it into a million tiny slices. One of those slices is like one microvolt compared to the whole pizza, which is one volt. So, one microvolt is just a super tiny piece of a volt—specifically, one one-millionth of it. It's like comparing a single grain of sand to a whole beach!

\subsubsection{Advanced Explanation}
In the International System of Units (SI), the prefix micro ($\mu$) denotes a factor of \(10^{-6}\). Therefore, one microvolt (\(\mu V\)) is defined as:
\[
1 \mu V = 10^{-6} V
\]
This means that one microvolt is one one-millionth of a volt. To put it into perspective, if you have a voltage of 1 volt, dividing it by one million gives you one microvolt. This is a very small unit of voltage, often used in applications where extremely low voltages are measured, such as in biomedical sensors or certain types of electronic circuits.

% Diagram prompt: A simple diagram showing a volt divided into one million equal parts, with one part labeled as one microvolt.