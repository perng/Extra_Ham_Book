\subsection{Understanding Milliwatts and Watts}
\label{T5B05}

\begin{tcolorbox}[colback=gray!10!white,colframe=black!75!black,title=T5B05]
Which is equal to 500 milliwatts?
\begin{enumerate}[noitemsep]
    \item 0.02 watts
    \item \textbf{0.5 watts}
    \item 5 watts
    \item 50 watts
\end{enumerate}
\end{tcolorbox}

\subsubsection*{Intuitive Explanation}
Imagine you have a small LED light that uses 500 milliwatts of power. To understand how much power that is in watts, think of it like this: 1 watt is like a big glass of water, and 1 milliwatt is like a tiny drop from that glass. So, 500 milliwatts is like 500 tiny drops, which is half of the big glass. Therefore, 500 milliwatts is equal to 0.5 watts.

\subsubsection*{Advanced Explanation}
The unit milliwatt (mW) is a subunit of the watt (W), where 1 milliwatt is equal to one-thousandth of a watt. Mathematically, this can be expressed as:
\[
1 \text{ mW} = \frac{1}{1000} \text{ W} = 0.001 \text{ W}
\]
To convert 500 milliwatts to watts, you multiply by the conversion factor:
\[
500 \text{ mW} \times 0.001 \frac{\text{W}}{\text{mW}} = 0.5 \text{ W}
\]
Thus, 500 milliwatts is equal to 0.5 watts. This conversion is essential in radio technology when dealing with power levels in different units.