\subsection{Understanding Milliwatts and Watts}
\label{T5B05}

\begin{tcolorbox}[colback=gray!10!white,colframe=black!75!black,title=T5B05]
Which is equal to 500 milliwatts?
\begin{enumerate}[label=\Alph*)]
    \item 0.02 watts
    \item \textbf{0.5 watts}
    \item 5 watts
    \item 50 watts
\end{enumerate}
\end{tcolorbox}

\subsubsection{Intuitive Explanation}
Imagine you have a tiny light bulb that uses 500 milliwatts of power. Now, think of a milliwatt as a tiny, tiny piece of a watt—like a crumb from a big cookie. If you have 500 of these crumbs, how many cookies do you have? Well, since 1,000 milliwatts make up 1 watt, 500 milliwatts is like half a cookie, or 0.5 watts. So, the correct answer is 0.5 watts!

\subsubsection{Advanced Explanation}
To understand this question, we need to know the relationship between milliwatts (mW) and watts (W). The prefix milli- means one-thousandth, so 1 milliwatt is equal to \(1 \times 10^{-3}\) watts. Therefore, to convert milliwatts to watts, we use the following formula:

\[
\text{Watts} = \text{Milliwatts} \times 10^{-3}
\]

Given that we have 500 milliwatts, we can calculate the equivalent in watts:

\[
\text{Watts} = 500 \, \text{mW} \times 10^{-3} = 0.5 \, \text{W}
\]

Thus, 500 milliwatts is equal to 0.5 watts. This conversion is essential in radio technology, where power levels are often measured in milliwatts or watts, depending on the application.

% Prompt for generating a diagram: A simple diagram showing the conversion from milliwatts to watts with an arrow pointing from 500 mW to 0.5 W.