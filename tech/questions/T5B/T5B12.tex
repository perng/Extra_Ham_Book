\subsection{Frequency Conversion}\label{T5B12}

\begin{tcolorbox}[colback=gray!10!white,colframe=black!75!black,title=T5B12]
Which is equal to 28400 kHz?
\begin{enumerate}[label=\Alph*)]
    \item 28.400 kHz
    \item 2.800 MHz
    \item 284.00 MHz
    \item \textbf{28.400 MHz}
\end{enumerate}
\end{tcolorbox}

\subsubsection{Intuitive Explanation}
Imagine you have a big number like 28400 kHz, and you want to make it easier to say and understand. Think of it like converting a big pile of pennies into dollars. Just like 100 pennies make a dollar, 1000 kHz make a MHz. So, if you take 28400 kHz and divide it by 1000, you get 28.400 MHz. That's like saying you have 28 dollars and 40 cents instead of 28400 pennies. Easy, right?

\subsubsection{Advanced Explanation}
To convert a frequency from kilohertz (kHz) to megahertz (MHz), you need to understand the relationship between these units. The prefix kilo means $10^3$, and mega means $10^6$. Therefore, 1 MHz is equal to 1000 kHz. 

Given the frequency 28400 kHz, the conversion to MHz is done by dividing by 1000:

\[
28400 \, \text{kHz} \div 1000 = 28.400 \, \text{MHz}
\]

This calculation shows that 28400 kHz is equivalent to 28.400 MHz. The other options either do not convert correctly or use the wrong units. For example, 28.400 kHz is much smaller than 28400 kHz, and 2.800 MHz is only 2800 kHz, which is also incorrect. 284.00 MHz would be 284000 kHz, which is ten times larger than the given frequency.

Understanding these conversions is crucial in radio technology, where frequencies are often expressed in different units depending on the context.

% Prompt for generating a diagram: A simple diagram showing the conversion from kHz to MHz with an arrow labeled divide by 1000 could be helpful here.