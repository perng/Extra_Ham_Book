\subsection{Power Decrease in Decibels}
\label{T5B10}

\begin{tcolorbox}[colback=gray!10!white,colframe=black!75!black,title=T5B10]
Which decibel value most closely represents a power decrease from 12 watts to 3 watts?
\begin{enumerate}[label=\Alph*)]
    \item -1 dB
    \item -3 dB
    \item \textbf{-6 dB}
    \item -9 dB
\end{enumerate}
\end{tcolorbox}

\subsubsection{Intuitive Explanation}
Imagine you have a big bag of 12 candies, and you give away some of them until you only have 3 left. That's a big drop in your candy stash! In the world of radio, we measure this kind of drop using something called decibels (dB). A decibel is just a way to compare two amounts of power. In this case, the power dropped from 12 watts to 3 watts, which is a big decrease. The correct answer, -6 dB, is like saying you lost half of your candy stash twice. First, you went from 12 to 6 (that's -3 dB), and then from 6 to 3 (another -3 dB). So, total, you lost -6 dB of power.

\subsubsection{Advanced Explanation}
To calculate the power decrease in decibels, we use the formula:

\[
\text{dB} = 10 \log_{10}\left(\frac{P_2}{P_1}\right)
\]

where \(P_1\) is the initial power and \(P_2\) is the final power. In this case, \(P_1 = 12\) watts and \(P_2 = 3\) watts. Plugging these values into the formula:

\[
\text{dB} = 10 \log_{10}\left(\frac{3}{12}\right) = 10 \log_{10}\left(\frac{1}{4}\right)
\]

We know that \(\log_{10}\left(\frac{1}{4}\right) = \log_{10}(1) - \log_{10}(4) = 0 - 0.602 = -0.602\). Therefore:

\[
\text{dB} = 10 \times (-0.602) = -6.02 \text{ dB}
\]

This value is closest to -6 dB, which corresponds to option C. The decibel scale is logarithmic, meaning that a decrease of 3 dB represents a halving of power. Here, the power decreased by a factor of 4, which is equivalent to two halvings, hence a total decrease of 6 dB.

% Diagram prompt: A graph showing the relationship between power levels and decibel changes, highlighting the points at 12 watts and 3 watts with corresponding dB values.