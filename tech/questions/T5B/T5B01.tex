\subsection{Milliamperes Conversion}
\label{T5B01}

\begin{tcolorbox}[colback=gray!10!white,colframe=black!75!black,title=T5B01]
How many milliamperes is 1.5 amperes?
\begin{enumerate}[label=\Alph*)]
    \item 15 milliamperes
    \item 150 milliamperes
    \item \textbf{1500 milliamperes}
    \item 15,000 milliamperes
\end{enumerate}
\end{tcolorbox}

\subsubsection{Intuitive Explanation}
Imagine you have a big bottle of soda that holds 1.5 liters. Now, if you pour that soda into smaller bottles that each hold 1 milliliter, how many small bottles would you need? Well, 1 liter is 1000 milliliters, so 1.5 liters would be 1500 milliliters. Similarly, 1 ampere is 1000 milliamperes, so 1.5 amperes is 1500 milliamperes. Easy peasy!

\subsubsection{Advanced Explanation}
The question involves converting amperes (A) to milliamperes (mA). The prefix milli- denotes a factor of \(10^{-3}\). Therefore, 1 ampere is equivalent to 1000 milliamperes. Mathematically, this can be expressed as:

\[
1 \text{ A} = 1000 \text{ mA}
\]

To convert 1.5 amperes to milliamperes, we multiply by the conversion factor:

\[
1.5 \text{ A} \times 1000 \text{ mA/A} = 1500 \text{ mA}
\]

Thus, 1.5 amperes is equal to 1500 milliamperes. This conversion is fundamental in electrical engineering, where current measurements often span multiple orders of magnitude.

% Diagram Prompt: A simple diagram showing the conversion from amperes to milliamperes with a scale or a conversion chart could be helpful here.