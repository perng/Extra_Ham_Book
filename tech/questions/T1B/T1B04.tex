\subsection{Which Amateur Band Includes 146.52 MHz?}
\label{T1B04}

\begin{tcolorbox}[colback=gray!10!white,colframe=black!75!black,title=T1B04]
Which amateur band includes 146.52 MHz?  
\begin{enumerate}[label=\Alph*)]
    \item 6 meters
    \item 20 meters
    \item 70 centimeters
    \item \textbf{2 meters}
\end{enumerate}
\end{tcolorbox}

\subsubsection{Explanation}
The frequency 146.52 MHz falls within the VHF (Very High Frequency) range, specifically in the 2-meter amateur radio band. The 2-meter band spans from 144 MHz to 148 MHz, as defined by the International Telecommunication Union (ITU). To determine which band includes 146.52 MHz, we can analyze the frequency ranges of the given options:

\begin{itemize}
    \item \textbf{6 meters}: 50-54 MHz
    \item \textbf{20 meters}: 14.0-14.35 MHz
    \item \textbf{70 centimeters}: 420-450 MHz
    \item \textbf{2 meters}: 144-148 MHz
\end{itemize}

The relationship between frequency and wavelength is given by the equation:

\[ \lambda = \frac{c}{f} \]

where:
\begin{itemize}
    \item \(\lambda\) is the wavelength in meters
    \item \(c\) is the speed of light (approximately $3 \cdot 10^8$ m/s)
    \item \(f\) is the frequency in Hz
\end{itemize}

For 146.52 MHz:

\[ \lambda = \frac{3 \times 10^8}{146.52 \times 10^6} \approx 2.047 \text{ meters} \]

This calculation shows why this frequency falls in the "2-meter" band, as its wavelength is approximately 2 meters.

Since 146.52 MHz lies within the 144-148 MHz range, it is part of the 2-meter band. This band is widely used for local and regional communication due to its balance between range and signal penetration.

% Diagram Prompt: Generate a diagram showing the frequency spectrum with the 2-meter band highlighted and 146.52 MHz marked within it.