\subsection{CW-Only VHF/UHF Band Segments}
\label{T1B07}

\begin{tcolorbox}[colback=gray!10!white,colframe=black!75!black,title=T1B07]
Which of the following VHF/UHF band segments are limited to CW only?
\begin{enumerate}[label=\Alph*),noitemsep]
    \item \textbf{50.0 MHz to 50.1 MHz and 144.0 MHz to 144.1 MHz}
    \item 219 MHz to 220 MHz and 420.0 MHz to 420.1 MHz
    \item 902.0 MHz to 902.1 MHz
    \item All these choices are correct
\end{enumerate}
\end{tcolorbox}

\subsubsection{Intuitive Explanation}
Imagine the radio spectrum as a big highway with different lanes for different types of traffic. Some lanes are reserved for specific types of vehicles, like motorcycles or trucks. In the same way, certain frequency ranges in the VHF/UHF bands are reserved exclusively for CW (Continuous Wave) transmissions, which are like the motorcycles of the radio world—simple and efficient. The question is asking which of these lanes (frequency ranges) are for CW only.

\subsubsection{Advanced Explanation}
In the VHF (Very High Frequency) and UHF (Ultra High Frequency) bands, specific segments are allocated exclusively for CW (Continuous Wave) transmissions. CW is a type of Morse code transmission that uses a single frequency carrier wave. The segments 50.0 MHz to 50.1 MHz and 144.0 MHz to 144.1 MHz are designated for CW only, ensuring that these frequencies are used for this specific mode of communication. This allocation helps in minimizing interference and maintaining clear communication channels for CW operators. The other options listed do not fall under the CW-only restrictions, making option A the correct answer.