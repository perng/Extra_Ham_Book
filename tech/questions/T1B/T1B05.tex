\subsection{Usage of 219 to 220 MHz Segment in 1.25 Meter Band}
\label{T1B05}

\begin{tcolorbox}[colback=gray!10!white,colframe=black!75!black,title=T1B05]
How may amateurs use the 219 to 220 MHz segment of 1.25 meter band?
\begin{enumerate}[label=\Alph*),noitemsep]
    \item Spread spectrum only
    \item Fast-scan television only
    \item Emergency traffic only
    \item \textbf{Fixed digital message forwarding systems only}
\end{enumerate}
\end{tcolorbox}

\subsubsection{Intuitive Explanation}
Imagine the 219 to 220 MHz frequency range as a special lane on a highway. Just like certain lanes are reserved for specific types of vehicles, this frequency segment is reserved for a specific type of communication. In this case, it's like a digital post office where messages are forwarded from one place to another, but only in a fixed manner. So, no spreading out, no TV broadcasts, and no emergency calls—just digital message forwarding.

\subsubsection{Advanced Explanation}
The 219 to 220 MHz segment of the 1.25 meter band is allocated for fixed digital message forwarding systems. This means that amateur radio operators can use this frequency range exclusively for the purpose of forwarding digital messages in a fixed manner. Spread spectrum, fast-scan television, and emergency traffic are not permitted in this segment. This allocation ensures that the frequency is used efficiently for its intended purpose, minimizing interference and maximizing utility for digital communication systems.