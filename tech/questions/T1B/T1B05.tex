\subsection{Usage of the 219 to 220 MHz Segment in the 1.25 Meter Band}
\label{T1B05}

\begin{tcolorbox}[colback=gray!10!white,colframe=black!75!black,title=T1B05]
How may amateurs use the 219 to 220 MHz segment of 1.25 meter band?
\begin{enumerate}[label=\Alph*)]
    \item Spread spectrum only
    \item Fast-scan television only
    \item Emergency traffic only
    \item \textbf{Fixed digital message forwarding systems only}
\end{enumerate}
\end{tcolorbox}

\subsubsection{Explanation}
The 219 to 220 MHz segment of the 1.25 meter band is allocated for specific uses under the regulations set by the Federal Communications Commission (FCC). This segment is designated for fixed digital message forwarding systems, which are automated systems that relay digital messages between stations. These systems are crucial for efficient and reliable communication in amateur radio networks. 

The FCC has restricted this segment to ensure that it is used exclusively for these purposes, thereby preventing interference from other types of communication such as spread spectrum, fast-scan television, or emergency traffic. This allocation helps maintain the integrity and efficiency of the communication systems operating within this frequency range.

% Diagram prompt: A diagram showing the frequency allocation of the 1.25 meter band with the 219 to 220 MHz segment highlighted and labeled for fixed digital message forwarding systems.