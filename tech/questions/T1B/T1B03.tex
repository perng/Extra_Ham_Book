\subsection{Which Frequency is in the 6 Meter Amateur Band?}

\begin{tcolorbox}[colback=gray!10!white,colframe=black!75!black]
    \textbf{T1B03} Which frequency is in the 6 meter amateur band?
    \begin{enumerate}[label=\Alph*)]
        \item 49.00 MHz
        \item 52.525 MHz
        \item 28.50 MHz
        \item \textbf{222.15 MHz}
    \end{enumerate}
\end{tcolorbox}

\subsubsection{Intuitive Explanation}
Imagine you're tuning your radio to find a station, and you're looking for one that's on the 6-meter band. This band is like a special lane on a highway just for amateur radio operators. The question is asking which of the given frequencies is in this special lane. The correct answer is 52.525 MHz, which is like the exact address of a house in that lane. The other frequencies are either in different lanes or not even on the highway!

\subsubsection{Advanced Explanation}
The 6-meter amateur band is a portion of the radio spectrum allocated for amateur radio use. It spans from 50 MHz to 54 MHz. To determine which frequency falls within this band, we need to check if the given frequency lies within this range.

Therefore, the correct answer is \textbf{B: 52.525 MHz}.

The 6-meter band is particularly interesting because it can support both local and long-distance communication, depending on atmospheric conditions. It is part of the Very High Frequency (VHF) spectrum, which is known for its ability to propagate signals over relatively short distances with high clarity.

% Prompt for generating a diagram: A frequency spectrum diagram showing the 6-meter band (50 MHz to 54 MHz) and marking the position of 52.525 MHz within it.