\subsection{Frequency Ranges for Phone Operation by Technician Licensees}

\begin{tcolorbox}[colback=gray!10!white,colframe=black!75!black,title=T1B01]
Which of the following frequency ranges are available for phone operation by Technician licensees?
\begin{enumerate}[label=\Alph*)]
    \item 28.050 MHz to 28.150 MHz
    \item 28.100 MHz to 28.300 MHz
    \item \textbf{28.300 MHz to 28.500 MHz}
    \item 28.500 MHz to 28.600 MHz
\end{enumerate}
\end{tcolorbox}

\subsubsection{Intuitive Explanation}
Imagine you’re a Technician licensee, and you’re given a walkie-talkie that can only talk on certain channels. The question is asking which of these channels (frequency ranges) you’re allowed to use for making phone calls. Think of it like being given a specific set of keys to unlock certain doors. The correct key here is the frequency range from 28.300 MHz to 28.500 MHz. So, if you’re in this range, you’re good to go!

\subsubsection{Advanced Explanation}
Technician licensees in the United States are granted privileges on specific frequency bands within the radio spectrum. The 10-meter band, which spans from 28.000 MHz to 29.700 MHz, is one of these bands. However, not all frequencies within this band are available for phone (voice) operation. 

The Federal Communications Commission (FCC) allocates specific segments of the 10-meter band for different modes of communication. For phone operation, Technician licensees are permitted to use the frequency range from 28.300 MHz to 28.500 MHz. This segment is designated for voice communication, typically using Single Sideband (SSB) modulation.

To understand why this specific range is allocated, consider the following:
\begin{itemize}
    \item \textbf{Bandwidth Requirements}: Voice communication requires a certain bandwidth to transmit audio signals effectively. The 28.300 MHz to 28.500 MHz range provides sufficient bandwidth for clear voice transmission.
    \item \textbf{Interference Management}: By restricting phone operation to this range, the FCC minimizes interference with other modes of communication, such as digital or CW (Morse code) signals, which may operate on adjacent frequencies.
    \item \textbf{Regulatory Compliance}: The allocation ensures that Technician licensees operate within the legal limits of their license, avoiding penalties and ensuring efficient use of the radio spectrum.
\end{itemize}

Therefore, the correct answer is \textbf{C: 28.300 MHz to 28.500 MHz}.

% Prompt for generating a diagram: A frequency spectrum diagram showing the 10-meter band with highlighted segments for phone operation, digital modes, and CW.