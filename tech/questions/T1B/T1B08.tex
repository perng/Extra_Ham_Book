\subsection{US Amateurs Restrictions in Secondary Band Segments}
\label{T1B08}

\begin{tcolorbox}[colback=gray!10!white,colframe=black!75!black,title=T1B08]
How are US amateurs restricted in segments of bands where the Amateur Radio Service is secondary?
\begin{enumerate}[label=\Alph*)]
    \item \textbf{U.S. amateurs may find non-amateur stations in those segments, and must avoid interfering with them}
    \item U.S. amateurs must give foreign amateur stations priority in those segments
    \item International communications are not permitted in those segments
    \item Digital transmissions are not permitted in those segments
\end{enumerate}
\end{tcolorbox}

\subsubsection{Intuitive Explanation}
Imagine you're at a playground, and there's a big sandbox where everyone can play. But sometimes, other kids who are not part of your group also want to play in the same sandbox. In this case, you need to be careful and not disturb them while you're playing. Similarly, in certain radio frequency bands, amateur radio operators (like you) share the space with other non-amateur stations. The rule is simple: you can use the space, but you must make sure you don't interfere with the other stations. It's like being a good neighbor in the radio world!

% Prompt for generating a diagram: A diagram showing the frequency band allocation with primary and secondary users, highlighting the segments where the Amateur Radio Service is secondary.