\subsection{SSB Phone Be Above 50 MHz?}
\label{T1B10}

\begin{tcolorbox}[colback=gray!10!white,colframe=black!75!black,title=T1B10]
Where may SSB phone be used in amateur bands above 50 MHz?
\begin{enumerate}[label=\Alph*)]
    \item Only in sub-bands allocated to General class or higher licensees
    \item Only on repeaters
    \item \textbf{In at least some segment of all these bands}
    \item On any band if the power is limited to 25 watts
\end{enumerate}
\end{tcolorbox}

\subsubsection{Intuitive Explanation}
Imagine you have a walkie-talkie, but it's a super fancy one that can talk on different channels. Now, think of these channels as different bands like different radio stations. The question is asking where you can use this fancy walkie-talkie (SSB phone) on channels above 50 MHz. The answer is that you can use it on at least some part of all these channels, not just specific ones or only on repeaters (which are like radio boosters). So, it's like saying you can tune into any station, but maybe not every single song on that station.

\subsubsection{Advanced Explanation}
Single Sideband (SSB) phone is a mode of communication used in amateur radio that is efficient in terms of bandwidth and power. The question pertains to the usage of SSB phone in amateur bands above 50 MHz. According to the FCC regulations, SSB phone can be used in at least some segment of all amateur bands above 50 MHz. This means that while not every frequency within these bands may be allocated for SSB phone, there are segments within each band where its use is permitted. This is different from options A and B, which restrict usage to specific sub-bands or repeaters, and option D, which incorrectly implies that power limitation alone determines usage across any band.

% Prompt for generating a diagram:
% A diagram showing the frequency spectrum of amateur bands above 50 MHz, highlighting the segments where SSB phone is permitted, would be beneficial for visual understanding.