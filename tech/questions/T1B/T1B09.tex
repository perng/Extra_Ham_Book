\subsection{Transmit Frequency Placement}
\label{T1B09}

\begin{tcolorbox}[colback=gray!10!white,colframe=black!75!black,title=T1B09]
Why should you not set your transmit frequency to be exactly at the edge of an amateur band or sub-band?
\begin{enumerate}[label=\Alph*),noitemsep]
    \item To allow for calibration error in the transmitter frequency display
    \item So that modulation sidebands do not extend beyond the band edge
    \item To allow for transmitter frequency drift
    \item \textbf{All these choices are correct}
\end{enumerate}
\end{tcolorbox}

\subsubsection*{Intuitive Explanation}
Imagine you're trying to park your car right at the edge of a parking spot. If you're even a tiny bit off, you might end up in the next spot or even outside the parking area. Similarly, setting your transmit frequency exactly at the edge of a band is risky because small errors or changes can push your signal outside the allowed range, causing interference or violating regulations.

\subsubsection*{Advanced Explanation}
When transmitting, several factors can affect the exact frequency of your signal:
\begin{itemize}
    \item \textbf{Calibration Error}: The frequency display on your transmitter might not be perfectly accurate. Even a small error can push your signal out of the allowed band.
    \item \textbf{Modulation Sidebands}: When you modulate your signal (e.g., with voice or data), sidebands are created. These sidebands extend beyond your carrier frequency, and if your carrier is at the edge, the sidebands can spill over into adjacent bands.
    \item \textbf{Frequency Drift}: Transmitters can experience frequency drift due to temperature changes or component aging. This drift can move your signal out of the allowed band if it's initially set too close to the edge.
\end{itemize}
Therefore, it's prudent to set your transmit frequency slightly inside the band edge to account for these potential issues.