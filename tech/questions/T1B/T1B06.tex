\subsection{Technician Class HF Phone Privileges}
\label{T1B06}

\begin{tcolorbox}[colback=gray!10!white,colframe=black!75!black,title=T1B06]
On which HF bands does a Technician class operator have phone privileges?
\begin{enumerate}[label=\Alph*),noitemsep]
    \item None
    \item \textbf{10 meter band only}
    \item 80 meter, 40 meter, 15 meter, and 10 meter bands
    \item 30 meter band only
\end{enumerate}
\end{tcolorbox}

\subsubsection*{Intuitive Explanation}
Think of the HF bands as different channels on a radio. A Technician class operator is like a new driver who can only drive on certain roads. In this case, the road they can use for phone (voice) communication is the 10 meter band. Other bands are either off-limits or reserved for different types of communication.

\subsubsection*{Advanced Explanation}
The HF (High Frequency) bands range from 3 to 30 MHz and are divided into several segments, each with specific allocations for different modes of communication. Technician class operators in the United States are granted limited privileges on the HF bands. Specifically, they are allowed to use phone (voice) communication on the 10 meter band (28.000 - 29.700 MHz). This band is particularly useful for local and long-distance communication, especially during periods of good propagation. Other HF bands, such as 80 meters, 40 meters, and 30 meters, are either not allocated for phone use by Technician class operators or are reserved for other modes like CW (Continuous Wave) or digital communication.