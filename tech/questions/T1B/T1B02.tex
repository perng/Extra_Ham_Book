\subsection{Contacting the International Space Station (ISS) on VHF Bands}
\subsubsection{Question T1B02}
\begin{tcolorbox}[colback=gray!10!white,colframe=black!75!black,title=T1B02]
Which amateurs may contact the International Space Station (ISS) on VHF bands?
\begin{enumerate}[label=\Alph*]
    \item Any amateur holding a General class or higher license
    \item \textbf{Any amateur holding a Technician class or higher license}
    \item Any amateur holding a General class or higher license who has applied for and received approval from NASA
    \item Any amateur holding a Technician class or higher license who has applied for and received approval from NASA
\end{enumerate}
\end{tcolorbox}

\subsubsection{Intuitive Explanation}
Imagine the International Space Station (ISS) as a super cool club in space. To talk to the astronauts in this club, you need a special pass. In the world of amateur radio, this pass is called a license. The good news is, you don’t need to be a super expert to get this pass. If you have a Technician class license or higher, you’re in! You don’t need to ask NASA for permission either. So, grab your radio, aim it at the ISS, and start chatting with the astronauts!

\subsubsection{Advanced Explanation}
The International Space Station (ISS) operates on VHF (Very High Frequency) bands, which are accessible to amateur radio operators. The Federal Communications Commission (FCC) in the United States regulates amateur radio licenses. The Technician class license is the entry-level license, and it grants privileges on VHF bands, including the frequencies used by the ISS.

To communicate with the ISS, an amateur radio operator must have at least a Technician class license. This license allows the operator to transmit on the 2-meter band (144-148 MHz), which is one of the frequencies used by the ISS for amateur radio communications. There is no additional requirement to seek approval from NASA for making contact with the ISS, as long as the operator adheres to the FCC regulations and the guidelines provided by the Amateur Radio on the International Space Station (ARISS) program.

In summary, the correct answer is that any amateur holding a Technician class or higher license may contact the ISS on VHF bands. This is because the Technician class license provides the necessary privileges for VHF communication, and no additional approval from NASA is required.

% Diagram prompt: Generate a diagram showing the hierarchy of amateur radio licenses and the privileges associated with each class, highlighting the VHF band access for Technician class and above.