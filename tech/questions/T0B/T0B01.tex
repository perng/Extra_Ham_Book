\subsection{Good Practices for Ground Wire Installation on Towers}
\label{T0B01}

\begin{tcolorbox}[colback=gray!10!white,colframe=black!75!black,title=T0B01]
Which of the following is good practice when installing ground wires on a tower for lightning protection?
\begin{enumerate}[noitemsep]
    \item Put a drip loop in the ground connection to prevent water damage to the ground system
    \item Make sure all ground wire bends are right angles
    \item \textbf{Ensure that connections are short and direct}
    \item All these choices are correct
\end{enumerate}
\end{tcolorbox}

\subsubsection*{Explanation}
When installing ground wires on a tower for lightning protection, it is crucial to ensure that the connections are short and direct. This minimizes the resistance and inductance in the grounding path, which is essential for effective lightning protection. While drip loops and right-angle bends might seem like good ideas, they can introduce unnecessary complications and potential points of failure. Therefore, the best practice is to keep the connections as straightforward as possible.