\subsection{Safety Rules for Crank-Up Towers}
\label{T0B07}

\begin{tcolorbox}[colback=gray!10!white,colframe=black!75!black,title=T0B07]
Which of the following is an important safety rule to remember when using a crank-up tower?
\begin{enumerate}[label=\Alph*]
    \item This type of tower must never be painted
    \item This type of tower must never be grounded
    \item \textbf{This type of tower must not be climbed unless it is retracted, or mechanical safety locking devices have been installed}
    \item All these choices are correct
\end{enumerate}
\end{tcolorbox}

\subsubsection{Intuitive Explanation}
Imagine you have a toy tower that can go up and down like a jack-in-the-box. Now, if you try to climb it while it's going up or down, you might get squished or fall off! That’s why you should only climb it when it’s all the way down or when it has special locks to keep it from moving. Safety first, always!

\subsubsection{Advanced Explanation}
Crank-up towers are designed to be extended and retracted using a mechanical system, often involving a crank or motor. When the tower is in the process of being extended or retracted, the mechanical components are under tension, and the structure may not be stable. Climbing the tower during this time poses a significant risk of structural failure or personal injury. 

Mechanical safety locking devices are installed to ensure that once the tower is fully extended, it remains securely in place, preventing any accidental retraction. These locks are crucial for maintaining the tower's stability and ensuring the safety of anyone who needs to climb it. Therefore, it is imperative to only climb the tower when it is fully retracted or when these safety devices are engaged.

% Diagram prompt: Generate a diagram showing a crank-up tower in both retracted and extended positions, highlighting the mechanical safety locking devices.