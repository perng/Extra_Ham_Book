\subsection{Grounding Conductors for Lightning Protection}
\label{T0B10}

\begin{tcolorbox}[colback=gray!10!white,colframe=black!75!black,title=T0B10]
Which of the following is true when installing grounding conductors used for lightning protection?
\begin{enumerate}[label=\Alph*]
    \item Use only non-insulated wire
    \item Wires must be carefully routed with precise right-angle bends
    \item \textbf{Sharp bends must be avoided}
    \item Common grounds must be avoided
\end{enumerate}
\end{tcolorbox}

\subsubsection{Intuitive Explanation}
Imagine you're setting up a lightning rod to protect your treehouse. You want the lightning to travel smoothly down the wire and into the ground, right? If you make sharp bends in the wire, it's like putting speed bumps on a highway—it slows things down and can cause problems. So, avoid sharp bends to keep the lightning's path clear and safe!

\subsubsection{Advanced Explanation}
When installing grounding conductors for lightning protection, it is crucial to ensure that the path for the electrical discharge is as direct and unobstructed as possible. Sharp bends in the conductor can create points of high electrical resistance, which can lead to localized heating and potential failure of the conductor during a lightning strike. 

The correct approach is to use smooth, gradual bends to minimize resistance and ensure efficient dissipation of the electrical energy into the ground. This principle is supported by the laws of electromagnetism, particularly Ohm's Law, which states that the current \( I \) through a conductor is directly proportional to the voltage \( V \) and inversely proportional to the resistance \( R \):

\[
I = \frac{V}{R}
\]

By avoiding sharp bends, we reduce the resistance \( R \), allowing the current to flow more freely and safely.

Additionally, using insulated wire is not necessary for grounding conductors, as the primary goal is to provide a low-resistance path to the ground. However, the insulation can help protect the conductor from environmental factors. Common grounds are generally acceptable and often necessary to ensure a unified grounding system.

% Diagram prompt: Generate a diagram showing a grounding conductor with smooth bends versus sharp bends, illustrating the difference in electrical resistance.