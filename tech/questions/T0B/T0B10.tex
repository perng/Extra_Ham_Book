\subsection{Grounding Conductors for Lightning Protection}
\label{T0B10}

\begin{tcolorbox}[colback=gray!10!white,colframe=black!75!black,title=T0B10]
Which of the following is true when installing grounding conductors used for lightning protection?
\begin{enumerate}[noitemsep]
    \item Use only non-insulated wire
    \item Wires must be carefully routed with precise right-angle bends
    \item \textbf{Sharp bends must be avoided}
    \item Common grounds must be avoided
\end{enumerate}
\end{tcolorbox}

\subsubsection*{Intuitive Explanation}
When installing grounding conductors for lightning protection, think of it like setting up a path for lightning to follow safely to the ground. You wouldn't want the path to have sharp turns because that could cause problems, just like how a sharp turn in a water pipe could cause a blockage or a burst. So, avoiding sharp bends ensures that the lightning can travel smoothly and safely to the ground.

\subsubsection*{Advanced Explanation}
Grounding conductors are crucial for safely dissipating the energy from a lightning strike into the earth. Sharp bends in the conductor can create points of high electrical stress, which can lead to arcing or even failure of the conductor. This is because the electric field is more concentrated at sharp bends, increasing the risk of breakdown. Therefore, it is essential to avoid sharp bends to maintain the integrity and effectiveness of the grounding system. Additionally, using insulated wire, precise right-angle bends, or avoiding common grounds are not standard practices and can compromise the safety and efficiency of the grounding system.