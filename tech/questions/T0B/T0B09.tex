\subsection{Antenna Attachment to Utility Poles}
\label{T0B09}

\begin{tcolorbox}[colback=gray!10!white,colframe=black!75!black,title=T0B09]
Why should you avoid attaching an antenna to a utility pole?
\begin{enumerate}[label=\Alph*)]
    \item The antenna will not work properly because of induced voltages
    \item The 60 Hz radiations from the feed line may increase the SWR
    \item \textbf{The antenna could contact high-voltage power lines}
    \item All these choices are correct
\end{enumerate}
\end{tcolorbox}

\subsubsection{Intuitive Explanation}
Imagine you’re trying to fly a kite, but instead of an open field, you’re in a place with lots of power lines. If your kite string touches one of those lines, ZAP! That’s not a fun day. Attaching an antenna to a utility pole is like flying a kite near power lines—it’s risky because the antenna might touch the high-voltage wires, and that could be very dangerous. So, it’s best to keep your antenna far away from utility poles to avoid any shocking surprises!

\subsubsection{Advanced Explanation}
Utility poles often carry high-voltage power lines, which can have voltages ranging from hundreds to thousands of volts. When an antenna is attached to a utility pole, there is a significant risk that the antenna or its supporting structure could come into contact with these high-voltage lines. This contact can lead to catastrophic consequences, including electrical arcing, equipment damage, and severe injury or death to anyone nearby.

The primary concern is the potential for the antenna to act as a conductor, creating a path for the high voltage to travel. This can result in a short circuit, which can cause fires or other hazardous conditions. Additionally, the high voltage can induce dangerous currents in the antenna and its feed line, posing a risk to both the equipment and the operator.

To avoid these risks, it is crucial to maintain a safe distance from utility poles and high-voltage power lines when installing antennas. The National Electrical Safety Code (NESC) provides guidelines for minimum clearance distances to ensure safety. For example, the NESC recommends a minimum horizontal clearance of 10 feet for voltages up to 50 kV and greater distances for higher voltages.

In summary, attaching an antenna to a utility pole is highly discouraged due to the risk of contact with high-voltage power lines, which can lead to severe safety hazards. Always follow safety guidelines and maintain proper clearance distances when installing antennas.

% Diagram prompt: Generate a diagram showing a utility pole with high-voltage power lines and an antenna being installed too close, with a warning sign indicating the danger of electrical contact.