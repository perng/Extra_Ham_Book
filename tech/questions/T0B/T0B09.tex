\subsection{Antenna Attachment to Utility Poles}
\label{T0B09}

\begin{tcolorbox}[colback=gray!10!white,colframe=black!75!black,title=T0B09]
Why should you avoid attaching an antenna to a utility pole?
\begin{enumerate}[noitemsep]
    \item The antenna will not work properly because of induced voltages
    \item The 60 Hz radiations from the feed line may increase the SWR
    \item \textbf{The antenna could contact high-voltage power lines}
    \item All these choices are correct
\end{enumerate}
\end{tcolorbox}

\subsubsection*{Intuitive Explanation}
Attaching an antenna to a utility pole might seem like a quick way to get it up high, but it's a bit like playing with fire—literally! Utility poles carry high-voltage power lines, and if your antenna touches one, it could lead to a dangerous situation. Think of it as trying to balance a metal rod near a live wire; not the best idea, right?

\subsubsection*{Advanced Explanation}
Utility poles are designed to support electrical power lines, which often carry high voltages. When you attach an antenna to such a pole, there is a significant risk that the antenna or its supporting structure could come into contact with these high-voltage lines. This contact can result in electrical arcing, which is extremely hazardous and can cause severe injury or even death. Additionally, the proximity to high-voltage lines can induce unwanted currents in the antenna, potentially damaging your equipment. Therefore, it is crucial to maintain a safe distance from utility poles when installing antennas to avoid these risks.