\subsection{Proper Grounding Method for a Tower}
\label{T0B08}

\begin{tcolorbox}[colback=gray!10!white,colframe=black!75!black,title=T0B08]
Which is a proper grounding method for a tower?
\begin{enumerate}[label=\Alph*]
    \item A single four-foot ground rod, driven into the ground no more than 12 inches from the base
    \item A ferrite-core RF choke connected between the tower and ground
    \item A connection between the tower base and a cold water pipe
    \item \textbf{Separate eight-foot ground rods for each tower leg, bonded to the tower and each other}
\end{enumerate}
\end{tcolorbox}

\subsubsection{Intuitive Explanation}
Imagine your tower is like a giant lightning rod. If lightning strikes, you want it to safely go into the ground without causing any damage. Think of grounding as giving the lightning a safe path to follow. Using a single small rod (like in option A) is like trying to catch a waterfall with a teacup—it’s just not enough! Option B is like putting a tiny filter in the path, which doesn’t help much. Option C is like connecting the tower to a pipe that might not always be reliable. But option D is like giving each leg of the tower its own sturdy path to the ground, and they all work together to keep everything safe. It’s like having multiple fire exits instead of just one!

\subsubsection{Advanced Explanation}
Proper grounding for a tower involves ensuring a low-resistance path to the earth to safely dissipate electrical energy, such as from a lightning strike. The key factors are the depth and number of ground rods, as well as their bonding.

\begin{itemize}
    \item \textbf{Depth of Ground Rods}: The National Electrical Code (NEC) recommends ground rods to be at least 8 feet deep to ensure good contact with the earth. A four-foot rod (Option A) does not provide sufficient grounding.
    \item \textbf{Number of Ground Rods}: A single ground rod (Option A) may not provide enough surface area for effective grounding. Multiple rods (Option D) increase the contact area with the earth, reducing the overall resistance.
    \item \textbf{Bonding}: Proper bonding between the ground rods and the tower ensures that all parts of the system are at the same potential, reducing the risk of electrical arcing. Option D specifies bonding between the rods and the tower, which is crucial for safety.
    \item \textbf{Ferrite-Core RF Choke}: This is used to block high-frequency signals, not for grounding (Option B).
    \item \textbf{Cold Water Pipe}: While sometimes used as a ground, it is not as reliable as dedicated ground rods (Option C).
\end{itemize}

Thus, the correct method is to use separate eight-foot ground rods for each tower leg, bonded to the tower and each other (Option D).

% Diagram Prompt: Generate a diagram showing a tower with multiple ground rods, each connected to a leg of the tower and bonded together.