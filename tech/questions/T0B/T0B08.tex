\subsection{Proper Grounding Method for a Tower}
\label{T0B08}

\begin{tcolorbox}[colback=gray!10!white,colframe=black!75!black,title=T0B08]
Which is a proper grounding method for a tower?
\begin{enumerate}[noitemsep]
    \item A single four-foot ground rod, driven into the ground no more than 12 inches from the base
    \item A ferrite-core RF choke connected between the tower and ground
    \item A connection between the tower base and a cold water pipe
    \item \textbf{Separate eight-foot ground rods for each tower leg, bonded to the tower and each other}
\end{enumerate}
\end{tcolorbox}

\subsubsection*{Intuitive Explanation}
Imagine your tower is like a giant lightning rod. If it's not properly grounded, it could attract lightning and cause damage. The best way to protect it is to have multiple ground rods, one for each leg of the tower, and make sure they're all connected. This way, if lightning strikes, the electricity has multiple paths to safely dissipate into the ground.

\subsubsection*{Advanced Explanation}
Proper grounding for a tower involves ensuring that each leg of the tower has its own ground rod, typically eight feet long, which is driven into the ground. These ground rods should be bonded to the tower and to each other to create a low-resistance path for electrical currents, such as those from lightning strikes, to safely dissipate into the earth. This method reduces the risk of electrical damage to the tower and connected equipment. Using a single ground rod or relying on a cold water pipe is insufficient because it does not provide the same level of protection. A ferrite-core RF choke is not a grounding method but rather a device used to suppress high-frequency interference.