\subsection{Understanding Simplex Communication in Amateur Radio}
\label{T2A11}

\begin{tcolorbox}[colback=gray!10!white,colframe=black!75!black,title=T2A11]
What term describes an amateur station that is transmitting and receiving on the same frequency?
\begin{enumerate}[label=\Alph*)]
    \item Full duplex
    \item Diplex
    \item \textbf{Simplex}
    \item Multiplex
\end{enumerate}
\end{tcolorbox}

\subsubsection{Intuitive Explanation}
Imagine you and your friend are talking on walkie-talkies. If you both can only talk one at a time, and you’re using the same channel, that’s like simplex communication. It’s like saying, “Over to you!” after you finish speaking, so your friend knows it’s their turn to talk. Simplex is simple because it uses just one frequency for both sending and receiving, just like a walkie-talkie!

\subsubsection{Advanced Explanation}
Simplex communication refers to a mode of transmission where both the transmitter and receiver operate on the same frequency. This is in contrast to duplex communication, where separate frequencies are used for transmitting and receiving simultaneously. In simplex mode, communication is unidirectional at any given time, meaning only one party can transmit while the other receives. This is mathematically represented as:

\[
f_{\text{transmit}} = f_{\text{receive}}
\]

where \( f_{\text{transmit}} \) is the transmitting frequency and \( f_{\text{receive}} \) is the receiving frequency. Simplex is commonly used in amateur radio for its simplicity and efficiency, especially in scenarios where simultaneous two-way communication is not required.

% Diagram prompt: Generate a diagram showing two radios communicating on the same frequency in simplex mode.