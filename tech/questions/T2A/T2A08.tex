\subsection{Meaning of the Procedural Signal “CQ”}
\label{T2A08}

\begin{tcolorbox}[colback=gray!10!white,colframe=black!75!black,title=T2A08]
What is the meaning of the procedural signal “CQ”?  
\begin{enumerate}[noitemsep]
    \item Call on the quarter hour
    \item Test transmission, no reply expected
    \item Only the called station should transmit
    \item \textbf{Calling any station}
\end{enumerate}
\end{tcolorbox}

The procedural signal “CQ” is commonly used in amateur radio to indicate a general call to any station that may be listening. It is not specific to any particular station and is used to initiate contact with any operator who is available to respond. The correct answer is \textbf{D}, which means “Calling any station.”