\subsection{Responding to a Station Calling CQ}
\label{T2A05}

\begin{tcolorbox}[colback=gray!10!white,colframe=black!75!black,title=T2A05]
How should you respond to a station calling CQ?
\begin{enumerate}[label=\Alph*]
    \item Transmit CQ followed by the other station’s call sign
    \item Transmit your call sign followed by the other station’s call sign
    \item \textbf{Transmit the other station’s call sign followed by your call sign}
    \item Transmit a signal report followed by your call sign
\end{enumerate}
\end{tcolorbox}

\subsubsection{Intuitive Explanation}
Imagine you're at a party, and someone shouts, Hey, anyone want to chat? You wouldn't just yell back, Hey, anyone want to chat? That would be weird, right? Instead, you'd say, Hey, I want to chat! In the same way, when a radio station calls CQ (which means Calling any station), you respond by saying their name (call sign) first, followed by your name (call sign). This way, they know you're talking to them and who you are. So, the correct answer is to say their call sign first, then yours.

\subsubsection{Advanced Explanation}
In radio communication, CQ is a general call to any station that may be listening. When you hear a station calling CQ, it means they are seeking a contact. The proper protocol for responding to a CQ call is to first acknowledge the calling station by transmitting their call sign, followed by your own call sign. This ensures clarity and avoids confusion, as it directly addresses the station that initiated the call and identifies yourself as the responder. 

For example, if Station A calls CQ and you are Station B, you would respond with Station A, this is Station B. This format is standardized in amateur radio communication to maintain order and efficiency in establishing contacts. 

% Diagram prompt: A simple flowchart showing the sequence of call signs when responding to a CQ call.