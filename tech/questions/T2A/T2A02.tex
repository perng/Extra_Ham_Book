\subsection{National Calling Frequency for FM Simplex Operations in the 2 Meter Band}
\label{T2A02}

\begin{tcolorbox}[colback=gray!10!white,colframe=black!75!black,title=T2A02]
What is the national calling frequency for FM simplex operations in the 2 meter band?
\begin{enumerate}[label=\Alph*)]
    \item \textbf{146.520 MHz}
    \item 145.000 MHz
    \item 432.100 MHz
    \item 446.000 MHz
\end{enumerate}
\end{tcolorbox}

\subsubsection{Intuitive Explanation}
Imagine you're at a big party, and everyone is talking at the same time. How do you find your friend? You might agree to meet at a specific spot, like the snack table. In the world of radio, the 2 meter band is like the party, and the national calling frequency (146.520 MHz) is the snack table. It's the agreed-upon spot where people can start a conversation before moving to a different frequency to chat more privately. So, if you want to talk to someone on the 2 meter band, you start by tuning into 146.520 MHz and saying, Hey, anyone out there?

\subsubsection{Advanced Explanation}
The 2 meter band refers to the VHF (Very High Frequency) range of radio frequencies from 144 MHz to 148 MHz. FM simplex operations mean that communication occurs on a single frequency, without the need for a repeater. The national calling frequency for FM simplex operations in the 2 meter band is standardized at 146.520 MHz. This frequency is designated as the initial contact point for amateur radio operators to establish communication before moving to another frequency for further conversation.

The choice of 146.520 MHz is based on its central location within the 2 meter band, making it easily accessible and minimizing interference from other services. This frequency is widely recognized and used across the United States, ensuring that operators can reliably make initial contact with others.

% Diagram Prompt: Generate a diagram showing the 2 meter band frequency range with 146.520 MHz highlighted as the national calling frequency.