\subsection{Common Repeater Frequency Offset in the 70 cm Band}
\label{T2A03}

\begin{tcolorbox}[colback=gray!10!white,colframe=black!75!black,title=T2A03]
What is a common repeater frequency offset in the 70 cm band?
\begin{enumerate}[noitemsep]
    \item \textbf{Plus or minus 5 MHz}
    \item Plus or minus 600 kHz
    \item Plus or minus 500 kHz
    \item Plus or minus 1 MHz
\end{enumerate}
\end{tcolorbox}

In the 70 cm band, which is part of the UHF spectrum, repeaters often use a frequency offset to separate the transmit and receive frequencies. This offset helps to avoid interference and allows for simultaneous transmission and reception. The most common offset in this band is \textbf{plus or minus 5 MHz}. This means that if a repeater is receiving on a certain frequency, it will transmit on a frequency that is 5 MHz higher or lower than the receive frequency.