\subsection{Band Plan Beyond FCC Privileges}
\label{T2A10}

\begin{tcolorbox}[colback=gray!10!white,colframe=black!75!black,title=T2A10]
What is a band plan, beyond the privileges established by the FCC?
\begin{enumerate}[noitemsep]
    \item \textbf{A voluntary guideline for using different modes or activities within an amateur band}
    \item A list of operating schedules
    \item A list of available net frequencies
    \item A plan devised by a club to indicate frequency band usage
\end{enumerate}
\end{tcolorbox}

\subsubsection*{Intuitive Explanation}
Think of a band plan as a set of gentleman's agreements among amateur radio operators. While the FCC sets the rules for what frequencies you can use, the band plan helps everyone play nicely together by suggesting which parts of the band are best for certain activities, like voice communication, digital modes, or Morse code. It's like having a playground where everyone agrees to use different areas for different games.

\subsubsection*{Advanced Explanation}
A band plan is a detailed guideline that organizes the use of the amateur radio spectrum beyond the basic regulatory framework provided by the FCC. It is developed by the amateur radio community to optimize the use of available frequencies and minimize interference between different modes of communication. For example, within a specific band, certain segments might be designated for voice communication (SSB, FM), while others are reserved for digital modes (PSK31, FT8) or CW (Morse code). These plans are not legally binding but are widely adopted to ensure efficient and harmonious use of the spectrum.