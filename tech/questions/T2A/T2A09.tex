\subsection{Identifying a Station Listening on a Repeater}
\label{T2A09}

\begin{tcolorbox}[colback=gray!10!white,colframe=black!75!black,title=T2A09]
Which of the following indicates that a station is listening on a repeater and looking for a contact?
\begin{enumerate}[label=\Alph*]
    \item “CQ CQ” followed by the repeater’s call sign
    \item \textbf{The station’s call sign followed by the word “monitoring”}
    \item The repeater call sign followed by the station’s call sign
    \item “QSY” followed by your call sign
\end{enumerate}
\end{tcolorbox}

\subsubsection{Intuitive Explanation}
Imagine you're at a party, and you're standing by the snack table, just listening to the conversations around you. You're not actively talking, but you're ready to jump in if someone says something interesting. Now, if someone wants to talk to you, they might say your name followed by Hey, are you listening? That's like a radio operator saying their call sign followed by monitoring. It’s their way of saying, I’m here, and I’m ready to chat if you want to talk!

\subsubsection{Advanced Explanation}
In radio communication, particularly when using repeaters, it's essential to understand the protocols for indicating availability for contact. A repeater is a device that receives a signal and retransmits it at a higher power or from a better location, extending the range of communication. When a station is listening on a repeater and is open to making contact, they will typically transmit their call sign followed by the word monitoring. This indicates that they are actively listening on the repeater frequency and are available for communication.

The correct answer, \textbf{B}, reflects this standard practice. The other options are incorrect because:
\begin{itemize}
    \item \textbf{A}: CQ CQ is a general call for any station to respond, not specific to repeaters.
    \item \textbf{C}: This format is not standard for indicating availability on a repeater.
    \item \textbf{D}: QSY is a Q-code meaning change frequency, which is unrelated to indicating availability for contact.
\end{itemize}

Understanding these protocols ensures efficient and clear communication in amateur radio operations.

% Prompt for diagram: A diagram showing a radio operator listening on a repeater and transmitting their call sign followed by monitoring could help visualize the concept.