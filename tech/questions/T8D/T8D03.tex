\subsection{APRS Data Transmission}
\label{T8D03}

APRS (Automatic Packet Reporting System) is a versatile communication protocol used in amateur radio to transmit various types of data. Below is a question that explores the kinds of data that can be transmitted using APRS.

\begin{tcolorbox}[colback=gray!10!white,colframe=black!75!black,title=T8D03]
What kind of data can be transmitted by APRS?
\begin{enumerate}[noitemsep]
    \item GPS position data
    \item Text messages
    \item Weather data
    \item \textbf{All these choices are correct}
\end{enumerate}
\end{tcolorbox}

APRS is designed to handle multiple types of data, including GPS position data, text messages, and weather data. This makes it a powerful tool for amateur radio operators to share information in real-time. The correct answer is \textbf{D}, as all the listed data types can be transmitted using APRS.