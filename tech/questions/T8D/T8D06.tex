\subsection{Meaning of PSK}\label{T8D06}

\begin{tcolorbox}[colback=gray!10!white,colframe=black!75!black,title=T8D06]
What does the abbreviation PSK mean?
\begin{enumerate}[label=\Alph*)]
    \item Pulse Shift Keying
    \item \textbf{Phase Shift Keying}
    \item Packet Short Keying
    \item Phased Slide Keying
\end{enumerate}
\end{tcolorbox}

\subsubsection{Intuitive Explanation}
Imagine you're sending secret messages using a flashlight. Instead of turning the light on and off (which is like Morse code), you decide to twist the flashlight slightly to change the angle of the light beam. This twisting is like changing the phase of the light. In radio terms, PSK stands for Phase Shift Keying, which is a fancy way of saying we change the angle (or phase) of the radio wave to send information. It's like twisting the flashlight to send different messages!

\subsubsection{Advanced Explanation}
Phase Shift Keying (PSK) is a digital modulation technique used in communication systems to transmit data by varying the phase of the carrier wave. The phase of the wave is altered to represent different binary states (e.g., 0 and 1). For example, in Binary Phase Shift Keying (BPSK), a phase shift of 0 degrees represents a binary 0, and a phase shift of 180 degrees represents a binary 1. Mathematically, the modulated signal can be represented as:

\[
s(t) = A \cos(2\pi f_c t + \phi)
\]

where \( A \) is the amplitude, \( f_c \) is the carrier frequency, and \( \phi \) is the phase shift. PSK is widely used in various communication systems due to its efficiency and robustness against noise.

% Diagram Prompt: Generate a diagram showing a carrier wave with different phase shifts representing binary data.