\subsection{DMR Technique Description}
\label{T8D07}

\begin{tcolorbox}[colback=gray!10!white,colframe=black!75!black,title=T8D07]
Which of the following describes DMR?
\begin{enumerate}[label=\Alph*)]
    \item \textbf{A technique for time-multiplexing two digital voice signals on a single 12.5 kHz repeater channel}
    \item An automatic position tracking mode for FM mobiles communicating through repeaters
    \item An automatic computer logging technique for hands-off logging when communicating while operating a vehicle
    \item A digital technique for transmitting on two repeater inputs simultaneously for automatic error correction
\end{enumerate}
\end{tcolorbox}

\subsubsection*{Intuitive Explanation}
Imagine you have a single lane road, but you want to let two cars drive on it at the same time without crashing. DMR is like a traffic light system that allows two digital voice signals to share the same 12.5 kHz channel by taking turns. It’s like saying, You go first, then I’ll go next, and they keep switching back and forth. This way, both signals can travel smoothly without interfering with each other.

\subsubsection*{Advanced Explanation}
DMR, or Digital Mobile Radio, is a digital radio standard that uses Time Division Multiple Access (TDMA) to multiplex two digital voice signals on a single 12.5 kHz repeater channel. TDMA divides the channel into time slots, allowing two separate signals to share the same frequency by transmitting in alternating time intervals. Mathematically, if the total bandwidth is \( B \), each signal effectively uses \( \frac{B}{2} \) bandwidth during its assigned time slot.

The key concept here is the efficient use of spectrum resources. By time-multiplexing two signals, DMR doubles the capacity of a single channel without requiring additional bandwidth. This is particularly useful in crowded frequency bands where spectrum is a limited resource.

% Prompt for diagram: Generate a diagram showing two digital voice signals being time-multiplexed on a single 12.5 kHz channel using TDMA.