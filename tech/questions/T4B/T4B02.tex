\subsection{Entering Transceiver Operating Frequency}
\label{T4B02}

\begin{tcolorbox}[colback=gray!10!white,colframe=black!75!black,title=T4B02]
Which of the following can be used to enter a transceiver’s operating frequency?
\begin{enumerate}[label=\Alph*)]
    \item \textbf{The keypad or VFO knob}
    \item The CTCSS or DTMF encoder
    \item The Automatic Frequency Control
    \item All these choices are correct
\end{enumerate}
\end{tcolorbox}

\subsubsection{Intuitive Explanation}
Imagine your transceiver is like a fancy radio that you can tune to different stations. To change the station (or frequency), you can either type in the number directly using a keypad (like typing in a phone number) or twist a knob (like tuning an old-school radio). The other options, like CTCSS or DTMF, are more like secret codes that help you talk to specific people, and Automatic Frequency Control is like a helper that keeps your station from drifting. So, the best way to change your station is by using the keypad or the knob!

\subsubsection{Advanced Explanation}
In radio transceivers, the operating frequency is the specific frequency at which the device transmits and receives signals. To set this frequency, users typically have two primary methods:

1. \textbf{Keypad}: This allows the user to directly input the desired frequency numerically. It is precise and straightforward, especially when the exact frequency is known.

2. \textbf{VFO Knob (Variable Frequency Oscillator)}: This is a rotary control that allows the user to adjust the frequency incrementally. It is useful for fine-tuning or scanning through a range of frequencies.

The other options mentioned in the question serve different purposes:
- \textbf{CTCSS (Continuous Tone-Coded Squelch System)** and **DTMF (Dual-Tone Multi-Frequency)} encoders are used for signaling and access control, not for setting the operating frequency.
- \textbf{Automatic Frequency Control (AFC)} is a feature that helps maintain the receiver's frequency stability, especially in the presence of signal drift, but it does not set the initial operating frequency.

Therefore, the correct method to enter a transceiver’s operating frequency is using the keypad or VFO knob.

% Diagram Prompt: Generate a diagram showing a transceiver with labeled keypad and VFO knob for frequency input.