\subsection{Adjusting Voice Pitch in Single-Sideband Signals}
\label{T4B06}

\begin{tcolorbox}[colback=gray!10!white,colframe=black!75!black,title=T4B06]
Which of the following controls could be used if the voice pitch of a single-sideband signal returning to your CQ call seems too high or low?
\begin{enumerate}[noitemsep]
    \item The AGC or limiter
    \item The bandwidth selection
    \item The tone squelch
    \item \textbf{The RIT or Clarifier}
\end{enumerate}
\end{tcolorbox}

\subsubsection*{Explanation}
When receiving a single-sideband (SSB) signal, the voice pitch may appear too high or low due to a slight frequency mismatch between the transmitter and receiver. The RIT (Receiver Incremental Tuning) or Clarifier is specifically designed to fine-tune the receiver's frequency to match the incoming signal, thereby correcting the voice pitch. The other options (AGC, bandwidth selection, and tone squelch) do not directly affect the frequency tuning and thus are not suitable for this purpose.