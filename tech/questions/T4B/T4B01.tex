\subsection{Effect of Excessive Microphone Gain on SSB Transmissions}
\label{T4B01}

\begin{tcolorbox}[colback=gray!10!white,colframe=black!75!black,title=T4B01]
What is the effect of excessive microphone gain on SSB transmissions?
\begin{enumerate}[label=\Alph*)]
    \item Frequency instability
    \item \textbf{Distorted transmitted audio}
    \item Increased SWR
    \item All these choices are correct
\end{enumerate}
\end{tcolorbox}

\subsubsection{Intuitive Explanation}
Imagine you're trying to talk to your friend on a walkie-talkie, but you accidentally turn up the volume on your microphone too high. What happens? Your voice gets all crackly and weird, right? That's because the microphone is picking up too much sound and making it distorted. In SSB (Single Sideband) transmissions, the same thing happens if you set the microphone gain too high. Your voice gets all messed up, and it’s harder for the person on the other end to understand you. So, keep that microphone gain just right—not too high, not too low!

\subsubsection{Advanced Explanation}
In SSB (Single Sideband) transmissions, the microphone gain controls the amplitude of the audio signal that modulates the carrier wave. When the microphone gain is set too high, the audio signal can exceed the linear range of the modulator, leading to overmodulation. Overmodulation causes the transmitted audio to become distorted, as the peaks of the audio signal are clipped or compressed. This distortion manifests as a harsh, unnatural sound at the receiver, making it difficult to understand the transmitted message.

Mathematically, the modulation index \( m \) is given by:
\[
m = \frac{A_m}{A_c}
\]
where \( A_m \) is the amplitude of the modulating signal (audio) and \( A_c \) is the amplitude of the carrier signal. When \( A_m \) becomes too large due to excessive microphone gain, \( m \) exceeds 1, leading to overmodulation and distortion.

Additionally, excessive microphone gain does not cause frequency instability or increased SWR (Standing Wave Ratio). These issues are typically related to other factors such as oscillator stability or antenna matching, respectively. Therefore, the correct answer is \textbf{B: Distorted transmitted audio}.

% Diagram prompt: A diagram showing the relationship between microphone gain, modulation index, and the resulting waveform distortion in SSB transmissions would be helpful here.