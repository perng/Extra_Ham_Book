\subsection{Effect of Excessive Microphone Gain on SSB Transmissions}
\label{T4B01}

\begin{tcolorbox}[colback=gray!10!white,colframe=black!75!black,title=T4B01]
What is the effect of excessive microphone gain on SSB transmissions?
\begin{enumerate}[noitemsep]
    \item Frequency instability
    \item \textbf{Distorted transmitted audio}
    \item Increased SWR
    \item All these choices are correct
\end{enumerate}
\end{tcolorbox}

\subsubsection*{Intuitive Explanation}
Think of your microphone gain like the volume knob on a stereo. If you turn it up too high, the sound gets distorted and unpleasant to listen to. Similarly, in SSB (Single Sideband) transmissions, if the microphone gain is set too high, the audio signal becomes distorted, making it harder for the receiver to understand the transmission.

\subsubsection*{Advanced Explanation}
In SSB transmissions, the microphone gain controls the amplitude of the audio signal that modulates the carrier wave. Excessive gain can cause the audio signal to exceed the linear range of the modulator, leading to clipping and distortion. This distortion manifests as a garbled or unintelligible audio signal at the receiver's end. Unlike frequency instability or increased SWR, which are not directly caused by microphone gain, distorted audio is a direct consequence of overdriving the microphone input. Therefore, the correct answer is \textbf{B: Distorted transmitted audio}.