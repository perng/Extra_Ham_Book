\subsection{Tuning an FM Receiver Above or Below a Signal’s Frequency}
\label{T4B12}

\begin{tcolorbox}[colback=gray!10!white,colframe=black!75!black,title=T4B12]
What is the result of tuning an FM receiver above or below a signal’s frequency?
\begin{enumerate}[noitemsep]
    \item Change in audio pitch
    \item Sideband inversion
    \item Generation of a heterodyne tone
    \item \textbf{Distortion of the signal’s audio}
\end{enumerate}
\end{tcolorbox}

\subsubsection*{Intuitive Explanation}
Imagine you’re trying to listen to your favorite radio station, but instead of tuning directly to the station’s frequency, you accidentally tune slightly above or below it. What happens? The audio you hear becomes distorted, like trying to listen to a song with a scratched CD. This is because the FM receiver is designed to decode the signal precisely at the correct frequency. If you’re off, the audio gets messed up.

\subsubsection*{Advanced Explanation}
FM (Frequency Modulation) receivers are designed to demodulate the signal based on the exact frequency of the carrier wave. When you tune the receiver above or below the signal’s frequency, the demodulation process is no longer aligned with the carrier wave’s frequency. This misalignment causes the audio signal to be distorted because the receiver is not correctly interpreting the frequency variations that encode the audio information. The distortion occurs because the receiver’s demodulator is not synchronized with the incoming signal, leading to errors in the decoded audio.