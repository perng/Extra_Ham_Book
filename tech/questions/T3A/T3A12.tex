\subsection{Effect of Fog and Rain on 10m and 6m Bands}
\label{T3A12}

\begin{tcolorbox}[colback=gray!10!white,colframe=black!75!black,title=T3A12]
What is the effect of fog and rain on signals in the 10 meter and 6 meter bands?
\begin{enumerate}[noitemsep]
    \item Absorption
    \item \textbf{There is little effect}
    \item Deflection
    \item Range increase
\end{enumerate}
\end{tcolorbox}

\subsubsection*{Intuitive Explanation}
Imagine you're trying to talk to someone through a foggy or rainy day. If you're close enough, the fog or rain doesn't really stop your voice from reaching them. Similarly, signals in the 10 meter and 6 meter bands are like your voice in this scenario—they aren't significantly affected by fog or rain.

\subsubsection*{Advanced Explanation}
The 10 meter (28-29.7 MHz) and 6 meter (50-54 MHz) bands are part of the Very High Frequency (VHF) spectrum. At these frequencies, the wavelength is relatively long, and the interaction with atmospheric particles like fog and rain is minimal. Unlike higher frequency bands (e.g., microwave bands), where absorption and scattering by water droplets can be significant, VHF signals pass through fog and rain with little attenuation. This is why fog and rain have little effect on signals in these bands.