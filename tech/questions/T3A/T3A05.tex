\subsection{Communicating with a Distant Repeater Using a Directional Antenna}
\label{T3A05}

\begin{tcolorbox}[colback=gray!10!white,colframe=black!75!black,title=T3A05]
When using a directional antenna, how might your station be able to communicate with a distant repeater if buildings or obstructions are blocking the direct line of sight path?
\begin{enumerate}[noitemsep]
    \item Change from vertical to horizontal polarization
    \item \textbf{Try to find a path that reflects signals to the repeater}
    \item Try the long path
    \item Increase the antenna SWR
\end{enumerate}
\end{tcolorbox}

\subsubsection{Intuitive Explanation}
Imagine you're trying to throw a ball to a friend, but there's a big wall in the way. You can't throw the ball straight to them, so what do you do? You might try bouncing the ball off the ground or a nearby wall to get it to your friend. Similarly, when buildings or obstructions block the direct path to a repeater, you can try to bounce your radio signals off other surfaces to reach it. This is called signal reflection.

\subsubsection{Advanced Explanation}
In radio communication, line of sight (LOS) is often the most direct and efficient path for signal transmission. However, when obstacles like buildings or terrain block the LOS, signals can be reflected off other surfaces to reach the intended receiver. This phenomenon is known as multipath propagation. By using a directional antenna, you can aim the signal towards a reflective surface, such as a building or a hill, which can then redirect the signal towards the repeater. This method leverages the principles of wave reflection and can be an effective way to overcome obstructions in the communication path.