\subsection{Effect of Vegetation on UHF and Microwave Signals}
\label{T3A02}

\begin{tcolorbox}[colback=gray!10!white,colframe=black!75!black,title=T3A02]
What is the effect of vegetation on UHF and microwave signals?
\begin{enumerate}[noitemsep]
    \item Knife-edge diffraction
    \item \textbf{Absorption}
    \item Amplification
    \item Polarization rotation
\end{enumerate}
\end{tcolorbox}

\subsubsection*{Intuitive Explanation}
Imagine you're trying to shout through a dense forest. The trees and leaves absorb some of your sound, making it harder for someone on the other side to hear you. Similarly, vegetation absorbs UHF and microwave signals, reducing their strength as they pass through.

\subsubsection*{Advanced Explanation}
Vegetation, especially when dense and moist, can absorb radio waves in the UHF and microwave frequency ranges. This absorption occurs because the water content in the vegetation interacts with the electromagnetic waves, converting some of the wave energy into heat. This effect is more pronounced at higher frequencies, such as those in the UHF and microwave bands, because these wavelengths are closer in size to the water molecules and plant structures, leading to more efficient energy transfer and absorption.