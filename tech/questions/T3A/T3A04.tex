\subsection{Polarization Mismatch in VHF/UHF Links}
\label{T3A04}

\begin{tcolorbox}[colback=gray!10!white,colframe=black!75!black,title=T3A04]
What happens when antennas at opposite ends of a VHF or UHF line of sight radio link are not using the same polarization?
\begin{enumerate}[noitemsep]
    \item The modulation sidebands might become inverted
    \item \textbf{Received signal strength is reduced}
    \item Signals have an echo effect
    \item Nothing significant will happen
\end{enumerate}
\end{tcolorbox}

\subsubsection*{Intuitive Explanation}
Imagine you're trying to catch a frisbee. If you hold your hands horizontally, but your friend throws the frisbee vertically, you're going to have a hard time catching it. Similarly, if the antennas at both ends of a radio link are not aligned in the same polarization (horizontal, vertical, etc.), the signal strength will be weaker because the antennas aren't catching the signal as effectively.

\subsubsection*{Advanced Explanation}
Polarization refers to the orientation of the electric field of the radio wave. For optimal signal transmission and reception, the transmitting and receiving antennas should have the same polarization. When the polarizations are mismatched, the received signal strength is reduced due to polarization loss. The amount of loss depends on the angle between the polarizations of the transmitting and receiving antennas. For example, if one antenna is vertically polarized and the other is horizontally polarized, the signal loss can be significant, often reducing the received signal strength by more than 20 dB. This is why it's crucial to ensure that antennas in a VHF or UHF line of sight link are aligned in the same polarization.