\subsection{Antenna Polarization for VHF/UHF Long-Distance Contacts}
\label{T3A03}

\begin{tcolorbox}[colback=gray!10!white,colframe=black!75!black,title=T3A03]
What antenna polarization is normally used for long-distance CW and SSB contacts on the VHF and UHF bands?
\begin{enumerate}[noitemsep]
    \item Right-hand circular
    \item Left-hand circular
    \item \textbf{Horizontal}
    \item Vertical
\end{enumerate}
\end{tcolorbox}

\subsubsection*{Intuitive Explanation}
Imagine you're trying to throw a frisbee as far as possible. You'd probably throw it horizontally, right? Similarly, for long-distance communication on VHF and UHF bands, horizontal polarization is like throwing that frisbee—it helps the signal travel farther.

\subsubsection*{Advanced Explanation}
Horizontal polarization is preferred for long-distance VHF and UHF communications because it reduces ground wave attenuation and minimizes interference from vertically polarized signals, which are more common in local communications. This polarization aligns the electric field of the radio wave parallel to the Earth's surface, which is more efficient for long-distance propagation, especially when using directional antennas like Yagis or dipoles.