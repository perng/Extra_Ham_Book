\subsection{Effects of Multi-Path Propagation on Data Transmissions}
\label{T3A10}

\begin{tcolorbox}[colback=gray!10!white,colframe=black!75!black,title=T3A10]
What effect does multi-path propagation have on data transmissions?
\begin{enumerate}[noitemsep]
    \item Transmission rates must be increased by a factor equal to the number of separate paths observed
    \item Transmission rates must be decreased by a factor equal to the number of separate paths observed
    \item No significant changes will occur if the signals are transmitted using FM
    \item \textbf{Error rates are likely to increase}
\end{enumerate}
\end{tcolorbox}

Multi-path propagation occurs when a signal travels from the transmitter to the receiver via multiple paths due to reflections, diffractions, and scattering. This can cause the signal to arrive at the receiver at different times, leading to interference and distortion. As a result, the error rates in data transmissions are likely to increase, making option D the correct answer.

% Diagram Prompt: Generate a diagram showing multi-path propagation with a transmitter, receiver, and multiple signal paths due to reflections and diffractions. Use SVG format for clarity and scalability.