\section{Capacitance and Inductance}
\label{section:capacitance_and_inductance}

\subsection*{Capacitance}
Capacitance describes the ability of a component, such as a capacitor, to store energy in an electric field. The unit of capacitance is the farad (F), named after the English physicist Michael Faraday. A capacitor consists of two conductive plates separated by an insulating material, known as a dielectric. When a voltage is applied across the plates, an electric field is established, and energy is stored in this field. The capacitance \( C \) of a capacitor is given by the formula:
\begin{equation}
    C = \frac{Q}{V}
\end{equation}
where \( Q \) is the charge stored on the plates, and \( V \) is the voltage across the plates.

\subsection*{Inductance}
Inductance, on the other hand, describes the ability of a component, such as an inductor, to store energy in a magnetic field. The unit of inductance is the henry (H), named after the American scientist Joseph Henry. An inductor typically consists of a coil of wire, and when current flows through it, a magnetic field is generated. The energy is stored in this magnetic field. The inductance \( L \) of an inductor is given by:
\begin{equation}
    L = \frac{\Phi}{I}
\end{equation}
where \( \Phi \) is the magnetic flux through the coil, and \( I \) is the current flowing through it.

\subsection*{Impedance}
Impedance is a measure of the opposition to alternating current (AC) flow in a circuit. It is a complex quantity that combines resistance and reactance (both inductive and capacitive). The unit of impedance is the ohm (\(\Omega\)). Impedance is significant in AC circuits because it determines the relationship between voltage and current in the presence of reactance. The impedance \( Z \) of a circuit is given by:
\begin{equation}
    Z = R + jX
\end{equation}
where \( R \) is the resistance, \( X \) is the reactance, and \( j \) is the imaginary unit.

\subsection*{Energy Storage in Electric and Magnetic Fields}
Energy can be stored in both electric and magnetic fields. In a capacitor, energy is stored in the electric field between its plates. The energy \( E \) stored in a capacitor is given by:
\begin{equation}
    E = \frac{1}{2}CV^2
\end{equation}
In an inductor, energy is stored in the magnetic field generated by the current flowing through it. The energy \( E \) stored in an inductor is given by:
\begin{equation}
    E = \frac{1}{2}LI^2
\end{equation}

\begin{figure}[h]
    \centering
    %\includegraphics[width=0.8\textwidth]{figures/capacitor_inductor.svg}
    \caption{Capacitor and inductor in a circuit. The diagram shows a capacitor and an inductor connected in a simple circuit, illustrating their roles in energy storage.}
    \label{fig:capacitor_inductor}
    % Prompt: Diagram showing a capacitor and inductor in a circuit. The capacitor is represented by two parallel plates, and the inductor is represented by a coil. The circuit includes a voltage source and a resistor to complete the loop.
\end{figure}

\subsection*{Questions}
\begin{tcolorbox}[colback=gray!10!white,colframe=black!75!black,title={T5C01}]
    What describes the ability to store energy in an electric field?
    \begin{enumerate}[label=\Alph*,noitemsep]
        \item Inductance
        \item Resistance
        \item Tolerance
        \item \textbf{Capacitance}
    \end{enumerate}
\end{tcolorbox}
Capacitance is the ability to store energy in an electric field, as described by the behavior of a capacitor. The other options are incorrect because inductance relates to magnetic fields, resistance opposes current flow, and tolerance is a measure of component variability.

%memory_trick T5C01

\begin{tcolorbox}[colback=gray!10!white,colframe=black!75!black,title={T5C02}]
    What is the unit of capacitance?
    \begin{enumerate}[label=\Alph*,noitemsep]
        \item \textbf{The farad}
        \item The ohm
        \item The volt
        \item The henry
    \end{enumerate}
\end{tcolorbox}
The unit of capacitance is the farad (F). The ohm is the unit of resistance, the volt is the unit of voltage, and the henry is the unit of inductance.

%memory_trick T5C02

\begin{tcolorbox}[colback=gray!10!white,colframe=black!75!black,title={T5C03}]
    What describes the ability to store energy in a magnetic field?
    \begin{enumerate}[label=\Alph*,noitemsep]
        \item Admittance
        \item Capacitance
        \item Resistance
        \item \textbf{Inductance}
    \end{enumerate}
\end{tcolorbox}
Inductance describes the ability to store energy in a magnetic field, as seen in inductors. Admittance is the inverse of impedance, capacitance relates to electric fields, and resistance opposes current flow.

%memory_trick T5C03

\begin{tcolorbox}[colback=gray!10!white,colframe=black!75!black,title={T5C04}]
    What is the unit of inductance?
    \begin{enumerate}[label=\Alph*,noitemsep]
        \item The coulomb
        \item The farad
        \item \textbf{The henry}
        \item The ohm
    \end{enumerate}
\end{tcolorbox}
The unit of inductance is the henry (H). The coulomb is the unit of electric charge, the farad is the unit of capacitance, and the ohm is the unit of resistance.

%memory_trick T5C04

\begin{tcolorbox}[colback=gray!10!white,colframe=black!75!black,title={T5C05}]
    What is the unit of impedance?
    \begin{enumerate}[label=\Alph*,noitemsep]
        \item The volt
        \item The ampere
        \item The coulomb
        \item \textbf{The ohm}
    \end{enumerate}
\end{tcolorbox}
The unit of impedance is the ohm (\(\Omega\)). The volt is the unit of voltage, the ampere is the unit of current, and the coulomb is the unit of electric charge.

%memory_trick T5C05

\begin{tcolorbox}[colback=gray!10!white,colframe=black!75!black,title={T5C12}]
    What is impedance?
    \begin{enumerate}[label=\Alph*,noitemsep]
        \item \textbf{The opposition to AC current flow}
        \item The inverse of resistance
        \item The Q or Quality Factor of a component
        \item The power handling capability of a component
    \end{enumerate}
\end{tcolorbox}
Impedance is the opposition to AC current flow, combining resistance and reactance. The inverse of resistance is conductance, the Q factor relates to the efficiency of an inductor or capacitor, and power handling capability is unrelated to impedance.

%memory_trick T5C12

\subsection*{Summary}
\begin{itemize}
    \item \textbf{Capacitance}: The ability to store energy in an electric field, measured in farads (F).
    \item \textbf{Inductance}: The ability to store energy in a magnetic field, measured in henries (H).
    \item \textbf{Impedance}: The opposition to AC current flow, measured in ohms (\(\Omega\)).
    \item \textbf{Energy Storage}: Energy is stored in electric fields (capacitors) and magnetic fields (inductors), with formulas \( E = \frac{1}{2}CV^2 \) and \( E = \frac{1}{2}LI^2 \), respectively.
\end{itemize}
