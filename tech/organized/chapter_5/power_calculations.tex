\section{Power Calculations}
\label{section:power_calculations}

\subsection*{Introduction}
In this section, we will explore the fundamental concepts of electrical power calculations in DC circuits. We will discuss the relationship between power, voltage, and current, and derive the formula used to calculate electrical power. Additionally, we will solve practical problems to reinforce these concepts.

\subsection*{Power Formula in DC Circuits}
The electrical power \( P \) in a DC circuit is calculated using the formula:
\begin{equation}
    P = I \cdot E
    \label{eq:power_formula}
\end{equation}
where:
\begin{itemize}
    \item \( P \) is the power in watts (W),
    \item \( I \) is the current in amperes (A),
    \item \( E \) is the voltage in volts (V).
\end{itemize}

This formula is derived from the basic principles of electrical circuits, where power is the product of voltage and current. It is essential to understand this relationship to analyze and design electrical systems effectively.

\subsection*{Calculating Power Given Voltage and Current}
To calculate the power delivered in a circuit, you need to know the voltage and current. For example, if a circuit has a voltage of 13.8 volts and a current of 10 amperes, the power can be calculated as follows:
\[
P = I \cdot E = 10 \, \text{A} \cdot 13.8 \, \text{V} = 138 \, \text{W}
\]
This calculation demonstrates how the power formula is applied in practical scenarios.

\subsection*{Relationship Between Power, Voltage, and Current}
The relationship between power, voltage, and current is linear. As either voltage or current increases, the power delivered by the circuit also increases proportionally. This relationship is crucial for understanding how electrical systems behave under different conditions.

\subsection*{Example Calculations}
Let's solve a few example problems to solidify our understanding:
\begin{enumerate}
    \item \textbf{Example 1:} Calculate the power delivered by a voltage of 12 volts DC and a current of 2.5 amperes.
    \[
    P = I \cdot E = 2.5 \, \text{A} \cdot 12 \, \text{V} = 30 \, \text{W}
    \]
    \item \textbf{Example 2:} Determine the current required to deliver 120 watts at a voltage of 12 volts DC.
    \[
    I = \frac{P}{E} = \frac{120 \, \text{W}}{12 \, \text{V}} = 10 \, \text{A}
    \]
\end{enumerate}

\subsection*{Questions}
\begin{tcolorbox}[colback=gray!10!white,colframe=black!75!black,title={T5C08}]
    What is the formula used to calculate electrical power (P) in a DC circuit?
    \begin{enumerate}[label=\Alph*,noitemsep]
        \item \textbf{P = I E}
        \item P = E / I
        \item P = E – I
        \item P = I + E
    \end{enumerate}
\end{tcolorbox}
The formula for electrical power in a DC circuit is \( P = I \cdot E \), where \( P \) is power, \( I \) is current, and \( E \) is voltage. This is derived from the basic principles of electrical circuits.

%memory_trick T5C08

\begin{tcolorbox}[colback=gray!10!white,colframe=black!75!black,title={T5C09}]
    How much power is delivered by a voltage of 13.8 volts DC and a current of 10 amperes?
    \begin{enumerate}[label=\Alph*,noitemsep]
        \item \textbf{138 watts}
        \item 0.7 watts
        \item 23.8 watts
        \item 3.8 watts
    \end{enumerate}
\end{tcolorbox}
Using the power formula \( P = I \cdot E \), we calculate:
\[
P = 10 \, \text{A} \cdot 13.8 \, \text{V} = 138 \, \text{W}
\]
The other options are incorrect because they do not follow the correct formula.

%memory_trick T5C09

\begin{tcolorbox}[colback=gray!10!white,colframe=black!75!black,title={T5C10}]
    How much power is delivered by a voltage of 12 volts DC and a current of 2.5 amperes?
    \begin{enumerate}[label=\Alph*,noitemsep]
        \item 4.8 watts
        \item \textbf{30 watts}
        \item 14.5 watts
        \item 0.208 watts
    \end{enumerate}
\end{tcolorbox}
Using the power formula:
\[
P = 2.5 \, \text{A} \cdot 12 \, \text{V} = 30 \, \text{W}
\]
The other options are incorrect because they do not follow the correct formula.

%memory_trick T5C10

\begin{tcolorbox}[colback=gray!10!white,colframe=black!75!black,title={T5C11}]
    How much current is required to deliver 120 watts at a voltage of 12 volts DC?
    \begin{enumerate}[label=\Alph*,noitemsep]
        \item 0.1 amperes
        \item \textbf{10 amperes}
        \item 12 amperes
        \item 132 amperes
    \end{enumerate}
\end{tcolorbox}
Rearranging the power formula \( P = I \cdot E \) to solve for current:
\[
I = \frac{P}{E} = \frac{120 \, \text{W}}{12 \, \text{V}} = 10 \, \text{A}
\]
The other options are incorrect because they do not follow the correct formula.

%memory_trick T5C11

\subsection*{Summary}
In this section, we discussed the following key concepts:
\begin{itemize}
    \item \textbf{Electrical Power:} The rate at which electrical energy is transferred by an electric circuit.
    \item \textbf{Power Formula:} \( P = I \cdot E \), where \( P \) is power, \( I \) is current, and \( E \) is voltage.
    \item \textbf{Voltage and Current:} The two fundamental quantities that determine the power in a DC circuit.
\end{itemize}
Understanding these concepts is essential for analyzing and designing electrical systems effectively.
