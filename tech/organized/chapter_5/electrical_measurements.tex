\section{Electrical Measurements}
\label{section:electrical_measurements}

\subsection*{Introduction}
Electrical measurements are fundamental in understanding and analyzing electrical circuits and systems. This section covers the conversion between different units of electrical measurement, the significance of frequency, and the relationship between various units of voltage and capacitance.

\subsection*{Unit Conversions in Electrical Measurements}
Electrical measurements often involve converting between different units, such as amperes to milliamperes, volts to microvolts, or hertz to kilohertz. Understanding these conversions is crucial for accurate measurements and calculations. For example, to convert amperes to milliamperes, you multiply by 1000:
\begin{equation}
1 \, \text{A} = 1000 \, \text{mA}
\end{equation}
Similarly, to convert hertz to kilohertz, you divide by 1000:
\begin{equation}
1 \, \text{kHz} = 1000 \, \text{Hz}
\end{equation}

\subsection*{Frequency in Electrical Measurements}
Frequency is a critical parameter in electrical measurements, especially in radio technology. It represents the number of cycles per second and is measured in hertz (Hz). Higher frequencies, such as kilohertz (kHz) and megahertz (MHz), are commonly used in radio communications. For example, 1.5 MHz is equivalent to 1500 kHz:
\begin{equation}
1.5 \, \text{MHz} = 1500 \, \text{kHz}
\end{equation}

\subsection*{Voltage Units}
Voltage is measured in volts (V), but smaller units like millivolts (mV) and microvolts (µV) are often used for precision measurements. The relationships between these units are:
\begin{equation}
1 \, \text{V} = 1000 \, \text{mV} = 1,000,000 \, \text{µV}
\end{equation}
For example, one microvolt is one-millionth of a volt:
\begin{equation}
1 \, \text{µV} = 10^{-6} \, \text{V}
\end{equation}

\subsection*{Capacitance and Its Units}
Capacitance is the ability of a system to store an electric charge and is measured in farads (F). Smaller units like microfarads (µF) and picofarads (pF) are commonly used. The conversion between these units is:
\begin{equation}
1 \, \text{F} = 1,000,000 \, \text{µF} = 1,000,000,000,000 \, \text{pF}
\end{equation}
For example, one microfarad is equal to one million picofarads:
\begin{equation}
1 \, \text{µF} = 1,000,000 \, \text{pF}
\end{equation}

\subsection*{Figures and Tables}
\begin{figure}[h]
    \centering
    %\includegraphics{fig:unit_conversions}
    \caption{Unit conversions for electrical measurements.}
    \label{fig:unit_conversions}
    % Diagram showing unit conversions for electrical measurements, including amperes to milliamperes, volts to microvolts, and hertz to kilohertz.
\end{figure}

\begin{table}[h]
    \centering
    \caption{Common Electrical Unit Conversions}
    \label{tab:unit_conversions}
    \begin{tabular}{|c|c|}
        \hline
        \textbf{Unit} & \textbf{Conversion} \\
        \hline
        1 A & 1000 mA \\
        1 kV & 1000 V \\
        1 MHz & 1000 kHz \\
        1 µF & 1,000,000 pF \\
        \hline
    \end{tabular}
\end{table}

\subsection*{Questions}
\begin{tcolorbox}[colback=gray!10!white,colframe=black!75!black,title={T5B01}]
    How many milliamperes is 1.5 amperes?
    \begin{enumerate}[label=\Alph*),noitemsep]
        \item 15 milliamperes
        \item 150 milliamperes
        \item \textbf{1500 milliamperes}
        \item 15,000 milliamperes
    \end{enumerate}
\end{tcolorbox}
To convert amperes to milliamperes, multiply by 1000: \(1.5 \, \text{A} \times 1000 = 1500 \, \text{mA}\). The other options are incorrect because they do not reflect this conversion.
%memory_trick T5B01

\begin{tcolorbox}[colback=gray!10!white,colframe=black!75!black,title={T5B02}]
    Which is equal to 1,500,000 hertz?
    \begin{enumerate}[label=\Alph*),noitemsep]
        \item \textbf{1500 kHz}
        \item 1500 MHz
        \item 15 GHz
        \item 150 kHz
    \end{enumerate}
\end{tcolorbox}
To convert hertz to kilohertz, divide by 1000: \(1,500,000 \, \text{Hz} \div 1000 = 1500 \, \text{kHz}\). The other options are incorrect because they do not match this conversion.
%memory_trick T5B02

\begin{tcolorbox}[colback=gray!10!white,colframe=black!75!black,title={T5B03}]
    Which is equal to one kilovolt?
    \begin{enumerate}[label=\Alph*),noitemsep]
        \item One one-thousandth of a volt
        \item One hundred volts
        \item \textbf{One thousand volts}
        \item One million volts
    \end{enumerate}
\end{tcolorbox}
One kilovolt is equal to 1000 volts. The other options are incorrect because they do not match this definition.
%memory_trick T5B03

\begin{tcolorbox}[colback=gray!10!white,colframe=black!75!black,title={T5B04}]
    Which is equal to one microvolt?
    \begin{enumerate}[label=\Alph*),noitemsep]
        \item \textbf{One one-millionth of a volt}
        \item One million volts
        \item One thousand kilovolts
        \item One one-thousandth of a volt
    \end{enumerate}
\end{tcolorbox}
One microvolt is equal to one-millionth of a volt. The other options are incorrect because they do not match this definition.
%memory_trick T5B04

\begin{tcolorbox}[colback=gray!10!white,colframe=black!75!black,title={T5B05}]
    Which is equal to 500 milliwatts?
    \begin{enumerate}[label=\Alph*),noitemsep]
        \item 0.02 watts
        \item \textbf{0.5 watts}
        \item 5 watts
        \item 50 watts
    \end{enumerate}
\end{tcolorbox}
To convert milliwatts to watts, divide by 1000: \(500 \, \text{mW} \div 1000 = 0.5 \, \text{W}\). The other options are incorrect because they do not reflect this conversion.
%memory_trick T5B05

\begin{tcolorbox}[colback=gray!10!white,colframe=black!75!black,title={T5B06}]
    Which is equal to 3000 milliamperes?
    \begin{enumerate}[label=\Alph*),noitemsep]
        \item 0.003 amperes
        \item 0.3 amperes
        \item 3,000,000 amperes
        \item \textbf{3 amperes}
    \end{enumerate}
\end{tcolorbox}
To convert milliamperes to amperes, divide by 1000: \(3000 \, \text{mA} \div 1000 = 3 \, \text{A}\). The other options are incorrect because they do not reflect this conversion.
%memory_trick T5B06

\begin{tcolorbox}[colback=gray!10!white,colframe=black!75!black,title={T5B07}]
    Which is equal to 3.525 MHz?
    \begin{enumerate}[label=\Alph*),noitemsep]
        \item 0.003525 kHz
        \item 35.25 kHz
        \item \textbf{3525 kHz}
        \item 3,525,000 kHz
    \end{enumerate}
\end{tcolorbox}
To convert megahertz to kilohertz, multiply by 1000: \(3.525 \, \text{MHz} \times 1000 = 3525 \, \text{kHz}\). The other options are incorrect because they do not match this conversion.
%memory_trick T5B07

\begin{tcolorbox}[colback=gray!10!white,colframe=black!75!black,title={T5B08}]
    Which is equal to 1,000,000 picofarads?
    \begin{enumerate}[label=\Alph*),noitemsep]
        \item 0.001 microfarads
        \item \textbf{1 microfarad}
        \item 1000 microfarads
        \item 1,000,000,000 microfarads
    \end{enumerate}
\end{tcolorbox}
To convert picofarads to microfarads, divide by 1,000,000: \(1,000,000 \, \text{pF} \div 1,000,000 = 1 \, \text{µF}\). The other options are incorrect because they do not reflect this conversion.
%memory_trick T5B08

\subsection*{Summary}
This section covered the following key concepts:
\begin{itemize}
    \item \textbf{Milliamperes}: A unit of electric current equal to one-thousandth of an ampere.
    \item \textbf{Hertz and kilohertz}: Units of frequency, with 1 kHz equal to 1000 Hz.
    \item \textbf{Volts and kilovolts}: Units of voltage, with 1 kV equal to 1000 V.
    \item \textbf{Microvolts}: A unit of voltage equal to one-millionth of a volt.
    \item \textbf{Milliwatts}: A unit of power equal to one-thousandth of a watt.
    \item \textbf{Picofarads and microfarads}: Units of capacitance, with 1 µF equal to 1,000,000 pF.
\end{itemize}
Understanding these units and their conversions is essential for working with electrical measurements in radio technology.
