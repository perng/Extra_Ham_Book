\section{Satellite Communications}
\label{section:satellite_communications}

\subsection*{Introduction}
Satellite communications play a crucial role in modern radio technology, enabling long-distance communication, data transmission, and global connectivity. This section explores key concepts such as satellite telemetry, Doppler shift, U/V mode, spin fading, and the characteristics of Low Earth Orbit (LEO) satellites.

\subsection*{Satellite Telemetry}
Telemetry is essential for monitoring the health and status of satellites. It involves the transmission of data such as temperature, battery voltage, and system performance from the satellite to ground stations. This information is critical for ensuring the satellite operates correctly and for diagnosing any issues that may arise.

\subsection*{Doppler Shift}
Doppler shift is a phenomenon observed in satellite communications where the frequency of the signal changes due to the relative motion between the satellite and the ground station. This effect is particularly pronounced in LEO satellites, which move rapidly relative to the Earth's surface. The Doppler shift can be calculated using the formula:
\begin{equation}
    \Delta f = \frac{v \cdot f_0}{c}
    \label{eq:doppler_shift}
\end{equation}
where \( \Delta f \) is the frequency shift, \( v \) is the relative velocity, \( f_0 \) is the original frequency, and \( c \) is the speed of light.

\subsection*{U/V Mode}
U/V mode refers to a common configuration in amateur radio satellites where the uplink is in the 70 centimeter band (UHF) and the downlink is in the 2 meter band (VHF). This mode allows for efficient communication and is widely used due to the availability of equipment and the favorable propagation characteristics of these frequency bands.

\subsection*{Spin Fading}
Spin fading occurs when a satellite's rotation causes periodic variations in signal strength. This effect is due to the changing orientation of the satellite's antennas relative to the ground station. Spin fading can be mitigated by using diversity reception techniques or by tracking the satellite's orientation.

\subsection*{LEO Satellites}
Low Earth Orbit (LEO) satellites are characterized by their relatively low altitude, typically between 160 and 2,000 kilometers above the Earth's surface. This proximity results in shorter communication delays and lower power requirements for transmission. Table \ref{tab:leo_satellites} summarizes the key characteristics of LEO satellites.

\begin{table}[h!]
    \centering
    \caption{Characteristics of LEO Satellites}
    \label{tab:leo_satellites}
    \begin{tabular}{|l|l|}
        \hline
        \textbf{Characteristic} & \textbf{Description} \\
        \hline
        Altitude & 160 - 2,000 km \\
        Orbital Period & 90 - 120 minutes \\
        Signal Delay & 5 - 10 ms \\
        Power Requirements & Low \\
        \hline
    \end{tabular}
\end{table}

\subsection*{Figures}
\begin{figure}[h!]
    \centering
    %\includegraphics[width=0.8\textwidth]{doppler_shift.png}
    \caption{Doppler shift in satellite communications. The diagram illustrates the change in frequency as the satellite moves relative to the ground station.}
    \label{fig:doppler_shift}
    % Diagram illustrating Doppler shift in satellite communications. The figure should show a satellite moving towards and away from a ground station, with the corresponding frequency shifts indicated.
\end{figure}

\begin{figure}[h!]
    \centering
    %\includegraphics[width=0.8\textwidth]{spin_fading.png}
    \caption{Effect of spin fading on satellite signals. The graph shows the periodic variation in signal strength due to the satellite's rotation.}
    \label{fig:spin_fading}
    % Graph showing the effect of spin fading on satellite signals. The figure should depict a sinusoidal signal strength variation over time, with annotations explaining the cause of the fading.
\end{figure}

\subsection*{Questions}
\begin{tcolorbox}[colback=gray!10!white,colframe=black!75!black,title={T8B01}]
    What telemetry information is typically transmitted by satellite beacons?
    \begin{enumerate}[label=\Alph*,noitemsep]
        \item The signal strength of received signals
        \item Time of day accurate to plus or minus 1/10 second
        \item \textbf{Health and status of the satellite}
        \item All these choices are correct
    \end{enumerate}
\end{tcolorbox}
Satellite beacons transmit health and status information, which is crucial for monitoring the satellite's condition. This includes data on battery voltage, temperature, and system performance. The other options, while potentially useful, are not the primary purpose of satellite telemetry.

%memory_trick T8B01

\begin{tcolorbox}[colback=gray!10!white,colframe=black!75!black,title={T8B02}]
    What is the impact of using excessive effective radiated power on a satellite uplink?
    \begin{enumerate}[label=\Alph*,noitemsep]
        \item Possibility of commanding the satellite to an improper mode
        \item \textbf{Blocking access by other users}
        \item Overloading the satellite batteries
        \item Possibility of rebooting the satellite control computer
    \end{enumerate}
\end{tcolorbox}
Using excessive power on a satellite uplink can block access for other users, as the strong signal may dominate the satellite's receiver. This is why it's important to use the minimum necessary power for communication.

%memory_trick T8B02

\begin{tcolorbox}[colback=gray!10!white,colframe=black!75!black,title={T8B03}]
    Which of the following are provided by satellite tracking programs?
    \begin{enumerate}[label=\Alph*,noitemsep]
        \item Maps showing the real-time position of the satellite track over Earth
        \item The time, azimuth, and elevation of the start, maximum altitude, and end of a pass
        \item The apparent frequency of the satellite transmission, including effects of Doppler shift
        \item \textbf{All these choices are correct}
    \end{enumerate}
\end{tcolorbox}
Satellite tracking programs provide comprehensive information, including real-time position maps, pass details, and frequency adjustments for Doppler shift. All the listed options are correct.

%memory_trick T8B03

\begin{tcolorbox}[colback=gray!10!white,colframe=black!75!black,title={T8B04}]
    What mode of transmission is commonly used by amateur radio satellites?
    \begin{enumerate}[label=\Alph*,noitemsep]
        \item SSB
        \item FM
        \item CW/data
        \item \textbf{All these choices are correct}
    \end{enumerate}
\end{tcolorbox}
Amateur radio satellites commonly use SSB, FM, and CW/data modes. Each mode has its advantages depending on the application and available equipment.

%memory_trick T8B04

\begin{tcolorbox}[colback=gray!10!white,colframe=black!75!black,title={T8B05}]
    What is a satellite beacon?
    \begin{enumerate}[label=\Alph*,noitemsep]
        \item The primary transmit antenna on the satellite
        \item An indicator light that shows where to point your antenna
        \item A reflective surface on the satellite
        \item \textbf{A transmission from a satellite that contains status information}
    \end{enumerate}
\end{tcolorbox}
A satellite beacon is a transmission that provides status information about the satellite, such as its health and operational status. This is essential for monitoring and maintaining the satellite.

%memory_trick T8B05

\begin{tcolorbox}[colback=gray!10!white,colframe=black!75!black,title={T8B06}]
    Which of the following are inputs to a satellite tracking program?
    \begin{enumerate}[label=\Alph*,noitemsep]
        \item The satellite transmitted power
        \item \textbf{The Keplerian elements}
        \item The last observed time of zero Doppler shift
        \item All these choices are correct
    \end{enumerate}
\end{tcolorbox}
Keplerian elements are the primary inputs to a satellite tracking program. These elements describe the satellite's orbit and are used to predict its position and movement.

%memory_trick T8B06

\begin{tcolorbox}[colback=gray!10!white,colframe=black!75!black,title={T8B07}]
    What is Doppler shift in reference to satellite communications?
    \begin{enumerate}[label=\Alph*,noitemsep]
        \item A change in the satellite orbit
        \item A mode where the satellite receives signals on one band and transmits on another
        \item \textbf{An observed change in signal frequency caused by relative motion between the satellite and Earth station}
        \item A special digital communications mode for some satellites
    \end{enumerate}
\end{tcolorbox}
Doppler shift is the change in frequency observed due to the relative motion between the satellite and the ground station. This effect is particularly important in LEO satellites, where the relative velocity is high.

%memory_trick T8B07

\begin{tcolorbox}[colback=gray!10!white,colframe=black!75!black,title={T8B08}]
    What is meant by the statement that a satellite is operating in U/V mode?
    \begin{enumerate}[label=\Alph*,noitemsep]
        \item The satellite uplink is in the 15 meter band and the downlink is in the 10 meter band
        \item \textbf{The satellite uplink is in the 70 centimeter band and the downlink is in the 2 meter band}
        \item The satellite operates using ultraviolet frequencies
        \item The satellite frequencies are usually variable
    \end{enumerate}
\end{tcolorbox}
U/V mode refers to a configuration where the uplink is in the 70 centimeter band (UHF) and the downlink is in the 2 meter band (VHF). This mode is commonly used in amateur radio satellites.

%memory_trick T8B08

\subsection*{Summary}
This section covered several key concepts in satellite communications:
\begin{itemize}
    \item \textbf{Satellite Telemetry}: The transmission of health and status information from the satellite to ground stations.
    \item \textbf{Doppler Shift}: The change in signal frequency due to relative motion between the satellite and the ground station.
    \item \textbf{U/V Mode}: A common configuration where the uplink is in the 70 cm band and the downlink is in the 2 m band.
    \item \textbf{Spin Fading}: Periodic variations in signal strength caused by the satellite's rotation.
    \item \textbf{LEO Satellites}: Satellites in low Earth orbit, characterized by low altitude and short orbital periods.
\end{itemize}
