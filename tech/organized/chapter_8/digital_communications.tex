\section{Digital Communications}
\label{section:digital_communications}

\subsection*{Digital Modes in Amateur Radio}
Digital communication modes have become increasingly popular in amateur radio due to their efficiency and versatility. Some of the most commonly used digital modes include DMR (Digital Mobile Radio), APRS (Automatic Packet Reporting System), and PSK (Phase Shift Keying). Each of these modes serves different purposes and operates on distinct principles.

\subsection*{DMR and Talkgroups}
DMR is a digital voice mode that allows for efficient use of radio spectrum by time-multiplexing two digital voice signals on a single 12.5 kHz repeater channel. A key feature of DMR is the use of \textit{talkgroups}, which enable groups of users to share a channel at different times without hearing other users on the same channel. This is particularly useful for organizing communications among users with common interests or objectives.

\begin{figure}[h]
    \centering
    % \includegraphics[width=0.8\textwidth]{dmr_talkgroup}
    \caption{Structure of a DMR talkgroup. The diagram illustrates how multiple users are grouped into talkgroups, allowing for efficient channel sharing.}
    \label{fig:dmr_talkgroup}
\end{figure}

\subsection*{APRS Applications}
APRS is a digital communication mode that enables the transmission of various types of data, including GPS position data, text messages, and weather information. It is widely used for real-time tactical digital communications, often in conjunction with a map showing the locations of stations. This makes APRS particularly useful for emergency communications and tracking.

\begin{figure}[h]
    \centering
    % \includegraphics[width=0.8\textwidth]{aprs_flow}
    \caption{Data flow in an APRS network. The graph shows how data is transmitted and received across an APRS network, including GPS data and text messages.}
    \label{fig:aprs_flow}
\end{figure}

\subsection*{ARQ Transmission and Error Correction}
ARQ (Automatic Repeat Request) is a protocol used in digital communications to ensure data integrity. When an error is detected in a transmitted packet, the receiver requests a retransmission of the corrupted data. This process is crucial for maintaining reliable communication, especially in environments with high noise or interference.

\subsection*{Comparison of Digital Modes}
Table \ref{tab:digital_modes} summarizes the key features of the digital modes discussed in this section.

\begin{table}[h]
    \centering
    \caption{Comparison of Digital Modes}
    \label{tab:digital_modes}
    \begin{tabular}{|l|l|l|}
        \hline
        \textbf{Mode} & \textbf{Key Feature} & \textbf{Application} \\
        \hline
        DMR & Time-multiplexing of voice signals & Efficient spectrum usage \\
        APRS & Transmission of GPS and text data & Real-time tracking and messaging \\
        PSK & Phase modulation for data transmission & Low-bandwidth data transfer \\
        ARQ & Error detection and correction & Reliable data transmission \\
        \hline
    \end{tabular}
\end{table}

\subsection*{Questions}
\begin{tcolorbox}[colback=gray!10!white,colframe=black!75!black,title={T8D01}]
    Which of the following is a digital communications mode?
    \begin{enumerate}[label=\Alph*,noitemsep]
        \item Packet radio
        \item IEEE 802.11
        \item FT8
        \item \textbf{All these choices are correct}
    \end{enumerate}
\end{tcolorbox}
Packet radio, IEEE 802.11, and FT8 are all examples of digital communication modes. Packet radio is used for data transmission, IEEE 802.11 is a standard for wireless networking, and FT8 is a popular digital mode for weak signal communication.

%memory_trick T8D01

\begin{tcolorbox}[colback=gray!10!white,colframe=black!75!black,title={T8D02}]
    What is a “talkgroup” on a DMR repeater?
    \begin{enumerate}[label=\Alph*,noitemsep]
        \item A group of operators sharing common interests
        \item \textbf{A way for groups of users to share a channel at different times without hearing other users on the channel}
        \item A protocol that increases the signal-to-noise ratio when multiple repeaters are linked together
        \item A net that meets at a specified time
    \end{enumerate}
\end{tcolorbox}
A talkgroup in DMR is a method for organizing users into groups that share a channel without interfering with other groups. This allows for efficient use of the repeater channel.

%memory_trick T8D02

\begin{tcolorbox}[colback=gray!10!white,colframe=black!75!black,title={T8D03}]
    What kind of data can be transmitted by APRS?
    \begin{enumerate}[label=\Alph*,noitemsep]
        \item GPS position data
        \item Text messages
        \item Weather data
        \item \textbf{All these choices are correct}
    \end{enumerate}
\end{tcolorbox}
APRS can transmit various types of data, including GPS position data, text messages, and weather information. This versatility makes it a valuable tool for amateur radio operators.

%memory_trick T8D03

\subsection*{Summary}
This section introduced several key concepts in digital communications, including:
\begin{itemize}
    \item \textbf{Digital Modes}: Techniques like DMR, APRS, and PSK that enable efficient and reliable data transmission.
    \item \textbf{DMR}: A digital voice mode that uses talkgroups to organize users and optimize channel usage.
    \item \textbf{APRS}: A system for transmitting GPS, text, and weather data, often used for real-time tracking and messaging.
    \item \textbf{PSK}: A modulation technique for low-bandwidth data transfer.
    \item \textbf{ARQ Transmission}: A protocol for error detection and correction, ensuring reliable communication.
\end{itemize}
