\section{Control Operator Responsibilities}
\label{sec:control_operator}

\subsection*{Introduction}
This section discusses the responsibilities and requirements of control operators in amateur radio stations. It also covers the rules for operating through amateur satellites or space stations, the determination of transmitting frequency privileges, and the definition of a control point.

\subsection*{Control Operator Requirements}
The control operator is responsible for ensuring that the station operates in compliance with FCC regulations. The station licensee must designate the control operator, and this operator must hold the appropriate class of license for the frequencies being used. The control operator and the station licensee share responsibility for the proper operation of the station when the control operator is not the licensee.

\subsection*{Satellite and Space Station Operations}
When operating through an amateur satellite or space station, the control operator must be authorized to transmit on the satellite's uplink frequency. There are no additional certifications required beyond the operator's license class.

\subsection*{Transmitting Frequency Privileges}
The transmitting frequency privileges of an amateur station are determined by the class of operator license held by the control operator. This ensures that the station operates within the frequency bands authorized for the control operator's license class.

\subsection*{Control Point Definition}
The control point is the location at which the control operator function is performed. This is distinct from the location of the transmitting apparatus or antenna. The control point is crucial for ensuring that the station is operated in compliance with regulations.

\subsection*{Questions}
\begin{tcolorbox}[colback=gray!10!white,colframe=black!75!black,title={T1E01}]
When may an amateur station transmit without a control operator?
\begin{enumerate}[label=\Alph*,noitemsep]
    \item When using automatic control, such as in the case of a repeater
    \item When the station licensee is away and another licensed amateur is using the station
    \item When the transmitting station is an auxiliary station
    \item \textbf{Never}
\end{enumerate}
\end{tcolorbox}
An amateur station must always have a control operator when transmitting. This is a fundamental rule in amateur radio operations to ensure compliance with regulations.

%memory_trick T1E01

\begin{tcolorbox}[colback=gray!10!white,colframe=black!75!black,title={T1E02}]
Who may be the control operator of a station communicating through an amateur satellite or space station?
\begin{enumerate}[label=\Alph*,noitemsep]
    \item Only an Amateur Extra Class operator
    \item A General class or higher licensee with a satellite operator certification
    \item Only an Amateur Extra Class operator who is also an AMSAT member
    \item \textbf{Any amateur allowed to transmit on the satellite uplink frequency}
\end{enumerate}
\end{tcolorbox}
Any licensed amateur who is authorized to transmit on the satellite's uplink frequency may act as the control operator. No additional certifications are required.

%memory_trick T1E02

\begin{tcolorbox}[colback=gray!10!white,colframe=black!75!black,title={T1E03}]
Who must designate the station control operator?
\begin{enumerate}[label=\Alph*,noitemsep]
    \item \textbf{The station licensee}
    \item The FCC
    \item The frequency coordinator
    \item Any licensed operator
\end{enumerate}
\end{tcolorbox}
The station licensee is responsible for designating the control operator. This ensures that the operator is aware of and complies with the station's operational parameters.

%memory_trick T1E03

\begin{tcolorbox}[colback=gray!10!white,colframe=black!75!black,title={T1E04}]
What determines the transmitting frequency privileges of an amateur station?
\begin{enumerate}[label=\Alph*,noitemsep]
    \item The frequency authorized by the frequency coordinator
    \item The frequencies printed on the license grant
    \item The highest class of operator license held by anyone on the premises
    \item \textbf{The class of operator license held by the control operator}
\end{enumerate}
\end{tcolorbox}
The transmitting frequency privileges are determined by the class of operator license held by the control operator. This ensures that the station operates within the authorized frequency bands.

%memory_trick T1E04

\begin{tcolorbox}[colback=gray!10!white,colframe=black!75!black,title={T1E05}]
What is an amateur station’s control point?
\begin{enumerate}[label=\Alph*,noitemsep]
    \item The location of the station’s transmitting antenna
    \item The location of the station’s transmitting apparatus
    \item \textbf{The location at which the control operator function is performed}
    \item The mailing address of the station licensee
\end{enumerate}
\end{tcolorbox}
The control point is the location where the control operator function is performed. This is distinct from the physical location of the transmitting apparatus or antenna.

%memory_trick T1E05

\begin{tcolorbox}[colback=gray!10!white,colframe=black!75!black,title={T1E06}]
When, under normal circumstances, may a Technician class licensee be the control operator of a station operating in an Amateur Extra Class band segment?
\begin{enumerate}[label=\Alph*,noitemsep]
    \item \textbf{At no time}
    \item When designated as the control operator by an Amateur Extra Class licensee
    \item As part of a multi-operator contest team
    \item When using a club station whose trustee holds an Amateur Extra Class license
\end{enumerate}
\end{tcolorbox}
A Technician class licensee cannot be the control operator for a station operating in an Amateur Extra Class band segment. The control operator must hold the appropriate license class for the frequency band being used.

%memory_trick T1E06

\begin{tcolorbox}[colback=gray!10!white,colframe=black!75!black,title={T1E07}]
When the control operator is not the station licensee, who is responsible for the proper operation of the station?
\begin{enumerate}[label=\Alph*,noitemsep]
    \item All licensed amateurs who are present at the operation
    \item Only the station licensee
    \item Only the control operator
    \item \textbf{The control operator and the station licensee}
\end{enumerate}
\end{tcolorbox}
Both the control operator and the station licensee are responsible for the proper operation of the station when the control operator is not the licensee.

%memory_trick T1E07

\begin{tcolorbox}[colback=gray!10!white,colframe=black!75!black,title={T1E08}]
Which of the following is an example of automatic control?
\begin{enumerate}[label=\Alph*,noitemsep]
    \item \textbf{Repeater operation}
    \item Controlling a station over the internet
    \item Using a computer or other device to send CW automatically
    \item Using a computer or other device to identify automatically
\end{enumerate}
\end{tcolorbox}
Repeater operation is an example of automatic control, where the station operates without direct human intervention.

%memory_trick T1E08

\subsection*{Summary}
This section covered the key responsibilities and requirements of control operators in amateur radio stations. The main concepts discussed include:
\begin{itemize}
    \item \textbf{Control operator requirements}: The control operator must be designated by the station licensee and must hold the appropriate license class for the frequencies being used.
    \item \textbf{Satellite and space station operations}: Any licensed amateur authorized to transmit on the satellite's uplink frequency may act as the control operator.
    \item \textbf{Transmitting frequency privileges}: These are determined by the class of operator license held by the control operator.
    \item \textbf{Control point definition}: The control point is the location where the control operator function is performed, distinct from the physical location of the transmitting apparatus or antenna.
\end{itemize}
