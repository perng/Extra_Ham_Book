
% ************************************************************************
% CHAPTER 4: Sublement E4 - Amateur Practices
% ************************************************************************
\chapter{E4: Amateur Practices - Working with the Tools!}
% ------------------------------------------------------------------------
% SECTION E4A: Test Equipment and RF Measurements
% ------------------------------------------------------------------------
\section{E4A: Test Equipment and RF Measurements - Making it Measurable!}

\subsection*{Understanding the Basics}
Get ready to become best friends with a range of \textcolor{myblue}{\textbf{test equipment}}! This section introduces the main tools that are essential for measuring and diagnosing radio signals. We will explore \textcolor{myblue}{\textbf{analog and digital instruments}} such as voltmeters and ammeters. We will learn about more advanced gear like \textcolor{myblue}{\textbf{spectrum analyzers}} and \textcolor{myblue}{\textbf{antenna analyzers}} used for examining signal qualities and antenna performance. You will also be introduced to \textcolor{myblue}{\textbf{oscilloscopes}} that allows you to see the radio signals with time. Lastly, we will discuss the basics of \textcolor{myblue}{\textbf{RF measurements}} which utilizes these tools to understand and improve the operation of radio circuits.
\mybox{mygreen}{
  \textbf{Fun Fact}: Did you know that some modern oscilloscopes use the same principle as the TV picture tubes from the 20th century? But now we get the advantage of digital signal processing to measure even complex waveforms with excellent precision!
}
\subsection*{Key Concepts for the Questions}
\begin{itemize}
    \item \textbf{Oscilloscope:} Instrument that displays signal characteristics over time.
     \item \textbf{Digital Oscilloscope:}  Uses analog-to-digital converters for sampling signal.
        \item \textbf{Sampling rate:} The rate at which a digital scope measures data samples.
        \item \textbf{Spectrum Analyzer:} Instrument for analyzing signal strength versus the frequency.
         \item \textbf{SWR (Standing Wave Ratio):} A measure of impedance match in antenna systems.
      \item \textbf{Directional Wattmeter:} Instrument for measurement of RF power.
       \item \textbf{Prescaler:} A device to divide frequency before measurement.
         \item  \textbf{Aliasing:} Erroneous digital display due to an insufficient sampling rate.
    \item \textbf{Vector Network Analyzer:} Instrument for measuring both amplitude and phase of signals.
         \item \textbf{Probes:} Specialized equipment for capturing the electrical signal.
\end{itemize}
\subsection*{Practice Questions}
\begin{enumerate}
   \item Which of the following limits the highest frequency signal that can be accurately displayed on a digital oscilloscope?
    \begin{enumerate}
    \item \textbf{A. Sampling rate of the analog-to-digital converter}
        \item  Analog-to-digital converter reference frequency
      \item  Q of the circuit
      \item  All these choices are correct
    \end{enumerate}
    \textcolor{myred}{Explanation:}
       The Nyquist-Shannon sampling theorem dictates the signal needs to be samples at twice the maximum frequency in order to be accurately reproduced, thus limiting the maximum frequency.
        
   \item Which of the following parameters does a spectrum analyzer display on the vertical and horizontal axes?
       \begin{enumerate}
         \item  Signal amplitude and time
    \item \textbf{B. Signal amplitude and frequency}
      \item  SWR and frequency
         \item  SWR and time
       \end{enumerate}
       \textcolor{myred}{Explanation:}
        Spectrum analyzer is designed to display the relationship between signal amplitude with respect to frequency.
    
    \item Which of the following test instruments is used to display spurious signals and/or intermodulation distortion products generated by an SSB transmitter?
        \begin{enumerate}
         \item  Differential resolver
      \item \textbf{B. Spectrum analyzer}
       \item  Logic analyzer
        \item  Network analyzer
      \end{enumerate}
    \textcolor{myred}{Explanation:}
      Spectrum analyzer can visualize the spurious output of an SSB signal by displaying all frequencies transmitted.
   
      \item How is compensation of an oscilloscope probe performed?
        \begin{enumerate}
       \item \textbf{A. A square wave is displayed, and the probe is adjusted until the horizontal portions of the displayed wave are as nearly flat as possible}
       \item  A high frequency sine wave is displayed, and the probe is adjusted for maximum amplitude
      \item  A frequency standard is displayed, and the probe is adjusted until the deflection time is accurate
    \item  A DC voltage standard is displayed, and the probe is adjusted until the displayed voltage is accurate
        \end{enumerate}
    \textcolor{myred}{Explanation:}
      An oscilloscope probe is compensated to accurately reproduce square waves, and can be performed by looking at the flat portion of the square wave.
        
    \item What is the purpose of using a prescaler with a frequency counter?
      \begin{enumerate}
      \item  Amplify low-level signals for more accurate counting
     \item  Multiply a higher frequency signal so a low-frequency counter can display the operating frequency
   \item  Prevent oscillation in a low-frequency counter circuit
       \item \textbf{D. Reduce the signal frequency to within the counter's operating range}
        \end{enumerate}
          \textcolor{myred}{Explanation:}
        A prescaler is used to divide high frequencies down to a range suitable for the counter's measuring limits.
        
   \item What is the effect of aliasing on a digital oscilloscope when displaying a waveform?
      \begin{enumerate}
      \item \textbf{A. A false, jittery low-frequency version of the waveform is displayed}
        \item  The waveform DC offset will be inaccurate
      \item  Calibration of the vertical scale is no longer valid
         \item  Excessive blanking occurs, which prevents display of the waveform
        \end{enumerate}
          \textcolor{myred}{Explanation:}
    When digital scopes can not sample signal at more than twice the maximum frequency (Nyquist Shannon theorem), aliasing errors generate lower frequencies that are not present in the original signal.
   
    \item Which of the following is an advantage of using an antenna analyzer compared to an SWR bridge?
      \begin{enumerate}
          \item  Antenna analyzers automatically tune your antenna for resonance
       \item \textbf{B. Antenna analyzers compute SWR and impedance automatically}
      \item  Antenna analyzers display a time-varying representation of the modulation envelope
       \item  All these choices are correct
        \end{enumerate}
       \textcolor{myred}{Explanation:}
    Unlike SWR bridges, antenna analyzers can automatically compute the SWR values from measured impedance at the antenna feedpoint.
    
    \item Which of the following is used to measure SWR?
        \begin{enumerate}
       \item  Directional wattmeter
     \item  Vector network analyzer
        \item  Antenna analyzer
       \item \textbf{D. All these choices are correct}
        \end{enumerate}
          \textcolor{myred}{Explanation:}
        SWR can be measured by a directional wattmeter, antenna analyzer, or network analyzer.

   \item Which of the following is good practice when using an oscilloscope probe?
        \begin{enumerate}
        \item \textbf{A. Minimize the length of the probe's ground connection}
       \item  Never use a high-impedance probe to measure a low-impedance circuit
         \item  Never use a DC-coupled probe to measure an AC circuit
     \item  All these choices are correct
     \end{enumerate}
    \textcolor{myred}{Explanation:}
   Long ground lead creates noise and inaccuracy in measurements, therefore the probe's ground connections should always be as short as possible.

     \item Which trigger mode is most effective when using an oscilloscope to measure a linear power supply's output ripple?
       \begin{enumerate}
       \item  Single-shot
         \item  Edge
      \item  Level
       \item \textbf{D. Line}
        \end{enumerate}
         \textcolor{myred}{Explanation:}
        Line triggering will start the scope's time base in sync with the 50/60Hz frequency of AC power, useful for viewing ripple frequency of power supplies.
     
      \item Which of the following can be measured with an antenna analyzer?
         \begin{enumerate}
         \item  Velocity factor
       \item  Cable length
      \item  Resonant frequency of a tuned circuit
      \item \textbf{D. All these choices are correct}
    \end{enumerate}
        \textcolor{myred}{Explanation:}
       Antenna analyzer can be used to measure velocity factor of transmission lines, the cable length as well as the resonant frequencies of tuned circuits.

\end{enumerate}

% ------------------------------------------------------------------------
% SECTION E4B: Measurement technique and limitations
% ------------------------------------------------------------------------
\section{E4B: Measurement Technique and Limitations - Knowing your Tools!}

\subsection*{Understanding the Basics}
In this section, we will be discussing how to use our tools and their limitations. We will learn that for accurate measurement, the \textcolor{myblue}{\textbf{instrument accuracy and performance limitations}} must be considered. We will delve into details of  \textcolor{myblue}{\textbf{probes}}, which are critical accessories for signal measurements and proper techniques are required to \textcolor{myblue}{\textbf{minimize errors}}. We will be introduced with measurement of \textcolor{myblue}{\textbf{Q}} values in components and circuit and understand the importance of \textcolor{myblue}{\textbf{instrument calibration}}. Finally you will understand what are \textcolor{myblue}{\textbf{S parameters}}, a special type of measurement in RF world and become familiar with \textcolor{myblue}{\textbf{vector network analyzers}}, one of the most useful tools in design and analysis of radio circuits.

\mybox{mygreen}{
  \textbf{Fun Fact}: Did you know that the ‘S’ in ‘S-parameters’ stands for ‘scattering’? These parameters relate the input and output signals of an electrical network to understand how much power is scattered and reflected rather than being fully utilized!
  }

\subsection*{Key Concepts for the Questions}
\begin{itemize}
     \item \textbf{Frequency Counter:} A device used to measure frequency of signals.
    \item \textbf{Voltmeter Sensitivity:} Indication of the voltmeter's internal impedance.
       \item \textbf{S-parameters:} Mathematical description of how RF signals propagate though a device or a network, with both magnitude and phase components.
    \item \textbf{Vector Network Analyzer (VNA):} An instrument that can measure S-parameters of networks.
    \item \textbf{Load:} An electrical device that provides load on a signal.
    \item \textbf{Q:} Quality factor. How close an element is to an ideal reactive component.
     \item  \textbf{Resonance:} When reactances are equal in magnitude but opposite in phase.

\end{itemize}

\subsection*{Practice Questions}
\begin{enumerate}
     \item Which of the following factors most affects the accuracy of a frequency counter?
       \begin{enumerate}
      \item  Input attenuator accuracy
        \item \textbf{B. Time base accuracy}
        \item  Decade divider accuracy
      \item  Temperature coefficient of the logic
        \end{enumerate}
    \textcolor{myred}{Explanation:}
        Frequency counters use a crystal time-base for its measurements, which directly affects the accuracy.
   
    \item What is the significance of voltmeter sensitivity expressed in ohms per volt?
     \begin{enumerate}
      \item \textbf{A. The full scale reading of the voltmeter multiplied by its ohms per volt rating is the input impedance of the voltmeter}
      \item  The reading in volts multiplied by the ohms per volt rating will determine the power drawn by the device under test
      \item  The reading in ohms divided by the ohms per volt rating will determine the voltage applied to the circuit
     \item  The full scale reading in amps divided by ohms per volt rating will determine the size of shunt needed
        \end{enumerate}
    \textcolor{myred}{Explanation:}
    The full-scale reading multiplied by its sensitivity provides the input impedance of the device under test.

   \item Which S parameter is equivalent to forward gain?
        \begin{enumerate}
         \item  S11
         \item  S12
      \item \textbf{C. S21}
      \item  S22
    \end{enumerate}
         \textcolor{myred}{Explanation:}
     S21 defines how much signal that enters at port 1 is present at output port 2. Therefore, it is forward gain.

    \item Which S parameter represents input port return loss or reflection coefficient (equivalent to VSWR)?
      \begin{enumerate}
        \item \textbf{A. S11}
         \item  S12
        \item  S21
        \item  S22
    \end{enumerate}
        \textcolor{myred}{Explanation:}
     S11 represents the magnitude of the reflected wave when power is incident at port 1, and is often called reflection coefficient.

     \item What three test loads are used to calibrate an RF vector network analyzer?
    \begin{enumerate}
        \item  50 ohms, 75 ohms, and 90 ohms
    \item \textbf{B. Short circuit, open circuit, and 50 ohms}
        \item  Short circuit, open circuit, and resonant circuit
         \item  50 ohms through 1/8 wavelength, 1/4 wavelength, and 1/2 wavelength of coaxial cable
        \end{enumerate}
        \textcolor{myred}{Explanation:}
      Open, Short and known impedance such as a 50 Ohm load, is used for VNA calibration.

     \item How much power is being absorbed by the load when a directional power meter connected between a transmitter and a terminating load reads 100 watts forward power and 25 watts reflected power?
        \begin{enumerate}
        \item  100 watts
       \item  125 watts
      \item  112.5 watts
      \item \textbf{D. 75 watts}
        \end{enumerate}
      \textcolor{myred}{Explanation:}
       The absorbed power by load is a difference of forward power and reflected power, thus in this case 100 - 25 = 75 watts is absorbed by the load.
        
       \item What do the subscripts of S parameters represent?
       \begin{enumerate}
       \item \textbf{A. The port or ports at which measurements are made}
        \item  The relative time between measurements
       \item  Relative quality of the data
        \item  Frequency order of the measurements
      \end{enumerate}
       \textcolor{myred}{Explanation:}
       The subscripts of an S-parameters refers to port numbers. For instance S21 indicates a two port measurement from port 1 to port 2.
        
      \item Which of the following can be used to determine the Q of a series-tuned circuit?
     \begin{enumerate}
       \item  The ratio of inductive reactance to capacitive reactance
       \item  The frequency shift
       \item \textbf{C. The bandwidth of the circuit's frequency response}
       \item  The resonant frequency of the circuit
        \end{enumerate}
       \textcolor{myred}{Explanation:}
    The Q of a circuit can be determined by measuring the bandwidth of its resonant curve. Lower bandwidth implies higher Q value.

        \item Which of the following can be measured by a two-port vector network analyzer?
      \begin{enumerate}
      \item  Phase noise
       \item \textbf{B. Filter frequency response}
      \item  Pulse rise time
        \item  Forward power
        \end{enumerate}
        \textcolor{myred}{Explanation:}
         A network analyzer measures signal transmission across two port. A filter performance can be measured by a two port measurement.
        
     \item Which of the following methods measures intermodulation distortion in an SSB transmitter?
    \begin{enumerate}
       \item  Modulate the transmitter using two RF signals having non-harmonically related frequencies and observe the RF output with a spectrum analyzer
         \item \textbf{B. Modulate the transmitter using two AF signals having non-harmonically related frequencies and observe the RF output with a spectrum analyzer}
    \item  Modulate the transmitter using two AF signals having harmonically related frequencies and observe the RF output with a peak reading wattmeter
       \item  Modulate the transmitter using two RF signals having harmonically related frequencies and observe the RF output with a logic analyzer
        \end{enumerate}
     \textcolor{myred}{Explanation:}
      When using two non harmonically related frequencies, a spectral analysis of the output signal will show the presence of any intermodulation distortion products. This testing is typically done at audio frequencies.

    \item Which of the following can be measured with a vector network analyzer?
        \begin{enumerate}
        \item  Input impedance
        \item  Output impedance
       \item  Reflection coefficient
        \item \textbf{D. All these choices are correct}
        \end{enumerate}
         \textcolor{myred}{Explanation:}
        A vector network analyzer can measure all these parameters as it provides the complete picture of magnitude and phase of signals transmitted through the measurement port.

\end{enumerate}


% ------------------------------------------------------------------------
% SECTION E4C: Receiver Performance
% ------------------------------------------------------------------------
\section{E4C: Receiver Performance - Decoding the Signals!}

\subsection*{Understanding the Basics}
Get ready to explore the art of receiving radio waves, where we examine what it means for your receiver to be top-notch! This section will explore several concepts involved in \textcolor{myblue}{\textbf{receiver performance}}, such as \textcolor{myblue}{\textbf{phase noise}} and \textcolor{myblue}{\textbf{noise floor}}, which determines how well a receiver can decode signals even when weak or corrupted. You'll learn how to reject unwanted signals, described with  \textcolor{myblue}{\textbf{image rejection}}. We will also explore the \textcolor{myblue}{\textbf{minimum detectable signal (MDS)}} limits of your receiver, what determines the weakest signal that can be properly received. We'll get to know methods for  \textcolor{myblue}{\textbf{increasing signal-to-noise ratio and dynamic range}} for better signal reception. Concepts like \textcolor{myblue}{\textbf{noise figure}} and \textcolor{myblue}{\textbf{reciprocal mixing}} and their effects on receiver performance will be presented. Finally, we will cover how \textcolor{myblue}{\textbf{selectivity}} is related to filter characteristics, and will discuss how \textcolor{myblue}{\textbf{SDR non-linearity}} and \textcolor{myblue}{\textbf{use of attenuators at low frequencies}} can have an impact on reception.

\mybox{mygreen}{
  \textbf{Fun Fact}: Did you know that the noise in a receiver is not all bad?  In fact, you can actually measure temperature of your antenna by measuring the noise it receives from the space!
}

\subsection*{Key Concepts for the Questions}
\begin{itemize}
    \item \textbf{Phase Noise:} Random variations in phase in the oscillator signal.
     \item \textbf{Noise Floor:}  The level of inherent noise in a receiver.
     \item  \textbf{Image Rejection:} The ability of a receiver to reject unwanted image frequencies.
     \item  \textbf{Minimum Detectable Signal (MDS):} The minimum signal level that can be detected above the noise floor.
        \item \textbf{Signal-to-Noise Ratio:} A measure of signal strength with respect to the noise in the background.
        \item  \textbf{Dynamic Range:} The range of signal levels a receiver can accurately receive.
    \item \textbf{Noise Figure:} A measure of noise added by receiver circuit itself.
    \item \textbf{Reciprocal Mixing:} A form of interference generated by local oscillator phase noise.
    \item \textbf{Selectivity:} Ability of a filter to block signals far from designed frequency.
     \item \textbf{SDR non-linearity:} How a Software defined radio's non ideal circuits can create distortion to the signal.
    
\end{itemize}

\subsection*{Practice Questions}
\begin{enumerate}
     \item What is an effect of excessive phase noise in an SDR receiver's master clock oscillator?
       \begin{enumerate}
          \item  It limits the receiver's ability to receive strong signals
          \item  It can affect the receiver's frequency calibration
         \item  It decreases the receiver's third-order intercept point
        \item \textbf{D. It can combine with strong signals on nearby frequencies to generate interference}
       \end{enumerate}
         \textcolor{myred}{Explanation:}
     High phase noise generates spurious signals that may interfere with other signals.

     \item Which of the following receiver circuits can be effective in eliminating interference from strong out-of-band signals?
      \begin{enumerate}
         \item \textbf{A. A front-end filter or preselector}
        \item  A narrow IF filter
         \item  A notch filter
        \item  A properly adjusted product detector
    \end{enumerate}
    \textcolor{myred}{Explanation:}
    Front end filters block out-of-band signals before amplification, protecting circuits from overload.
       
      \item What is the term for the suppression in an FM receiver of one signal by another stronger signal on the same frequency?
        \begin{enumerate}
        \item  Desensitization
       \item  Cross-modulation interference
    \item \textbf{C. Capture effect}
     \item  Frequency discrimination
     \end{enumerate}
        \textcolor{myred}{Explanation:}
       Capture effect is when the stronger FM signal suppress the weaker FM signal and renders it inaudible.
        
        \item What is the noise figure of a receiver?
      \begin{enumerate}
     \item  The ratio of atmospheric noise to phase noise
    \item  The ratio of the noise bandwidth in hertz to the theoretical bandwidth of a resistive network
       \item  The ratio in dB of the noise generated in the receiver to atmospheric noise
      \item \textbf{D. The ratio in dB of the noise generated by the receiver to the theoretical minimum noise}
     \end{enumerate}
      \textcolor{myred}{Explanation:}
        Noise figure defines the difference between thermal noise level and noise added by receiver.

    \item What does a receiver noise floor of -174 dBm represent?
        \begin{enumerate}
       \item  The receiver noise is 6 dB above the theoretical minimum
   \item \textbf{B. The theoretical noise in a 1 Hz bandwidth at the input of a perfect receiver at room temperature}
   \item  The noise figure of a 1 Hz bandwidth receiver
        \item  The receiver noise is 3 dB above theoretical minimum
    \end{enumerate}
   \textcolor{myred}{Explanation:}
     -174 dBm indicates the absolute noise level at input of receiver with 1 Hz bandwidth at room temperature, without taking in account the performance of receiver.

    \item How much does increasing a receiver's bandwidth from 50 Hz to 1,000 Hz increase the receiver's noise floor?
    \begin{enumerate}
      \item  3 dB
        \item  5 dB
         \item  10 dB
       \item \textbf{D. 13 dB}
    \end{enumerate}
        \textcolor{myred}{Explanation:}
       As the bandwidth increases 20*log(1000/50) = 13dB more noise will be captured in the receiver.
     
   \item What does the MDS of a receiver represent?
        \begin{enumerate}
        \item  The meter display sensitivity
       \item \textbf{B. The minimum discernible signal}
        \item  The modulation distortion specification
       \item  The maximum detectable spectrum
       \end{enumerate}
       \textcolor{myred}{Explanation:}
          Minimum discernable signal (MDS) defines the weakest signal detectable by the receiver.
          
   \item An SDR receiver is overloaded when input signals exceed what level?
        \begin{enumerate}
           \item  One-half of the maximum sample rate
           \item  One-half of the maximum sampling buffer size
       \item  The maximum count value of the analog-to-digital converter
      \item \textbf{D. The reference voltage of the analog-to-digital converter}
       \end{enumerate}
      \textcolor{myred}{Explanation:}
        When the magnitude of the input signals exceeds the analog-to-digital reference voltage, the signal will be clipped and thus become distorted.

    \item Which of the following choices is a good reason for selecting a high IF for a superheterodyne HF or VHF communications receiver?
        \begin{enumerate}
         \item  Fewer components in the receiver
      \item  Reduced drift
       \item \textbf{C. Easier for front-end circuitry to eliminate image responses}
        \item  Improved receiver noise figure
       \end{enumerate}
       \textcolor{myred}{Explanation:}
         A high Intermediate Frequency (IF) simplifies image rejection filter and performance.

     \item What is an advantage of having a variety of receiver bandwidths from which to select?
      \begin{enumerate}
        \item  The noise figure of the RF amplifier can be adjusted to match the modulation type, thus increasing receiver sensitivity
       \item  Receiver power consumption can be reduced when wider bandwidth is not required
         \item \textbf{C. Receive bandwidth can be set to match the modulation bandwidth, maximizing signal-to-noise ratio and minimizing interference}
         \item  Multiple frequencies can be received simultaneously if desired
        \end{enumerate}
          \textcolor{myred}{Explanation:}
         Selecting appropriate bandwidth improves the ratio of signal over noise for optimal sensitivity and reducing interference from adjacent signals.

  \item Why does input attenuation reduce receiver overload on the lower frequency HF bands with little or no impact on signal-to-noise ratio?
     \begin{enumerate}
      \item  The attenuator has a low-pass filter to increase the strength of lower frequency signals
         \item  The attenuator has a noise filter to suppress interference
       \item  Signals are attenuated separately from the noise
        \item \textbf{D. Atmospheric noise is generally greater than internally generated noise even after attenuation}
        \end{enumerate}
       \textcolor{myred}{Explanation:}
    HF bands especially at low frequencies have very high level of atmospheric noises. Input attenuator will reduce both the signal and the atmospheric noise and therefore has minimal impact in the signal to noise ratio.

        \item How does a narrow-band roofing filter affect receiver performance?
       \begin{enumerate}
     \item  It improves sensitivity by reducing front-end noise
      \item  It improves intelligibility by using low Q circuitry to reduce ringing
    \item \textbf{C. It improves blocking dynamic range by attenuating strong signals near the receive frequency}
        \item  All these choices are correct
    \end{enumerate}
     \textcolor{myred}{Explanation:}
        A roofing filter blocks strong adjacent signals, which increase dynamic range by reducing desensitization of receivers.
        
         \item What is reciprocal mixing?
       \begin{enumerate}
         \item  Two out-of-band signals mixing to generate an in-band spurious signal
          \item  In-phase signals cancelling in a mixer resulting in loss of receiver sensitivity
         \item  Two digital signals combining from alternate time slots
        \item \textbf{D. Local oscillator phase noise mixing with adjacent strong signals to create interference to desired signals}
       \end{enumerate}
      \textcolor{myred}{Explanation:}
     Reciprocal mixing creates interference by mixing phase noise with adjacent strong signals.

      \item What is the purpose of the receiver IF Shift control?
       \begin{enumerate}
        \item  To permit listening on a different frequency from the transmitting frequency
        \item  To change frequency rapidly
      \item \textbf{C. To reduce interference from stations transmitting on adjacent frequencies}
        \item  To tune in stations slightly off frequency without changing the transmit frequency
    \end{enumerate}
         \textcolor{myred}{Explanation:}
         IF shift helps reduce adjacent channel interferences by tuning the signal at the side of IF filter.

\end{enumerate}

% ------------------------------------------------------------------------
% SECTION E4D: Receiver Performance Characteristics
% ------------------------------------------------------------------------
\section{E4D: Receiver Performance Characteristics - Fine Tuning your Reception!}
\subsection*{Understanding the Basics}
In this section, we will continue to study receiver parameters, but focus on signal dynamics! This chapter is all about \textcolor{myblue}{\textbf{receiver performance characteristics}}. We'll start with \textcolor{myblue}{\textbf{dynamic range}} and how to measure it, a parameter that shows the ability of receiver to capture both weak and strong signals. We will delve into types of interference such as  \textcolor{myblue}{\textbf{intermodulation and cross-modulation interference}}, and how a receiver is affected by the presence of these unintended signals. We’ll also look at  \textcolor{myblue}{\textbf{third-order intercept}}, an important measure of a receiver's ability to withstand strong signals without generating distortion. We will study concepts such as \textcolor{myblue}{\textbf{desensitization}}, how the strong signals can reduce the sensitivity of the receiver, and methods to alleviate the problems with  \textcolor{myblue}{\textbf{preselectors}},. We also study receiver \textcolor{myblue}{\textbf{sensitivity}} and the concept of \textcolor{myblue}{\textbf{link margin}}.

\mybox{mygreen}{
  \textbf{Fun Fact}: Did you know that many of the methods used in radio receivers to reduce interference and distortion were invented by amateur radio operators? Such as direct conversion receivers or regenerative circuits. These innovations have greatly impacted how we build communication devices.
  }

\subsection*{Key Concepts for the Questions}
\begin{itemize}
       \item \textbf{Dynamic Range:} A measure of how well a receiver handles both strong and weak signals at the same time.
     \item \textbf{Blocking Dynamic Range:}  The level of the unwanted signal that will cause gain compression by 1dB.
    \item \textbf{Intermodulation:} When mixing of two or more signals create interference.
         \item \textbf{Cross-modulation:} Modulation of one signal by other interfering signal.
    \item \textbf{Third-Order Intercept Point:}  A measure of a receiver's ability to handle strong signals without distortion.
    \item \textbf{Desensitization:} The reduction in receiver sensitivity caused by strong adjacent signals.
    \item \textbf{Preselector:} Filters placed before amplifier stages to minimize unwanted signals.
    \item \textbf{Sensitivity:} A measure of a receivers ability to pick up weak signals.
       \item \textbf{Link Margin:}  The difference between received power and power needed for signal decoding.
       \item \textbf{Feed-line loss:} Signal loss within the cables used for feeding signals to the antennas.

\end{itemize}

\subsection*{Practice Questions}
\begin{enumerate}
      \item What is meant by the blocking dynamic range of a receiver?
    \begin{enumerate}
       \item \textbf{A. The difference in dB between the noise floor and the level of an incoming signal that will cause 1 dB of gain compression}
        \item  The minimum difference in dB between the levels of two FM signals that will cause one signal to block the other
        \item  The difference in dB between the noise floor and the third-order intercept point
        \item  The minimum difference in dB between two signals which produce third-order intermodulation products greater than the noise floor
      \end{enumerate}
    \textcolor{myred}{Explanation:}
    Blocking dynamic range defines the signal power at which gain compression of 1 dB starts.
     
        \item Which of the following describes problems caused by poor dynamic range in a receiver?
        \begin{enumerate}
        \item \textbf{A. Spurious signals caused by cross modulation and desensitization from strong adjacent signals}
       \item  Oscillator instability requiring frequent retuning and loss of ability to recover the opposite sideband
          \item  Poor weak signal reception caused by insufficient local oscillator injection
         \item  Oscillator instability and severe audio distortion of all but the strongest received signals
        \end{enumerate}
       \textcolor{myred}{Explanation:}
     Poor dynamic range can result in cross modulation and desensitization when strong adjacent signals are present.

    \item What creates intermodulation interference between two repeaters in close proximity?
     \begin{enumerate}
      \item  The output signals cause feedback in the final amplifier of one or both transmitters
       \item \textbf{B. The output signals mix in the final amplifier of one or both transmitters}
    \item  The input frequencies are harmonically related
     \item  The output frequencies are harmonically related
        \end{enumerate}
       \textcolor{myred}{Explanation:}
      Intermodulation is a non-linear mixing product generated by power amplifiers.
      
   \item Which of the following is used to reduce or eliminate intermodulation interference in a repeater caused by a nearby transmitter?
        \begin{enumerate}
           \item  A band-pass filter in the feed line between the transmitter and receiver
        \item \textbf{B. A properly terminated circulator at the output of the repeater's transmitter}
       \item  Utilizing a Class C final amplifier
          \item  Utilizing a Class D final amplifier
        \end{enumerate}
        \textcolor{myred}{Explanation:}
     A circulator provides isolation between transmitter output and receiver input ports and reduces intermodulation distortions.
        
     \item What transmitter frequencies would create an intermodulation-product signal in a receiver tuned to 146.70 MHz when a nearby station transmits on 146.52 MHz?
       \begin{enumerate}
       \item \textbf{A. 146.34 MHz and 146.61 MHz}
    \item  146.88 MHz and 146.34 MHz
     \item  146.10 MHz and 147.30 MHz
     \item  146.30 MHz and 146.90 MHz
     \end{enumerate}
       \textcolor{myred}{Explanation:}
   Intermod products can be calculated with (2*146.52)-146.70 = 146.34 and (2*146.70)-146.52 = 146.88.  Of the choices, 146.34 and 146.61 produce the 3rd order IMD
        
       \item What is the term for the reduction in receiver sensitivity caused by a strong signal near the received frequency?
         \begin{enumerate}
        \item  Reciprocal mixing
       \item  Quieting
       \item \textbf{C. Desensitization}
        \item  Cross modulation interference
       \end{enumerate}
         \textcolor{myred}{Explanation:}
       Desensitization occurs when a strong adjacent frequency signal prevents a receiver from detecting the weak signals.

   \item Which of the following reduces the likelihood of receiver desensitization?
        \begin{enumerate}
         \item \textbf{A. Insert attenuation before the first RF stage}
         \item  Raise the receiver's IF frequency
        \item  Increase the receiver's front-end gain
         \item  Switch from fast AGC to slow AGC
     \end{enumerate}
        \textcolor{myred}{Explanation:}
      Adding an attenuator before the receiver will reduce both the strong and weak signals preventing desensitization of sensitive front end circuit components.

    \item What causes intermodulation in an electronic circuit?
        \begin{enumerate}
     \item  Negative feedback
        \item  Lack of neutralization
       \item \textbf{C. Nonlinear circuits or devices}
        \item  Positive feedback
        \end{enumerate}
          \textcolor{myred}{Explanation:}
        Nonlinear behavior in a component causes the generation of new signals which are linear combinations of the original ones and called intermodulation.
        
    \item What is the purpose of the preselector in a communications receiver?
      \begin{enumerate}
      \item  To store frequencies that are often used
       \item  To provide broadband attenuation before the first RF stage to prevent intermodulation
      \item \textbf{C. To increase the rejection of signals outside the band being received}
       \item  To allow selection of the optimum RF amplifier device
        \end{enumerate}
        \textcolor{myred}{Explanation:}
          Preselector filters are tuned to reduce the unwanted signals outside the intended band for reducing interference, therefore improving signal selection.
   
    \item What does a third-order intercept level of 40 dBm mean with respect to receiver performance?
         \begin{enumerate}
       \item  Signals less than 40 dBm will not generate audible third-order intermodulation products
        \item \textbf{B. The receiver can tolerate signals up to 40 dB above the noise floor without producing third order intermodulation products}
          \item  A pair of 40 dBm input signals will theoretically generate a third-order intermodulation product that has the same output amplitude as either of the input signals
       \item  A pair of 1 mW input signals will produce a third-order intermodulation product that is 40 dB stronger than the input signal
       \end{enumerate}
     \textcolor{myred}{Explanation:}
      Third order intermodulation points specify how a device will start generating unwanted signals with respect to the fundamental signal level.
        
      \item Why are odd-order intermodulation products, created within a receiver, of particular interest compared to other products?
    \begin{enumerate}
    \item \textbf{A. Odd-order products of two signals in the band being received are also likely to be within the band}
       \item  Odd-order products are more likely to overload the IF filters
        \item  Odd-order products are an indication of poor image rejection
        \item  Odd-order intermodulation produces three products for every input signal within the band of interest
   \end{enumerate}
        \textcolor{myred}{Explanation:}
      The frequencies of 3rd order intermodulation distortion products are located close to the desired signals and therefore is harder to filter out than other distortions, as they fall in the range of the receiver.
        
         \item What is the link margin in a system with a transmit power level of 10 W (+40 dBm), a system antenna gain of 10 dBi, a cable loss of 3 dB, a path loss of 136 dB, a receiver minimum discernable signal of -103 dBm, and a required signal-to-noise ratio of 6 dB?
       \begin{enumerate}
     \item  -8dB
        \item  -14dB
        \item \textbf{C. +8dB}
        \item  +14dB
      \end{enumerate}
   \textcolor{myred}{Explanation:}
     The power at the input of receiver will be 40dBm+10dB-3dB-136dB= -89dBm. The required power at the input is -103dBm+6dB=-97dBm. The margin is the received power vs the required, therefore -89-(-97)=8dBm.

   \item What is the received signal level with a transmit power of 10 W (+40 dBm), a transmit antenna gain of 6 dBi, a receive antenna gain of 3 dBi, and a path loss of 100 dB?
       \begin{enumerate}
     \item \textbf{A. -51 dBm}
      \item  -54 dBm
       \item  -57 dBm
        \item  -60 dBm
       \end{enumerate}
   \textcolor{myred}{Explanation:}
    The received signal is 40dBm +6dBi + 3dBi - 100dB = -51dBm.
     
        \item What power level does a receiver minimum discernible signal of -100 dBm represent?
        \begin{enumerate}
      \item  100 microwatts
      \item  0.1 microwatt
     \item  0.001 microwatts
        \item \textbf{D. 0.1 picowatts}
       \end{enumerate}
    \textcolor{myred}{Explanation:}
      -100 dBm is approximately equivalent to 0.1 pico watts, which can be calculated by converting -100dBm into watt.
\end{enumerate}


% ------------------------------------------------------------------------
% SECTION E4E: Noise and Interference
% ------------------------------------------------------------------------
\section{E4E: Noise and Interference - Taming the Static!}

\subsection*{Understanding the Basics}
Let's dive into the challenging, but necessary topic of \textcolor{myblue}{\textbf{noise and interference}}, and how to deal with them! In this section, we will study different sources of \textcolor{myblue}{\textbf{external RF interference}}, which can come from a wide range of external sources. Then we will dive into the world of \textcolor{myblue}{\textbf{electrical and computer noise}} including line noises that might interfere with your reception. We will then examine the nature of \textcolor{myblue}{\textbf{line noise}} that may originate from our power circuits. Then we will study methods to improve the signal quality with \textcolor{myblue}{\textbf{DSP filtering and noise reduction}}. We also explore what \textcolor{myblue}{\textbf{common-mode current}} is, and finally discuss various methods to reduce the impact of interference, such as \textcolor{myblue}{\textbf{surge protectors}} and \textcolor{myblue}{\textbf{single point ground panels}}.

\mybox{mygreen}{
  \textbf{Fun Fact}: Did you know that even the smallest crackling sound from household appliances, powerlines, and many electronic devices can be a form of radio frequency noise and interfere with your transmissions!
  }
\subsection*{Key Concepts for the Questions}
\begin{itemize}
        \item \textbf{Automatic Notch Filter (ANF):} A digital filter for removing a constant signal such as a carrier.
     \item  \textbf{Broadband White Noise:} Random noise across a wide range of frequencies.
         \item  \textbf{Impulse Noise:} Short, sudden bursts of energy, like from motors or switches.
         \item \textbf{Noise Blanker:} A circuit that eliminates short impulses of noise.
        \item \textbf{Line Noise:}  Interference caused by unwanted signals in power lines.
         \item  \textbf{Ferrite Chokes:} Passive components used for suppressing common mode signals.
      \item \textbf{Common-Mode Current:} Current that flows equally on all conductors of a cable.
       \item \textbf{Surge Protector:} Devices that help to protect electronic gear from over-voltage spikes
       \item \textbf{Single Point Ground Panel:} Connection of multiple earth points of a station to a single panel.
       \item \textbf{Skin effect:} High-frequency RF current flowing closer to the surface.
      \item \textbf{Switch-mode Power Supplies:} A class of power supplies that use high frequency switching technique to reduce size and weight of power supplies.
\end{itemize}

\subsection*{Practice Questions}
\begin{enumerate}
   \item What problem can occur when using an automatic notch filter (ANF) to remove interfering carriers while receiving CW signals?
    \begin{enumerate}
        \item \textbf{A. Removal of the CW signal as well as the interfering carrier}
        \item  Any nearby signal passing through the DSP system will overwhelm the desired signal
        \item  Excessive ringing
       \item  All these choices are correct
    \end{enumerate}
      \textcolor{myred}{Explanation:}
    An automatic notch filter can remove intended signal as well as unwanted signals.
        
    \item Which of the following types of noise can often be reduced by a digital noise reduction?
       \begin{enumerate}
         \item  Broadband white noise
        \item  Ignition noise
      \item  Power line noise
         \item \textbf{D. All these choices are correct}
       \end{enumerate}
      \textcolor{myred}{Explanation:}
       Digital noise reduction can reduce multiple forms of noises.
       
   \item Which of the following types of noise are removed by a noise blanker?
    \begin{enumerate}
     \item  Broadband white noise
     \item \textbf{B. Impulse noise}
       \item  Hum and buzz
      \item  All these choices are correct
        \end{enumerate}
     \textcolor{myred}{Explanation:}
       A noise blanker detects and eliminates impulse type noise.
   
   \item How can conducted noise from an automobile battery charging system be suppressed?
     \begin{enumerate}
    \item  By installing filter capacitors in series with the alternator leads
        \item  By installing a noise suppression resistor and a blocking capacitor at the battery
       \item  By installing a high-pass filter in series with the radio's power lead and a low-pass filter in parallel with the antenna feed line
     \item \textbf{D. By installing ferrite chokes on the charging system leads}
       \end{enumerate}
      \textcolor{myred}{Explanation:}
      Ferrite chokes block the radio waves through absorption and thereby help prevent it from traveling on power lines.

  \item What is used to suppress radio frequency interference from a line-driven AC motor?
        \begin{enumerate}
          \item  A high-pass filter in series with the motor's power leads
        \item \textbf{B. A brute-force AC-line filter in series with the motor's power leads}
         \item  A bypass capacitor in series with the motor's field winding
        \item  A bypass choke in parallel with the motor's field winding
        \end{enumerate}
   \textcolor{myred}{Explanation:}
         An AC line filter blocks high frequencies by inductors and capacitors before reaching AC motor.

   \item What type of electrical interference can be caused by computer network equipment?
     \begin{enumerate}
     \item  A loud AC hum in the audio output of your station's receiver
       \item  A clicking noise at intervals of a few seconds
         \item \textbf{C. The appearance of unstable modulated or unmodulated signals at specific frequencies}
         \item  A whining-type noise that continually pulses off and on
     \end{enumerate}
     \textcolor{myred}{Explanation:}
    Network equipment may emit high frequency noise on frequencies used for digital communications.

    \item Which of the following can cause shielded cables to radiate or receive interference?
      \begin{enumerate}
          \item  Low inductance ground connections at both ends of the shield
       \item \textbf{B. Common-mode currents on the shield and conductors}
        \item  Use of braided shielding material
     \item  Tying all ground connections to a common point resulting in differential-mode currents in the shield
     \end{enumerate}
        \textcolor{myred}{Explanation:}
         Common mode currents, which travels the length of the shield, can generate strong interference.

     \item What current flows equally on all conductors of an unshielded multiconductor cable?
        \begin{enumerate}
      \item  Differential-mode current
          \item \textbf{B. Common-mode current}
         \item  Reactive current only
       \item  Magnetically-coupled current only
        \end{enumerate}
     \textcolor{myred}{Explanation:}
      Common mode current is when currents on the signal carrying lines are equal in magnitude and direction.

    \item What undesirable effect can occur when using a noise blanker?
        \begin{enumerate}
          \item  Received audio in the speech range might have an echo effect
         \item  The audio frequency bandwidth of the received signal might be compressed
      \item \textbf{C. Strong signals may be distorted and appear to cause spurious emissions}
       \item  FM signals can no longer be demodulated
      \end{enumerate}
       \textcolor{myred}{Explanation:}
       Strong signals can cause a receiver to generate artifacts by its blanking circuits.
       
        \item Which of the following can create intermittent loud roaring or buzzing AC line interference?
        \begin{enumerate}
        \item  Arcing contacts in a thermostatically controlled device
        \item  A defective doorbell or doorbell transformer inside a nearby residence
        \item  A malfunctioning illuminated advertising display
       \item \textbf{D. All these choices are correct}
       \end{enumerate}
        \textcolor{myred}{Explanation:}
       All of these are known culprits for generating power line noise due to arcing on loose connections.
       
    \item What could be the cause of local AM broadcast band signals combining to generate spurious signals on the MF or HF bands?
         \begin{enumerate}
         \item  One or more of the broadcast stations is transmitting an over-modulated signal
          \item \textbf{B. Nearby corroded metal connections are mixing and reradiating the broadcast signals}
         \item  You are receiving skywave signals from a distant station
      \item  Your station receiver IF amplifier stage is overloaded
      \end{enumerate}
   \textcolor{myred}{Explanation:}
      Corrosion is commonly cause non-linear behavior of metal contacts, and cause creation of spurious emissions through intermodulation.
      
      \item What causes interference received as a series of carriers at regular intervals across a wide frequency range?
       \begin{enumerate}
        \item \textbf{A. Switch-mode power supplies}
       \item  Radar transmitters
        \item  Wireless security camera transmitters
      \item  Electric fences
       \end{enumerate}
     \textcolor{myred}{Explanation:}
     Switch mode power supplies with square waveform can generate strong harmonic frequencies in wide frequency ranges.
     
     \item Where should a station AC surge protector be installed?
      \begin{enumerate}
       \item  At the AC service panel
       \item  At an AC outlet
      \item \textbf{C. On the single point ground panel}
        \item  On a ground rod outside the station
        \end{enumerate}
       \textcolor{myred}{Explanation:}
       To provide best protection, the AC surge protector must be connected at the single point ground panel.
         
      \item What is the purpose of a single point ground panel?
       \begin{enumerate}
       \item  Remove AC power in case of a short-circuit
       \item  Prevent common-mode transients in multi-wire systems
         \item  Eliminate air gaps between protected and non-protected circuits
        \item \textbf{D. Ensure all lightning protectors activate at the same time}
        \end{enumerate}
        \textcolor{myred}{Explanation:}
    A single point ground panels ensure that grounding will happen at a single point, preventing current loops in the system and also that all components get connected to the ground at same time to provide best protection in case of lightning.
\end{enumerate}
