\subsection{Total Bandwidth of FM Phone Transmission}
\label{G8B06}

\begin{tcolorbox}[colback=gray!10!white,colframe=black!75!black,title=G8B06]
What is the total bandwidth of an FM phone transmission having 5 kHz deviation and 3 kHz modulating frequency?
\begin{enumerate}[label=\Alph*),noitemsep]
    \item 3 kHz
    \item 5 kHz
    \item 8 kHz
    \item \textbf{16 kHz}
\end{enumerate}
\end{tcolorbox}

\subsubsection{Intuitive Explanation}
Imagine you're at a concert, and the singer is moving around the stage. The singer's movement is like the frequency deviation (5 kHz), and the speed at which they move is like the modulating frequency (3 kHz). The total area the singer covers is like the bandwidth. In FM radio, the total bandwidth is calculated by considering both how much the frequency changes and how fast it changes. So, the total bandwidth is much larger than just the deviation or the modulating frequency alone. In this case, it's 16 kHz, which means the radio signal covers a wide range of frequencies to carry the information.

\subsubsection{Advanced Explanation}
In FM (Frequency Modulation), the total bandwidth \( B \) can be approximated using Carson's rule:
\[
B \approx 2(\Delta f + f_m)
\]
where \( \Delta f \) is the frequency deviation and \( f_m \) is the modulating frequency. Given \( \Delta f = 5 \) kHz and \( f_m = 3 \) kHz, we can calculate the bandwidth as follows:
\[
B \approx 2(5 \text{ kHz} + 3 \text{ kHz}) = 2 \times 8 \text{ kHz} = 16 \text{ kHz}
\]
This formula accounts for the maximum frequency swing due to modulation and the rate at which the frequency changes. The result, 16 kHz, is the total bandwidth required for the FM phone transmission.

% Diagram prompt: Generate a diagram showing the relationship between frequency deviation, modulating frequency, and total bandwidth in FM transmission.