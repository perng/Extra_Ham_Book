\subsection{Minimum Age to Qualify as an Accredited Volunteer Examiner}
\label{G1D10}

\begin{tcolorbox}[colback=gray!10!white,colframe=black!75!black,title=G1D10]
What is the minimum age that one must be to qualify as an accredited Volunteer Examiner?
\begin{enumerate}[label=\Alph*,noitemsep]
    \item 16 years
    \item \textbf{18 years}
    \item 21 years
    \item There is no age limit
\end{enumerate}
\end{tcolorbox}

\subsubsection{Intuitive Explanation}
Alright, imagine you're trying to become a superhero who helps people pass their radio exams. But just like you can't drive a car until you're 16, you can't become a Volunteer Examiner until you're 18. It's like the universe saying, Hey, you need to be a bit older to take on this responsibility! So, if you're 18 or older, you're good to go. If not, you'll have to wait a bit longer to join the team of exam superheroes.

\subsubsection{Advanced Explanation}
The requirement for becoming an accredited Volunteer Examiner (VE) is governed by the Federal Communications Commission (FCC) regulations. According to these regulations, an individual must be at least 18 years old to qualify as a VE. This age requirement ensures that the individual has reached a level of maturity and responsibility necessary to administer and oversee amateur radio examinations. 

The rationale behind this regulation is to maintain the integrity and credibility of the examination process. By setting a minimum age, the FCC ensures that VEs are capable of understanding and adhering to the rules and procedures involved in conducting these exams. This age requirement is consistent with other professional certifications and responsibilities that require a certain level of maturity and experience.

% Diagram prompt: A flowchart showing the age requirements for different roles in amateur radio, including Volunteer Examiner.