\subsection{Remote Control Operation of a US Station from Outside the Country}
\label{G1D05}

\begin{tcolorbox}[colback=gray!10!white,colframe=black!75!black,title=G1D05]
When operating a US station by remote control from outside the country, what license is required of the control operator?
\begin{enumerate}[label=\Alph*)]
    \item \textbf{A US operator/primary station license}
    \item Only an appropriate US operator/primary license and a special remote station permit from the FCC
    \item Only a license from the foreign country, as long as the call sign includes identification of portable operation in the US
    \item A license from the foreign country and a special remote station permit from the FCC
\end{enumerate}
\end{tcolorbox}

\subsubsection{Intuitive Explanation}
Imagine you have a toy car that you can control from your house, but you want to drive it from your friend's house in another country. Even though you're not in your own house, you still need to follow the rules of your house to drive the car. Similarly, if you're controlling a radio station in the US from another country, you still need to follow the US rules and have the right license to do so. It's like saying, Hey, I'm still playing by the rules even if I'm not at home!

\subsubsection{Advanced Explanation}
When operating a US radio station by remote control from outside the United States, the control operator must hold a valid US operator/primary station license. This requirement ensures that the operator is authorized to operate the station in accordance with the Federal Communications Commission (FCC) regulations, regardless of their physical location. The FCC mandates that the operator must have the appropriate license to ensure compliance with US radio operation standards, including frequency usage, power limits, and identification protocols.

The correct answer is \textbf{A}, as it aligns with the FCC's regulations that the control operator must possess a US operator/primary station license. This license serves as the legal authorization to operate the station, even when the control is exercised from a foreign country. The other options either introduce unnecessary additional permits or incorrectly suggest that a foreign license alone would suffice, which is not in line with FCC requirements.

% Prompt for generating a diagram:
% A diagram showing a person in a foreign country remotely controlling a radio station in the US, with a label indicating the required US operator/primary station license.