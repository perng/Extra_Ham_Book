\subsection{Criteria for a Non-US Citizen to be an Accredited Volunteer Examiner}
\label{G1D08}

\begin{tcolorbox}[colback=gray!10!white,colframe=black!75!black,title=G1D08]
Which of the following criteria must be met for a non-US citizen to be an accredited Volunteer Examiner?
\begin{enumerate}[label=\Alph*,noitemsep]
    \item The person must be a resident of the US for a minimum of 5 years
    \item \textbf{The person must hold an FCC granted amateur radio license of General class or above}
    \item The person’s home citizenship must be in ITU region 2
    \item None of these choices is correct; a non-US citizen cannot be a Volunteer Examiner
\end{enumerate}
\end{tcolorbox}

\subsubsection{Intuitive Explanation}
Imagine you're playing a game where you need to be a referee. To be a referee, you need to have a special badge that shows you know the rules really well. Now, if you're not from the country where the game is being played, you still need that special badge to be a referee. In this case, the special badge is an FCC granted amateur radio license of General class or above. So, even if you're not from the US, as long as you have that license, you can be a Volunteer Examiner!

\subsubsection{Advanced Explanation}
To be an accredited Volunteer Examiner (VE) in the United States, certain criteria must be met. For non-US citizens, the primary requirement is to hold an amateur radio license granted by the Federal Communications Commission (FCC) of at least General class. This ensures that the individual has a sufficient understanding of the rules and regulations governing amateur radio operations in the US.

The FCC is the regulatory body responsible for managing the radio spectrum and issuing licenses. The General class license is one of the three primary license classes (Technician, General, and Extra) and requires passing an examination that covers a broad range of topics, including regulations, operating practices, and technical knowledge.

The other options provided in the question are incorrect:
\begin{itemize}
    \item Option A: There is no residency requirement for non-US citizens to become VEs.
    \item Option C: The ITU region of citizenship is irrelevant for this purpose.
    \item Option D: Non-US citizens can indeed become VEs if they meet the licensing requirement.
\end{itemize}

Therefore, the correct answer is B: The person must hold an FCC granted amateur radio license of General class or above.

% Prompt for generating a diagram: A flowchart showing the criteria for becoming a Volunteer Examiner, with a branch for US citizens and non-US citizens, highlighting the FCC license requirement for non-US citizens.