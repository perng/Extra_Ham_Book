\subsection{Band Plan Restrictions in 50.1-50.125 MHz}
\label{G2B08}

\begin{tcolorbox}[colback=gray!10!white,colframe=black!75!black,title=G2B08]
What is the voluntary band plan restriction for US stations transmitting within the 48 contiguous states in the 50.1 MHz to 50.125 MHz band segment?
\begin{enumerate}[label=\Alph*),noitemsep]
    \item \textbf{Only contacts with stations not within the 48 contiguous states}
    \item Only contacts with other stations within the 48 contiguous states
    \item Only digital contacts
    \item Only SSTV contacts
\end{enumerate}
\end{tcolorbox}

\subsubsection*{Intuitive Explanation}
Imagine you’re at a big party, but there’s a rule: you can only talk to people who aren’t from your hometown. That’s kind of what’s happening here! In the 50.1 MHz to 50.125 MHz frequency range, radio stations in the 48 contiguous states can only chat with stations outside those states. It’s like a rule to keep the conversation interesting and diverse!

\subsubsection*{Advanced Explanation}
The 50.1 MHz to 50.125 MHz band segment is part of the 6-meter amateur radio band. The voluntary band plan for this segment restricts US stations within the 48 contiguous states to only communicate with stations outside these states. This restriction helps to minimize interference and ensures that the band is used efficiently for long-distance (DX) communications. 

Mathematically, this can be represented as:
\[
\text{Allowed Contacts} = \{ \text{Stations} \mid \text{Stations} \notin \text{48 contiguous states} \}
\]

This restriction is part of a broader effort to manage the limited radio spectrum and promote international communication. By limiting contacts to stations outside the 48 contiguous states, operators can avoid local congestion and focus on DX operations, which are often more challenging and rewarding.

% Diagram prompt: A map showing the 48 contiguous states and highlighting the restriction on communication with stations outside this region.