\subsection{RACES Training Drills and Tests Frequency}
\label{G2B11}

\begin{tcolorbox}[colback=gray!10!white,colframe=black!75!black,title=G2B11]
How often may RACES training drills and tests be routinely conducted without special authorization?
\begin{enumerate}[label=\Alph*),noitemsep]
    \item No more than 1 hour per month
    \item No more than 2 hours per month
    \item \textbf{No more than 1 hour per week}
    \item No more than 2 hours per week
\end{enumerate}
\end{tcolorbox}

\subsubsection*{Intuitive Explanation}
Imagine you're part of a team that practices for emergencies, like a fire drill at school. You don't want to practice too much and take away from your regular activities, but you also don't want to practice too little and forget what to do. The rules say you can practice for up to 1 hour every week without needing special permission. That's like having a quick review session every week to make sure you're ready if something happens.

\subsubsection*{Advanced Explanation}
RACES (Radio Amateur Civil Emergency Service) training drills and tests are essential for ensuring that amateur radio operators are prepared to assist in emergency communications. The Federal Communications Commission (FCC) and RACES guidelines specify that these drills and tests can be conducted routinely without special authorization for up to 1 hour per week. This frequency strikes a balance between maintaining operational readiness and not overburdening participants with excessive training sessions. 

The rationale behind this regulation is to ensure that operators remain proficient in their skills without requiring frequent special permissions, which could complicate the scheduling and execution of these drills. This weekly limit allows for consistent practice while keeping the training manageable and effective.

% Prompt for diagram: A simple timeline showing weekly 1-hour training sessions over a month could help visualize the frequency.