\subsection{Choosing a Frequency for Initiating a Call}
\label{G2B07}

\begin{tcolorbox}[colback=gray!10!white,colframe=black!75!black,title=G2B07]
Which of the following complies with commonly accepted amateur practice when choosing a frequency on which to initiate a call?
\begin{enumerate}[label=\Alph*),noitemsep]
    \item Listen on the frequency for at least two minutes to be sure it is clear
    \item Identify your station by transmitting your call sign at least 3 times
    \item \textbf{Follow the voluntary band plan}
    \item All these choices are correct
\end{enumerate}
\end{tcolorbox}

\subsubsection{Intuitive Explanation}
Imagine you're at a big party with different rooms for different types of conversations. You wouldn't just barge into a room and start talking without checking if it's the right place, right? Similarly, when you want to start a call on the radio, you need to follow the room rules or the band plan. This plan helps everyone know where to talk so no one steps on each other's toes. It's like a map that keeps the party organized!

\subsubsection{Advanced Explanation}
In amateur radio, the voluntary band plan is a set of guidelines that helps operators choose appropriate frequencies for different modes of communication (e.g., voice, digital, Morse code). These plans are not legally binding but are widely accepted to ensure efficient use of the radio spectrum and minimize interference.

When initiating a call, it is crucial to follow the band plan to avoid disrupting ongoing communications. For example, the 2-meter band (144-148 MHz) has specific segments allocated for FM voice, digital modes, and simplex operations. By adhering to these guidelines, operators can ensure that their transmissions are within the expected frequency ranges for their intended mode of communication.

Additionally, while listening to ensure a frequency is clear (Option A) and identifying your station (Option B) are good practices, they are not sufficient on their own. The band plan provides a structured approach to frequency selection, which is essential for maintaining order in the amateur radio community.

% Prompt for generating a diagram: 
% Diagram showing the 2-meter band with segments labeled for FM voice, digital modes, and simplex operations.