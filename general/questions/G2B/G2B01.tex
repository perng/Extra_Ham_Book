\subsection{Access to Frequencies in Amateur Radio}
\label{G2B01}

\begin{tcolorbox}[colback=gray!10!white,colframe=black!75!black,title=G2B01]
Which of the following is true concerning access to frequencies?
\begin{enumerate}[label=\Alph*]
    \item Nets have priority
    \item QSOs in progress have priority
    \item \textbf{Except during emergencies, no amateur station has priority access to any frequency}
    \item Contest operations should yield to non-contest use of frequencies
\end{enumerate}
\end{tcolorbox}

\subsubsection{Intuitive Explanation}
Imagine you and your friends are at a playground, and there’s only one swing. Everyone wants to use it, but there’s no rule saying who gets to swing first. In amateur radio, frequencies are like that swing. Except during emergencies (like if someone gets hurt on the playground), no one has special rights to use a frequency. Everyone gets a fair chance to communicate. So, the correct answer is that no amateur station has priority access to any frequency, except during emergencies.

\subsubsection{Advanced Explanation}
In amateur radio, the allocation and use of frequencies are governed by regulations to ensure fair and efficient use of the radio spectrum. According to the International Telecommunication Union (ITU) and national regulatory bodies, amateur radio operators must share frequencies and avoid causing harmful interference to other users. 

The principle of \textit{equitable access} dictates that no single amateur station has priority over another, except in emergency situations where immediate communication is necessary to protect life or property. This principle is codified in various regulations, such as the FCC rules in the United States and the Ofcom regulations in the United Kingdom.

Mathematically, the concept can be represented as:
\[
\text{Priority}(f) = 
\begin{cases}
1 & \text{if emergency,} \\
0 & \text{otherwise,}
\end{cases}
\]
where \( \text{Priority}(f) \) denotes the priority level of a station on frequency \( f \).

This ensures that during normal operations, all stations have an equal opportunity to use any frequency, promoting fairness and cooperation within the amateur radio community.

% Prompt for diagram: A diagram showing a frequency spectrum with multiple amateur stations sharing frequencies, with an emergency station highlighted to show priority during emergencies.