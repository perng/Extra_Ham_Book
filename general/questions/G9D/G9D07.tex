\subsection{Log-Periodic Antenna Characteristics}
\label{G9D07}

\begin{tcolorbox}[colback=gray!10!white,colframe=black!75!black,title=G9D07]
Which of the following describes a log-periodic antenna?
\begin{enumerate}[label=\Alph*)]
    \item \textbf{Element length and spacing vary logarithmically along the boom}
    \item Impedance varies periodically as a function of frequency
    \item Gain varies logarithmically as a function of frequency
    \item SWR varies periodically as a function of boom length
\end{enumerate}
\end{tcolorbox}

\subsubsection*{Intuitive Explanation}
Imagine you have a bunch of sticks of different lengths, and you arrange them in a line. The sticks get longer and the spaces between them get bigger in a special way—like how numbers on a ruler get bigger as you move along it. This is what a log-periodic antenna does! The sticks (called elements) and the spaces between them change in a pattern that follows a logarithmic scale. This helps the antenna pick up a wide range of radio signals, kind of like how a multi-tool can do lots of different jobs.

\subsubsection*{Advanced Explanation}
A log-periodic antenna is designed with elements whose lengths and spacings vary logarithmically along the boom. This means that the length of each element and the distance between consecutive elements follow a logarithmic progression. Mathematically, if \( L_n \) is the length of the \( n \)-th element and \( d_n \) is the spacing between the \( n \)-th and \( (n+1) \)-th elements, then:

\[
L_{n+1} = \tau L_n \quad \text{and} \quad d_{n+1} = \tau d_n
\]

where \( \tau \) is the scaling factor, typically less than 1. This logarithmic scaling allows the antenna to operate over a wide frequency range, as each element is resonant at a different frequency. The impedance of the antenna remains relatively constant across this range, which is a key advantage of the log-periodic design.

The other options describe characteristics that are not typical of log-periodic antennas. For example, the impedance of a log-periodic antenna does not vary periodically with frequency, nor does the gain vary logarithmically with frequency. The SWR (Standing Wave Ratio) is not a function of the boom length but rather a measure of how well the antenna is matched to the transmission line.

% Diagram Prompt: Generate a diagram showing the arrangement of elements in a log-periodic antenna, with element lengths and spacings labeled to illustrate the logarithmic progression.