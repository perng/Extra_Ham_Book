\subsection{Primary Use of a Beverage Antenna}
\label{G9D09}

\begin{tcolorbox}[colback=gray!10!white,colframe=black!75!black,title=G9D09]
What is the primary use of a Beverage antenna?
\begin{enumerate}[label=\Alph*),noitemsep]
    \item \textbf{Directional receiving for MF and low HF bands}
    \item Directional transmitting for low HF bands
    \item Portable direction finding at higher HF frequencies
    \item Portable direction finding at lower HF frequencies
\end{enumerate}
\end{tcolorbox}

\subsubsection{Intuitive Explanation}
Imagine you're trying to listen to a radio station that's really far away, like across the ocean. A Beverage antenna is like a super long ear that you can point in the direction of the station to hear it better. It's not for talking back, just for listening, and it works best for those low and medium frequency radio waves that travel long distances.

\subsubsection{Advanced Explanation}
The Beverage antenna, named after its inventor Harold H. Beverage, is primarily used for directional reception in the medium frequency (MF) and low high frequency (HF) bands. This antenna is characterized by its long wire, typically several wavelengths long, which is laid out horizontally close to the ground. The key principle behind its operation is the wave tilt effect, where the antenna captures the vertically polarized component of the incoming radio waves that have been refracted by the ionosphere.

The Beverage antenna is not suitable for transmitting due to its low radiation efficiency and high losses. Its directional receiving capability is achieved by the phase difference between the signals arriving at different points along the wire, which allows it to favor signals coming from one direction while rejecting those from others. This makes it particularly useful for long-distance communication in the MF and low HF bands, where signals can travel thousands of kilometers by skywave propagation.

% Diagram prompt: A diagram showing the layout of a Beverage antenna with labels indicating the direction of incoming radio waves and the orientation of the wire relative to the ground.