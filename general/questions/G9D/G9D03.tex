\subsection{Maximum Radiation Direction of a VHF/UHF Halo Antenna}
\label{G9D03}

\begin{tcolorbox}[colback=gray!10!white,colframe=black!75!black,title=G9D03]
In which direction is the maximum radiation from a VHF/UHF “halo” antenna?
\begin{enumerate}[label=\Alph*,noitemsep]
    \item Broadside to the plane of the halo
    \item Opposite the feed point
    \item \textbf{Omnidirectional in the plane of the halo}
    \item On the same side as the feed point
\end{enumerate}
\end{tcolorbox}

\subsubsection{Intuitive Explanation}
Imagine you’re holding a hula hoop (that’s your halo antenna) and spinning it around your waist. The radio waves it sends out are like the glow from a glow stick—it shines equally in all directions around the hoop, but not up or down. So, the maximum radiation is all around the hoop, not just in one spot. That’s why the answer is “omnidirectional in the plane of the halo.”

\subsubsection{Advanced Explanation}
A VHF/UHF halo antenna is a type of loop antenna that is typically circular in shape. The radiation pattern of such an antenna is determined by the current distribution around the loop. In the case of a halo antenna, the current is uniformly distributed around the loop, leading to an omnidirectional radiation pattern in the plane of the loop. This means that the antenna radiates equally in all directions within the plane of the loop, but not perpendicular to it.

Mathematically, the radiation pattern \( E(\theta, \phi) \) of a small loop antenna can be approximated as:
\[
E(\theta, \phi) = E_0 \sin(\theta)
\]
where \( \theta \) is the angle from the axis perpendicular to the plane of the loop, and \( \phi \) is the azimuthal angle in the plane of the loop. For a halo antenna, the maximum radiation occurs when \( \theta = 90^\circ \), which corresponds to the plane of the loop. This confirms that the radiation is omnidirectional in the plane of the halo.

% Diagram prompt: Generate a diagram showing a circular loop antenna with arrows indicating the omnidirectional radiation pattern in the plane of the loop.