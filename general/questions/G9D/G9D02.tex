\subsection{Feed Point Impedance of an End-Fed Half-Wave Antenna}
\label{G9D02}

\begin{tcolorbox}[colback=gray!10!white,colframe=black!75!black,title=G9D02]
What is the feed point impedance of an end-fed half-wave antenna?
\begin{enumerate}[label=\Alph*,noitemsep]
    \item Very low
    \item Approximately 50 ohms
    \item Approximately 300 ohms
    \item \textbf{Very high}
\end{enumerate}
\end{tcolorbox}

\subsubsection{Intuitive Explanation}
Imagine you're trying to push a swing at just the right moment to keep it going. If you push at the wrong time, it’s hard to get it moving. An end-fed half-wave antenna is like that swing—it’s tricky to feed energy into it because the end of the antenna is where the voltage is highest and the current is lowest. This makes the impedance (the resistance to the flow of energy) very high. So, the correct answer is that the feed point impedance is very high.

\subsubsection{Advanced Explanation}
The feed point impedance of an antenna is determined by the ratio of voltage to current at the feed point. For an end-fed half-wave antenna, the current is at a minimum and the voltage is at a maximum at the end of the antenna. This results in a very high impedance. Mathematically, the impedance \( Z \) is given by:

\[
Z = \frac{V}{I}
\]

where \( V \) is the voltage and \( I \) is the current. Since \( V \) is high and \( I \) is low at the end of the antenna, \( Z \) becomes very high. This is why the correct answer is that the feed point impedance is very high.

% Diagram prompt: Generate a diagram showing the voltage and current distribution along a half-wave antenna, highlighting the high voltage and low current at the end.