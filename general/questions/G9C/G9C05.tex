\subsection{Effect of Boom Length and Directors on Yagi Antenna}
\label{G9C05}

\begin{tcolorbox}[colback=gray!10!white,colframe=black!75!black,title=G9C05]
What is the primary effect of increasing boom length and adding directors to a Yagi antenna?
\begin{enumerate}[label=\Alph*),noitemsep]
    \item \textbf{Gain increases}
    \item Beamwidth increases
    \item Front-to-back ratio decreases
    \item Resonant frequency is lower
\end{enumerate}
\end{tcolorbox}

\subsubsection{Intuitive Explanation}
Imagine you're trying to catch a ball with a net. If you make the net bigger and add more strings (directors), you can catch more balls, right? Similarly, when you make the boom (the long stick) of a Yagi antenna longer and add more directors (the smaller sticks), the antenna can catch more radio waves. This means the antenna becomes better at picking up signals, which we call gain. So, the primary effect is that the gain increases!

\subsubsection{Advanced Explanation}
The Yagi antenna is a directional antenna that consists of a driven element, a reflector, and one or more directors. The boom length and the number of directors directly influence the antenna's performance. 

When the boom length is increased and more directors are added, the antenna's gain increases. This is because the directors help to focus the radio waves in a specific direction, making the antenna more efficient at transmitting and receiving signals in that direction. The gain \( G \) of a Yagi antenna can be approximated by the formula:

\[
G \approx 10 \log_{10} \left( \frac{4 \pi A}{\lambda^2} \right)
\]

where \( A \) is the effective aperture of the antenna and \( \lambda \) is the wavelength of the signal. As the boom length and the number of directors increase, the effective aperture \( A \) increases, leading to a higher gain.

Additionally, the beamwidth of the antenna decreases as the gain increases, which means the antenna becomes more directional. The front-to-back ratio, which is the ratio of the power radiated in the forward direction to the power radiated in the backward direction, typically increases with the addition of directors. The resonant frequency of the antenna is primarily determined by the length of the driven element and is not significantly affected by the boom length or the number of directors.

% Diagram prompt: Generate a diagram showing a Yagi antenna with varying boom lengths and numbers of directors, illustrating the increase in gain and directionality.