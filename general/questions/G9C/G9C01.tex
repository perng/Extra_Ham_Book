\subsection{Increasing Bandwidth of a Yagi Antenna}
\label{G9C01}

\begin{tcolorbox}[colback=gray!10!white,colframe=black!75!black,title=G9C01]
Which of the following would increase the bandwidth of a Yagi antenna?
\begin{enumerate}[label=\Alph*),noitemsep]
    \item \textbf{Larger-diameter elements}
    \item Closer element spacing
    \item Loading coils in series with the element
    \item Tapered-diameter elements
\end{enumerate}
\end{tcolorbox}

\subsubsection*{Intuitive Explanation}
Imagine a Yagi antenna as a team of cheerleaders. The larger the pom-poms (elements), the more energy they can spread out, making the cheer (signal) cover a wider area (bandwidth). So, bigger elements mean more bandwidth! Simple, right?

\subsubsection*{Advanced Explanation}
The bandwidth of a Yagi antenna is influenced by the physical characteristics of its elements. Larger-diameter elements reduce the Q-factor (quality factor) of the antenna, which in turn increases the bandwidth. The Q-factor is inversely proportional to the bandwidth, as given by the formula:

\[
\text{Bandwidth} \propto \frac{1}{Q}
\]

Larger-diameter elements have lower Q-factors because they have lower resistance and higher capacitance, which allows them to operate over a wider range of frequencies. Conversely, closer element spacing and loading coils increase the Q-factor, reducing the bandwidth. Tapered-diameter elements do not significantly affect the bandwidth in the same way as larger-diameter elements.

% Diagram Prompt: Generate a diagram showing a Yagi antenna with larger-diameter elements and label the bandwidth increase.