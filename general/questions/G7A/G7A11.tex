\subsection{NPN Junction Transistor Symbol}
\label{G7A11}

\begin{tcolorbox}[colback=gray!10!white,colframe=black!75!black,title=G7A11]
Which symbol in figure G7-1 represents an NPN junction transistor?
\begin{enumerate}[label=\Alph*),noitemsep]
    \item Symbol 1
    \item \textbf{Symbol 2}
    \item Symbol 7
    \item Symbol 11
\end{enumerate}
\end{tcolorbox}

\subsubsection{Intuitive Explanation}
Imagine you have a sandwich with three layers: bread, ham, and bread again. Now, think of an NPN transistor as a special kind of sandwich where the ham is in the middle, and the bread is on the outside. In the world of electronics, the NPN transistor has three parts: two bread layers (called the emitter and collector) and one ham layer (called the base). The symbol for an NPN transistor in figure G7-1 is like a little drawing that shows this sandwich. Symbol 2 is the one that represents this NPN sandwich correctly!

\subsubsection{Advanced Explanation}
An NPN junction transistor is a type of bipolar junction transistor (BJT) that consists of three semiconductor layers: an N-type layer (emitter), a P-type layer (base), and another N-type layer (collector). The symbol for an NPN transistor typically includes an arrow on the emitter terminal pointing outward, indicating the direction of conventional current flow from the base to the emitter.

In figure G7-1, Symbol 2 correctly represents an NPN transistor. The arrow on the emitter points away from the base, which is the standard representation for an NPN transistor. The other symbols either represent different types of transistors (such as PNP) or other electronic components.

To identify the correct symbol, one must understand the following:
\begin{itemize}
    \item The arrow direction: In NPN transistors, the arrow points outward from the base.
    \item The labeling of terminals: The emitter, base, and collector must be correctly identified.
\end{itemize}

% Prompt for generating the diagram:
% Include a diagram showing the symbols of different transistors, highlighting Symbol 2 as the NPN transistor. Label the emitter, base, and collector clearly.