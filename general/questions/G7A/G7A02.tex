\subsection{Power Supply Filter Network Components}
\label{G7A02}

\begin{tcolorbox}[colback=gray!10!white,colframe=black!75!black,title=G7A02]
Which of the following components are used in a power supply filter network?
\begin{enumerate}[label=\Alph*]
    \item Diodes
    \item Transformers and transducers
    \item \textbf{Capacitors and inductors}
    \item All these choices are correct
\end{enumerate}
\end{tcolorbox}

\subsubsection{Intuitive Explanation}
Imagine you're trying to clean up a messy room. You have a vacuum cleaner (capacitor) and a broom (inductor). The vacuum cleaner sucks up the dust (high-frequency noise), and the broom sweeps away the larger debris (low-frequency noise). Together, they make the room (your power supply) nice and clean. So, in a power supply filter network, capacitors and inductors are the cleaning crew that keeps the power smooth and steady.

\subsubsection{Advanced Explanation}
A power supply filter network is designed to remove unwanted AC components (ripple) from the DC output of a power supply. This is achieved using a combination of capacitors and inductors. 

Capacitors store electrical energy and can smooth out voltage fluctuations by charging and discharging. The impedance of a capacitor is given by:
\[
Z_C = \frac{1}{j\omega C}
\]
where \( \omega \) is the angular frequency and \( C \) is the capacitance. At high frequencies, the impedance is low, allowing the capacitor to short out high-frequency noise.

Inductors, on the other hand, store energy in a magnetic field and resist changes in current. The impedance of an inductor is:
\[
Z_L = j\omega L
\]
where \( L \) is the inductance. At high frequencies, the impedance is high, blocking high-frequency noise.

Together, capacitors and inductors form a low-pass filter that allows DC to pass while attenuating AC components. This is essential for providing a stable DC voltage to electronic circuits.

% Diagram prompt: Generate a diagram showing a simple LC filter circuit with an input, capacitor, inductor, and output.