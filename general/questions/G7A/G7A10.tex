\subsection{Symbol Representation of a Zener Diode}
\label{G7A10}

\begin{tcolorbox}[colback=gray!10!white,colframe=black!75!black,title=G7A10]
Which symbol in figure G7-1 represents a Zener diode?
\begin{enumerate}[label=\Alph*]
    \item Symbol 4
    \item Symbol 1
    \item Symbol 11
    \item \textbf{Symbol 5}
\end{enumerate}
\end{tcolorbox}

\subsubsection{Intuitive Explanation}
Imagine you’re looking at a bunch of road signs, and you need to find the one that says Zener Diode. It’s like a special kind of diode that’s really good at controlling voltage. In this case, the correct road sign is Symbol 5. Think of it as the superhero of diodes, always ready to save the day by keeping the voltage in check!

\subsubsection{Advanced Explanation}
A Zener diode is a type of diode that allows current to flow not only in the forward direction but also in the reverse direction when the voltage exceeds a certain value, known as the Zener breakdown voltage. This characteristic makes it useful for voltage regulation. In the context of the question, Symbol 5 represents a Zener diode. The symbol typically includes a small Z shape or a line at the cathode end to distinguish it from a regular diode.

% Prompt for generating the diagram:
% Include a diagram showing various diode symbols, with Symbol 5 clearly labeled as the Zener diode. The diagram should illustrate the distinct features of a Zener diode symbol compared to other diode symbols.