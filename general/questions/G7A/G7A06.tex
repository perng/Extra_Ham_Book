\subsection{AC Cycle Conversion by Full-Wave Rectifier}
\label{G7A06}

\begin{tcolorbox}[colback=gray!10!white,colframe=black!75!black,title=G7A06]
What portion of the AC cycle is converted to DC by a full-wave rectifier?
\begin{enumerate}[label=\Alph*,noitemsep]
    \item 90 degrees
    \item 180 degrees
    \item 270 degrees
    \item \textbf{360 degrees}
\end{enumerate}
\end{tcolorbox}

\subsubsection{Intuitive Explanation}
Imagine you have a seesaw that goes up and down, just like the AC current. A full-wave rectifier is like a magical tool that makes sure the seesaw only goes up, no matter which way it was going before. So, instead of going up and down, it just goes up, up, up! This means it uses the entire cycle of the seesaw (360 degrees) to keep things moving in one direction. Cool, right?

\subsubsection{Advanced Explanation}
A full-wave rectifier converts both the positive and negative halves of the AC cycle into DC. In an AC cycle, one complete cycle is 360 degrees. The full-wave rectifier utilizes both the positive half-cycle (0 to 180 degrees) and the negative half-cycle (180 to 360 degrees) by inverting the negative half-cycle to positive. Therefore, the entire 360 degrees of the AC cycle are effectively converted to DC. Mathematically, this can be represented as:

\[
V_{\text{DC}} = \frac{2V_{\text{peak}}}{\pi}
\]

where \( V_{\text{peak}} \) is the peak voltage of the AC signal. This equation shows that the full-wave rectifier averages the absolute value of the AC signal over the entire cycle, confirming that 360 degrees are utilized.

% Diagram Prompt: Generate a diagram showing an AC sine wave and its corresponding full-wave rectified output.