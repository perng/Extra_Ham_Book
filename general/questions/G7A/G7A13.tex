\subsection{Symbol Representation in Figure G7-1}
\label{G7A13}

\begin{tcolorbox}[colback=gray!10!white,colframe=black!75!black,title=G7A13]
Which symbol in Figure G7-1 represents a tapped inductor?
\begin{enumerate}[label=\Alph*,noitemsep]
    \item \textbf{Symbol 7}
    \item Symbol 11
    \item Symbol 6
    \item Symbol 1
\end{enumerate}
\end{tcolorbox}

\subsubsection{Intuitive Explanation}
Imagine you have a piece of string that you can pull from different points to make it longer or shorter. A tapped inductor is like that string, but instead of pulling it, you can connect to it at different points to get different amounts of inductance magic. In Figure G7-1, Symbol 7 is the one that shows this special kind of inductor where you can tap into it at different points. It's like having a secret door that lets you choose how much magic you want to use!

\subsubsection{Advanced Explanation}
A tapped inductor is an inductor with multiple connection points along its winding, allowing for different inductance values to be accessed. In circuit diagrams, this is represented by a symbol that shows a coil with one or more additional taps. Symbol 7 in Figure G7-1 is the correct representation of a tapped inductor. 

Inductance, \( L \), is a property of an inductor that opposes changes in current. The inductance value can be calculated using the formula:
\[
L = \frac{N^2 \mu A}{l}
\]
where \( N \) is the number of turns, \( \mu \) is the permeability of the core, \( A \) is the cross-sectional area, and \( l \) is the length of the coil. By tapping into different points of the inductor, you effectively change the number of turns \( N \), thus altering the inductance.

% Prompt for generating the diagram: 
% Include a diagram showing various inductor symbols, with Symbol 7 highlighted as the tapped inductor.