\subsection{Identifying Field Effect Transistor Symbol}
\label{G7A09}

\begin{tcolorbox}[colback=gray!10!white,colframe=black!75!black,title=G7A09]
Which symbol in figure G7-1 represents a field effect transistor?
\begin{enumerate}[label=\Alph*)]
    \item Symbol 2
    \item Symbol 5
    \item \textbf{Symbol 1}
    \item Symbol 4
\end{enumerate}
\end{tcolorbox}

\subsubsection{Intuitive Explanation}
Imagine you're looking at a bunch of emojis, and you need to pick the one that represents a cool gadget. In this case, the cool gadget is a field effect transistor (FET). FETs are like tiny switches that control the flow of electricity. The correct emoji (symbol) is the one that looks like it has a gate, a source, and a drain—just like a FET. So, if you see a symbol that fits this description, that's your FET!

\subsubsection{Advanced Explanation}
A field effect transistor (FET) is a type of transistor that uses an electric field to control the flow of current. The three main terminals of a FET are the gate (G), source (S), and drain (D). The gate controls the conductivity between the source and the drain by modulating the electric field within the device. In schematic diagrams, FETs are represented by specific symbols that depict these terminals and their connections. The correct symbol for a FET in figure G7-1 is Symbol 1, which accurately represents the gate, source, and drain configuration.

% Prompt for generating the diagram:
% Include a diagram labeled Figure G7-1 showing various electronic symbols, with Symbol 1 clearly marked as the field effect transistor. The diagram should illustrate the gate, source, and drain terminals of the FET.