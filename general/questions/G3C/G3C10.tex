\subsection{Near Vertical Incidence Skywave (NVIS) Propagation}
\label{G3C10}

\begin{tcolorbox}[colback=gray!10!white,colframe=black!75!black,title=G3C10]
What is near vertical incidence skywave (NVIS) propagation?
\begin{enumerate}[label=\Alph*,noitemsep]
    \item Propagation near the MUF
    \item \textbf{Short distance MF or HF propagation at high elevation angles}
    \item Long path HF propagation at sunrise and sunset
    \item Double hop propagation near the LUF
\end{enumerate}
\end{tcolorbox}

\subsubsection{Intuitive Explanation}
Imagine you're trying to throw a ball straight up into the air and catch it yourself. You don't need to throw it far, just high enough so it comes back down to you. NVIS propagation is like that but with radio waves! Instead of sending signals far away, you send them almost straight up into the sky, and they bounce back down to cover a short distance. It's perfect for talking to someone nearby without needing to go around the Earth's curve.

\subsubsection{Advanced Explanation}
Near Vertical Incidence Skywave (NVIS) propagation is a technique used in radio communication where signals are transmitted at high elevation angles, typically between 70 and 90 degrees. This method is particularly effective for short to medium distances, usually within a few hundred kilometers. The signals are reflected back to Earth by the ionosphere, which acts as a mirror for radio waves. 

The key advantage of NVIS is its ability to provide reliable communication over areas that are otherwise difficult to cover due to terrain obstacles, such as mountains or dense forests. The frequency range for NVIS typically falls within the Medium Frequency (MF) and High Frequency (HF) bands, specifically between 2 to 10 MHz. 

Mathematically, the critical frequency \( f_c \) for NVIS can be approximated using the formula:
\[
f_c = \sqrt{80.8 \times N_e}
\]
where \( N_e \) is the electron density in the ionosphere. This frequency determines the maximum usable frequency (MUF) for NVIS propagation. 

NVIS is particularly useful in emergency communication scenarios, where establishing reliable short-distance communication is crucial. It is also employed in military operations and amateur radio activities.

% Prompt for generating a diagram: A diagram showing radio waves being transmitted at a high elevation angle, reflecting off the ionosphere, and returning to Earth within a short distance.