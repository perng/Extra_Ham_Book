\subsection{Long-Distance Communication Challenges on Certain Bands}
\label{G3C05}

\begin{tcolorbox}[colback=gray!10!white,colframe=black!75!black,title=G3C05]
Why is long-distance communication on the 40-, 60-, 80-, and 160-meter bands more difficult during the day?
\begin{enumerate}[label=\Alph*,noitemsep]
    \item The F region absorbs signals at these frequencies during daylight hours
    \item The F region is unstable during daylight hours
    \item \textbf{The D region absorbs signals at these frequencies during daylight hours}
    \item The E region is unstable during daylight hours
\end{enumerate}
\end{tcolorbox}

\subsubsection*{Intuitive Explanation}
Imagine the Earth's atmosphere is like a big sandwich with different layers. During the day, one of these layers, called the D region, acts like a sponge and soaks up radio signals, especially on the 40-, 60-, 80-, and 160-meter bands. This makes it harder for these signals to travel long distances. At night, the D region goes to sleep, and the signals can travel much farther without getting soaked up!

\subsubsection*{Advanced Explanation}
The Earth's ionosphere is divided into several regions: D, E, and F. The D region, located at altitudes of 60 to 90 km, is primarily responsible for the absorption of radio waves during daylight hours. This absorption is due to the ionization of atmospheric gases by solar radiation, which increases the electron density in the D region. The absorption coefficient \(\alpha\) can be approximated by:

\[
\alpha \propto \frac{N_e \nu^2}{\nu^2 + \nu_c^2}
\]

where \(N_e\) is the electron density, \(\nu\) is the frequency of the radio wave, and \(\nu_c\) is the collision frequency. For frequencies in the 40-, 60-, 80-, and 160-meter bands, the absorption is significant during the day, making long-distance communication more challenging. At night, the D region's electron density decreases, reducing absorption and allowing signals to propagate further.

% Diagram prompt: Generate a diagram showing the Earth's ionosphere layers (D, E, F) and the path of radio waves during day and night.