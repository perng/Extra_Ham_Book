\subsection{Ionospheric Absorption of Signals Below 10 MHz}
\label{G3C11}

\begin{tcolorbox}[colback=gray!10!white,colframe=black!75!black,title=G3C11]
Which ionospheric region is the most absorbent of signals below 10 MHz during daylight hours?
\begin{enumerate}[label=\Alph*),noitemsep]
    \item The F2 region
    \item The F1 region
    \item The E region
    \item \textbf{The D region}
\end{enumerate}
\end{tcolorbox}

\subsubsection*{Intuitive Explanation}
Imagine the ionosphere as a giant sponge in the sky that soaks up radio signals. During the day, the sun shines on this sponge, making it extra thirsty for signals, especially the ones below 10 MHz. The D region is like the bottom layer of the sponge, and it’s the most absorbent part during daylight hours. So, if you’re trying to send a signal below 10 MHz during the day, the D region is going to slurp it up like a smoothie!

\subsubsection*{Advanced Explanation}
The ionosphere is composed of several layers, each with distinct characteristics. The D region, located at altitudes between 60 to 90 km, is particularly significant for its absorption of radio waves. During daylight hours, solar radiation ionizes the D region, increasing its electron density. This ionization leads to higher absorption of radio signals, especially those below 10 MHz, due to collisions between electrons and neutral particles. The absorption coefficient \(\alpha\) in the D region can be approximated by:

\[
\alpha \propto \frac{N_e \nu}{f^2}
\]

where \(N_e\) is the electron density, \(\nu\) is the collision frequency, and \(f\) is the frequency of the radio wave. Since \(f\) is in the denominator, lower frequencies experience higher absorption. The D region’s high electron density and collision frequency make it the most absorbent layer for signals below 10 MHz during daylight hours.

% Diagram prompt: Generate a diagram showing the ionospheric layers (D, E, F1, F2) with annotations indicating their altitudes and absorption characteristics for signals below 10 MHz.