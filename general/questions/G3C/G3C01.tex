\subsection{Ionospheric Regions and Their Proximity to Earth}
\label{G3C01}

\begin{tcolorbox}[colback=gray!10!white,colframe=black!75!black,title=G3C01]
Which ionospheric region is closest to the surface of Earth?
\begin{enumerate}[label=\Alph*,noitemsep]
    \item \textbf{The D region}
    \item The E region
    \item The F1 region
    \item The F2 region
\end{enumerate}
\end{tcolorbox}

\subsubsection{Intuitive Explanation}
Imagine the Earth is like a giant onion, and the ionosphere is one of its layers. The ionosphere is divided into different regions, kind of like how an onion has different layers. The D region is the layer closest to the Earth's surface, just like the first layer of an onion is the one you peel off first. So, if you were to start peeling the Earth's ionosphere, the D region would be the first layer you'd encounter!

\subsubsection{Advanced Explanation}
The ionosphere is a region of the Earth's upper atmosphere, ionized by solar radiation. It is divided into several distinct layers based on their altitude and ionization characteristics. The D region is the lowest of these layers, typically located between 60 to 90 kilometers above the Earth's surface. This region is primarily responsible for the absorption of high-frequency radio waves during daylight hours, which can affect radio communication.

The other regions, in order of increasing altitude, are:
\begin{itemize}
    \item The E region, located between 90 to 120 kilometers.
    \item The F1 region, located between 150 to 200 kilometers.
    \item The F2 region, located above 200 kilometers.
\end{itemize}

The D region's proximity to the Earth's surface makes it the closest ionospheric region. Its lower altitude means it is more directly influenced by the Earth's atmosphere and weather conditions, which can impact its ionization levels and, consequently, its effect on radio wave propagation.

% Diagram Prompt: Generate a diagram showing the layers of the ionosphere with their respective altitudes, highlighting the D region as the closest to the Earth's surface.