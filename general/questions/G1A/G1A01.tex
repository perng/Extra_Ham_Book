\subsection{Portions of HF and MF Amateur Bands Restricted for General Class Licensees}\label{G1A01}

\begin{tcolorbox}[colback=gray!10!white,colframe=black!75!black,title=G1A01]
On which HF and/or MF amateur bands are there portions where General class licensees cannot transmit?
\begin{enumerate}[label=\Alph*]
    \item 60 meters, 30 meters, 17 meters, and 12 meters
    \item 160 meters, 60 meters, 15 meters, and 12 meters
    \item \textbf{80 meters, 40 meters, 20 meters, and 15 meters}
    \item 80 meters, 20 meters, 15 meters, and 10 meters
\end{enumerate}
\end{tcolorbox}

\subsubsection{Intuitive Explanation}
Imagine you’re at a big party with different rooms for different age groups. The General class licensees are like teenagers—they can hang out in most rooms, but there are a few areas where only the adults (Extra class licensees) are allowed. In the world of radio, these restricted rooms are parts of the 80 meters, 40 meters, 20 meters, and 15 meters bands. So, if you’re a General class licensee, you can’t transmit in those specific areas, but you’re free to roam in the rest of the radio spectrum!

\subsubsection{Advanced Explanation}
The Federal Communications Commission (FCC) allocates specific frequency ranges within the HF (High Frequency) and MF (Medium Frequency) bands for amateur radio use. These allocations are further divided into sub-bands based on license class. General class licensees have access to most of these sub-bands, but there are certain portions where only Extra class licensees are permitted to transmit.

The restricted portions for General class licensees are as follows:
\begin{itemize}
    \item \textbf{80 meters (3.5-4.0 MHz)}: General class licensees are restricted from transmitting in the 3.8-4.0 MHz range.
    \item \textbf{40 meters (7.0-7.3 MHz)}: General class licensees are restricted from transmitting in the 7.125-7.3 MHz range.
    \item \textbf{20 meters (14.0-14.35 MHz)}: General class licensees are restricted from transmitting in the 14.15-14.35 MHz range.
    \item \textbf{15 meters (21.0-21.45 MHz)}: General class licensees are restricted from transmitting in the 21.2-21.45 MHz range.
\end{itemize}

These restrictions are in place to ensure that higher-class licensees have exclusive access to certain frequencies, which can be particularly useful for long-distance communication and contesting. The exact frequency ranges and restrictions are detailed in the FCC regulations, and it’s important for operators to be aware of these to avoid unintentional interference.

% Diagram Prompt: A frequency spectrum chart showing the HF and MF bands with highlighted areas indicating the restricted portions for General class licensees.