\subsection{G1A04: Amateur Bands Restricted to Specific Channels}
\label{G1A04}

\begin{tcolorbox}[colback=gray!10!white,colframe=black!75!black,title=G1A04]
Which of the following amateur bands is restricted to communication only on specific channels, rather than frequency ranges?
\begin{enumerate}[label=\Alph*),noitemsep]
    \item 11 meters
    \item 12 meters
    \item 30 meters
    \item \textbf{60 meters}
\end{enumerate}
\end{tcolorbox}

\subsubsection{Intuitive Explanation}
Imagine you're at a school dance, and the DJ has set up different channels for different types of music. You can only dance to the music on the channel you're tuned into. Similarly, in the 60-meter amateur band, you can only communicate on specific channels, like specific songs, rather than having the freedom to choose any frequency within a range. It's like being told, You can only dance to the Cha-Cha Slide, not any song you want!

\subsubsection{Advanced Explanation}
In amateur radio, most bands allow operators to transmit on any frequency within a specified range, subject to certain rules and regulations. However, the 60-meter band (5.3 MHz to 5.4 MHz) is unique because it is restricted to specific channels rather than a continuous frequency range. This restriction is due to international agreements and the need to avoid interference with other services.

The 60-meter band is allocated for amateur use on a secondary basis, meaning that amateurs must not cause harmful interference to primary users and must accept interference from them. The specific channels are:

\begin{itemize}
    \item 5.3305 MHz
    \item 5.3465 MHz
    \item 5.3570 MHz
    \item 5.3715 MHz
    \item 5.4035 MHz
\end{itemize}

These channels are spaced to minimize interference and ensure efficient use of the spectrum. The restriction to specific channels is a compromise that allows amateur radio operators to use this band while protecting primary users.

% Diagram Prompt: Generate a diagram showing the frequency spectrum of the 60-meter band with the specific channels marked.