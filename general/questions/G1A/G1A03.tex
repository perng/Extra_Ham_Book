\subsection{Prohibited Bands for Image Transmission}
\label{G1A03}

\begin{tcolorbox}[colback=gray!10!white,colframe=black!75!black,title=G1A03]
On which of the following bands is image transmission prohibited?
\begin{enumerate}[label=\Alph*)]
    \item 160 meters
    \item \textbf{30 meters}
    \item 20 meters
    \item 12 meters
\end{enumerate}
\end{tcolorbox}

\subsubsection{Intuitive Explanation}
Imagine you're at a party, and there are different rooms where you can play different types of music. Now, the party organizers have a rule: in one specific room, you can only talk, no music allowed! In the world of radio, the 30 meters band is like that room—only certain types of communication are allowed, and image transmission is not one of them. So, if you try to send pictures on the 30 meters band, you're breaking the party rules!

\subsubsection{Advanced Explanation}
The 30 meters band (10.1 MHz to 10.15 MHz) is part of the High Frequency (HF) spectrum and is allocated for amateur radio use. However, it is strictly limited to specific modes of communication, primarily Morse code (CW) and data modes like RTTY and PSK31. Image transmission, which typically involves modes like SSTV (Slow Scan Television) or digital image modes, is prohibited on this band. This restriction is in place to minimize interference and ensure efficient use of the limited bandwidth available.

The International Telecommunication Union (ITU) and national regulatory bodies, such as the FCC in the United States, set these rules to manage the radio spectrum effectively. The 30 meters band is particularly sensitive due to its narrow bandwidth and its use for long-distance communication, especially during periods of low solar activity when higher frequency bands may not be as effective.

% Prompt for generating a diagram: A frequency spectrum chart showing the 30 meters band and its allocated modes of communication, with a clear indication of prohibited modes like image transmission.