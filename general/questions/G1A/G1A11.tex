\subsection{Available Voice Segment for General Class Licensees}
\label{G1A11}

\begin{tcolorbox}[colback=gray!10!white,colframe=black!75!black,title=G1A11]
When General class licensees are not permitted to use the entire voice portion of a band, which portion of the voice segment is available to them?
\begin{enumerate}[label=\Alph*]
    \item The lower frequency portion
    \item \textbf{The upper frequency portion}
    \item The lower frequency portion on frequencies below 7.3 MHz, and the upper portion on frequencies above 14.150 MHz
    \item The upper frequency portion on frequencies below 7.3 MHz, and the lower portion on frequencies above 14.150 MHz
\end{enumerate}
\end{tcolorbox}

\subsubsection{Intuitive Explanation}
Imagine you and your friends are sharing a big pizza, but there are rules about which slices you can take. For General class licensees, the rule is simple: they can only take the slices from the top half of the pizza. In radio terms, this means they can use the upper frequency portion of the voice segment. So, if the whole band is like the pizza, they get the upper part, not the lower part. Easy, right?

\subsubsection{Advanced Explanation}
In radio frequency allocation, different classes of licenses have access to specific portions of the frequency bands. For General class licensees, when the entire voice portion of a band is not available, they are typically granted access to the upper frequency portion of the voice segment. This is due to regulatory decisions aimed at optimizing spectrum usage and minimizing interference between different user groups.

For example, in the HF bands, General class licensees might be restricted from using certain lower frequency segments to avoid interference with other services or license classes. Therefore, they are allocated the upper frequency portion of the voice segment. This ensures efficient use of the spectrum while maintaining clear communication channels for all users.

% Diagram Prompt: Generate a diagram showing the frequency band with labeled segments indicating the upper and lower portions available to General class licensees.