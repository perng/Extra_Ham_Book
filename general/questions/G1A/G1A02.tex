\subsection{Prohibited Phone Operation Bands}
\label{G1A02}

\begin{tcolorbox}[colback=gray!10!white,colframe=black!75!black,title=G1A02]
On which of the following bands is phone operation prohibited?
\begin{enumerate}[label=\Alph*)]
    \item 160 meters
    \item \textbf{30 meters}
    \item 17 meters
    \item 12 meters
\end{enumerate}
\end{tcolorbox}

\subsubsection{Intuitive Explanation}
Imagine you're at a party, and there are different rooms where people are chatting. In some rooms, you can talk loudly, in others, you can only whisper, and in one special room, you're not allowed to talk at all! The 30-meter band is like that special room where phone operation (talking) is not allowed. Instead, you can only use Morse code or digital modes to communicate. So, if you want to talk on the radio, you better avoid the 30-meter band!

\subsubsection{Advanced Explanation}
The 30-meter band (10.1 MHz to 10.15 MHz) is part of the High Frequency (HF) spectrum. This band is allocated for amateur radio use, but it is restricted to specific modes of operation. According to the International Telecommunication Union (ITU) regulations and the Federal Communications Commission (FCC) rules in the United States, phone operation (voice communication) is prohibited on the 30-meter band. This restriction is in place to minimize interference with other services that share this frequency range, such as fixed and mobile services.

The 30-meter band is primarily used for digital modes like RTTY, PSK31, and CW (Morse code). These modes are more efficient in terms of bandwidth usage and are less likely to cause interference with other users. The band is also relatively narrow, which further necessitates the restriction on phone operation to ensure efficient use of the spectrum.

In summary, the 30-meter band is unique in that it does not allow phone operation, making it a specialized band for digital and CW communications.

% Diagram Prompt: Generate a diagram showing the frequency allocation of the 30-meter band compared to other HF bands, highlighting the prohibition of phone operation.