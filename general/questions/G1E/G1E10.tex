\subsection{Why Should an Amateur Operator Avoid Transmitting on Specific Frequencies?}\label{G1E10}

\begin{tcolorbox}[colback=gray!10!white,colframe=black!75!black,title=G1E10]
Why should an amateur operator normally avoid transmitting on 14.100, 18.110, 21.150, 24.930 and 28.200 MHz?
\begin{enumerate}[label=\Alph*,noitemsep]
    \item \textbf{A system of propagation beacon stations operates on those frequencies}
    \item A system of automatic digital stations operates on those frequencies
    \item These frequencies are set aside for emergency operations
    \item These frequencies are set aside for bulletins from the FCC
\end{enumerate}
\end{tcolorbox}

\subsubsection{Intuitive Explanation}
Imagine you're at a concert, and there's a special microphone on stage that helps everyone in the audience hear the music better. Now, if someone starts talking loudly into that microphone, it messes up the music for everyone. In the world of radio, certain frequencies are like that special microphone—they're used by special stations called propagation beacons to help radio operators understand how signals are traveling through the air. If amateur operators transmit on these frequencies, it's like talking into that special microphone and messing up the important information for everyone else. So, it's best to avoid those frequencies to keep the radio world harmonious!

\subsubsection{Advanced Explanation}
Propagation beacon stations are critical for understanding radio wave propagation characteristics, such as ionospheric conditions. These beacons transmit continuous signals on specific frequencies (e.g., 14.100 MHz, 18.110 MHz, 21.150 MHz, 24.930 MHz, and 28.200 MHz) to allow operators to monitor and analyze propagation paths. Transmitting on these frequencies can interfere with the beacon signals, disrupting the data collection process. 

The ionosphere, a layer of the Earth's atmosphere, plays a significant role in radio wave propagation. By reflecting or refracting radio waves, it enables long-distance communication. Propagation beacons help operators determine the optimal frequencies and times for communication by providing real-time data on ionospheric conditions. 

Mathematically, the critical frequency \( f_c \) of the ionosphere can be calculated using the formula:
\[
f_c = \sqrt{80.8 \cdot N_e}
\]
where \( N_e \) is the electron density in electrons per cubic meter. This frequency is crucial for determining the maximum usable frequency (MUF) for communication.

In summary, avoiding these frequencies ensures that propagation beacon stations can operate without interference, providing valuable data for amateur and professional radio operators alike.

% Diagram Prompt: Generate a diagram showing the ionosphere layers and how radio waves propagate through them, with labels for the critical frequency and MUF.