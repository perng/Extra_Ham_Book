\subsection{ITU Region Frequency Allocations for Radio Amateurs in North and South America}
\label{G1E06}

\begin{tcolorbox}[colback=gray!10!white,colframe=black!75!black,title=G1E06]
The frequency allocations of which ITU region apply to radio amateurs operating in North and South America?
\begin{enumerate}[label=\Alph*)]
    \item Region 4
    \item Region 3
    \item \textbf{Region 2}
    \item Region 1
\end{enumerate}
\end{tcolorbox}

\subsubsection{Intuitive Explanation}
Imagine the world is divided into different neighborhoods, and each neighborhood has its own set of rules for using the radio. For radio amateurs in North and South America, the neighborhood they belong to is called Region 2. Just like how your neighborhood might have specific rules for playing music loudly, Region 2 has specific frequency allocations that radio amateurs need to follow. So, if you're in North or South America, you need to play by the rules of Region 2!

\subsubsection{Advanced Explanation}
The International Telecommunication Union (ITU) divides the world into three regions for the purpose of managing radio frequency allocations:

\begin{itemize}
    \item \textbf{Region 1}: Europe, Africa, the Middle East, and the northern part of Asia.
    \item \textbf{Region 2}: North and South America, including the Caribbean.
    \item \textbf{Region 3}: The rest of Asia, including Southeast Asia, and the Pacific.
\end{itemize}

Radio amateurs operating in North and South America fall under ITU Region 2. This region has specific frequency bands allocated for amateur radio use, which are different from those in Regions 1 and 3. These allocations are designed to minimize interference and ensure efficient use of the radio spectrum.

For example, the 40-meter band (7.0–7.3 MHz) is allocated differently in each region. In Region 2, the entire band is available for amateur use, while in Region 1, only a portion of the band is allocated. This regional allocation ensures that radio amateurs in different parts of the world can operate without causing interference to each other.

% Prompt for generating a diagram: A world map divided into ITU Regions 1, 2, and 3, with North and South America highlighted in Region 2.