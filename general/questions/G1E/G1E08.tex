\subsection{Maximum PEP Output for Spread Spectrum Transmissions}
\label{G1E08}

\begin{tcolorbox}[colback=gray!10!white,colframe=black!75!black,title=G1E08]
What is the maximum PEP output allowed for spread spectrum transmissions?
\begin{enumerate}[label=\Alph*)]
    \item 100 milliwatts
    \item \textbf{10 watts}
    \item 100 watts
    \item 1500 watts
\end{enumerate}
\end{tcolorbox}

\subsubsection{Intuitive Explanation}
Imagine you're playing a game where you can only shout as loud as 10 watts. If you shout louder, you might disturb the neighbors or even break the rules of the game. In the world of radio, spread spectrum transmissions are like this game. The rules say you can't use more than 10 watts of power when you're sending your signals. This keeps everything fair and prevents interference with other radios. So, the maximum power you're allowed to use is 10 watts—just like the maximum volume you're allowed to shout in the game!

\subsubsection{Advanced Explanation}
In radio communications, the Peak Envelope Power (PEP) is the maximum power that a transmitter can output during a transmission. For spread spectrum transmissions, which are a type of radio communication that spreads the signal over a wide frequency band, the Federal Communications Commission (FCC) has set specific limits to ensure that these transmissions do not interfere with other communications.

The FCC regulations state that the maximum PEP output for spread spectrum transmissions is 10 watts. This limit is designed to balance the need for effective communication with the need to minimize interference with other radio services. The calculation of PEP involves measuring the power at the peak of the transmitted signal's envelope, which is the highest instantaneous power level during the transmission.

To calculate PEP, you would use the following formula:
\[
\text{PEP} = \frac{V_{\text{peak}}^2}{R}
\]
where \( V_{\text{peak}} \) is the peak voltage of the signal and \( R \) is the resistance of the transmission line. However, in practice, PEP is often measured directly using specialized equipment.

Understanding this limit is crucial for anyone involved in designing or operating spread spectrum communication systems, as exceeding the PEP limit can result in regulatory penalties and interference with other radio services.

% Prompt for diagram: A diagram showing the relationship between PEP, signal envelope, and the 10-watt limit for spread spectrum transmissions would be helpful here.