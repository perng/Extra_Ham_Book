\subsection{Minimum Wire Size for 20-Ampere Circuit Breaker}
\label{G0B02}

\begin{tcolorbox}[colback=gray!10!white,colframe=black!75!black,title=G0B02]
According to the National Electrical Code, what is the minimum wire size that may be used safely for wiring with a 20-ampere circuit breaker?
\begin{enumerate}[label=\Alph*]
    \item AWG number 20
    \item AWG number 16
    \item \textbf{AWG number 12}
    \item AWG number 8
\end{enumerate}
\end{tcolorbox}

\subsubsection{Intuitive Explanation}
Imagine you're trying to pour water through a hose. If the hose is too small, the water won't flow smoothly, and the hose might burst! Similarly, when electricity flows through a wire, the wire needs to be thick enough to handle the current without overheating. For a 20-ampere circuit breaker, the wire needs to be at least AWG 12, which is like a medium-sized hose—just right for the job.

\subsubsection{Advanced Explanation}
The National Electrical Code (NEC) specifies the minimum wire size based on the current-carrying capacity of the wire and the circuit breaker rating. For a 20-ampere circuit, the NEC requires a minimum wire size of AWG 12. This is determined by the wire's ampacity, which is the maximum current it can safely carry without exceeding its temperature rating. 

The ampacity of AWG 12 copper wire is 20 amperes, which matches the circuit breaker rating. Using a smaller wire, such as AWG 16 or AWG 20, would result in excessive heat generation, potentially leading to insulation damage or even fire. Conversely, using a larger wire, such as AWG 8, is unnecessary and increases material costs without providing additional benefits.

The relationship between wire size and current capacity can be understood through Ohm's Law and the power dissipation formula:
\[
P = I^2 \times R
\]
where \( P \) is the power dissipated as heat, \( I \) is the current, and \( R \) is the resistance of the wire. Larger wires have lower resistance, reducing heat generation for a given current.

% Diagram Prompt: Generate a diagram showing the relationship between wire gauge (AWG) and current capacity, highlighting the minimum wire size for a 20-ampere circuit breaker.