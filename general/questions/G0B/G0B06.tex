\subsection{National Electrical Code Coverage}
\label{G0B06}

\begin{tcolorbox}[colback=gray!10!white,colframe=black!75!black,title=G0B06]
Which of the following is covered by the National Electrical Code?
\begin{enumerate}[label=\Alph*)]
    \item Acceptable bandwidth limits
    \item Acceptable modulation limits
    \item \textbf{Electrical safety of the station}
    \item RF exposure limits of the human body
\end{enumerate}
\end{tcolorbox}

\subsubsection{Intuitive Explanation}
Imagine the National Electrical Code (NEC) as the rulebook for making sure your house doesn't catch fire because of bad wiring. It's like the safety manual for anything that uses electricity. So, when you're setting up your radio station, the NEC is there to make sure you don't accidentally zap yourself or start a fire. It doesn't care about how loud your radio is or what kind of music you play—it's all about keeping you safe from electrical hazards.

\subsubsection{Advanced Explanation}
The National Electrical Code (NEC) is a set of standards designed to ensure the safe installation of electrical wiring and equipment. It is published by the National Fire Protection Association (NFPA) and is widely adopted in the United States. The NEC covers various aspects of electrical safety, including wiring methods, materials, and equipment. 

In the context of a radio station, the NEC would be concerned with the electrical safety of the station, such as proper grounding, circuit protection, and the safe installation of electrical components. It does not address bandwidth limits, modulation techniques, or RF exposure limits, which are typically regulated by other standards or organizations such as the Federal Communications Commission (FCC).

For example, the NEC would specify the correct gauge of wire to use for a given current load to prevent overheating and potential fire hazards. It would also require the use of circuit breakers or fuses to protect against overcurrent conditions. These safety measures are crucial in preventing electrical accidents and ensuring the safe operation of the station.

% Diagram prompt: A simple diagram showing a radio station setup with proper grounding and circuit protection as per NEC standards.