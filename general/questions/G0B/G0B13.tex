\subsection{Lightning Arrestor Placement}
\label{G0B13}

\begin{tcolorbox}[colback=gray!10!white,colframe=black!75!black,title=G0B13]
Where should lightning arrestors be located?
\begin{enumerate}[label=\Alph*,noitemsep]
    \item \textbf{Where the feed lines enter the building}
    \item On the antenna, opposite the feed point
    \item In series with each ground lead
    \item At the closest power pole ground electrode
\end{enumerate}
\end{tcolorbox}

\subsubsection{Intuitive Explanation}
Imagine your house is a castle, and lightning is a dragon trying to attack. The lightning arrestor is like a magical shield that protects your castle. But where should you place this shield? If you put it where the dragon (lightning) first tries to enter your castle (where the feed lines enter the building), you can stop it right at the gate! Placing it anywhere else would be like putting the shield on the roof or in the basement—it just wouldn’t work as well.

\subsubsection{Advanced Explanation}
Lightning arrestors are designed to protect electrical systems from the damaging effects of lightning strikes. They work by providing a low-impedance path to ground for the lightning current, thereby preventing it from entering the building's electrical system. The optimal location for a lightning arrestor is where the feed lines enter the building. This placement ensures that any lightning-induced surge is intercepted before it can propagate into the building's internal wiring. 

Mathematically, the effectiveness of a lightning arrestor can be understood in terms of the voltage drop across the arrestor, given by Ohm's Law:
\[
V = I \cdot R
\]
where \( V \) is the voltage drop, \( I \) is the lightning current, and \( R \) is the resistance of the arrestor. By placing the arrestor at the entry point, the voltage drop is minimized, reducing the risk of damage to the internal circuitry.

Related concepts include the principles of grounding, surge protection, and the behavior of electrical systems under high-voltage conditions. Proper grounding ensures that the lightning current is safely dissipated into the earth, while surge protection devices (like arrestors) prevent transient voltages from damaging sensitive equipment.

% Prompt for generating a diagram: A diagram showing the placement of a lightning arrestor at the point where feed lines enter a building, with arrows indicating the path of lightning current to ground.