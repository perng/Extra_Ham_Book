\subsection{Lightning Protection Ground Rods Requirements}
\label{G0B11}

\begin{tcolorbox}[colback=gray!10!white,colframe=black!75!black,title=G0B11]
Which of the following is required for lightning protection ground rods?
\begin{enumerate}[label=\Alph*,noitemsep]
    \item They must be bonded to all buried water and gas lines
    \item Bends in ground wires must be made as close as possible to a right angle
    \item Lightning grounds must be connected to all ungrounded wiring
    \item \textbf{They must be bonded together with all other grounds}
\end{enumerate}
\end{tcolorbox}

\subsubsection{Intuitive Explanation}
Imagine you're building a fort to protect yourself from a thunderstorm. You wouldn't just stick a single metal rod into the ground and call it a day, right? You'd want to connect all your defenses together to make sure the lightning has a safe path to the ground. That's exactly what bonding all the ground rods together does—it ensures that if lightning strikes, it has a clear and safe path to follow, reducing the risk of damage.

\subsubsection{Advanced Explanation}
In lightning protection systems, grounding is crucial to safely dissipate the electrical energy from a lightning strike. Ground rods are used to provide a low-resistance path to the earth. According to the National Electrical Code (NEC) and other standards, all grounding electrodes, including ground rods, must be bonded together to form a single grounding system. This ensures that the electrical potential is equalized across all parts of the system, minimizing the risk of side flashes and other hazards.

The bonding of ground rods is typically achieved using a grounding conductor, which connects all the rods together. This conductor must be of sufficient size and properly installed to handle the high currents associated with a lightning strike. The resistance of the grounding system should be as low as possible, often less than 25 ohms, to ensure effective dissipation of the lightning energy.

Mathematically, the resistance \( R \) of a grounding system can be approximated using the formula:
\[
R = \frac{\rho}{2\pi L} \left( \ln\left(\frac{4L}{d}\right) - 1 \right)
\]
where \( \rho \) is the soil resistivity, \( L \) is the length of the ground rod, and \( d \) is the diameter of the rod. By bonding multiple rods together, the overall resistance of the grounding system is reduced, enhancing its effectiveness.

% Diagram Prompt: Generate a diagram showing multiple ground rods bonded together with a grounding conductor, illustrating the path of lightning current to the earth.