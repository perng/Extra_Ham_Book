\subsection{Purpose of a Power Supply Interlock}
\label{G0B12}

\begin{tcolorbox}[colback=gray!10!white,colframe=black!75!black,title=G0B12]
What is the purpose of a power supply interlock?
\begin{enumerate}[label=\Alph*)]
    \item To prevent unauthorized changes to the circuit that would void the manufacturer’s warranty
    \item To shut down the unit if it becomes too hot
    \item \textbf{To ensure that dangerous voltages are removed if the cabinet is opened}
    \item To shut off the power supply if too much voltage is produced
\end{enumerate}
\end{tcolorbox}

\subsubsection{Intuitive Explanation}
Imagine you have a toy robot that runs on batteries. Now, what if you accidentally opened the robot while it was still on? You might get a little shock, right? A power supply interlock is like a safety switch that makes sure the robot turns off as soon as you open it. This way, you don’t get zapped by any dangerous electricity. It’s like having a guardian angel for your electronics!

\subsubsection{Advanced Explanation}
A power supply interlock is a safety mechanism designed to disconnect the power supply when the cabinet or enclosure of an electronic device is opened. This is crucial in high-voltage systems where exposure to live circuits can be hazardous. The interlock typically consists of a switch or sensor that detects when the cabinet is opened and immediately cuts off the power supply, thereby removing any dangerous voltages.

In mathematical terms, the interlock can be modeled as a switch \( S \) in series with the power supply \( V \). When the cabinet is opened, the switch \( S \) opens, breaking the circuit:

\[
V_{\text{output}} = V \times S
\]

where \( S = 0 \) when the cabinet is open, ensuring \( V_{\text{output}} = 0 \).

This mechanism is essential in ensuring the safety of technicians and users who might need to access the internal components of the device. It is a fundamental aspect of electrical safety protocols and is often mandated by regulatory standards.

% Diagram prompt: Generate a diagram showing a power supply circuit with an interlock switch that opens when the cabinet is opened.