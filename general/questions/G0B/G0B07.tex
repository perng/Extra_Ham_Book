\subsection{Safety Harness Guidelines for Tower Climbing}
\label{G0B07}

\begin{tcolorbox}[colback=gray!10!white,colframe=black!75!black,title=G0B07]
Which of these choices should be observed when climbing a tower using a safety harness?
\begin{enumerate}[label=\Alph*)]
    \item Always hold on to the tower with one hand
    \item \textbf{Confirm that the harness is rated for the weight of the climber and that it is within its allowable service life}
    \item Ensure that all heavy tools are securely fastened to the harness
    \item All these choices are correct
\end{enumerate}
\end{tcolorbox}

\subsubsection{Intuitive Explanation}
Imagine you're climbing a really tall ladder, but instead of a ladder, it's a giant tower. You’re wearing a special belt called a safety harness to keep you from falling. Now, would you trust a belt that’s too small for you or one that’s super old and might break? Probably not! That’s why you need to make sure the harness fits your weight and isn’t too old. It’s like checking if your bike helmet still fits and isn’t cracked before you ride. Safety first!

\subsubsection{Advanced Explanation}
When climbing a tower, the safety harness is a critical piece of equipment designed to prevent falls and ensure the climber's safety. The harness must be rated for the climber's weight to ensure it can withstand the forces exerted during a fall. Additionally, the harness must be within its allowable service life, as materials degrade over time due to environmental factors like UV exposure, moisture, and mechanical wear. 

The force exerted on the harness during a fall can be calculated using the formula:
\[
F = m \cdot a
\]
where \( F \) is the force, \( m \) is the mass of the climber, and \( a \) is the acceleration due to gravity (approximately \( 9.81 \, \text{m/s}^2 \)). If the harness is not rated for this force, it could fail, leading to serious injury or death. Therefore, it is essential to confirm that the harness is both appropriately rated and within its service life before use.

Other considerations, such as securing heavy tools to the harness, are also important but secondary to ensuring the harness itself is safe and functional. Holding onto the tower with one hand is not a substitute for a properly functioning harness.

% Prompt for diagram: A diagram showing a climber on a tower with a safety harness, highlighting the weight rating and service life checkpoints.