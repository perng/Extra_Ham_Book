\subsection{Effect of Modulation Duty Cycle on RF Exposure}
\label{G0A07}

\begin{tcolorbox}[colback=gray!10!white,colframe=black!75!black,title=G0A07]
What is the effect of modulation duty cycle on RF exposure?
\begin{enumerate}[label=\Alph*,noitemsep]
    \item \textbf{A lower duty cycle permits greater power levels to be transmitted}
    \item A higher duty cycle permits greater power levels to be transmitted
    \item Low duty cycle transmitters are exempt from RF exposure evaluation requirements
    \item High duty cycle transmitters are exempt from RF exposure requirements
\end{enumerate}
\end{tcolorbox}

\subsubsection{Intuitive Explanation}
Imagine you have a flashlight that you can turn on and off really quickly. If you leave it on all the time (high duty cycle), it gets hot and uses a lot of battery. But if you blink it on and off quickly (low duty cycle), it stays cooler and uses less battery. In radio terms, a lower duty cycle means the transmitter is on for shorter periods, so it can use more power without overheating or causing too much RF exposure. It's like blinking the flashlight really fast to make it brighter without burning out.

\subsubsection{Advanced Explanation}
The duty cycle of a modulated signal is defined as the ratio of the time the signal is active (on) to the total period of the signal. Mathematically, it is expressed as:

\[
\text{Duty Cycle} = \frac{T_{\text{on}}}{T_{\text{total}}}
\]

where \( T_{\text{on}} \) is the time the signal is active, and \( T_{\text{total}} \) is the total period of the signal.

When the duty cycle is lower, the transmitter is active for a shorter duration, which reduces the average power output. This allows for higher peak power levels to be transmitted without exceeding the average power limits set by RF exposure regulations. The relationship between peak power (\( P_{\text{peak}} \)) and average power (\( P_{\text{avg}} \)) is given by:

\[
P_{\text{avg}} = P_{\text{peak}} \times \text{Duty Cycle}
\]

Thus, for a given average power limit, a lower duty cycle permits higher peak power levels. This is crucial in applications where high peak power is needed, such as in radar systems or certain communication protocols.

% Diagram Prompt: Generate a diagram showing the relationship between duty cycle, peak power, and average power in a modulated signal.