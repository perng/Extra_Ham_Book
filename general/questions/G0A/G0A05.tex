\subsection{RF Energy and Human Absorption}
\label{G0A05}

\begin{tcolorbox}[colback=gray!10!white,colframe=black!75!black,title=G0A05]
What must you do if an evaluation of your station shows that the RF energy radiated by your station exceeds permissible limits for possible human absorption?
\begin{enumerate}[label=\Alph*,noitemsep]
    \item \textbf{Take action to prevent human exposure to the excessive RF fields}
    \item File an Environmental Impact Statement (EIS-97) with the FCC
    \item Secure written permission from your neighbors to operate above the controlled MPE limits
    \item All these choices are correct
\end{enumerate}
\end{tcolorbox}

\subsubsection{Intuitive Explanation}
Imagine your radio station is like a giant microwave oven, but instead of heating food, it’s sending out invisible waves called RF (Radio Frequency) energy. If these waves are too strong, they can be harmful to people nearby, just like standing too close to a microwave can be bad for you. So, if your station is sending out too much RF energy, you need to turn it down or shield it to keep people safe. It’s like turning down the heat on your stove if the pot is boiling over!

\subsubsection{Advanced Explanation}
RF energy is a form of electromagnetic radiation that can be absorbed by the human body, leading to potential health risks such as tissue heating. The Maximum Permissible Exposure (MPE) limits are set by regulatory bodies like the FCC to ensure that RF energy levels remain safe for human exposure. If an evaluation of your station indicates that the RF energy exceeds these limits, you must take immediate action to mitigate the risk. This could involve reducing the power output, increasing the distance between the antenna and people, or installing shielding to block the RF energy. 

The calculation for determining whether RF energy exceeds MPE limits involves measuring the power density \( S \) at a given distance \( d \) from the antenna. The power density can be calculated using the formula:

\[ S = \frac{P}{4 \pi d^2} \]

where \( P \) is the power transmitted by the antenna. If \( S \) exceeds the MPE limit, corrective actions must be taken to ensure compliance and safety.

% Prompt for generating a diagram: A diagram showing the relationship between RF energy, distance from the antenna, and MPE limits would be helpful here.