\subsection{RF Energy and Human Body Tissue}
\label{G0A01}

\begin{tcolorbox}[colback=gray!10!white,colframe=black!75!black,title=G0A01]
What is one way that RF energy can affect human body tissue?
\begin{enumerate}[label=\Alph*),noitemsep]
    \item \textbf{It heats body tissue}
    \item It causes radiation poisoning
    \item It causes the blood count to reach a dangerously low level
    \item It cools body tissue
\end{enumerate}
\end{tcolorbox}

\subsubsection{Intuitive Explanation}
Imagine you’re holding a microwave oven (don’t actually do this!). When you heat up your leftover pizza, the microwave uses RF (radio frequency) energy to make the water molecules in the pizza wiggle around really fast, which heats it up. Now, think of your body as the pizza. RF energy can make the water molecules in your body wiggle too, which can heat up your tissues. So, RF energy can heat up your body, just like it heats up your pizza!

\subsubsection{Advanced Explanation}
RF energy, or radio frequency energy, is a form of non-ionizing electromagnetic radiation. When RF energy interacts with human body tissue, it primarily causes dielectric heating. This occurs because the electric field of the RF wave induces a force on the polar molecules (like water) in the tissue, causing them to rotate and align with the field. This rapid movement generates heat due to molecular friction.

The amount of heat generated can be described by the specific absorption rate (SAR), which is the rate at which energy is absorbed by the body when exposed to an RF electromagnetic field. The SAR is given by:

\[
\text{SAR} = \frac{\sigma |E|^2}{\rho}
\]

where:
\begin{itemize}
    \item \(\sigma\) is the conductivity of the tissue,
    \item \(E\) is the electric field strength,
    \item \(\rho\) is the mass density of the tissue.
\end{itemize}

This heating effect is the basis for various medical applications, such as diathermy, where RF energy is used to heat tissues for therapeutic purposes. However, excessive exposure to RF energy can lead to thermal damage, which is why safety standards limit the permissible exposure levels.

% Diagram prompt: Generate a diagram showing the interaction of RF energy with human tissue, illustrating the heating effect due to molecular friction.