\subsection{RF Exposure Determination}\label{G0A02}

\begin{tcolorbox}[colback=gray!10!white,colframe=black!75!black,title=G0A02]
Which of the following is used to determine RF exposure from a transmitted signal?
\begin{enumerate}[label=\Alph*]
    \item Its duty cycle
    \item Its frequency
    \item Its power density
    \item \textbf{All these choices are correct}
\end{enumerate}
\end{tcolorbox}

\subsubsection{Intuitive Explanation}
Imagine you're baking cookies. To make sure they turn out just right, you need to consider the oven temperature (frequency), how long you bake them (duty cycle), and how much dough you put in each cookie (power density). Similarly, to figure out how much RF exposure you're getting from a signal, you need to look at all these factors together. It's not just one thing—it's the whole recipe!

\subsubsection{Advanced Explanation}
To determine RF exposure from a transmitted signal, multiple factors must be considered:

1. \textbf{Duty Cycle}: This represents the fraction of time the transmitter is active during a given period. It is calculated as:
   \[
   \text{Duty Cycle} = \frac{\text{Transmit Time}}{\text{Total Time}}
   \]
   A higher duty cycle means more exposure over time.

2. \textbf{Frequency}: The frequency of the RF signal affects how it interacts with biological tissues. Higher frequencies can penetrate less deeply but may cause more localized heating.

3. \textbf{Power Density}: This is the amount of power per unit area and is given by:
   \[
   \text{Power Density} = \frac{P}{4\pi r^2}
   \]
   where \( P \) is the transmitted power and \( r \) is the distance from the source. Higher power density means greater exposure.

All these factors—duty cycle, frequency, and power density—are crucial in determining the overall RF exposure. Therefore, the correct answer is that all these choices are correct.

% Prompt for generating a diagram: A diagram showing the relationship between RF exposure, duty cycle, frequency, and power density would be helpful here.