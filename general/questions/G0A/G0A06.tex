\subsection{FCC RF Exposure Exemption Criteria}
\label{G0A06}

\begin{tcolorbox}[colback=gray!10!white,colframe=black!75!black,title=G0A06]
What must you do if your station fails to meet the FCC RF exposure exemption criteria?
\begin{enumerate}[label=\Alph*)]
    \item \textbf{Perform an RF Exposure Evaluation in accordance with FCC OET Bulletin 65}
    \item Contact the FCC for permission to transmit
    \item Perform an RF exposure evaluation in accordance with World Meteorological Organization guidelines
    \item Use an FCC-approved band-pass filter
\end{enumerate}
\end{tcolorbox}

\subsubsection{Intuitive Explanation}
Imagine you're baking cookies, and the recipe says you can only use a certain amount of sugar to make them safe to eat. If you accidentally use too much sugar, you need to check the recipe again to make sure your cookies are still safe. Similarly, if your radio station is emitting too much RF (radio frequency) energy, you need to follow the FCC's guidelines (like a recipe) to make sure it's safe for everyone around.

\subsubsection{Advanced Explanation}
The FCC (Federal Communications Commission) sets limits on the amount of RF energy that radio stations can emit to ensure public safety. These limits are based on guidelines provided in the FCC OET Bulletin 65, which outlines the methods for evaluating RF exposure. If a station exceeds the exemption criteria, it must conduct an RF Exposure Evaluation to determine the actual levels of RF energy being emitted and ensure they are within safe limits. This evaluation involves measuring the power output, antenna gain, and distance from the antenna to the public, and then comparing these values to the permissible exposure limits (PELs) specified by the FCC. The formula for calculating the power density \( S \) at a distance \( d \) from the antenna is given by:

\[ S = \frac{P \cdot G}{4 \pi d^2} \]

where \( P \) is the power output, \( G \) is the antenna gain, and \( d \) is the distance from the antenna. The calculated power density must be less than the PELs to ensure compliance with FCC regulations.

% Diagram prompt: Generate a diagram showing the relationship between power output, antenna gain, and distance in RF exposure evaluation.