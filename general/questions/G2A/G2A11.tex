\subsection{Who Should Respond to a Station in the Contiguous 48 States Calling CQ DX?}
\label{G2A11}

\begin{tcolorbox}[colback=gray!10!white,colframe=black!75!black,title=G2A11]
Generally, who should respond to a station in the contiguous 48 states calling “CQ DX”?
\begin{enumerate}[label=\Alph*,noitemsep]
    \item Any caller is welcome to respond
    \item Only stations in Germany
    \item \textbf{Any stations outside the lower 48 states}
    \item Only contest stations
\end{enumerate}
\end{tcolorbox}

\subsubsection{Intuitive Explanation}
Imagine you're at a big party, and someone shouts, Hey, anyone from out of town? They're not looking for the locals to answer; they want to hear from the people who traveled from far away. Similarly, when a radio station in the contiguous 48 states calls CQ DX, they're saying, Hey, anyone outside these 48 states? So, if you're outside these states, it's your time to shine and respond!

\subsubsection{Advanced Explanation}
The term CQ DX is a specific call used in amateur radio to indicate that the station is seeking contacts with distant stations, particularly those outside the contiguous 48 states. The CQ part is a general call to all stations, while DX stands for distance or distant stations. 

In this context, the correct response should come from stations located outside the contiguous 48 states. This is because the primary goal of a CQ DX call is to establish communication with distant or foreign stations, which can be more challenging and thus more rewarding for amateur radio operators.

The contiguous 48 states refer to the United States excluding Alaska and Hawaii. Therefore, stations in Alaska, Hawaii, or any other country are the intended respondents to a CQ DX call from the contiguous 48 states.

% Diagram prompt: A map showing the contiguous 48 states with an arrow pointing to stations outside this region responding to a CQ DX call.