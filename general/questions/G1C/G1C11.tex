\subsection{Measurement Specified by FCC Rules for Maximum Power}
\label{G1C11}

\begin{tcolorbox}[colback=gray!10!white,colframe=black!75!black,title=G1C11]
What measurement is specified by FCC rules that regulate maximum power?
\begin{enumerate}[label=\Alph*]
    \item RMS output from the transmitter
    \item RMS input to the antenna
    \item PEP input to the antenna
    \item \textbf{PEP output from the transmitter}
\end{enumerate}
\end{tcolorbox}

\subsubsection{Intuitive Explanation}
Imagine you have a super loud speaker, and the government wants to make sure you don’t blow out everyone’s eardrums. They don’t care about the average volume (RMS) or what’s going into the speaker (antenna). They want to know the absolute loudest sound (PEP) that comes out of the speaker (transmitter). That’s what the FCC is checking—how loud your radio can get at its peak!

\subsubsection{Advanced Explanation}
The Federal Communications Commission (FCC) regulates the maximum power output of transmitters to ensure they do not cause interference or exceed safety limits. The key measurement here is the Peak Envelope Power (PEP), which represents the maximum power level that the transmitter can produce during a signal cycle. This is different from Root Mean Square (RMS) power, which measures the average power over time.

The FCC specifies the PEP output from the transmitter because it directly relates to the potential for interference and the effective radiated power. The formula for PEP is given by:

\[
\text{PEP} = \frac{V_{\text{peak}}^2}{R}
\]

where \( V_{\text{peak}} \) is the peak voltage and \( R \) is the load resistance. This measurement ensures that the transmitter does not exceed the maximum allowable power, which is crucial for maintaining the integrity of the radio spectrum.

% Diagram Prompt: Generate a diagram showing the relationship between PEP and RMS power in a transmitter signal cycle.