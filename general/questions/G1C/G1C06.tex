\subsection{Limit for Transmitter Power on the 1.8 MHz Band}
\label{G1C06}

\begin{tcolorbox}[colback=gray!10!white,colframe=black!75!black,title=G1C06]
What is the limit for transmitter power on the 1.8 MHz band?
\begin{enumerate}[label=\Alph*,noitemsep]
    \item 200 watts PEP output
    \item 1000 watts PEP output
    \item 1200 watts PEP output
    \item \textbf{1500 watts PEP output}
\end{enumerate}
\end{tcolorbox}

\subsubsection{Intuitive Explanation}
Imagine you're at a concert, and the band is playing really loud. If they play too loud, the sound can damage your ears or even the speakers! Similarly, when you're transmitting radio signals, there's a limit to how powerful your transmitter can be. On the 1.8 MHz band, the maximum power allowed is like the volume knob on your stereo—it can go up to 1500 watts PEP output. This ensures that your signal is strong enough to reach far but not so strong that it causes problems for others.

\subsubsection{Advanced Explanation}
The 1.8 MHz band, also known as the 160-meter band, is part of the High Frequency (HF) spectrum. The Federal Communications Commission (FCC) regulates the maximum permissible transmitter power to prevent interference and ensure efficient use of the spectrum. The limit for transmitter power on this band is set at 1500 watts PEP (Peak Envelope Power) output. 

PEP is a measure of the maximum power level of a signal during one complete cycle of modulation. It is calculated as:
\[
\text{PEP} = \frac{V_{\text{peak}}^2}{R}
\]
where \( V_{\text{peak}} \) is the peak voltage and \( R \) is the load resistance. 

This power limit is crucial for maintaining the integrity of the radio spectrum and ensuring that all users can communicate effectively without causing harmful interference to each other.

% Diagram Prompt: Generate a diagram showing the power levels of different transmitter outputs on the 1.8 MHz band, highlighting the 1500 watts PEP limit.