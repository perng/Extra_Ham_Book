\subsection{Maximum Symbol Rate for RTTY or Data Emission Below 28 MHz}\label{G1C08}

\begin{tcolorbox}[colback=gray!10!white,colframe=black!75!black,title=G1C08]
What is the maximum symbol rate permitted for RTTY or data emission transmitted at frequencies below 28 MHz?
\begin{enumerate}[label=\Alph*,noitemsep]
    \item 56 kilobaud
    \item 19.6 kilobaud
    \item 1200 baud
    \item \textbf{300 baud}
\end{enumerate}
\end{tcolorbox}

\subsubsection{Intuitive Explanation}
Imagine you're sending a text message, but instead of using your phone, you're using a radio. The speed at which you can send these messages is limited by the rules of the radio world. For frequencies below 28 MHz, the rule is simple: you can't send messages faster than 300 baud. Think of it like a speed limit on a road—you can't drive faster than the posted limit, or you'll get in trouble. So, in this case, 300 baud is the speed limit for sending RTTY or data messages below 28 MHz.

\subsubsection{Advanced Explanation}
The maximum symbol rate for RTTY (Radio Teletype) or data emissions below 28 MHz is governed by regulatory standards to ensure efficient use of the radio spectrum and to minimize interference. The International Telecommunication Union (ITU) and national regulatory bodies, such as the Federal Communications Commission (FCC) in the United States, set these limits.

For frequencies below 28 MHz, the maximum permitted symbol rate is 300 baud. This limit is based on the bandwidth requirements and the need to avoid excessive interference with other users of the spectrum. The symbol rate, measured in baud, represents the number of symbol changes (or signaling events) per second. A lower symbol rate means that the signal occupies less bandwidth, which is crucial in the crowded HF (High Frequency) bands.

Mathematically, the relationship between symbol rate (\( R_s \)) and bandwidth (\( B \)) can be approximated by:
\[ B \approx R_s \]
For a symbol rate of 300 baud, the required bandwidth is approximately 300 Hz. This ensures that the signal remains within the allocated channel and does not interfere with adjacent channels.

In summary, the maximum symbol rate of 300 baud for RTTY or data emissions below 28 MHz is a regulatory requirement designed to maintain orderly and efficient use of the radio spectrum.

% Diagram prompt: A diagram showing the relationship between symbol rate and bandwidth in the context of RTTY or data emissions below 28 MHz.