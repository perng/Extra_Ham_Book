\subsection{Representation of Reactance}
\label{G5A11}

\begin{tcolorbox}[colback=gray!10!white,colframe=black!75!black,title=G5A11]
What letter is used to represent reactance?
\begin{enumerate}[label=\Alph*,noitemsep]
    \item Z
    \item \textbf{X}
    \item B
    \item Y
\end{enumerate}
\end{tcolorbox}

\subsubsection{Intuitive Explanation}
Imagine you're playing a game where you have to give nicknames to different characters. Reactance is like a superhero that resists changes in electrical current. Just like how Superman has the letter S on his chest, reactance has its own special letter: X. So, when you see X in electrical stuff, you know it's talking about reactance!

\subsubsection{Advanced Explanation}
Reactance is a property of electrical circuits that opposes the change in current due to inductance or capacitance. It is denoted by the letter \( X \). Reactance can be either inductive (\( X_L \)) or capacitive (\( X_C \)). The total reactance in a circuit is given by:
\[
X = X_L - X_C
\]
where \( X_L = 2\pi fL \) and \( X_C = \frac{1}{2\pi fC} \). Here, \( f \) is the frequency, \( L \) is the inductance, and \( C \) is the capacitance. Reactance is measured in ohms ($\Omega$) and plays a crucial role in determining the impedance of a circuit, which is represented by \( Z \).

% Prompt for generating a diagram: A simple circuit diagram showing an inductor and a capacitor in series, with labels for \( X_L \) and \( X_C \).