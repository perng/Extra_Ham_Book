\subsection{Unit of Reactance Measurement}
\label{G5A09}

\begin{tcolorbox}[colback=gray!10!white,colframe=black!75!black,title=G5A09]
What unit is used to measure reactance?
\begin{enumerate}[label=\Alph*,noitemsep]
    \item Farad
    \item \textbf{Ohm}
    \item Ampere
    \item Siemens
\end{enumerate}
\end{tcolorbox}

\subsubsection*{Intuitive Explanation}
Imagine you're trying to push a swing. If the swing is heavy, it's harder to push, right? Reactance is like the heaviness that electricity feels when it tries to flow through certain parts of a circuit. Just like we measure weight in pounds or kilograms, we measure this heaviness in ohms. So, when someone asks what unit measures reactance, think of the swing and say ohms!

\subsubsection*{Advanced Explanation}
Reactance is a property of electrical circuits that opposes the change in current due to inductance or capacitance. It is measured in ohms ($\Omega$), the same unit used for resistance. Reactance can be either inductive ($X_L$) or capacitive ($X_C$), and they are calculated as follows:

\[
X_L = 2\pi f L
\]
\[
X_C = \frac{1}{2\pi f C}
\]

where:
\begin{itemize}
    \item $X_L$ is the inductive reactance in ohms ($\Omega$),
    \item $X_C$ is the capacitive reactance in ohms ($\Omega$),
    \item $f$ is the frequency in hertz (Hz),
    \item $L$ is the inductance in henries (H),
    \item $C$ is the capacitance in farads (F).
\end{itemize}

Both inductive and capacitive reactance are measured in ohms because they represent the opposition to the flow of alternating current (AC) in a circuit. The unit ohm is universally recognized for measuring impedance, which is the combination of resistance and reactance in an AC circuit.

% Diagram Prompt: A simple circuit diagram showing an AC source connected to an inductor and a capacitor, with labels for inductive reactance (X_L) and capacitive reactance (X_C).