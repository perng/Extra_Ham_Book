\subsection{Inverse of Impedance}
\label{G5A07}

\begin{tcolorbox}[colback=gray!10!white,colframe=black!75!black,title=G5A07]
What is the term for the inverse of impedance?
\begin{enumerate}[label=\Alph*),noitemsep]
    \item Conductance
    \item Susceptance
    \item Reluctance
    \item \textbf{Admittance}
\end{enumerate}
\end{tcolorbox}

\subsubsection{Intuitive Explanation}
Imagine you're trying to push a shopping cart through a crowded store. The impedance is like the resistance you feel when trying to move the cart. Now, if you think about how easily you can push the cart, that's like the inverse of impedance. In the world of electronics, this ease is called \textbf{admittance}. It's like saying, How easily can electricity flow through this circuit? So, the inverse of impedance is admittance!

\subsubsection{Advanced Explanation}
In electrical engineering, impedance (\(Z\)) is a measure of opposition to the flow of alternating current (AC) in a circuit. It is a complex quantity that combines resistance (\(R\)) and reactance (\(X\)), and is given by:
\[
Z = R + jX
\]
where \(j\) is the imaginary unit.

The inverse of impedance is known as \textbf{admittance} (\(Y\)), which is also a complex quantity. Admittance is defined as:
\[
Y = \frac{1}{Z}
\]
Admittance can be broken down into its real and imaginary parts, known as conductance (\(G\)) and susceptance (\(B\)), respectively:
\[
Y = G + jB
\]
Conductance (\(G\)) is the real part of admittance and represents the ease with which current flows through a resistive element. Susceptance (\(B\)) is the imaginary part and represents the ease with which current flows through a reactive element (inductor or capacitor).

Therefore, the correct answer to the question is \textbf{Admittance}, as it is the term that directly represents the inverse of impedance.

% Diagram Prompt: Consider generating a diagram showing the relationship between impedance, admittance, conductance, and susceptance in a complex plane.