\subsection{Opposition to AC in an Inductor}
\label{G5A03}

\begin{tcolorbox}[colback=gray!10!white,colframe=black!75!black,title=G5A03]
Which of the following is opposition to the flow of alternating current in an inductor?
\begin{enumerate}[label=\Alph*]
    \item Conductance
    \item Reluctance
    \item Admittance
    \item \textbf{Reactance}
\end{enumerate}
\end{tcolorbox}

\subsubsection{Intuitive Explanation}
Imagine you're trying to push a swing. If you push it at the right time, it swings higher. But if you push it at the wrong time, it feels like the swing is pushing back. In an inductor, alternating current (AC) is like pushing the swing at different times. The inductor pushes back against the changing current, and this push back is called reactance. So, reactance is the opposition to the flow of AC in an inductor.

\subsubsection{Advanced Explanation}
In an inductor, the opposition to the flow of alternating current (AC) is known as reactance, specifically inductive reactance. Inductive reactance (\(X_L\)) is given by the formula:

\[
X_L = 2\pi f L
\]

where:
\begin{itemize}
    \item \(X_L\) is the inductive reactance in ohms (\(\Omega\)),
    \item \(f\) is the frequency of the AC in hertz (Hz),
    \item \(L\) is the inductance in henries (H).
\end{itemize}

Inductive reactance increases with both the frequency of the AC and the inductance of the inductor. This is because a higher frequency means the current is changing more rapidly, and a higher inductance means the inductor resists changes in current more strongly. Therefore, the correct answer is \textbf{Reactance}.

% Diagram prompt: A diagram showing an inductor connected to an AC source, with arrows indicating the opposition (reactance) to the flow of current.