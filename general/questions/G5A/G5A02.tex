\subsection{Understanding Reactance}
\label{G5A02}

\begin{tcolorbox}[colback=gray!10!white,colframe=black!75!black,title=G5A02]
What is reactance?
\begin{enumerate}[label=\Alph*)]
    \item Opposition to the flow of direct current caused by resistance
    \item \textbf{Opposition to the flow of alternating current caused by capacitance or inductance}
    \item Reinforcement of the flow of direct current caused by resistance
    \item Reinforcement of the flow of alternating current caused by capacitance or inductance
\end{enumerate}
\end{tcolorbox}

\subsubsection{Intuitive Explanation}
Imagine you're trying to push a swing. If you push it at just the right time, it swings higher. But if you push it at the wrong time, it doesn't move much. Reactance is like that wrong time push for electricity. It’s the opposition to the flow of alternating current (AC) caused by things like capacitors and inductors. These components don’t like sudden changes in current or voltage, so they push back in a way that makes it harder for the AC to flow smoothly.

\subsubsection{Advanced Explanation}
Reactance is a property of electrical circuits that opposes the flow of alternating current (AC) due to the presence of capacitance or inductance. It is denoted by the symbol \(X\) and is measured in ohms (\(\Omega\)). There are two types of reactance:

1. \textbf{Capacitive Reactance (\(X_C\))}: This is the opposition to the change in voltage across a capacitor. It is given by the formula:
   \[
   X_C = \frac{1}{2\pi f C}
   \]
   where \(f\) is the frequency of the AC signal and \(C\) is the capacitance.

2. \textbf{Inductive Reactance (\(X_L\))}: This is the opposition to the change in current through an inductor. It is given by the formula:
   \[
   X_L = 2\pi f L
   \]
   where \(f\) is the frequency of the AC signal and \(L\) is the inductance.

Reactance differs from resistance in that it depends on the frequency of the AC signal. At higher frequencies, capacitive reactance decreases, while inductive reactance increases. This frequency dependence is crucial in designing filters and tuning circuits in radio technology.

% Diagram Prompt: Generate a diagram showing the relationship between frequency and reactance for both capacitive and inductive reactance.