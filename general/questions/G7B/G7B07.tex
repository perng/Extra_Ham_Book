\subsection{Basic Components of a Sine Wave Oscillator}
\label{G7B07}

\begin{tcolorbox}[colback=gray!10!white,colframe=black!75!black,title=G7B07]
Which of the following are basic components of a sine wave oscillator?
\begin{enumerate}[label=\Alph*]
    \item An amplifier and a divider
    \item A frequency multiplier and a mixer
    \item A circulator and a filter operating in a feed-forward loop
    \item \textbf{A filter and an amplifier operating in a feedback loop}
\end{enumerate}
\end{tcolorbox}

\subsubsection{Intuitive Explanation}
Imagine you’re trying to keep a swing moving back and forth without stopping. You’d need two things: something to push the swing (like your legs) and something to control how fast it swings (like the length of the swing’s ropes). In the world of electronics, a sine wave oscillator works similarly. It needs an amplifier to keep the signal strong (like your legs pushing the swing) and a filter to control the frequency (like the ropes controlling the swing’s speed). Together, they create a smooth, repeating wave, just like the swing’s motion.

\subsubsection{Advanced Explanation}
A sine wave oscillator is designed to generate a continuous sinusoidal waveform. The fundamental components required for this are:

1. \textbf{Amplifier}: This component provides the necessary gain to sustain the oscillations. It compensates for any losses in the circuit, ensuring that the signal does not decay over time.

2. \textbf{Filter}: This component determines the frequency of the oscillation. It ensures that only the desired frequency is amplified and fed back into the system.

3. \textbf{Feedback Loop}: The feedback loop is crucial as it allows a portion of the output signal to be fed back into the input. This feedback must be positive and of the correct phase to sustain oscillations.

Mathematically, the condition for sustained oscillations is given by the Barkhausen criterion:
\[
|A \beta| = 1 \quad \text{and} \quad \angle A \beta = 0^\circ
\]
where \( A \) is the gain of the amplifier and \( \beta \) is the feedback factor.

In the context of the question, the correct answer is D, as it correctly identifies the essential components: a filter and an amplifier operating in a feedback loop.

% Diagram Prompt: Generate a diagram showing the basic components of a sine wave oscillator, including the amplifier, filter, and feedback loop.