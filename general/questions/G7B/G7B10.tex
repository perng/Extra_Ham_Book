\subsection{Linear Amplifier Characteristics}
\label{G7B10}

\begin{tcolorbox}[colback=gray!10!white,colframe=black!75!black,title=G7B10]
Which of the following describes a linear amplifier?  
\begin{enumerate}[label=\Alph*)]
    \item Any RF power amplifier used in conjunction with an amateur transceiver
    \item \textbf{An amplifier in which the output preserves the input waveform}
    \item A Class C high efficiency amplifier
    \item An amplifier used as a frequency multiplier
\end{enumerate}
\end{tcolorbox}

\subsubsection{Intuitive Explanation}
Imagine you have a photocopier. If you put a picture in and the copy looks exactly the same, that’s like a linear amplifier! It takes your signal (the picture) and makes a bigger version of it without changing how it looks. If the copier messed up the colors or made the picture blurry, that would be like a non-linear amplifier. So, a linear amplifier is like a perfect copier for signals!

\subsubsection{Advanced Explanation}
A linear amplifier is designed to amplify a signal while maintaining the fidelity of the input waveform. Mathematically, if the input signal is \( x(t) \), the output signal \( y(t) \) is given by:
\[
y(t) = A \cdot x(t)
\]
where \( A \) is the amplification factor. This relationship ensures that the output is a scaled version of the input without distortion. 

Linear amplifiers are crucial in applications where signal integrity is paramount, such as in communication systems. Non-linear amplifiers, like Class C amplifiers, introduce harmonics and distortions, which are undesirable in such contexts. Frequency multipliers, on the other hand, alter the frequency of the input signal, which is not the function of a linear amplifier.

% Diagram prompt: Generate a diagram showing the input and output waveforms of a linear amplifier, highlighting the preservation of the waveform shape.