\subsection{Amplifier Efficiency Classes}
\label{G7B02}

\begin{tcolorbox}[colback=gray!10!white,colframe=black!75!black,title=G7B02]
Which of these classes of amplifiers has the highest efficiency?
\begin{enumerate}[label=\Alph*)]
    \item Class A
    \item Class B
    \item Class AB
    \item \textbf{Class C}
\end{enumerate}
\end{tcolorbox}

\subsubsection{Intuitive Explanation}
Imagine you have a group of friends who are helping you carry a heavy box. Class A friends are always carrying the box, even when it's not heavy. Class B friends take turns, but they still work a lot. Class AB friends are a mix of both, working more than Class B but less than Class A. Now, Class C friends are the smartest—they only help when the box is super heavy, so they don't waste energy. That's why Class C amplifiers are the most efficient—they only work when they really need to!

\subsubsection{Advanced Explanation}
Amplifier efficiency is defined as the ratio of the output power to the input power, expressed as:
\[
\eta = \frac{P_{\text{out}}}{P_{\text{in}}} \times 100\%
\]
Class C amplifiers are designed to conduct for less than half of the input signal cycle, which minimizes power dissipation and maximizes efficiency. This is achieved by biasing the transistor such that it operates in the cutoff region for most of the input signal cycle, only conducting during the peaks. 

In contrast, Class A amplifiers conduct for the entire input signal cycle, leading to higher power dissipation and lower efficiency. Class B amplifiers conduct for half of the cycle, and Class AB amplifiers conduct for more than half but less than the full cycle, resulting in intermediate efficiency levels.

The efficiency of Class C amplifiers can theoretically reach up to 100\%, although practical designs typically achieve efficiencies around 70-80\%. This makes Class C amplifiers the most efficient among the classes listed.

% Diagram Prompt: Generate a diagram showing the conduction angles of Class A, B, AB, and C amplifiers for better visualization.