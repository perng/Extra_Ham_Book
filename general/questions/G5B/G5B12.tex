\subsection{RMS Voltage Across a Dummy Load}
\label{G5B12}

\begin{tcolorbox}[colback=gray!10!white,colframe=black!75!black,title=G5B12]
What is the RMS voltage across a 50-ohm dummy load dissipating 1200 watts?
\begin{enumerate}[label=\Alph*)]
    \item 173 volts
    \item \textbf{245 volts}
    \item 346 volts
    \item 692 volts
\end{enumerate}
\end{tcolorbox}

\subsubsection{Intuitive Explanation}
Imagine you have a big, heavy box (the dummy load) that you need to push across the floor. The amount of energy you use to push it is like the power (1200 watts). The floor is like the resistance (50 ohms). Now, the question is asking: how hard do you need to push (the voltage) to use that much energy? The answer is 245 volts, which is like saying you need to push with a certain amount of force to move the box using 1200 watts of energy.

\subsubsection{Advanced Explanation}
To find the RMS voltage across a load, we can use the relationship between power, voltage, and resistance. The formula for power in terms of voltage and resistance is:

\[
P = \frac{V^2}{R}
\]

Where:
\begin{itemize}
    \item \( P \) is the power in watts (1200 W),
    \item \( V \) is the RMS voltage in volts (unknown),
    \item \( R \) is the resistance in ohms (50 \(\Omega\)).
\end{itemize}

Rearranging the formula to solve for \( V \):

\[
V = \sqrt{P \times R}
\]

Substituting the given values:

\[
V = \sqrt{1200 \times 50} = \sqrt{60000} \approx 245 \text{ volts}
\]

Thus, the RMS voltage across the 50-ohm dummy load dissipating 1200 watts is approximately 245 volts.

% Diagram Prompt: Generate a diagram showing a 50-ohm dummy load connected to a power source with the RMS voltage calculated as 245 volts.