\subsection{Peak-to-Peak Voltage of a Sine Wave}
\label{G5B08}

\begin{tcolorbox}[colback=gray!10!white,colframe=black!75!black,title=G5B08]
What is the peak-to-peak voltage of a sine wave with an RMS voltage of 120 volts?
\begin{enumerate}[label=\Alph*,noitemsep]
    \item 84.8 volts
    \item 169.7 volts
    \item 240.0 volts
    \item \textbf{339.4 volts}
\end{enumerate}
\end{tcolorbox}

\subsubsection{Intuitive Explanation}
Imagine you’re on a swing, going back and forth. The highest point you reach is like the peak of a sine wave. Now, if you measure from the highest point to the lowest point, that’s the peak-to-peak distance. For a sine wave with an RMS (Root Mean Square) voltage of 120 volts, the peak-to-peak voltage is like measuring the total distance from the top of your swing to the bottom. It’s about 339.4 volts, which is more than double the RMS voltage. So, the correct answer is \textbf{D}.

\subsubsection{Advanced Explanation}
The RMS voltage of a sine wave is related to its peak voltage (\(V_{\text{peak}}\)) by the following equation:
\[
V_{\text{RMS}} = \frac{V_{\text{peak}}}{\sqrt{2}}
\]
Given \(V_{\text{RMS}} = 120\) volts, we can solve for \(V_{\text{peak}}\):
\[
V_{\text{peak}} = V_{\text{RMS}} \times \sqrt{2} = 120 \times 1.414 \approx 169.7 \text{ volts}
\]
The peak-to-peak voltage (\(V_{\text{pp}}\)) is twice the peak voltage:
\[
V_{\text{pp}} = 2 \times V_{\text{peak}} = 2 \times 169.7 \approx 339.4 \text{ volts}
\]
Therefore, the correct answer is \textbf{D}.

% Diagram Prompt: Generate a diagram showing a sine wave with labeled peak, peak-to-peak, and RMS voltage values.