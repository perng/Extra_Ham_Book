\subsection{Power Consumption of a 12 VDC Light Bulb}
\label{G5B04}

\begin{tcolorbox}[colback=gray!10!white,colframe=black!75!black,title=G5B04]
How many watts of electrical power are consumed by a 12 VDC light bulb that draws 0.2 amperes?
\begin{enumerate}[label=\Alph*)]
    \item \textbf{2.4 watts}
    \item 24 watts
    \item 6 watts
    \item 60 watts
\end{enumerate}
\end{tcolorbox}

\subsubsection{Intuitive Explanation}
Imagine your light bulb is like a tiny little robot that needs energy to work. The energy it needs comes from electricity. The electricity is like the robot's food. The robot eats 0.2 bites of electricity every second, and each bite is worth 12 energy points (volts). To find out how much energy the robot uses in total, you multiply the number of bites by the energy points per bite. So, 0.2 bites times 12 energy points equals 2.4 energy points per second, or 2.4 watts. That's how much energy your light bulb robot needs to shine!

\subsubsection{Advanced Explanation}
The power consumed by an electrical device can be calculated using the formula:
\[
P = V \times I
\]
where:
\begin{itemize}
    \item \( P \) is the power in watts (W),
    \item \( V \) is the voltage in volts (V),
    \item \( I \) is the current in amperes (A).
\end{itemize}

Given:
\[
V = 12 \, \text{V}, \quad I = 0.2 \, \text{A}
\]

Substituting the values into the formula:
\[
P = 12 \, \text{V} \times 0.2 \, \text{A} = 2.4 \, \text{W}
\]

Thus, the power consumed by the 12 VDC light bulb is 2.4 watts.

This calculation is based on Ohm's Law, which relates voltage, current, and resistance in an electrical circuit. In this case, the resistance of the light bulb can be calculated using:
\[
R = \frac{V}{I} = \frac{12 \, \text{V}}{0.2 \, \text{A}} = 60 \, \Omega
\]

Understanding these relationships is crucial for analyzing and designing electrical circuits.

% Prompt for generating a diagram: A simple circuit diagram showing a 12 VDC power source connected to a light bulb with a current of 0.2 A flowing through it.