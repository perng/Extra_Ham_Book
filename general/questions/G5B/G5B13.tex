\subsection{Output PEP of an Unmodulated Carrier}
\label{G5B13}

\begin{tcolorbox}[colback=gray!10!white,colframe=black!75!black,title=G5B13]
What is the output PEP of an unmodulated carrier if the average power is 1060 watts?
\begin{enumerate}[label=\Alph*),noitemsep]
    \item 530 watts
    \item \textbf{1060 watts}
    \item 1500 watts
    \item 2120 watts
\end{enumerate}
\end{tcolorbox}

\subsubsection{Intuitive Explanation}
Imagine you have a flashlight that shines with a steady, unchanging light. The brightness of this light is like the power of an unmodulated carrier in radio terms. If someone tells you the average brightness is 1060 watts, and asks what the peak brightness is, you'd say it's the same because the light doesn't flicker or change. So, the peak envelope power (PEP) is also 1060 watts. Easy, right?

\subsubsection{Advanced Explanation}
In radio communications, an unmodulated carrier is a continuous wave (CW) signal that does not vary in amplitude or frequency. The average power \( P_{\text{avg}} \) of such a signal is equal to its peak envelope power (PEP) because there are no variations in the signal's amplitude. Mathematically, this can be expressed as:

\[
P_{\text{PEP}} = P_{\text{avg}}
\]

Given that the average power \( P_{\text{avg}} \) is 1060 watts, the peak envelope power \( P_{\text{PEP}} \) is also 1060 watts. This is because the signal's amplitude remains constant over time, and thus the peak power does not exceed the average power.

\subsubsection{Related Concepts}
\begin{itemize}
    \item \textbf{Continuous Wave (CW):} A signal that is constant in amplitude and frequency, often used in Morse code transmissions.
    \item \textbf{Average Power:} The mean power of a signal over a period of time.
    \item \textbf{Peak Envelope Power (PEP):} The highest instantaneous power of a signal, typically measured over one cycle of the modulation envelope.
\end{itemize}

% Diagram Prompt: A simple graph showing a constant amplitude signal over time, labeled with Unmodulated Carrier and indicating the average power and PEP as the same value.