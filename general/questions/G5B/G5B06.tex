\subsection{Peak Envelope Power Calculation}
\label{G5B06}

\begin{tcolorbox}[colback=gray!10!white,colframe=black!75!black,title=G5B06]
What is the PEP produced by 200 volts peak-to-peak across a 50-ohm dummy load?
\begin{enumerate}[label=\Alph*)]
    \item 1.4 watts
    \item \textbf{100 watts}
    \item 353.5 watts
    \item 400 watts
\end{enumerate}
\end{tcolorbox}

\subsubsection*{Intuitive Explanation}
Imagine you have a water hose, and you're trying to measure how much water it can spray out. The voltage is like the pressure in the hose, and the dummy load is like the nozzle. If you have a certain pressure (200 volts peak-to-peak) and a specific nozzle size (50 ohms), you can figure out how much water (power) is coming out. In this case, it's 100 watts, which is like a strong, steady stream of water!

\subsubsection*{Advanced Explanation}
To calculate the Peak Envelope Power (PEP), we use the formula for power in a resistive load:

\[
P = \frac{V_{\text{rms}}^2}{R}
\]

First, we need to find the root mean square (RMS) voltage from the peak-to-peak voltage. The peak-to-peak voltage \(V_{\text{pp}}\) is 200 volts. The peak voltage \(V_{\text{peak}}\) is half of that:

\[
V_{\text{peak}} = \frac{V_{\text{pp}}}{2} = \frac{200}{2} = 100 \text{ volts}
\]

The RMS voltage \(V_{\text{rms}}\) is then:

\[
V_{\text{rms}} = \frac{V_{\text{peak}}}{\sqrt{2}} = \frac{100}{\sqrt{2}} \approx 70.71 \text{ volts}
\]

Now, we can calculate the power:

\[
P = \frac{V_{\text{rms}}^2}{R} = \frac{(70.71)^2}{50} = \frac{5000}{50} = 100 \text{ watts}
\]

Thus, the PEP produced by 200 volts peak-to-peak across a 50-ohm dummy load is 100 watts.

% Diagram prompt: A diagram showing the relationship between peak-to-peak voltage, RMS voltage, and power in a resistive load could be helpful here.