\subsection{Power Consumption in a Resistor}
\label{G5B05}

\begin{tcolorbox}[colback=gray!10!white,colframe=black!75!black,title=G5B05]
How many watts are consumed when a current of 7.0 milliamperes flows through a 1,250-ohm resistance?
\begin{enumerate}[label=\Alph*)]
    \item \textbf{Approximately 61 milliwatts}
    \item Approximately 61 watts
    \item Approximately 11 milliwatts
    \item Approximately 11 watts
\end{enumerate}
\end{tcolorbox}

\subsubsection{Intuitive Explanation}
Imagine you have a tiny river (the current) flowing through a narrow canyon (the resistor). The river is only 7.0 milliamperes wide, and the canyon is 1,250 ohms deep. Now, think of the power consumed as the energy the river uses to push through the canyon. It’s like the river is doing a little workout! In this case, the river’s workout is pretty light, so it only uses about 61 milliwatts of energy. That’s like a tiny, tiny light bulb glowing very faintly.

\subsubsection{Advanced Explanation}
To calculate the power consumed in a resistor, we use the formula:
\[
P = I^2 \times R
\]
where \( P \) is the power in watts, \( I \) is the current in amperes, and \( R \) is the resistance in ohms.

Given:
\[
I = 7.0 \, \text{mA} = 7.0 \times 10^{-3} \, \text{A}
\]
\[
R = 1,250 \, \Omega
\]

Substitute the values into the formula:
\[
P = (7.0 \times 10^{-3})^2 \times 1,250
\]
\[
P = 49 \times 10^{-6} \times 1,250
\]
\[
P = 61.25 \times 10^{-3} \, \text{W} = 61.25 \, \text{mW}
\]

Thus, the power consumed is approximately 61 milliwatts.

This calculation shows how the power dissipated in a resistor depends on both the current flowing through it and the resistance of the resistor. The relationship is quadratic with respect to current, meaning that even small increases in current can lead to significant increases in power dissipation.

% Diagram Prompt: A simple circuit diagram showing a resistor with a current flowing through it, labeled with the given values.