\subsection{RMS Voltage of a Sine Wave}
\label{G5B09}

\begin{tcolorbox}[colback=gray!10!white,colframe=black!75!black,title=G5B09]
What is the RMS voltage of a sine wave with a value of 17 volts peak?
\begin{enumerate}[label=\Alph*),noitemsep]
    \item 8.5 volts
    \item \textbf{12 volts}
    \item 24 volts
    \item 34 volts
\end{enumerate}
\end{tcolorbox}

\subsubsection{Intuitive Explanation}
Imagine you have a sine wave that goes up and down like a roller coaster. The highest point it reaches is 17 volts. But we want to know the average voltage that would give the same power as this wavy voltage. This average is called the RMS (Root Mean Square) voltage. For a sine wave, the RMS voltage is about 0.707 times the peak voltage. So, if the peak is 17 volts, the RMS voltage is roughly 12 volts. It's like saying, Even though the roller coaster goes up to 17 volts, on average, it feels like 12 volts.

\subsubsection{Advanced Explanation}
The RMS voltage of a sine wave is calculated using the formula:
\[
V_{\text{RMS}} = \frac{V_{\text{peak}}}{\sqrt{2}}
\]
where \( V_{\text{peak}} \) is the peak voltage. For a sine wave with a peak voltage of 17 volts, the RMS voltage is:
\[
V_{\text{RMS}} = \frac{17}{\sqrt{2}} \approx 12 \text{ volts}
\]
The factor \( \frac{1}{\sqrt{2}} \) (approximately 0.707) is derived from the mathematical process of taking the square root of the mean (average) of the square of the voltage over one cycle of the sine wave. This RMS value is crucial because it represents the equivalent DC voltage that would deliver the same power to a load as the AC voltage does.

% Diagram prompt: Generate a diagram showing a sine wave with a peak voltage of 17 volts and indicate the RMS voltage of 12 volts on the same graph.