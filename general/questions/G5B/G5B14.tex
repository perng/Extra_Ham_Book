\subsection{Output PEP Calculation}
\label{G5B14}

\begin{tcolorbox}[colback=gray!10!white,colframe=black!75!black,title=G5B14]
What is the output PEP of 500 volts peak-to-peak across a 50-ohm load?
\begin{enumerate}[label=\Alph*,noitemsep]
    \item 8.75 watts
    \item \textbf{625 watts}
    \item 2500 watts
    \item 5000 watts
\end{enumerate}
\end{tcolorbox}

\subsubsection{Intuitive Explanation}
Imagine you have a water hose, and you're trying to measure how much water is coming out. The voltage is like the pressure of the water, and the load is like the size of the hose. If you have a lot of pressure (voltage) and a small hose (load), you can figure out how much water (power) is coming out. In this case, 500 volts peak-to-peak across a 50-ohm load gives us 625 watts of power. That's like a lot of water coming out of your hose!

\subsubsection{Advanced Explanation}
To calculate the Peak Envelope Power (PEP), we use the formula for power in a resistive load:

\[
P = \frac{V_{\text{rms}}^2}{R}
\]

First, we need to convert the peak-to-peak voltage (\(V_{\text{pp}}\)) to the root mean square voltage (\(V_{\text{rms}}\)):

\[
V_{\text{rms}} = \frac{V_{\text{pp}}}{2\sqrt{2}}
\]

Given \(V_{\text{pp}} = 500\) volts and \(R = 50\) ohms, we can calculate \(V_{\text{rms}}\):

\[
V_{\text{rms}} = \frac{500}{2\sqrt{2}} = \frac{500}{2 \times 1.414} \approx 176.78 \text{ volts}
\]

Now, we can calculate the power:

\[
P = \frac{(176.78)^2}{50} = \frac{31250}{50} = 625 \text{ watts}
\]

Thus, the output PEP is 625 watts.

% Diagram prompt: Generate a diagram showing the relationship between peak-to-peak voltage, RMS voltage, and power across a resistive load.