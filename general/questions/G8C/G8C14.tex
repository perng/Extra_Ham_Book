\subsection{Waterfall Display Description}
\label{G8C14}

\begin{tcolorbox}[colback=gray!10!white,colframe=black!75!black,title=G8C14]
Which of the following describes a waterfall display?
\begin{enumerate}[label=\Alph*),noitemsep]
    \item Frequency is horizontal, signal strength is vertical, time is intensity
    \item Frequency is vertical, signal strength is intensity, time is horizontal
    \item \textbf{Frequency is horizontal, signal strength is intensity, time is vertical}
    \item Frequency is vertical, signal strength is horizontal, time is intensity
\end{enumerate}
\end{tcolorbox}

\subsubsection{Intuitive Explanation}
Imagine you're watching a waterfall, but instead of water, you're seeing signals! The waterfall display is like a magical picture that shows how signals change over time. The frequency (how high or low the signal sounds) is shown from left to right, like the width of the waterfall. The signal strength (how loud or quiet the signal is) is shown by how bright or dark the colors are. And time? That’s shown from top to bottom, like the water flowing down. So, it’s like a colorful, flowing picture of signals!

\subsubsection{Advanced Explanation}
A waterfall display is a graphical representation used in signal analysis to visualize the frequency spectrum of a signal over time. The x-axis represents frequency, the y-axis represents time, and the intensity (or color) represents the signal strength. This type of display is particularly useful for identifying transient signals or monitoring frequency usage over time.

Mathematically, the waterfall display can be represented as a three-dimensional plot where:
\begin{itemize}
    \item The x-axis (\(f\)) represents frequency.
    \item The y-axis (\(t\)) represents time.
    \item The z-axis (\(S(f, t)\)) represents the signal strength at a given frequency and time.
\end{itemize}

The intensity of the color at each point \((f, t)\) corresponds to the magnitude of \(S(f, t)\). This allows for a comprehensive visualization of how the frequency components of a signal evolve over time.

% Prompt for generating a diagram: 
% A diagram showing a waterfall display with frequency on the x-axis, time on the y-axis, and signal strength represented by color intensity.