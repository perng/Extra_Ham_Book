\subsection{Amateur Band Sharing with Wi-Fi}
\label{G8C01}

\begin{tcolorbox}[colback=gray!10!white,colframe=black!75!black,title=G8C01]
On what band do amateurs share channels with the unlicensed Wi-Fi service?
\begin{enumerate}[label=\Alph*,noitemsep]
    \item 432 MHz
    \item 902 MHz
    \item \textbf{2.4 GHz}
    \item 10.7 GHz
\end{enumerate}
\end{tcolorbox}

\subsubsection{Intuitive Explanation}
Imagine the radio spectrum as a big highway with different lanes. Some lanes are reserved for specific vehicles (like emergency services), while others are open for everyone to use. The 2.4 GHz band is like a shared lane where both amateur radio operators and Wi-Fi devices can drive. It’s a popular lane because it’s wide enough for many devices to use without causing too much traffic jam. So, when you’re using Wi-Fi at home, you’re sharing the same lane with amateur radio enthusiasts!

\subsubsection{Advanced Explanation}
The 2.4 GHz band is part of the Industrial, Scientific, and Medical (ISM) radio bands, which are internationally reserved for unlicensed use. This band ranges from 2.400 GHz to 2.4835 GHz. Amateur radio operators are allowed to operate in this band under specific regulations, sharing the spectrum with Wi-Fi devices, Bluetooth devices, and other unlicensed services. The sharing is managed through frequency coordination and power limits to minimize interference.

Mathematically, the frequency range can be expressed as:
\[ f_{\text{min}} = 2.400 \, \text{GHz} \]
\[ f_{\text{max}} = 2.4835 \, \text{GHz} \]

The wavelength (\(\lambda\)) of a signal in this band can be calculated using the formula:
\[ \lambda = \frac{c}{f} \]
where \(c\) is the speed of light (\(3 \times 10^8 \, \text{m/s}\)) and \(f\) is the frequency. For example, at 2.4 GHz:
\[ \lambda = \frac{3 \times 10^8}{2.4 \times 10^9} = 0.125 \, \text{m} \]

This band is particularly useful for both amateur radio and Wi-Fi due to its balance between range and data throughput. The higher frequency allows for higher data rates, while the wavelength is still long enough to provide reasonable coverage.

% Diagram Prompt: Generate a diagram showing the frequency spectrum of the 2.4 GHz band, highlighting the shared usage between amateur radio and Wi-Fi.