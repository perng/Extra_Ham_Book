\subsection{Inductor Operation Above Self-Resonant Frequency}
\label{G6A11}

\begin{tcolorbox}[colback=gray!10!white,colframe=black!75!black,title=G6A11]
What happens when an inductor is operated above its self-resonant frequency?
\begin{enumerate}[label=\Alph*,noitemsep]
    \item Its reactance increases
    \item Harmonics are generated
    \item \textbf{It becomes capacitive}
    \item Catastrophic failure is likely
\end{enumerate}
\end{tcolorbox}

\subsubsection*{Intuitive Explanation}
Imagine you have a spring that you can stretch and compress. When you push and pull it at just the right speed, it bounces back perfectly. This is like the inductor at its self-resonant frequency. But if you start shaking it really fast (above its natural speed), it doesn't bounce back the same way—it starts acting more like a squishy cushion. Similarly, when an inductor is used above its self-resonant frequency, it stops acting like an inductor and starts behaving like a capacitor. It’s like the spring forgot how to spring!

\subsubsection*{Advanced Explanation}
An inductor has a self-resonant frequency (SRF) determined by its inductance \(L\) and parasitic capacitance \(C_p\). Below the SRF, the inductor behaves as expected, with its reactance \(X_L = 2\pi f L\) increasing with frequency. However, above the SRF, the parasitic capacitance dominates, and the reactance becomes negative, indicating capacitive behavior. The impedance \(Z\) of the inductor can be modeled as:

\[
Z = j\omega L + \frac{1}{j\omega C_p}
\]

At frequencies above the SRF, the term \(\frac{1}{j\omega C_p}\) dominates, making the impedance negative and the inductor effectively capacitive. This transition is critical in RF circuits, where operating above the SRF can lead to unintended circuit behavior.

% Diagram prompt: A graph showing the impedance of an inductor vs. frequency, with the self-resonant frequency marked and the transition from inductive to capacitive behavior clearly indicated.