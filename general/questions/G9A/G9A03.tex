\subsection{Nominal Characteristic Impedance of Window Line}
\label{G9A03}

\begin{tcolorbox}[colback=gray!10!white,colframe=black!75!black,title=G9A03]
What is the nominal characteristic impedance of “window line” transmission line?
\begin{enumerate}[label=\Alph*)]
    \item 50 ohms
    \item 75 ohms
    \item 100 ohms
    \item \textbf{450 ohms}
\end{enumerate}
\end{tcolorbox}

\subsubsection{Intuitive Explanation}
Imagine you're trying to send a message through a long, narrow tunnel. The tunnel's shape and size affect how easily your message can travel. In the world of radio, window line is like a special tunnel for signals. Unlike the usual tunnels (like coaxial cables) that have lower impedance, window line has a much higher impedance—450 ohms. This means it’s designed to handle signals in a unique way, making it perfect for certain types of antennas. So, if someone asks you about the impedance of window line, you can confidently say it’s 450 ohms!

\subsubsection{Advanced Explanation}
The characteristic impedance of a transmission line is a measure of how the line resists the flow of electrical energy. It is determined by the physical properties of the line, such as its geometry and the materials used. For window line, which is a type of open-wire transmission line, the nominal characteristic impedance is typically 450 ohms. This high impedance is due to the large spacing between the conductors and the air dielectric, which reduces the capacitance and increases the inductance per unit length.

The characteristic impedance \( Z_0 \) of a transmission line can be calculated using the formula:

\[
Z_0 = \sqrt{\frac{L}{C}}
\]

where \( L \) is the inductance per unit length and \( C \) is the capacitance per unit length. For window line, the large spacing between conductors results in a low capacitance and a relatively high inductance, leading to the high characteristic impedance of 450 ohms.

This high impedance makes window line particularly suitable for certain types of antennas, such as dipoles, where a good match between the antenna and the transmission line is crucial for efficient signal transmission.

% Diagram prompt: A diagram showing the structure of window line transmission line, highlighting the spacing between conductors and the air dielectric, would be helpful for visual understanding.