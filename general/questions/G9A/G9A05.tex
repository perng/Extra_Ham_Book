\subsection{Attenuation of Coaxial Cable with Frequency}
\label{G9A05}

\begin{tcolorbox}[colback=gray!10!white,colframe=black!75!black,title=G9A05]
How does the attenuation of coaxial cable change with increasing frequency?
\begin{enumerate}[label=\Alph*]
    \item Attenuation is independent of frequency
    \item \textbf{Attenuation increases}
    \item Attenuation decreases
    \item Attenuation follows Marconi’s Law of Attenuation
\end{enumerate}
\end{tcolorbox}

\subsubsection*{Intuitive Explanation}
Imagine you're trying to shout a message through a long, narrow tunnel. If you shout slowly (low frequency), your voice travels pretty far. But if you start shouting really fast (high frequency), your voice gets weaker and weaker as it travels through the tunnel. That's kind of what happens with coaxial cables! As the frequency of the signal increases, the cable gets tired and the signal weakens more quickly. So, the higher the frequency, the more the signal gets lost along the way.

\subsubsection*{Advanced Explanation}
Attenuation in coaxial cables is primarily due to two factors: conductor loss and dielectric loss. Conductor loss is caused by the resistance of the inner and outer conductors, which increases with frequency due to the skin effect. The skin effect causes the current to flow more on the surface of the conductor as frequency increases, effectively reducing the cross-sectional area through which current flows and increasing resistance. Dielectric loss is due to the imperfect insulating material between the conductors, which dissipates energy as heat. This loss also increases with frequency.

The total attenuation \( \alpha \) in a coaxial cable can be expressed as:
\[
\alpha = \alpha_c + \alpha_d
\]
where \( \alpha_c \) is the conductor loss and \( \alpha_d \) is the dielectric loss. Both \( \alpha_c \) and \( \alpha_d \) increase with frequency, leading to an overall increase in attenuation as frequency increases.

% Diagram Prompt: Generate a diagram showing the relationship between frequency and attenuation in a coaxial cable, illustrating how both conductor loss and dielectric loss contribute to the total attenuation.