\subsection{Standing Wave Ratio Calculation}
\label{G9A10}

\begin{tcolorbox}[colback=gray!10!white,colframe=black!75!black,title=G9A10]
What standing wave ratio results from connecting a 50-ohm feed line to a 10-ohm resistive load?
\begin{enumerate}[label=\Alph*,noitemsep]
    \item 2:1
    \item 1:2
    \item 1:5
    \item \textbf{5:1}
\end{enumerate}
\end{tcolorbox}

\subsubsection{Intuitive Explanation}
Imagine you're trying to push a swing. If the swing is too light, your push doesn't do much, and if it's too heavy, you can't move it at all. Now, think of the 50-ohm feed line as your push and the 10-ohm load as the swing. The mismatch between the push and the swing creates a standing wave, which is like the swing moving back and forth unevenly. The ratio of this unevenness is 5:1, meaning the swing moves five times more in one direction than the other. So, the standing wave ratio is 5:1!

\subsubsection{Advanced Explanation}
The standing wave ratio (SWR) is a measure of impedance mismatch between a transmission line and its load. It is calculated using the formula:

\[
\text{SWR} = \frac{Z_0}{Z_L} \quad \text{if} \quad Z_0 > Z_L
\]

where \( Z_0 \) is the characteristic impedance of the feed line (50 ohms) and \( Z_L \) is the load impedance (10 ohms). Plugging in the values:

\[
\text{SWR} = \frac{50}{10} = 5
\]

Thus, the SWR is 5:1. This indicates a significant mismatch, which can lead to power loss and potential damage to the transmitter. Understanding SWR is crucial for optimizing radio frequency (RF) systems and ensuring efficient power transfer.

% Diagram prompt: A diagram showing a 50-ohm feed line connected to a 10-ohm resistive load, with arrows indicating the direction of the standing wave and labels for SWR calculation.