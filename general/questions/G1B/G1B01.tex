\subsection{Maximum Height for Antenna Structure Without FAA Notification and FCC Registration}
\label{G1B01}

\begin{tcolorbox}[colback=gray!10!white,colframe=black!75!black,title=G1B01]
What is the maximum height above ground for an antenna structure not near a public use airport without requiring notification to the FAA and registration with the FCC?
\begin{enumerate}[label=\Alph*,noitemsep]
    \item 50 feet
    \item 100 feet
    \item \textbf{200 feet}
    \item 250 feet
\end{enumerate}
\end{tcolorbox}

\subsubsection{Intuitive Explanation}
Imagine you're building a giant flagpole in your backyard. You want it to be as tall as possible, but you don't want to bother the airport or fill out a bunch of paperwork. The magic number here is 200 feet. If your flagpole (or antenna) is shorter than 200 feet, you're good to go! No need to call the airport or register it with the government. It's like the no permission needed height limit for your backyard projects.

\subsubsection{Advanced Explanation}
The Federal Aviation Administration (FAA) and the Federal Communications Commission (FCC) have specific regulations regarding the height of antenna structures to ensure they do not interfere with aviation safety. According to these regulations, any antenna structure that exceeds 200 feet above ground level (AGL) must be registered with the FCC and the FAA must be notified. This is to prevent potential hazards to aircraft, especially in areas not near public use airports. 

The calculation for determining whether an antenna requires notification and registration is straightforward:
\[
\text{Height of Antenna} \leq 200 \text{ feet}
\]
If the height is less than or equal to 200 feet, no further action is required. If it exceeds this height, the appropriate regulatory bodies must be informed. This regulation ensures that all structures are accounted for and do not pose a risk to aviation safety.

% Prompt for generating a diagram: A diagram showing an antenna structure with a height marker at 200 feet, indicating the threshold for FAA notification and FCC registration.