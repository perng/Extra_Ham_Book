\subsection{Power Limit for Beacon Stations}
\label{G1B10}

\begin{tcolorbox}[colback=gray!10!white,colframe=black!75!black,title=G1B10]
What is the power limit for beacon stations?
\begin{enumerate}[label=\Alph*,noitemsep]
    \item 10 watts PEP output
    \item 20 watts PEP output
    \item \textbf{100 watts PEP output}
    \item 200 watts PEP output
\end{enumerate}
\end{tcolorbox}

\subsubsection{Intuitive Explanation}
Imagine you're trying to send a message using a flashlight. If the flashlight is too dim (like 10 watts), no one will see it. If it's too bright (like 200 watts), it might blind people or use up all your battery. A beacon station is like a special flashlight that helps pilots or ships know where they are. The perfect brightness for this flashlight is 100 watts—it’s bright enough to be seen from far away but not so bright that it causes problems. So, the power limit for beacon stations is 100 watts PEP output.

\subsubsection{Advanced Explanation}
Beacon stations are used to transmit continuous signals for navigation, identification, or other purposes. The power limit for these stations is regulated to ensure efficient communication without causing interference or excessive power consumption. The Peak Envelope Power (PEP) is a measure of the maximum power output during a transmission cycle.

The Federal Communications Commission (FCC) and other regulatory bodies have set the power limit for beacon stations at 100 watts PEP output. This limit is chosen to balance the need for sufficient signal strength with the need to minimize interference and power usage. 

Mathematically, PEP is calculated as:
\[
\text{PEP} = \frac{V_{\text{peak}}^2}{R}
\]
where \( V_{\text{peak}} \) is the peak voltage and \( R \) is the load resistance. For beacon stations, this calculation ensures that the power output does not exceed the regulatory limit of 100 watts PEP.

Understanding this limit is crucial for designing and operating beacon stations efficiently while complying with legal requirements.

% Prompt for diagram: A diagram showing the relationship between power output, distance, and signal strength for beacon stations could be helpful here.