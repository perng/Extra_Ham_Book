\subsection{On What HF Frequencies Are Automatically Controlled Beacons Permitted?}
\label{G1B09}

\begin{tcolorbox}[colback=gray!10!white,colframe=black!75!black,title=G1B09]
On what HF frequencies are automatically controlled beacons permitted?
\begin{enumerate}[label=\Alph*)]
    \item On any frequency if power is less than 1 watt
    \item On any frequency if transmissions are in Morse code
    \item 21.08 MHz to 21.09 MHz
    \item \textbf{28.20 MHz to 28.30 MHz}
\end{enumerate}
\end{tcolorbox}

\subsubsection{Intuitive Explanation}
Imagine you have a bunch of walkie-talkies, and you want to set up a special kind of walkie-talkie that sends out signals automatically, like a lighthouse for radio waves. But you can't just use any channel you want; there are rules! The rule for these special walkie-talkies (called beacons) is that they can only use a specific channel on the radio dial. In this case, that channel is between 28.20 MHz and 28.30 MHz. So, if you want to set up one of these beacons, you have to tune it to this specific range, just like tuning your radio to your favorite station.

\subsubsection{Advanced Explanation}
Automatically controlled beacons are specialized transmitters that operate on specific frequency ranges within the High Frequency (HF) spectrum. The HF spectrum ranges from 3 MHz to 30 MHz and is divided into various bands allocated for different purposes, including amateur radio, broadcasting, and beacon operations.

The International Telecommunication Union (ITU) and national regulatory bodies, such as the Federal Communications Commission (FCC) in the United States, allocate specific frequency ranges for beacon operations to avoid interference with other services. In this case, the frequency range allocated for automatically controlled beacons is 28.20 MHz to 28.30 MHz. This range falls within the 10-meter amateur radio band, which is commonly used for long-distance communication.

The correct answer, \textbf{D}, specifies this exact frequency range. The other options are incorrect because:
\begin{itemize}
    \item Option A is incorrect because power limitations do not determine the frequency allocation for beacons.
    \item Option B is incorrect because the mode of transmission (Morse code) does not determine the frequency allocation.
    \item Option C is incorrect because 21.08 MHz to 21.09 MHz is not the allocated range for automatically controlled beacons.
\end{itemize}

Understanding the allocation of frequency bands and the regulations governing their use is crucial for anyone involved in radio communications, especially when setting up specialized equipment like beacons.

% Prompt for generating a diagram: A frequency spectrum chart showing the HF bands and highlighting the 28.20 MHz to 28.30 MHz range for automatically controlled beacons.