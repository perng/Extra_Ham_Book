\subsection{Upper Frequency Limit for BNC Connectors}
\label{G6B04}

\begin{tcolorbox}[colback=gray!10!white,colframe=black!75!black,title=G6B04]
What is a typical upper frequency limit for low SWR operation of 50-ohm BNC connectors?
\begin{enumerate}[label=\Alph*,noitemsep]
    \item 50 MHz
    \item 500 MHz
    \item \textbf{4 GHz}
    \item 40 GHz
\end{enumerate}
\end{tcolorbox}

\subsubsection{Intuitive Explanation}
Imagine you’re trying to send a message through a pipe. If the pipe is too small or the message is too fast, it might get stuck or distorted. BNC connectors are like special pipes for radio signals. They work really well up to a certain speed, but if you go too fast (like 4 GHz), they start to struggle. So, 4 GHz is the speed limit where these connectors can still do their job without messing up the message.

\subsubsection{Advanced Explanation}
BNC (Bayonet Neill–Concelman) connectors are commonly used in RF (Radio Frequency) applications due to their ease of use and reliable performance. The upper frequency limit for low SWR (Standing Wave Ratio) operation is determined by the connector's design and the wavelength of the signal. 

For a 50-ohm BNC connector, the typical upper frequency limit is around 4 GHz. Beyond this frequency, the connector's impedance may not remain constant, leading to reflections and increased SWR. The SWR is a measure of how well the impedance of the connector matches the impedance of the transmission line. A low SWR indicates efficient power transfer, while a high SWR indicates power reflection and potential signal loss.

The relationship between frequency and wavelength is given by:
\[
\lambda = \frac{c}{f}
\]
where \(\lambda\) is the wavelength, \(c\) is the speed of light (\(3 \times 10^8\) m/s), and \(f\) is the frequency. At 4 GHz, the wavelength is:
\[
\lambda = \frac{3 \times 10^8}{4 \times 10^9} = 0.075 \text{ meters} = 7.5 \text{ cm}
\]
This wavelength is compatible with the physical dimensions of the BNC connector, ensuring minimal signal loss and reflection.

% Diagram prompt: A diagram showing the relationship between frequency, wavelength, and the physical dimensions of a BNC connector would be helpful here.