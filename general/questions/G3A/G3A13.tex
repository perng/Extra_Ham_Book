\subsection{A-Index Measurement}
\label{G3A13}

\begin{tcolorbox}[colback=gray!10!white,colframe=black!75!black,title=G3A13]
What does the A-index measure?
\begin{enumerate}[label=\Alph*)]
    \item The relative position of sunspots on the surface of the Sun
    \item The amount of polarization of the Sun’s electric field
    \item \textbf{The long-term stability of Earth’s geomagnetic field}
    \item The solar radio flux at Boulder, Colorado
\end{enumerate}
\end{tcolorbox}

\subsubsection{Intuitive Explanation}
Imagine the Earth is like a giant magnet, and the A-index is like a report card that tells us how well this magnet is behaving over time. It doesn’t care about sunspots or how the Sun’s electric field is polarized. Instead, it focuses on how stable the Earth’s magnetic field is in the long run. So, if you’re curious about the Earth’s magnetic mood over weeks or months, the A-index is your go-to!

\subsubsection{Advanced Explanation}
The A-index is a measure derived from the K-index, which quantifies geomagnetic activity on a scale from 0 to 9. The K-index is calculated every three hours based on the maximum fluctuations of the horizontal component of the Earth’s magnetic field. The A-index is then computed as the average of the eight daily K-index values, converted to a linear scale. Mathematically, the A-index \( A \) is given by:

\[
A = \frac{1}{8} \sum_{i=1}^{8} K_i
\]

where \( K_i \) represents the K-index values for each three-hour interval. The A-index provides a long-term perspective on geomagnetic activity, smoothing out short-term fluctuations. This is crucial for understanding the overall stability of the Earth’s geomagnetic field, which can be influenced by solar wind and other space weather phenomena.

% Diagram Prompt: Generate a diagram showing the relationship between the K-index and A-index, with a timeline illustrating how the A-index is derived from the K-index values over a day.