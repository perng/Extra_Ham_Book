\subsection{K-Index Measurement}
\label{G3A12}

\begin{tcolorbox}[colback=gray!10!white,colframe=black!75!black,title=G3A12]
What does the K-index measure?
\begin{enumerate}[label=\Alph*),noitemsep]
    \item The relative position of sunspots on the surface of the Sun
    \item \textbf{The short-term stability of Earth’s geomagnetic field}
    \item The short-term stability of the Sun’s magnetic field
    \item The solar radio flux at Boulder, Colorado
\end{enumerate}
\end{tcolorbox}

\subsubsection{Intuitive Explanation}
Imagine Earth is like a giant magnet, and sometimes it gets a little shaky because of the Sun's mood swings. The K-index is like a shakiness meter for Earth's magnetic field. It tells us how wobbly the magnetic field is over a short period of time. So, if the K-index is high, it means Earth's magnetic field is having a bit of a rough day!

\subsubsection{Advanced Explanation}
The K-index is a quantized measure of the geomagnetic activity, specifically the horizontal component of the Earth's magnetic field. It is derived from the maximum fluctuations of the magnetic field observed over a three-hour interval at a given magnetometer station. The K-index ranges from 0 to 9, with higher values indicating greater geomagnetic activity. 

Mathematically, the K-index is calculated based on the range of the magnetic field variation in nanoteslas (nT). The formula for converting the range \( R \) in nT to the K-index is:

\[
K = \left\lfloor \frac{\log_{10}(R) - a}{b} \right\rfloor
\]

where \( a \) and \( b \) are constants specific to the magnetometer station. The K-index is crucial for understanding space weather and its effects on satellite communications, power grids, and other technological systems.

% Prompt for diagram: A diagram showing Earth's magnetic field lines and how they fluctuate with geomagnetic activity would be helpful here.