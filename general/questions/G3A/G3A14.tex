\subsection{Effect of Solar Coronal Particles on Long Distance Radio Communication}
\label{G3A14}

\begin{tcolorbox}[colback=gray!10!white,colframe=black!75!black,title=G3A14]
How is long distance radio communication usually affected by the charged particles that reach Earth from solar coronal holes?
\begin{enumerate}[label=\Alph*),noitemsep]
    \item HF communication is improved
    \item \textbf{HF communication is disturbed}
    \item VHF/UHF ducting is improved
    \item VHF/UHF ducting is disturbed
\end{enumerate}
\end{tcolorbox}

\subsubsection{Intuitive Explanation}
Imagine the Sun is like a giant popcorn machine, and sometimes it pops out a bunch of charged particles (like tiny, invisible popcorn kernels) that fly towards Earth. These particles can mess with the radio waves we use to talk to people far away, especially the ones that bounce off the sky (HF waves). It's like trying to have a conversation while someone is popping popcorn right next to you—it’s going to be harder to hear each other! So, instead of making the radio waves better, these solar particles usually make them worse.

\subsubsection{Advanced Explanation}
Solar coronal holes are regions on the Sun's corona where the magnetic field opens into space, allowing charged particles to escape. These particles, primarily electrons and protons, are ejected as part of the solar wind. When they reach Earth, they interact with the Earth's magnetosphere and ionosphere.

The ionosphere is crucial for HF (High Frequency) radio communication, as it reflects HF radio waves back to Earth, enabling long-distance communication. However, the influx of charged particles from solar coronal holes can cause disturbances in the ionosphere. These disturbances can lead to increased ionization, which in turn can cause absorption or scattering of HF radio waves, thereby disrupting HF communication.

Mathematically, the increased ionization can be described by the electron density \( N_e \) in the ionosphere. The critical frequency \( f_c \) of the ionosphere, which determines the highest frequency that can be reflected, is given by:

\[
f_c = 9 \sqrt{N_e}
\]

When \( N_e \) increases due to the influx of charged particles, \( f_c \) also increases. However, this can lead to increased absorption of HF radio waves, reducing their effective range and quality.

In contrast, VHF (Very High Frequency) and UHF (Ultra High Frequency) communications are less affected by ionospheric disturbances because they typically propagate through the troposphere rather than being reflected by the ionosphere. Therefore, the impact on VHF/UHF ducting is minimal compared to HF communication.

% Prompt for diagram: A diagram showing the interaction of solar wind particles with the Earth's magnetosphere and ionosphere, and how this affects HF radio wave propagation.