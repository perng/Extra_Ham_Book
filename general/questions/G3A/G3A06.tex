\subsection{Geomagnetic Storms}
\label{G3A06}

\begin{tcolorbox}[colback=gray!10!white,colframe=black!75!black,title=G3A06]
What is a geomagnetic storm?
\begin{enumerate}[label=\Alph*),noitemsep]
    \item A sudden drop in the solar flux index
    \item A thunderstorm that affects radio propagation
    \item Ripples in the geomagnetic force
    \item \textbf{A temporary disturbance in Earth’s geomagnetic field}
\end{enumerate}
\end{tcolorbox}

\subsubsection{Intuitive Explanation}
Imagine Earth is like a giant magnet, with invisible magnetic lines stretching from the North Pole to the South Pole. Sometimes, the Sun sends out a big burst of energy, like a cosmic sneeze. When this energy hits Earth, it shakes up our magnetic field, causing a geomagnetic storm. It's like a temporary hiccup in Earth's magnetic personality, but don't worry, it doesn't last forever!

\subsubsection{Advanced Explanation}
A geomagnetic storm is a temporary disturbance of Earth's magnetosphere caused by a solar wind shock wave and/or cloud of magnetic field that interacts with the Earth's magnetic field. The solar wind pressure on the magnetosphere will increase or decrease depending on the Sun's activity. These changes can induce electric currents in the Earth's crust and ionosphere, which can affect power grids, satellite operations, and radio communications.

Mathematically, the disturbance can be described by changes in the geomagnetic indices such as the Kp index, which quantifies the level of geomagnetic activity. The Kp index ranges from 0 to 9, with higher values indicating more severe geomagnetic storms. The relationship between the solar wind parameters and the geomagnetic indices can be complex, involving interactions between the solar wind's magnetic field and the Earth's magnetosphere.

% Diagram prompt: Generate a diagram showing the interaction between the solar wind and Earth's magnetosphere during a geomagnetic storm.