\subsection{Coronal Mass Ejection Impact on Radio Propagation}
\label{G3A11}

\begin{tcolorbox}[colback=gray!10!white,colframe=black!75!black,title=G3A11]
How long does it take a coronal mass ejection to affect radio propagation on Earth?
\begin{enumerate}[label=\Alph*)]
    \item 28 days
    \item 14 days
    \item 4 to 8 minutes
    \item \textbf{15 hours to several days}
\end{enumerate}
\end{tcolorbox}

\subsubsection{Intuitive Explanation}
Imagine the Sun as a giant sneeze machine. Sometimes, it sneezes out a huge cloud of particles called a coronal mass ejection (CME). This sneeze travels through space like a slow-moving storm. It doesn’t reach Earth instantly—it takes a while, like waiting for a package to arrive. Depending on how fast the sneeze is, it can take anywhere from 15 hours to a few days to reach us. When it finally arrives, it can mess with radio signals, making them act all wonky. So, the answer is D: 15 hours to several days.

\subsubsection{Advanced Explanation}
A coronal mass ejection (CME) is a massive burst of solar wind and magnetic fields rising above the solar corona or being released into space. The speed of a CME can vary widely, typically ranging from 250 km/s to over 3000 km/s. The distance from the Sun to Earth is approximately 150 million kilometers (1 astronomical unit, AU).

To calculate the time it takes for a CME to reach Earth, we use the formula:
\[
\text{Time} = \frac{\text{Distance}}{\text{Speed}}
\]
For a CME traveling at 1000 km/s:
\[
\text{Time} = \frac{150,000,000 \text{ km}}{1000 \text{ km/s}} = 150,000 \text{ seconds} \approx 41.67 \text{ hours} \approx 1.74 \text{ days}
\]
Given the range of CME speeds, the time can vary from 15 hours (for the fastest CMEs) to several days (for slower ones). This variability explains why the correct answer is D: 15 hours to several days.

CMEs interact with Earth's magnetosphere, causing geomagnetic storms that can disrupt radio propagation by altering the ionosphere. This can lead to phenomena such as increased absorption of radio waves, changes in the maximum usable frequency (MUF), and the creation of auroras.

% Diagram Prompt: Generate a diagram showing the path of a CME from the Sun to Earth, with labels indicating the distance and the time it takes for the CME to travel.