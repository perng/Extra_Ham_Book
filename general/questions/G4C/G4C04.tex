\subsection{RF Interference from a CW Transmitter}
\label{G4C04}

\begin{tcolorbox}[colback=gray!10!white,colframe=black!75!black,title=G4C04]
What sound is heard from an audio device experiencing RF interference from a CW transmitter?
\begin{enumerate}[label=\Alph*)]
    \item \textbf{On-and-off humming or clicking}
    \item A CW signal at a nearly pure audio frequency
    \item A chirpy CW signal
    \item Severely distorted audio
\end{enumerate}
\end{tcolorbox}

\subsubsection{Intuitive Explanation}
Imagine you're listening to your favorite song on the radio, and suddenly, you hear a weird humming or clicking noise that keeps turning on and off. This is like someone turning a light switch on and off really fast, but instead of light, it's sound! This happens because a CW transmitter (which sends out a steady radio signal) is interfering with your audio device. The on-and-off humming or clicking is the result of this interference, making your music sound like it's being interrupted by a pesky ghost!

\subsubsection{Advanced Explanation}
RF (Radio Frequency) interference occurs when a CW (Continuous Wave) transmitter's signal is picked up by an audio device. A CW transmitter emits a steady, unmodulated radio signal at a specific frequency. When this signal interferes with an audio device, it can cause the device's circuitry to demodulate the RF signal, converting it into an audible sound.

The demodulation process often results in an on-and-off humming or clicking sound because the CW signal is essentially a pure tone that is being turned on and off at a rate that the audio device can interpret. This is different from other types of interference, such as a chirpy CW signal or severely distorted audio, which would result from different modulation or interference patterns.

Mathematically, the CW signal can be represented as:
\[
s(t) = A \cos(2\pi f_c t)
\]
where \( A \) is the amplitude and \( f_c \) is the carrier frequency. When this signal is demodulated by the audio device, it can produce an audible signal that corresponds to the on-and-off pattern of the CW transmission.

% Diagram prompt: Generate a diagram showing a CW signal interfering with an audio device, illustrating the on-and-off humming or clicking sound.