\subsection{Causes of Wide-Range Frequency Interference}
\label{G4C02}

\begin{tcolorbox}[colback=gray!10!white,colframe=black!75!black,title=G4C02]
Which of the following could be a cause of interference covering a wide range of frequencies?
\begin{enumerate}[label=\Alph*,noitemsep]
    \item Not using a balun or line isolator to feed balanced antennas
    \item Lack of rectification of the transmitter’s signal in power conductors
    \item \textbf{Arcing at a poor electrical connection}
    \item Using a balun to feed an unbalanced antenna
\end{enumerate}
\end{tcolorbox}

\subsubsection{Intuitive Explanation}
Imagine you’re trying to listen to your favorite radio station, but suddenly, you hear a bunch of weird noises—like static, crackling, or buzzing—that cover a lot of different stations. What could be causing this? Well, think of a bad electrical connection as a tiny lightning bolt that keeps zapping. Each zap sends out a burst of energy that messes up a wide range of frequencies. This is called arcing, and it’s like a naughty gremlin that loves to ruin your radio experience!

\subsubsection{Advanced Explanation}
Arcing at a poor electrical connection generates broadband noise, which spans a wide range of frequencies. This phenomenon occurs when there is a breakdown of insulation or a gap in the electrical circuit, causing intermittent sparks. These sparks produce electromagnetic radiation across a broad spectrum, leading to interference in multiple frequency bands.

Mathematically, the power spectral density \( S(f) \) of the noise generated by arcing can be approximated by:

\[
S(f) \propto \frac{1}{f^n}
\]

where \( f \) is the frequency and \( n \) is a constant typically between 1 and 2. This inverse relationship indicates that the noise power decreases with increasing frequency, but it still affects a wide range of frequencies.

In contrast, the other options do not produce such wideband interference:
\begin{itemize}
    \item Not using a balun or line isolator (Option A) can cause impedance mismatch, leading to reflections and standing waves, but this is usually limited to specific frequencies.
    \item Lack of rectification (Option B) would not generate interference; it would simply result in a loss of signal.
    \item Using a balun to feed an unbalanced antenna (Option D) might cause some inefficiency, but it does not produce wideband noise.
\end{itemize}

% Diagram Prompt: Generate a diagram showing the frequency spectrum of noise generated by arcing compared to other types of interference.