\subsection{Soldered Joints in Lightning Protection}
\label{G4C07}

\begin{tcolorbox}[colback=gray!10!white,colframe=black!75!black,title=G4C07]
Why should soldered joints not be used in lightning protection ground connections?
\begin{enumerate}[label=\Alph*)]
    \item \textbf{A soldered joint will likely be destroyed by the heat of a lightning strike}
    \item Solder flux will prevent a low conductivity connection
    \item Solder has too high a dielectric constant to provide adequate lightning protection
    \item All these choices are correct
\end{enumerate}
\end{tcolorbox}

\subsubsection{Intuitive Explanation}
Imagine you're trying to protect your house from a giant lightning bolt. You set up a special wire to guide the lightning safely into the ground. Now, if you use a soldered joint (like a tiny metal glue) to connect parts of this wire, the heat from the lightning is so intense that it would melt the solder, just like how a marshmallow melts in a campfire. So, the connection would break, and the lightning might not go where you want it to. That's why we don't use soldered joints in lightning protection—they can't handle the heat!

\subsubsection{Advanced Explanation}
Lightning strikes can generate temperatures up to 30,000 Kelvin, which is hotter than the surface of the sun. Soldered joints, typically made of tin-lead or other low-melting-point alloys, have melting points around 183°C to 250°C. When a lightning strike occurs, the immense heat can easily melt the solder, causing the joint to fail. This failure can disrupt the path of the lightning, potentially leading to damage or injury.

Additionally, the electrical conductivity of solder is lower than that of copper or other metals typically used in grounding systems. While this is not the primary reason for avoiding solder in lightning protection, it is a secondary consideration. The dielectric constant of solder is irrelevant in this context, as the primary concern is the mechanical integrity of the joint under extreme thermal stress.

In summary, soldered joints are unsuitable for lightning protection due to their low melting points and the extreme heat generated by lightning strikes.

% Diagram Prompt: Generate a diagram showing a lightning strike hitting a ground wire with a soldered joint, illustrating the melting of the solder due to the intense heat.