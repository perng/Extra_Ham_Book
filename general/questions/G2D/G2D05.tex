\subsection{Indicating an HF Contact Request}
\label{G2D05}

\begin{tcolorbox}[colback=gray!10!white,colframe=black!75!black,title=G2D05]
Which of the following indicates that you are looking for an HF contact with any station?
\begin{enumerate}[label=\Alph*,noitemsep]
    \item Sign your call sign once, followed by the words “listening for a call” -- if no answer, change frequency and repeat
    \item Say “QTC” followed by “this is” and your call sign -- if no answer, change frequency and repeat
    \item \textbf{Repeat “CQ” a few times, followed by “this is,” then your call sign a few times, then pause to listen, repeat as necessary}
    \item Transmit an unmodulated carried for approximately 10 seconds, followed by “this is” and your call sign, and pause to listen -- repeat as necessary
\end{enumerate}
\end{tcolorbox}

\subsubsection{Intuitive Explanation}
Imagine you're at a big party and you want to talk to someone, but you don't know who yet. You might shout, Hey, anyone want to chat? That's what CQ is like in the radio world. It's a way of saying, Hey, anyone out there want to talk? You say it a few times, then say who you are (your call sign), and then you listen to see if anyone answers. If no one does, you try again. It's like calling out in a crowded room to see who's interested in a conversation.

\subsubsection{Advanced Explanation}
In High Frequency (HF) radio communication, CQ is a general call to any station. It is derived from the French word sécurité, which means safety, but in radio communication, it has come to mean calling any station. The correct procedure to initiate a contact is to repeat CQ a few times, followed by this is, and then your call sign a few times. This sequence ensures that your call is heard clearly and that any station listening can identify you. After transmitting, you pause to listen for a response. If no response is received, you repeat the process. This method is standardized to ensure clarity and efficiency in establishing communication.

% Diagram prompt: Generate a diagram showing the sequence of transmitting CQ, call sign, and listening for a response in HF communication.