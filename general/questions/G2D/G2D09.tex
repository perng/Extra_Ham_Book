\subsection{Contest Participation on HF Frequencies}
\label{G2D09}

\begin{tcolorbox}[colback=gray!10!white,colframe=black!75!black,title=G2D09]
Which of the following is required when participating in a contest on HF frequencies?
\begin{enumerate}[label=\Alph*,noitemsep]
    \item Submit a log to the contest sponsor
    \item Send a QSL card to the stations worked, or QSL via Logbook of The World
    \item \textbf{Identify your station according to normal FCC regulations}
    \item All these choices are correct
\end{enumerate}
\end{tcolorbox}

\subsubsection{Intuitive Explanation}
Imagine you're playing a game where you need to talk to as many people as possible on a special radio. But there's a rule: you have to tell everyone who you are, just like saying your name when you meet someone new. This is so everyone knows who's talking and it keeps things fair and fun. So, when you're in this radio game, you must always say your name (or your station's name) according to the rules.

\subsubsection{Advanced Explanation}
When participating in a contest on HF (High Frequency) frequencies, it is essential to adhere to the regulations set forth by the Federal Communications Commission (FCC). One of the fundamental requirements is the proper identification of your station. According to FCC regulations, you must identify your station by transmitting your call sign at the end of each communication and at least every 10 minutes during a communication. This ensures transparency and accountability in radio communications.

The other options, such as submitting a log to the contest sponsor or sending QSL cards, are not mandatory but are often encouraged as part of good amateur radio practice. However, the only requirement that is strictly enforced by the FCC is the proper identification of your station.

% Diagram prompt: A flowchart showing the steps of participating in an HF contest, highlighting the mandatory station identification step.