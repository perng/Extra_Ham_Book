\subsection{Objectives of the Volunteer Monitor Program}
\label{G2D02}

\begin{tcolorbox}[colback=gray!10!white,colframe=black!75!black,title=G2D02]
Which of the following are objectives of the Volunteer Monitor Program?
\begin{enumerate}[label=\Alph*),noitemsep]
    \item To conduct efficient and orderly amateur licensing examinations
    \item To provide emergency and public safety communications
    \item To coordinate repeaters for efficient and orderly spectrum usage
    \item \textbf{To encourage amateur radio operators to self-regulate and comply with the rules}
\end{enumerate}
\end{tcolorbox}

\subsubsection{Intuitive Explanation}
Imagine you're part of a big club where everyone loves playing with radios. Now, just like in any club, there are rules to make sure everyone has fun and stays safe. The Volunteer Monitor Program is like the friendly club monitors who remind everyone to follow the rules. They don't give tests, they don't handle emergencies, and they don't manage the radios directly. Instead, they encourage everyone to be good radio citizens and follow the rules on their own. So, the right answer is the one that says they help people follow the rules themselves!

\subsubsection{Advanced Explanation}
The Volunteer Monitor Program (VMP) is an initiative by the Federal Communications Commission (FCC) to promote self-regulation within the amateur radio community. The primary objective of the VMP is to encourage amateur radio operators to adhere to the rules and regulations set forth by the FCC. This is achieved through monitoring and reporting any violations, as well as educating operators about compliance.

The other options listed in the question pertain to different aspects of amateur radio operations:
\begin{itemize}
    \item \textbf{Option A} refers to the role of Volunteer Examiners (VEs) who conduct licensing exams.
    \item \textbf{Option B} relates to the Amateur Radio Emergency Service (ARES) and other public service groups.
    \item \textbf{Option C} involves the coordination of repeaters, which is typically managed by local or regional repeater councils.
\end{itemize}

Therefore, the correct answer is \textbf{D}, as it directly aligns with the objectives of the Volunteer Monitor Program.

% Prompt for diagram: A diagram showing the relationship between the Volunteer Monitor Program, amateur radio operators, and the FCC could be helpful here.