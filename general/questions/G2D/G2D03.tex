\subsection{Localizing a Station with Continuous Carrier}\label{G2D03}

\begin{tcolorbox}[colback=gray!10!white,colframe=black!75!black,title=G2D03]
What procedure may be used by Volunteer Monitors to localize a station whose continuous carrier is holding a repeater on in their area?
\begin{enumerate}[label=\Alph*),noitemsep]
    \item Compare vertical and horizontal signal strengths on the input frequency
    \item \textbf{Compare beam headings on the repeater input from their home locations with that of other Volunteer Monitors}
    \item Compare signal strengths between the input and output of the repeater
    \item All these choices are correct
\end{enumerate}
\end{tcolorbox}

\subsubsection{Intuitive Explanation}
Imagine you and your friends are trying to find out where a loud, annoying noise is coming from in your neighborhood. Instead of just guessing, you all decide to point in the direction you think the noise is coming from. If everyone points in the same direction, you can be pretty sure that's where the noise is coming from. In this case, the noise is a radio signal, and the pointing is done using antennas. By comparing the directions everyone is pointing, you can figure out where the signal is coming from and stop it from messing up the repeater.

\subsubsection{Advanced Explanation}
To localize a station with a continuous carrier, Volunteer Monitors can use a technique called \textit{direction finding}. This involves using directional antennas to determine the bearing of the signal. By comparing the beam headings (the direction the antennas are pointing) from multiple locations, the monitors can triangulate the source of the signal. 

Mathematically, if you have two or more bearings from different locations, you can find the intersection point of these bearings to locate the source. For example, if Monitor A reports a bearing of \( \theta_1 \) and Monitor B reports a bearing of \( \theta_2 \), the intersection of these two lines will give the approximate location of the signal source. This method is more accurate than comparing signal strengths, as it directly uses the direction of the signal rather than its intensity.

% Diagram prompt: Generate a diagram showing two monitors with directional antennas pointing towards a signal source, with the bearings labeled as theta1 and theta2, and the intersection point marked as the signal source.