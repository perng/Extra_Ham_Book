\subsection{Lower HF Frequencies in Summer}
\label{G3B12}

\begin{tcolorbox}[colback=gray!10!white,colframe=black!75!black,title=G3B12]
Which of the following is typical of the lower HF frequencies during the summer?
\begin{enumerate}[label=\Alph*]
    \item Poor propagation at any time of day
    \item World-wide propagation during daylight hours
    \item Heavy distortion on signals due to photon absorption
    \item \textbf{High levels of atmospheric noise or static}
\end{enumerate}
\end{tcolorbox}

\subsubsection{Intuitive Explanation}
Imagine you're trying to listen to your favorite radio station during a summer thunderstorm. The crackling and popping sounds you hear are caused by atmospheric noise, which is like nature's way of adding static to your radio. In the summer, especially at lower HF frequencies, this noise is much more common because of all the thunderstorms and lightning happening around the world. So, if you're tuning into these frequencies in the summer, expect a lot of static!

\subsubsection{Advanced Explanation}
The lower HF (High Frequency) band, typically ranging from 3 to 10 MHz, is significantly affected by atmospheric conditions, particularly during the summer months. One of the primary sources of atmospheric noise in this frequency range is lightning discharges, which are more frequent in summer due to increased thunderstorm activity. 

The noise generated by lightning propagates over long distances via the ionosphere, leading to high levels of atmospheric noise or static. This phenomenon is quantified by the noise figure, which is higher in the summer months. Mathematically, the noise power \( P_n \) can be expressed as:

\[ P_n = kTB \]

where:
\begin{itemize}
    \item \( k \) is the Boltzmann constant (\( 1.38 \times 10^{-23} \) J/K),
    \item \( T \) is the effective noise temperature,
    \item \( B \) is the bandwidth of the receiver.
\end{itemize}

During summer, the effective noise temperature \( T \) increases due to the higher incidence of lightning, leading to an increase in \( P_n \). This results in the observed high levels of atmospheric noise or static.

Additionally, the ionospheric conditions during summer can also contribute to the propagation of this noise. The D-layer of the ionosphere, which absorbs HF signals during the day, is more pronounced in summer, further enhancing the noise levels.

% Diagram prompt: Generate a diagram showing the propagation of atmospheric noise from lightning discharges through the ionosphere to a receiver on the ground.