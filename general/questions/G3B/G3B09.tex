\subsection{Maximum Distance in One Hop Using F2 Region}
\label{G3B09}

\begin{tcolorbox}[colback=gray!10!white,colframe=black!75!black,title=G3B09]
What is the approximate maximum distance along the Earth’s surface normally covered in one hop using the F2 region?
\begin{enumerate}[label=\Alph*,noitemsep]
    \item 180 miles
    \item 1,200 miles
    \item \textbf{2,500 miles}
    \item 12,000 miles
\end{enumerate}
\end{tcolorbox}

\subsubsection{Intuitive Explanation}
Imagine you're playing a game of radio wave hopscotch on Earth. The F2 region is like a trampoline in the sky that helps radio waves bounce really far. Normally, when you use this trampoline, the radio waves can jump up to about 2,500 miles in one go. That's like bouncing from New York City to Los Angeles in a single hop! So, if you're trying to send a radio signal across the country, the F2 region is your best buddy.

\subsubsection{Advanced Explanation}
The F2 region is a layer of the ionosphere located approximately 200 to 400 km above the Earth's surface. This region is crucial for long-distance radio communication because it reflects high-frequency (HF) radio waves back to Earth. The maximum distance covered in one hop using the F2 region is influenced by the height of the ionosphere and the curvature of the Earth.

To calculate the maximum distance, we can use the following formula:

\[
d = 2 \times \sqrt{2 \times R \times h}
\]

where:
\begin{itemize}
    \item \( d \) is the maximum distance,
    \item \( R \) is the Earth's radius (approximately 6,371 km),
    \item \( h \) is the height of the F2 region (approximately 300 km).
\end{itemize}

Plugging in the values:

\[
d = 2 \times \sqrt{2 \times 6371 \times 300} \approx 2,500 \text{ miles}
\]

This calculation shows that the maximum distance covered in one hop using the F2 region is approximately 2,500 miles. This distance can vary slightly depending on atmospheric conditions and the angle of incidence of the radio waves.

% Diagram Prompt: Generate a diagram showing the Earth's curvature, the F2 region, and the path of a radio wave bouncing off the F2 region to cover a distance of 2,500 miles.