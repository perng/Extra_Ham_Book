\subsection{Placement of Radial Wires in Ground-Mounted Vertical Antenna Systems}
\label{G9B06}

\begin{tcolorbox}[colback=gray!10!white,colframe=black!75!black,title=G9B06]
Where should the radial wires of a ground-mounted vertical antenna system be placed?
\begin{enumerate}[label=\Alph*,noitemsep]
    \item As high as possible above the ground
    \item Parallel to the antenna element
    \item \textbf{On the surface or buried a few inches below the ground}
    \item At the center of the antenna
\end{enumerate}
\end{tcolorbox}

\subsubsection{Intuitive Explanation}
Imagine you have a giant metal stick (the antenna) stuck in the ground, and you want it to work really well. The radial wires are like the roots of a tree—they help the antenna grip the ground better. If you put the roots (radial wires) on the surface or just a little bit under the dirt, the antenna can talk to the ground more effectively. If you put them too high or in the wrong place, it’s like the tree has no roots—it won’t work as well!

\subsubsection{Advanced Explanation}
In a ground-mounted vertical antenna system, the radial wires serve as the ground plane, which is essential for the antenna's performance. The ground plane provides a reflective surface for the radio waves, improving the antenna's efficiency and radiation pattern. 

The optimal placement for the radial wires is on the surface or buried a few inches below the ground. This placement ensures that the radial wires are in close contact with the earth, which enhances the conductivity and reduces ground losses. The radial wires should be evenly distributed around the base of the antenna to create a symmetrical ground plane.

Mathematically, the effectiveness of the ground plane can be analyzed using the concept of ground conductivity (\(\sigma\)) and the skin depth (\(\delta\)) of the radio waves in the ground. The skin depth is given by:

\[
\delta = \sqrt{\frac{2}{\omega \mu \sigma}}
\]

where \(\omega\) is the angular frequency of the radio wave, \(\mu\) is the permeability of the ground, and \(\sigma\) is the conductivity. By placing the radial wires close to the ground, we minimize the skin depth and maximize the ground plane's effectiveness.

% Prompt for generating a diagram: A diagram showing a vertical antenna with radial wires placed on the surface and buried a few inches below the ground, with arrows indicating the flow of radio waves and the ground plane's reflective surface.