\subsection{Length of a 1/4 Wave Monopole Antenna}\label{G9B12}

\begin{tcolorbox}[colback=gray!10!white,colframe=black!75!black,title=G9B12]
What is the approximate length for a 1/4 wave monopole antenna cut for 28.5 MHz?
\begin{enumerate}[label=\Alph*)]
    \item \textbf{8 feet}
    \item 11 feet
    \item 16 feet
    \item 21 feet
\end{enumerate}
\end{tcolorbox}

\subsubsection{Intuitive Explanation}
Imagine you have a string that you want to make wiggle just right so it can send out radio waves. For a 1/4 wave monopole antenna, you only need a piece of the string that's one-fourth the length of the wave you're trying to send. At 28.5 MHz, the wave is pretty short, so the antenna doesn't need to be very long. Think of it like cutting a piece of spaghetti to match the size of a small plate—it’s just a little bit, not the whole noodle!

\subsubsection{Advanced Explanation}
The length of a 1/4 wave monopole antenna can be calculated using the formula:

\[
L = \frac{c}{4f}
\]

where:
\begin{itemize}
    \item \(L\) is the length of the antenna,
    \item \(c\) is the speed of light (\(3 \times 10^8\) meters/second),
    \item \(f\) is the frequency in Hertz.
\end{itemize}

For a frequency of 28.5 MHz (\(28.5 \times 10^6\) Hz), the calculation is:

\[
L = \frac{3 \times 10^8}{4 \times 28.5 \times 10^6} \approx 2.63 \text{ meters}
\]

Converting meters to feet (1 meter $\approx$ 3.28 feet):

\[
2.63 \text{ meters} \times 3.28 \approx 8.63 \text{ feet}
\]

Thus, the approximate length of the antenna is 8 feet.

\subsubsection{Related Concepts}
A monopole antenna is a type of radio antenna that consists of a straight rod-shaped conductor, often mounted perpendicularly over some type of conductive surface, called a ground plane. The length of the antenna is crucial because it determines the frequency at which the antenna resonates. The 1/4 wave monopole is a common design because it is relatively simple and effective for many applications.

% Prompt for generating a diagram: 
% Diagram showing a 1/4 wave monopole antenna with the length labeled and the ground plane indicated.