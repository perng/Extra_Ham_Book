\subsection{Length of a 1/2 Wave Dipole Antenna}
\label{G9B10}

\begin{tcolorbox}[colback=gray!10!white,colframe=black!75!black,title=G9B10]
What is the approximate length for a 1/2 wave dipole antenna cut for 14.250 MHz?
\begin{enumerate}[label=\Alph*,noitemsep]
    \item 8 feet
    \item 16 feet
    \item 24 feet
    \item \textbf{33 feet}
\end{enumerate}
\end{tcolorbox}

\subsubsection{Intuitive Explanation}
Imagine you’re trying to make a jump rope that’s just the right length to swing up and down exactly once every time you say 1, 2, 3, 4, 5, 6, 7, 8, 9, 10, 11, 12, 13, 14, 15, 16, 17, 18, 19, 20, 21, 22, 23, 24, 25, 26, 27, 28, 29, 30, 31, 32, 33! That’s kind of what a 1/2 wave dipole antenna does, but instead of counting, it’s tuned to a specific radio frequency. For 14.250 MHz, the magic length is about 33 feet. Too short or too long, and it won’t work as well!

\subsubsection{Advanced Explanation}
The length of a 1/2 wave dipole antenna can be calculated using the formula:
\[
L = \frac{468}{f}
\]
where \(L\) is the length in feet and \(f\) is the frequency in MHz. For a frequency of 14.250 MHz, the calculation is:
\[
L = \frac{468}{14.250} \approx 32.84 \text{ feet}
\]
Rounding to the nearest whole number gives us 33 feet. This formula is derived from the relationship between the wavelength (\(\lambda\)) of the radio wave and the speed of light (\(c\)):
\[
\lambda = \frac{c}{f}
\]
where \(c \approx 3 \times 10^8\) meters per second. The 1/2 wave dipole antenna is designed to be half of this wavelength, hence the name.

% Prompt for generating a diagram: A diagram showing the relationship between wavelength, frequency, and the length of a 1/2 wave dipole antenna would be helpful here.