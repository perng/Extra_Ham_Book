\subsection{Digital Mode Operations in the 20-Meter Band}
\label{G2E08}

\begin{tcolorbox}[colback=gray!10!white,colframe=black!75!black,title=G2E08]
In what segment of the 20-meter band are most digital mode operations commonly found?
\begin{enumerate}[label=\Alph*)]
    \item At the bottom of the slow-scan TV segment, near 14.230 MHz
    \item At the top of the SSB phone segment, near 14.325 MHz
    \item In the middle of the CW segment, near 14.100 MHz
    \item \textbf{Between 14.070 MHz and 14.100 MHz}
\end{enumerate}
\end{tcolorbox}

\subsubsection{Intuitive Explanation}
Imagine the 20-meter band as a big playground for radio signals. Different activities happen in different parts of the playground. Digital mode operations, like sending messages using computers, usually hang out in a specific area. Think of it as the digital corner of the playground. This corner is between 14.070 MHz and 14.100 MHz. So, if you're looking for digital mode operations, that's where you'll find them!

\subsubsection{Advanced Explanation}
The 20-meter band, ranging from 14.000 MHz to 14.350 MHz, is divided into segments allocated for different types of communications. Digital mode operations, such as RTTY, PSK31, and FT8, are typically found in the lower part of the band, specifically between 14.070 MHz and 14.100 MHz. This segment is reserved for digital communications to avoid interference with other modes like CW (Continuous Wave) and SSB (Single Side Band).

The allocation of frequencies is managed by international agreements to ensure efficient use of the radio spectrum. Digital modes often require precise frequency control and minimal interference, which is why they are confined to this specific segment. Understanding these allocations helps operators choose the correct frequency for their intended mode of communication.

% Diagram prompt: Generate a diagram showing the 20-meter band with segments labeled for CW, SSB, and digital modes, highlighting the 14.070 MHz to 14.100 MHz range.