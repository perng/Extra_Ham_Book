\subsection{RTTY Signals and SSB Transmitter Modes}
\label{G2E01}

\begin{tcolorbox}[colback=gray!10!white,colframe=black!75!black,title=G2E01]
Which mode is normally used when sending RTTY signals via AFSK with an SSB transmitter?
\begin{enumerate}[label=\Alph*,noitemsep]
    \item USB
    \item DSB
    \item CW
    \item \textbf{LSB}
\end{enumerate}
\end{tcolorbox}

\subsubsection{Intuitive Explanation}
Imagine you're sending a secret message to your friend using a walkie-talkie. You have two ways to send it: one where you talk normally (USB) and another where you talk in a funny, low voice (LSB). When sending RTTY signals, which are like digital messages, you usually use the low voice (LSB) because it works better with the equipment and keeps things clear. So, LSB is the way to go!

\subsubsection{Advanced Explanation}
RTTY (Radio Teletype) signals are typically transmitted using AFSK (Audio Frequency Shift Keying) with an SSB (Single Sideband) transmitter. SSB transmission can be either USB (Upper Sideband) or LSB (Lower Sideband). For RTTY signals, LSB is the standard mode used in the HF (High Frequency) bands. This is because LSB is traditionally used for voice communication in the HF bands, and RTTY signals are often sent in the same frequency range. 

The choice of LSB over USB is largely historical and based on convention. When modulating the audio signal for RTTY, the lower sideband is filtered out and transmitted, which is why LSB is used. This ensures compatibility with existing equipment and practices in the amateur radio community.

% Diagram prompt: Generate a diagram showing the frequency spectrum of an SSB signal with USB and LSB, highlighting the LSB used for RTTY transmission.