\subsection{Common Location for FT8}
\label{G2E15}

\begin{tcolorbox}[colback=gray!10!white,colframe=black!75!black,title=G2E15]
Which of the following is a common location for FT8?
\begin{enumerate}[label=\Alph*,noitemsep]
    \item Anywhere in the voice portion of the band
    \item Anywhere in the CW portion of the band
    \item \textbf{Approximately 14.074 MHz to 14.077 MHz}
    \item Approximately 14.110 MHz to 14.113 MHz
\end{enumerate}
\end{tcolorbox}

\subsubsection{Intuitive Explanation}
Imagine the radio band as a big highway with different lanes for different types of traffic. FT8 is like a specific car that always drives in a particular lane. That lane is between 14.074 MHz and 14.077 MHz. So, if you're looking for FT8, you know exactly where to find it on this radio highway!

\subsubsection{Advanced Explanation}
FT8 is a digital mode used in amateur radio for weak signal communication. It operates within a very narrow bandwidth, typically around 50 Hz. The frequency range of 14.074 MHz to 14.077 MHz is specifically allocated for FT8 within the 20-meter amateur band. This narrow range ensures that FT8 signals do not interfere with other modes of communication and allows for efficient use of the spectrum.

The 20-meter band (14.000 MHz to 14.350 MHz) is one of the most popular bands for amateur radio operators due to its good propagation characteristics during both day and night. Within this band, FT8 has a dedicated segment to avoid overlap with other digital modes like JT65 or PSK31, which operate at different frequencies.

% Diagram prompt: Generate a diagram showing the 20-meter band with labeled segments for FT8, JT65, and PSK31.