\subsection{Choosing a Transmitting Frequency for FT8}
\label{G2E04}

\begin{tcolorbox}[colback=gray!10!white,colframe=black!75!black,title=G2E04]
Which of the following is good practice when choosing a transmitting frequency to answer a station calling CQ using FT8?
\begin{enumerate}[label=\Alph*)]
    \item Always call on the station’s frequency
    \item Call on any frequency in the waterfall except the station’s frequency
    \item Find a clear frequency during the same time slot as the calling station
    \item \textbf{Find a clear frequency during the alternate time slot to the calling station}
\end{enumerate}
\end{tcolorbox}

\subsubsection{Intuitive Explanation}
Imagine you're at a party, and someone is shouting Hey, anyone want to chat? (that's the CQ call). Instead of shouting back right in their ear (which would be rude and confusing), you find a quiet corner and wait for them to finish their shout. Then, during the next quiet moment, you respond from your corner. This way, everyone can hear each other clearly without stepping on each other's toes. In FT8, this quiet corner is a clear frequency during the alternate time slot.

\subsubsection{Advanced Explanation}
FT8 (Franke-Taylor design, 8-FSK modulation) is a digital mode used in amateur radio that operates in fixed time slots, typically 15 seconds long. When a station calls CQ, it is transmitting during its designated time slot. To avoid interference and ensure clear communication, it is best practice to respond during the alternate time slot. This means if the calling station is transmitting in the first 15-second slot, you should transmit in the second 15-second slot, and vice versa.

Additionally, selecting a clear frequency ensures that your transmission does not overlap with others, reducing the likelihood of collisions and increasing the chances of a successful QSO (contact). This practice is crucial in crowded band conditions where multiple stations may be operating simultaneously.

% Diagram prompt: A diagram showing the time slots and frequency allocation for FT8 communication could be helpful here.