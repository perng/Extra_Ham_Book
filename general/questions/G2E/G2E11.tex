\subsection{Primary Purpose of AREDN Mesh Network}
\label{G2E11}

\begin{tcolorbox}[colback=gray!10!white,colframe=black!75!black,title=G2E11]
What is the primary purpose of an Amateur Radio Emergency Data Network (AREDN) mesh network?
\begin{enumerate}[label=\Alph*,noitemsep]
    \item To provide FM repeater coverage in remote areas
    \item To provide real time propagation data by monitoring amateur radio transmissions worldwide
    \item \textbf{To provide high-speed data services during an emergency or community event}
    \item To provide DX spotting reports to aid contesters and DXers
\end{enumerate}
\end{tcolorbox}

\subsubsection{Intuitive Explanation}
Imagine you're at a big event, like a fair or a concert, and suddenly the internet goes down. Panic! But wait, there's a superhero network called AREDN that jumps in to save the day. It’s like a giant web of walkie-talkies that can send data super fast, even when everything else is broken. So, when there’s an emergency or a big event, AREDN is there to keep everyone connected and informed. It’s not for chatting on the radio or checking the weather—it’s all about keeping the data flowing when it’s needed most!

\subsubsection{Advanced Explanation}
An Amateur Radio Emergency Data Network (AREDN) mesh network is designed to provide high-speed data services, particularly in scenarios where traditional communication infrastructure may be compromised, such as during emergencies or large community events. The network operates by creating a mesh of interconnected nodes, each capable of relaying data to other nodes within the network. This decentralized architecture ensures robust and resilient communication, even if some nodes fail.

The primary function of AREDN is not to provide FM repeater coverage (Option A), which is typically used for voice communication, nor is it to monitor propagation data (Option B) or provide DX spotting reports (Option D), which are more relevant to amateur radio contesting and long-distance communication. Instead, AREDN focuses on delivering high-speed data services, making Option C the correct answer.

The network utilizes wireless technologies such as Wi-Fi, operating on amateur radio frequencies, to establish links between nodes. Each node in the mesh network can communicate with multiple other nodes, creating redundant paths for data transmission. This redundancy is crucial in emergency situations where traditional communication channels may be unavailable or overloaded.

In summary, AREDN mesh networks are engineered to ensure reliable, high-speed data communication during critical situations, leveraging the principles of mesh networking and amateur radio technology.

% Diagram Prompt: Generate a diagram showing a mesh network with multiple interconnected nodes, illustrating how data can be routed through various paths in the network.