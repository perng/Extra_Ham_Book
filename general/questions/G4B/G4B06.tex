\subsection{Advantage of Digital Multimeters}
\label{G4B06}

\begin{tcolorbox}[colback=gray!10!white,colframe=black!75!black,title=G4B06]
What is an advantage of a digital multimeter as compared to an analog multimeter?
\begin{enumerate}[label=\Alph*)]
    \item Better for measuring computer circuits
    \item Less prone to overload
    \item \textbf{Higher precision}
    \item Faster response
\end{enumerate}
\end{tcolorbox}

\subsubsection*{Intuitive Explanation}
Imagine you’re trying to measure how much juice is left in your soda bottle. An analog multimeter is like using a ruler with big, chunky marks—it gives you a rough idea, but you might not know exactly how much is left. A digital multimeter, on the other hand, is like using a super precise measuring cup with tiny lines—it tells you exactly how much soda you have left, down to the last drop! So, the big advantage of a digital multimeter is that it’s much more precise than an analog one.

\subsubsection*{Advanced Explanation}
Digital multimeters (DMMs) offer higher precision compared to analog multimeters due to their ability to convert analog signals into digital data using an analog-to-digital converter (ADC). This conversion process allows for more accurate readings, often with resolutions down to several decimal places. 

Mathematically, the precision of a digital multimeter can be expressed as:
\[
\text{Precision} = \frac{\text{Resolution}}{\text{Full Scale Reading}}
\]
where the resolution is the smallest change in the input signal that the multimeter can detect, and the full scale reading is the maximum value the multimeter can measure. For example, if a DMM has a resolution of 0.001 V and a full scale reading of 10 V, its precision is:
\[
\text{Precision} = \frac{0.001}{10} = 0.0001 \text{ or } 0.01\%
\]

Analog multimeters, which rely on a moving needle to indicate measurements, are inherently less precise due to parallax errors and the mechanical limitations of the needle movement. Additionally, digital multimeters often include features like auto-ranging and data logging, further enhancing their utility and accuracy in various measurement scenarios.

% Diagram Prompt: Generate a diagram comparing the precision of an analog multimeter (with a needle) and a digital multimeter (with a digital display).