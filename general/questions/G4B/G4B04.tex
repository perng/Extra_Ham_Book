\subsection{Signal Source for Oscilloscope RF Envelope Check}
\label{G4B04}

\begin{tcolorbox}[colback=gray!10!white,colframe=black!75!black,title=G4B04]
What signal source is connected to the vertical input of an oscilloscope when checking the RF envelope pattern of a transmitted signal?
\begin{enumerate}[label=\Alph*]
    \item The local oscillator of the transmitter
    \item An external RF oscillator
    \item The transmitter balanced mixer output
    \item \textbf{The attenuated RF output of the transmitter}
\end{enumerate}
\end{tcolorbox}

\subsubsection{Intuitive Explanation}
Imagine you’re trying to see what a radio signal looks like as it’s being sent out. You’re using a special tool called an oscilloscope, which is like a fancy TV screen that shows you the shape of the signal. Now, to see the signal properly, you need to connect the right part of the radio to the oscilloscope. If you connect the wrong part, it’s like trying to watch a movie by plugging the DVD player into the wrong input—it just won’t work! The correct part to connect is the RF output of the radio, but you have to make it weaker (attenuated) so it doesn’t overwhelm the oscilloscope. That way, you can see the signal’s “envelope,” which is like the outline of the signal’s shape.

\subsubsection{Advanced Explanation}
When analyzing the RF envelope pattern of a transmitted signal using an oscilloscope, the vertical input must be connected to the RF output of the transmitter. This is because the RF output contains the modulated signal that represents the actual transmitted information. However, the RF signal is typically too strong to be directly connected to the oscilloscope, so it must be attenuated to avoid damaging the oscilloscope and to ensure accurate measurement.

The RF envelope pattern is the shape of the modulated signal over time, and it is crucial for understanding the characteristics of the transmitted signal. The local oscillator (Option A) and the balanced mixer output (Option C) are internal components of the transmitter that are not directly related to the final RF output. An external RF oscillator (Option B) is unrelated to the transmitter’s output signal. Therefore, the correct choice is the attenuated RF output of the transmitter (Option D).

% Prompt for diagram: A diagram showing the connection between the transmitter's RF output, an attenuator, and the oscilloscope's vertical input would be helpful here.