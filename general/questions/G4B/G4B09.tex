\subsection{Preference of Analog Multimeter over Digital Multimeter}
\label{G4B09}

\begin{tcolorbox}[colback=gray!10!white,colframe=black!75!black,title=G4B09]
When is an analog multimeter preferred to a digital multimeter?
\begin{enumerate}[label=\Alph*,noitemsep]
    \item When testing logic circuits
    \item When high precision is desired
    \item When measuring the frequency of an oscillator
    \item \textbf{When adjusting circuits for maximum or minimum values}
\end{enumerate}
\end{tcolorbox}

\subsubsection{Intuitive Explanation}
Imagine you're tuning a guitar. You twist the tuning pegs and listen carefully to the sound. You don't need a super precise digital tuner to tell you exactly how many cents off you are; you just need to hear when the string is in tune. Similarly, when adjusting circuits for maximum or minimum values, an analog multimeter is like your ears—it gives you a quick, visual sense of the changes as you tweak the circuit. It's like using a dial instead of a digital readout to find the sweet spot.

\subsubsection{Advanced Explanation}
Analog multimeters are often preferred over digital multimeters when adjusting circuits for maximum or minimum values because they provide a continuous, real-time response to changes in the circuit. This is particularly useful in scenarios where the exact numerical value is less important than the trend or direction of change. 

Analog multimeters use a moving coil mechanism to display measurements, which allows for a smooth and immediate response to variations in the measured parameter. This is in contrast to digital multimeters, which sample the signal at discrete intervals and display the results numerically. The continuous response of an analog multimeter makes it easier to observe the effects of adjustments in real-time, which is crucial when tuning circuits for optimal performance.

Mathematically, the response of an analog multimeter can be represented as a continuous function of time, \( V(t) \), where \( V \) is the voltage being measured. This continuous function allows for immediate visual feedback, which is not possible with the discrete sampling of a digital multimeter.

In summary, analog multimeters are preferred in situations where the trend or direction of change is more important than the exact numerical value, such as when adjusting circuits for maximum or minimum values.

% Diagram prompt: Generate a diagram comparing the response of an analog multimeter (continuous line) versus a digital multimeter (discrete points) over time.