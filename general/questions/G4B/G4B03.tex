\subsection{Instrument for Checking CW Keying Waveform}
\label{G4B03}

\begin{tcolorbox}[colback=gray!10!white,colframe=black!75!black,title=G4B03]
Which of the following is the best instrument to use for checking the keying waveform of a CW transmitter?
\begin{enumerate}[label=\Alph*,noitemsep]
    \item \textbf{An oscilloscope}
    \item A field strength meter
    \item A sidetone monitor
    \item A wavemeter
\end{enumerate}
\end{tcolorbox}

\subsubsection{Intuitive Explanation}
Imagine you're trying to see how a light bulb flickers when you turn it on and off really quickly. You wouldn't use a thermometer to measure the flicker, right? Instead, you'd use a tool that can show you the exact pattern of the light turning on and off. Similarly, when you want to see how a CW transmitter's keying waveform looks, you need a tool that can display the exact shape of the signal. That tool is an oscilloscope! It’s like a TV screen for electrical signals, showing you exactly how the signal changes over time.

\subsubsection{Advanced Explanation}
An oscilloscope is an essential instrument for analyzing the keying waveform of a CW (Continuous Wave) transmitter. The keying waveform represents the on-off pattern of the transmitted signal, which is crucial for ensuring proper Morse code transmission. 

The oscilloscope works by displaying the voltage of the signal over time. When connected to the transmitter, it captures the rising and falling edges of the keying waveform, allowing you to observe the shape, timing, and any potential distortions. This is particularly important for diagnosing issues such as key clicks or improper keying speeds.

Mathematically, the oscilloscope plots the signal \( V(t) \) as a function of time \( t \). For a CW transmitter, the waveform can be represented as a square wave:
\[
V(t) = 
\begin{cases} 
V_0 & \text{when the key is pressed} \\
0 & \text{when the key is released}
\end{cases}
\]
where \( V_0 \) is the amplitude of the transmitted signal. The oscilloscope allows you to measure the rise time \( t_r \), fall time \( t_f \), and the duty cycle of the waveform, ensuring that the transmitter operates within the desired parameters.

Other instruments like a field strength meter, sidetone monitor, or wavemeter are not suitable for this purpose. A field strength meter measures the strength of the radiated signal, a sidetone monitor provides audio feedback for the operator, and a wavemeter measures the frequency of the signal. None of these instruments can display the detailed waveform required for analyzing keying performance.

% Diagram Prompt: Generate a diagram showing a CW keying waveform displayed on an oscilloscope screen, with labeled axes for time (t) and voltage (V).