\subsection{Two-Tone Test Signals}
\label{G4B07}

\begin{tcolorbox}[colback=gray!10!white,colframe=black!75!black,title=G4B07]
What signals are used to conduct a two-tone test?
\begin{enumerate}[label=\Alph*,noitemsep]
    \item Two audio signals of the same frequency shifted 90 degrees
    \item \textbf{Two non-harmonically related audio signals}
    \item Two swept frequency tones
    \item Two audio frequency range square wave signals of equal amplitude
\end{enumerate}
\end{tcolorbox}

\subsubsection{Intuitive Explanation}
Imagine you’re trying to test how well a speaker can handle two different sounds at the same time. If you use two sounds that are completely unrelated (like a dog barking and a piano playing), you can see if the speaker gets confused or distorts the sounds. This is like a two-tone test! The key is to use two sounds that don’t have any special relationship, so they don’t interfere with each other in a predictable way. That’s why we use two non-harmonically related audio signals—they’re like two random sounds that don’t “match” in any way.

\subsubsection{Advanced Explanation}
A two-tone test is used to evaluate the linearity and distortion characteristics of a system, such as an amplifier or a transmitter. The test involves applying two sinusoidal signals with different frequencies to the system and analyzing the output. The frequencies of these signals should be non-harmonically related, meaning they are not integer multiples of each other. This ensures that any intermodulation products (distortions) generated by the system can be easily identified and measured.

Mathematically, if we have two tones with frequencies \( f_1 \) and \( f_2 \), the intermodulation products will appear at frequencies such as \( f_1 \pm f_2 \), \( 2f_1 \pm f_2 \), \( 2f_2 \pm f_1 \), etc. By choosing \( f_1 \) and \( f_2 \) to be non-harmonically related, we can avoid overlap between the fundamental tones and the intermodulation products, making it easier to analyze the system's performance.

For example, if \( f_1 = 1 \) kHz and \( f_2 = 1.5 \) kHz, the intermodulation products will appear at frequencies like \( 0.5 \) kHz, \( 2.5 \) kHz, \( 3.5 \) kHz, etc. These products are distinct from the original tones, allowing for clear measurement of distortion.

% Diagram prompt: Generate a diagram showing two non-harmonically related sine waves and their intermodulation products in the frequency domain.