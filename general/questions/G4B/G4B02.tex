\subsection{Oscilloscope vs. Digital Voltmeter Advantages}
\label{G4B02}

\begin{tcolorbox}[colback=gray!10!white,colframe=black!75!black,title=G4B02]
Which of the following is an advantage of an oscilloscope versus a digital voltmeter?
\begin{enumerate}[label=\Alph*,noitemsep]
    \item An oscilloscope uses less power
    \item Complex impedances can be easily measured
    \item Greater precision
    \item \textbf{Complex waveforms can be measured}
\end{enumerate}
\end{tcolorbox}

\subsubsection{Intuitive Explanation}
Imagine you’re trying to see what’s happening in a river. A digital voltmeter is like a single sensor that tells you how fast the water is flowing at one spot. But an oscilloscope is like a camera that shows you the entire river, including all the waves and ripples. So, if you want to see the whole picture of what’s happening, especially if the water is moving in a complicated way, you’d use the camera (oscilloscope) instead of just the sensor (digital voltmeter). That’s why an oscilloscope is better for measuring complex waveforms!

\subsubsection{Advanced Explanation}
An oscilloscope is a device that graphically displays electrical signals as a function of time. It captures the voltage variations over time, allowing the user to visualize waveforms, including complex ones that may have multiple frequencies, amplitudes, and phases. This is particularly useful in analyzing signals that are not purely sinusoidal or have transient components.

A digital voltmeter (DVM), on the other hand, measures the voltage at a specific point in time and provides a numerical readout. While it is highly accurate for steady-state measurements, it lacks the ability to display the temporal variations of a signal. Therefore, for complex waveforms—such as those found in modulated signals, pulse trains, or non-periodic signals—an oscilloscope is indispensable.

Mathematically, an oscilloscope can represent a signal \( V(t) \) as a function of time \( t \), whereas a DVM provides a single value \( V \) at a specific time \( t_0 \). For example, if the signal is a sine wave with noise, the oscilloscope can display the entire waveform \( V(t) = A \sin(2\pi ft) + \text{noise} \), while the DVM would only show \( V(t_0) \).

In summary, the oscilloscope’s ability to visualize complex waveforms makes it a superior tool for analyzing dynamic electrical signals compared to a digital voltmeter.

% Prompt for generating a diagram: 
% Diagram showing a comparison between an oscilloscope displaying a complex waveform and a digital voltmeter showing a single voltage reading.