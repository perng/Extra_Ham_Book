\subsection{Noise Blanker Operation}
\label{G4A03}

\begin{tcolorbox}[colback=gray!10!white,colframe=black!75!black,title=G4A03]
How does a noise blanker work?
\begin{enumerate}[label=\Alph*,noitemsep]
    \item By temporarily increasing received bandwidth
    \item By redirecting noise pulses into a filter capacitor
    \item \textbf{By reducing receiver gain during a noise pulse}
    \item By clipping noise peaks
\end{enumerate}
\end{tcolorbox}

\subsubsection*{Intuitive Explanation}
Imagine you're listening to your favorite radio station, but suddenly, someone starts popping bubble wrap right next to your ear. Annoying, right? A noise blanker is like a smart friend who quickly turns down the volume every time they hear a pop, so you can keep enjoying your music without the annoying interruptions. It doesn’t stop the pops from happening, but it makes sure they don’t ruin your listening experience.

\subsubsection*{Advanced Explanation}
A noise blanker is a circuit designed to mitigate the impact of impulsive noise, such as electrical spikes or interference, on a radio receiver. When a noise pulse is detected, the noise blanker temporarily reduces the receiver's gain, effectively attenuating the noise signal. This is achieved by using a fast-acting control loop that detects the noise pulse and adjusts the gain of the receiver's amplifier accordingly.

Mathematically, the gain reduction can be expressed as:
\[
G_{\text{reduced}} = G_{\text{normal}} - \Delta G
\]
where \( G_{\text{normal}} \) is the normal gain of the receiver, and \( \Delta G \) is the amount of gain reduction applied during the noise pulse.

The noise blanker operates by monitoring the signal for sudden increases in amplitude, which are characteristic of noise pulses. When such an increase is detected, the gain is reduced for a short duration, typically a few microseconds, to minimize the impact of the noise on the received signal. This process helps to maintain the clarity of the desired signal while suppressing unwanted noise.

% Diagram Prompt: Generate a diagram showing the operation of a noise blanker, including the normal signal, noise pulse, and the gain reduction applied by the noise blanker.