\subsection{Transformer Voltage Output}
\label{G5C06}

\begin{tcolorbox}[colback=gray!10!white,colframe=black!75!black,title=G5C06]
What is the voltage output of a transformer with a 500-turn primary and a 1500-turn secondary when 120 VAC is applied to the primary?
\begin{enumerate}[label=\Alph*,noitemsep]
    \item \textbf{360 volts}
    \item 120 volts
    \item 40 volts
    \item 25.5 volts
\end{enumerate}
\end{tcolorbox}

\subsubsection{Intuitive Explanation}
Imagine you have a magical box called a transformer. This box can change the voltage of electricity. If you put in 120 volts into the box, and the box has a special trick where it has 500 turns on the input side and 1500 turns on the output side, it will multiply the voltage by 3. So, 120 volts times 3 equals 360 volts. That's why the correct answer is 360 volts!

\subsubsection{Advanced Explanation}
The voltage transformation in a transformer is governed by the turns ratio, which is the ratio of the number of turns in the secondary coil (\(N_s\)) to the number of turns in the primary coil (\(N_p\)). The relationship is given by:

\[
\frac{V_s}{V_p} = \frac{N_s}{N_p}
\]

Where:
\begin{itemize}
    \item \(V_s\) is the secondary voltage
    \item \(V_p\) is the primary voltage
    \item \(N_s\) is the number of turns in the secondary coil
    \item \(N_p\) is the number of turns in the primary coil
\end{itemize}

Given:
\[
V_p = 120 \text{ V}, \quad N_p = 500, \quad N_s = 1500
\]

We can solve for \(V_s\):

\[
V_s = V_p \times \frac{N_s}{N_p} = 120 \text{ V} \times \frac{1500}{500} = 120 \text{ V} \times 3 = 360 \text{ V}
\]

Thus, the secondary voltage \(V_s\) is 360 volts.

% Diagram prompt: A diagram showing a transformer with labeled primary and secondary coils, indicating the number of turns and the input/output voltages.