\subsection{Increasing Capacitance}
\label{G5C13}

\begin{tcolorbox}[colback=gray!10!white,colframe=black!75!black,title=G5C13]
Which of the following components should be added to a capacitor to increase the capacitance?
\begin{enumerate}[label=\Alph*)]
    \item An inductor in series
    \item An inductor in parallel
    \item \textbf{A capacitor in parallel}
    \item A capacitor in series
\end{enumerate}
\end{tcolorbox}

\subsubsection{Intuitive Explanation}
Imagine you have a bucket (your capacitor) that can hold water (electric charge). If you want to hold more water, you can either get a bigger bucket or add another bucket next to it. Adding another bucket next to it is like adding a capacitor in parallel. This way, you have more space to store water. Adding a bucket on top of the first one (in series) doesn't help much because the water still has to go through the first bucket. So, to increase the capacitance, you add another capacitor in parallel, just like adding another bucket next to the first one.

\subsubsection{Advanced Explanation}
Capacitance \( C \) is a measure of a capacitor's ability to store charge. The total capacitance of capacitors in parallel is the sum of their individual capacitances:
\[
C_{\text{total}} = C_1 + C_2 + \dots + C_n
\]
Therefore, adding a capacitor in parallel increases the total capacitance. Conversely, the total capacitance of capacitors in series is given by:
\[
\frac{1}{C_{\text{total}}} = \frac{1}{C_1} + \frac{1}{C_2} + \dots + \frac{1}{C_n}
\]
Adding a capacitor in series decreases the total capacitance. Inductors, whether in series or parallel, do not affect the capacitance directly. Thus, the correct choice is to add a capacitor in parallel.

% Diagram prompt: A diagram showing two capacitors connected in parallel and another showing two capacitors connected in series would be helpful for visual comparison.