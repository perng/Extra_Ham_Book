\subsection{Inductance of Series Inductors}
\label{G5C11}

\begin{tcolorbox}[colback=gray!10!white,colframe=black!75!black,title=G5C11]
What is the inductance of a circuit with a 20-millihenry inductor connected in series with a 50-millihenry inductor?
\begin{enumerate}[label=\Alph*),noitemsep]
    \item 7 millihenries
    \item 14.3 millihenries
    \item \textbf{70 millihenries}
    \item 1,000 millihenries
\end{enumerate}
\end{tcolorbox}

\subsubsection{Intuitive Explanation}
Imagine you have two water pipes connected one after the other. The first pipe can hold 20 liters of water, and the second pipe can hold 50 liters. If you connect them in series, the total amount of water they can hold together is simply the sum of the two, which is 70 liters. Similarly, when you connect two inductors in series, their inductances add up just like the water in the pipes. So, a 20-millihenry inductor and a 50-millihenry inductor in series give you a total inductance of 70 millihenries.

\subsubsection{Advanced Explanation}
In a series circuit, the total inductance \( L_{\text{total}} \) is the sum of the individual inductances. Mathematically, this is expressed as:

\[
L_{\text{total}} = L_1 + L_2
\]

Given:
\[
L_1 = 20 \, \text{mH}, \quad L_2 = 50 \, \text{mH}
\]

The total inductance is calculated as:
\[
L_{\text{total}} = 20 \, \text{mH} + 50 \, \text{mH} = 70 \, \text{mH}
\]

This result is consistent with the principle that inductors in series add their inductances directly. This is because the magnetic fields generated by each inductor combine constructively, leading to an increased overall inductance.

% Diagram prompt: Generate a diagram showing two inductors connected in series with their respective inductance values labeled.