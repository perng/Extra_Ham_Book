\subsection{Increasing Inductance with Components}
\label{G5C14}

\begin{tcolorbox}[colback=gray!10!white,colframe=black!75!black,title=G5C14]
Which of the following components should be added to an inductor to increase the inductance?
\begin{enumerate}[label=\Alph*,noitemsep]
    \item A capacitor in series
    \item A capacitor in parallel
    \item An inductor in parallel
    \item \textbf{An inductor in series}
\end{enumerate}
\end{tcolorbox}

\subsubsection{Intuitive Explanation}
Imagine you have a spring (which is like an inductor). If you want to make it harder to stretch (increase its inductance), you can add another spring in line with it. This is like adding another inductor in series. Adding a spring side by side (in parallel) doesn't make it harder to stretch; it just gives you more springs to stretch. Similarly, adding a capacitor (which is like a rubber band) won't help you increase the inductance. So, the best way to increase inductance is to add another inductor in series.

\subsubsection{Advanced Explanation}
Inductance is a property of an inductor that resists changes in current. The total inductance of inductors connected in series is the sum of their individual inductances. Mathematically, for two inductors \( L_1 \) and \( L_2 \) in series, the total inductance \( L_{\text{total}} \) is given by:
\[
L_{\text{total}} = L_1 + L_2
\]
Adding an inductor in series increases the total inductance. On the other hand, inductors in parallel have a combined inductance given by:
\[
\frac{1}{L_{\text{total}}} = \frac{1}{L_1} + \frac{1}{L_2}
\]
This results in a total inductance that is less than the smallest individual inductance. Capacitors, whether in series or parallel, do not contribute to increasing inductance. Therefore, the correct choice is to add an inductor in series.

% Diagram prompt: Generate a diagram showing two inductors connected in series and another showing two inductors connected in parallel.