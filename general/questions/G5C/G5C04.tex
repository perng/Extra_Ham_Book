\subsection{Total Resistance of Resistors in Parallel}
\label{G5C04}

\begin{tcolorbox}[colback=gray!10!white,colframe=black!75!black,title=G5C04]
What is the approximate total resistance of a 100- and a 200-ohm resistor in parallel?
\begin{enumerate}[label=\Alph*),noitemsep]
    \item 300 ohms
    \item 150 ohms
    \item 75 ohms
    \item \textbf{67 ohms}
\end{enumerate}
\end{tcolorbox}

\subsubsection{Intuitive Explanation}
Imagine you have two water pipes, one is twice as wide as the other. If you connect them side by side (in parallel), water can flow through both pipes at the same time. The total resistance to the water flow is less than the resistance of either pipe alone because the water has more paths to take. Similarly, when you connect two resistors in parallel, the total resistance is less than the smallest resistor. In this case, the total resistance is about 67 ohms, which is less than both 100 ohms and 200 ohms.

\subsubsection{Advanced Explanation}
The total resistance \( R_{\text{total}} \) of resistors in parallel is given by the formula:
\[
\frac{1}{R_{\text{total}}} = \frac{1}{R_1} + \frac{1}{R_2}
\]
For a 100-ohm resistor (\( R_1 \)) and a 200-ohm resistor (\( R_2 \)):
\[
\frac{1}{R_{\text{total}}} = \frac{1}{100} + \frac{1}{200} = \frac{2}{200} + \frac{1}{200} = \frac{3}{200}
\]
Taking the reciprocal to find \( R_{\text{total}} \):
\[
R_{\text{total}} = \frac{200}{3} \approx 67 \text{ ohms}
\]
This calculation shows that the total resistance is approximately 67 ohms, which is less than either of the individual resistances.

% [Prompt for generating a diagram: A diagram showing two resistors connected in parallel with their respective resistances labeled (100 ohms and 200 ohms) and the total resistance (67 ohms) calculated.]