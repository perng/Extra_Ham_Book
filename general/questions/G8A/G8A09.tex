\subsection{FT8 Modulation Type}
\label{G8A09}

\begin{tcolorbox}[colback=gray!10!white,colframe=black!75!black,title=G8A09]
What type of modulation is used by FT8?
\begin{enumerate}[label=\Alph*,noitemsep]
    \item \textbf{8-tone frequency shift keying}
    \item Vestigial sideband
    \item Amplitude compressed AM
    \item 8-bit direct sequence spread spectrum
\end{enumerate}
\end{tcolorbox}

\subsubsection{Intuitive Explanation}
Imagine you’re sending a secret message to your friend using a walkie-talkie, but instead of just talking, you’re using different musical notes to represent your message. FT8 is like that—it uses 8 different notes (or tones) to send information. Each tone is like a different frequency, and by shifting between these frequencies, it can send data quickly and efficiently. So, FT8 uses a method called 8-tone frequency shift keying to communicate. Think of it as a musical Morse code!

\subsubsection{Advanced Explanation}
FT8 employs a modulation technique known as 8-tone frequency shift keying (8-FSK). In this method, the carrier signal is modulated by shifting its frequency among 8 distinct tones, each representing a different symbol. The frequency shift keying (FSK) is a form of digital modulation where the frequency of the carrier signal is varied in accordance with the digital data being transmitted. 

Mathematically, the modulated signal can be represented as:
\[
s(t) = A \cos(2\pi f_i t + \phi)
\]
where \( A \) is the amplitude, \( f_i \) is the frequency of the \( i \)-th tone, and \( \phi \) is the phase. In FT8, the frequency shifts are carefully chosen to minimize interference and maximize the efficiency of data transmission.

This method is particularly effective in weak signal conditions, making it popular among amateur radio operators for digital communication. The use of multiple tones allows for a higher data rate compared to traditional binary FSK, where only two frequencies are used.

% Diagram prompt: Generate a diagram showing the frequency spectrum of an 8-FSK signal, highlighting the 8 distinct tones used in FT8 modulation.