\subsection{Prosign for End of Formal Message in CW}
\label{G2C08}

\begin{tcolorbox}[colback=gray!10!white,colframe=black!75!black,title=G2C08]
What prosign is sent to indicate the end of a formal message when using CW?
\begin{enumerate}[label=\Alph*),noitemsep]
    \item SK
    \item BK
    \item \textbf{AR}
    \item KN
\end{enumerate}
\end{tcolorbox}

\subsubsection{Intuitive Explanation}
Imagine you're sending a secret message in Morse code to your friend. When you're done with your message, you need to let them know it's the end so they don't keep waiting for more. In Morse code, the prosign AR is like saying The End in a movie. It tells your friend, Okay, that's all I have to say! So, AR is the magic signal that wraps up your message.

\subsubsection{Advanced Explanation}
In Continuous Wave (CW) communication, prosigns are special sequences of Morse code characters used to convey specific instructions or signals. The prosign AR is used to indicate the end of a formal message. It is composed of the Morse code characters for A (·-) and R (·-·), sent together without a pause. This prosign is standardized in radio communication protocols to ensure clarity and consistency in message transmission. 

The other options provided are also prosigns but serve different purposes:
\begin{itemize}
    \item SK (··· -·-) signifies the end of a contact or communication session.
    \item BK (-··· -·-) is used to break into a conversation or interrupt.
    \item KN (-·- -·) is used to indicate that only the specific station being called should respond.
\end{itemize}

Understanding these prosigns is crucial for effective and standardized communication in CW, especially in amateur radio and other formal radio communication contexts.

% Prompt for generating a diagram: A diagram showing the Morse code sequences for AR, SK, BK, and KN with their respective meanings could be helpful for visual learners.