\subsection{Optimal Speed for Responding to a CQ in Morse Code}
\label{G2C05}

\begin{tcolorbox}[colback=gray!10!white,colframe=black!75!black,title=G2C05]
What is the best speed to use when answering a CQ in Morse code?
\begin{enumerate}[label=\Alph*,noitemsep]
    \item The fastest speed at which you are comfortable copying, but no slower than the CQ
    \item \textbf{The fastest speed at which you are comfortable copying, but no faster than the CQ}
    \item At the standard calling speed of 10 wpm
    \item At the standard calling speed of 5 wpm
\end{enumerate}
\end{tcolorbox}

\subsubsection{Intuitive Explanation}
Imagine you're playing a game of catch with a friend. If your friend throws the ball too fast, you might not catch it. If they throw it too slow, the game becomes boring. The same idea applies to Morse code! When someone sends a CQ (which is like saying, Hey, anyone want to chat?), you should respond at a speed that matches theirs. If you go faster, they might not understand you. If you go slower, it might feel like you're dragging the conversation. So, the best speed is one that matches the CQ speed, but not faster than what you can handle.

\subsubsection{Advanced Explanation}
In Morse code communication, the speed of transmission is measured in words per minute (wpm). When responding to a CQ (a general call to any station), it is crucial to match the speed of the calling station to ensure effective communication. The correct approach is to respond at the fastest speed you are comfortable copying, but not exceeding the speed of the CQ. This ensures that both parties can understand each other without confusion or delay. 

Mathematically, if the CQ is sent at a speed of \( S \) wpm, your response speed \( R \) should satisfy:
\[ R \leq S \]
where \( R \) is the maximum speed you can comfortably copy. This principle maintains synchronization and clarity in Morse code exchanges.

% Diagram prompt: A simple diagram showing two stations communicating via Morse code, with arrows indicating the speed of transmission and matching speeds for clarity.