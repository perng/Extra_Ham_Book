\subsection{CW Station Sends QRS}\label{G2C02}

\begin{tcolorbox}[colback=gray!10!white,colframe=black!75!black,title=G2C02]
What should you do if a CW station sends “QRS?”  
\begin{enumerate}[label=\Alph*),noitemsep]
    \item \textbf{Send slower}
    \item Change frequency
    \item Increase your power
    \item Repeat everything twice
\end{enumerate}
\end{tcolorbox}

\subsubsection{Intuitive Explanation}
Imagine you're talking to someone who’s speaking way too fast for you to understand. You’d probably say, “Hey, slow down!” That’s exactly what “QRS” means in Morse code. It’s like the other station is saying, “You’re sending messages too quickly for me to keep up. Please slow down!” So, the right thing to do is to send your messages at a slower pace. Easy, right?

\subsubsection{Advanced Explanation}
In CW (Continuous Wave) communication, operators use Morse code to send messages. The abbreviation “QRS” is part of the Q-code system, which is a standardized set of three-letter codes used in radio communication. Specifically, “QRS” means “Please send more slowly.” This is often used when the receiving operator is having difficulty decoding the incoming Morse code due to the sender’s high transmission speed.

To respond appropriately, the sender should reduce their sending speed, typically measured in words per minute (WPM). For example, if the sender was transmitting at 20 WPM, they might reduce their speed to 15 WPM or lower, depending on the receiver’s request. This ensures clear and accurate communication between both parties.

% Diagram Prompt: A diagram showing a comparison of Morse code transmission speeds (e.g., fast vs. slow) could help visualize the concept.