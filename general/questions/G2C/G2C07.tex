\subsection{RST Report and the C Indicator}
\label{G2C07}

\begin{tcolorbox}[colback=gray!10!white,colframe=black!75!black,title=G2C07]
When sending CW, what does a “C” mean when added to the RST report?
\begin{enumerate}[label=\Alph*),noitemsep]
    \item \textbf{Chirpy or unstable signal}
    \item Report was read from an S meter rather than estimated
    \item 100 percent copy
    \item Key clicks
\end{enumerate}
\end{tcolorbox}

\subsubsection*{Intuitive Explanation}
Imagine you're listening to a song on the radio, but the singer's voice keeps wobbling like they're on a rollercoaster. That's what a C in the RST report means when sending CW (Morse code). It tells the sender that their signal is a bit wobbly or unstable, like a chirping bird. So, if you hear a C, it's like saying, Hey, your signal is a bit shaky—fix it!

\subsubsection*{Advanced Explanation}
The RST report is a standardized way to describe the quality of a radio signal, particularly in CW (Continuous Wave) communication. The R stands for Readability, S for Strength, and T for Tone. When a C is appended to the RST report, it indicates that the tone of the CW signal is Chirpy or unstable. This instability can be caused by various factors, such as frequency drift or modulation issues in the transmitter.

Mathematically, a stable CW signal can be represented as a pure sine wave:
\[
s(t) = A \sin(2\pi f t + \phi)
\]
where \( A \) is the amplitude, \( f \) is the frequency, and \( \phi \) is the phase. A chirpy or unstable signal would introduce variations in frequency \( f \) or phase \( \phi \), leading to a less predictable waveform.

Understanding the RST report and its modifiers like C is crucial for effective communication in amateur radio, as it helps operators diagnose and improve their signal quality.

% Diagram prompt: Generate a diagram showing a stable sine wave versus a chirpy/unstable sine wave for visual comparison.