\subsection{Disadvantages of Shortened Mobile Antennas}
\label{G4E06}

\begin{tcolorbox}[colback=gray!10!white,colframe=black!75!black,title=G4E06]
What is one disadvantage of using a shortened mobile antenna as opposed to a full-size antenna?
\begin{enumerate}[label=\Alph*),noitemsep]
    \item Short antennas are more likely to cause distortion of transmitted signals
    \item Q of the antenna will be very low
    \item \textbf{Operating bandwidth may be very limited}
    \item Harmonic radiation may increase
\end{enumerate}
\end{tcolorbox}

\subsubsection{Intuitive Explanation}
Imagine you have a big, long antenna like a giant fishing rod. It can catch a lot of different fish (signals) because it’s big and covers a wide area. Now, if you use a tiny, short antenna, it’s like using a small fishing net. You can only catch a few fish (signals) at a time. So, the big disadvantage of using a short antenna is that it can’t handle as many different signals as a full-size antenna. It’s like trying to watch all your favorite TV channels with a tiny antenna—you might only get a few!

\subsubsection{Advanced Explanation}
The operating bandwidth of an antenna is the range of frequencies over which it can effectively transmit or receive signals. A full-size antenna, typically a quarter-wavelength or half-wavelength long, is designed to operate efficiently over a broad range of frequencies. In contrast, a shortened mobile antenna, which is physically shorter than a quarter-wavelength, often employs loading coils or other techniques to achieve resonance at the desired frequency. However, these modifications can significantly reduce the antenna's bandwidth.

The bandwidth \( B \) of an antenna is inversely proportional to its quality factor \( Q \), which is a measure of how sharply the antenna resonates at a particular frequency. Mathematically, this relationship can be expressed as:

\[
B = \frac{f_0}{Q}
\]

where \( f_0 \) is the resonant frequency. A high \( Q \) indicates a narrow bandwidth, meaning the antenna is highly selective and can only operate effectively over a limited range of frequencies. Shortened antennas typically have a higher \( Q \) due to their reduced physical size and the added components needed to achieve resonance, leading to a very limited operating bandwidth.

Additionally, the efficiency of a shortened antenna is often lower than that of a full-size antenna, as the added components introduce losses. This further restricts the antenna's ability to operate over a wide range of frequencies.

% Prompt for diagram: A diagram comparing the bandwidth of a full-size antenna versus a shortened antenna, showing the frequency range each can effectively cover.