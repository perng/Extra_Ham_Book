
% ************************************************************************
% CHAPTER 3: Sublement E3 - Radio Wave Propagation
% ************************************************************************
\chapter{E3: Radio Wave Propagation - Riding the Radio Waves!}

% ------------------------------------------------------------------------
% SECTION E3A: Electromagnetic Waves and Specialized Propagation
% ------------------------------------------------------------------------
\section{E3A: Electromagnetic Waves and Specialized Propagation - Where Do Radio Signals Go?}

\subsection*{Understanding the Basics}
Ready to understand what really happens when your radio signals leave your antenna? Get ready to take a journey through the fascinating world of \textcolor{myblue}{\textbf{electromagnetic waves and propagation}}! This section goes into several different methods of long and short range communications.
We'll delve into the specifics of \textcolor{myblue}{\textbf{Earth-Moon-Earth (EME) communications}}, a challenging yet awesome way of communicating through the moon. Then we will explore \textcolor{myblue}{\textbf{meteor scatter}}, that allows communication using meteor trails. Other modes such as \textcolor{myblue}{\textbf{tropospheric and scatter propagation}}, and \textcolor{myblue}{\textbf{auroral propagation}} also will be covered. Finally we will discuss \textcolor{myblue}{\textbf{daily variation of ionospheric propagation}} and the important topic of \textcolor{myblue}{\textbf{circular polarization}}.
\mybox{mygreen}{
  \textbf{Fun Fact}: Did you know that the ionosphere, a layer of Earth's atmosphere, is a major part of how radio waves travel across the globe?  It is an area of electrical activity that changes based on time of day, season and solar activity!
  }
\subsection*{Key Concepts for the Questions}
\begin{itemize}
    \item \textbf{Earth-Moon-Earth (EME):} Communication via bouncing signals off the Moon.
    \item \textbf{Meteor Scatter:} A communication technique that uses ionized trails of meteors to reflect signals.
    \item \textbf{Tropospheric Ducting:} The propagation of signals within atmospheric layers, that act as a duct to guide the signals over long distances.
        \item \textbf{Auroral Propagation:} A technique that uses reflections of signals off of the auroras for propagation.
        \item \textbf{Ionospheric Propagation:} Radio signals that are reflected by the ionosphere.
    \item \textbf{Libration Fading:} A type of fading specific to EME communication due to the moon's motion.
     \item \textbf{Perigee:} When moon is closest to the earth in its orbit.
         \item \textbf{Apogee:} When moon is farthest from the earth in its orbit.
        \item \textbf{MUF (Maximum Usable Frequency):} The highest frequency that can be used for ionospheric propagation.
        \item \textbf{Circular Polarization:} Signals with a rotating electric and magnetic field.

\end{itemize}

\subsection*{Practice Questions}
\begin{enumerate}
      \item What is the approximate maximum separation measured along the surface of the Earth between two stations communicating by EME?
       \begin{enumerate}
        \item  2,000 miles, if the moon is at perigee
          \item  2,000 miles, if the moon is at apogee
      \item  5,000 miles, if the moon is at perigee
         \item \textbf{D. 12,000 miles, if the moon is "visible" by both stations}
        \end{enumerate}
    \textcolor{myred}{Explanation:}
   EME allows contacts between two stations that have a common "view" of the moon. With no terrain obstruction it can be up to 12000 miles apart.

    \item What characterizes libration fading of an EME signal?
      \begin{enumerate}
      \item  A slow change in the pitch of the CW signal
       \item \textbf{B. A fluttery, irregular fading}
      \item  A gradual loss of signal as the sun rises
        \item  The returning echo is several hertz lower in frequency than the transmitted signal
    \end{enumerate}
       \textcolor{myred}{Explanation:}
        Libration fading is specific to EME communications which manifests in a very random and fluttery fashion.
         
   \item When scheduling EME contacts, which of these conditions will generally result in the least path loss?
     \begin{enumerate}
      \item \textbf{A. When the Moon is at perigee}
         \item  When the Moon is full
       \item  When the Moon is at apogee
        \item  When the MUF is above 30 MHz
       \end{enumerate}
    \textcolor{myred}{Explanation:}
    When the moon is at the perigee, it will be closer to the earth and experience a reduced free-space path loss.
   
  \item In what direction does an electromagnetic wave travel?
    \begin{enumerate}
         \item  It depends on the phase angle of the magnetic field
         \item  It travels parallel to the electric and magnetic fields
      \item  It depends on the phase angle of the electric field
         \item \textbf{D. It travels at a right angle to the electric and magnetic fields}
     \end{enumerate}
       \textcolor{myred}{Explanation:}
    Electromagnetic waves always travel at right angles to their electric and magnetic fields.
   
    \item How are the component fields of an electromagnetic wave oriented?
       \begin{enumerate}
        \item  They are parallel
      \item  They are tangential
        \item \textbf{C. They are at right angles}
         \item  They are 90 degrees out of phase
        \end{enumerate}
         \textcolor{myred}{Explanation:}
        Electromagnetic waves always have electric and magnetic field oriented 90 degrees to each other.
    
    \item What should be done to continue a long-distance contact when the MUF for that path decreases due to darkness?
       \begin{enumerate}
           \item  Switch to a higher frequency HF band
       \item \textbf{B. Switch to a lower frequency HF band}
        \item  Change to an antenna with a higher takeoff angle
         \item  Change to an antenna with greater beam width
        \end{enumerate}
    \textcolor{myred}{Explanation:}
     As the ionosphere changes due to sunset and darkness, the lower bands become useful due to lower MUF.

   \item Atmospheric ducts capable of propagating microwave signals often form over what geographic feature?
     \begin{enumerate}
        \item  Mountain ranges
      \item  Stratocumulus clouds
      \item \textbf{C. Large bodies of water}
       \item  Nimbus clouds
     \end{enumerate}
      \textcolor{myred}{Explanation:}
    Tropospheric ducting can occur over large bodies of water.
       
    \item When a meteor strikes the Earth's atmosphere, a linear ionized region is formed at what region of the ionosphere?
        \begin{enumerate}
         \item \textbf{A. The E region}
        \item  The F1 region
     \item  The F2 region
       \item  The D region
    \end{enumerate}
    \textcolor{myred}{Explanation:}
      Meteors entering the atmosphere ionize the E region.
        
       \item Which of the following frequency ranges is most suited for meteor-scatter communications?
       \begin{enumerate}
           \item  1.8 MHz - 1.9 MHz
     \item  10 MHz - 14 MHz
    \item \textbf{C. 28 MHz - 148 MHz}
      \item  220 MHz - 450 MHz
        \end{enumerate}
     \textcolor{myred}{Explanation:}
      Meteor scatter mostly occurs on 28 - 148 MHz range, where signals reflect on the ionized meteor trail.

        \item What determines the speed of electromagnetic waves through a medium?
    \begin{enumerate}
        \item  Resistance and reactance
         \item  Evanescence
         \item  Birefringence
        \item \textbf{D. The index of refraction}
    \end{enumerate}
     \textcolor{myred}{Explanation:}
      Index of refraction is defined as the speed of an EM wave in the given media relative to that of the vacuum.
        
        \item What is a typical range for tropospheric duct propagation of microwave signals?
       \begin{enumerate}
        \item  10 miles to 50 miles
          \item \textbf{B. 100 miles to 300 miles}
        \item  1,200 miles
       \item  2,500 miles
       \end{enumerate}
     \textcolor{myred}{Explanation:}
     Tropospheric ducting can commonly extend for up to 300 miles, but have occurred at greater distances.

    \item What is most likely to result in auroral propagation?
      \begin{enumerate}
        \item  Meteor showers
          \item  Quiet geomagnetic conditions
       \item \textbf{C. Severe geomagnetic storms}
        \item  Extreme low-pressure areas in polar regions
      \end{enumerate}
        \textcolor{myred}{Explanation:}
    Strong aurora are associated with increased geomagnetic storms.
   
        \item Which of these emission modes is best for auroral propagation?
        \begin{enumerate}
         \item \textbf{A. CW}
          \item  SSB
        \item  FM
         \item  RTTY
        \end{enumerate}
      \textcolor{myred}{Explanation:}
       Due to the specific behavior of auroral reflection and resulting signal distortion, CW mode is the most efficient for such communications.
       
        \item What are circularly polarized electromagnetic waves?
        \begin{enumerate}
       \item  Waves with an electric field bent into a circular shape
        \item \textbf{B. Waves with rotating electric and magnetic fields}
         \item  Waves that circle Earth
          \item  Waves produced by a loop antenna
        \end{enumerate}
       \textcolor{myred}{Explanation:}
    Circular polarization results in rotating electric and magnetic field when the signal is transmitted.

\end{enumerate}

% ------------------------------------------------------------------------
% SECTION E3B: Transequatorial Propagation and Other Modes
% ------------------------------------------------------------------------
\section{E3B: Transequatorial Propagation and Other Modes - Bouncing Around!}

\subsection*{Understanding the Basics}
In this section we will continue the fascinating topic of radio wave propagation. We will start with \textcolor{myblue}{\textbf{Transequatorial propagation}}, a rare form of HF propagation and will discuss some unique aspects of its use. We will then cover \textcolor{myblue}{\textbf{long-path propagation}}, which is using the longest path on earth circumference for transmissions, and cover differences between the two paths that result in \textcolor{myblue}{\textbf{ordinary and extraordinary waves}}, a concept that has impact in understanding signal propagation in ionosphere. Then, we will learn about \textcolor{myblue}{\textbf{chordal hop}}, that provides a different mode of propagating a wave in the ionosphere without reflection on the earth's surface, and  \textcolor{myblue}{\textbf{sporadic-E mechanisms}} which enables contacts on higher HF and VHF bands due to short-lived intense ionization. Finally, we'll understand how signals travel through the earth by use of \textcolor{myblue}{\textbf{ground-wave propagation}} in low frequencies.
\mybox{mygreen}{
   \textbf{Fun Fact}:   Sporadic-E can sometimes create openings on VHF bands, which would otherwise be unlikely! This can allow for unexpected and surprisingly long range contacts even at VHF frequencies!
  }

\subsection*{Key Concepts for the Questions}
\begin{itemize}
    \item \textbf{Transequatorial Propagation (TEP):}  A mode of propagation where signals travel across the geomagnetic equator.
     \item \textbf{Long Path:} Radio wave propagation that travels along the longer direction of earth’s curvature instead of the shortest one.
    \item \textbf{Ordinary/Extraordinary Waves:} Two types of signals created in the ionosphere that have different path behaviors.
        \item  \textbf{Chordal Hop:} When a signal is refracted in the ionosphere and returns to earth, skipping earth surface reflection.
    \item \textbf{Sporadic-E:} Temporary and sporadic high level of ionization in the E region.
      \item \textbf{Ground-Wave Propagation:}  The direct transmission of radio waves along the Earth's surface.
      \item \textbf{Ionospheric Skip:} Reflection of radio signals off the ionosphere.
     \item \textbf{Terminator:} The boundary line between the daylit and night portions of earth.
\end{itemize}

\subsection*{Practice Questions}
\begin{enumerate}
    \item Where is transequatorial propagation (TEP) most likely to occur?
         \begin{enumerate}
       \item \textbf{A. Between points separated by 2,000 miles to 3,000 miles over a path perpendicular to the geomagnetic equator}
        \item  Between points located 1,500 miles to 2,000 miles apart on the geomagnetic equator
     \item  Between points located at each other's antipode
      \item  Through the region where the terminator crosses the geographic equator
     \end{enumerate}
    \textcolor{myred}{Explanation:}
      Transequatorial propagation (TEP) typically occurs across the geomagnetic equator in 2000-3000 mile range.
    
    \item What is the approximate maximum range for signals using transequatorial propagation?
      \begin{enumerate}
        \item  1,000 miles
      \item  2,500 miles
         \item \textbf{C. 5,000 miles}
      \item  7,500 miles
    \end{enumerate}
     \textcolor{myred}{Explanation:}
       Transequatorial propagation can reach up to 5000 miles in some conditions.

    \item At what time of day is transequatorial propagation most likely to occur?
    \begin{enumerate}
    \item  Morning
    \item  Noon
     \item \textbf{C. Afternoon or early evening}
   \item  Late at night
      \end{enumerate}
      \textcolor{myred}{Explanation:}
       TEP is most likely to occur in afternoon or early evening.
        
       \item What are "extraordinary” and “ordinary” waves?
        \begin{enumerate}
          \item  Extraordinary waves exhibit rare long-skip propagation, compared to ordinary waves, which travel shorter distances
        \item \textbf{B. Independently propagating, elliptically polarized waves created in the ionosphere}
        \item  Long-path and short-path waves
        \item  Refracted rays and reflected waves
      \end{enumerate}
     \textcolor{myred}{Explanation:}
       Extraordinary and ordinary are names given to two types of waves in the ionosphere with different polarizations and path.

     \item Which of the following paths is most likely to support long-distance propagation on 160 meters?
        \begin{enumerate}
         \item  A path entirely in sunlight
          \item  Paths at high latitudes
       \item  A direct north-south path
       \item \textbf{D. A path entirely in darkness}
    \end{enumerate}
     \textcolor{myred}{Explanation:}
       160 meter operation relies on D layer absorption, which minimizes during night time for the longest distance operations.
       
       \item On which of the following amateur bands is long-path propagation most frequent?
      \begin{enumerate}
          \item  160 meters and 80 meters
      \item \textbf{B. 40 meters and 20 meters}
      \item  10 meters and 6 meters
        \item  6 meters and 2 meters
        \end{enumerate}
         \textcolor{myred}{Explanation:}
       Long path propagation is frequently observed at 40 and 20 meters bands.

     \item What effect does lowering a signal's transmitted elevation angle have on ionospheric HF skip propagation?
       \begin{enumerate}
         \item  Faraday rotation becomes stronger
         \item  The MUF decreases
       \item \textbf{C. The distance covered by each hop increases}
     \item  The critical frequency increases
        \end{enumerate}
       \textcolor{myred}{Explanation:}
    Lowering elevation angle allows the signal to bounce off the ionosphere at a greater distance.
        
     \item How does the maximum range of ground-wave propagation change when the signal frequency is increased?
        \begin{enumerate}
         \item  It stays the same
        \item  It increases
         \item \textbf{C. It decreases}
        \item  It peaks at roughly 8 MHz
        \end{enumerate}
       \textcolor{myred}{Explanation:}
    Ground wave propagation is more prevalent at lower frequencies. As the frequency increases the absorption by the ground decreases the range.
       
        \item At what time of year is sporadic-E propagation most likely to occur?
        \begin{enumerate}
    \item \textbf{A. Around the solstices, especially the summer solstice}
     \item  Around the solstices, especially the winter solstice
      \item  Around the equinoxes, especially the spring equinox
    \item  Around the equinoxes, especially the fall equinox
      \end{enumerate}
       \textcolor{myred}{Explanation:}
       Sporadic-E is more commonly observed during summer months and especially around solstice.

   \item What is the effect of chordal-hop propagation?
        \begin{enumerate}
           \item \textbf{A. The signal experiences less loss compared to multi-hop propagation, which uses Earth as a reflector}
        \item  The MUF for chordal-hop propagation is much lower than for normal skip propagation
     \item  Atmospheric noise is reduced in the direction of chordal-hop propagation
        \item  Signals travel faster along ionospheric chords
        \end{enumerate}
    \textcolor{myred}{Explanation:}
        Chordal hops have lower path losses because it does not involve reflecting off of the ground.
   
   \item At what time of day is sporadic-E propagation most likely to occur?
       \begin{enumerate}
         \item  Between midnight and sunrise
         \item  Between sunset and midnight
        \item  Between sunset and sunrise
         \item \textbf{D. Between sunrise and sunset}
       \end{enumerate}
    \textcolor{myred}{Explanation:}
      Sporadic E happens mostly during the day.

       \item What is chordal-hop propagation?
      \begin{enumerate}
          \item  Propagation away from the great circle bearing between stations
       \item \textbf{B. Successive ionospheric refractions without an intermediate reflection from the ground}
       \item  Propagation across the geomagnetic equator
      \item  Signals reflected back toward the transmitting station
     \end{enumerate}
        \textcolor{myred}{Explanation:}
        Chordal hops occur when a radio signal is repeatedly refracted, skipping reflection from the ground.

      \item What type of polarization is supported by ground-wave propagation?
        \begin{enumerate}
          \item \textbf{A. Vertical}
        \item  Horizontal
      \item  Circular
    \item  Elliptical
      \end{enumerate}
     \textcolor{myred}{Explanation:}
     Ground waves primarily propagate using vertical polarization.
\end{enumerate}


Okay, here's the LaTeX source code for sections E3C and E3D, continuing the previous structure and cheerful tone:

% ------------------------------------------------------------------------
% SECTION E3C: Propagation Prediction and Reporting
% ------------------------------------------------------------------------
\section{E3C: Propagation Prediction and Reporting - Forecasting the Signals!}

\subsection*{Understanding the Basics}
Get ready to explore how we can understand and sometimes even predict how the radio waves will behave! In this section we will discuss \textcolor{myblue}{\textbf{propagation prediction and reporting}}, including understanding different techniques to foresee signal strength and quality. We'll delve into the fascinating topic of the \textcolor{myblue}{\textbf{radio horizon}}, what determines the maximum line-of-sight distance for radio waves and we will also understand how \textcolor{myblue}{\textbf{effects of space weather phenomena}} can dramatically impact radio propagation. This includes the influences of solar flares, geomagnetic storms, and aurorae, on radio transmissions.

\mybox{mygreen}{
  \textbf{Fun Fact}: Did you know that you can monitor real time space weather activities to predict changes in radio propagation? There are numerous free website that provide data to plan your next communication contacts!
  }
\subsection*{Key Concepts for the Questions}
\begin{itemize}
    \item \textbf{Radio Horizon:}  The maximum distance at which radio waves can be transmitted directly without any reflection.
        \item \textbf{Solar flares:}  Sudden release of energy from the sun, that affect propagation.
    \item \textbf{A-index and K-index:} Two measures of geomagnetic activity and strength.
     \item \textbf{Auroral Oval:} A zone that encircles the polar region where auroras frequently occur.
    \item \textbf{Bz (B sub z):} A measure of north-south strength of interplanetary magnetic field.
       \item \textbf{Space Weather Terms:} The effects of space conditions (solar activity) on Earth's magnetic field.
     \item \textbf{Solar Flux:} A measurement of the energy output of the sun.
       \item \textbf{304A:}  A band used for measuring solar UV emissions.
           \item  \textbf{VOACAP:} A software tool used for propagation prediction.
            \item  \textbf{Coronal Mass Ejection:} Massive release of plasma from the sun.
            \item \textbf{RMS Noise:} A measure of the root mean square of noise.

\end{itemize}

\subsection*{Practice Questions}
\begin{enumerate}
    \item What is the cause of short-term radio blackouts?
        \begin{enumerate}
          \item  Coronal mass ejections
       \item  Sunspots on the solar equator
      \item  North-oriented interplanetary magnetic field
         \item \textbf{D. Solar flares}
     \end{enumerate}
    \textcolor{myred}{Explanation:}
     Solar flares emit strong X rays that can cause sudden and complete blackouts of some frequencies.
     
   \item What is indicated by a rising A-index or K-index?
         \begin{enumerate}
        \item \textbf{A. Increasing disturbance of the geomagnetic field}
        \item  Decreasing disturbance of the geomagnetic field
       \item  Higher levels of solar UV radiation
        \item  An increase in the critical frequency
    \end{enumerate}
     \textcolor{myred}{Explanation:}
     A-index and K-index are two measures of disturbance of the geomagnetic field, higher value of either implies higher levels of disturbance.

        \item Which of the following signal paths is most likely to experience high levels of absorption when the A-index or K-index is elevated?
        \begin{enumerate}
      \item  Transequatorial
       \item \textbf{B. Through the auroral oval}
       \item  Sporadic-E
       \item  NVIS
        \end{enumerate}
    \textcolor{myred}{Explanation:}
   Aurora regions are associated with absorption, a higher A/K index can lead to high level of absorption in this region.
        
        \item What does the value of Bz (B sub z) represent?
         \begin{enumerate}
        \item  Geomagnetic field stability
      \item  Critical frequency for vertical transmissions
    \item \textbf{C. North-south strength of the interplanetary magnetic field}
        \item  Duration of long-delayed echoes
        \end{enumerate}
        \textcolor{myred}{Explanation:}
          Bz refers to the north-south component of interplanetary magnetic field. When this is southward it is more likely to generate geomagnetic disturbances.
    
       \item What orientation of Bz (B sub z) increases the likelihood that charged particles from the Sun will cause disturbed conditions?
    \begin{enumerate}
        \item \textbf{A. Southward}
       \item  Northward
         \item  Eastward
         \item  Westward
        \end{enumerate}
   \textcolor{myred}{Explanation:}
       When interplanetary field is oriented southward, it can combine with Earth's magnetic field and allow charged particles to interact in the atmosphere, causing disturbances.
        
        \item How does the VHF/UHF radio horizon compare to the geographic horizon?
        \begin{enumerate}
        \item \textbf{A. It is approximately 15 percent farther}
      \item  It is approximately 20 percent nearer
        \item  It is approximately 50 percent farther
       \item  They are approximately the same
        \end{enumerate}
      \textcolor{myred}{Explanation:}
    Radio waves bend in the atmosphere which extend radio horizons roughly 15% further than the geographical horizon.

    \item Which of the following indicates the greatest solar flare intensity?
       \begin{enumerate}
        \item  Class A
       \item  Class Z
       \item  Class M
         \item \textbf{D. Class X}
        \end{enumerate}
      \textcolor{myred}{Explanation:}
       Class X flares are the most intense solar flares in the standard ranking system.
       
     \item Which of the following is the space-weather term for an extreme geomagnetic storm?
       \begin{enumerate}
         \item  B9
         \item  X5
       \item  M9
     \item \textbf{D. G5}
        \end{enumerate}
       \textcolor{myred}{Explanation:}
        G5 is the space weather term used for describing an extreme geomagnetic storm.
        
     \item What type of data is reported by amateur radio propagation reporting networks?
       \begin{enumerate}
       \item  Solar flux
      \item  Electric field intensity
         \item  Magnetic declination
     \item \textbf{D. Digital-mode and CW signals}
        \end{enumerate}
       \textcolor{myred}{Explanation:}
       Propagation reporting networks commonly use data based on transmissions of digital modes and CW signals.
       
       \item What does the 304A solar parameter measure?
    \begin{enumerate}
        \item  The ratio of X-ray flux to radio flux, correlated to sunspot number
        \item \textbf{B. UV emissions at 304 angstroms, correlated to the solar flux index}
    \item  The solar wind velocity at an angle of 304 degrees from the solar equator, correlated to geomagnetic storms
   \item  The solar emission at 304 GHz, correlated to X-ray flare levels
        \end{enumerate}
        \textcolor{myred}{Explanation:}
          304A is specific wavelength of UV emissions. It is often used for indicating solar flux levels.
          
     \item What does VOACAP software model?
         \begin{enumerate}
       \item  AC voltage and impedance
       \item  VHF radio propagation
        \item \textbf{C. HF propagation}
         \item  AC current and impedance
     \end{enumerate}
     \textcolor{myred}{Explanation:}
        VOACAP models HF radio wave propagations.
       
     \item Which of the following is indicated by a sudden rise in radio background noise across a large portion of the HF spectrum?
       \begin{enumerate}
      \item  A temperature inversion has occurred
         \item \textbf{B. A coronal mass ejection impact or a solar flare has occurred}
     \item  Transequatorial propagation on 6 meters is likely
        \item  Long-path propagation on the higher HF bands is likely
        \end{enumerate}
        \textcolor{myred}{Explanation:}
        Coronal mass ejections and solar flares are the most likely source of sudden rise in background noise in HF spectrum.

\end{enumerate}


